% !TEX root = ../thesis.tex
% !TeX spellcheck = en_US

In this chapter, the implementation of the hybrid routing algorithm is evaluated regarding performance, correctness and usefulness.
Next to the actual results, methods and design details on the evaluation are given as well.

\section{Performance evaluation}

	The performance evaluation uses different datasets to measure graph generation and routing times.
	Each of these two steps is measured more fine-grained on the level of separate method calls.
	% TODO Mention memory usage as well

	\subsection{Methods \& Measurements}

		\subsubsection{Collected data}
		
			The collected data consists of time and memory usage measurement, of which the time measurements are on the level of separate methods.
			Also, the amount of data is measures, namely the number of edges and vertices at various steps in the process, as well as the length of routes, which includes the beeline and actual route distances.
		
		\subsubsection{Datasets}
		\label{subsubsec:eval-datasets}
		
			Two overall types of datasets are used:
			Pattern-based datasets are created using a recurring pattern of various sizes.
			OSM-based datasets use differently sized extracts from OpenStreetMap.
			While the OSM-based datasets contain obstacles and roads (except in the \enquote{without roads/obstacles} datasets), the pattern-based datasets do no contain any roads.
			
			For both types, several categories of datasets exist with different properties:
			\begin{description}
				\item[Maze pattern] Datasets with touching and collinear linestrings forming a maze-like structure.
				\item[Rectangle pattern] Datasets with numerous differently sized, positioned and rotated rectangles.
				\item[Circle pattern] dataset with differently sized circles having a large amount of vertices.
				\item[OSM city] Real-world extracts from the OpenStreetMap database with data from the city of Hamburg, Germany. The data has been filtered to remove all over- and underground features.
				\item[OSM rural] Equivalent to the \enquote{OSM city} dataset, but located outside the city of Hamburg and therefore containing more natural obstacles (lakes, ditches, forest), more open spaces and less regular distributed buildings.
				\item[OSM export without roads] OSM extracts but without roads.
				\item[OSM export without obstacles] Analogous to the \enquote{without roads} datasets, but without any obstacles, i.e. buildings, walls and natural areas such as lakes and forests.
			\end{description}
			The \enquote{OSM city} and \enquote{OSM rural} categories each contain six dataset of the sizes 0.5, 1, 1.5, 2, 3 and 4 km\textsuperscript{2}.
			The two OSM categories \enquote{without roads/obstacles} both use the 4 km\textsuperscript{2} datasets from the city and rural categories.
			All OSM-based dataset are filtered to not contains any over- or underground features, because the hybrid routing algorithm was not made to handle this third spatial dimension.
			
			\begin{figure}[h!]
				\centering
				\begin{minipage}[t]{.38\textwidth}
					\begin{figcenter}
						\includegraphics[width=\textwidth]{images/qgis-overview-city-rural_city}
					\end{figcenter}
				\end{minipage}
				\hspace{0.04\textwidth}
				\begin{minipage}[t]{.38\textwidth}
					\begin{figcenter}
						\includegraphics[width=\textwidth]{images/qgis-overview-city-rural_rural}
					\end{figcenter}
				\end{minipage}
				\caption[Areas of the \enquote{OSM city} and \enquote{OSM rural} datasets.]{All six regions from 0.5km\textsuperscript{2} to 4km\textsuperscript{2} of the \enquote{OSM city} datasets (left) and \enquote{OSM rural} datasets (right).}
			\end{figure}
		
		\subsubsection{Optimizations}
		
			As described in \Cref{chap:implementation}, there are several optimizations made to the implementation.
			Some of which are on the level of data structures, some on an algorithmic level.
			The effectiveness of these optimization was evaluated using the 4 km\textsuperscript{2} \enquote{OSM city} dataset.
			Each of the following optimizations was deactivated or replaced for the evaluation:

			\begin{description}
				\item[Shadow areas] Instead, every visibility check was performed using the custom intersection check described in \Cref{subsubsec:intersection-checks}.
				\item[Custom intersection check] The custom intersection check was replaced by the \texttt{RobustLineIntersector} class from the NTS to determine intersections between line segments.
				\item[BinIndex] Instead, the \texttt{Bintree} from the NTS was used.
				\item[Convex hull] The restriction to only consider vertices on the convex hull of obstacles was removed.
				\item[Valid angle areas] Considering only potential visibility neighbors within certain angular ranges was deactivated.
				\item[kNN search] Instead, all visibility neighbors in all directions were determined.
			\end{description}		
			
		\subsubsection{Measurement method}
		
			Measuring the performance was done by a small agent-based simulation project called \texttt{HikerModel}, consisting of one agent consecutively visiting a given list of coordinates (routing waypoints) and using a given dataset as input to the hybrid routing algorithm.
			The coordinates are given via a linestring within a GeoJSON file, of which each coordinate of the linestring is visited by the agent in order.
			Each waypoint was within the range of the dataset, meaning each coordinate was surrounded by obstacles.
			The euclidean distances (beeline distances) of the line segments within this linestring were distributed evenly to measure the required routing time relative to the distance and dataset size.
			
			Because the OSM datasets within one category cover differently sized areas, each waypoint linestring of an OSM-based dataset contains all waypoints of the next smaller one plus some additional ones.
			This means that the waypoints of the smallest dataset are used by every other dataset as well.
		
		\subsubsection{Technical considerations}
		
			The measurement was done by an auxiliary class \texttt{PerformanceMeasurement}, which provides a method that mesaures the execution time of a passed function delegate.
			
			As part of its memory management, C\#/.NET uses automatic garbage collection adding unavoidable noise to the measurements.
			Unfortunately the garbage collector cannot be turned off and controlling it is only partially possible.
			
			Prior to each measurement, the garbage collector was triggered with the goal to provide equal circumstances to all iterations.
			This was done by the \texttt{GC.Collect()} and \texttt{GC.WaitForPendingFinalizers()} methods of the .NET framework.
			Using these two methods forces a garbage collection and waits for it to finish\footnote{\url{https://learn.microsoft.com/en-us/dotnet/api/system.gc.waitforpendingfinalizers?view=net-7.0}}.
			
			To prevent the garbage collection from interfering with the execution, a 256 MiB large no-GC-region is placed around the measured function call via \texttt{GC.TryStartNoGCRegion(256 * 1024 * 1024)}.
			Introducing this no-GC-region noticeably reduced noise in the measured times.
		
			C\#/.NET also uses \term*[just-in-time compilation]{just-in-time} (\term*{JIT}) compilation changing the code during runtime, which potentially yields different measurements for subsequent method calls.
			JIT compilation can not be turned off for normal .NET executions via the command \texttt{dotnet program.dll}.
			An alternative would be the usage of \term*[]ahead-of-time compilation]{ahead-of-time} (\term*{AOT}) compilation, which has a negative impact on the performance of LINQ operations\footnote{\url{https://learn.microsoft.com/en-us/dotnet/core/deploying/native-aot/?tabs=net7}}, which are frequently used in the implementation.
			Because both compilation strategies have disadvantages, the default JIT compilation was chosen.
			
			The dynamic behavior of JIT compilation was mitigated by calling the measured function three times without storing the measurement results (warm-ups) before executing it five times and storing these last five results.
			This ensures that any JIT compilation and garbage collection of prior code was performed during the warm-up iterations and interfered with the actual iteration as little as possible.
			
			Another step to get stable and reproducible results was the increase of the process priority to exclusively use one CPU core on which the single threaded application ran.
			Increasing the process priority was done by settings the \texttt{Thread.CurrentThread.Priority} to \texttt{ProcessPriorityClass.High}, which required root permissions on Linux systems.
		
		\subsubsection{System and hardware}
		
			The measurements were performed on an up-to-date Arch Linux operating system (Kernel 6.4.3) with .NET Core 7.0.107 and MARS framework 4.5.2.
			Apart from necessary operating system processes and a minimal desktop environment, no other applications ran during the performance measurements.
			
			The hardware consisted of an octa core Intel\textregistered\ Xeon\textregistered\ E3-1231 v3 CPU at 3.40 GHz, a total of 16GB DDR3 1333 MHz RAM and a Samsung EVO 850 SSD.
			However, the whole algorithm and the \texttt{HikerModel} simulation is single threaded.
			File system operations are only performed to initially load the input data and to write the measurement results after completing all executions.
	
\section{Performance evaluation}

	In this section the results of the performance evaluation of the hybrid routing algorithm are presented.
	First the OSM-based datasets are discussed followed by the artificial pattern-based datasets.

	\subsection{OSM-based datasets}
		
		\subsubsection{Import and graph generation}
		
			A few aspects regarding the runtime behavior can be inferred from the general graph generation times shown in \Cref{fig:eval-import-city} and \Cref{fig:eval-import-rural}.
			First, as mentioned at the beginning of \Cref{subsec:related-work:visibility-graph}, the process of generating a visibility graph has an inherent quadratic runtime.
			This fact is clearly visible in the import measurements shown in \Cref{fig:eval-import-city-abs} and \Cref{fig:eval-import-rural-abs}, even though it is less prominent in the \enquote{OSM rural} datasets.
%			Details on the contribution of each task to the overall graph generation time discussed below.
			Second, \Cref{fig:eval-import-city-rel} and \ref{fig:eval-import-rural-rel} show an increase in the per-vertex processing time, which not necessarily grows quadratic as the \enquote{OSM rural} dataset shows.
			
			\begin{figure}[h!]
				\begin{minipage}{.48\textwidth}
					\begin{subfigure}[t]{\linewidth}
						\begin{figcenter}
							%% Creator: Matplotlib, PGF backend
%%
%% To include the figure in your LaTeX document, write
%%   \input{<filename>.pgf}
%%
%% Make sure the required packages are loaded in your preamble
%%   \usepackage{pgf}
%%
%% Also ensure that all the required font packages are loaded; for instance,
%% the lmodern package is sometimes necessary when using math font.
%%   \usepackage{lmodern}
%%
%% Figures using additional raster images can only be included by \input if
%% they are in the same directory as the main LaTeX file. For loading figures
%% from other directories you can use the `import` package
%%   \usepackage{import}
%%
%% and then include the figures with
%%   \import{<path to file>}{<filename>.pgf}
%%
%% Matplotlib used the following preamble
%%   
%%   \usepackage{fontspec}
%%   \setmainfont{DejaVuSerif.ttf}[Path=\detokenize{/home/hauke/.local/lib/python3.11/site-packages/matplotlib/mpl-data/fonts/ttf/}]
%%   \setsansfont{DroidSans.ttf}[Path=\detokenize{/usr/share/fonts/droid/}]
%%   \setmonofont{DejaVuSansMono.ttf}[Path=\detokenize{/home/hauke/.local/lib/python3.11/site-packages/matplotlib/mpl-data/fonts/ttf/}]
%%   \makeatletter\@ifpackageloaded{underscore}{}{\usepackage[strings]{underscore}}\makeatother
%%
\begingroup%
\makeatletter%
\begin{pgfpicture}%
\pgfpathrectangle{\pgfpointorigin}{\pgfqpoint{2.681845in}{1.770898in}}%
\pgfusepath{use as bounding box, clip}%
\begin{pgfscope}%
\pgfsetbuttcap%
\pgfsetmiterjoin%
\definecolor{currentfill}{rgb}{1.000000,1.000000,1.000000}%
\pgfsetfillcolor{currentfill}%
\pgfsetlinewidth{0.000000pt}%
\definecolor{currentstroke}{rgb}{1.000000,1.000000,1.000000}%
\pgfsetstrokecolor{currentstroke}%
\pgfsetdash{}{0pt}%
\pgfpathmoveto{\pgfqpoint{0.000000in}{0.000000in}}%
\pgfpathlineto{\pgfqpoint{2.681845in}{0.000000in}}%
\pgfpathlineto{\pgfqpoint{2.681845in}{1.770898in}}%
\pgfpathlineto{\pgfqpoint{0.000000in}{1.770898in}}%
\pgfpathlineto{\pgfqpoint{0.000000in}{0.000000in}}%
\pgfpathclose%
\pgfusepath{fill}%
\end{pgfscope}%
\begin{pgfscope}%
\pgfsetbuttcap%
\pgfsetmiterjoin%
\definecolor{currentfill}{rgb}{1.000000,1.000000,1.000000}%
\pgfsetfillcolor{currentfill}%
\pgfsetlinewidth{0.000000pt}%
\definecolor{currentstroke}{rgb}{0.000000,0.000000,0.000000}%
\pgfsetstrokecolor{currentstroke}%
\pgfsetstrokeopacity{0.000000}%
\pgfsetdash{}{0pt}%
\pgfpathmoveto{\pgfqpoint{0.464084in}{0.451389in}}%
\pgfpathlineto{\pgfqpoint{2.661760in}{0.451389in}}%
\pgfpathlineto{\pgfqpoint{2.661760in}{1.770898in}}%
\pgfpathlineto{\pgfqpoint{0.464084in}{1.770898in}}%
\pgfpathlineto{\pgfqpoint{0.464084in}{0.451389in}}%
\pgfpathclose%
\pgfusepath{fill}%
\end{pgfscope}%
\begin{pgfscope}%
\pgfpathrectangle{\pgfqpoint{0.464084in}{0.451389in}}{\pgfqpoint{2.197676in}{1.319509in}}%
\pgfusepath{clip}%
\pgfsetroundcap%
\pgfsetroundjoin%
\pgfsetlinewidth{1.003750pt}%
\definecolor{currentstroke}{rgb}{0.800000,0.800000,0.800000}%
\pgfsetstrokecolor{currentstroke}%
\pgfsetdash{}{0pt}%
\pgfpathmoveto{\pgfqpoint{0.464084in}{0.451389in}}%
\pgfpathlineto{\pgfqpoint{0.464084in}{1.770898in}}%
\pgfusepath{stroke}%
\end{pgfscope}%
\begin{pgfscope}%
\definecolor{textcolor}{rgb}{0.150000,0.150000,0.150000}%
\pgfsetstrokecolor{textcolor}%
\pgfsetfillcolor{textcolor}%
\pgftext[x=0.464084in,y=0.319444in,,top]{\color{textcolor}\sffamily\fontsize{9.000000}{10.800000}\selectfont 0}%
\end{pgfscope}%
\begin{pgfscope}%
\pgfpathrectangle{\pgfqpoint{0.464084in}{0.451389in}}{\pgfqpoint{2.197676in}{1.319509in}}%
\pgfusepath{clip}%
\pgfsetroundcap%
\pgfsetroundjoin%
\pgfsetlinewidth{1.003750pt}%
\definecolor{currentstroke}{rgb}{0.800000,0.800000,0.800000}%
\pgfsetstrokecolor{currentstroke}%
\pgfsetdash{}{0pt}%
\pgfpathmoveto{\pgfqpoint{1.157439in}{0.451389in}}%
\pgfpathlineto{\pgfqpoint{1.157439in}{1.770898in}}%
\pgfusepath{stroke}%
\end{pgfscope}%
\begin{pgfscope}%
\definecolor{textcolor}{rgb}{0.150000,0.150000,0.150000}%
\pgfsetstrokecolor{textcolor}%
\pgfsetfillcolor{textcolor}%
\pgftext[x=1.157439in,y=0.319444in,,top]{\color{textcolor}\sffamily\fontsize{9.000000}{10.800000}\selectfont 2000}%
\end{pgfscope}%
\begin{pgfscope}%
\pgfpathrectangle{\pgfqpoint{0.464084in}{0.451389in}}{\pgfqpoint{2.197676in}{1.319509in}}%
\pgfusepath{clip}%
\pgfsetroundcap%
\pgfsetroundjoin%
\pgfsetlinewidth{1.003750pt}%
\definecolor{currentstroke}{rgb}{0.800000,0.800000,0.800000}%
\pgfsetstrokecolor{currentstroke}%
\pgfsetdash{}{0pt}%
\pgfpathmoveto{\pgfqpoint{1.850794in}{0.451389in}}%
\pgfpathlineto{\pgfqpoint{1.850794in}{1.770898in}}%
\pgfusepath{stroke}%
\end{pgfscope}%
\begin{pgfscope}%
\definecolor{textcolor}{rgb}{0.150000,0.150000,0.150000}%
\pgfsetstrokecolor{textcolor}%
\pgfsetfillcolor{textcolor}%
\pgftext[x=1.850794in,y=0.319444in,,top]{\color{textcolor}\sffamily\fontsize{9.000000}{10.800000}\selectfont 4000}%
\end{pgfscope}%
\begin{pgfscope}%
\pgfpathrectangle{\pgfqpoint{0.464084in}{0.451389in}}{\pgfqpoint{2.197676in}{1.319509in}}%
\pgfusepath{clip}%
\pgfsetroundcap%
\pgfsetroundjoin%
\pgfsetlinewidth{1.003750pt}%
\definecolor{currentstroke}{rgb}{0.800000,0.800000,0.800000}%
\pgfsetstrokecolor{currentstroke}%
\pgfsetdash{}{0pt}%
\pgfpathmoveto{\pgfqpoint{2.544149in}{0.451389in}}%
\pgfpathlineto{\pgfqpoint{2.544149in}{1.770898in}}%
\pgfusepath{stroke}%
\end{pgfscope}%
\begin{pgfscope}%
\definecolor{textcolor}{rgb}{0.150000,0.150000,0.150000}%
\pgfsetstrokecolor{textcolor}%
\pgfsetfillcolor{textcolor}%
\pgftext[x=2.544149in,y=0.319444in,,top]{\color{textcolor}\sffamily\fontsize{9.000000}{10.800000}\selectfont 6000}%
\end{pgfscope}%
\begin{pgfscope}%
\definecolor{textcolor}{rgb}{0.150000,0.150000,0.150000}%
\pgfsetstrokecolor{textcolor}%
\pgfsetfillcolor{textcolor}%
\pgftext[x=1.562922in,y=0.125000in,,top]{\color{textcolor}\sffamily\fontsize{9.000000}{10.800000}\selectfont Input obstacle vertices}%
\end{pgfscope}%
\begin{pgfscope}%
\pgfpathrectangle{\pgfqpoint{0.464084in}{0.451389in}}{\pgfqpoint{2.197676in}{1.319509in}}%
\pgfusepath{clip}%
\pgfsetroundcap%
\pgfsetroundjoin%
\pgfsetlinewidth{1.003750pt}%
\definecolor{currentstroke}{rgb}{0.800000,0.800000,0.800000}%
\pgfsetstrokecolor{currentstroke}%
\pgfsetdash{}{0pt}%
\pgfpathmoveto{\pgfqpoint{0.464084in}{0.451389in}}%
\pgfpathlineto{\pgfqpoint{2.661760in}{0.451389in}}%
\pgfusepath{stroke}%
\end{pgfscope}%
\begin{pgfscope}%
\definecolor{textcolor}{rgb}{0.150000,0.150000,0.150000}%
\pgfsetstrokecolor{textcolor}%
\pgfsetfillcolor{textcolor}%
\pgftext[x=0.263292in, y=0.403903in, left, base]{\color{textcolor}\sffamily\fontsize{9.000000}{10.800000}\selectfont 0}%
\end{pgfscope}%
\begin{pgfscope}%
\pgfpathrectangle{\pgfqpoint{0.464084in}{0.451389in}}{\pgfqpoint{2.197676in}{1.319509in}}%
\pgfusepath{clip}%
\pgfsetroundcap%
\pgfsetroundjoin%
\pgfsetlinewidth{1.003750pt}%
\definecolor{currentstroke}{rgb}{0.800000,0.800000,0.800000}%
\pgfsetstrokecolor{currentstroke}%
\pgfsetdash{}{0pt}%
\pgfpathmoveto{\pgfqpoint{0.464084in}{0.856036in}}%
\pgfpathlineto{\pgfqpoint{2.661760in}{0.856036in}}%
\pgfusepath{stroke}%
\end{pgfscope}%
\begin{pgfscope}%
\definecolor{textcolor}{rgb}{0.150000,0.150000,0.150000}%
\pgfsetstrokecolor{textcolor}%
\pgfsetfillcolor{textcolor}%
\pgftext[x=0.194444in, y=0.808551in, left, base]{\color{textcolor}\sffamily\fontsize{9.000000}{10.800000}\selectfont 10}%
\end{pgfscope}%
\begin{pgfscope}%
\pgfpathrectangle{\pgfqpoint{0.464084in}{0.451389in}}{\pgfqpoint{2.197676in}{1.319509in}}%
\pgfusepath{clip}%
\pgfsetroundcap%
\pgfsetroundjoin%
\pgfsetlinewidth{1.003750pt}%
\definecolor{currentstroke}{rgb}{0.800000,0.800000,0.800000}%
\pgfsetstrokecolor{currentstroke}%
\pgfsetdash{}{0pt}%
\pgfpathmoveto{\pgfqpoint{0.464084in}{1.260683in}}%
\pgfpathlineto{\pgfqpoint{2.661760in}{1.260683in}}%
\pgfusepath{stroke}%
\end{pgfscope}%
\begin{pgfscope}%
\definecolor{textcolor}{rgb}{0.150000,0.150000,0.150000}%
\pgfsetstrokecolor{textcolor}%
\pgfsetfillcolor{textcolor}%
\pgftext[x=0.194444in, y=1.213198in, left, base]{\color{textcolor}\sffamily\fontsize{9.000000}{10.800000}\selectfont 20}%
\end{pgfscope}%
\begin{pgfscope}%
\pgfpathrectangle{\pgfqpoint{0.464084in}{0.451389in}}{\pgfqpoint{2.197676in}{1.319509in}}%
\pgfusepath{clip}%
\pgfsetroundcap%
\pgfsetroundjoin%
\pgfsetlinewidth{1.003750pt}%
\definecolor{currentstroke}{rgb}{0.800000,0.800000,0.800000}%
\pgfsetstrokecolor{currentstroke}%
\pgfsetdash{}{0pt}%
\pgfpathmoveto{\pgfqpoint{0.464084in}{1.665331in}}%
\pgfpathlineto{\pgfqpoint{2.661760in}{1.665331in}}%
\pgfusepath{stroke}%
\end{pgfscope}%
\begin{pgfscope}%
\definecolor{textcolor}{rgb}{0.150000,0.150000,0.150000}%
\pgfsetstrokecolor{textcolor}%
\pgfsetfillcolor{textcolor}%
\pgftext[x=0.194444in, y=1.617845in, left, base]{\color{textcolor}\sffamily\fontsize{9.000000}{10.800000}\selectfont 30}%
\end{pgfscope}%
\begin{pgfscope}%
\definecolor{textcolor}{rgb}{0.150000,0.150000,0.150000}%
\pgfsetstrokecolor{textcolor}%
\pgfsetfillcolor{textcolor}%
\pgftext[x=0.125000in,y=1.111143in,,bottom,rotate=90.000000]{\color{textcolor}\sffamily\fontsize{9.000000}{10.800000}\selectfont Time in s}%
\end{pgfscope}%
\begin{pgfscope}%
\pgfpathrectangle{\pgfqpoint{0.464084in}{0.451389in}}{\pgfqpoint{2.197676in}{1.319509in}}%
\pgfusepath{clip}%
\pgfsetbuttcap%
\pgfsetroundjoin%
\definecolor{currentfill}{rgb}{0.003922,0.450980,0.698039}%
\pgfsetfillcolor{currentfill}%
\pgfsetfillopacity{0.200000}%
\pgfsetlinewidth{1.003750pt}%
\definecolor{currentstroke}{rgb}{0.003922,0.450980,0.698039}%
\pgfsetstrokecolor{currentstroke}%
\pgfsetstrokeopacity{0.200000}%
\pgfsetdash{}{0pt}%
\pgfsys@defobject{currentmarker}{\pgfqpoint{0.592355in}{0.487558in}}{\pgfqpoint{2.563217in}{1.709786in}}{%
\pgfpathmoveto{\pgfqpoint{0.592355in}{0.487916in}}%
\pgfpathlineto{\pgfqpoint{0.592355in}{0.487558in}}%
\pgfpathlineto{\pgfqpoint{0.781294in}{0.567670in}}%
\pgfpathlineto{\pgfqpoint{1.001781in}{0.639840in}}%
\pgfpathlineto{\pgfqpoint{1.249309in}{0.731963in}}%
\pgfpathlineto{\pgfqpoint{1.988772in}{1.241469in}}%
\pgfpathlineto{\pgfqpoint{2.563217in}{1.701472in}}%
\pgfpathlineto{\pgfqpoint{2.563217in}{1.709786in}}%
\pgfpathlineto{\pgfqpoint{2.563217in}{1.709786in}}%
\pgfpathlineto{\pgfqpoint{1.988772in}{1.243571in}}%
\pgfpathlineto{\pgfqpoint{1.249309in}{0.741756in}}%
\pgfpathlineto{\pgfqpoint{1.001781in}{0.644072in}}%
\pgfpathlineto{\pgfqpoint{0.781294in}{0.569432in}}%
\pgfpathlineto{\pgfqpoint{0.592355in}{0.487916in}}%
\pgfpathlineto{\pgfqpoint{0.592355in}{0.487916in}}%
\pgfpathclose%
\pgfusepath{stroke,fill}%
}%
\begin{pgfscope}%
\pgfsys@transformshift{0.000000in}{0.000000in}%
\pgfsys@useobject{currentmarker}{}%
\end{pgfscope}%
\end{pgfscope}%
\begin{pgfscope}%
\pgfsetrectcap%
\pgfsetmiterjoin%
\pgfsetlinewidth{1.254687pt}%
\definecolor{currentstroke}{rgb}{0.800000,0.800000,0.800000}%
\pgfsetstrokecolor{currentstroke}%
\pgfsetdash{}{0pt}%
\pgfpathmoveto{\pgfqpoint{0.464084in}{0.451389in}}%
\pgfpathlineto{\pgfqpoint{0.464084in}{1.770898in}}%
\pgfusepath{stroke}%
\end{pgfscope}%
\begin{pgfscope}%
\pgfsetrectcap%
\pgfsetmiterjoin%
\pgfsetlinewidth{1.254687pt}%
\definecolor{currentstroke}{rgb}{0.800000,0.800000,0.800000}%
\pgfsetstrokecolor{currentstroke}%
\pgfsetdash{}{0pt}%
\pgfpathmoveto{\pgfqpoint{2.661760in}{0.451389in}}%
\pgfpathlineto{\pgfqpoint{2.661760in}{1.770898in}}%
\pgfusepath{stroke}%
\end{pgfscope}%
\begin{pgfscope}%
\pgfsetrectcap%
\pgfsetmiterjoin%
\pgfsetlinewidth{1.254687pt}%
\definecolor{currentstroke}{rgb}{0.800000,0.800000,0.800000}%
\pgfsetstrokecolor{currentstroke}%
\pgfsetdash{}{0pt}%
\pgfpathmoveto{\pgfqpoint{0.464084in}{0.451389in}}%
\pgfpathlineto{\pgfqpoint{2.661760in}{0.451389in}}%
\pgfusepath{stroke}%
\end{pgfscope}%
\begin{pgfscope}%
\pgfsetrectcap%
\pgfsetmiterjoin%
\pgfsetlinewidth{1.254687pt}%
\definecolor{currentstroke}{rgb}{0.800000,0.800000,0.800000}%
\pgfsetstrokecolor{currentstroke}%
\pgfsetdash{}{0pt}%
\pgfpathmoveto{\pgfqpoint{0.464084in}{1.770898in}}%
\pgfpathlineto{\pgfqpoint{2.661760in}{1.770898in}}%
\pgfusepath{stroke}%
\end{pgfscope}%
\begin{pgfscope}%
\pgfsetroundcap%
\pgfsetroundjoin%
\pgfsetlinewidth{1.003750pt}%
\definecolor{currentstroke}{rgb}{0.003922,0.450980,0.698039}%
\pgfsetstrokecolor{currentstroke}%
\pgfsetdash{}{0pt}%
\pgfpathmoveto{\pgfqpoint{0.592355in}{0.487709in}}%
\pgfpathlineto{\pgfqpoint{0.781294in}{0.568906in}}%
\pgfpathlineto{\pgfqpoint{1.001781in}{0.641574in}}%
\pgfpathlineto{\pgfqpoint{1.249309in}{0.737136in}}%
\pgfpathlineto{\pgfqpoint{1.988772in}{1.242395in}}%
\pgfpathlineto{\pgfqpoint{2.563217in}{1.705259in}}%
\pgfusepath{stroke}%
\end{pgfscope}%
\begin{pgfscope}%
\pgfsetbuttcap%
\pgfsetroundjoin%
\definecolor{currentfill}{rgb}{0.003922,0.450980,0.698039}%
\pgfsetfillcolor{currentfill}%
\pgfsetlinewidth{0.752812pt}%
\definecolor{currentstroke}{rgb}{1.000000,1.000000,1.000000}%
\pgfsetstrokecolor{currentstroke}%
\pgfsetdash{}{0pt}%
\pgfsys@defobject{currentmarker}{\pgfqpoint{-0.034722in}{-0.034722in}}{\pgfqpoint{0.034722in}{0.034722in}}{%
\pgfpathmoveto{\pgfqpoint{0.000000in}{-0.034722in}}%
\pgfpathcurveto{\pgfqpoint{0.009208in}{-0.034722in}}{\pgfqpoint{0.018041in}{-0.031064in}}{\pgfqpoint{0.024552in}{-0.024552in}}%
\pgfpathcurveto{\pgfqpoint{0.031064in}{-0.018041in}}{\pgfqpoint{0.034722in}{-0.009208in}}{\pgfqpoint{0.034722in}{0.000000in}}%
\pgfpathcurveto{\pgfqpoint{0.034722in}{0.009208in}}{\pgfqpoint{0.031064in}{0.018041in}}{\pgfqpoint{0.024552in}{0.024552in}}%
\pgfpathcurveto{\pgfqpoint{0.018041in}{0.031064in}}{\pgfqpoint{0.009208in}{0.034722in}}{\pgfqpoint{0.000000in}{0.034722in}}%
\pgfpathcurveto{\pgfqpoint{-0.009208in}{0.034722in}}{\pgfqpoint{-0.018041in}{0.031064in}}{\pgfqpoint{-0.024552in}{0.024552in}}%
\pgfpathcurveto{\pgfqpoint{-0.031064in}{0.018041in}}{\pgfqpoint{-0.034722in}{0.009208in}}{\pgfqpoint{-0.034722in}{0.000000in}}%
\pgfpathcurveto{\pgfqpoint{-0.034722in}{-0.009208in}}{\pgfqpoint{-0.031064in}{-0.018041in}}{\pgfqpoint{-0.024552in}{-0.024552in}}%
\pgfpathcurveto{\pgfqpoint{-0.018041in}{-0.031064in}}{\pgfqpoint{-0.009208in}{-0.034722in}}{\pgfqpoint{0.000000in}{-0.034722in}}%
\pgfpathlineto{\pgfqpoint{0.000000in}{-0.034722in}}%
\pgfpathclose%
\pgfusepath{stroke,fill}%
}%
\begin{pgfscope}%
\pgfsys@transformshift{0.592355in}{0.487709in}%
\pgfsys@useobject{currentmarker}{}%
\end{pgfscope}%
\begin{pgfscope}%
\pgfsys@transformshift{0.781294in}{0.568906in}%
\pgfsys@useobject{currentmarker}{}%
\end{pgfscope}%
\begin{pgfscope}%
\pgfsys@transformshift{1.001781in}{0.641574in}%
\pgfsys@useobject{currentmarker}{}%
\end{pgfscope}%
\begin{pgfscope}%
\pgfsys@transformshift{1.249309in}{0.737136in}%
\pgfsys@useobject{currentmarker}{}%
\end{pgfscope}%
\begin{pgfscope}%
\pgfsys@transformshift{1.988772in}{1.242395in}%
\pgfsys@useobject{currentmarker}{}%
\end{pgfscope}%
\begin{pgfscope}%
\pgfsys@transformshift{2.563217in}{1.705259in}%
\pgfsys@useobject{currentmarker}{}%
\end{pgfscope}%
\end{pgfscope}%
\end{pgfpicture}%
\makeatother%
\endgroup%

						\end{figcenter}
						\caption{Total graph generation times.}
						\label{fig:eval-import-city-abs}
					\end{subfigure}
					\\[3ex]
					\begin{subfigure}[t]{\linewidth}
						\begin{figcenter}
							\begingroup%
\makeatletter%
\begin{pgfpicture}%
\pgfpathrectangle{\pgfpointorigin}{\pgfqpoint{3.042524in}{1.867995in}}%
\pgfusepath{use as bounding box}%
\begin{pgfscope}%
\pgfsetbuttcap%
\pgfsetmiterjoin%
\definecolor{currentfill}{rgb}{1.000000,1.000000,1.000000}%
\pgfsetfillcolor{currentfill}%
\pgfsetlinewidth{0.000000pt}%
\definecolor{currentstroke}{rgb}{1.000000,1.000000,1.000000}%
\pgfsetstrokecolor{currentstroke}%
\pgfsetdash{}{0pt}%
\pgfpathmoveto{\pgfqpoint{0.000000in}{0.000000in}}%
\pgfpathlineto{\pgfqpoint{3.042524in}{0.000000in}}%
\pgfpathlineto{\pgfqpoint{3.042524in}{1.867995in}}%
\pgfpathlineto{\pgfqpoint{0.000000in}{1.867995in}}%
\pgfpathlineto{\pgfqpoint{0.000000in}{0.000000in}}%
\pgfpathclose%
\pgfusepath{fill}%
\end{pgfscope}%
\begin{pgfscope}%
\pgfsetbuttcap%
\pgfsetmiterjoin%
\definecolor{currentfill}{rgb}{1.000000,1.000000,1.000000}%
\pgfsetfillcolor{currentfill}%
\pgfsetlinewidth{0.000000pt}%
\definecolor{currentstroke}{rgb}{0.000000,0.000000,0.000000}%
\pgfsetstrokecolor{currentstroke}%
\pgfsetstrokeopacity{0.000000}%
\pgfsetdash{}{0pt}%
\pgfpathmoveto{\pgfqpoint{0.566440in}{0.451389in}}%
\pgfpathlineto{\pgfqpoint{3.042524in}{0.451389in}}%
\pgfpathlineto{\pgfqpoint{3.042524in}{1.867995in}}%
\pgfpathlineto{\pgfqpoint{0.566440in}{1.867995in}}%
\pgfpathlineto{\pgfqpoint{0.566440in}{0.451389in}}%
\pgfpathclose%
\pgfusepath{fill}%
\end{pgfscope}%
\begin{pgfscope}%
\pgfpathrectangle{\pgfqpoint{0.566440in}{0.451389in}}{\pgfqpoint{2.476084in}{1.416606in}}%
\pgfusepath{clip}%
\pgfsetroundcap%
\pgfsetroundjoin%
\pgfsetlinewidth{1.003750pt}%
\definecolor{currentstroke}{rgb}{0.800000,0.800000,0.800000}%
\pgfsetstrokecolor{currentstroke}%
\pgfsetdash{}{0pt}%
\pgfpathmoveto{\pgfqpoint{0.566440in}{0.451389in}}%
\pgfpathlineto{\pgfqpoint{0.566440in}{1.867995in}}%
\pgfusepath{stroke}%
\end{pgfscope}%
\begin{pgfscope}%
\definecolor{textcolor}{rgb}{0.150000,0.150000,0.150000}%
\pgfsetstrokecolor{textcolor}%
\pgfsetfillcolor{textcolor}%
\pgftext[x=0.566440in,y=0.319444in,,top]{\color{textcolor}\sffamily\fontsize{9.000000}{10.800000}\selectfont 0}%
\end{pgfscope}%
\begin{pgfscope}%
\pgfpathrectangle{\pgfqpoint{0.566440in}{0.451389in}}{\pgfqpoint{2.476084in}{1.416606in}}%
\pgfusepath{clip}%
\pgfsetroundcap%
\pgfsetroundjoin%
\pgfsetlinewidth{1.003750pt}%
\definecolor{currentstroke}{rgb}{0.800000,0.800000,0.800000}%
\pgfsetstrokecolor{currentstroke}%
\pgfsetdash{}{0pt}%
\pgfpathmoveto{\pgfqpoint{1.094231in}{0.451389in}}%
\pgfpathlineto{\pgfqpoint{1.094231in}{1.867995in}}%
\pgfusepath{stroke}%
\end{pgfscope}%
\begin{pgfscope}%
\definecolor{textcolor}{rgb}{0.150000,0.150000,0.150000}%
\pgfsetstrokecolor{textcolor}%
\pgfsetfillcolor{textcolor}%
\pgftext[x=1.094231in,y=0.319444in,,top]{\color{textcolor}\sffamily\fontsize{9.000000}{10.800000}\selectfont 10000}%
\end{pgfscope}%
\begin{pgfscope}%
\pgfpathrectangle{\pgfqpoint{0.566440in}{0.451389in}}{\pgfqpoint{2.476084in}{1.416606in}}%
\pgfusepath{clip}%
\pgfsetroundcap%
\pgfsetroundjoin%
\pgfsetlinewidth{1.003750pt}%
\definecolor{currentstroke}{rgb}{0.800000,0.800000,0.800000}%
\pgfsetstrokecolor{currentstroke}%
\pgfsetdash{}{0pt}%
\pgfpathmoveto{\pgfqpoint{1.622022in}{0.451389in}}%
\pgfpathlineto{\pgfqpoint{1.622022in}{1.867995in}}%
\pgfusepath{stroke}%
\end{pgfscope}%
\begin{pgfscope}%
\definecolor{textcolor}{rgb}{0.150000,0.150000,0.150000}%
\pgfsetstrokecolor{textcolor}%
\pgfsetfillcolor{textcolor}%
\pgftext[x=1.622022in,y=0.319444in,,top]{\color{textcolor}\sffamily\fontsize{9.000000}{10.800000}\selectfont 20000}%
\end{pgfscope}%
\begin{pgfscope}%
\pgfpathrectangle{\pgfqpoint{0.566440in}{0.451389in}}{\pgfqpoint{2.476084in}{1.416606in}}%
\pgfusepath{clip}%
\pgfsetroundcap%
\pgfsetroundjoin%
\pgfsetlinewidth{1.003750pt}%
\definecolor{currentstroke}{rgb}{0.800000,0.800000,0.800000}%
\pgfsetstrokecolor{currentstroke}%
\pgfsetdash{}{0pt}%
\pgfpathmoveto{\pgfqpoint{2.149813in}{0.451389in}}%
\pgfpathlineto{\pgfqpoint{2.149813in}{1.867995in}}%
\pgfusepath{stroke}%
\end{pgfscope}%
\begin{pgfscope}%
\definecolor{textcolor}{rgb}{0.150000,0.150000,0.150000}%
\pgfsetstrokecolor{textcolor}%
\pgfsetfillcolor{textcolor}%
\pgftext[x=2.149813in,y=0.319444in,,top]{\color{textcolor}\sffamily\fontsize{9.000000}{10.800000}\selectfont 30000}%
\end{pgfscope}%
\begin{pgfscope}%
\pgfpathrectangle{\pgfqpoint{0.566440in}{0.451389in}}{\pgfqpoint{2.476084in}{1.416606in}}%
\pgfusepath{clip}%
\pgfsetroundcap%
\pgfsetroundjoin%
\pgfsetlinewidth{1.003750pt}%
\definecolor{currentstroke}{rgb}{0.800000,0.800000,0.800000}%
\pgfsetstrokecolor{currentstroke}%
\pgfsetdash{}{0pt}%
\pgfpathmoveto{\pgfqpoint{2.677604in}{0.451389in}}%
\pgfpathlineto{\pgfqpoint{2.677604in}{1.867995in}}%
\pgfusepath{stroke}%
\end{pgfscope}%
\begin{pgfscope}%
\definecolor{textcolor}{rgb}{0.150000,0.150000,0.150000}%
\pgfsetstrokecolor{textcolor}%
\pgfsetfillcolor{textcolor}%
\pgftext[x=2.677604in,y=0.319444in,,top]{\color{textcolor}\sffamily\fontsize{9.000000}{10.800000}\selectfont 40000}%
\end{pgfscope}%
\begin{pgfscope}%
\definecolor{textcolor}{rgb}{0.150000,0.150000,0.150000}%
\pgfsetstrokecolor{textcolor}%
\pgfsetfillcolor{textcolor}%
\pgftext[x=1.804482in,y=0.125000in,,top]{\color{textcolor}\sffamily\fontsize{9.000000}{10.800000}\selectfont Input obstacle vertices}%
\end{pgfscope}%
\begin{pgfscope}%
\pgfpathrectangle{\pgfqpoint{0.566440in}{0.451389in}}{\pgfqpoint{2.476084in}{1.416606in}}%
\pgfusepath{clip}%
\pgfsetroundcap%
\pgfsetroundjoin%
\pgfsetlinewidth{1.003750pt}%
\definecolor{currentstroke}{rgb}{0.800000,0.800000,0.800000}%
\pgfsetstrokecolor{currentstroke}%
\pgfsetdash{}{0pt}%
\pgfpathmoveto{\pgfqpoint{0.566440in}{0.451389in}}%
\pgfpathlineto{\pgfqpoint{3.042524in}{0.451389in}}%
\pgfusepath{stroke}%
\end{pgfscope}%
\begin{pgfscope}%
\definecolor{textcolor}{rgb}{0.150000,0.150000,0.150000}%
\pgfsetstrokecolor{textcolor}%
\pgfsetfillcolor{textcolor}%
\pgftext[x=0.263292in, y=0.403903in, left, base]{\color{textcolor}\sffamily\fontsize{9.000000}{10.800000}\selectfont 0.0}%
\end{pgfscope}%
\begin{pgfscope}%
\pgfpathrectangle{\pgfqpoint{0.566440in}{0.451389in}}{\pgfqpoint{2.476084in}{1.416606in}}%
\pgfusepath{clip}%
\pgfsetroundcap%
\pgfsetroundjoin%
\pgfsetlinewidth{1.003750pt}%
\definecolor{currentstroke}{rgb}{0.800000,0.800000,0.800000}%
\pgfsetstrokecolor{currentstroke}%
\pgfsetdash{}{0pt}%
\pgfpathmoveto{\pgfqpoint{0.566440in}{0.791652in}}%
\pgfpathlineto{\pgfqpoint{3.042524in}{0.791652in}}%
\pgfusepath{stroke}%
\end{pgfscope}%
\begin{pgfscope}%
\definecolor{textcolor}{rgb}{0.150000,0.150000,0.150000}%
\pgfsetstrokecolor{textcolor}%
\pgfsetfillcolor{textcolor}%
\pgftext[x=0.263292in, y=0.744167in, left, base]{\color{textcolor}\sffamily\fontsize{9.000000}{10.800000}\selectfont 2.5}%
\end{pgfscope}%
\begin{pgfscope}%
\pgfpathrectangle{\pgfqpoint{0.566440in}{0.451389in}}{\pgfqpoint{2.476084in}{1.416606in}}%
\pgfusepath{clip}%
\pgfsetroundcap%
\pgfsetroundjoin%
\pgfsetlinewidth{1.003750pt}%
\definecolor{currentstroke}{rgb}{0.800000,0.800000,0.800000}%
\pgfsetstrokecolor{currentstroke}%
\pgfsetdash{}{0pt}%
\pgfpathmoveto{\pgfqpoint{0.566440in}{1.131915in}}%
\pgfpathlineto{\pgfqpoint{3.042524in}{1.131915in}}%
\pgfusepath{stroke}%
\end{pgfscope}%
\begin{pgfscope}%
\definecolor{textcolor}{rgb}{0.150000,0.150000,0.150000}%
\pgfsetstrokecolor{textcolor}%
\pgfsetfillcolor{textcolor}%
\pgftext[x=0.263292in, y=1.084430in, left, base]{\color{textcolor}\sffamily\fontsize{9.000000}{10.800000}\selectfont 5.0}%
\end{pgfscope}%
\begin{pgfscope}%
\pgfpathrectangle{\pgfqpoint{0.566440in}{0.451389in}}{\pgfqpoint{2.476084in}{1.416606in}}%
\pgfusepath{clip}%
\pgfsetroundcap%
\pgfsetroundjoin%
\pgfsetlinewidth{1.003750pt}%
\definecolor{currentstroke}{rgb}{0.800000,0.800000,0.800000}%
\pgfsetstrokecolor{currentstroke}%
\pgfsetdash{}{0pt}%
\pgfpathmoveto{\pgfqpoint{0.566440in}{1.472179in}}%
\pgfpathlineto{\pgfqpoint{3.042524in}{1.472179in}}%
\pgfusepath{stroke}%
\end{pgfscope}%
\begin{pgfscope}%
\definecolor{textcolor}{rgb}{0.150000,0.150000,0.150000}%
\pgfsetstrokecolor{textcolor}%
\pgfsetfillcolor{textcolor}%
\pgftext[x=0.263292in, y=1.424693in, left, base]{\color{textcolor}\sffamily\fontsize{9.000000}{10.800000}\selectfont 7.5}%
\end{pgfscope}%
\begin{pgfscope}%
\pgfpathrectangle{\pgfqpoint{0.566440in}{0.451389in}}{\pgfqpoint{2.476084in}{1.416606in}}%
\pgfusepath{clip}%
\pgfsetroundcap%
\pgfsetroundjoin%
\pgfsetlinewidth{1.003750pt}%
\definecolor{currentstroke}{rgb}{0.800000,0.800000,0.800000}%
\pgfsetstrokecolor{currentstroke}%
\pgfsetdash{}{0pt}%
\pgfpathmoveto{\pgfqpoint{0.566440in}{1.812442in}}%
\pgfpathlineto{\pgfqpoint{3.042524in}{1.812442in}}%
\pgfusepath{stroke}%
\end{pgfscope}%
\begin{pgfscope}%
\definecolor{textcolor}{rgb}{0.150000,0.150000,0.150000}%
\pgfsetstrokecolor{textcolor}%
\pgfsetfillcolor{textcolor}%
\pgftext[x=0.194444in, y=1.764957in, left, base]{\color{textcolor}\sffamily\fontsize{9.000000}{10.800000}\selectfont 10.0}%
\end{pgfscope}%
\begin{pgfscope}%
\definecolor{textcolor}{rgb}{0.150000,0.150000,0.150000}%
\pgfsetstrokecolor{textcolor}%
\pgfsetfillcolor{textcolor}%
\pgftext[x=0.125000in,y=1.159692in,,bottom,rotate=90.000000]{\color{textcolor}\sffamily\fontsize{9.000000}{10.800000}\selectfont Time in ms}%
\end{pgfscope}%
\begin{pgfscope}%
\pgfpathrectangle{\pgfqpoint{0.566440in}{0.451389in}}{\pgfqpoint{2.476084in}{1.416606in}}%
\pgfusepath{clip}%
\pgfsetbuttcap%
\pgfsetroundjoin%
\definecolor{currentfill}{rgb}{0.003922,0.450980,0.698039}%
\pgfsetfillcolor{currentfill}%
\pgfsetfillopacity{0.200000}%
\pgfsetlinewidth{1.003750pt}%
\definecolor{currentstroke}{rgb}{0.003922,0.450980,0.698039}%
\pgfsetstrokecolor{currentstroke}%
\pgfsetstrokeopacity{0.200000}%
\pgfsetdash{}{0pt}%
\pgfsys@defobject{currentmarker}{\pgfqpoint{0.940961in}{0.636736in}}{\pgfqpoint{2.942450in}{1.809363in}}{%
\pgfpathmoveto{\pgfqpoint{0.940961in}{0.643347in}}%
\pgfpathlineto{\pgfqpoint{0.940961in}{0.636736in}}%
\pgfpathlineto{\pgfqpoint{1.245074in}{0.753707in}}%
\pgfpathlineto{\pgfqpoint{1.510869in}{0.858688in}}%
\pgfpathlineto{\pgfqpoint{1.769857in}{0.988946in}}%
\pgfpathlineto{\pgfqpoint{2.372224in}{1.312447in}}%
\pgfpathlineto{\pgfqpoint{2.942450in}{1.741654in}}%
\pgfpathlineto{\pgfqpoint{2.942450in}{1.809363in}}%
\pgfpathlineto{\pgfqpoint{2.942450in}{1.809363in}}%
\pgfpathlineto{\pgfqpoint{2.372224in}{1.323634in}}%
\pgfpathlineto{\pgfqpoint{1.769857in}{1.000971in}}%
\pgfpathlineto{\pgfqpoint{1.510869in}{0.867902in}}%
\pgfpathlineto{\pgfqpoint{1.245074in}{0.763513in}}%
\pgfpathlineto{\pgfqpoint{0.940961in}{0.643347in}}%
\pgfpathlineto{\pgfqpoint{0.940961in}{0.643347in}}%
\pgfpathclose%
\pgfusepath{stroke,fill}%
}%
\begin{pgfscope}%
\pgfsys@transformshift{0.000000in}{0.000000in}%
\pgfsys@useobject{currentmarker}{}%
\end{pgfscope}%
\end{pgfscope}%
\begin{pgfscope}%
\pgfsetrectcap%
\pgfsetmiterjoin%
\pgfsetlinewidth{1.254687pt}%
\definecolor{currentstroke}{rgb}{0.800000,0.800000,0.800000}%
\pgfsetstrokecolor{currentstroke}%
\pgfsetdash{}{0pt}%
\pgfpathmoveto{\pgfqpoint{0.566440in}{0.451389in}}%
\pgfpathlineto{\pgfqpoint{0.566440in}{1.867995in}}%
\pgfusepath{stroke}%
\end{pgfscope}%
\begin{pgfscope}%
\pgfsetrectcap%
\pgfsetmiterjoin%
\pgfsetlinewidth{1.254687pt}%
\definecolor{currentstroke}{rgb}{0.800000,0.800000,0.800000}%
\pgfsetstrokecolor{currentstroke}%
\pgfsetdash{}{0pt}%
\pgfpathmoveto{\pgfqpoint{3.042524in}{0.451389in}}%
\pgfpathlineto{\pgfqpoint{3.042524in}{1.867995in}}%
\pgfusepath{stroke}%
\end{pgfscope}%
\begin{pgfscope}%
\pgfsetrectcap%
\pgfsetmiterjoin%
\pgfsetlinewidth{1.254687pt}%
\definecolor{currentstroke}{rgb}{0.800000,0.800000,0.800000}%
\pgfsetstrokecolor{currentstroke}%
\pgfsetdash{}{0pt}%
\pgfpathmoveto{\pgfqpoint{0.566440in}{0.451389in}}%
\pgfpathlineto{\pgfqpoint{3.042524in}{0.451389in}}%
\pgfusepath{stroke}%
\end{pgfscope}%
\begin{pgfscope}%
\pgfsetrectcap%
\pgfsetmiterjoin%
\pgfsetlinewidth{1.254687pt}%
\definecolor{currentstroke}{rgb}{0.800000,0.800000,0.800000}%
\pgfsetstrokecolor{currentstroke}%
\pgfsetdash{}{0pt}%
\pgfpathmoveto{\pgfqpoint{0.566440in}{1.867995in}}%
\pgfpathlineto{\pgfqpoint{3.042524in}{1.867995in}}%
\pgfusepath{stroke}%
\end{pgfscope}%
\begin{pgfscope}%
\pgfsetroundcap%
\pgfsetroundjoin%
\pgfsetlinewidth{1.003750pt}%
\definecolor{currentstroke}{rgb}{0.003922,0.450980,0.698039}%
\pgfsetstrokecolor{currentstroke}%
\pgfsetdash{}{0pt}%
\pgfpathmoveto{\pgfqpoint{0.940961in}{0.639510in}}%
\pgfpathlineto{\pgfqpoint{1.245074in}{0.759212in}}%
\pgfpathlineto{\pgfqpoint{1.510869in}{0.863992in}}%
\pgfpathlineto{\pgfqpoint{1.769857in}{0.996407in}}%
\pgfpathlineto{\pgfqpoint{2.372224in}{1.317860in}}%
\pgfpathlineto{\pgfqpoint{2.942450in}{1.773630in}}%
\pgfusepath{stroke}%
\end{pgfscope}%
\begin{pgfscope}%
\pgfsetbuttcap%
\pgfsetroundjoin%
\definecolor{currentfill}{rgb}{0.003922,0.450980,0.698039}%
\pgfsetfillcolor{currentfill}%
\pgfsetlinewidth{0.752812pt}%
\definecolor{currentstroke}{rgb}{1.000000,1.000000,1.000000}%
\pgfsetstrokecolor{currentstroke}%
\pgfsetdash{}{0pt}%
\pgfsys@defobject{currentmarker}{\pgfqpoint{-0.034722in}{-0.034722in}}{\pgfqpoint{0.034722in}{0.034722in}}{%
\pgfpathmoveto{\pgfqpoint{0.000000in}{-0.034722in}}%
\pgfpathcurveto{\pgfqpoint{0.009208in}{-0.034722in}}{\pgfqpoint{0.018041in}{-0.031064in}}{\pgfqpoint{0.024552in}{-0.024552in}}%
\pgfpathcurveto{\pgfqpoint{0.031064in}{-0.018041in}}{\pgfqpoint{0.034722in}{-0.009208in}}{\pgfqpoint{0.034722in}{0.000000in}}%
\pgfpathcurveto{\pgfqpoint{0.034722in}{0.009208in}}{\pgfqpoint{0.031064in}{0.018041in}}{\pgfqpoint{0.024552in}{0.024552in}}%
\pgfpathcurveto{\pgfqpoint{0.018041in}{0.031064in}}{\pgfqpoint{0.009208in}{0.034722in}}{\pgfqpoint{0.000000in}{0.034722in}}%
\pgfpathcurveto{\pgfqpoint{-0.009208in}{0.034722in}}{\pgfqpoint{-0.018041in}{0.031064in}}{\pgfqpoint{-0.024552in}{0.024552in}}%
\pgfpathcurveto{\pgfqpoint{-0.031064in}{0.018041in}}{\pgfqpoint{-0.034722in}{0.009208in}}{\pgfqpoint{-0.034722in}{0.000000in}}%
\pgfpathcurveto{\pgfqpoint{-0.034722in}{-0.009208in}}{\pgfqpoint{-0.031064in}{-0.018041in}}{\pgfqpoint{-0.024552in}{-0.024552in}}%
\pgfpathcurveto{\pgfqpoint{-0.018041in}{-0.031064in}}{\pgfqpoint{-0.009208in}{-0.034722in}}{\pgfqpoint{0.000000in}{-0.034722in}}%
\pgfpathlineto{\pgfqpoint{0.000000in}{-0.034722in}}%
\pgfpathclose%
\pgfusepath{stroke,fill}%
}%
\begin{pgfscope}%
\pgfsys@transformshift{0.940961in}{0.639510in}%
\pgfsys@useobject{currentmarker}{}%
\end{pgfscope}%
\begin{pgfscope}%
\pgfsys@transformshift{1.245074in}{0.759212in}%
\pgfsys@useobject{currentmarker}{}%
\end{pgfscope}%
\begin{pgfscope}%
\pgfsys@transformshift{1.510869in}{0.863992in}%
\pgfsys@useobject{currentmarker}{}%
\end{pgfscope}%
\begin{pgfscope}%
\pgfsys@transformshift{1.769857in}{0.996407in}%
\pgfsys@useobject{currentmarker}{}%
\end{pgfscope}%
\begin{pgfscope}%
\pgfsys@transformshift{2.372224in}{1.317860in}%
\pgfsys@useobject{currentmarker}{}%
\end{pgfscope}%
\begin{pgfscope}%
\pgfsys@transformshift{2.942450in}{1.773630in}%
\pgfsys@useobject{currentmarker}{}%
\end{pgfscope}%
\end{pgfscope}%
\end{pgfpicture}%
\makeatother%
\endgroup%

						\end{figcenter}
						\caption{Graph generation times per input vertex.}
						\label{fig:eval-import-city-rel}
					\end{subfigure}
%					\\[3ex]
%					\begin{subfigure}[t]{\linewidth}
%						\begin{figcenter}
%							%% Creator: Matplotlib, PGF backend
%%
%% To include the figure in your LaTeX document, write
%%   \input{<filename>.pgf}
%%
%% Make sure the required packages are loaded in your preamble
%%   \usepackage{pgf}
%%
%% Also ensure that all the required font packages are loaded; for instance,
%% the lmodern package is sometimes necessary when using math font.
%%   \usepackage{lmodern}
%%
%% Figures using additional raster images can only be included by \input if
%% they are in the same directory as the main LaTeX file. For loading figures
%% from other directories you can use the `import` package
%%   \usepackage{import}
%%
%% and then include the figures with
%%   \import{<path to file>}{<filename>.pgf}
%%
%% Matplotlib used the following preamble
%%   
%%   \usepackage{fontspec}
%%   \setmainfont{DejaVuSerif.ttf}[Path=\detokenize{/home/hauke/.local/lib/python3.11/site-packages/matplotlib/mpl-data/fonts/ttf/}]
%%   \setsansfont{DroidSans.ttf}[Path=\detokenize{/usr/share/fonts/droid/}]
%%   \setmonofont{DejaVuSansMono.ttf}[Path=\detokenize{/home/hauke/.local/lib/python3.11/site-packages/matplotlib/mpl-data/fonts/ttf/}]
%%   \makeatletter\@ifpackageloaded{underscore}{}{\usepackage[strings]{underscore}}\makeatother
%%
\begingroup%
\makeatletter%
\begin{pgfpicture}%
\pgfpathrectangle{\pgfpointorigin}{\pgfqpoint{2.681552in}{1.770898in}}%
\pgfusepath{use as bounding box, clip}%
\begin{pgfscope}%
\pgfsetbuttcap%
\pgfsetmiterjoin%
\definecolor{currentfill}{rgb}{1.000000,1.000000,1.000000}%
\pgfsetfillcolor{currentfill}%
\pgfsetlinewidth{0.000000pt}%
\definecolor{currentstroke}{rgb}{1.000000,1.000000,1.000000}%
\pgfsetstrokecolor{currentstroke}%
\pgfsetdash{}{0pt}%
\pgfpathmoveto{\pgfqpoint{0.000000in}{0.000000in}}%
\pgfpathlineto{\pgfqpoint{2.681552in}{0.000000in}}%
\pgfpathlineto{\pgfqpoint{2.681552in}{1.770898in}}%
\pgfpathlineto{\pgfqpoint{0.000000in}{1.770898in}}%
\pgfpathlineto{\pgfqpoint{0.000000in}{0.000000in}}%
\pgfpathclose%
\pgfusepath{fill}%
\end{pgfscope}%
\begin{pgfscope}%
\pgfsetbuttcap%
\pgfsetmiterjoin%
\definecolor{currentfill}{rgb}{1.000000,1.000000,1.000000}%
\pgfsetfillcolor{currentfill}%
\pgfsetlinewidth{0.000000pt}%
\definecolor{currentstroke}{rgb}{0.000000,0.000000,0.000000}%
\pgfsetstrokecolor{currentstroke}%
\pgfsetstrokeopacity{0.000000}%
\pgfsetdash{}{0pt}%
\pgfpathmoveto{\pgfqpoint{0.395236in}{0.451389in}}%
\pgfpathlineto{\pgfqpoint{2.667104in}{0.451389in}}%
\pgfpathlineto{\pgfqpoint{2.667104in}{1.770898in}}%
\pgfpathlineto{\pgfqpoint{0.395236in}{1.770898in}}%
\pgfpathlineto{\pgfqpoint{0.395236in}{0.451389in}}%
\pgfpathclose%
\pgfusepath{fill}%
\end{pgfscope}%
\begin{pgfscope}%
\pgfpathrectangle{\pgfqpoint{0.395236in}{0.451389in}}{\pgfqpoint{2.271868in}{1.319509in}}%
\pgfusepath{clip}%
\pgfsetroundcap%
\pgfsetroundjoin%
\pgfsetlinewidth{1.003750pt}%
\definecolor{currentstroke}{rgb}{0.800000,0.800000,0.800000}%
\pgfsetstrokecolor{currentstroke}%
\pgfsetdash{}{0pt}%
\pgfpathmoveto{\pgfqpoint{1.090675in}{0.451389in}}%
\pgfpathlineto{\pgfqpoint{1.090675in}{1.770898in}}%
\pgfusepath{stroke}%
\end{pgfscope}%
\begin{pgfscope}%
\definecolor{textcolor}{rgb}{0.150000,0.150000,0.150000}%
\pgfsetstrokecolor{textcolor}%
\pgfsetfillcolor{textcolor}%
\pgftext[x=1.090675in,y=0.319444in,,top]{\color{textcolor}\sffamily\fontsize{9.000000}{10.800000}\selectfont 2000}%
\end{pgfscope}%
\begin{pgfscope}%
\pgfpathrectangle{\pgfqpoint{0.395236in}{0.451389in}}{\pgfqpoint{2.271868in}{1.319509in}}%
\pgfusepath{clip}%
\pgfsetroundcap%
\pgfsetroundjoin%
\pgfsetlinewidth{1.003750pt}%
\definecolor{currentstroke}{rgb}{0.800000,0.800000,0.800000}%
\pgfsetstrokecolor{currentstroke}%
\pgfsetdash{}{0pt}%
\pgfpathmoveto{\pgfqpoint{1.817266in}{0.451389in}}%
\pgfpathlineto{\pgfqpoint{1.817266in}{1.770898in}}%
\pgfusepath{stroke}%
\end{pgfscope}%
\begin{pgfscope}%
\definecolor{textcolor}{rgb}{0.150000,0.150000,0.150000}%
\pgfsetstrokecolor{textcolor}%
\pgfsetfillcolor{textcolor}%
\pgftext[x=1.817266in,y=0.319444in,,top]{\color{textcolor}\sffamily\fontsize{9.000000}{10.800000}\selectfont 4000}%
\end{pgfscope}%
\begin{pgfscope}%
\pgfpathrectangle{\pgfqpoint{0.395236in}{0.451389in}}{\pgfqpoint{2.271868in}{1.319509in}}%
\pgfusepath{clip}%
\pgfsetroundcap%
\pgfsetroundjoin%
\pgfsetlinewidth{1.003750pt}%
\definecolor{currentstroke}{rgb}{0.800000,0.800000,0.800000}%
\pgfsetstrokecolor{currentstroke}%
\pgfsetdash{}{0pt}%
\pgfpathmoveto{\pgfqpoint{2.543856in}{0.451389in}}%
\pgfpathlineto{\pgfqpoint{2.543856in}{1.770898in}}%
\pgfusepath{stroke}%
\end{pgfscope}%
\begin{pgfscope}%
\definecolor{textcolor}{rgb}{0.150000,0.150000,0.150000}%
\pgfsetstrokecolor{textcolor}%
\pgfsetfillcolor{textcolor}%
\pgftext[x=2.543856in,y=0.319444in,,top]{\color{textcolor}\sffamily\fontsize{9.000000}{10.800000}\selectfont 6000}%
\end{pgfscope}%
\begin{pgfscope}%
\definecolor{textcolor}{rgb}{0.150000,0.150000,0.150000}%
\pgfsetstrokecolor{textcolor}%
\pgfsetfillcolor{textcolor}%
\pgftext[x=1.531170in,y=0.125000in,,top]{\color{textcolor}\sffamily\fontsize{9.000000}{10.800000}\selectfont Input obstacle vertices}%
\end{pgfscope}%
\begin{pgfscope}%
\pgfpathrectangle{\pgfqpoint{0.395236in}{0.451389in}}{\pgfqpoint{2.271868in}{1.319509in}}%
\pgfusepath{clip}%
\pgfsetroundcap%
\pgfsetroundjoin%
\pgfsetlinewidth{1.003750pt}%
\definecolor{currentstroke}{rgb}{0.800000,0.800000,0.800000}%
\pgfsetstrokecolor{currentstroke}%
\pgfsetdash{}{0pt}%
\pgfpathmoveto{\pgfqpoint{0.395236in}{0.752767in}}%
\pgfpathlineto{\pgfqpoint{2.667104in}{0.752767in}}%
\pgfusepath{stroke}%
\end{pgfscope}%
\begin{pgfscope}%
\definecolor{textcolor}{rgb}{0.150000,0.150000,0.150000}%
\pgfsetstrokecolor{textcolor}%
\pgfsetfillcolor{textcolor}%
\pgftext[x=0.194444in, y=0.705281in, left, base]{\color{textcolor}\sffamily\fontsize{9.000000}{10.800000}\selectfont 2}%
\end{pgfscope}%
\begin{pgfscope}%
\pgfpathrectangle{\pgfqpoint{0.395236in}{0.451389in}}{\pgfqpoint{2.271868in}{1.319509in}}%
\pgfusepath{clip}%
\pgfsetroundcap%
\pgfsetroundjoin%
\pgfsetlinewidth{1.003750pt}%
\definecolor{currentstroke}{rgb}{0.800000,0.800000,0.800000}%
\pgfsetstrokecolor{currentstroke}%
\pgfsetdash{}{0pt}%
\pgfpathmoveto{\pgfqpoint{0.395236in}{1.169918in}}%
\pgfpathlineto{\pgfqpoint{2.667104in}{1.169918in}}%
\pgfusepath{stroke}%
\end{pgfscope}%
\begin{pgfscope}%
\definecolor{textcolor}{rgb}{0.150000,0.150000,0.150000}%
\pgfsetstrokecolor{textcolor}%
\pgfsetfillcolor{textcolor}%
\pgftext[x=0.194444in, y=1.122432in, left, base]{\color{textcolor}\sffamily\fontsize{9.000000}{10.800000}\selectfont 4}%
\end{pgfscope}%
\begin{pgfscope}%
\pgfpathrectangle{\pgfqpoint{0.395236in}{0.451389in}}{\pgfqpoint{2.271868in}{1.319509in}}%
\pgfusepath{clip}%
\pgfsetroundcap%
\pgfsetroundjoin%
\pgfsetlinewidth{1.003750pt}%
\definecolor{currentstroke}{rgb}{0.800000,0.800000,0.800000}%
\pgfsetstrokecolor{currentstroke}%
\pgfsetdash{}{0pt}%
\pgfpathmoveto{\pgfqpoint{0.395236in}{1.587069in}}%
\pgfpathlineto{\pgfqpoint{2.667104in}{1.587069in}}%
\pgfusepath{stroke}%
\end{pgfscope}%
\begin{pgfscope}%
\definecolor{textcolor}{rgb}{0.150000,0.150000,0.150000}%
\pgfsetstrokecolor{textcolor}%
\pgfsetfillcolor{textcolor}%
\pgftext[x=0.194444in, y=1.539583in, left, base]{\color{textcolor}\sffamily\fontsize{9.000000}{10.800000}\selectfont 6}%
\end{pgfscope}%
\begin{pgfscope}%
\definecolor{textcolor}{rgb}{0.150000,0.150000,0.150000}%
\pgfsetstrokecolor{textcolor}%
\pgfsetfillcolor{textcolor}%
\pgftext[x=0.125000in,y=1.111143in,,bottom,rotate=90.000000]{\color{textcolor}\sffamily\fontsize{9.000000}{10.800000}\selectfont Time in µs}%
\end{pgfscope}%
\begin{pgfscope}%
\pgfpathrectangle{\pgfqpoint{0.395236in}{0.451389in}}{\pgfqpoint{2.271868in}{1.319509in}}%
\pgfusepath{clip}%
\pgfsetbuttcap%
\pgfsetroundjoin%
\definecolor{currentfill}{rgb}{0.003922,0.450980,0.698039}%
\pgfsetfillcolor{currentfill}%
\pgfsetfillopacity{0.200000}%
\pgfsetlinewidth{1.003750pt}%
\definecolor{currentstroke}{rgb}{0.003922,0.450980,0.698039}%
\pgfsetstrokecolor{currentstroke}%
\pgfsetstrokeopacity{0.200000}%
\pgfsetdash{}{0pt}%
\pgfsys@defobject{currentmarker}{\pgfqpoint{0.498503in}{0.511366in}}{\pgfqpoint{2.563838in}{1.710920in}}{%
\pgfpathmoveto{\pgfqpoint{0.498503in}{1.710920in}}%
\pgfpathlineto{\pgfqpoint{0.498503in}{1.697441in}}%
\pgfpathlineto{\pgfqpoint{0.696499in}{1.051520in}}%
\pgfpathlineto{\pgfqpoint{0.927555in}{0.739411in}}%
\pgfpathlineto{\pgfqpoint{1.186948in}{0.617518in}}%
\pgfpathlineto{\pgfqpoint{1.961857in}{0.546162in}}%
\pgfpathlineto{\pgfqpoint{2.563838in}{0.511366in}}%
\pgfpathlineto{\pgfqpoint{2.563838in}{0.512535in}}%
\pgfpathlineto{\pgfqpoint{2.563838in}{0.512535in}}%
\pgfpathlineto{\pgfqpoint{1.961857in}{0.546722in}}%
\pgfpathlineto{\pgfqpoint{1.186948in}{0.627357in}}%
\pgfpathlineto{\pgfqpoint{0.927555in}{0.748479in}}%
\pgfpathlineto{\pgfqpoint{0.696499in}{1.062369in}}%
\pgfpathlineto{\pgfqpoint{0.498503in}{1.710920in}}%
\pgfpathlineto{\pgfqpoint{0.498503in}{1.710920in}}%
\pgfpathclose%
\pgfusepath{stroke,fill}%
}%
\begin{pgfscope}%
\pgfsys@transformshift{0.000000in}{0.000000in}%
\pgfsys@useobject{currentmarker}{}%
\end{pgfscope}%
\end{pgfscope}%
\begin{pgfscope}%
\pgfsetrectcap%
\pgfsetmiterjoin%
\pgfsetlinewidth{1.254687pt}%
\definecolor{currentstroke}{rgb}{0.800000,0.800000,0.800000}%
\pgfsetstrokecolor{currentstroke}%
\pgfsetdash{}{0pt}%
\pgfpathmoveto{\pgfqpoint{0.395236in}{0.451389in}}%
\pgfpathlineto{\pgfqpoint{0.395236in}{1.770898in}}%
\pgfusepath{stroke}%
\end{pgfscope}%
\begin{pgfscope}%
\pgfsetrectcap%
\pgfsetmiterjoin%
\pgfsetlinewidth{1.254687pt}%
\definecolor{currentstroke}{rgb}{0.800000,0.800000,0.800000}%
\pgfsetstrokecolor{currentstroke}%
\pgfsetdash{}{0pt}%
\pgfpathmoveto{\pgfqpoint{2.667104in}{0.451389in}}%
\pgfpathlineto{\pgfqpoint{2.667104in}{1.770898in}}%
\pgfusepath{stroke}%
\end{pgfscope}%
\begin{pgfscope}%
\pgfsetrectcap%
\pgfsetmiterjoin%
\pgfsetlinewidth{1.254687pt}%
\definecolor{currentstroke}{rgb}{0.800000,0.800000,0.800000}%
\pgfsetstrokecolor{currentstroke}%
\pgfsetdash{}{0pt}%
\pgfpathmoveto{\pgfqpoint{0.395236in}{0.451389in}}%
\pgfpathlineto{\pgfqpoint{2.667104in}{0.451389in}}%
\pgfusepath{stroke}%
\end{pgfscope}%
\begin{pgfscope}%
\pgfsetrectcap%
\pgfsetmiterjoin%
\pgfsetlinewidth{1.254687pt}%
\definecolor{currentstroke}{rgb}{0.800000,0.800000,0.800000}%
\pgfsetstrokecolor{currentstroke}%
\pgfsetdash{}{0pt}%
\pgfpathmoveto{\pgfqpoint{0.395236in}{1.770898in}}%
\pgfpathlineto{\pgfqpoint{2.667104in}{1.770898in}}%
\pgfusepath{stroke}%
\end{pgfscope}%
\begin{pgfscope}%
\pgfsetroundcap%
\pgfsetroundjoin%
\pgfsetlinewidth{1.003750pt}%
\definecolor{currentstroke}{rgb}{0.003922,0.450980,0.698039}%
\pgfsetstrokecolor{currentstroke}%
\pgfsetdash{}{0pt}%
\pgfpathmoveto{\pgfqpoint{0.498503in}{1.703124in}}%
\pgfpathlineto{\pgfqpoint{0.696499in}{1.059132in}}%
\pgfpathlineto{\pgfqpoint{0.927555in}{0.743127in}}%
\pgfpathlineto{\pgfqpoint{1.186948in}{0.622715in}}%
\pgfpathlineto{\pgfqpoint{1.961857in}{0.546408in}}%
\pgfpathlineto{\pgfqpoint{2.563838in}{0.511899in}}%
\pgfusepath{stroke}%
\end{pgfscope}%
\begin{pgfscope}%
\pgfsetbuttcap%
\pgfsetroundjoin%
\definecolor{currentfill}{rgb}{0.003922,0.450980,0.698039}%
\pgfsetfillcolor{currentfill}%
\pgfsetlinewidth{0.752812pt}%
\definecolor{currentstroke}{rgb}{1.000000,1.000000,1.000000}%
\pgfsetstrokecolor{currentstroke}%
\pgfsetdash{}{0pt}%
\pgfsys@defobject{currentmarker}{\pgfqpoint{-0.034722in}{-0.034722in}}{\pgfqpoint{0.034722in}{0.034722in}}{%
\pgfpathmoveto{\pgfqpoint{0.000000in}{-0.034722in}}%
\pgfpathcurveto{\pgfqpoint{0.009208in}{-0.034722in}}{\pgfqpoint{0.018041in}{-0.031064in}}{\pgfqpoint{0.024552in}{-0.024552in}}%
\pgfpathcurveto{\pgfqpoint{0.031064in}{-0.018041in}}{\pgfqpoint{0.034722in}{-0.009208in}}{\pgfqpoint{0.034722in}{0.000000in}}%
\pgfpathcurveto{\pgfqpoint{0.034722in}{0.009208in}}{\pgfqpoint{0.031064in}{0.018041in}}{\pgfqpoint{0.024552in}{0.024552in}}%
\pgfpathcurveto{\pgfqpoint{0.018041in}{0.031064in}}{\pgfqpoint{0.009208in}{0.034722in}}{\pgfqpoint{0.000000in}{0.034722in}}%
\pgfpathcurveto{\pgfqpoint{-0.009208in}{0.034722in}}{\pgfqpoint{-0.018041in}{0.031064in}}{\pgfqpoint{-0.024552in}{0.024552in}}%
\pgfpathcurveto{\pgfqpoint{-0.031064in}{0.018041in}}{\pgfqpoint{-0.034722in}{0.009208in}}{\pgfqpoint{-0.034722in}{0.000000in}}%
\pgfpathcurveto{\pgfqpoint{-0.034722in}{-0.009208in}}{\pgfqpoint{-0.031064in}{-0.018041in}}{\pgfqpoint{-0.024552in}{-0.024552in}}%
\pgfpathcurveto{\pgfqpoint{-0.018041in}{-0.031064in}}{\pgfqpoint{-0.009208in}{-0.034722in}}{\pgfqpoint{0.000000in}{-0.034722in}}%
\pgfpathlineto{\pgfqpoint{0.000000in}{-0.034722in}}%
\pgfpathclose%
\pgfusepath{stroke,fill}%
}%
\begin{pgfscope}%
\pgfsys@transformshift{0.498503in}{1.703124in}%
\pgfsys@useobject{currentmarker}{}%
\end{pgfscope}%
\begin{pgfscope}%
\pgfsys@transformshift{0.696499in}{1.059132in}%
\pgfsys@useobject{currentmarker}{}%
\end{pgfscope}%
\begin{pgfscope}%
\pgfsys@transformshift{0.927555in}{0.743127in}%
\pgfsys@useobject{currentmarker}{}%
\end{pgfscope}%
\begin{pgfscope}%
\pgfsys@transformshift{1.186948in}{0.622715in}%
\pgfsys@useobject{currentmarker}{}%
\end{pgfscope}%
\begin{pgfscope}%
\pgfsys@transformshift{1.961857in}{0.546408in}%
\pgfsys@useobject{currentmarker}{}%
\end{pgfscope}%
\begin{pgfscope}%
\pgfsys@transformshift{2.563838in}{0.511899in}%
\pgfsys@useobject{currentmarker}{}%
\end{pgfscope}%
\end{pgfscope}%
\end{pgfpicture}%
\makeatother%
\endgroup%

%						\end{figcenter}
%						\caption{Increase in graph generation time per additionally added vertex.}
%						\label{fig:eval-import-city-rel-increase}
%					\end{subfigure}
					\caption{Graph generation times using the \enquote{OSM city} datasets.}
					\label{fig:eval-import-city}
				\end{minipage}
				\hfill
				\begin{minipage}{.48\textwidth}
					\begin{subfigure}[t]{\linewidth}
						\begin{figcenter}
							%% Creator: Matplotlib, PGF backend
%%
%% To include the figure in your LaTeX document, write
%%   \input{<filename>.pgf}
%%
%% Make sure the required packages are loaded in your preamble
%%   \usepackage{pgf}
%%
%% Also ensure that all the required font packages are loaded; for instance,
%% the lmodern package is sometimes necessary when using math font.
%%   \usepackage{lmodern}
%%
%% Figures using additional raster images can only be included by \input if
%% they are in the same directory as the main LaTeX file. For loading figures
%% from other directories you can use the `import` package
%%   \usepackage{import}
%%
%% and then include the figures with
%%   \import{<path to file>}{<filename>.pgf}
%%
%% Matplotlib used the following preamble
%%   
%%   \usepackage{fontspec}
%%   \setmainfont{DejaVuSerif.ttf}[Path=\detokenize{/home/hauke/.local/lib/python3.11/site-packages/matplotlib/mpl-data/fonts/ttf/}]
%%   \setsansfont{DroidSans.ttf}[Path=\detokenize{/usr/share/fonts/droid/}]
%%   \setmonofont{DejaVuSansMono.ttf}[Path=\detokenize{/home/hauke/.local/lib/python3.11/site-packages/matplotlib/mpl-data/fonts/ttf/}]
%%   \makeatletter\@ifpackageloaded{underscore}{}{\usepackage[strings]{underscore}}\makeatother
%%
\begingroup%
\makeatletter%
\begin{pgfpicture}%
\pgfpathrectangle{\pgfpointorigin}{\pgfqpoint{2.681845in}{1.770898in}}%
\pgfusepath{use as bounding box, clip}%
\begin{pgfscope}%
\pgfsetbuttcap%
\pgfsetmiterjoin%
\definecolor{currentfill}{rgb}{1.000000,1.000000,1.000000}%
\pgfsetfillcolor{currentfill}%
\pgfsetlinewidth{0.000000pt}%
\definecolor{currentstroke}{rgb}{1.000000,1.000000,1.000000}%
\pgfsetstrokecolor{currentstroke}%
\pgfsetdash{}{0pt}%
\pgfpathmoveto{\pgfqpoint{0.000000in}{0.000000in}}%
\pgfpathlineto{\pgfqpoint{2.681845in}{0.000000in}}%
\pgfpathlineto{\pgfqpoint{2.681845in}{1.770898in}}%
\pgfpathlineto{\pgfqpoint{0.000000in}{1.770898in}}%
\pgfpathlineto{\pgfqpoint{0.000000in}{0.000000in}}%
\pgfpathclose%
\pgfusepath{fill}%
\end{pgfscope}%
\begin{pgfscope}%
\pgfsetbuttcap%
\pgfsetmiterjoin%
\definecolor{currentfill}{rgb}{1.000000,1.000000,1.000000}%
\pgfsetfillcolor{currentfill}%
\pgfsetlinewidth{0.000000pt}%
\definecolor{currentstroke}{rgb}{0.000000,0.000000,0.000000}%
\pgfsetstrokecolor{currentstroke}%
\pgfsetstrokeopacity{0.000000}%
\pgfsetdash{}{0pt}%
\pgfpathmoveto{\pgfqpoint{0.464084in}{0.451389in}}%
\pgfpathlineto{\pgfqpoint{2.661760in}{0.451389in}}%
\pgfpathlineto{\pgfqpoint{2.661760in}{1.770898in}}%
\pgfpathlineto{\pgfqpoint{0.464084in}{1.770898in}}%
\pgfpathlineto{\pgfqpoint{0.464084in}{0.451389in}}%
\pgfpathclose%
\pgfusepath{fill}%
\end{pgfscope}%
\begin{pgfscope}%
\pgfpathrectangle{\pgfqpoint{0.464084in}{0.451389in}}{\pgfqpoint{2.197676in}{1.319509in}}%
\pgfusepath{clip}%
\pgfsetroundcap%
\pgfsetroundjoin%
\pgfsetlinewidth{1.003750pt}%
\definecolor{currentstroke}{rgb}{0.800000,0.800000,0.800000}%
\pgfsetstrokecolor{currentstroke}%
\pgfsetdash{}{0pt}%
\pgfpathmoveto{\pgfqpoint{0.464084in}{0.451389in}}%
\pgfpathlineto{\pgfqpoint{0.464084in}{1.770898in}}%
\pgfusepath{stroke}%
\end{pgfscope}%
\begin{pgfscope}%
\definecolor{textcolor}{rgb}{0.150000,0.150000,0.150000}%
\pgfsetstrokecolor{textcolor}%
\pgfsetfillcolor{textcolor}%
\pgftext[x=0.464084in,y=0.319444in,,top]{\color{textcolor}\sffamily\fontsize{9.000000}{10.800000}\selectfont 0}%
\end{pgfscope}%
\begin{pgfscope}%
\pgfpathrectangle{\pgfqpoint{0.464084in}{0.451389in}}{\pgfqpoint{2.197676in}{1.319509in}}%
\pgfusepath{clip}%
\pgfsetroundcap%
\pgfsetroundjoin%
\pgfsetlinewidth{1.003750pt}%
\definecolor{currentstroke}{rgb}{0.800000,0.800000,0.800000}%
\pgfsetstrokecolor{currentstroke}%
\pgfsetdash{}{0pt}%
\pgfpathmoveto{\pgfqpoint{1.157439in}{0.451389in}}%
\pgfpathlineto{\pgfqpoint{1.157439in}{1.770898in}}%
\pgfusepath{stroke}%
\end{pgfscope}%
\begin{pgfscope}%
\definecolor{textcolor}{rgb}{0.150000,0.150000,0.150000}%
\pgfsetstrokecolor{textcolor}%
\pgfsetfillcolor{textcolor}%
\pgftext[x=1.157439in,y=0.319444in,,top]{\color{textcolor}\sffamily\fontsize{9.000000}{10.800000}\selectfont 2000}%
\end{pgfscope}%
\begin{pgfscope}%
\pgfpathrectangle{\pgfqpoint{0.464084in}{0.451389in}}{\pgfqpoint{2.197676in}{1.319509in}}%
\pgfusepath{clip}%
\pgfsetroundcap%
\pgfsetroundjoin%
\pgfsetlinewidth{1.003750pt}%
\definecolor{currentstroke}{rgb}{0.800000,0.800000,0.800000}%
\pgfsetstrokecolor{currentstroke}%
\pgfsetdash{}{0pt}%
\pgfpathmoveto{\pgfqpoint{1.850794in}{0.451389in}}%
\pgfpathlineto{\pgfqpoint{1.850794in}{1.770898in}}%
\pgfusepath{stroke}%
\end{pgfscope}%
\begin{pgfscope}%
\definecolor{textcolor}{rgb}{0.150000,0.150000,0.150000}%
\pgfsetstrokecolor{textcolor}%
\pgfsetfillcolor{textcolor}%
\pgftext[x=1.850794in,y=0.319444in,,top]{\color{textcolor}\sffamily\fontsize{9.000000}{10.800000}\selectfont 4000}%
\end{pgfscope}%
\begin{pgfscope}%
\pgfpathrectangle{\pgfqpoint{0.464084in}{0.451389in}}{\pgfqpoint{2.197676in}{1.319509in}}%
\pgfusepath{clip}%
\pgfsetroundcap%
\pgfsetroundjoin%
\pgfsetlinewidth{1.003750pt}%
\definecolor{currentstroke}{rgb}{0.800000,0.800000,0.800000}%
\pgfsetstrokecolor{currentstroke}%
\pgfsetdash{}{0pt}%
\pgfpathmoveto{\pgfqpoint{2.544149in}{0.451389in}}%
\pgfpathlineto{\pgfqpoint{2.544149in}{1.770898in}}%
\pgfusepath{stroke}%
\end{pgfscope}%
\begin{pgfscope}%
\definecolor{textcolor}{rgb}{0.150000,0.150000,0.150000}%
\pgfsetstrokecolor{textcolor}%
\pgfsetfillcolor{textcolor}%
\pgftext[x=2.544149in,y=0.319444in,,top]{\color{textcolor}\sffamily\fontsize{9.000000}{10.800000}\selectfont 6000}%
\end{pgfscope}%
\begin{pgfscope}%
\definecolor{textcolor}{rgb}{0.150000,0.150000,0.150000}%
\pgfsetstrokecolor{textcolor}%
\pgfsetfillcolor{textcolor}%
\pgftext[x=1.562922in,y=0.125000in,,top]{\color{textcolor}\sffamily\fontsize{9.000000}{10.800000}\selectfont Input obstacle vertices}%
\end{pgfscope}%
\begin{pgfscope}%
\pgfpathrectangle{\pgfqpoint{0.464084in}{0.451389in}}{\pgfqpoint{2.197676in}{1.319509in}}%
\pgfusepath{clip}%
\pgfsetroundcap%
\pgfsetroundjoin%
\pgfsetlinewidth{1.003750pt}%
\definecolor{currentstroke}{rgb}{0.800000,0.800000,0.800000}%
\pgfsetstrokecolor{currentstroke}%
\pgfsetdash{}{0pt}%
\pgfpathmoveto{\pgfqpoint{0.464084in}{0.451389in}}%
\pgfpathlineto{\pgfqpoint{2.661760in}{0.451389in}}%
\pgfusepath{stroke}%
\end{pgfscope}%
\begin{pgfscope}%
\definecolor{textcolor}{rgb}{0.150000,0.150000,0.150000}%
\pgfsetstrokecolor{textcolor}%
\pgfsetfillcolor{textcolor}%
\pgftext[x=0.263292in, y=0.403903in, left, base]{\color{textcolor}\sffamily\fontsize{9.000000}{10.800000}\selectfont 0}%
\end{pgfscope}%
\begin{pgfscope}%
\pgfpathrectangle{\pgfqpoint{0.464084in}{0.451389in}}{\pgfqpoint{2.197676in}{1.319509in}}%
\pgfusepath{clip}%
\pgfsetroundcap%
\pgfsetroundjoin%
\pgfsetlinewidth{1.003750pt}%
\definecolor{currentstroke}{rgb}{0.800000,0.800000,0.800000}%
\pgfsetstrokecolor{currentstroke}%
\pgfsetdash{}{0pt}%
\pgfpathmoveto{\pgfqpoint{0.464084in}{0.856036in}}%
\pgfpathlineto{\pgfqpoint{2.661760in}{0.856036in}}%
\pgfusepath{stroke}%
\end{pgfscope}%
\begin{pgfscope}%
\definecolor{textcolor}{rgb}{0.150000,0.150000,0.150000}%
\pgfsetstrokecolor{textcolor}%
\pgfsetfillcolor{textcolor}%
\pgftext[x=0.194444in, y=0.808551in, left, base]{\color{textcolor}\sffamily\fontsize{9.000000}{10.800000}\selectfont 10}%
\end{pgfscope}%
\begin{pgfscope}%
\pgfpathrectangle{\pgfqpoint{0.464084in}{0.451389in}}{\pgfqpoint{2.197676in}{1.319509in}}%
\pgfusepath{clip}%
\pgfsetroundcap%
\pgfsetroundjoin%
\pgfsetlinewidth{1.003750pt}%
\definecolor{currentstroke}{rgb}{0.800000,0.800000,0.800000}%
\pgfsetstrokecolor{currentstroke}%
\pgfsetdash{}{0pt}%
\pgfpathmoveto{\pgfqpoint{0.464084in}{1.260683in}}%
\pgfpathlineto{\pgfqpoint{2.661760in}{1.260683in}}%
\pgfusepath{stroke}%
\end{pgfscope}%
\begin{pgfscope}%
\definecolor{textcolor}{rgb}{0.150000,0.150000,0.150000}%
\pgfsetstrokecolor{textcolor}%
\pgfsetfillcolor{textcolor}%
\pgftext[x=0.194444in, y=1.213198in, left, base]{\color{textcolor}\sffamily\fontsize{9.000000}{10.800000}\selectfont 20}%
\end{pgfscope}%
\begin{pgfscope}%
\pgfpathrectangle{\pgfqpoint{0.464084in}{0.451389in}}{\pgfqpoint{2.197676in}{1.319509in}}%
\pgfusepath{clip}%
\pgfsetroundcap%
\pgfsetroundjoin%
\pgfsetlinewidth{1.003750pt}%
\definecolor{currentstroke}{rgb}{0.800000,0.800000,0.800000}%
\pgfsetstrokecolor{currentstroke}%
\pgfsetdash{}{0pt}%
\pgfpathmoveto{\pgfqpoint{0.464084in}{1.665331in}}%
\pgfpathlineto{\pgfqpoint{2.661760in}{1.665331in}}%
\pgfusepath{stroke}%
\end{pgfscope}%
\begin{pgfscope}%
\definecolor{textcolor}{rgb}{0.150000,0.150000,0.150000}%
\pgfsetstrokecolor{textcolor}%
\pgfsetfillcolor{textcolor}%
\pgftext[x=0.194444in, y=1.617845in, left, base]{\color{textcolor}\sffamily\fontsize{9.000000}{10.800000}\selectfont 30}%
\end{pgfscope}%
\begin{pgfscope}%
\definecolor{textcolor}{rgb}{0.150000,0.150000,0.150000}%
\pgfsetstrokecolor{textcolor}%
\pgfsetfillcolor{textcolor}%
\pgftext[x=0.125000in,y=1.111143in,,bottom,rotate=90.000000]{\color{textcolor}\sffamily\fontsize{9.000000}{10.800000}\selectfont Time in s}%
\end{pgfscope}%
\begin{pgfscope}%
\pgfpathrectangle{\pgfqpoint{0.464084in}{0.451389in}}{\pgfqpoint{2.197676in}{1.319509in}}%
\pgfusepath{clip}%
\pgfsetbuttcap%
\pgfsetroundjoin%
\definecolor{currentfill}{rgb}{0.003922,0.450980,0.698039}%
\pgfsetfillcolor{currentfill}%
\pgfsetfillopacity{0.200000}%
\pgfsetlinewidth{1.003750pt}%
\definecolor{currentstroke}{rgb}{0.003922,0.450980,0.698039}%
\pgfsetstrokecolor{currentstroke}%
\pgfsetstrokeopacity{0.200000}%
\pgfsetdash{}{0pt}%
\pgfsys@defobject{currentmarker}{\pgfqpoint{0.592355in}{0.487558in}}{\pgfqpoint{2.563217in}{1.709786in}}{%
\pgfpathmoveto{\pgfqpoint{0.592355in}{0.487916in}}%
\pgfpathlineto{\pgfqpoint{0.592355in}{0.487558in}}%
\pgfpathlineto{\pgfqpoint{0.781294in}{0.567670in}}%
\pgfpathlineto{\pgfqpoint{1.001781in}{0.639840in}}%
\pgfpathlineto{\pgfqpoint{1.249309in}{0.731963in}}%
\pgfpathlineto{\pgfqpoint{1.988772in}{1.241469in}}%
\pgfpathlineto{\pgfqpoint{2.563217in}{1.701472in}}%
\pgfpathlineto{\pgfqpoint{2.563217in}{1.709786in}}%
\pgfpathlineto{\pgfqpoint{2.563217in}{1.709786in}}%
\pgfpathlineto{\pgfqpoint{1.988772in}{1.243571in}}%
\pgfpathlineto{\pgfqpoint{1.249309in}{0.741756in}}%
\pgfpathlineto{\pgfqpoint{1.001781in}{0.644072in}}%
\pgfpathlineto{\pgfqpoint{0.781294in}{0.569432in}}%
\pgfpathlineto{\pgfqpoint{0.592355in}{0.487916in}}%
\pgfpathlineto{\pgfqpoint{0.592355in}{0.487916in}}%
\pgfpathclose%
\pgfusepath{stroke,fill}%
}%
\begin{pgfscope}%
\pgfsys@transformshift{0.000000in}{0.000000in}%
\pgfsys@useobject{currentmarker}{}%
\end{pgfscope}%
\end{pgfscope}%
\begin{pgfscope}%
\pgfsetrectcap%
\pgfsetmiterjoin%
\pgfsetlinewidth{1.254687pt}%
\definecolor{currentstroke}{rgb}{0.800000,0.800000,0.800000}%
\pgfsetstrokecolor{currentstroke}%
\pgfsetdash{}{0pt}%
\pgfpathmoveto{\pgfqpoint{0.464084in}{0.451389in}}%
\pgfpathlineto{\pgfqpoint{0.464084in}{1.770898in}}%
\pgfusepath{stroke}%
\end{pgfscope}%
\begin{pgfscope}%
\pgfsetrectcap%
\pgfsetmiterjoin%
\pgfsetlinewidth{1.254687pt}%
\definecolor{currentstroke}{rgb}{0.800000,0.800000,0.800000}%
\pgfsetstrokecolor{currentstroke}%
\pgfsetdash{}{0pt}%
\pgfpathmoveto{\pgfqpoint{2.661760in}{0.451389in}}%
\pgfpathlineto{\pgfqpoint{2.661760in}{1.770898in}}%
\pgfusepath{stroke}%
\end{pgfscope}%
\begin{pgfscope}%
\pgfsetrectcap%
\pgfsetmiterjoin%
\pgfsetlinewidth{1.254687pt}%
\definecolor{currentstroke}{rgb}{0.800000,0.800000,0.800000}%
\pgfsetstrokecolor{currentstroke}%
\pgfsetdash{}{0pt}%
\pgfpathmoveto{\pgfqpoint{0.464084in}{0.451389in}}%
\pgfpathlineto{\pgfqpoint{2.661760in}{0.451389in}}%
\pgfusepath{stroke}%
\end{pgfscope}%
\begin{pgfscope}%
\pgfsetrectcap%
\pgfsetmiterjoin%
\pgfsetlinewidth{1.254687pt}%
\definecolor{currentstroke}{rgb}{0.800000,0.800000,0.800000}%
\pgfsetstrokecolor{currentstroke}%
\pgfsetdash{}{0pt}%
\pgfpathmoveto{\pgfqpoint{0.464084in}{1.770898in}}%
\pgfpathlineto{\pgfqpoint{2.661760in}{1.770898in}}%
\pgfusepath{stroke}%
\end{pgfscope}%
\begin{pgfscope}%
\pgfsetroundcap%
\pgfsetroundjoin%
\pgfsetlinewidth{1.003750pt}%
\definecolor{currentstroke}{rgb}{0.003922,0.450980,0.698039}%
\pgfsetstrokecolor{currentstroke}%
\pgfsetdash{}{0pt}%
\pgfpathmoveto{\pgfqpoint{0.592355in}{0.487709in}}%
\pgfpathlineto{\pgfqpoint{0.781294in}{0.568906in}}%
\pgfpathlineto{\pgfqpoint{1.001781in}{0.641574in}}%
\pgfpathlineto{\pgfqpoint{1.249309in}{0.737136in}}%
\pgfpathlineto{\pgfqpoint{1.988772in}{1.242395in}}%
\pgfpathlineto{\pgfqpoint{2.563217in}{1.705259in}}%
\pgfusepath{stroke}%
\end{pgfscope}%
\begin{pgfscope}%
\pgfsetbuttcap%
\pgfsetroundjoin%
\definecolor{currentfill}{rgb}{0.003922,0.450980,0.698039}%
\pgfsetfillcolor{currentfill}%
\pgfsetlinewidth{0.752812pt}%
\definecolor{currentstroke}{rgb}{1.000000,1.000000,1.000000}%
\pgfsetstrokecolor{currentstroke}%
\pgfsetdash{}{0pt}%
\pgfsys@defobject{currentmarker}{\pgfqpoint{-0.034722in}{-0.034722in}}{\pgfqpoint{0.034722in}{0.034722in}}{%
\pgfpathmoveto{\pgfqpoint{0.000000in}{-0.034722in}}%
\pgfpathcurveto{\pgfqpoint{0.009208in}{-0.034722in}}{\pgfqpoint{0.018041in}{-0.031064in}}{\pgfqpoint{0.024552in}{-0.024552in}}%
\pgfpathcurveto{\pgfqpoint{0.031064in}{-0.018041in}}{\pgfqpoint{0.034722in}{-0.009208in}}{\pgfqpoint{0.034722in}{0.000000in}}%
\pgfpathcurveto{\pgfqpoint{0.034722in}{0.009208in}}{\pgfqpoint{0.031064in}{0.018041in}}{\pgfqpoint{0.024552in}{0.024552in}}%
\pgfpathcurveto{\pgfqpoint{0.018041in}{0.031064in}}{\pgfqpoint{0.009208in}{0.034722in}}{\pgfqpoint{0.000000in}{0.034722in}}%
\pgfpathcurveto{\pgfqpoint{-0.009208in}{0.034722in}}{\pgfqpoint{-0.018041in}{0.031064in}}{\pgfqpoint{-0.024552in}{0.024552in}}%
\pgfpathcurveto{\pgfqpoint{-0.031064in}{0.018041in}}{\pgfqpoint{-0.034722in}{0.009208in}}{\pgfqpoint{-0.034722in}{0.000000in}}%
\pgfpathcurveto{\pgfqpoint{-0.034722in}{-0.009208in}}{\pgfqpoint{-0.031064in}{-0.018041in}}{\pgfqpoint{-0.024552in}{-0.024552in}}%
\pgfpathcurveto{\pgfqpoint{-0.018041in}{-0.031064in}}{\pgfqpoint{-0.009208in}{-0.034722in}}{\pgfqpoint{0.000000in}{-0.034722in}}%
\pgfpathlineto{\pgfqpoint{0.000000in}{-0.034722in}}%
\pgfpathclose%
\pgfusepath{stroke,fill}%
}%
\begin{pgfscope}%
\pgfsys@transformshift{0.592355in}{0.487709in}%
\pgfsys@useobject{currentmarker}{}%
\end{pgfscope}%
\begin{pgfscope}%
\pgfsys@transformshift{0.781294in}{0.568906in}%
\pgfsys@useobject{currentmarker}{}%
\end{pgfscope}%
\begin{pgfscope}%
\pgfsys@transformshift{1.001781in}{0.641574in}%
\pgfsys@useobject{currentmarker}{}%
\end{pgfscope}%
\begin{pgfscope}%
\pgfsys@transformshift{1.249309in}{0.737136in}%
\pgfsys@useobject{currentmarker}{}%
\end{pgfscope}%
\begin{pgfscope}%
\pgfsys@transformshift{1.988772in}{1.242395in}%
\pgfsys@useobject{currentmarker}{}%
\end{pgfscope}%
\begin{pgfscope}%
\pgfsys@transformshift{2.563217in}{1.705259in}%
\pgfsys@useobject{currentmarker}{}%
\end{pgfscope}%
\end{pgfscope}%
\end{pgfpicture}%
\makeatother%
\endgroup%

						\end{figcenter}
						\caption{Total graph generation times.}
						\label{fig:eval-import-rural-abs}
					\end{subfigure}
					\\[3ex]
					\begin{subfigure}[t]{\linewidth}
						\begin{figcenter}
							\begingroup%
\makeatletter%
\begin{pgfpicture}%
\pgfpathrectangle{\pgfpointorigin}{\pgfqpoint{3.042524in}{1.867995in}}%
\pgfusepath{use as bounding box}%
\begin{pgfscope}%
\pgfsetbuttcap%
\pgfsetmiterjoin%
\definecolor{currentfill}{rgb}{1.000000,1.000000,1.000000}%
\pgfsetfillcolor{currentfill}%
\pgfsetlinewidth{0.000000pt}%
\definecolor{currentstroke}{rgb}{1.000000,1.000000,1.000000}%
\pgfsetstrokecolor{currentstroke}%
\pgfsetdash{}{0pt}%
\pgfpathmoveto{\pgfqpoint{0.000000in}{0.000000in}}%
\pgfpathlineto{\pgfqpoint{3.042524in}{0.000000in}}%
\pgfpathlineto{\pgfqpoint{3.042524in}{1.867995in}}%
\pgfpathlineto{\pgfqpoint{0.000000in}{1.867995in}}%
\pgfpathlineto{\pgfqpoint{0.000000in}{0.000000in}}%
\pgfpathclose%
\pgfusepath{fill}%
\end{pgfscope}%
\begin{pgfscope}%
\pgfsetbuttcap%
\pgfsetmiterjoin%
\definecolor{currentfill}{rgb}{1.000000,1.000000,1.000000}%
\pgfsetfillcolor{currentfill}%
\pgfsetlinewidth{0.000000pt}%
\definecolor{currentstroke}{rgb}{0.000000,0.000000,0.000000}%
\pgfsetstrokecolor{currentstroke}%
\pgfsetstrokeopacity{0.000000}%
\pgfsetdash{}{0pt}%
\pgfpathmoveto{\pgfqpoint{0.566440in}{0.451389in}}%
\pgfpathlineto{\pgfqpoint{3.042524in}{0.451389in}}%
\pgfpathlineto{\pgfqpoint{3.042524in}{1.867995in}}%
\pgfpathlineto{\pgfqpoint{0.566440in}{1.867995in}}%
\pgfpathlineto{\pgfqpoint{0.566440in}{0.451389in}}%
\pgfpathclose%
\pgfusepath{fill}%
\end{pgfscope}%
\begin{pgfscope}%
\pgfpathrectangle{\pgfqpoint{0.566440in}{0.451389in}}{\pgfqpoint{2.476084in}{1.416606in}}%
\pgfusepath{clip}%
\pgfsetroundcap%
\pgfsetroundjoin%
\pgfsetlinewidth{1.003750pt}%
\definecolor{currentstroke}{rgb}{0.800000,0.800000,0.800000}%
\pgfsetstrokecolor{currentstroke}%
\pgfsetdash{}{0pt}%
\pgfpathmoveto{\pgfqpoint{0.566440in}{0.451389in}}%
\pgfpathlineto{\pgfqpoint{0.566440in}{1.867995in}}%
\pgfusepath{stroke}%
\end{pgfscope}%
\begin{pgfscope}%
\definecolor{textcolor}{rgb}{0.150000,0.150000,0.150000}%
\pgfsetstrokecolor{textcolor}%
\pgfsetfillcolor{textcolor}%
\pgftext[x=0.566440in,y=0.319444in,,top]{\color{textcolor}\sffamily\fontsize{9.000000}{10.800000}\selectfont 0}%
\end{pgfscope}%
\begin{pgfscope}%
\pgfpathrectangle{\pgfqpoint{0.566440in}{0.451389in}}{\pgfqpoint{2.476084in}{1.416606in}}%
\pgfusepath{clip}%
\pgfsetroundcap%
\pgfsetroundjoin%
\pgfsetlinewidth{1.003750pt}%
\definecolor{currentstroke}{rgb}{0.800000,0.800000,0.800000}%
\pgfsetstrokecolor{currentstroke}%
\pgfsetdash{}{0pt}%
\pgfpathmoveto{\pgfqpoint{1.094231in}{0.451389in}}%
\pgfpathlineto{\pgfqpoint{1.094231in}{1.867995in}}%
\pgfusepath{stroke}%
\end{pgfscope}%
\begin{pgfscope}%
\definecolor{textcolor}{rgb}{0.150000,0.150000,0.150000}%
\pgfsetstrokecolor{textcolor}%
\pgfsetfillcolor{textcolor}%
\pgftext[x=1.094231in,y=0.319444in,,top]{\color{textcolor}\sffamily\fontsize{9.000000}{10.800000}\selectfont 10000}%
\end{pgfscope}%
\begin{pgfscope}%
\pgfpathrectangle{\pgfqpoint{0.566440in}{0.451389in}}{\pgfqpoint{2.476084in}{1.416606in}}%
\pgfusepath{clip}%
\pgfsetroundcap%
\pgfsetroundjoin%
\pgfsetlinewidth{1.003750pt}%
\definecolor{currentstroke}{rgb}{0.800000,0.800000,0.800000}%
\pgfsetstrokecolor{currentstroke}%
\pgfsetdash{}{0pt}%
\pgfpathmoveto{\pgfqpoint{1.622022in}{0.451389in}}%
\pgfpathlineto{\pgfqpoint{1.622022in}{1.867995in}}%
\pgfusepath{stroke}%
\end{pgfscope}%
\begin{pgfscope}%
\definecolor{textcolor}{rgb}{0.150000,0.150000,0.150000}%
\pgfsetstrokecolor{textcolor}%
\pgfsetfillcolor{textcolor}%
\pgftext[x=1.622022in,y=0.319444in,,top]{\color{textcolor}\sffamily\fontsize{9.000000}{10.800000}\selectfont 20000}%
\end{pgfscope}%
\begin{pgfscope}%
\pgfpathrectangle{\pgfqpoint{0.566440in}{0.451389in}}{\pgfqpoint{2.476084in}{1.416606in}}%
\pgfusepath{clip}%
\pgfsetroundcap%
\pgfsetroundjoin%
\pgfsetlinewidth{1.003750pt}%
\definecolor{currentstroke}{rgb}{0.800000,0.800000,0.800000}%
\pgfsetstrokecolor{currentstroke}%
\pgfsetdash{}{0pt}%
\pgfpathmoveto{\pgfqpoint{2.149813in}{0.451389in}}%
\pgfpathlineto{\pgfqpoint{2.149813in}{1.867995in}}%
\pgfusepath{stroke}%
\end{pgfscope}%
\begin{pgfscope}%
\definecolor{textcolor}{rgb}{0.150000,0.150000,0.150000}%
\pgfsetstrokecolor{textcolor}%
\pgfsetfillcolor{textcolor}%
\pgftext[x=2.149813in,y=0.319444in,,top]{\color{textcolor}\sffamily\fontsize{9.000000}{10.800000}\selectfont 30000}%
\end{pgfscope}%
\begin{pgfscope}%
\pgfpathrectangle{\pgfqpoint{0.566440in}{0.451389in}}{\pgfqpoint{2.476084in}{1.416606in}}%
\pgfusepath{clip}%
\pgfsetroundcap%
\pgfsetroundjoin%
\pgfsetlinewidth{1.003750pt}%
\definecolor{currentstroke}{rgb}{0.800000,0.800000,0.800000}%
\pgfsetstrokecolor{currentstroke}%
\pgfsetdash{}{0pt}%
\pgfpathmoveto{\pgfqpoint{2.677604in}{0.451389in}}%
\pgfpathlineto{\pgfqpoint{2.677604in}{1.867995in}}%
\pgfusepath{stroke}%
\end{pgfscope}%
\begin{pgfscope}%
\definecolor{textcolor}{rgb}{0.150000,0.150000,0.150000}%
\pgfsetstrokecolor{textcolor}%
\pgfsetfillcolor{textcolor}%
\pgftext[x=2.677604in,y=0.319444in,,top]{\color{textcolor}\sffamily\fontsize{9.000000}{10.800000}\selectfont 40000}%
\end{pgfscope}%
\begin{pgfscope}%
\definecolor{textcolor}{rgb}{0.150000,0.150000,0.150000}%
\pgfsetstrokecolor{textcolor}%
\pgfsetfillcolor{textcolor}%
\pgftext[x=1.804482in,y=0.125000in,,top]{\color{textcolor}\sffamily\fontsize{9.000000}{10.800000}\selectfont Input obstacle vertices}%
\end{pgfscope}%
\begin{pgfscope}%
\pgfpathrectangle{\pgfqpoint{0.566440in}{0.451389in}}{\pgfqpoint{2.476084in}{1.416606in}}%
\pgfusepath{clip}%
\pgfsetroundcap%
\pgfsetroundjoin%
\pgfsetlinewidth{1.003750pt}%
\definecolor{currentstroke}{rgb}{0.800000,0.800000,0.800000}%
\pgfsetstrokecolor{currentstroke}%
\pgfsetdash{}{0pt}%
\pgfpathmoveto{\pgfqpoint{0.566440in}{0.451389in}}%
\pgfpathlineto{\pgfqpoint{3.042524in}{0.451389in}}%
\pgfusepath{stroke}%
\end{pgfscope}%
\begin{pgfscope}%
\definecolor{textcolor}{rgb}{0.150000,0.150000,0.150000}%
\pgfsetstrokecolor{textcolor}%
\pgfsetfillcolor{textcolor}%
\pgftext[x=0.263292in, y=0.403903in, left, base]{\color{textcolor}\sffamily\fontsize{9.000000}{10.800000}\selectfont 0.0}%
\end{pgfscope}%
\begin{pgfscope}%
\pgfpathrectangle{\pgfqpoint{0.566440in}{0.451389in}}{\pgfqpoint{2.476084in}{1.416606in}}%
\pgfusepath{clip}%
\pgfsetroundcap%
\pgfsetroundjoin%
\pgfsetlinewidth{1.003750pt}%
\definecolor{currentstroke}{rgb}{0.800000,0.800000,0.800000}%
\pgfsetstrokecolor{currentstroke}%
\pgfsetdash{}{0pt}%
\pgfpathmoveto{\pgfqpoint{0.566440in}{0.791652in}}%
\pgfpathlineto{\pgfqpoint{3.042524in}{0.791652in}}%
\pgfusepath{stroke}%
\end{pgfscope}%
\begin{pgfscope}%
\definecolor{textcolor}{rgb}{0.150000,0.150000,0.150000}%
\pgfsetstrokecolor{textcolor}%
\pgfsetfillcolor{textcolor}%
\pgftext[x=0.263292in, y=0.744167in, left, base]{\color{textcolor}\sffamily\fontsize{9.000000}{10.800000}\selectfont 2.5}%
\end{pgfscope}%
\begin{pgfscope}%
\pgfpathrectangle{\pgfqpoint{0.566440in}{0.451389in}}{\pgfqpoint{2.476084in}{1.416606in}}%
\pgfusepath{clip}%
\pgfsetroundcap%
\pgfsetroundjoin%
\pgfsetlinewidth{1.003750pt}%
\definecolor{currentstroke}{rgb}{0.800000,0.800000,0.800000}%
\pgfsetstrokecolor{currentstroke}%
\pgfsetdash{}{0pt}%
\pgfpathmoveto{\pgfqpoint{0.566440in}{1.131915in}}%
\pgfpathlineto{\pgfqpoint{3.042524in}{1.131915in}}%
\pgfusepath{stroke}%
\end{pgfscope}%
\begin{pgfscope}%
\definecolor{textcolor}{rgb}{0.150000,0.150000,0.150000}%
\pgfsetstrokecolor{textcolor}%
\pgfsetfillcolor{textcolor}%
\pgftext[x=0.263292in, y=1.084430in, left, base]{\color{textcolor}\sffamily\fontsize{9.000000}{10.800000}\selectfont 5.0}%
\end{pgfscope}%
\begin{pgfscope}%
\pgfpathrectangle{\pgfqpoint{0.566440in}{0.451389in}}{\pgfqpoint{2.476084in}{1.416606in}}%
\pgfusepath{clip}%
\pgfsetroundcap%
\pgfsetroundjoin%
\pgfsetlinewidth{1.003750pt}%
\definecolor{currentstroke}{rgb}{0.800000,0.800000,0.800000}%
\pgfsetstrokecolor{currentstroke}%
\pgfsetdash{}{0pt}%
\pgfpathmoveto{\pgfqpoint{0.566440in}{1.472179in}}%
\pgfpathlineto{\pgfqpoint{3.042524in}{1.472179in}}%
\pgfusepath{stroke}%
\end{pgfscope}%
\begin{pgfscope}%
\definecolor{textcolor}{rgb}{0.150000,0.150000,0.150000}%
\pgfsetstrokecolor{textcolor}%
\pgfsetfillcolor{textcolor}%
\pgftext[x=0.263292in, y=1.424693in, left, base]{\color{textcolor}\sffamily\fontsize{9.000000}{10.800000}\selectfont 7.5}%
\end{pgfscope}%
\begin{pgfscope}%
\pgfpathrectangle{\pgfqpoint{0.566440in}{0.451389in}}{\pgfqpoint{2.476084in}{1.416606in}}%
\pgfusepath{clip}%
\pgfsetroundcap%
\pgfsetroundjoin%
\pgfsetlinewidth{1.003750pt}%
\definecolor{currentstroke}{rgb}{0.800000,0.800000,0.800000}%
\pgfsetstrokecolor{currentstroke}%
\pgfsetdash{}{0pt}%
\pgfpathmoveto{\pgfqpoint{0.566440in}{1.812442in}}%
\pgfpathlineto{\pgfqpoint{3.042524in}{1.812442in}}%
\pgfusepath{stroke}%
\end{pgfscope}%
\begin{pgfscope}%
\definecolor{textcolor}{rgb}{0.150000,0.150000,0.150000}%
\pgfsetstrokecolor{textcolor}%
\pgfsetfillcolor{textcolor}%
\pgftext[x=0.194444in, y=1.764957in, left, base]{\color{textcolor}\sffamily\fontsize{9.000000}{10.800000}\selectfont 10.0}%
\end{pgfscope}%
\begin{pgfscope}%
\definecolor{textcolor}{rgb}{0.150000,0.150000,0.150000}%
\pgfsetstrokecolor{textcolor}%
\pgfsetfillcolor{textcolor}%
\pgftext[x=0.125000in,y=1.159692in,,bottom,rotate=90.000000]{\color{textcolor}\sffamily\fontsize{9.000000}{10.800000}\selectfont Time in ms}%
\end{pgfscope}%
\begin{pgfscope}%
\pgfpathrectangle{\pgfqpoint{0.566440in}{0.451389in}}{\pgfqpoint{2.476084in}{1.416606in}}%
\pgfusepath{clip}%
\pgfsetbuttcap%
\pgfsetroundjoin%
\definecolor{currentfill}{rgb}{0.003922,0.450980,0.698039}%
\pgfsetfillcolor{currentfill}%
\pgfsetfillopacity{0.200000}%
\pgfsetlinewidth{1.003750pt}%
\definecolor{currentstroke}{rgb}{0.003922,0.450980,0.698039}%
\pgfsetstrokecolor{currentstroke}%
\pgfsetstrokeopacity{0.200000}%
\pgfsetdash{}{0pt}%
\pgfsys@defobject{currentmarker}{\pgfqpoint{0.940961in}{0.636736in}}{\pgfqpoint{2.942450in}{1.809363in}}{%
\pgfpathmoveto{\pgfqpoint{0.940961in}{0.643347in}}%
\pgfpathlineto{\pgfqpoint{0.940961in}{0.636736in}}%
\pgfpathlineto{\pgfqpoint{1.245074in}{0.753707in}}%
\pgfpathlineto{\pgfqpoint{1.510869in}{0.858688in}}%
\pgfpathlineto{\pgfqpoint{1.769857in}{0.988946in}}%
\pgfpathlineto{\pgfqpoint{2.372224in}{1.312447in}}%
\pgfpathlineto{\pgfqpoint{2.942450in}{1.741654in}}%
\pgfpathlineto{\pgfqpoint{2.942450in}{1.809363in}}%
\pgfpathlineto{\pgfqpoint{2.942450in}{1.809363in}}%
\pgfpathlineto{\pgfqpoint{2.372224in}{1.323634in}}%
\pgfpathlineto{\pgfqpoint{1.769857in}{1.000971in}}%
\pgfpathlineto{\pgfqpoint{1.510869in}{0.867902in}}%
\pgfpathlineto{\pgfqpoint{1.245074in}{0.763513in}}%
\pgfpathlineto{\pgfqpoint{0.940961in}{0.643347in}}%
\pgfpathlineto{\pgfqpoint{0.940961in}{0.643347in}}%
\pgfpathclose%
\pgfusepath{stroke,fill}%
}%
\begin{pgfscope}%
\pgfsys@transformshift{0.000000in}{0.000000in}%
\pgfsys@useobject{currentmarker}{}%
\end{pgfscope}%
\end{pgfscope}%
\begin{pgfscope}%
\pgfsetrectcap%
\pgfsetmiterjoin%
\pgfsetlinewidth{1.254687pt}%
\definecolor{currentstroke}{rgb}{0.800000,0.800000,0.800000}%
\pgfsetstrokecolor{currentstroke}%
\pgfsetdash{}{0pt}%
\pgfpathmoveto{\pgfqpoint{0.566440in}{0.451389in}}%
\pgfpathlineto{\pgfqpoint{0.566440in}{1.867995in}}%
\pgfusepath{stroke}%
\end{pgfscope}%
\begin{pgfscope}%
\pgfsetrectcap%
\pgfsetmiterjoin%
\pgfsetlinewidth{1.254687pt}%
\definecolor{currentstroke}{rgb}{0.800000,0.800000,0.800000}%
\pgfsetstrokecolor{currentstroke}%
\pgfsetdash{}{0pt}%
\pgfpathmoveto{\pgfqpoint{3.042524in}{0.451389in}}%
\pgfpathlineto{\pgfqpoint{3.042524in}{1.867995in}}%
\pgfusepath{stroke}%
\end{pgfscope}%
\begin{pgfscope}%
\pgfsetrectcap%
\pgfsetmiterjoin%
\pgfsetlinewidth{1.254687pt}%
\definecolor{currentstroke}{rgb}{0.800000,0.800000,0.800000}%
\pgfsetstrokecolor{currentstroke}%
\pgfsetdash{}{0pt}%
\pgfpathmoveto{\pgfqpoint{0.566440in}{0.451389in}}%
\pgfpathlineto{\pgfqpoint{3.042524in}{0.451389in}}%
\pgfusepath{stroke}%
\end{pgfscope}%
\begin{pgfscope}%
\pgfsetrectcap%
\pgfsetmiterjoin%
\pgfsetlinewidth{1.254687pt}%
\definecolor{currentstroke}{rgb}{0.800000,0.800000,0.800000}%
\pgfsetstrokecolor{currentstroke}%
\pgfsetdash{}{0pt}%
\pgfpathmoveto{\pgfqpoint{0.566440in}{1.867995in}}%
\pgfpathlineto{\pgfqpoint{3.042524in}{1.867995in}}%
\pgfusepath{stroke}%
\end{pgfscope}%
\begin{pgfscope}%
\pgfsetroundcap%
\pgfsetroundjoin%
\pgfsetlinewidth{1.003750pt}%
\definecolor{currentstroke}{rgb}{0.003922,0.450980,0.698039}%
\pgfsetstrokecolor{currentstroke}%
\pgfsetdash{}{0pt}%
\pgfpathmoveto{\pgfqpoint{0.940961in}{0.639510in}}%
\pgfpathlineto{\pgfqpoint{1.245074in}{0.759212in}}%
\pgfpathlineto{\pgfqpoint{1.510869in}{0.863992in}}%
\pgfpathlineto{\pgfqpoint{1.769857in}{0.996407in}}%
\pgfpathlineto{\pgfqpoint{2.372224in}{1.317860in}}%
\pgfpathlineto{\pgfqpoint{2.942450in}{1.773630in}}%
\pgfusepath{stroke}%
\end{pgfscope}%
\begin{pgfscope}%
\pgfsetbuttcap%
\pgfsetroundjoin%
\definecolor{currentfill}{rgb}{0.003922,0.450980,0.698039}%
\pgfsetfillcolor{currentfill}%
\pgfsetlinewidth{0.752812pt}%
\definecolor{currentstroke}{rgb}{1.000000,1.000000,1.000000}%
\pgfsetstrokecolor{currentstroke}%
\pgfsetdash{}{0pt}%
\pgfsys@defobject{currentmarker}{\pgfqpoint{-0.034722in}{-0.034722in}}{\pgfqpoint{0.034722in}{0.034722in}}{%
\pgfpathmoveto{\pgfqpoint{0.000000in}{-0.034722in}}%
\pgfpathcurveto{\pgfqpoint{0.009208in}{-0.034722in}}{\pgfqpoint{0.018041in}{-0.031064in}}{\pgfqpoint{0.024552in}{-0.024552in}}%
\pgfpathcurveto{\pgfqpoint{0.031064in}{-0.018041in}}{\pgfqpoint{0.034722in}{-0.009208in}}{\pgfqpoint{0.034722in}{0.000000in}}%
\pgfpathcurveto{\pgfqpoint{0.034722in}{0.009208in}}{\pgfqpoint{0.031064in}{0.018041in}}{\pgfqpoint{0.024552in}{0.024552in}}%
\pgfpathcurveto{\pgfqpoint{0.018041in}{0.031064in}}{\pgfqpoint{0.009208in}{0.034722in}}{\pgfqpoint{0.000000in}{0.034722in}}%
\pgfpathcurveto{\pgfqpoint{-0.009208in}{0.034722in}}{\pgfqpoint{-0.018041in}{0.031064in}}{\pgfqpoint{-0.024552in}{0.024552in}}%
\pgfpathcurveto{\pgfqpoint{-0.031064in}{0.018041in}}{\pgfqpoint{-0.034722in}{0.009208in}}{\pgfqpoint{-0.034722in}{0.000000in}}%
\pgfpathcurveto{\pgfqpoint{-0.034722in}{-0.009208in}}{\pgfqpoint{-0.031064in}{-0.018041in}}{\pgfqpoint{-0.024552in}{-0.024552in}}%
\pgfpathcurveto{\pgfqpoint{-0.018041in}{-0.031064in}}{\pgfqpoint{-0.009208in}{-0.034722in}}{\pgfqpoint{0.000000in}{-0.034722in}}%
\pgfpathlineto{\pgfqpoint{0.000000in}{-0.034722in}}%
\pgfpathclose%
\pgfusepath{stroke,fill}%
}%
\begin{pgfscope}%
\pgfsys@transformshift{0.940961in}{0.639510in}%
\pgfsys@useobject{currentmarker}{}%
\end{pgfscope}%
\begin{pgfscope}%
\pgfsys@transformshift{1.245074in}{0.759212in}%
\pgfsys@useobject{currentmarker}{}%
\end{pgfscope}%
\begin{pgfscope}%
\pgfsys@transformshift{1.510869in}{0.863992in}%
\pgfsys@useobject{currentmarker}{}%
\end{pgfscope}%
\begin{pgfscope}%
\pgfsys@transformshift{1.769857in}{0.996407in}%
\pgfsys@useobject{currentmarker}{}%
\end{pgfscope}%
\begin{pgfscope}%
\pgfsys@transformshift{2.372224in}{1.317860in}%
\pgfsys@useobject{currentmarker}{}%
\end{pgfscope}%
\begin{pgfscope}%
\pgfsys@transformshift{2.942450in}{1.773630in}%
\pgfsys@useobject{currentmarker}{}%
\end{pgfscope}%
\end{pgfscope}%
\end{pgfpicture}%
\makeatother%
\endgroup%

						\end{figcenter}
						\caption{Graph generation times per input vertex.}
						\label{fig:eval-import-rural-rel}
					\end{subfigure}
%					\\[3ex]
%					\begin{subfigure}[t]{\linewidth}
%						\begin{figcenter}
%							%% Creator: Matplotlib, PGF backend
%%
%% To include the figure in your LaTeX document, write
%%   \input{<filename>.pgf}
%%
%% Make sure the required packages are loaded in your preamble
%%   \usepackage{pgf}
%%
%% Also ensure that all the required font packages are loaded; for instance,
%% the lmodern package is sometimes necessary when using math font.
%%   \usepackage{lmodern}
%%
%% Figures using additional raster images can only be included by \input if
%% they are in the same directory as the main LaTeX file. For loading figures
%% from other directories you can use the `import` package
%%   \usepackage{import}
%%
%% and then include the figures with
%%   \import{<path to file>}{<filename>.pgf}
%%
%% Matplotlib used the following preamble
%%   
%%   \usepackage{fontspec}
%%   \setmainfont{DejaVuSerif.ttf}[Path=\detokenize{/home/hauke/.local/lib/python3.11/site-packages/matplotlib/mpl-data/fonts/ttf/}]
%%   \setsansfont{DroidSans.ttf}[Path=\detokenize{/usr/share/fonts/droid/}]
%%   \setmonofont{DejaVuSansMono.ttf}[Path=\detokenize{/home/hauke/.local/lib/python3.11/site-packages/matplotlib/mpl-data/fonts/ttf/}]
%%   \makeatletter\@ifpackageloaded{underscore}{}{\usepackage[strings]{underscore}}\makeatother
%%
\begingroup%
\makeatletter%
\begin{pgfpicture}%
\pgfpathrectangle{\pgfpointorigin}{\pgfqpoint{2.681552in}{1.770898in}}%
\pgfusepath{use as bounding box, clip}%
\begin{pgfscope}%
\pgfsetbuttcap%
\pgfsetmiterjoin%
\definecolor{currentfill}{rgb}{1.000000,1.000000,1.000000}%
\pgfsetfillcolor{currentfill}%
\pgfsetlinewidth{0.000000pt}%
\definecolor{currentstroke}{rgb}{1.000000,1.000000,1.000000}%
\pgfsetstrokecolor{currentstroke}%
\pgfsetdash{}{0pt}%
\pgfpathmoveto{\pgfqpoint{0.000000in}{0.000000in}}%
\pgfpathlineto{\pgfqpoint{2.681552in}{0.000000in}}%
\pgfpathlineto{\pgfqpoint{2.681552in}{1.770898in}}%
\pgfpathlineto{\pgfqpoint{0.000000in}{1.770898in}}%
\pgfpathlineto{\pgfqpoint{0.000000in}{0.000000in}}%
\pgfpathclose%
\pgfusepath{fill}%
\end{pgfscope}%
\begin{pgfscope}%
\pgfsetbuttcap%
\pgfsetmiterjoin%
\definecolor{currentfill}{rgb}{1.000000,1.000000,1.000000}%
\pgfsetfillcolor{currentfill}%
\pgfsetlinewidth{0.000000pt}%
\definecolor{currentstroke}{rgb}{0.000000,0.000000,0.000000}%
\pgfsetstrokecolor{currentstroke}%
\pgfsetstrokeopacity{0.000000}%
\pgfsetdash{}{0pt}%
\pgfpathmoveto{\pgfqpoint{0.395236in}{0.451389in}}%
\pgfpathlineto{\pgfqpoint{2.667104in}{0.451389in}}%
\pgfpathlineto{\pgfqpoint{2.667104in}{1.770898in}}%
\pgfpathlineto{\pgfqpoint{0.395236in}{1.770898in}}%
\pgfpathlineto{\pgfqpoint{0.395236in}{0.451389in}}%
\pgfpathclose%
\pgfusepath{fill}%
\end{pgfscope}%
\begin{pgfscope}%
\pgfpathrectangle{\pgfqpoint{0.395236in}{0.451389in}}{\pgfqpoint{2.271868in}{1.319509in}}%
\pgfusepath{clip}%
\pgfsetroundcap%
\pgfsetroundjoin%
\pgfsetlinewidth{1.003750pt}%
\definecolor{currentstroke}{rgb}{0.800000,0.800000,0.800000}%
\pgfsetstrokecolor{currentstroke}%
\pgfsetdash{}{0pt}%
\pgfpathmoveto{\pgfqpoint{1.090675in}{0.451389in}}%
\pgfpathlineto{\pgfqpoint{1.090675in}{1.770898in}}%
\pgfusepath{stroke}%
\end{pgfscope}%
\begin{pgfscope}%
\definecolor{textcolor}{rgb}{0.150000,0.150000,0.150000}%
\pgfsetstrokecolor{textcolor}%
\pgfsetfillcolor{textcolor}%
\pgftext[x=1.090675in,y=0.319444in,,top]{\color{textcolor}\sffamily\fontsize{9.000000}{10.800000}\selectfont 2000}%
\end{pgfscope}%
\begin{pgfscope}%
\pgfpathrectangle{\pgfqpoint{0.395236in}{0.451389in}}{\pgfqpoint{2.271868in}{1.319509in}}%
\pgfusepath{clip}%
\pgfsetroundcap%
\pgfsetroundjoin%
\pgfsetlinewidth{1.003750pt}%
\definecolor{currentstroke}{rgb}{0.800000,0.800000,0.800000}%
\pgfsetstrokecolor{currentstroke}%
\pgfsetdash{}{0pt}%
\pgfpathmoveto{\pgfqpoint{1.817266in}{0.451389in}}%
\pgfpathlineto{\pgfqpoint{1.817266in}{1.770898in}}%
\pgfusepath{stroke}%
\end{pgfscope}%
\begin{pgfscope}%
\definecolor{textcolor}{rgb}{0.150000,0.150000,0.150000}%
\pgfsetstrokecolor{textcolor}%
\pgfsetfillcolor{textcolor}%
\pgftext[x=1.817266in,y=0.319444in,,top]{\color{textcolor}\sffamily\fontsize{9.000000}{10.800000}\selectfont 4000}%
\end{pgfscope}%
\begin{pgfscope}%
\pgfpathrectangle{\pgfqpoint{0.395236in}{0.451389in}}{\pgfqpoint{2.271868in}{1.319509in}}%
\pgfusepath{clip}%
\pgfsetroundcap%
\pgfsetroundjoin%
\pgfsetlinewidth{1.003750pt}%
\definecolor{currentstroke}{rgb}{0.800000,0.800000,0.800000}%
\pgfsetstrokecolor{currentstroke}%
\pgfsetdash{}{0pt}%
\pgfpathmoveto{\pgfqpoint{2.543856in}{0.451389in}}%
\pgfpathlineto{\pgfqpoint{2.543856in}{1.770898in}}%
\pgfusepath{stroke}%
\end{pgfscope}%
\begin{pgfscope}%
\definecolor{textcolor}{rgb}{0.150000,0.150000,0.150000}%
\pgfsetstrokecolor{textcolor}%
\pgfsetfillcolor{textcolor}%
\pgftext[x=2.543856in,y=0.319444in,,top]{\color{textcolor}\sffamily\fontsize{9.000000}{10.800000}\selectfont 6000}%
\end{pgfscope}%
\begin{pgfscope}%
\definecolor{textcolor}{rgb}{0.150000,0.150000,0.150000}%
\pgfsetstrokecolor{textcolor}%
\pgfsetfillcolor{textcolor}%
\pgftext[x=1.531170in,y=0.125000in,,top]{\color{textcolor}\sffamily\fontsize{9.000000}{10.800000}\selectfont Input obstacle vertices}%
\end{pgfscope}%
\begin{pgfscope}%
\pgfpathrectangle{\pgfqpoint{0.395236in}{0.451389in}}{\pgfqpoint{2.271868in}{1.319509in}}%
\pgfusepath{clip}%
\pgfsetroundcap%
\pgfsetroundjoin%
\pgfsetlinewidth{1.003750pt}%
\definecolor{currentstroke}{rgb}{0.800000,0.800000,0.800000}%
\pgfsetstrokecolor{currentstroke}%
\pgfsetdash{}{0pt}%
\pgfpathmoveto{\pgfqpoint{0.395236in}{0.752767in}}%
\pgfpathlineto{\pgfqpoint{2.667104in}{0.752767in}}%
\pgfusepath{stroke}%
\end{pgfscope}%
\begin{pgfscope}%
\definecolor{textcolor}{rgb}{0.150000,0.150000,0.150000}%
\pgfsetstrokecolor{textcolor}%
\pgfsetfillcolor{textcolor}%
\pgftext[x=0.194444in, y=0.705281in, left, base]{\color{textcolor}\sffamily\fontsize{9.000000}{10.800000}\selectfont 2}%
\end{pgfscope}%
\begin{pgfscope}%
\pgfpathrectangle{\pgfqpoint{0.395236in}{0.451389in}}{\pgfqpoint{2.271868in}{1.319509in}}%
\pgfusepath{clip}%
\pgfsetroundcap%
\pgfsetroundjoin%
\pgfsetlinewidth{1.003750pt}%
\definecolor{currentstroke}{rgb}{0.800000,0.800000,0.800000}%
\pgfsetstrokecolor{currentstroke}%
\pgfsetdash{}{0pt}%
\pgfpathmoveto{\pgfqpoint{0.395236in}{1.169918in}}%
\pgfpathlineto{\pgfqpoint{2.667104in}{1.169918in}}%
\pgfusepath{stroke}%
\end{pgfscope}%
\begin{pgfscope}%
\definecolor{textcolor}{rgb}{0.150000,0.150000,0.150000}%
\pgfsetstrokecolor{textcolor}%
\pgfsetfillcolor{textcolor}%
\pgftext[x=0.194444in, y=1.122432in, left, base]{\color{textcolor}\sffamily\fontsize{9.000000}{10.800000}\selectfont 4}%
\end{pgfscope}%
\begin{pgfscope}%
\pgfpathrectangle{\pgfqpoint{0.395236in}{0.451389in}}{\pgfqpoint{2.271868in}{1.319509in}}%
\pgfusepath{clip}%
\pgfsetroundcap%
\pgfsetroundjoin%
\pgfsetlinewidth{1.003750pt}%
\definecolor{currentstroke}{rgb}{0.800000,0.800000,0.800000}%
\pgfsetstrokecolor{currentstroke}%
\pgfsetdash{}{0pt}%
\pgfpathmoveto{\pgfqpoint{0.395236in}{1.587069in}}%
\pgfpathlineto{\pgfqpoint{2.667104in}{1.587069in}}%
\pgfusepath{stroke}%
\end{pgfscope}%
\begin{pgfscope}%
\definecolor{textcolor}{rgb}{0.150000,0.150000,0.150000}%
\pgfsetstrokecolor{textcolor}%
\pgfsetfillcolor{textcolor}%
\pgftext[x=0.194444in, y=1.539583in, left, base]{\color{textcolor}\sffamily\fontsize{9.000000}{10.800000}\selectfont 6}%
\end{pgfscope}%
\begin{pgfscope}%
\definecolor{textcolor}{rgb}{0.150000,0.150000,0.150000}%
\pgfsetstrokecolor{textcolor}%
\pgfsetfillcolor{textcolor}%
\pgftext[x=0.125000in,y=1.111143in,,bottom,rotate=90.000000]{\color{textcolor}\sffamily\fontsize{9.000000}{10.800000}\selectfont Time in µs}%
\end{pgfscope}%
\begin{pgfscope}%
\pgfpathrectangle{\pgfqpoint{0.395236in}{0.451389in}}{\pgfqpoint{2.271868in}{1.319509in}}%
\pgfusepath{clip}%
\pgfsetbuttcap%
\pgfsetroundjoin%
\definecolor{currentfill}{rgb}{0.003922,0.450980,0.698039}%
\pgfsetfillcolor{currentfill}%
\pgfsetfillopacity{0.200000}%
\pgfsetlinewidth{1.003750pt}%
\definecolor{currentstroke}{rgb}{0.003922,0.450980,0.698039}%
\pgfsetstrokecolor{currentstroke}%
\pgfsetstrokeopacity{0.200000}%
\pgfsetdash{}{0pt}%
\pgfsys@defobject{currentmarker}{\pgfqpoint{0.498503in}{0.511366in}}{\pgfqpoint{2.563838in}{1.710920in}}{%
\pgfpathmoveto{\pgfqpoint{0.498503in}{1.710920in}}%
\pgfpathlineto{\pgfqpoint{0.498503in}{1.697441in}}%
\pgfpathlineto{\pgfqpoint{0.696499in}{1.051520in}}%
\pgfpathlineto{\pgfqpoint{0.927555in}{0.739411in}}%
\pgfpathlineto{\pgfqpoint{1.186948in}{0.617518in}}%
\pgfpathlineto{\pgfqpoint{1.961857in}{0.546162in}}%
\pgfpathlineto{\pgfqpoint{2.563838in}{0.511366in}}%
\pgfpathlineto{\pgfqpoint{2.563838in}{0.512535in}}%
\pgfpathlineto{\pgfqpoint{2.563838in}{0.512535in}}%
\pgfpathlineto{\pgfqpoint{1.961857in}{0.546722in}}%
\pgfpathlineto{\pgfqpoint{1.186948in}{0.627357in}}%
\pgfpathlineto{\pgfqpoint{0.927555in}{0.748479in}}%
\pgfpathlineto{\pgfqpoint{0.696499in}{1.062369in}}%
\pgfpathlineto{\pgfqpoint{0.498503in}{1.710920in}}%
\pgfpathlineto{\pgfqpoint{0.498503in}{1.710920in}}%
\pgfpathclose%
\pgfusepath{stroke,fill}%
}%
\begin{pgfscope}%
\pgfsys@transformshift{0.000000in}{0.000000in}%
\pgfsys@useobject{currentmarker}{}%
\end{pgfscope}%
\end{pgfscope}%
\begin{pgfscope}%
\pgfsetrectcap%
\pgfsetmiterjoin%
\pgfsetlinewidth{1.254687pt}%
\definecolor{currentstroke}{rgb}{0.800000,0.800000,0.800000}%
\pgfsetstrokecolor{currentstroke}%
\pgfsetdash{}{0pt}%
\pgfpathmoveto{\pgfqpoint{0.395236in}{0.451389in}}%
\pgfpathlineto{\pgfqpoint{0.395236in}{1.770898in}}%
\pgfusepath{stroke}%
\end{pgfscope}%
\begin{pgfscope}%
\pgfsetrectcap%
\pgfsetmiterjoin%
\pgfsetlinewidth{1.254687pt}%
\definecolor{currentstroke}{rgb}{0.800000,0.800000,0.800000}%
\pgfsetstrokecolor{currentstroke}%
\pgfsetdash{}{0pt}%
\pgfpathmoveto{\pgfqpoint{2.667104in}{0.451389in}}%
\pgfpathlineto{\pgfqpoint{2.667104in}{1.770898in}}%
\pgfusepath{stroke}%
\end{pgfscope}%
\begin{pgfscope}%
\pgfsetrectcap%
\pgfsetmiterjoin%
\pgfsetlinewidth{1.254687pt}%
\definecolor{currentstroke}{rgb}{0.800000,0.800000,0.800000}%
\pgfsetstrokecolor{currentstroke}%
\pgfsetdash{}{0pt}%
\pgfpathmoveto{\pgfqpoint{0.395236in}{0.451389in}}%
\pgfpathlineto{\pgfqpoint{2.667104in}{0.451389in}}%
\pgfusepath{stroke}%
\end{pgfscope}%
\begin{pgfscope}%
\pgfsetrectcap%
\pgfsetmiterjoin%
\pgfsetlinewidth{1.254687pt}%
\definecolor{currentstroke}{rgb}{0.800000,0.800000,0.800000}%
\pgfsetstrokecolor{currentstroke}%
\pgfsetdash{}{0pt}%
\pgfpathmoveto{\pgfqpoint{0.395236in}{1.770898in}}%
\pgfpathlineto{\pgfqpoint{2.667104in}{1.770898in}}%
\pgfusepath{stroke}%
\end{pgfscope}%
\begin{pgfscope}%
\pgfsetroundcap%
\pgfsetroundjoin%
\pgfsetlinewidth{1.003750pt}%
\definecolor{currentstroke}{rgb}{0.003922,0.450980,0.698039}%
\pgfsetstrokecolor{currentstroke}%
\pgfsetdash{}{0pt}%
\pgfpathmoveto{\pgfqpoint{0.498503in}{1.703124in}}%
\pgfpathlineto{\pgfqpoint{0.696499in}{1.059132in}}%
\pgfpathlineto{\pgfqpoint{0.927555in}{0.743127in}}%
\pgfpathlineto{\pgfqpoint{1.186948in}{0.622715in}}%
\pgfpathlineto{\pgfqpoint{1.961857in}{0.546408in}}%
\pgfpathlineto{\pgfqpoint{2.563838in}{0.511899in}}%
\pgfusepath{stroke}%
\end{pgfscope}%
\begin{pgfscope}%
\pgfsetbuttcap%
\pgfsetroundjoin%
\definecolor{currentfill}{rgb}{0.003922,0.450980,0.698039}%
\pgfsetfillcolor{currentfill}%
\pgfsetlinewidth{0.752812pt}%
\definecolor{currentstroke}{rgb}{1.000000,1.000000,1.000000}%
\pgfsetstrokecolor{currentstroke}%
\pgfsetdash{}{0pt}%
\pgfsys@defobject{currentmarker}{\pgfqpoint{-0.034722in}{-0.034722in}}{\pgfqpoint{0.034722in}{0.034722in}}{%
\pgfpathmoveto{\pgfqpoint{0.000000in}{-0.034722in}}%
\pgfpathcurveto{\pgfqpoint{0.009208in}{-0.034722in}}{\pgfqpoint{0.018041in}{-0.031064in}}{\pgfqpoint{0.024552in}{-0.024552in}}%
\pgfpathcurveto{\pgfqpoint{0.031064in}{-0.018041in}}{\pgfqpoint{0.034722in}{-0.009208in}}{\pgfqpoint{0.034722in}{0.000000in}}%
\pgfpathcurveto{\pgfqpoint{0.034722in}{0.009208in}}{\pgfqpoint{0.031064in}{0.018041in}}{\pgfqpoint{0.024552in}{0.024552in}}%
\pgfpathcurveto{\pgfqpoint{0.018041in}{0.031064in}}{\pgfqpoint{0.009208in}{0.034722in}}{\pgfqpoint{0.000000in}{0.034722in}}%
\pgfpathcurveto{\pgfqpoint{-0.009208in}{0.034722in}}{\pgfqpoint{-0.018041in}{0.031064in}}{\pgfqpoint{-0.024552in}{0.024552in}}%
\pgfpathcurveto{\pgfqpoint{-0.031064in}{0.018041in}}{\pgfqpoint{-0.034722in}{0.009208in}}{\pgfqpoint{-0.034722in}{0.000000in}}%
\pgfpathcurveto{\pgfqpoint{-0.034722in}{-0.009208in}}{\pgfqpoint{-0.031064in}{-0.018041in}}{\pgfqpoint{-0.024552in}{-0.024552in}}%
\pgfpathcurveto{\pgfqpoint{-0.018041in}{-0.031064in}}{\pgfqpoint{-0.009208in}{-0.034722in}}{\pgfqpoint{0.000000in}{-0.034722in}}%
\pgfpathlineto{\pgfqpoint{0.000000in}{-0.034722in}}%
\pgfpathclose%
\pgfusepath{stroke,fill}%
}%
\begin{pgfscope}%
\pgfsys@transformshift{0.498503in}{1.703124in}%
\pgfsys@useobject{currentmarker}{}%
\end{pgfscope}%
\begin{pgfscope}%
\pgfsys@transformshift{0.696499in}{1.059132in}%
\pgfsys@useobject{currentmarker}{}%
\end{pgfscope}%
\begin{pgfscope}%
\pgfsys@transformshift{0.927555in}{0.743127in}%
\pgfsys@useobject{currentmarker}{}%
\end{pgfscope}%
\begin{pgfscope}%
\pgfsys@transformshift{1.186948in}{0.622715in}%
\pgfsys@useobject{currentmarker}{}%
\end{pgfscope}%
\begin{pgfscope}%
\pgfsys@transformshift{1.961857in}{0.546408in}%
\pgfsys@useobject{currentmarker}{}%
\end{pgfscope}%
\begin{pgfscope}%
\pgfsys@transformshift{2.563838in}{0.511899in}%
\pgfsys@useobject{currentmarker}{}%
\end{pgfscope}%
\end{pgfscope}%
\end{pgfpicture}%
\makeatother%
\endgroup%

%						\end{figcenter}
%						\caption{Increase in graph generation time per additionally added vertex.}
%						\label{fig:eval-import-rural-rel-increase}
%					\end{subfigure}
					\caption{Graph generation times using the \enquote{OSM rural} datasets.}
					\label{fig:eval-import-rural}
				\end{minipage}
			\end{figure}
			
			\begin{figure}[h!]
				\begin{figcenter}
					\begin{subfigure}[t]{\textwidth}
						\begin{figcenter}
							%% Creator: Matplotlib, PGF backend
%%
%% To include the figure in your LaTeX document, write
%%   \input{<filename>.pgf}
%%
%% Make sure the required packages are loaded in your preamble
%%   \usepackage{pgf}
%%
%% Also ensure that all the required font packages are loaded; for instance,
%% the lmodern package is sometimes necessary when using math font.
%%   \usepackage{lmodern}
%%
%% Figures using additional raster images can only be included by \input if
%% they are in the same directory as the main LaTeX file. For loading figures
%% from other directories you can use the `import` package
%%   \usepackage{import}
%%
%% and then include the figures with
%%   \import{<path to file>}{<filename>.pgf}
%%
%% Matplotlib used the following preamble
%%   
%%   \usepackage{fontspec}
%%   \setmainfont{DejaVuSerif.ttf}[Path=\detokenize{/home/hauke/.local/lib/python3.11/site-packages/matplotlib/mpl-data/fonts/ttf/}]
%%   \setsansfont{DroidSans.ttf}[Path=\detokenize{/usr/share/fonts/droid/}]
%%   \setmonofont{DejaVuSansMono.ttf}[Path=\detokenize{/home/hauke/.local/lib/python3.11/site-packages/matplotlib/mpl-data/fonts/ttf/}]
%%   \makeatletter\@ifpackageloaded{underscore}{}{\usepackage[strings]{underscore}}\makeatother
%%
\begingroup%
\makeatletter%
\begin{pgfpicture}%
\pgfpathrectangle{\pgfpointorigin}{\pgfqpoint{5.694946in}{2.407400in}}%
\pgfusepath{use as bounding box, clip}%
\begin{pgfscope}%
\pgfsetbuttcap%
\pgfsetmiterjoin%
\definecolor{currentfill}{rgb}{1.000000,1.000000,1.000000}%
\pgfsetfillcolor{currentfill}%
\pgfsetlinewidth{0.000000pt}%
\definecolor{currentstroke}{rgb}{1.000000,1.000000,1.000000}%
\pgfsetstrokecolor{currentstroke}%
\pgfsetdash{}{0pt}%
\pgfpathmoveto{\pgfqpoint{0.000000in}{0.000000in}}%
\pgfpathlineto{\pgfqpoint{5.694946in}{0.000000in}}%
\pgfpathlineto{\pgfqpoint{5.694946in}{2.407400in}}%
\pgfpathlineto{\pgfqpoint{0.000000in}{2.407400in}}%
\pgfpathlineto{\pgfqpoint{0.000000in}{0.000000in}}%
\pgfpathclose%
\pgfusepath{fill}%
\end{pgfscope}%
\begin{pgfscope}%
\pgfsetbuttcap%
\pgfsetmiterjoin%
\definecolor{currentfill}{rgb}{1.000000,1.000000,1.000000}%
\pgfsetfillcolor{currentfill}%
\pgfsetlinewidth{0.000000pt}%
\definecolor{currentstroke}{rgb}{0.000000,0.000000,0.000000}%
\pgfsetstrokecolor{currentstroke}%
\pgfsetstrokeopacity{0.000000}%
\pgfsetdash{}{0pt}%
\pgfpathmoveto{\pgfqpoint{0.592976in}{0.451389in}}%
\pgfpathlineto{\pgfqpoint{4.295331in}{0.451389in}}%
\pgfpathlineto{\pgfqpoint{4.295331in}{2.407400in}}%
\pgfpathlineto{\pgfqpoint{0.592976in}{2.407400in}}%
\pgfpathlineto{\pgfqpoint{0.592976in}{0.451389in}}%
\pgfpathclose%
\pgfusepath{fill}%
\end{pgfscope}%
\begin{pgfscope}%
\pgfpathrectangle{\pgfqpoint{0.592976in}{0.451389in}}{\pgfqpoint{3.702355in}{1.956011in}}%
\pgfusepath{clip}%
\pgfsetroundcap%
\pgfsetroundjoin%
\pgfsetlinewidth{1.003750pt}%
\definecolor{currentstroke}{rgb}{0.800000,0.800000,0.800000}%
\pgfsetstrokecolor{currentstroke}%
\pgfsetdash{}{0pt}%
\pgfpathmoveto{\pgfqpoint{0.727267in}{0.451389in}}%
\pgfpathlineto{\pgfqpoint{0.727267in}{2.407400in}}%
\pgfusepath{stroke}%
\end{pgfscope}%
\begin{pgfscope}%
\definecolor{textcolor}{rgb}{0.150000,0.150000,0.150000}%
\pgfsetstrokecolor{textcolor}%
\pgfsetfillcolor{textcolor}%
\pgftext[x=0.727267in,y=0.319444in,,top]{\color{textcolor}\sffamily\fontsize{9.000000}{10.800000}\selectfont 0}%
\end{pgfscope}%
\begin{pgfscope}%
\pgfpathrectangle{\pgfqpoint{0.592976in}{0.451389in}}{\pgfqpoint{3.702355in}{1.956011in}}%
\pgfusepath{clip}%
\pgfsetroundcap%
\pgfsetroundjoin%
\pgfsetlinewidth{1.003750pt}%
\definecolor{currentstroke}{rgb}{0.800000,0.800000,0.800000}%
\pgfsetstrokecolor{currentstroke}%
\pgfsetdash{}{0pt}%
\pgfpathmoveto{\pgfqpoint{1.492982in}{0.451389in}}%
\pgfpathlineto{\pgfqpoint{1.492982in}{2.407400in}}%
\pgfusepath{stroke}%
\end{pgfscope}%
\begin{pgfscope}%
\definecolor{textcolor}{rgb}{0.150000,0.150000,0.150000}%
\pgfsetstrokecolor{textcolor}%
\pgfsetfillcolor{textcolor}%
\pgftext[x=1.492982in,y=0.319444in,,top]{\color{textcolor}\sffamily\fontsize{9.000000}{10.800000}\selectfont 10000}%
\end{pgfscope}%
\begin{pgfscope}%
\pgfpathrectangle{\pgfqpoint{0.592976in}{0.451389in}}{\pgfqpoint{3.702355in}{1.956011in}}%
\pgfusepath{clip}%
\pgfsetroundcap%
\pgfsetroundjoin%
\pgfsetlinewidth{1.003750pt}%
\definecolor{currentstroke}{rgb}{0.800000,0.800000,0.800000}%
\pgfsetstrokecolor{currentstroke}%
\pgfsetdash{}{0pt}%
\pgfpathmoveto{\pgfqpoint{2.258697in}{0.451389in}}%
\pgfpathlineto{\pgfqpoint{2.258697in}{2.407400in}}%
\pgfusepath{stroke}%
\end{pgfscope}%
\begin{pgfscope}%
\definecolor{textcolor}{rgb}{0.150000,0.150000,0.150000}%
\pgfsetstrokecolor{textcolor}%
\pgfsetfillcolor{textcolor}%
\pgftext[x=2.258697in,y=0.319444in,,top]{\color{textcolor}\sffamily\fontsize{9.000000}{10.800000}\selectfont 20000}%
\end{pgfscope}%
\begin{pgfscope}%
\pgfpathrectangle{\pgfqpoint{0.592976in}{0.451389in}}{\pgfqpoint{3.702355in}{1.956011in}}%
\pgfusepath{clip}%
\pgfsetroundcap%
\pgfsetroundjoin%
\pgfsetlinewidth{1.003750pt}%
\definecolor{currentstroke}{rgb}{0.800000,0.800000,0.800000}%
\pgfsetstrokecolor{currentstroke}%
\pgfsetdash{}{0pt}%
\pgfpathmoveto{\pgfqpoint{3.024412in}{0.451389in}}%
\pgfpathlineto{\pgfqpoint{3.024412in}{2.407400in}}%
\pgfusepath{stroke}%
\end{pgfscope}%
\begin{pgfscope}%
\definecolor{textcolor}{rgb}{0.150000,0.150000,0.150000}%
\pgfsetstrokecolor{textcolor}%
\pgfsetfillcolor{textcolor}%
\pgftext[x=3.024412in,y=0.319444in,,top]{\color{textcolor}\sffamily\fontsize{9.000000}{10.800000}\selectfont 30000}%
\end{pgfscope}%
\begin{pgfscope}%
\pgfpathrectangle{\pgfqpoint{0.592976in}{0.451389in}}{\pgfqpoint{3.702355in}{1.956011in}}%
\pgfusepath{clip}%
\pgfsetroundcap%
\pgfsetroundjoin%
\pgfsetlinewidth{1.003750pt}%
\definecolor{currentstroke}{rgb}{0.800000,0.800000,0.800000}%
\pgfsetstrokecolor{currentstroke}%
\pgfsetdash{}{0pt}%
\pgfpathmoveto{\pgfqpoint{3.790128in}{0.451389in}}%
\pgfpathlineto{\pgfqpoint{3.790128in}{2.407400in}}%
\pgfusepath{stroke}%
\end{pgfscope}%
\begin{pgfscope}%
\definecolor{textcolor}{rgb}{0.150000,0.150000,0.150000}%
\pgfsetstrokecolor{textcolor}%
\pgfsetfillcolor{textcolor}%
\pgftext[x=3.790128in,y=0.319444in,,top]{\color{textcolor}\sffamily\fontsize{9.000000}{10.800000}\selectfont 40000}%
\end{pgfscope}%
\begin{pgfscope}%
\definecolor{textcolor}{rgb}{0.150000,0.150000,0.150000}%
\pgfsetstrokecolor{textcolor}%
\pgfsetfillcolor{textcolor}%
\pgftext[x=2.444154in,y=0.125000in,,top]{\color{textcolor}\sffamily\fontsize{9.000000}{10.800000}\selectfont Input obstacle vertices}%
\end{pgfscope}%
\begin{pgfscope}%
\pgfpathrectangle{\pgfqpoint{0.592976in}{0.451389in}}{\pgfqpoint{3.702355in}{1.956011in}}%
\pgfusepath{clip}%
\pgfsetroundcap%
\pgfsetroundjoin%
\pgfsetlinewidth{1.003750pt}%
\definecolor{currentstroke}{rgb}{0.800000,0.800000,0.800000}%
\pgfsetstrokecolor{currentstroke}%
\pgfsetdash{}{0pt}%
\pgfpathmoveto{\pgfqpoint{0.592976in}{0.772958in}}%
\pgfpathlineto{\pgfqpoint{4.295331in}{0.772958in}}%
\pgfusepath{stroke}%
\end{pgfscope}%
\begin{pgfscope}%
\definecolor{textcolor}{rgb}{0.150000,0.150000,0.150000}%
\pgfsetstrokecolor{textcolor}%
\pgfsetfillcolor{textcolor}%
\pgftext[x=0.194444in, y=0.725473in, left, base]{\color{textcolor}\sffamily\fontsize{9.000000}{10.800000}\selectfont \(\displaystyle {10^{-4}}\)}%
\end{pgfscope}%
\begin{pgfscope}%
\pgfpathrectangle{\pgfqpoint{0.592976in}{0.451389in}}{\pgfqpoint{3.702355in}{1.956011in}}%
\pgfusepath{clip}%
\pgfsetroundcap%
\pgfsetroundjoin%
\pgfsetlinewidth{1.003750pt}%
\definecolor{currentstroke}{rgb}{0.800000,0.800000,0.800000}%
\pgfsetstrokecolor{currentstroke}%
\pgfsetdash{}{0pt}%
\pgfpathmoveto{\pgfqpoint{0.592976in}{1.197168in}}%
\pgfpathlineto{\pgfqpoint{4.295331in}{1.197168in}}%
\pgfusepath{stroke}%
\end{pgfscope}%
\begin{pgfscope}%
\definecolor{textcolor}{rgb}{0.150000,0.150000,0.150000}%
\pgfsetstrokecolor{textcolor}%
\pgfsetfillcolor{textcolor}%
\pgftext[x=0.194444in, y=1.149682in, left, base]{\color{textcolor}\sffamily\fontsize{9.000000}{10.800000}\selectfont \(\displaystyle {10^{-2}}\)}%
\end{pgfscope}%
\begin{pgfscope}%
\pgfpathrectangle{\pgfqpoint{0.592976in}{0.451389in}}{\pgfqpoint{3.702355in}{1.956011in}}%
\pgfusepath{clip}%
\pgfsetroundcap%
\pgfsetroundjoin%
\pgfsetlinewidth{1.003750pt}%
\definecolor{currentstroke}{rgb}{0.800000,0.800000,0.800000}%
\pgfsetstrokecolor{currentstroke}%
\pgfsetdash{}{0pt}%
\pgfpathmoveto{\pgfqpoint{0.592976in}{1.621377in}}%
\pgfpathlineto{\pgfqpoint{4.295331in}{1.621377in}}%
\pgfusepath{stroke}%
\end{pgfscope}%
\begin{pgfscope}%
\definecolor{textcolor}{rgb}{0.150000,0.150000,0.150000}%
\pgfsetstrokecolor{textcolor}%
\pgfsetfillcolor{textcolor}%
\pgftext[x=0.274690in, y=1.573892in, left, base]{\color{textcolor}\sffamily\fontsize{9.000000}{10.800000}\selectfont \(\displaystyle {10^{0}}\)}%
\end{pgfscope}%
\begin{pgfscope}%
\pgfpathrectangle{\pgfqpoint{0.592976in}{0.451389in}}{\pgfqpoint{3.702355in}{1.956011in}}%
\pgfusepath{clip}%
\pgfsetroundcap%
\pgfsetroundjoin%
\pgfsetlinewidth{1.003750pt}%
\definecolor{currentstroke}{rgb}{0.800000,0.800000,0.800000}%
\pgfsetstrokecolor{currentstroke}%
\pgfsetdash{}{0pt}%
\pgfpathmoveto{\pgfqpoint{0.592976in}{2.045587in}}%
\pgfpathlineto{\pgfqpoint{4.295331in}{2.045587in}}%
\pgfusepath{stroke}%
\end{pgfscope}%
\begin{pgfscope}%
\definecolor{textcolor}{rgb}{0.150000,0.150000,0.150000}%
\pgfsetstrokecolor{textcolor}%
\pgfsetfillcolor{textcolor}%
\pgftext[x=0.274690in, y=1.998102in, left, base]{\color{textcolor}\sffamily\fontsize{9.000000}{10.800000}\selectfont \(\displaystyle {10^{2}}\)}%
\end{pgfscope}%
\begin{pgfscope}%
\definecolor{textcolor}{rgb}{0.150000,0.150000,0.150000}%
\pgfsetstrokecolor{textcolor}%
\pgfsetfillcolor{textcolor}%
\pgftext[x=0.125000in,y=1.429394in,,bottom,rotate=90.000000]{\color{textcolor}\sffamily\fontsize{9.000000}{10.800000}\selectfont Time in s}%
\end{pgfscope}%
\begin{pgfscope}%
\pgfpathrectangle{\pgfqpoint{0.592976in}{0.451389in}}{\pgfqpoint{3.702355in}{1.956011in}}%
\pgfusepath{clip}%
\pgfsetbuttcap%
\pgfsetroundjoin%
\definecolor{currentfill}{rgb}{0.003922,0.450980,0.698039}%
\pgfsetfillcolor{currentfill}%
\pgfsetfillopacity{0.200000}%
\pgfsetlinewidth{1.003750pt}%
\definecolor{currentstroke}{rgb}{0.003922,0.450980,0.698039}%
\pgfsetstrokecolor{currentstroke}%
\pgfsetstrokeopacity{0.200000}%
\pgfsetdash{}{0pt}%
\pgfsys@defobject{currentmarker}{\pgfqpoint{0.761265in}{1.455970in}}{\pgfqpoint{4.127042in}{2.318490in}}{%
\pgfpathmoveto{\pgfqpoint{0.761265in}{1.459695in}}%
\pgfpathlineto{\pgfqpoint{0.761265in}{1.455970in}}%
\pgfpathlineto{\pgfqpoint{0.863258in}{1.698752in}}%
\pgfpathlineto{\pgfqpoint{1.033247in}{1.839247in}}%
\pgfpathlineto{\pgfqpoint{1.271231in}{1.948660in}}%
\pgfpathlineto{\pgfqpoint{1.577211in}{2.034583in}}%
\pgfpathlineto{\pgfqpoint{1.951186in}{2.105413in}}%
\pgfpathlineto{\pgfqpoint{2.393157in}{2.171804in}}%
\pgfpathlineto{\pgfqpoint{2.903123in}{2.231881in}}%
\pgfpathlineto{\pgfqpoint{3.481085in}{2.275011in}}%
\pgfpathlineto{\pgfqpoint{4.127042in}{2.317590in}}%
\pgfpathlineto{\pgfqpoint{4.127042in}{2.318490in}}%
\pgfpathlineto{\pgfqpoint{4.127042in}{2.318490in}}%
\pgfpathlineto{\pgfqpoint{3.481085in}{2.275480in}}%
\pgfpathlineto{\pgfqpoint{2.903123in}{2.232255in}}%
\pgfpathlineto{\pgfqpoint{2.393157in}{2.172637in}}%
\pgfpathlineto{\pgfqpoint{1.951186in}{2.106090in}}%
\pgfpathlineto{\pgfqpoint{1.577211in}{2.036093in}}%
\pgfpathlineto{\pgfqpoint{1.271231in}{1.949698in}}%
\pgfpathlineto{\pgfqpoint{1.033247in}{1.839946in}}%
\pgfpathlineto{\pgfqpoint{0.863258in}{1.699562in}}%
\pgfpathlineto{\pgfqpoint{0.761265in}{1.459695in}}%
\pgfpathlineto{\pgfqpoint{0.761265in}{1.459695in}}%
\pgfpathclose%
\pgfusepath{stroke,fill}%
}%
\begin{pgfscope}%
\pgfsys@transformshift{0.000000in}{0.000000in}%
\pgfsys@useobject{currentmarker}{}%
\end{pgfscope}%
\end{pgfscope}%
\begin{pgfscope}%
\pgfpathrectangle{\pgfqpoint{0.592976in}{0.451389in}}{\pgfqpoint{3.702355in}{1.956011in}}%
\pgfusepath{clip}%
\pgfsetbuttcap%
\pgfsetroundjoin%
\definecolor{currentfill}{rgb}{0.870588,0.560784,0.019608}%
\pgfsetfillcolor{currentfill}%
\pgfsetfillopacity{0.200000}%
\pgfsetlinewidth{1.003750pt}%
\definecolor{currentstroke}{rgb}{0.870588,0.560784,0.019608}%
\pgfsetstrokecolor{currentstroke}%
\pgfsetstrokeopacity{0.200000}%
\pgfsetdash{}{0pt}%
\pgfsys@defobject{currentmarker}{\pgfqpoint{0.761265in}{1.443516in}}{\pgfqpoint{4.127042in}{2.318333in}}{%
\pgfpathmoveto{\pgfqpoint{0.761265in}{1.445992in}}%
\pgfpathlineto{\pgfqpoint{0.761265in}{1.443516in}}%
\pgfpathlineto{\pgfqpoint{0.863258in}{1.694446in}}%
\pgfpathlineto{\pgfqpoint{1.033247in}{1.837070in}}%
\pgfpathlineto{\pgfqpoint{1.271231in}{1.947364in}}%
\pgfpathlineto{\pgfqpoint{1.577211in}{2.033736in}}%
\pgfpathlineto{\pgfqpoint{1.951186in}{2.104814in}}%
\pgfpathlineto{\pgfqpoint{2.393157in}{2.171401in}}%
\pgfpathlineto{\pgfqpoint{2.903123in}{2.231635in}}%
\pgfpathlineto{\pgfqpoint{3.481085in}{2.274812in}}%
\pgfpathlineto{\pgfqpoint{4.127042in}{2.317434in}}%
\pgfpathlineto{\pgfqpoint{4.127042in}{2.318333in}}%
\pgfpathlineto{\pgfqpoint{4.127042in}{2.318333in}}%
\pgfpathlineto{\pgfqpoint{3.481085in}{2.275284in}}%
\pgfpathlineto{\pgfqpoint{2.903123in}{2.232006in}}%
\pgfpathlineto{\pgfqpoint{2.393157in}{2.172236in}}%
\pgfpathlineto{\pgfqpoint{1.951186in}{2.105492in}}%
\pgfpathlineto{\pgfqpoint{1.577211in}{2.035233in}}%
\pgfpathlineto{\pgfqpoint{1.271231in}{1.948337in}}%
\pgfpathlineto{\pgfqpoint{1.033247in}{1.837620in}}%
\pgfpathlineto{\pgfqpoint{0.863258in}{1.695130in}}%
\pgfpathlineto{\pgfqpoint{0.761265in}{1.445992in}}%
\pgfpathlineto{\pgfqpoint{0.761265in}{1.445992in}}%
\pgfpathclose%
\pgfusepath{stroke,fill}%
}%
\begin{pgfscope}%
\pgfsys@transformshift{0.000000in}{0.000000in}%
\pgfsys@useobject{currentmarker}{}%
\end{pgfscope}%
\end{pgfscope}%
\begin{pgfscope}%
\pgfpathrectangle{\pgfqpoint{0.592976in}{0.451389in}}{\pgfqpoint{3.702355in}{1.956011in}}%
\pgfusepath{clip}%
\pgfsetbuttcap%
\pgfsetroundjoin%
\definecolor{currentfill}{rgb}{0.007843,0.619608,0.450980}%
\pgfsetfillcolor{currentfill}%
\pgfsetfillopacity{0.200000}%
\pgfsetlinewidth{1.003750pt}%
\definecolor{currentstroke}{rgb}{0.007843,0.619608,0.450980}%
\pgfsetstrokecolor{currentstroke}%
\pgfsetstrokeopacity{0.200000}%
\pgfsetdash{}{0pt}%
\pgfsys@defobject{currentmarker}{\pgfqpoint{0.761265in}{1.124594in}}{\pgfqpoint{4.127042in}{1.654244in}}{%
\pgfpathmoveto{\pgfqpoint{0.761265in}{1.133971in}}%
\pgfpathlineto{\pgfqpoint{0.761265in}{1.124594in}}%
\pgfpathlineto{\pgfqpoint{0.863258in}{1.326006in}}%
\pgfpathlineto{\pgfqpoint{1.033247in}{1.403940in}}%
\pgfpathlineto{\pgfqpoint{1.271231in}{1.470302in}}%
\pgfpathlineto{\pgfqpoint{1.577211in}{1.526399in}}%
\pgfpathlineto{\pgfqpoint{1.951186in}{1.549531in}}%
\pgfpathlineto{\pgfqpoint{2.393157in}{1.599545in}}%
\pgfpathlineto{\pgfqpoint{2.903123in}{1.604145in}}%
\pgfpathlineto{\pgfqpoint{3.481085in}{1.627141in}}%
\pgfpathlineto{\pgfqpoint{4.127042in}{1.648914in}}%
\pgfpathlineto{\pgfqpoint{4.127042in}{1.654244in}}%
\pgfpathlineto{\pgfqpoint{4.127042in}{1.654244in}}%
\pgfpathlineto{\pgfqpoint{3.481085in}{1.630110in}}%
\pgfpathlineto{\pgfqpoint{2.903123in}{1.606052in}}%
\pgfpathlineto{\pgfqpoint{2.393157in}{1.602590in}}%
\pgfpathlineto{\pgfqpoint{1.951186in}{1.571006in}}%
\pgfpathlineto{\pgfqpoint{1.577211in}{1.544497in}}%
\pgfpathlineto{\pgfqpoint{1.271231in}{1.493922in}}%
\pgfpathlineto{\pgfqpoint{1.033247in}{1.407747in}}%
\pgfpathlineto{\pgfqpoint{0.863258in}{1.327811in}}%
\pgfpathlineto{\pgfqpoint{0.761265in}{1.133971in}}%
\pgfpathlineto{\pgfqpoint{0.761265in}{1.133971in}}%
\pgfpathclose%
\pgfusepath{stroke,fill}%
}%
\begin{pgfscope}%
\pgfsys@transformshift{0.000000in}{0.000000in}%
\pgfsys@useobject{currentmarker}{}%
\end{pgfscope}%
\end{pgfscope}%
\begin{pgfscope}%
\pgfpathrectangle{\pgfqpoint{0.592976in}{0.451389in}}{\pgfqpoint{3.702355in}{1.956011in}}%
\pgfusepath{clip}%
\pgfsetbuttcap%
\pgfsetroundjoin%
\definecolor{currentfill}{rgb}{0.835294,0.368627,0.000000}%
\pgfsetfillcolor{currentfill}%
\pgfsetfillopacity{0.200000}%
\pgfsetlinewidth{1.003750pt}%
\definecolor{currentstroke}{rgb}{0.835294,0.368627,0.000000}%
\pgfsetstrokecolor{currentstroke}%
\pgfsetstrokeopacity{0.200000}%
\pgfsetdash{}{0pt}%
\pgfsys@defobject{currentmarker}{\pgfqpoint{0.761265in}{1.236018in}}{\pgfqpoint{4.127042in}{1.675506in}}{%
\pgfpathmoveto{\pgfqpoint{0.761265in}{1.237726in}}%
\pgfpathlineto{\pgfqpoint{0.761265in}{1.236018in}}%
\pgfpathlineto{\pgfqpoint{0.863258in}{1.364291in}}%
\pgfpathlineto{\pgfqpoint{1.033247in}{1.439446in}}%
\pgfpathlineto{\pgfqpoint{1.271231in}{1.493302in}}%
\pgfpathlineto{\pgfqpoint{1.577211in}{1.534060in}}%
\pgfpathlineto{\pgfqpoint{1.951186in}{1.581022in}}%
\pgfpathlineto{\pgfqpoint{2.393157in}{1.608007in}}%
\pgfpathlineto{\pgfqpoint{2.903123in}{1.631255in}}%
\pgfpathlineto{\pgfqpoint{3.481085in}{1.652086in}}%
\pgfpathlineto{\pgfqpoint{4.127042in}{1.673084in}}%
\pgfpathlineto{\pgfqpoint{4.127042in}{1.675506in}}%
\pgfpathlineto{\pgfqpoint{4.127042in}{1.675506in}}%
\pgfpathlineto{\pgfqpoint{3.481085in}{1.654816in}}%
\pgfpathlineto{\pgfqpoint{2.903123in}{1.632744in}}%
\pgfpathlineto{\pgfqpoint{2.393157in}{1.609090in}}%
\pgfpathlineto{\pgfqpoint{1.951186in}{1.582701in}}%
\pgfpathlineto{\pgfqpoint{1.577211in}{1.536338in}}%
\pgfpathlineto{\pgfqpoint{1.271231in}{1.499967in}}%
\pgfpathlineto{\pgfqpoint{1.033247in}{1.451798in}}%
\pgfpathlineto{\pgfqpoint{0.863258in}{1.365406in}}%
\pgfpathlineto{\pgfqpoint{0.761265in}{1.237726in}}%
\pgfpathlineto{\pgfqpoint{0.761265in}{1.237726in}}%
\pgfpathclose%
\pgfusepath{stroke,fill}%
}%
\begin{pgfscope}%
\pgfsys@transformshift{0.000000in}{0.000000in}%
\pgfsys@useobject{currentmarker}{}%
\end{pgfscope}%
\end{pgfscope}%
\begin{pgfscope}%
\pgfpathrectangle{\pgfqpoint{0.592976in}{0.451389in}}{\pgfqpoint{3.702355in}{1.956011in}}%
\pgfusepath{clip}%
\pgfsetbuttcap%
\pgfsetroundjoin%
\definecolor{currentfill}{rgb}{0.800000,0.470588,0.737255}%
\pgfsetfillcolor{currentfill}%
\pgfsetfillopacity{0.200000}%
\pgfsetlinewidth{1.003750pt}%
\definecolor{currentstroke}{rgb}{0.800000,0.470588,0.737255}%
\pgfsetstrokecolor{currentstroke}%
\pgfsetstrokeopacity{0.200000}%
\pgfsetdash{}{0pt}%
\pgfsys@defobject{currentmarker}{\pgfqpoint{0.761265in}{0.576100in}}{\pgfqpoint{4.127042in}{0.985925in}}{%
\pgfpathmoveto{\pgfqpoint{0.761265in}{0.639950in}}%
\pgfpathlineto{\pgfqpoint{0.761265in}{0.576100in}}%
\pgfpathlineto{\pgfqpoint{0.863258in}{0.622842in}}%
\pgfpathlineto{\pgfqpoint{1.033247in}{0.668572in}}%
\pgfpathlineto{\pgfqpoint{1.271231in}{0.700218in}}%
\pgfpathlineto{\pgfqpoint{1.577211in}{0.730397in}}%
\pgfpathlineto{\pgfqpoint{1.951186in}{0.766075in}}%
\pgfpathlineto{\pgfqpoint{2.393157in}{0.828525in}}%
\pgfpathlineto{\pgfqpoint{2.903123in}{0.846835in}}%
\pgfpathlineto{\pgfqpoint{3.481085in}{0.890130in}}%
\pgfpathlineto{\pgfqpoint{4.127042in}{0.908215in}}%
\pgfpathlineto{\pgfqpoint{4.127042in}{0.964208in}}%
\pgfpathlineto{\pgfqpoint{4.127042in}{0.964208in}}%
\pgfpathlineto{\pgfqpoint{3.481085in}{0.953150in}}%
\pgfpathlineto{\pgfqpoint{2.903123in}{0.928643in}}%
\pgfpathlineto{\pgfqpoint{2.393157in}{0.923613in}}%
\pgfpathlineto{\pgfqpoint{1.951186in}{0.916952in}}%
\pgfpathlineto{\pgfqpoint{1.577211in}{0.904845in}}%
\pgfpathlineto{\pgfqpoint{1.271231in}{0.891463in}}%
\pgfpathlineto{\pgfqpoint{1.033247in}{0.985925in}}%
\pgfpathlineto{\pgfqpoint{0.863258in}{0.664477in}}%
\pgfpathlineto{\pgfqpoint{0.761265in}{0.639950in}}%
\pgfpathlineto{\pgfqpoint{0.761265in}{0.639950in}}%
\pgfpathclose%
\pgfusepath{stroke,fill}%
}%
\begin{pgfscope}%
\pgfsys@transformshift{0.000000in}{0.000000in}%
\pgfsys@useobject{currentmarker}{}%
\end{pgfscope}%
\end{pgfscope}%
\begin{pgfscope}%
\pgfpathrectangle{\pgfqpoint{0.592976in}{0.451389in}}{\pgfqpoint{3.702355in}{1.956011in}}%
\pgfusepath{clip}%
\pgfsetbuttcap%
\pgfsetroundjoin%
\definecolor{currentfill}{rgb}{0.792157,0.568627,0.380392}%
\pgfsetfillcolor{currentfill}%
\pgfsetfillopacity{0.200000}%
\pgfsetlinewidth{1.003750pt}%
\definecolor{currentstroke}{rgb}{0.792157,0.568627,0.380392}%
\pgfsetstrokecolor{currentstroke}%
\pgfsetstrokeopacity{0.200000}%
\pgfsetdash{}{0pt}%
\pgfsys@defobject{currentmarker}{\pgfqpoint{0.761265in}{0.540298in}}{\pgfqpoint{4.127042in}{0.930294in}}{%
\pgfpathmoveto{\pgfqpoint{0.761265in}{0.662053in}}%
\pgfpathlineto{\pgfqpoint{0.761265in}{0.540298in}}%
\pgfpathlineto{\pgfqpoint{0.863258in}{0.610810in}}%
\pgfpathlineto{\pgfqpoint{1.033247in}{0.706112in}}%
\pgfpathlineto{\pgfqpoint{1.271231in}{0.735792in}}%
\pgfpathlineto{\pgfqpoint{1.577211in}{0.775143in}}%
\pgfpathlineto{\pgfqpoint{1.951186in}{0.800603in}}%
\pgfpathlineto{\pgfqpoint{2.393157in}{0.862800in}}%
\pgfpathlineto{\pgfqpoint{2.903123in}{0.883383in}}%
\pgfpathlineto{\pgfqpoint{3.481085in}{0.900196in}}%
\pgfpathlineto{\pgfqpoint{4.127042in}{0.918975in}}%
\pgfpathlineto{\pgfqpoint{4.127042in}{0.930294in}}%
\pgfpathlineto{\pgfqpoint{4.127042in}{0.930294in}}%
\pgfpathlineto{\pgfqpoint{3.481085in}{0.909730in}}%
\pgfpathlineto{\pgfqpoint{2.903123in}{0.896127in}}%
\pgfpathlineto{\pgfqpoint{2.393157in}{0.878359in}}%
\pgfpathlineto{\pgfqpoint{1.951186in}{0.838270in}}%
\pgfpathlineto{\pgfqpoint{1.577211in}{0.800875in}}%
\pgfpathlineto{\pgfqpoint{1.271231in}{0.774782in}}%
\pgfpathlineto{\pgfqpoint{1.033247in}{0.752633in}}%
\pgfpathlineto{\pgfqpoint{0.863258in}{0.713427in}}%
\pgfpathlineto{\pgfqpoint{0.761265in}{0.662053in}}%
\pgfpathlineto{\pgfqpoint{0.761265in}{0.662053in}}%
\pgfpathclose%
\pgfusepath{stroke,fill}%
}%
\begin{pgfscope}%
\pgfsys@transformshift{0.000000in}{0.000000in}%
\pgfsys@useobject{currentmarker}{}%
\end{pgfscope}%
\end{pgfscope}%
\begin{pgfscope}%
\pgfsetrectcap%
\pgfsetmiterjoin%
\pgfsetlinewidth{1.254687pt}%
\definecolor{currentstroke}{rgb}{0.800000,0.800000,0.800000}%
\pgfsetstrokecolor{currentstroke}%
\pgfsetdash{}{0pt}%
\pgfpathmoveto{\pgfqpoint{0.592976in}{0.451389in}}%
\pgfpathlineto{\pgfqpoint{0.592976in}{2.407400in}}%
\pgfusepath{stroke}%
\end{pgfscope}%
\begin{pgfscope}%
\pgfsetrectcap%
\pgfsetmiterjoin%
\pgfsetlinewidth{1.254687pt}%
\definecolor{currentstroke}{rgb}{0.800000,0.800000,0.800000}%
\pgfsetstrokecolor{currentstroke}%
\pgfsetdash{}{0pt}%
\pgfpathmoveto{\pgfqpoint{4.295331in}{0.451389in}}%
\pgfpathlineto{\pgfqpoint{4.295331in}{2.407400in}}%
\pgfusepath{stroke}%
\end{pgfscope}%
\begin{pgfscope}%
\pgfsetrectcap%
\pgfsetmiterjoin%
\pgfsetlinewidth{1.254687pt}%
\definecolor{currentstroke}{rgb}{0.800000,0.800000,0.800000}%
\pgfsetstrokecolor{currentstroke}%
\pgfsetdash{}{0pt}%
\pgfpathmoveto{\pgfqpoint{0.592976in}{0.451389in}}%
\pgfpathlineto{\pgfqpoint{4.295331in}{0.451389in}}%
\pgfusepath{stroke}%
\end{pgfscope}%
\begin{pgfscope}%
\pgfsetrectcap%
\pgfsetmiterjoin%
\pgfsetlinewidth{1.254687pt}%
\definecolor{currentstroke}{rgb}{0.800000,0.800000,0.800000}%
\pgfsetstrokecolor{currentstroke}%
\pgfsetdash{}{0pt}%
\pgfpathmoveto{\pgfqpoint{0.592976in}{2.407400in}}%
\pgfpathlineto{\pgfqpoint{4.295331in}{2.407400in}}%
\pgfusepath{stroke}%
\end{pgfscope}%
\begin{pgfscope}%
\pgfsetbuttcap%
\pgfsetmiterjoin%
\definecolor{currentfill}{rgb}{1.000000,1.000000,1.000000}%
\pgfsetfillcolor{currentfill}%
\pgfsetfillopacity{0.800000}%
\pgfsetlinewidth{1.003750pt}%
\definecolor{currentstroke}{rgb}{0.800000,0.800000,0.800000}%
\pgfsetstrokecolor{currentstroke}%
\pgfsetstrokeopacity{0.800000}%
\pgfsetdash{}{0pt}%
\pgfpathmoveto{\pgfqpoint{4.475390in}{0.538404in}}%
\pgfpathlineto{\pgfqpoint{5.669946in}{0.538404in}}%
\pgfpathquadraticcurveto{\pgfqpoint{5.694946in}{0.538404in}}{\pgfqpoint{5.694946in}{0.563404in}}%
\pgfpathlineto{\pgfqpoint{5.694946in}{2.295385in}}%
\pgfpathquadraticcurveto{\pgfqpoint{5.694946in}{2.320385in}}{\pgfqpoint{5.669946in}{2.320385in}}%
\pgfpathlineto{\pgfqpoint{4.475390in}{2.320385in}}%
\pgfpathquadraticcurveto{\pgfqpoint{4.450390in}{2.320385in}}{\pgfqpoint{4.450390in}{2.295385in}}%
\pgfpathlineto{\pgfqpoint{4.450390in}{0.563404in}}%
\pgfpathquadraticcurveto{\pgfqpoint{4.450390in}{0.538404in}}{\pgfqpoint{4.475390in}{0.538404in}}%
\pgfpathlineto{\pgfqpoint{4.475390in}{0.538404in}}%
\pgfpathclose%
\pgfusepath{stroke,fill}%
\end{pgfscope}%
\begin{pgfscope}%
\definecolor{textcolor}{rgb}{0.150000,0.150000,0.150000}%
\pgfsetstrokecolor{textcolor}%
\pgfsetfillcolor{textcolor}%
\pgftext[x=4.869268in,y=2.175414in,left,base]{\color{textcolor}\sffamily\fontsize{9.000000}{10.800000}\selectfont Legend}%
\end{pgfscope}%
\begin{pgfscope}%
\pgfsetroundcap%
\pgfsetroundjoin%
\pgfsetlinewidth{1.505625pt}%
\definecolor{currentstroke}{rgb}{0.003922,0.450980,0.698039}%
\pgfsetstrokecolor{currentstroke}%
\pgfsetdash{}{0pt}%
\pgfpathmoveto{\pgfqpoint{4.500390in}{2.031664in}}%
\pgfpathlineto{\pgfqpoint{4.625390in}{2.031664in}}%
\pgfpathlineto{\pgfqpoint{4.750390in}{2.031664in}}%
\pgfusepath{stroke}%
\end{pgfscope}%
\begin{pgfscope}%
\definecolor{textcolor}{rgb}{0.150000,0.150000,0.150000}%
\pgfsetstrokecolor{textcolor}%
\pgfsetfillcolor{textcolor}%
\pgftext[x=4.850390in,y=1.987914in,left,base]{\color{textcolor}\sffamily\fontsize{9.000000}{10.800000}\selectfont Total time}%
\end{pgfscope}%
\begin{pgfscope}%
\pgfsetroundcap%
\pgfsetroundjoin%
\pgfsetlinewidth{1.505625pt}%
\definecolor{currentstroke}{rgb}{0.870588,0.560784,0.019608}%
\pgfsetstrokecolor{currentstroke}%
\pgfsetdash{}{0pt}%
\pgfpathmoveto{\pgfqpoint{4.500390in}{1.844164in}}%
\pgfpathlineto{\pgfqpoint{4.625390in}{1.844164in}}%
\pgfpathlineto{\pgfqpoint{4.750390in}{1.844164in}}%
\pgfusepath{stroke}%
\end{pgfscope}%
\begin{pgfscope}%
\definecolor{textcolor}{rgb}{0.150000,0.150000,0.150000}%
\pgfsetstrokecolor{textcolor}%
\pgfsetfillcolor{textcolor}%
\pgftext[x=4.850390in,y=1.800414in,left,base]{\color{textcolor}\sffamily\fontsize{9.000000}{10.800000}\selectfont kNN search}%
\end{pgfscope}%
\begin{pgfscope}%
\pgfsetroundcap%
\pgfsetroundjoin%
\pgfsetlinewidth{1.505625pt}%
\definecolor{currentstroke}{rgb}{0.007843,0.619608,0.450980}%
\pgfsetstrokecolor{currentstroke}%
\pgfsetdash{}{0pt}%
\pgfpathmoveto{\pgfqpoint{4.500390in}{1.656664in}}%
\pgfpathlineto{\pgfqpoint{4.625390in}{1.656664in}}%
\pgfpathlineto{\pgfqpoint{4.750390in}{1.656664in}}%
\pgfusepath{stroke}%
\end{pgfscope}%
\begin{pgfscope}%
\definecolor{textcolor}{rgb}{0.150000,0.150000,0.150000}%
\pgfsetstrokecolor{textcolor}%
\pgfsetfillcolor{textcolor}%
\pgftext[x=4.850390in,y=1.612914in,left,base]{\color{textcolor}\sffamily\fontsize{9.000000}{10.800000}\selectfont Create graph}%
\end{pgfscope}%
\begin{pgfscope}%
\pgfsetroundcap%
\pgfsetroundjoin%
\pgfsetlinewidth{1.505625pt}%
\definecolor{currentstroke}{rgb}{0.835294,0.368627,0.000000}%
\pgfsetstrokecolor{currentstroke}%
\pgfsetdash{}{0pt}%
\pgfpathmoveto{\pgfqpoint{4.500390in}{1.382153in}}%
\pgfpathlineto{\pgfqpoint{4.625390in}{1.382153in}}%
\pgfpathlineto{\pgfqpoint{4.750390in}{1.382153in}}%
\pgfusepath{stroke}%
\end{pgfscope}%
\begin{pgfscope}%
\definecolor{textcolor}{rgb}{0.150000,0.150000,0.150000}%
\pgfsetstrokecolor{textcolor}%
\pgfsetfillcolor{textcolor}%
\pgftext[x=4.850390in, y=1.425415in, left, base]{\color{textcolor}\sffamily\fontsize{9.000000}{10.800000}\selectfont Get \& prepare}%
\end{pgfscope}%
\begin{pgfscope}%
\definecolor{textcolor}{rgb}{0.150000,0.150000,0.150000}%
\pgfsetstrokecolor{textcolor}%
\pgfsetfillcolor{textcolor}%
\pgftext[x=4.850390in, y=1.281421in, left, base]{\color{textcolor}\sffamily\fontsize{9.000000}{10.800000}\selectfont obstacles}%
\end{pgfscope}%
\begin{pgfscope}%
\pgfsetroundcap%
\pgfsetroundjoin%
\pgfsetlinewidth{1.505625pt}%
\definecolor{currentstroke}{rgb}{0.800000,0.470588,0.737255}%
\pgfsetstrokecolor{currentstroke}%
\pgfsetdash{}{0pt}%
\pgfpathmoveto{\pgfqpoint{4.500390in}{1.050659in}}%
\pgfpathlineto{\pgfqpoint{4.625390in}{1.050659in}}%
\pgfpathlineto{\pgfqpoint{4.750390in}{1.050659in}}%
\pgfusepath{stroke}%
\end{pgfscope}%
\begin{pgfscope}%
\definecolor{textcolor}{rgb}{0.150000,0.150000,0.150000}%
\pgfsetstrokecolor{textcolor}%
\pgfsetfillcolor{textcolor}%
\pgftext[x=4.850390in, y=1.093921in, left, base]{\color{textcolor}\sffamily\fontsize{9.000000}{10.800000}\selectfont Merge road}%
\end{pgfscope}%
\begin{pgfscope}%
\definecolor{textcolor}{rgb}{0.150000,0.150000,0.150000}%
\pgfsetstrokecolor{textcolor}%
\pgfsetfillcolor{textcolor}%
\pgftext[x=4.850390in, y=0.949927in, left, base]{\color{textcolor}\sffamily\fontsize{9.000000}{10.800000}\selectfont edges}%
\end{pgfscope}%
\begin{pgfscope}%
\pgfsetroundcap%
\pgfsetroundjoin%
\pgfsetlinewidth{1.505625pt}%
\definecolor{currentstroke}{rgb}{0.792157,0.568627,0.380392}%
\pgfsetstrokecolor{currentstroke}%
\pgfsetdash{}{0pt}%
\pgfpathmoveto{\pgfqpoint{4.500390in}{0.719165in}}%
\pgfpathlineto{\pgfqpoint{4.625390in}{0.719165in}}%
\pgfpathlineto{\pgfqpoint{4.750390in}{0.719165in}}%
\pgfusepath{stroke}%
\end{pgfscope}%
\begin{pgfscope}%
\definecolor{textcolor}{rgb}{0.150000,0.150000,0.150000}%
\pgfsetstrokecolor{textcolor}%
\pgfsetfillcolor{textcolor}%
\pgftext[x=4.850390in, y=0.762427in, left, base]{\color{textcolor}\sffamily\fontsize{9.000000}{10.800000}\selectfont Add POI}%
\end{pgfscope}%
\begin{pgfscope}%
\definecolor{textcolor}{rgb}{0.150000,0.150000,0.150000}%
\pgfsetstrokecolor{textcolor}%
\pgfsetfillcolor{textcolor}%
\pgftext[x=4.850390in, y=0.618433in, left, base]{\color{textcolor}\sffamily\fontsize{9.000000}{10.800000}\selectfont attributes}%
\end{pgfscope}%
\begin{pgfscope}%
\pgfsetroundcap%
\pgfsetroundjoin%
\pgfsetlinewidth{1.003750pt}%
\definecolor{currentstroke}{rgb}{0.003922,0.450980,0.698039}%
\pgfsetstrokecolor{currentstroke}%
\pgfsetdash{}{0pt}%
\pgfpathmoveto{\pgfqpoint{0.761265in}{1.457007in}}%
\pgfpathlineto{\pgfqpoint{0.863258in}{1.699189in}}%
\pgfpathlineto{\pgfqpoint{1.033247in}{1.839654in}}%
\pgfpathlineto{\pgfqpoint{1.271231in}{1.949225in}}%
\pgfpathlineto{\pgfqpoint{1.577211in}{2.035316in}}%
\pgfpathlineto{\pgfqpoint{1.951186in}{2.105655in}}%
\pgfpathlineto{\pgfqpoint{2.393157in}{2.172206in}}%
\pgfpathlineto{\pgfqpoint{2.903123in}{2.232027in}}%
\pgfpathlineto{\pgfqpoint{3.481085in}{2.275222in}}%
\pgfpathlineto{\pgfqpoint{4.127042in}{2.318086in}}%
\pgfusepath{stroke}%
\end{pgfscope}%
\begin{pgfscope}%
\pgfsetbuttcap%
\pgfsetroundjoin%
\definecolor{currentfill}{rgb}{0.003922,0.450980,0.698039}%
\pgfsetfillcolor{currentfill}%
\pgfsetlinewidth{0.752812pt}%
\definecolor{currentstroke}{rgb}{1.000000,1.000000,1.000000}%
\pgfsetstrokecolor{currentstroke}%
\pgfsetdash{}{0pt}%
\pgfsys@defobject{currentmarker}{\pgfqpoint{-0.034722in}{-0.034722in}}{\pgfqpoint{0.034722in}{0.034722in}}{%
\pgfpathmoveto{\pgfqpoint{0.000000in}{-0.034722in}}%
\pgfpathcurveto{\pgfqpoint{0.009208in}{-0.034722in}}{\pgfqpoint{0.018041in}{-0.031064in}}{\pgfqpoint{0.024552in}{-0.024552in}}%
\pgfpathcurveto{\pgfqpoint{0.031064in}{-0.018041in}}{\pgfqpoint{0.034722in}{-0.009208in}}{\pgfqpoint{0.034722in}{0.000000in}}%
\pgfpathcurveto{\pgfqpoint{0.034722in}{0.009208in}}{\pgfqpoint{0.031064in}{0.018041in}}{\pgfqpoint{0.024552in}{0.024552in}}%
\pgfpathcurveto{\pgfqpoint{0.018041in}{0.031064in}}{\pgfqpoint{0.009208in}{0.034722in}}{\pgfqpoint{0.000000in}{0.034722in}}%
\pgfpathcurveto{\pgfqpoint{-0.009208in}{0.034722in}}{\pgfqpoint{-0.018041in}{0.031064in}}{\pgfqpoint{-0.024552in}{0.024552in}}%
\pgfpathcurveto{\pgfqpoint{-0.031064in}{0.018041in}}{\pgfqpoint{-0.034722in}{0.009208in}}{\pgfqpoint{-0.034722in}{0.000000in}}%
\pgfpathcurveto{\pgfqpoint{-0.034722in}{-0.009208in}}{\pgfqpoint{-0.031064in}{-0.018041in}}{\pgfqpoint{-0.024552in}{-0.024552in}}%
\pgfpathcurveto{\pgfqpoint{-0.018041in}{-0.031064in}}{\pgfqpoint{-0.009208in}{-0.034722in}}{\pgfqpoint{0.000000in}{-0.034722in}}%
\pgfpathlineto{\pgfqpoint{0.000000in}{-0.034722in}}%
\pgfpathclose%
\pgfusepath{stroke,fill}%
}%
\begin{pgfscope}%
\pgfsys@transformshift{0.761265in}{1.457007in}%
\pgfsys@useobject{currentmarker}{}%
\end{pgfscope}%
\begin{pgfscope}%
\pgfsys@transformshift{0.863258in}{1.699189in}%
\pgfsys@useobject{currentmarker}{}%
\end{pgfscope}%
\begin{pgfscope}%
\pgfsys@transformshift{1.033247in}{1.839654in}%
\pgfsys@useobject{currentmarker}{}%
\end{pgfscope}%
\begin{pgfscope}%
\pgfsys@transformshift{1.271231in}{1.949225in}%
\pgfsys@useobject{currentmarker}{}%
\end{pgfscope}%
\begin{pgfscope}%
\pgfsys@transformshift{1.577211in}{2.035316in}%
\pgfsys@useobject{currentmarker}{}%
\end{pgfscope}%
\begin{pgfscope}%
\pgfsys@transformshift{1.951186in}{2.105655in}%
\pgfsys@useobject{currentmarker}{}%
\end{pgfscope}%
\begin{pgfscope}%
\pgfsys@transformshift{2.393157in}{2.172206in}%
\pgfsys@useobject{currentmarker}{}%
\end{pgfscope}%
\begin{pgfscope}%
\pgfsys@transformshift{2.903123in}{2.232027in}%
\pgfsys@useobject{currentmarker}{}%
\end{pgfscope}%
\begin{pgfscope}%
\pgfsys@transformshift{3.481085in}{2.275222in}%
\pgfsys@useobject{currentmarker}{}%
\end{pgfscope}%
\begin{pgfscope}%
\pgfsys@transformshift{4.127042in}{2.318086in}%
\pgfsys@useobject{currentmarker}{}%
\end{pgfscope}%
\end{pgfscope}%
\begin{pgfscope}%
\pgfsetroundcap%
\pgfsetroundjoin%
\pgfsetlinewidth{1.003750pt}%
\definecolor{currentstroke}{rgb}{0.870588,0.560784,0.019608}%
\pgfsetstrokecolor{currentstroke}%
\pgfsetdash{}{0pt}%
\pgfpathmoveto{\pgfqpoint{0.761265in}{1.444228in}}%
\pgfpathlineto{\pgfqpoint{0.863258in}{1.694869in}}%
\pgfpathlineto{\pgfqpoint{1.033247in}{1.837426in}}%
\pgfpathlineto{\pgfqpoint{1.271231in}{1.947920in}}%
\pgfpathlineto{\pgfqpoint{1.577211in}{2.034456in}}%
\pgfpathlineto{\pgfqpoint{1.951186in}{2.105077in}}%
\pgfpathlineto{\pgfqpoint{2.393157in}{2.171802in}}%
\pgfpathlineto{\pgfqpoint{2.903123in}{2.231779in}}%
\pgfpathlineto{\pgfqpoint{3.481085in}{2.275025in}}%
\pgfpathlineto{\pgfqpoint{4.127042in}{2.317930in}}%
\pgfusepath{stroke}%
\end{pgfscope}%
\begin{pgfscope}%
\pgfsetbuttcap%
\pgfsetroundjoin%
\definecolor{currentfill}{rgb}{0.870588,0.560784,0.019608}%
\pgfsetfillcolor{currentfill}%
\pgfsetlinewidth{0.752812pt}%
\definecolor{currentstroke}{rgb}{1.000000,1.000000,1.000000}%
\pgfsetstrokecolor{currentstroke}%
\pgfsetdash{}{0pt}%
\pgfsys@defobject{currentmarker}{\pgfqpoint{-0.034722in}{-0.034722in}}{\pgfqpoint{0.034722in}{0.034722in}}{%
\pgfpathmoveto{\pgfqpoint{0.000000in}{-0.034722in}}%
\pgfpathcurveto{\pgfqpoint{0.009208in}{-0.034722in}}{\pgfqpoint{0.018041in}{-0.031064in}}{\pgfqpoint{0.024552in}{-0.024552in}}%
\pgfpathcurveto{\pgfqpoint{0.031064in}{-0.018041in}}{\pgfqpoint{0.034722in}{-0.009208in}}{\pgfqpoint{0.034722in}{0.000000in}}%
\pgfpathcurveto{\pgfqpoint{0.034722in}{0.009208in}}{\pgfqpoint{0.031064in}{0.018041in}}{\pgfqpoint{0.024552in}{0.024552in}}%
\pgfpathcurveto{\pgfqpoint{0.018041in}{0.031064in}}{\pgfqpoint{0.009208in}{0.034722in}}{\pgfqpoint{0.000000in}{0.034722in}}%
\pgfpathcurveto{\pgfqpoint{-0.009208in}{0.034722in}}{\pgfqpoint{-0.018041in}{0.031064in}}{\pgfqpoint{-0.024552in}{0.024552in}}%
\pgfpathcurveto{\pgfqpoint{-0.031064in}{0.018041in}}{\pgfqpoint{-0.034722in}{0.009208in}}{\pgfqpoint{-0.034722in}{0.000000in}}%
\pgfpathcurveto{\pgfqpoint{-0.034722in}{-0.009208in}}{\pgfqpoint{-0.031064in}{-0.018041in}}{\pgfqpoint{-0.024552in}{-0.024552in}}%
\pgfpathcurveto{\pgfqpoint{-0.018041in}{-0.031064in}}{\pgfqpoint{-0.009208in}{-0.034722in}}{\pgfqpoint{0.000000in}{-0.034722in}}%
\pgfpathlineto{\pgfqpoint{0.000000in}{-0.034722in}}%
\pgfpathclose%
\pgfusepath{stroke,fill}%
}%
\begin{pgfscope}%
\pgfsys@transformshift{0.761265in}{1.444228in}%
\pgfsys@useobject{currentmarker}{}%
\end{pgfscope}%
\begin{pgfscope}%
\pgfsys@transformshift{0.863258in}{1.694869in}%
\pgfsys@useobject{currentmarker}{}%
\end{pgfscope}%
\begin{pgfscope}%
\pgfsys@transformshift{1.033247in}{1.837426in}%
\pgfsys@useobject{currentmarker}{}%
\end{pgfscope}%
\begin{pgfscope}%
\pgfsys@transformshift{1.271231in}{1.947920in}%
\pgfsys@useobject{currentmarker}{}%
\end{pgfscope}%
\begin{pgfscope}%
\pgfsys@transformshift{1.577211in}{2.034456in}%
\pgfsys@useobject{currentmarker}{}%
\end{pgfscope}%
\begin{pgfscope}%
\pgfsys@transformshift{1.951186in}{2.105077in}%
\pgfsys@useobject{currentmarker}{}%
\end{pgfscope}%
\begin{pgfscope}%
\pgfsys@transformshift{2.393157in}{2.171802in}%
\pgfsys@useobject{currentmarker}{}%
\end{pgfscope}%
\begin{pgfscope}%
\pgfsys@transformshift{2.903123in}{2.231779in}%
\pgfsys@useobject{currentmarker}{}%
\end{pgfscope}%
\begin{pgfscope}%
\pgfsys@transformshift{3.481085in}{2.275025in}%
\pgfsys@useobject{currentmarker}{}%
\end{pgfscope}%
\begin{pgfscope}%
\pgfsys@transformshift{4.127042in}{2.317930in}%
\pgfsys@useobject{currentmarker}{}%
\end{pgfscope}%
\end{pgfscope}%
\begin{pgfscope}%
\pgfsetroundcap%
\pgfsetroundjoin%
\pgfsetlinewidth{1.003750pt}%
\definecolor{currentstroke}{rgb}{0.007843,0.619608,0.450980}%
\pgfsetstrokecolor{currentstroke}%
\pgfsetdash{}{0pt}%
\pgfpathmoveto{\pgfqpoint{0.761265in}{1.128142in}}%
\pgfpathlineto{\pgfqpoint{0.863258in}{1.326642in}}%
\pgfpathlineto{\pgfqpoint{1.033247in}{1.405241in}}%
\pgfpathlineto{\pgfqpoint{1.271231in}{1.480098in}}%
\pgfpathlineto{\pgfqpoint{1.577211in}{1.538581in}}%
\pgfpathlineto{\pgfqpoint{1.951186in}{1.558755in}}%
\pgfpathlineto{\pgfqpoint{2.393157in}{1.601164in}}%
\pgfpathlineto{\pgfqpoint{2.903123in}{1.604980in}}%
\pgfpathlineto{\pgfqpoint{3.481085in}{1.628330in}}%
\pgfpathlineto{\pgfqpoint{4.127042in}{1.651244in}}%
\pgfusepath{stroke}%
\end{pgfscope}%
\begin{pgfscope}%
\pgfsetbuttcap%
\pgfsetroundjoin%
\definecolor{currentfill}{rgb}{0.007843,0.619608,0.450980}%
\pgfsetfillcolor{currentfill}%
\pgfsetlinewidth{0.752812pt}%
\definecolor{currentstroke}{rgb}{1.000000,1.000000,1.000000}%
\pgfsetstrokecolor{currentstroke}%
\pgfsetdash{}{0pt}%
\pgfsys@defobject{currentmarker}{\pgfqpoint{-0.034722in}{-0.034722in}}{\pgfqpoint{0.034722in}{0.034722in}}{%
\pgfpathmoveto{\pgfqpoint{0.000000in}{-0.034722in}}%
\pgfpathcurveto{\pgfqpoint{0.009208in}{-0.034722in}}{\pgfqpoint{0.018041in}{-0.031064in}}{\pgfqpoint{0.024552in}{-0.024552in}}%
\pgfpathcurveto{\pgfqpoint{0.031064in}{-0.018041in}}{\pgfqpoint{0.034722in}{-0.009208in}}{\pgfqpoint{0.034722in}{0.000000in}}%
\pgfpathcurveto{\pgfqpoint{0.034722in}{0.009208in}}{\pgfqpoint{0.031064in}{0.018041in}}{\pgfqpoint{0.024552in}{0.024552in}}%
\pgfpathcurveto{\pgfqpoint{0.018041in}{0.031064in}}{\pgfqpoint{0.009208in}{0.034722in}}{\pgfqpoint{0.000000in}{0.034722in}}%
\pgfpathcurveto{\pgfqpoint{-0.009208in}{0.034722in}}{\pgfqpoint{-0.018041in}{0.031064in}}{\pgfqpoint{-0.024552in}{0.024552in}}%
\pgfpathcurveto{\pgfqpoint{-0.031064in}{0.018041in}}{\pgfqpoint{-0.034722in}{0.009208in}}{\pgfqpoint{-0.034722in}{0.000000in}}%
\pgfpathcurveto{\pgfqpoint{-0.034722in}{-0.009208in}}{\pgfqpoint{-0.031064in}{-0.018041in}}{\pgfqpoint{-0.024552in}{-0.024552in}}%
\pgfpathcurveto{\pgfqpoint{-0.018041in}{-0.031064in}}{\pgfqpoint{-0.009208in}{-0.034722in}}{\pgfqpoint{0.000000in}{-0.034722in}}%
\pgfpathlineto{\pgfqpoint{0.000000in}{-0.034722in}}%
\pgfpathclose%
\pgfusepath{stroke,fill}%
}%
\begin{pgfscope}%
\pgfsys@transformshift{0.761265in}{1.128142in}%
\pgfsys@useobject{currentmarker}{}%
\end{pgfscope}%
\begin{pgfscope}%
\pgfsys@transformshift{0.863258in}{1.326642in}%
\pgfsys@useobject{currentmarker}{}%
\end{pgfscope}%
\begin{pgfscope}%
\pgfsys@transformshift{1.033247in}{1.405241in}%
\pgfsys@useobject{currentmarker}{}%
\end{pgfscope}%
\begin{pgfscope}%
\pgfsys@transformshift{1.271231in}{1.480098in}%
\pgfsys@useobject{currentmarker}{}%
\end{pgfscope}%
\begin{pgfscope}%
\pgfsys@transformshift{1.577211in}{1.538581in}%
\pgfsys@useobject{currentmarker}{}%
\end{pgfscope}%
\begin{pgfscope}%
\pgfsys@transformshift{1.951186in}{1.558755in}%
\pgfsys@useobject{currentmarker}{}%
\end{pgfscope}%
\begin{pgfscope}%
\pgfsys@transformshift{2.393157in}{1.601164in}%
\pgfsys@useobject{currentmarker}{}%
\end{pgfscope}%
\begin{pgfscope}%
\pgfsys@transformshift{2.903123in}{1.604980in}%
\pgfsys@useobject{currentmarker}{}%
\end{pgfscope}%
\begin{pgfscope}%
\pgfsys@transformshift{3.481085in}{1.628330in}%
\pgfsys@useobject{currentmarker}{}%
\end{pgfscope}%
\begin{pgfscope}%
\pgfsys@transformshift{4.127042in}{1.651244in}%
\pgfsys@useobject{currentmarker}{}%
\end{pgfscope}%
\end{pgfscope}%
\begin{pgfscope}%
\pgfsetroundcap%
\pgfsetroundjoin%
\pgfsetlinewidth{1.003750pt}%
\definecolor{currentstroke}{rgb}{0.835294,0.368627,0.000000}%
\pgfsetstrokecolor{currentstroke}%
\pgfsetdash{}{0pt}%
\pgfpathmoveto{\pgfqpoint{0.761265in}{1.236872in}}%
\pgfpathlineto{\pgfqpoint{0.863258in}{1.364862in}}%
\pgfpathlineto{\pgfqpoint{1.033247in}{1.444129in}}%
\pgfpathlineto{\pgfqpoint{1.271231in}{1.496647in}}%
\pgfpathlineto{\pgfqpoint{1.577211in}{1.535210in}}%
\pgfpathlineto{\pgfqpoint{1.951186in}{1.581804in}}%
\pgfpathlineto{\pgfqpoint{2.393157in}{1.608464in}}%
\pgfpathlineto{\pgfqpoint{2.903123in}{1.632055in}}%
\pgfpathlineto{\pgfqpoint{3.481085in}{1.653635in}}%
\pgfpathlineto{\pgfqpoint{4.127042in}{1.674145in}}%
\pgfusepath{stroke}%
\end{pgfscope}%
\begin{pgfscope}%
\pgfsetbuttcap%
\pgfsetroundjoin%
\definecolor{currentfill}{rgb}{0.835294,0.368627,0.000000}%
\pgfsetfillcolor{currentfill}%
\pgfsetlinewidth{0.752812pt}%
\definecolor{currentstroke}{rgb}{1.000000,1.000000,1.000000}%
\pgfsetstrokecolor{currentstroke}%
\pgfsetdash{}{0pt}%
\pgfsys@defobject{currentmarker}{\pgfqpoint{-0.034722in}{-0.034722in}}{\pgfqpoint{0.034722in}{0.034722in}}{%
\pgfpathmoveto{\pgfqpoint{0.000000in}{-0.034722in}}%
\pgfpathcurveto{\pgfqpoint{0.009208in}{-0.034722in}}{\pgfqpoint{0.018041in}{-0.031064in}}{\pgfqpoint{0.024552in}{-0.024552in}}%
\pgfpathcurveto{\pgfqpoint{0.031064in}{-0.018041in}}{\pgfqpoint{0.034722in}{-0.009208in}}{\pgfqpoint{0.034722in}{0.000000in}}%
\pgfpathcurveto{\pgfqpoint{0.034722in}{0.009208in}}{\pgfqpoint{0.031064in}{0.018041in}}{\pgfqpoint{0.024552in}{0.024552in}}%
\pgfpathcurveto{\pgfqpoint{0.018041in}{0.031064in}}{\pgfqpoint{0.009208in}{0.034722in}}{\pgfqpoint{0.000000in}{0.034722in}}%
\pgfpathcurveto{\pgfqpoint{-0.009208in}{0.034722in}}{\pgfqpoint{-0.018041in}{0.031064in}}{\pgfqpoint{-0.024552in}{0.024552in}}%
\pgfpathcurveto{\pgfqpoint{-0.031064in}{0.018041in}}{\pgfqpoint{-0.034722in}{0.009208in}}{\pgfqpoint{-0.034722in}{0.000000in}}%
\pgfpathcurveto{\pgfqpoint{-0.034722in}{-0.009208in}}{\pgfqpoint{-0.031064in}{-0.018041in}}{\pgfqpoint{-0.024552in}{-0.024552in}}%
\pgfpathcurveto{\pgfqpoint{-0.018041in}{-0.031064in}}{\pgfqpoint{-0.009208in}{-0.034722in}}{\pgfqpoint{0.000000in}{-0.034722in}}%
\pgfpathlineto{\pgfqpoint{0.000000in}{-0.034722in}}%
\pgfpathclose%
\pgfusepath{stroke,fill}%
}%
\begin{pgfscope}%
\pgfsys@transformshift{0.761265in}{1.236872in}%
\pgfsys@useobject{currentmarker}{}%
\end{pgfscope}%
\begin{pgfscope}%
\pgfsys@transformshift{0.863258in}{1.364862in}%
\pgfsys@useobject{currentmarker}{}%
\end{pgfscope}%
\begin{pgfscope}%
\pgfsys@transformshift{1.033247in}{1.444129in}%
\pgfsys@useobject{currentmarker}{}%
\end{pgfscope}%
\begin{pgfscope}%
\pgfsys@transformshift{1.271231in}{1.496647in}%
\pgfsys@useobject{currentmarker}{}%
\end{pgfscope}%
\begin{pgfscope}%
\pgfsys@transformshift{1.577211in}{1.535210in}%
\pgfsys@useobject{currentmarker}{}%
\end{pgfscope}%
\begin{pgfscope}%
\pgfsys@transformshift{1.951186in}{1.581804in}%
\pgfsys@useobject{currentmarker}{}%
\end{pgfscope}%
\begin{pgfscope}%
\pgfsys@transformshift{2.393157in}{1.608464in}%
\pgfsys@useobject{currentmarker}{}%
\end{pgfscope}%
\begin{pgfscope}%
\pgfsys@transformshift{2.903123in}{1.632055in}%
\pgfsys@useobject{currentmarker}{}%
\end{pgfscope}%
\begin{pgfscope}%
\pgfsys@transformshift{3.481085in}{1.653635in}%
\pgfsys@useobject{currentmarker}{}%
\end{pgfscope}%
\begin{pgfscope}%
\pgfsys@transformshift{4.127042in}{1.674145in}%
\pgfsys@useobject{currentmarker}{}%
\end{pgfscope}%
\end{pgfscope}%
\begin{pgfscope}%
\pgfsetroundcap%
\pgfsetroundjoin%
\pgfsetlinewidth{1.003750pt}%
\definecolor{currentstroke}{rgb}{0.800000,0.470588,0.737255}%
\pgfsetstrokecolor{currentstroke}%
\pgfsetdash{}{0pt}%
\pgfpathmoveto{\pgfqpoint{0.761265in}{0.611875in}}%
\pgfpathlineto{\pgfqpoint{0.863258in}{0.650278in}}%
\pgfpathlineto{\pgfqpoint{1.033247in}{0.889508in}}%
\pgfpathlineto{\pgfqpoint{1.271231in}{0.827000in}}%
\pgfpathlineto{\pgfqpoint{1.577211in}{0.838542in}}%
\pgfpathlineto{\pgfqpoint{1.951186in}{0.859620in}}%
\pgfpathlineto{\pgfqpoint{2.393157in}{0.876283in}}%
\pgfpathlineto{\pgfqpoint{2.903123in}{0.891106in}}%
\pgfpathlineto{\pgfqpoint{3.481085in}{0.921102in}}%
\pgfpathlineto{\pgfqpoint{4.127042in}{0.931025in}}%
\pgfusepath{stroke}%
\end{pgfscope}%
\begin{pgfscope}%
\pgfsetbuttcap%
\pgfsetroundjoin%
\definecolor{currentfill}{rgb}{0.800000,0.470588,0.737255}%
\pgfsetfillcolor{currentfill}%
\pgfsetlinewidth{0.752812pt}%
\definecolor{currentstroke}{rgb}{1.000000,1.000000,1.000000}%
\pgfsetstrokecolor{currentstroke}%
\pgfsetdash{}{0pt}%
\pgfsys@defobject{currentmarker}{\pgfqpoint{-0.034722in}{-0.034722in}}{\pgfqpoint{0.034722in}{0.034722in}}{%
\pgfpathmoveto{\pgfqpoint{0.000000in}{-0.034722in}}%
\pgfpathcurveto{\pgfqpoint{0.009208in}{-0.034722in}}{\pgfqpoint{0.018041in}{-0.031064in}}{\pgfqpoint{0.024552in}{-0.024552in}}%
\pgfpathcurveto{\pgfqpoint{0.031064in}{-0.018041in}}{\pgfqpoint{0.034722in}{-0.009208in}}{\pgfqpoint{0.034722in}{0.000000in}}%
\pgfpathcurveto{\pgfqpoint{0.034722in}{0.009208in}}{\pgfqpoint{0.031064in}{0.018041in}}{\pgfqpoint{0.024552in}{0.024552in}}%
\pgfpathcurveto{\pgfqpoint{0.018041in}{0.031064in}}{\pgfqpoint{0.009208in}{0.034722in}}{\pgfqpoint{0.000000in}{0.034722in}}%
\pgfpathcurveto{\pgfqpoint{-0.009208in}{0.034722in}}{\pgfqpoint{-0.018041in}{0.031064in}}{\pgfqpoint{-0.024552in}{0.024552in}}%
\pgfpathcurveto{\pgfqpoint{-0.031064in}{0.018041in}}{\pgfqpoint{-0.034722in}{0.009208in}}{\pgfqpoint{-0.034722in}{0.000000in}}%
\pgfpathcurveto{\pgfqpoint{-0.034722in}{-0.009208in}}{\pgfqpoint{-0.031064in}{-0.018041in}}{\pgfqpoint{-0.024552in}{-0.024552in}}%
\pgfpathcurveto{\pgfqpoint{-0.018041in}{-0.031064in}}{\pgfqpoint{-0.009208in}{-0.034722in}}{\pgfqpoint{0.000000in}{-0.034722in}}%
\pgfpathlineto{\pgfqpoint{0.000000in}{-0.034722in}}%
\pgfpathclose%
\pgfusepath{stroke,fill}%
}%
\begin{pgfscope}%
\pgfsys@transformshift{0.761265in}{0.611875in}%
\pgfsys@useobject{currentmarker}{}%
\end{pgfscope}%
\begin{pgfscope}%
\pgfsys@transformshift{0.863258in}{0.650278in}%
\pgfsys@useobject{currentmarker}{}%
\end{pgfscope}%
\begin{pgfscope}%
\pgfsys@transformshift{1.033247in}{0.889508in}%
\pgfsys@useobject{currentmarker}{}%
\end{pgfscope}%
\begin{pgfscope}%
\pgfsys@transformshift{1.271231in}{0.827000in}%
\pgfsys@useobject{currentmarker}{}%
\end{pgfscope}%
\begin{pgfscope}%
\pgfsys@transformshift{1.577211in}{0.838542in}%
\pgfsys@useobject{currentmarker}{}%
\end{pgfscope}%
\begin{pgfscope}%
\pgfsys@transformshift{1.951186in}{0.859620in}%
\pgfsys@useobject{currentmarker}{}%
\end{pgfscope}%
\begin{pgfscope}%
\pgfsys@transformshift{2.393157in}{0.876283in}%
\pgfsys@useobject{currentmarker}{}%
\end{pgfscope}%
\begin{pgfscope}%
\pgfsys@transformshift{2.903123in}{0.891106in}%
\pgfsys@useobject{currentmarker}{}%
\end{pgfscope}%
\begin{pgfscope}%
\pgfsys@transformshift{3.481085in}{0.921102in}%
\pgfsys@useobject{currentmarker}{}%
\end{pgfscope}%
\begin{pgfscope}%
\pgfsys@transformshift{4.127042in}{0.931025in}%
\pgfsys@useobject{currentmarker}{}%
\end{pgfscope}%
\end{pgfscope}%
\begin{pgfscope}%
\pgfsetroundcap%
\pgfsetroundjoin%
\pgfsetlinewidth{1.003750pt}%
\definecolor{currentstroke}{rgb}{0.792157,0.568627,0.380392}%
\pgfsetstrokecolor{currentstroke}%
\pgfsetdash{}{0pt}%
\pgfpathmoveto{\pgfqpoint{0.761265in}{0.609733in}}%
\pgfpathlineto{\pgfqpoint{0.863258in}{0.670833in}}%
\pgfpathlineto{\pgfqpoint{1.033247in}{0.730689in}}%
\pgfpathlineto{\pgfqpoint{1.271231in}{0.752862in}}%
\pgfpathlineto{\pgfqpoint{1.577211in}{0.786471in}}%
\pgfpathlineto{\pgfqpoint{1.951186in}{0.818528in}}%
\pgfpathlineto{\pgfqpoint{2.393157in}{0.868719in}}%
\pgfpathlineto{\pgfqpoint{2.903123in}{0.887618in}}%
\pgfpathlineto{\pgfqpoint{3.481085in}{0.904580in}}%
\pgfpathlineto{\pgfqpoint{4.127042in}{0.924471in}}%
\pgfusepath{stroke}%
\end{pgfscope}%
\begin{pgfscope}%
\pgfsetbuttcap%
\pgfsetroundjoin%
\definecolor{currentfill}{rgb}{0.792157,0.568627,0.380392}%
\pgfsetfillcolor{currentfill}%
\pgfsetlinewidth{0.752812pt}%
\definecolor{currentstroke}{rgb}{1.000000,1.000000,1.000000}%
\pgfsetstrokecolor{currentstroke}%
\pgfsetdash{}{0pt}%
\pgfsys@defobject{currentmarker}{\pgfqpoint{-0.034722in}{-0.034722in}}{\pgfqpoint{0.034722in}{0.034722in}}{%
\pgfpathmoveto{\pgfqpoint{0.000000in}{-0.034722in}}%
\pgfpathcurveto{\pgfqpoint{0.009208in}{-0.034722in}}{\pgfqpoint{0.018041in}{-0.031064in}}{\pgfqpoint{0.024552in}{-0.024552in}}%
\pgfpathcurveto{\pgfqpoint{0.031064in}{-0.018041in}}{\pgfqpoint{0.034722in}{-0.009208in}}{\pgfqpoint{0.034722in}{0.000000in}}%
\pgfpathcurveto{\pgfqpoint{0.034722in}{0.009208in}}{\pgfqpoint{0.031064in}{0.018041in}}{\pgfqpoint{0.024552in}{0.024552in}}%
\pgfpathcurveto{\pgfqpoint{0.018041in}{0.031064in}}{\pgfqpoint{0.009208in}{0.034722in}}{\pgfqpoint{0.000000in}{0.034722in}}%
\pgfpathcurveto{\pgfqpoint{-0.009208in}{0.034722in}}{\pgfqpoint{-0.018041in}{0.031064in}}{\pgfqpoint{-0.024552in}{0.024552in}}%
\pgfpathcurveto{\pgfqpoint{-0.031064in}{0.018041in}}{\pgfqpoint{-0.034722in}{0.009208in}}{\pgfqpoint{-0.034722in}{0.000000in}}%
\pgfpathcurveto{\pgfqpoint{-0.034722in}{-0.009208in}}{\pgfqpoint{-0.031064in}{-0.018041in}}{\pgfqpoint{-0.024552in}{-0.024552in}}%
\pgfpathcurveto{\pgfqpoint{-0.018041in}{-0.031064in}}{\pgfqpoint{-0.009208in}{-0.034722in}}{\pgfqpoint{0.000000in}{-0.034722in}}%
\pgfpathlineto{\pgfqpoint{0.000000in}{-0.034722in}}%
\pgfpathclose%
\pgfusepath{stroke,fill}%
}%
\begin{pgfscope}%
\pgfsys@transformshift{0.761265in}{0.609733in}%
\pgfsys@useobject{currentmarker}{}%
\end{pgfscope}%
\begin{pgfscope}%
\pgfsys@transformshift{0.863258in}{0.670833in}%
\pgfsys@useobject{currentmarker}{}%
\end{pgfscope}%
\begin{pgfscope}%
\pgfsys@transformshift{1.033247in}{0.730689in}%
\pgfsys@useobject{currentmarker}{}%
\end{pgfscope}%
\begin{pgfscope}%
\pgfsys@transformshift{1.271231in}{0.752862in}%
\pgfsys@useobject{currentmarker}{}%
\end{pgfscope}%
\begin{pgfscope}%
\pgfsys@transformshift{1.577211in}{0.786471in}%
\pgfsys@useobject{currentmarker}{}%
\end{pgfscope}%
\begin{pgfscope}%
\pgfsys@transformshift{1.951186in}{0.818528in}%
\pgfsys@useobject{currentmarker}{}%
\end{pgfscope}%
\begin{pgfscope}%
\pgfsys@transformshift{2.393157in}{0.868719in}%
\pgfsys@useobject{currentmarker}{}%
\end{pgfscope}%
\begin{pgfscope}%
\pgfsys@transformshift{2.903123in}{0.887618in}%
\pgfsys@useobject{currentmarker}{}%
\end{pgfscope}%
\begin{pgfscope}%
\pgfsys@transformshift{3.481085in}{0.904580in}%
\pgfsys@useobject{currentmarker}{}%
\end{pgfscope}%
\begin{pgfscope}%
\pgfsys@transformshift{4.127042in}{0.924471in}%
\pgfsys@useobject{currentmarker}{}%
\end{pgfscope}%
\end{pgfscope}%
\end{pgfpicture}%
\makeatother%
\endgroup%

						\end{figcenter}
						\caption{Import time of the \enquote{OSM city} dataset by tasks.}
					\end{subfigure}
					\\[3ex]
					\begin{subfigure}[t]{\textwidth}
						\begin{figcenter}
							%% Creator: Matplotlib, PGF backend
%%
%% To include the figure in your LaTeX document, write
%%   \input{<filename>.pgf}
%%
%% Make sure the required packages are loaded in your preamble
%%   \usepackage{pgf}
%%
%% Also ensure that all the required font packages are loaded; for instance,
%% the lmodern package is sometimes necessary when using math font.
%%   \usepackage{lmodern}
%%
%% Figures using additional raster images can only be included by \input if
%% they are in the same directory as the main LaTeX file. For loading figures
%% from other directories you can use the `import` package
%%   \usepackage{import}
%%
%% and then include the figures with
%%   \import{<path to file>}{<filename>.pgf}
%%
%% Matplotlib used the following preamble
%%   
%%   \usepackage{fontspec}
%%   \setmainfont{DejaVuSerif.ttf}[Path=\detokenize{/home/hauke/.local/lib/python3.11/site-packages/matplotlib/mpl-data/fonts/ttf/}]
%%   \setsansfont{DroidSans.ttf}[Path=\detokenize{/usr/share/fonts/droid/}]
%%   \setmonofont{DejaVuSansMono.ttf}[Path=\detokenize{/home/hauke/.local/lib/python3.11/site-packages/matplotlib/mpl-data/fonts/ttf/}]
%%   \makeatletter\@ifpackageloaded{underscore}{}{\usepackage[strings]{underscore}}\makeatother
%%
\begingroup%
\makeatletter%
\begin{pgfpicture}%
\pgfpathrectangle{\pgfpointorigin}{\pgfqpoint{5.694946in}{2.407400in}}%
\pgfusepath{use as bounding box, clip}%
\begin{pgfscope}%
\pgfsetbuttcap%
\pgfsetmiterjoin%
\definecolor{currentfill}{rgb}{1.000000,1.000000,1.000000}%
\pgfsetfillcolor{currentfill}%
\pgfsetlinewidth{0.000000pt}%
\definecolor{currentstroke}{rgb}{1.000000,1.000000,1.000000}%
\pgfsetstrokecolor{currentstroke}%
\pgfsetdash{}{0pt}%
\pgfpathmoveto{\pgfqpoint{0.000000in}{0.000000in}}%
\pgfpathlineto{\pgfqpoint{5.694946in}{0.000000in}}%
\pgfpathlineto{\pgfqpoint{5.694946in}{2.407400in}}%
\pgfpathlineto{\pgfqpoint{0.000000in}{2.407400in}}%
\pgfpathlineto{\pgfqpoint{0.000000in}{0.000000in}}%
\pgfpathclose%
\pgfusepath{fill}%
\end{pgfscope}%
\begin{pgfscope}%
\pgfsetbuttcap%
\pgfsetmiterjoin%
\definecolor{currentfill}{rgb}{1.000000,1.000000,1.000000}%
\pgfsetfillcolor{currentfill}%
\pgfsetlinewidth{0.000000pt}%
\definecolor{currentstroke}{rgb}{0.000000,0.000000,0.000000}%
\pgfsetstrokecolor{currentstroke}%
\pgfsetstrokeopacity{0.000000}%
\pgfsetdash{}{0pt}%
\pgfpathmoveto{\pgfqpoint{0.592976in}{0.451389in}}%
\pgfpathlineto{\pgfqpoint{4.295331in}{0.451389in}}%
\pgfpathlineto{\pgfqpoint{4.295331in}{2.407400in}}%
\pgfpathlineto{\pgfqpoint{0.592976in}{2.407400in}}%
\pgfpathlineto{\pgfqpoint{0.592976in}{0.451389in}}%
\pgfpathclose%
\pgfusepath{fill}%
\end{pgfscope}%
\begin{pgfscope}%
\pgfpathrectangle{\pgfqpoint{0.592976in}{0.451389in}}{\pgfqpoint{3.702355in}{1.956011in}}%
\pgfusepath{clip}%
\pgfsetroundcap%
\pgfsetroundjoin%
\pgfsetlinewidth{1.003750pt}%
\definecolor{currentstroke}{rgb}{0.800000,0.800000,0.800000}%
\pgfsetstrokecolor{currentstroke}%
\pgfsetdash{}{0pt}%
\pgfpathmoveto{\pgfqpoint{0.727267in}{0.451389in}}%
\pgfpathlineto{\pgfqpoint{0.727267in}{2.407400in}}%
\pgfusepath{stroke}%
\end{pgfscope}%
\begin{pgfscope}%
\definecolor{textcolor}{rgb}{0.150000,0.150000,0.150000}%
\pgfsetstrokecolor{textcolor}%
\pgfsetfillcolor{textcolor}%
\pgftext[x=0.727267in,y=0.319444in,,top]{\color{textcolor}\sffamily\fontsize{9.000000}{10.800000}\selectfont 0}%
\end{pgfscope}%
\begin{pgfscope}%
\pgfpathrectangle{\pgfqpoint{0.592976in}{0.451389in}}{\pgfqpoint{3.702355in}{1.956011in}}%
\pgfusepath{clip}%
\pgfsetroundcap%
\pgfsetroundjoin%
\pgfsetlinewidth{1.003750pt}%
\definecolor{currentstroke}{rgb}{0.800000,0.800000,0.800000}%
\pgfsetstrokecolor{currentstroke}%
\pgfsetdash{}{0pt}%
\pgfpathmoveto{\pgfqpoint{1.492982in}{0.451389in}}%
\pgfpathlineto{\pgfqpoint{1.492982in}{2.407400in}}%
\pgfusepath{stroke}%
\end{pgfscope}%
\begin{pgfscope}%
\definecolor{textcolor}{rgb}{0.150000,0.150000,0.150000}%
\pgfsetstrokecolor{textcolor}%
\pgfsetfillcolor{textcolor}%
\pgftext[x=1.492982in,y=0.319444in,,top]{\color{textcolor}\sffamily\fontsize{9.000000}{10.800000}\selectfont 10000}%
\end{pgfscope}%
\begin{pgfscope}%
\pgfpathrectangle{\pgfqpoint{0.592976in}{0.451389in}}{\pgfqpoint{3.702355in}{1.956011in}}%
\pgfusepath{clip}%
\pgfsetroundcap%
\pgfsetroundjoin%
\pgfsetlinewidth{1.003750pt}%
\definecolor{currentstroke}{rgb}{0.800000,0.800000,0.800000}%
\pgfsetstrokecolor{currentstroke}%
\pgfsetdash{}{0pt}%
\pgfpathmoveto{\pgfqpoint{2.258697in}{0.451389in}}%
\pgfpathlineto{\pgfqpoint{2.258697in}{2.407400in}}%
\pgfusepath{stroke}%
\end{pgfscope}%
\begin{pgfscope}%
\definecolor{textcolor}{rgb}{0.150000,0.150000,0.150000}%
\pgfsetstrokecolor{textcolor}%
\pgfsetfillcolor{textcolor}%
\pgftext[x=2.258697in,y=0.319444in,,top]{\color{textcolor}\sffamily\fontsize{9.000000}{10.800000}\selectfont 20000}%
\end{pgfscope}%
\begin{pgfscope}%
\pgfpathrectangle{\pgfqpoint{0.592976in}{0.451389in}}{\pgfqpoint{3.702355in}{1.956011in}}%
\pgfusepath{clip}%
\pgfsetroundcap%
\pgfsetroundjoin%
\pgfsetlinewidth{1.003750pt}%
\definecolor{currentstroke}{rgb}{0.800000,0.800000,0.800000}%
\pgfsetstrokecolor{currentstroke}%
\pgfsetdash{}{0pt}%
\pgfpathmoveto{\pgfqpoint{3.024412in}{0.451389in}}%
\pgfpathlineto{\pgfqpoint{3.024412in}{2.407400in}}%
\pgfusepath{stroke}%
\end{pgfscope}%
\begin{pgfscope}%
\definecolor{textcolor}{rgb}{0.150000,0.150000,0.150000}%
\pgfsetstrokecolor{textcolor}%
\pgfsetfillcolor{textcolor}%
\pgftext[x=3.024412in,y=0.319444in,,top]{\color{textcolor}\sffamily\fontsize{9.000000}{10.800000}\selectfont 30000}%
\end{pgfscope}%
\begin{pgfscope}%
\pgfpathrectangle{\pgfqpoint{0.592976in}{0.451389in}}{\pgfqpoint{3.702355in}{1.956011in}}%
\pgfusepath{clip}%
\pgfsetroundcap%
\pgfsetroundjoin%
\pgfsetlinewidth{1.003750pt}%
\definecolor{currentstroke}{rgb}{0.800000,0.800000,0.800000}%
\pgfsetstrokecolor{currentstroke}%
\pgfsetdash{}{0pt}%
\pgfpathmoveto{\pgfqpoint{3.790128in}{0.451389in}}%
\pgfpathlineto{\pgfqpoint{3.790128in}{2.407400in}}%
\pgfusepath{stroke}%
\end{pgfscope}%
\begin{pgfscope}%
\definecolor{textcolor}{rgb}{0.150000,0.150000,0.150000}%
\pgfsetstrokecolor{textcolor}%
\pgfsetfillcolor{textcolor}%
\pgftext[x=3.790128in,y=0.319444in,,top]{\color{textcolor}\sffamily\fontsize{9.000000}{10.800000}\selectfont 40000}%
\end{pgfscope}%
\begin{pgfscope}%
\definecolor{textcolor}{rgb}{0.150000,0.150000,0.150000}%
\pgfsetstrokecolor{textcolor}%
\pgfsetfillcolor{textcolor}%
\pgftext[x=2.444154in,y=0.125000in,,top]{\color{textcolor}\sffamily\fontsize{9.000000}{10.800000}\selectfont Input obstacle vertices}%
\end{pgfscope}%
\begin{pgfscope}%
\pgfpathrectangle{\pgfqpoint{0.592976in}{0.451389in}}{\pgfqpoint{3.702355in}{1.956011in}}%
\pgfusepath{clip}%
\pgfsetroundcap%
\pgfsetroundjoin%
\pgfsetlinewidth{1.003750pt}%
\definecolor{currentstroke}{rgb}{0.800000,0.800000,0.800000}%
\pgfsetstrokecolor{currentstroke}%
\pgfsetdash{}{0pt}%
\pgfpathmoveto{\pgfqpoint{0.592976in}{0.772958in}}%
\pgfpathlineto{\pgfqpoint{4.295331in}{0.772958in}}%
\pgfusepath{stroke}%
\end{pgfscope}%
\begin{pgfscope}%
\definecolor{textcolor}{rgb}{0.150000,0.150000,0.150000}%
\pgfsetstrokecolor{textcolor}%
\pgfsetfillcolor{textcolor}%
\pgftext[x=0.194444in, y=0.725473in, left, base]{\color{textcolor}\sffamily\fontsize{9.000000}{10.800000}\selectfont \(\displaystyle {10^{-4}}\)}%
\end{pgfscope}%
\begin{pgfscope}%
\pgfpathrectangle{\pgfqpoint{0.592976in}{0.451389in}}{\pgfqpoint{3.702355in}{1.956011in}}%
\pgfusepath{clip}%
\pgfsetroundcap%
\pgfsetroundjoin%
\pgfsetlinewidth{1.003750pt}%
\definecolor{currentstroke}{rgb}{0.800000,0.800000,0.800000}%
\pgfsetstrokecolor{currentstroke}%
\pgfsetdash{}{0pt}%
\pgfpathmoveto{\pgfqpoint{0.592976in}{1.197168in}}%
\pgfpathlineto{\pgfqpoint{4.295331in}{1.197168in}}%
\pgfusepath{stroke}%
\end{pgfscope}%
\begin{pgfscope}%
\definecolor{textcolor}{rgb}{0.150000,0.150000,0.150000}%
\pgfsetstrokecolor{textcolor}%
\pgfsetfillcolor{textcolor}%
\pgftext[x=0.194444in, y=1.149682in, left, base]{\color{textcolor}\sffamily\fontsize{9.000000}{10.800000}\selectfont \(\displaystyle {10^{-2}}\)}%
\end{pgfscope}%
\begin{pgfscope}%
\pgfpathrectangle{\pgfqpoint{0.592976in}{0.451389in}}{\pgfqpoint{3.702355in}{1.956011in}}%
\pgfusepath{clip}%
\pgfsetroundcap%
\pgfsetroundjoin%
\pgfsetlinewidth{1.003750pt}%
\definecolor{currentstroke}{rgb}{0.800000,0.800000,0.800000}%
\pgfsetstrokecolor{currentstroke}%
\pgfsetdash{}{0pt}%
\pgfpathmoveto{\pgfqpoint{0.592976in}{1.621377in}}%
\pgfpathlineto{\pgfqpoint{4.295331in}{1.621377in}}%
\pgfusepath{stroke}%
\end{pgfscope}%
\begin{pgfscope}%
\definecolor{textcolor}{rgb}{0.150000,0.150000,0.150000}%
\pgfsetstrokecolor{textcolor}%
\pgfsetfillcolor{textcolor}%
\pgftext[x=0.274690in, y=1.573892in, left, base]{\color{textcolor}\sffamily\fontsize{9.000000}{10.800000}\selectfont \(\displaystyle {10^{0}}\)}%
\end{pgfscope}%
\begin{pgfscope}%
\pgfpathrectangle{\pgfqpoint{0.592976in}{0.451389in}}{\pgfqpoint{3.702355in}{1.956011in}}%
\pgfusepath{clip}%
\pgfsetroundcap%
\pgfsetroundjoin%
\pgfsetlinewidth{1.003750pt}%
\definecolor{currentstroke}{rgb}{0.800000,0.800000,0.800000}%
\pgfsetstrokecolor{currentstroke}%
\pgfsetdash{}{0pt}%
\pgfpathmoveto{\pgfqpoint{0.592976in}{2.045587in}}%
\pgfpathlineto{\pgfqpoint{4.295331in}{2.045587in}}%
\pgfusepath{stroke}%
\end{pgfscope}%
\begin{pgfscope}%
\definecolor{textcolor}{rgb}{0.150000,0.150000,0.150000}%
\pgfsetstrokecolor{textcolor}%
\pgfsetfillcolor{textcolor}%
\pgftext[x=0.274690in, y=1.998102in, left, base]{\color{textcolor}\sffamily\fontsize{9.000000}{10.800000}\selectfont \(\displaystyle {10^{2}}\)}%
\end{pgfscope}%
\begin{pgfscope}%
\definecolor{textcolor}{rgb}{0.150000,0.150000,0.150000}%
\pgfsetstrokecolor{textcolor}%
\pgfsetfillcolor{textcolor}%
\pgftext[x=0.125000in,y=1.429394in,,bottom,rotate=90.000000]{\color{textcolor}\sffamily\fontsize{9.000000}{10.800000}\selectfont Time in s}%
\end{pgfscope}%
\begin{pgfscope}%
\pgfpathrectangle{\pgfqpoint{0.592976in}{0.451389in}}{\pgfqpoint{3.702355in}{1.956011in}}%
\pgfusepath{clip}%
\pgfsetbuttcap%
\pgfsetroundjoin%
\definecolor{currentfill}{rgb}{0.003922,0.450980,0.698039}%
\pgfsetfillcolor{currentfill}%
\pgfsetfillopacity{0.200000}%
\pgfsetlinewidth{1.003750pt}%
\definecolor{currentstroke}{rgb}{0.003922,0.450980,0.698039}%
\pgfsetstrokecolor{currentstroke}%
\pgfsetstrokeopacity{0.200000}%
\pgfsetdash{}{0pt}%
\pgfsys@defobject{currentmarker}{\pgfqpoint{0.761265in}{1.455970in}}{\pgfqpoint{4.127042in}{2.318490in}}{%
\pgfpathmoveto{\pgfqpoint{0.761265in}{1.459695in}}%
\pgfpathlineto{\pgfqpoint{0.761265in}{1.455970in}}%
\pgfpathlineto{\pgfqpoint{0.863258in}{1.698752in}}%
\pgfpathlineto{\pgfqpoint{1.033247in}{1.839247in}}%
\pgfpathlineto{\pgfqpoint{1.271231in}{1.948660in}}%
\pgfpathlineto{\pgfqpoint{1.577211in}{2.034583in}}%
\pgfpathlineto{\pgfqpoint{1.951186in}{2.105413in}}%
\pgfpathlineto{\pgfqpoint{2.393157in}{2.171804in}}%
\pgfpathlineto{\pgfqpoint{2.903123in}{2.231881in}}%
\pgfpathlineto{\pgfqpoint{3.481085in}{2.275011in}}%
\pgfpathlineto{\pgfqpoint{4.127042in}{2.317590in}}%
\pgfpathlineto{\pgfqpoint{4.127042in}{2.318490in}}%
\pgfpathlineto{\pgfqpoint{4.127042in}{2.318490in}}%
\pgfpathlineto{\pgfqpoint{3.481085in}{2.275480in}}%
\pgfpathlineto{\pgfqpoint{2.903123in}{2.232255in}}%
\pgfpathlineto{\pgfqpoint{2.393157in}{2.172637in}}%
\pgfpathlineto{\pgfqpoint{1.951186in}{2.106090in}}%
\pgfpathlineto{\pgfqpoint{1.577211in}{2.036093in}}%
\pgfpathlineto{\pgfqpoint{1.271231in}{1.949698in}}%
\pgfpathlineto{\pgfqpoint{1.033247in}{1.839946in}}%
\pgfpathlineto{\pgfqpoint{0.863258in}{1.699562in}}%
\pgfpathlineto{\pgfqpoint{0.761265in}{1.459695in}}%
\pgfpathlineto{\pgfqpoint{0.761265in}{1.459695in}}%
\pgfpathclose%
\pgfusepath{stroke,fill}%
}%
\begin{pgfscope}%
\pgfsys@transformshift{0.000000in}{0.000000in}%
\pgfsys@useobject{currentmarker}{}%
\end{pgfscope}%
\end{pgfscope}%
\begin{pgfscope}%
\pgfpathrectangle{\pgfqpoint{0.592976in}{0.451389in}}{\pgfqpoint{3.702355in}{1.956011in}}%
\pgfusepath{clip}%
\pgfsetbuttcap%
\pgfsetroundjoin%
\definecolor{currentfill}{rgb}{0.870588,0.560784,0.019608}%
\pgfsetfillcolor{currentfill}%
\pgfsetfillopacity{0.200000}%
\pgfsetlinewidth{1.003750pt}%
\definecolor{currentstroke}{rgb}{0.870588,0.560784,0.019608}%
\pgfsetstrokecolor{currentstroke}%
\pgfsetstrokeopacity{0.200000}%
\pgfsetdash{}{0pt}%
\pgfsys@defobject{currentmarker}{\pgfqpoint{0.761265in}{1.443516in}}{\pgfqpoint{4.127042in}{2.318333in}}{%
\pgfpathmoveto{\pgfqpoint{0.761265in}{1.445992in}}%
\pgfpathlineto{\pgfqpoint{0.761265in}{1.443516in}}%
\pgfpathlineto{\pgfqpoint{0.863258in}{1.694446in}}%
\pgfpathlineto{\pgfqpoint{1.033247in}{1.837070in}}%
\pgfpathlineto{\pgfqpoint{1.271231in}{1.947364in}}%
\pgfpathlineto{\pgfqpoint{1.577211in}{2.033736in}}%
\pgfpathlineto{\pgfqpoint{1.951186in}{2.104814in}}%
\pgfpathlineto{\pgfqpoint{2.393157in}{2.171401in}}%
\pgfpathlineto{\pgfqpoint{2.903123in}{2.231635in}}%
\pgfpathlineto{\pgfqpoint{3.481085in}{2.274812in}}%
\pgfpathlineto{\pgfqpoint{4.127042in}{2.317434in}}%
\pgfpathlineto{\pgfqpoint{4.127042in}{2.318333in}}%
\pgfpathlineto{\pgfqpoint{4.127042in}{2.318333in}}%
\pgfpathlineto{\pgfqpoint{3.481085in}{2.275284in}}%
\pgfpathlineto{\pgfqpoint{2.903123in}{2.232006in}}%
\pgfpathlineto{\pgfqpoint{2.393157in}{2.172236in}}%
\pgfpathlineto{\pgfqpoint{1.951186in}{2.105492in}}%
\pgfpathlineto{\pgfqpoint{1.577211in}{2.035233in}}%
\pgfpathlineto{\pgfqpoint{1.271231in}{1.948337in}}%
\pgfpathlineto{\pgfqpoint{1.033247in}{1.837620in}}%
\pgfpathlineto{\pgfqpoint{0.863258in}{1.695130in}}%
\pgfpathlineto{\pgfqpoint{0.761265in}{1.445992in}}%
\pgfpathlineto{\pgfqpoint{0.761265in}{1.445992in}}%
\pgfpathclose%
\pgfusepath{stroke,fill}%
}%
\begin{pgfscope}%
\pgfsys@transformshift{0.000000in}{0.000000in}%
\pgfsys@useobject{currentmarker}{}%
\end{pgfscope}%
\end{pgfscope}%
\begin{pgfscope}%
\pgfpathrectangle{\pgfqpoint{0.592976in}{0.451389in}}{\pgfqpoint{3.702355in}{1.956011in}}%
\pgfusepath{clip}%
\pgfsetbuttcap%
\pgfsetroundjoin%
\definecolor{currentfill}{rgb}{0.007843,0.619608,0.450980}%
\pgfsetfillcolor{currentfill}%
\pgfsetfillopacity{0.200000}%
\pgfsetlinewidth{1.003750pt}%
\definecolor{currentstroke}{rgb}{0.007843,0.619608,0.450980}%
\pgfsetstrokecolor{currentstroke}%
\pgfsetstrokeopacity{0.200000}%
\pgfsetdash{}{0pt}%
\pgfsys@defobject{currentmarker}{\pgfqpoint{0.761265in}{1.124594in}}{\pgfqpoint{4.127042in}{1.654244in}}{%
\pgfpathmoveto{\pgfqpoint{0.761265in}{1.133971in}}%
\pgfpathlineto{\pgfqpoint{0.761265in}{1.124594in}}%
\pgfpathlineto{\pgfqpoint{0.863258in}{1.326006in}}%
\pgfpathlineto{\pgfqpoint{1.033247in}{1.403940in}}%
\pgfpathlineto{\pgfqpoint{1.271231in}{1.470302in}}%
\pgfpathlineto{\pgfqpoint{1.577211in}{1.526399in}}%
\pgfpathlineto{\pgfqpoint{1.951186in}{1.549531in}}%
\pgfpathlineto{\pgfqpoint{2.393157in}{1.599545in}}%
\pgfpathlineto{\pgfqpoint{2.903123in}{1.604145in}}%
\pgfpathlineto{\pgfqpoint{3.481085in}{1.627141in}}%
\pgfpathlineto{\pgfqpoint{4.127042in}{1.648914in}}%
\pgfpathlineto{\pgfqpoint{4.127042in}{1.654244in}}%
\pgfpathlineto{\pgfqpoint{4.127042in}{1.654244in}}%
\pgfpathlineto{\pgfqpoint{3.481085in}{1.630110in}}%
\pgfpathlineto{\pgfqpoint{2.903123in}{1.606052in}}%
\pgfpathlineto{\pgfqpoint{2.393157in}{1.602590in}}%
\pgfpathlineto{\pgfqpoint{1.951186in}{1.571006in}}%
\pgfpathlineto{\pgfqpoint{1.577211in}{1.544497in}}%
\pgfpathlineto{\pgfqpoint{1.271231in}{1.493922in}}%
\pgfpathlineto{\pgfqpoint{1.033247in}{1.407747in}}%
\pgfpathlineto{\pgfqpoint{0.863258in}{1.327811in}}%
\pgfpathlineto{\pgfqpoint{0.761265in}{1.133971in}}%
\pgfpathlineto{\pgfqpoint{0.761265in}{1.133971in}}%
\pgfpathclose%
\pgfusepath{stroke,fill}%
}%
\begin{pgfscope}%
\pgfsys@transformshift{0.000000in}{0.000000in}%
\pgfsys@useobject{currentmarker}{}%
\end{pgfscope}%
\end{pgfscope}%
\begin{pgfscope}%
\pgfpathrectangle{\pgfqpoint{0.592976in}{0.451389in}}{\pgfqpoint{3.702355in}{1.956011in}}%
\pgfusepath{clip}%
\pgfsetbuttcap%
\pgfsetroundjoin%
\definecolor{currentfill}{rgb}{0.835294,0.368627,0.000000}%
\pgfsetfillcolor{currentfill}%
\pgfsetfillopacity{0.200000}%
\pgfsetlinewidth{1.003750pt}%
\definecolor{currentstroke}{rgb}{0.835294,0.368627,0.000000}%
\pgfsetstrokecolor{currentstroke}%
\pgfsetstrokeopacity{0.200000}%
\pgfsetdash{}{0pt}%
\pgfsys@defobject{currentmarker}{\pgfqpoint{0.761265in}{1.236018in}}{\pgfqpoint{4.127042in}{1.675506in}}{%
\pgfpathmoveto{\pgfqpoint{0.761265in}{1.237726in}}%
\pgfpathlineto{\pgfqpoint{0.761265in}{1.236018in}}%
\pgfpathlineto{\pgfqpoint{0.863258in}{1.364291in}}%
\pgfpathlineto{\pgfqpoint{1.033247in}{1.439446in}}%
\pgfpathlineto{\pgfqpoint{1.271231in}{1.493302in}}%
\pgfpathlineto{\pgfqpoint{1.577211in}{1.534060in}}%
\pgfpathlineto{\pgfqpoint{1.951186in}{1.581022in}}%
\pgfpathlineto{\pgfqpoint{2.393157in}{1.608007in}}%
\pgfpathlineto{\pgfqpoint{2.903123in}{1.631255in}}%
\pgfpathlineto{\pgfqpoint{3.481085in}{1.652086in}}%
\pgfpathlineto{\pgfqpoint{4.127042in}{1.673084in}}%
\pgfpathlineto{\pgfqpoint{4.127042in}{1.675506in}}%
\pgfpathlineto{\pgfqpoint{4.127042in}{1.675506in}}%
\pgfpathlineto{\pgfqpoint{3.481085in}{1.654816in}}%
\pgfpathlineto{\pgfqpoint{2.903123in}{1.632744in}}%
\pgfpathlineto{\pgfqpoint{2.393157in}{1.609090in}}%
\pgfpathlineto{\pgfqpoint{1.951186in}{1.582701in}}%
\pgfpathlineto{\pgfqpoint{1.577211in}{1.536338in}}%
\pgfpathlineto{\pgfqpoint{1.271231in}{1.499967in}}%
\pgfpathlineto{\pgfqpoint{1.033247in}{1.451798in}}%
\pgfpathlineto{\pgfqpoint{0.863258in}{1.365406in}}%
\pgfpathlineto{\pgfqpoint{0.761265in}{1.237726in}}%
\pgfpathlineto{\pgfqpoint{0.761265in}{1.237726in}}%
\pgfpathclose%
\pgfusepath{stroke,fill}%
}%
\begin{pgfscope}%
\pgfsys@transformshift{0.000000in}{0.000000in}%
\pgfsys@useobject{currentmarker}{}%
\end{pgfscope}%
\end{pgfscope}%
\begin{pgfscope}%
\pgfpathrectangle{\pgfqpoint{0.592976in}{0.451389in}}{\pgfqpoint{3.702355in}{1.956011in}}%
\pgfusepath{clip}%
\pgfsetbuttcap%
\pgfsetroundjoin%
\definecolor{currentfill}{rgb}{0.800000,0.470588,0.737255}%
\pgfsetfillcolor{currentfill}%
\pgfsetfillopacity{0.200000}%
\pgfsetlinewidth{1.003750pt}%
\definecolor{currentstroke}{rgb}{0.800000,0.470588,0.737255}%
\pgfsetstrokecolor{currentstroke}%
\pgfsetstrokeopacity{0.200000}%
\pgfsetdash{}{0pt}%
\pgfsys@defobject{currentmarker}{\pgfqpoint{0.761265in}{0.576100in}}{\pgfqpoint{4.127042in}{0.985925in}}{%
\pgfpathmoveto{\pgfqpoint{0.761265in}{0.639950in}}%
\pgfpathlineto{\pgfqpoint{0.761265in}{0.576100in}}%
\pgfpathlineto{\pgfqpoint{0.863258in}{0.622842in}}%
\pgfpathlineto{\pgfqpoint{1.033247in}{0.668572in}}%
\pgfpathlineto{\pgfqpoint{1.271231in}{0.700218in}}%
\pgfpathlineto{\pgfqpoint{1.577211in}{0.730397in}}%
\pgfpathlineto{\pgfqpoint{1.951186in}{0.766075in}}%
\pgfpathlineto{\pgfqpoint{2.393157in}{0.828525in}}%
\pgfpathlineto{\pgfqpoint{2.903123in}{0.846835in}}%
\pgfpathlineto{\pgfqpoint{3.481085in}{0.890130in}}%
\pgfpathlineto{\pgfqpoint{4.127042in}{0.908215in}}%
\pgfpathlineto{\pgfqpoint{4.127042in}{0.964208in}}%
\pgfpathlineto{\pgfqpoint{4.127042in}{0.964208in}}%
\pgfpathlineto{\pgfqpoint{3.481085in}{0.953150in}}%
\pgfpathlineto{\pgfqpoint{2.903123in}{0.928643in}}%
\pgfpathlineto{\pgfqpoint{2.393157in}{0.923613in}}%
\pgfpathlineto{\pgfqpoint{1.951186in}{0.916952in}}%
\pgfpathlineto{\pgfqpoint{1.577211in}{0.904845in}}%
\pgfpathlineto{\pgfqpoint{1.271231in}{0.891463in}}%
\pgfpathlineto{\pgfqpoint{1.033247in}{0.985925in}}%
\pgfpathlineto{\pgfqpoint{0.863258in}{0.664477in}}%
\pgfpathlineto{\pgfqpoint{0.761265in}{0.639950in}}%
\pgfpathlineto{\pgfqpoint{0.761265in}{0.639950in}}%
\pgfpathclose%
\pgfusepath{stroke,fill}%
}%
\begin{pgfscope}%
\pgfsys@transformshift{0.000000in}{0.000000in}%
\pgfsys@useobject{currentmarker}{}%
\end{pgfscope}%
\end{pgfscope}%
\begin{pgfscope}%
\pgfpathrectangle{\pgfqpoint{0.592976in}{0.451389in}}{\pgfqpoint{3.702355in}{1.956011in}}%
\pgfusepath{clip}%
\pgfsetbuttcap%
\pgfsetroundjoin%
\definecolor{currentfill}{rgb}{0.792157,0.568627,0.380392}%
\pgfsetfillcolor{currentfill}%
\pgfsetfillopacity{0.200000}%
\pgfsetlinewidth{1.003750pt}%
\definecolor{currentstroke}{rgb}{0.792157,0.568627,0.380392}%
\pgfsetstrokecolor{currentstroke}%
\pgfsetstrokeopacity{0.200000}%
\pgfsetdash{}{0pt}%
\pgfsys@defobject{currentmarker}{\pgfqpoint{0.761265in}{0.540298in}}{\pgfqpoint{4.127042in}{0.930294in}}{%
\pgfpathmoveto{\pgfqpoint{0.761265in}{0.662053in}}%
\pgfpathlineto{\pgfqpoint{0.761265in}{0.540298in}}%
\pgfpathlineto{\pgfqpoint{0.863258in}{0.610810in}}%
\pgfpathlineto{\pgfqpoint{1.033247in}{0.706112in}}%
\pgfpathlineto{\pgfqpoint{1.271231in}{0.735792in}}%
\pgfpathlineto{\pgfqpoint{1.577211in}{0.775143in}}%
\pgfpathlineto{\pgfqpoint{1.951186in}{0.800603in}}%
\pgfpathlineto{\pgfqpoint{2.393157in}{0.862800in}}%
\pgfpathlineto{\pgfqpoint{2.903123in}{0.883383in}}%
\pgfpathlineto{\pgfqpoint{3.481085in}{0.900196in}}%
\pgfpathlineto{\pgfqpoint{4.127042in}{0.918975in}}%
\pgfpathlineto{\pgfqpoint{4.127042in}{0.930294in}}%
\pgfpathlineto{\pgfqpoint{4.127042in}{0.930294in}}%
\pgfpathlineto{\pgfqpoint{3.481085in}{0.909730in}}%
\pgfpathlineto{\pgfqpoint{2.903123in}{0.896127in}}%
\pgfpathlineto{\pgfqpoint{2.393157in}{0.878359in}}%
\pgfpathlineto{\pgfqpoint{1.951186in}{0.838270in}}%
\pgfpathlineto{\pgfqpoint{1.577211in}{0.800875in}}%
\pgfpathlineto{\pgfqpoint{1.271231in}{0.774782in}}%
\pgfpathlineto{\pgfqpoint{1.033247in}{0.752633in}}%
\pgfpathlineto{\pgfqpoint{0.863258in}{0.713427in}}%
\pgfpathlineto{\pgfqpoint{0.761265in}{0.662053in}}%
\pgfpathlineto{\pgfqpoint{0.761265in}{0.662053in}}%
\pgfpathclose%
\pgfusepath{stroke,fill}%
}%
\begin{pgfscope}%
\pgfsys@transformshift{0.000000in}{0.000000in}%
\pgfsys@useobject{currentmarker}{}%
\end{pgfscope}%
\end{pgfscope}%
\begin{pgfscope}%
\pgfsetrectcap%
\pgfsetmiterjoin%
\pgfsetlinewidth{1.254687pt}%
\definecolor{currentstroke}{rgb}{0.800000,0.800000,0.800000}%
\pgfsetstrokecolor{currentstroke}%
\pgfsetdash{}{0pt}%
\pgfpathmoveto{\pgfqpoint{0.592976in}{0.451389in}}%
\pgfpathlineto{\pgfqpoint{0.592976in}{2.407400in}}%
\pgfusepath{stroke}%
\end{pgfscope}%
\begin{pgfscope}%
\pgfsetrectcap%
\pgfsetmiterjoin%
\pgfsetlinewidth{1.254687pt}%
\definecolor{currentstroke}{rgb}{0.800000,0.800000,0.800000}%
\pgfsetstrokecolor{currentstroke}%
\pgfsetdash{}{0pt}%
\pgfpathmoveto{\pgfqpoint{4.295331in}{0.451389in}}%
\pgfpathlineto{\pgfqpoint{4.295331in}{2.407400in}}%
\pgfusepath{stroke}%
\end{pgfscope}%
\begin{pgfscope}%
\pgfsetrectcap%
\pgfsetmiterjoin%
\pgfsetlinewidth{1.254687pt}%
\definecolor{currentstroke}{rgb}{0.800000,0.800000,0.800000}%
\pgfsetstrokecolor{currentstroke}%
\pgfsetdash{}{0pt}%
\pgfpathmoveto{\pgfqpoint{0.592976in}{0.451389in}}%
\pgfpathlineto{\pgfqpoint{4.295331in}{0.451389in}}%
\pgfusepath{stroke}%
\end{pgfscope}%
\begin{pgfscope}%
\pgfsetrectcap%
\pgfsetmiterjoin%
\pgfsetlinewidth{1.254687pt}%
\definecolor{currentstroke}{rgb}{0.800000,0.800000,0.800000}%
\pgfsetstrokecolor{currentstroke}%
\pgfsetdash{}{0pt}%
\pgfpathmoveto{\pgfqpoint{0.592976in}{2.407400in}}%
\pgfpathlineto{\pgfqpoint{4.295331in}{2.407400in}}%
\pgfusepath{stroke}%
\end{pgfscope}%
\begin{pgfscope}%
\pgfsetbuttcap%
\pgfsetmiterjoin%
\definecolor{currentfill}{rgb}{1.000000,1.000000,1.000000}%
\pgfsetfillcolor{currentfill}%
\pgfsetfillopacity{0.800000}%
\pgfsetlinewidth{1.003750pt}%
\definecolor{currentstroke}{rgb}{0.800000,0.800000,0.800000}%
\pgfsetstrokecolor{currentstroke}%
\pgfsetstrokeopacity{0.800000}%
\pgfsetdash{}{0pt}%
\pgfpathmoveto{\pgfqpoint{4.475390in}{0.538404in}}%
\pgfpathlineto{\pgfqpoint{5.669946in}{0.538404in}}%
\pgfpathquadraticcurveto{\pgfqpoint{5.694946in}{0.538404in}}{\pgfqpoint{5.694946in}{0.563404in}}%
\pgfpathlineto{\pgfqpoint{5.694946in}{2.295385in}}%
\pgfpathquadraticcurveto{\pgfqpoint{5.694946in}{2.320385in}}{\pgfqpoint{5.669946in}{2.320385in}}%
\pgfpathlineto{\pgfqpoint{4.475390in}{2.320385in}}%
\pgfpathquadraticcurveto{\pgfqpoint{4.450390in}{2.320385in}}{\pgfqpoint{4.450390in}{2.295385in}}%
\pgfpathlineto{\pgfqpoint{4.450390in}{0.563404in}}%
\pgfpathquadraticcurveto{\pgfqpoint{4.450390in}{0.538404in}}{\pgfqpoint{4.475390in}{0.538404in}}%
\pgfpathlineto{\pgfqpoint{4.475390in}{0.538404in}}%
\pgfpathclose%
\pgfusepath{stroke,fill}%
\end{pgfscope}%
\begin{pgfscope}%
\definecolor{textcolor}{rgb}{0.150000,0.150000,0.150000}%
\pgfsetstrokecolor{textcolor}%
\pgfsetfillcolor{textcolor}%
\pgftext[x=4.869268in,y=2.175414in,left,base]{\color{textcolor}\sffamily\fontsize{9.000000}{10.800000}\selectfont Legend}%
\end{pgfscope}%
\begin{pgfscope}%
\pgfsetroundcap%
\pgfsetroundjoin%
\pgfsetlinewidth{1.505625pt}%
\definecolor{currentstroke}{rgb}{0.003922,0.450980,0.698039}%
\pgfsetstrokecolor{currentstroke}%
\pgfsetdash{}{0pt}%
\pgfpathmoveto{\pgfqpoint{4.500390in}{2.031664in}}%
\pgfpathlineto{\pgfqpoint{4.625390in}{2.031664in}}%
\pgfpathlineto{\pgfqpoint{4.750390in}{2.031664in}}%
\pgfusepath{stroke}%
\end{pgfscope}%
\begin{pgfscope}%
\definecolor{textcolor}{rgb}{0.150000,0.150000,0.150000}%
\pgfsetstrokecolor{textcolor}%
\pgfsetfillcolor{textcolor}%
\pgftext[x=4.850390in,y=1.987914in,left,base]{\color{textcolor}\sffamily\fontsize{9.000000}{10.800000}\selectfont Total time}%
\end{pgfscope}%
\begin{pgfscope}%
\pgfsetroundcap%
\pgfsetroundjoin%
\pgfsetlinewidth{1.505625pt}%
\definecolor{currentstroke}{rgb}{0.870588,0.560784,0.019608}%
\pgfsetstrokecolor{currentstroke}%
\pgfsetdash{}{0pt}%
\pgfpathmoveto{\pgfqpoint{4.500390in}{1.844164in}}%
\pgfpathlineto{\pgfqpoint{4.625390in}{1.844164in}}%
\pgfpathlineto{\pgfqpoint{4.750390in}{1.844164in}}%
\pgfusepath{stroke}%
\end{pgfscope}%
\begin{pgfscope}%
\definecolor{textcolor}{rgb}{0.150000,0.150000,0.150000}%
\pgfsetstrokecolor{textcolor}%
\pgfsetfillcolor{textcolor}%
\pgftext[x=4.850390in,y=1.800414in,left,base]{\color{textcolor}\sffamily\fontsize{9.000000}{10.800000}\selectfont kNN search}%
\end{pgfscope}%
\begin{pgfscope}%
\pgfsetroundcap%
\pgfsetroundjoin%
\pgfsetlinewidth{1.505625pt}%
\definecolor{currentstroke}{rgb}{0.007843,0.619608,0.450980}%
\pgfsetstrokecolor{currentstroke}%
\pgfsetdash{}{0pt}%
\pgfpathmoveto{\pgfqpoint{4.500390in}{1.656664in}}%
\pgfpathlineto{\pgfqpoint{4.625390in}{1.656664in}}%
\pgfpathlineto{\pgfqpoint{4.750390in}{1.656664in}}%
\pgfusepath{stroke}%
\end{pgfscope}%
\begin{pgfscope}%
\definecolor{textcolor}{rgb}{0.150000,0.150000,0.150000}%
\pgfsetstrokecolor{textcolor}%
\pgfsetfillcolor{textcolor}%
\pgftext[x=4.850390in,y=1.612914in,left,base]{\color{textcolor}\sffamily\fontsize{9.000000}{10.800000}\selectfont Create graph}%
\end{pgfscope}%
\begin{pgfscope}%
\pgfsetroundcap%
\pgfsetroundjoin%
\pgfsetlinewidth{1.505625pt}%
\definecolor{currentstroke}{rgb}{0.835294,0.368627,0.000000}%
\pgfsetstrokecolor{currentstroke}%
\pgfsetdash{}{0pt}%
\pgfpathmoveto{\pgfqpoint{4.500390in}{1.382153in}}%
\pgfpathlineto{\pgfqpoint{4.625390in}{1.382153in}}%
\pgfpathlineto{\pgfqpoint{4.750390in}{1.382153in}}%
\pgfusepath{stroke}%
\end{pgfscope}%
\begin{pgfscope}%
\definecolor{textcolor}{rgb}{0.150000,0.150000,0.150000}%
\pgfsetstrokecolor{textcolor}%
\pgfsetfillcolor{textcolor}%
\pgftext[x=4.850390in, y=1.425415in, left, base]{\color{textcolor}\sffamily\fontsize{9.000000}{10.800000}\selectfont Get \& prepare}%
\end{pgfscope}%
\begin{pgfscope}%
\definecolor{textcolor}{rgb}{0.150000,0.150000,0.150000}%
\pgfsetstrokecolor{textcolor}%
\pgfsetfillcolor{textcolor}%
\pgftext[x=4.850390in, y=1.281421in, left, base]{\color{textcolor}\sffamily\fontsize{9.000000}{10.800000}\selectfont obstacles}%
\end{pgfscope}%
\begin{pgfscope}%
\pgfsetroundcap%
\pgfsetroundjoin%
\pgfsetlinewidth{1.505625pt}%
\definecolor{currentstroke}{rgb}{0.800000,0.470588,0.737255}%
\pgfsetstrokecolor{currentstroke}%
\pgfsetdash{}{0pt}%
\pgfpathmoveto{\pgfqpoint{4.500390in}{1.050659in}}%
\pgfpathlineto{\pgfqpoint{4.625390in}{1.050659in}}%
\pgfpathlineto{\pgfqpoint{4.750390in}{1.050659in}}%
\pgfusepath{stroke}%
\end{pgfscope}%
\begin{pgfscope}%
\definecolor{textcolor}{rgb}{0.150000,0.150000,0.150000}%
\pgfsetstrokecolor{textcolor}%
\pgfsetfillcolor{textcolor}%
\pgftext[x=4.850390in, y=1.093921in, left, base]{\color{textcolor}\sffamily\fontsize{9.000000}{10.800000}\selectfont Merge road}%
\end{pgfscope}%
\begin{pgfscope}%
\definecolor{textcolor}{rgb}{0.150000,0.150000,0.150000}%
\pgfsetstrokecolor{textcolor}%
\pgfsetfillcolor{textcolor}%
\pgftext[x=4.850390in, y=0.949927in, left, base]{\color{textcolor}\sffamily\fontsize{9.000000}{10.800000}\selectfont edges}%
\end{pgfscope}%
\begin{pgfscope}%
\pgfsetroundcap%
\pgfsetroundjoin%
\pgfsetlinewidth{1.505625pt}%
\definecolor{currentstroke}{rgb}{0.792157,0.568627,0.380392}%
\pgfsetstrokecolor{currentstroke}%
\pgfsetdash{}{0pt}%
\pgfpathmoveto{\pgfqpoint{4.500390in}{0.719165in}}%
\pgfpathlineto{\pgfqpoint{4.625390in}{0.719165in}}%
\pgfpathlineto{\pgfqpoint{4.750390in}{0.719165in}}%
\pgfusepath{stroke}%
\end{pgfscope}%
\begin{pgfscope}%
\definecolor{textcolor}{rgb}{0.150000,0.150000,0.150000}%
\pgfsetstrokecolor{textcolor}%
\pgfsetfillcolor{textcolor}%
\pgftext[x=4.850390in, y=0.762427in, left, base]{\color{textcolor}\sffamily\fontsize{9.000000}{10.800000}\selectfont Add POI}%
\end{pgfscope}%
\begin{pgfscope}%
\definecolor{textcolor}{rgb}{0.150000,0.150000,0.150000}%
\pgfsetstrokecolor{textcolor}%
\pgfsetfillcolor{textcolor}%
\pgftext[x=4.850390in, y=0.618433in, left, base]{\color{textcolor}\sffamily\fontsize{9.000000}{10.800000}\selectfont attributes}%
\end{pgfscope}%
\begin{pgfscope}%
\pgfsetroundcap%
\pgfsetroundjoin%
\pgfsetlinewidth{1.003750pt}%
\definecolor{currentstroke}{rgb}{0.003922,0.450980,0.698039}%
\pgfsetstrokecolor{currentstroke}%
\pgfsetdash{}{0pt}%
\pgfpathmoveto{\pgfqpoint{0.761265in}{1.457007in}}%
\pgfpathlineto{\pgfqpoint{0.863258in}{1.699189in}}%
\pgfpathlineto{\pgfqpoint{1.033247in}{1.839654in}}%
\pgfpathlineto{\pgfqpoint{1.271231in}{1.949225in}}%
\pgfpathlineto{\pgfqpoint{1.577211in}{2.035316in}}%
\pgfpathlineto{\pgfqpoint{1.951186in}{2.105655in}}%
\pgfpathlineto{\pgfqpoint{2.393157in}{2.172206in}}%
\pgfpathlineto{\pgfqpoint{2.903123in}{2.232027in}}%
\pgfpathlineto{\pgfqpoint{3.481085in}{2.275222in}}%
\pgfpathlineto{\pgfqpoint{4.127042in}{2.318086in}}%
\pgfusepath{stroke}%
\end{pgfscope}%
\begin{pgfscope}%
\pgfsetbuttcap%
\pgfsetroundjoin%
\definecolor{currentfill}{rgb}{0.003922,0.450980,0.698039}%
\pgfsetfillcolor{currentfill}%
\pgfsetlinewidth{0.752812pt}%
\definecolor{currentstroke}{rgb}{1.000000,1.000000,1.000000}%
\pgfsetstrokecolor{currentstroke}%
\pgfsetdash{}{0pt}%
\pgfsys@defobject{currentmarker}{\pgfqpoint{-0.034722in}{-0.034722in}}{\pgfqpoint{0.034722in}{0.034722in}}{%
\pgfpathmoveto{\pgfqpoint{0.000000in}{-0.034722in}}%
\pgfpathcurveto{\pgfqpoint{0.009208in}{-0.034722in}}{\pgfqpoint{0.018041in}{-0.031064in}}{\pgfqpoint{0.024552in}{-0.024552in}}%
\pgfpathcurveto{\pgfqpoint{0.031064in}{-0.018041in}}{\pgfqpoint{0.034722in}{-0.009208in}}{\pgfqpoint{0.034722in}{0.000000in}}%
\pgfpathcurveto{\pgfqpoint{0.034722in}{0.009208in}}{\pgfqpoint{0.031064in}{0.018041in}}{\pgfqpoint{0.024552in}{0.024552in}}%
\pgfpathcurveto{\pgfqpoint{0.018041in}{0.031064in}}{\pgfqpoint{0.009208in}{0.034722in}}{\pgfqpoint{0.000000in}{0.034722in}}%
\pgfpathcurveto{\pgfqpoint{-0.009208in}{0.034722in}}{\pgfqpoint{-0.018041in}{0.031064in}}{\pgfqpoint{-0.024552in}{0.024552in}}%
\pgfpathcurveto{\pgfqpoint{-0.031064in}{0.018041in}}{\pgfqpoint{-0.034722in}{0.009208in}}{\pgfqpoint{-0.034722in}{0.000000in}}%
\pgfpathcurveto{\pgfqpoint{-0.034722in}{-0.009208in}}{\pgfqpoint{-0.031064in}{-0.018041in}}{\pgfqpoint{-0.024552in}{-0.024552in}}%
\pgfpathcurveto{\pgfqpoint{-0.018041in}{-0.031064in}}{\pgfqpoint{-0.009208in}{-0.034722in}}{\pgfqpoint{0.000000in}{-0.034722in}}%
\pgfpathlineto{\pgfqpoint{0.000000in}{-0.034722in}}%
\pgfpathclose%
\pgfusepath{stroke,fill}%
}%
\begin{pgfscope}%
\pgfsys@transformshift{0.761265in}{1.457007in}%
\pgfsys@useobject{currentmarker}{}%
\end{pgfscope}%
\begin{pgfscope}%
\pgfsys@transformshift{0.863258in}{1.699189in}%
\pgfsys@useobject{currentmarker}{}%
\end{pgfscope}%
\begin{pgfscope}%
\pgfsys@transformshift{1.033247in}{1.839654in}%
\pgfsys@useobject{currentmarker}{}%
\end{pgfscope}%
\begin{pgfscope}%
\pgfsys@transformshift{1.271231in}{1.949225in}%
\pgfsys@useobject{currentmarker}{}%
\end{pgfscope}%
\begin{pgfscope}%
\pgfsys@transformshift{1.577211in}{2.035316in}%
\pgfsys@useobject{currentmarker}{}%
\end{pgfscope}%
\begin{pgfscope}%
\pgfsys@transformshift{1.951186in}{2.105655in}%
\pgfsys@useobject{currentmarker}{}%
\end{pgfscope}%
\begin{pgfscope}%
\pgfsys@transformshift{2.393157in}{2.172206in}%
\pgfsys@useobject{currentmarker}{}%
\end{pgfscope}%
\begin{pgfscope}%
\pgfsys@transformshift{2.903123in}{2.232027in}%
\pgfsys@useobject{currentmarker}{}%
\end{pgfscope}%
\begin{pgfscope}%
\pgfsys@transformshift{3.481085in}{2.275222in}%
\pgfsys@useobject{currentmarker}{}%
\end{pgfscope}%
\begin{pgfscope}%
\pgfsys@transformshift{4.127042in}{2.318086in}%
\pgfsys@useobject{currentmarker}{}%
\end{pgfscope}%
\end{pgfscope}%
\begin{pgfscope}%
\pgfsetroundcap%
\pgfsetroundjoin%
\pgfsetlinewidth{1.003750pt}%
\definecolor{currentstroke}{rgb}{0.870588,0.560784,0.019608}%
\pgfsetstrokecolor{currentstroke}%
\pgfsetdash{}{0pt}%
\pgfpathmoveto{\pgfqpoint{0.761265in}{1.444228in}}%
\pgfpathlineto{\pgfqpoint{0.863258in}{1.694869in}}%
\pgfpathlineto{\pgfqpoint{1.033247in}{1.837426in}}%
\pgfpathlineto{\pgfqpoint{1.271231in}{1.947920in}}%
\pgfpathlineto{\pgfqpoint{1.577211in}{2.034456in}}%
\pgfpathlineto{\pgfqpoint{1.951186in}{2.105077in}}%
\pgfpathlineto{\pgfqpoint{2.393157in}{2.171802in}}%
\pgfpathlineto{\pgfqpoint{2.903123in}{2.231779in}}%
\pgfpathlineto{\pgfqpoint{3.481085in}{2.275025in}}%
\pgfpathlineto{\pgfqpoint{4.127042in}{2.317930in}}%
\pgfusepath{stroke}%
\end{pgfscope}%
\begin{pgfscope}%
\pgfsetbuttcap%
\pgfsetroundjoin%
\definecolor{currentfill}{rgb}{0.870588,0.560784,0.019608}%
\pgfsetfillcolor{currentfill}%
\pgfsetlinewidth{0.752812pt}%
\definecolor{currentstroke}{rgb}{1.000000,1.000000,1.000000}%
\pgfsetstrokecolor{currentstroke}%
\pgfsetdash{}{0pt}%
\pgfsys@defobject{currentmarker}{\pgfqpoint{-0.034722in}{-0.034722in}}{\pgfqpoint{0.034722in}{0.034722in}}{%
\pgfpathmoveto{\pgfqpoint{0.000000in}{-0.034722in}}%
\pgfpathcurveto{\pgfqpoint{0.009208in}{-0.034722in}}{\pgfqpoint{0.018041in}{-0.031064in}}{\pgfqpoint{0.024552in}{-0.024552in}}%
\pgfpathcurveto{\pgfqpoint{0.031064in}{-0.018041in}}{\pgfqpoint{0.034722in}{-0.009208in}}{\pgfqpoint{0.034722in}{0.000000in}}%
\pgfpathcurveto{\pgfqpoint{0.034722in}{0.009208in}}{\pgfqpoint{0.031064in}{0.018041in}}{\pgfqpoint{0.024552in}{0.024552in}}%
\pgfpathcurveto{\pgfqpoint{0.018041in}{0.031064in}}{\pgfqpoint{0.009208in}{0.034722in}}{\pgfqpoint{0.000000in}{0.034722in}}%
\pgfpathcurveto{\pgfqpoint{-0.009208in}{0.034722in}}{\pgfqpoint{-0.018041in}{0.031064in}}{\pgfqpoint{-0.024552in}{0.024552in}}%
\pgfpathcurveto{\pgfqpoint{-0.031064in}{0.018041in}}{\pgfqpoint{-0.034722in}{0.009208in}}{\pgfqpoint{-0.034722in}{0.000000in}}%
\pgfpathcurveto{\pgfqpoint{-0.034722in}{-0.009208in}}{\pgfqpoint{-0.031064in}{-0.018041in}}{\pgfqpoint{-0.024552in}{-0.024552in}}%
\pgfpathcurveto{\pgfqpoint{-0.018041in}{-0.031064in}}{\pgfqpoint{-0.009208in}{-0.034722in}}{\pgfqpoint{0.000000in}{-0.034722in}}%
\pgfpathlineto{\pgfqpoint{0.000000in}{-0.034722in}}%
\pgfpathclose%
\pgfusepath{stroke,fill}%
}%
\begin{pgfscope}%
\pgfsys@transformshift{0.761265in}{1.444228in}%
\pgfsys@useobject{currentmarker}{}%
\end{pgfscope}%
\begin{pgfscope}%
\pgfsys@transformshift{0.863258in}{1.694869in}%
\pgfsys@useobject{currentmarker}{}%
\end{pgfscope}%
\begin{pgfscope}%
\pgfsys@transformshift{1.033247in}{1.837426in}%
\pgfsys@useobject{currentmarker}{}%
\end{pgfscope}%
\begin{pgfscope}%
\pgfsys@transformshift{1.271231in}{1.947920in}%
\pgfsys@useobject{currentmarker}{}%
\end{pgfscope}%
\begin{pgfscope}%
\pgfsys@transformshift{1.577211in}{2.034456in}%
\pgfsys@useobject{currentmarker}{}%
\end{pgfscope}%
\begin{pgfscope}%
\pgfsys@transformshift{1.951186in}{2.105077in}%
\pgfsys@useobject{currentmarker}{}%
\end{pgfscope}%
\begin{pgfscope}%
\pgfsys@transformshift{2.393157in}{2.171802in}%
\pgfsys@useobject{currentmarker}{}%
\end{pgfscope}%
\begin{pgfscope}%
\pgfsys@transformshift{2.903123in}{2.231779in}%
\pgfsys@useobject{currentmarker}{}%
\end{pgfscope}%
\begin{pgfscope}%
\pgfsys@transformshift{3.481085in}{2.275025in}%
\pgfsys@useobject{currentmarker}{}%
\end{pgfscope}%
\begin{pgfscope}%
\pgfsys@transformshift{4.127042in}{2.317930in}%
\pgfsys@useobject{currentmarker}{}%
\end{pgfscope}%
\end{pgfscope}%
\begin{pgfscope}%
\pgfsetroundcap%
\pgfsetroundjoin%
\pgfsetlinewidth{1.003750pt}%
\definecolor{currentstroke}{rgb}{0.007843,0.619608,0.450980}%
\pgfsetstrokecolor{currentstroke}%
\pgfsetdash{}{0pt}%
\pgfpathmoveto{\pgfqpoint{0.761265in}{1.128142in}}%
\pgfpathlineto{\pgfqpoint{0.863258in}{1.326642in}}%
\pgfpathlineto{\pgfqpoint{1.033247in}{1.405241in}}%
\pgfpathlineto{\pgfqpoint{1.271231in}{1.480098in}}%
\pgfpathlineto{\pgfqpoint{1.577211in}{1.538581in}}%
\pgfpathlineto{\pgfqpoint{1.951186in}{1.558755in}}%
\pgfpathlineto{\pgfqpoint{2.393157in}{1.601164in}}%
\pgfpathlineto{\pgfqpoint{2.903123in}{1.604980in}}%
\pgfpathlineto{\pgfqpoint{3.481085in}{1.628330in}}%
\pgfpathlineto{\pgfqpoint{4.127042in}{1.651244in}}%
\pgfusepath{stroke}%
\end{pgfscope}%
\begin{pgfscope}%
\pgfsetbuttcap%
\pgfsetroundjoin%
\definecolor{currentfill}{rgb}{0.007843,0.619608,0.450980}%
\pgfsetfillcolor{currentfill}%
\pgfsetlinewidth{0.752812pt}%
\definecolor{currentstroke}{rgb}{1.000000,1.000000,1.000000}%
\pgfsetstrokecolor{currentstroke}%
\pgfsetdash{}{0pt}%
\pgfsys@defobject{currentmarker}{\pgfqpoint{-0.034722in}{-0.034722in}}{\pgfqpoint{0.034722in}{0.034722in}}{%
\pgfpathmoveto{\pgfqpoint{0.000000in}{-0.034722in}}%
\pgfpathcurveto{\pgfqpoint{0.009208in}{-0.034722in}}{\pgfqpoint{0.018041in}{-0.031064in}}{\pgfqpoint{0.024552in}{-0.024552in}}%
\pgfpathcurveto{\pgfqpoint{0.031064in}{-0.018041in}}{\pgfqpoint{0.034722in}{-0.009208in}}{\pgfqpoint{0.034722in}{0.000000in}}%
\pgfpathcurveto{\pgfqpoint{0.034722in}{0.009208in}}{\pgfqpoint{0.031064in}{0.018041in}}{\pgfqpoint{0.024552in}{0.024552in}}%
\pgfpathcurveto{\pgfqpoint{0.018041in}{0.031064in}}{\pgfqpoint{0.009208in}{0.034722in}}{\pgfqpoint{0.000000in}{0.034722in}}%
\pgfpathcurveto{\pgfqpoint{-0.009208in}{0.034722in}}{\pgfqpoint{-0.018041in}{0.031064in}}{\pgfqpoint{-0.024552in}{0.024552in}}%
\pgfpathcurveto{\pgfqpoint{-0.031064in}{0.018041in}}{\pgfqpoint{-0.034722in}{0.009208in}}{\pgfqpoint{-0.034722in}{0.000000in}}%
\pgfpathcurveto{\pgfqpoint{-0.034722in}{-0.009208in}}{\pgfqpoint{-0.031064in}{-0.018041in}}{\pgfqpoint{-0.024552in}{-0.024552in}}%
\pgfpathcurveto{\pgfqpoint{-0.018041in}{-0.031064in}}{\pgfqpoint{-0.009208in}{-0.034722in}}{\pgfqpoint{0.000000in}{-0.034722in}}%
\pgfpathlineto{\pgfqpoint{0.000000in}{-0.034722in}}%
\pgfpathclose%
\pgfusepath{stroke,fill}%
}%
\begin{pgfscope}%
\pgfsys@transformshift{0.761265in}{1.128142in}%
\pgfsys@useobject{currentmarker}{}%
\end{pgfscope}%
\begin{pgfscope}%
\pgfsys@transformshift{0.863258in}{1.326642in}%
\pgfsys@useobject{currentmarker}{}%
\end{pgfscope}%
\begin{pgfscope}%
\pgfsys@transformshift{1.033247in}{1.405241in}%
\pgfsys@useobject{currentmarker}{}%
\end{pgfscope}%
\begin{pgfscope}%
\pgfsys@transformshift{1.271231in}{1.480098in}%
\pgfsys@useobject{currentmarker}{}%
\end{pgfscope}%
\begin{pgfscope}%
\pgfsys@transformshift{1.577211in}{1.538581in}%
\pgfsys@useobject{currentmarker}{}%
\end{pgfscope}%
\begin{pgfscope}%
\pgfsys@transformshift{1.951186in}{1.558755in}%
\pgfsys@useobject{currentmarker}{}%
\end{pgfscope}%
\begin{pgfscope}%
\pgfsys@transformshift{2.393157in}{1.601164in}%
\pgfsys@useobject{currentmarker}{}%
\end{pgfscope}%
\begin{pgfscope}%
\pgfsys@transformshift{2.903123in}{1.604980in}%
\pgfsys@useobject{currentmarker}{}%
\end{pgfscope}%
\begin{pgfscope}%
\pgfsys@transformshift{3.481085in}{1.628330in}%
\pgfsys@useobject{currentmarker}{}%
\end{pgfscope}%
\begin{pgfscope}%
\pgfsys@transformshift{4.127042in}{1.651244in}%
\pgfsys@useobject{currentmarker}{}%
\end{pgfscope}%
\end{pgfscope}%
\begin{pgfscope}%
\pgfsetroundcap%
\pgfsetroundjoin%
\pgfsetlinewidth{1.003750pt}%
\definecolor{currentstroke}{rgb}{0.835294,0.368627,0.000000}%
\pgfsetstrokecolor{currentstroke}%
\pgfsetdash{}{0pt}%
\pgfpathmoveto{\pgfqpoint{0.761265in}{1.236872in}}%
\pgfpathlineto{\pgfqpoint{0.863258in}{1.364862in}}%
\pgfpathlineto{\pgfqpoint{1.033247in}{1.444129in}}%
\pgfpathlineto{\pgfqpoint{1.271231in}{1.496647in}}%
\pgfpathlineto{\pgfqpoint{1.577211in}{1.535210in}}%
\pgfpathlineto{\pgfqpoint{1.951186in}{1.581804in}}%
\pgfpathlineto{\pgfqpoint{2.393157in}{1.608464in}}%
\pgfpathlineto{\pgfqpoint{2.903123in}{1.632055in}}%
\pgfpathlineto{\pgfqpoint{3.481085in}{1.653635in}}%
\pgfpathlineto{\pgfqpoint{4.127042in}{1.674145in}}%
\pgfusepath{stroke}%
\end{pgfscope}%
\begin{pgfscope}%
\pgfsetbuttcap%
\pgfsetroundjoin%
\definecolor{currentfill}{rgb}{0.835294,0.368627,0.000000}%
\pgfsetfillcolor{currentfill}%
\pgfsetlinewidth{0.752812pt}%
\definecolor{currentstroke}{rgb}{1.000000,1.000000,1.000000}%
\pgfsetstrokecolor{currentstroke}%
\pgfsetdash{}{0pt}%
\pgfsys@defobject{currentmarker}{\pgfqpoint{-0.034722in}{-0.034722in}}{\pgfqpoint{0.034722in}{0.034722in}}{%
\pgfpathmoveto{\pgfqpoint{0.000000in}{-0.034722in}}%
\pgfpathcurveto{\pgfqpoint{0.009208in}{-0.034722in}}{\pgfqpoint{0.018041in}{-0.031064in}}{\pgfqpoint{0.024552in}{-0.024552in}}%
\pgfpathcurveto{\pgfqpoint{0.031064in}{-0.018041in}}{\pgfqpoint{0.034722in}{-0.009208in}}{\pgfqpoint{0.034722in}{0.000000in}}%
\pgfpathcurveto{\pgfqpoint{0.034722in}{0.009208in}}{\pgfqpoint{0.031064in}{0.018041in}}{\pgfqpoint{0.024552in}{0.024552in}}%
\pgfpathcurveto{\pgfqpoint{0.018041in}{0.031064in}}{\pgfqpoint{0.009208in}{0.034722in}}{\pgfqpoint{0.000000in}{0.034722in}}%
\pgfpathcurveto{\pgfqpoint{-0.009208in}{0.034722in}}{\pgfqpoint{-0.018041in}{0.031064in}}{\pgfqpoint{-0.024552in}{0.024552in}}%
\pgfpathcurveto{\pgfqpoint{-0.031064in}{0.018041in}}{\pgfqpoint{-0.034722in}{0.009208in}}{\pgfqpoint{-0.034722in}{0.000000in}}%
\pgfpathcurveto{\pgfqpoint{-0.034722in}{-0.009208in}}{\pgfqpoint{-0.031064in}{-0.018041in}}{\pgfqpoint{-0.024552in}{-0.024552in}}%
\pgfpathcurveto{\pgfqpoint{-0.018041in}{-0.031064in}}{\pgfqpoint{-0.009208in}{-0.034722in}}{\pgfqpoint{0.000000in}{-0.034722in}}%
\pgfpathlineto{\pgfqpoint{0.000000in}{-0.034722in}}%
\pgfpathclose%
\pgfusepath{stroke,fill}%
}%
\begin{pgfscope}%
\pgfsys@transformshift{0.761265in}{1.236872in}%
\pgfsys@useobject{currentmarker}{}%
\end{pgfscope}%
\begin{pgfscope}%
\pgfsys@transformshift{0.863258in}{1.364862in}%
\pgfsys@useobject{currentmarker}{}%
\end{pgfscope}%
\begin{pgfscope}%
\pgfsys@transformshift{1.033247in}{1.444129in}%
\pgfsys@useobject{currentmarker}{}%
\end{pgfscope}%
\begin{pgfscope}%
\pgfsys@transformshift{1.271231in}{1.496647in}%
\pgfsys@useobject{currentmarker}{}%
\end{pgfscope}%
\begin{pgfscope}%
\pgfsys@transformshift{1.577211in}{1.535210in}%
\pgfsys@useobject{currentmarker}{}%
\end{pgfscope}%
\begin{pgfscope}%
\pgfsys@transformshift{1.951186in}{1.581804in}%
\pgfsys@useobject{currentmarker}{}%
\end{pgfscope}%
\begin{pgfscope}%
\pgfsys@transformshift{2.393157in}{1.608464in}%
\pgfsys@useobject{currentmarker}{}%
\end{pgfscope}%
\begin{pgfscope}%
\pgfsys@transformshift{2.903123in}{1.632055in}%
\pgfsys@useobject{currentmarker}{}%
\end{pgfscope}%
\begin{pgfscope}%
\pgfsys@transformshift{3.481085in}{1.653635in}%
\pgfsys@useobject{currentmarker}{}%
\end{pgfscope}%
\begin{pgfscope}%
\pgfsys@transformshift{4.127042in}{1.674145in}%
\pgfsys@useobject{currentmarker}{}%
\end{pgfscope}%
\end{pgfscope}%
\begin{pgfscope}%
\pgfsetroundcap%
\pgfsetroundjoin%
\pgfsetlinewidth{1.003750pt}%
\definecolor{currentstroke}{rgb}{0.800000,0.470588,0.737255}%
\pgfsetstrokecolor{currentstroke}%
\pgfsetdash{}{0pt}%
\pgfpathmoveto{\pgfqpoint{0.761265in}{0.611875in}}%
\pgfpathlineto{\pgfqpoint{0.863258in}{0.650278in}}%
\pgfpathlineto{\pgfqpoint{1.033247in}{0.889508in}}%
\pgfpathlineto{\pgfqpoint{1.271231in}{0.827000in}}%
\pgfpathlineto{\pgfqpoint{1.577211in}{0.838542in}}%
\pgfpathlineto{\pgfqpoint{1.951186in}{0.859620in}}%
\pgfpathlineto{\pgfqpoint{2.393157in}{0.876283in}}%
\pgfpathlineto{\pgfqpoint{2.903123in}{0.891106in}}%
\pgfpathlineto{\pgfqpoint{3.481085in}{0.921102in}}%
\pgfpathlineto{\pgfqpoint{4.127042in}{0.931025in}}%
\pgfusepath{stroke}%
\end{pgfscope}%
\begin{pgfscope}%
\pgfsetbuttcap%
\pgfsetroundjoin%
\definecolor{currentfill}{rgb}{0.800000,0.470588,0.737255}%
\pgfsetfillcolor{currentfill}%
\pgfsetlinewidth{0.752812pt}%
\definecolor{currentstroke}{rgb}{1.000000,1.000000,1.000000}%
\pgfsetstrokecolor{currentstroke}%
\pgfsetdash{}{0pt}%
\pgfsys@defobject{currentmarker}{\pgfqpoint{-0.034722in}{-0.034722in}}{\pgfqpoint{0.034722in}{0.034722in}}{%
\pgfpathmoveto{\pgfqpoint{0.000000in}{-0.034722in}}%
\pgfpathcurveto{\pgfqpoint{0.009208in}{-0.034722in}}{\pgfqpoint{0.018041in}{-0.031064in}}{\pgfqpoint{0.024552in}{-0.024552in}}%
\pgfpathcurveto{\pgfqpoint{0.031064in}{-0.018041in}}{\pgfqpoint{0.034722in}{-0.009208in}}{\pgfqpoint{0.034722in}{0.000000in}}%
\pgfpathcurveto{\pgfqpoint{0.034722in}{0.009208in}}{\pgfqpoint{0.031064in}{0.018041in}}{\pgfqpoint{0.024552in}{0.024552in}}%
\pgfpathcurveto{\pgfqpoint{0.018041in}{0.031064in}}{\pgfqpoint{0.009208in}{0.034722in}}{\pgfqpoint{0.000000in}{0.034722in}}%
\pgfpathcurveto{\pgfqpoint{-0.009208in}{0.034722in}}{\pgfqpoint{-0.018041in}{0.031064in}}{\pgfqpoint{-0.024552in}{0.024552in}}%
\pgfpathcurveto{\pgfqpoint{-0.031064in}{0.018041in}}{\pgfqpoint{-0.034722in}{0.009208in}}{\pgfqpoint{-0.034722in}{0.000000in}}%
\pgfpathcurveto{\pgfqpoint{-0.034722in}{-0.009208in}}{\pgfqpoint{-0.031064in}{-0.018041in}}{\pgfqpoint{-0.024552in}{-0.024552in}}%
\pgfpathcurveto{\pgfqpoint{-0.018041in}{-0.031064in}}{\pgfqpoint{-0.009208in}{-0.034722in}}{\pgfqpoint{0.000000in}{-0.034722in}}%
\pgfpathlineto{\pgfqpoint{0.000000in}{-0.034722in}}%
\pgfpathclose%
\pgfusepath{stroke,fill}%
}%
\begin{pgfscope}%
\pgfsys@transformshift{0.761265in}{0.611875in}%
\pgfsys@useobject{currentmarker}{}%
\end{pgfscope}%
\begin{pgfscope}%
\pgfsys@transformshift{0.863258in}{0.650278in}%
\pgfsys@useobject{currentmarker}{}%
\end{pgfscope}%
\begin{pgfscope}%
\pgfsys@transformshift{1.033247in}{0.889508in}%
\pgfsys@useobject{currentmarker}{}%
\end{pgfscope}%
\begin{pgfscope}%
\pgfsys@transformshift{1.271231in}{0.827000in}%
\pgfsys@useobject{currentmarker}{}%
\end{pgfscope}%
\begin{pgfscope}%
\pgfsys@transformshift{1.577211in}{0.838542in}%
\pgfsys@useobject{currentmarker}{}%
\end{pgfscope}%
\begin{pgfscope}%
\pgfsys@transformshift{1.951186in}{0.859620in}%
\pgfsys@useobject{currentmarker}{}%
\end{pgfscope}%
\begin{pgfscope}%
\pgfsys@transformshift{2.393157in}{0.876283in}%
\pgfsys@useobject{currentmarker}{}%
\end{pgfscope}%
\begin{pgfscope}%
\pgfsys@transformshift{2.903123in}{0.891106in}%
\pgfsys@useobject{currentmarker}{}%
\end{pgfscope}%
\begin{pgfscope}%
\pgfsys@transformshift{3.481085in}{0.921102in}%
\pgfsys@useobject{currentmarker}{}%
\end{pgfscope}%
\begin{pgfscope}%
\pgfsys@transformshift{4.127042in}{0.931025in}%
\pgfsys@useobject{currentmarker}{}%
\end{pgfscope}%
\end{pgfscope}%
\begin{pgfscope}%
\pgfsetroundcap%
\pgfsetroundjoin%
\pgfsetlinewidth{1.003750pt}%
\definecolor{currentstroke}{rgb}{0.792157,0.568627,0.380392}%
\pgfsetstrokecolor{currentstroke}%
\pgfsetdash{}{0pt}%
\pgfpathmoveto{\pgfqpoint{0.761265in}{0.609733in}}%
\pgfpathlineto{\pgfqpoint{0.863258in}{0.670833in}}%
\pgfpathlineto{\pgfqpoint{1.033247in}{0.730689in}}%
\pgfpathlineto{\pgfqpoint{1.271231in}{0.752862in}}%
\pgfpathlineto{\pgfqpoint{1.577211in}{0.786471in}}%
\pgfpathlineto{\pgfqpoint{1.951186in}{0.818528in}}%
\pgfpathlineto{\pgfqpoint{2.393157in}{0.868719in}}%
\pgfpathlineto{\pgfqpoint{2.903123in}{0.887618in}}%
\pgfpathlineto{\pgfqpoint{3.481085in}{0.904580in}}%
\pgfpathlineto{\pgfqpoint{4.127042in}{0.924471in}}%
\pgfusepath{stroke}%
\end{pgfscope}%
\begin{pgfscope}%
\pgfsetbuttcap%
\pgfsetroundjoin%
\definecolor{currentfill}{rgb}{0.792157,0.568627,0.380392}%
\pgfsetfillcolor{currentfill}%
\pgfsetlinewidth{0.752812pt}%
\definecolor{currentstroke}{rgb}{1.000000,1.000000,1.000000}%
\pgfsetstrokecolor{currentstroke}%
\pgfsetdash{}{0pt}%
\pgfsys@defobject{currentmarker}{\pgfqpoint{-0.034722in}{-0.034722in}}{\pgfqpoint{0.034722in}{0.034722in}}{%
\pgfpathmoveto{\pgfqpoint{0.000000in}{-0.034722in}}%
\pgfpathcurveto{\pgfqpoint{0.009208in}{-0.034722in}}{\pgfqpoint{0.018041in}{-0.031064in}}{\pgfqpoint{0.024552in}{-0.024552in}}%
\pgfpathcurveto{\pgfqpoint{0.031064in}{-0.018041in}}{\pgfqpoint{0.034722in}{-0.009208in}}{\pgfqpoint{0.034722in}{0.000000in}}%
\pgfpathcurveto{\pgfqpoint{0.034722in}{0.009208in}}{\pgfqpoint{0.031064in}{0.018041in}}{\pgfqpoint{0.024552in}{0.024552in}}%
\pgfpathcurveto{\pgfqpoint{0.018041in}{0.031064in}}{\pgfqpoint{0.009208in}{0.034722in}}{\pgfqpoint{0.000000in}{0.034722in}}%
\pgfpathcurveto{\pgfqpoint{-0.009208in}{0.034722in}}{\pgfqpoint{-0.018041in}{0.031064in}}{\pgfqpoint{-0.024552in}{0.024552in}}%
\pgfpathcurveto{\pgfqpoint{-0.031064in}{0.018041in}}{\pgfqpoint{-0.034722in}{0.009208in}}{\pgfqpoint{-0.034722in}{0.000000in}}%
\pgfpathcurveto{\pgfqpoint{-0.034722in}{-0.009208in}}{\pgfqpoint{-0.031064in}{-0.018041in}}{\pgfqpoint{-0.024552in}{-0.024552in}}%
\pgfpathcurveto{\pgfqpoint{-0.018041in}{-0.031064in}}{\pgfqpoint{-0.009208in}{-0.034722in}}{\pgfqpoint{0.000000in}{-0.034722in}}%
\pgfpathlineto{\pgfqpoint{0.000000in}{-0.034722in}}%
\pgfpathclose%
\pgfusepath{stroke,fill}%
}%
\begin{pgfscope}%
\pgfsys@transformshift{0.761265in}{0.609733in}%
\pgfsys@useobject{currentmarker}{}%
\end{pgfscope}%
\begin{pgfscope}%
\pgfsys@transformshift{0.863258in}{0.670833in}%
\pgfsys@useobject{currentmarker}{}%
\end{pgfscope}%
\begin{pgfscope}%
\pgfsys@transformshift{1.033247in}{0.730689in}%
\pgfsys@useobject{currentmarker}{}%
\end{pgfscope}%
\begin{pgfscope}%
\pgfsys@transformshift{1.271231in}{0.752862in}%
\pgfsys@useobject{currentmarker}{}%
\end{pgfscope}%
\begin{pgfscope}%
\pgfsys@transformshift{1.577211in}{0.786471in}%
\pgfsys@useobject{currentmarker}{}%
\end{pgfscope}%
\begin{pgfscope}%
\pgfsys@transformshift{1.951186in}{0.818528in}%
\pgfsys@useobject{currentmarker}{}%
\end{pgfscope}%
\begin{pgfscope}%
\pgfsys@transformshift{2.393157in}{0.868719in}%
\pgfsys@useobject{currentmarker}{}%
\end{pgfscope}%
\begin{pgfscope}%
\pgfsys@transformshift{2.903123in}{0.887618in}%
\pgfsys@useobject{currentmarker}{}%
\end{pgfscope}%
\begin{pgfscope}%
\pgfsys@transformshift{3.481085in}{0.904580in}%
\pgfsys@useobject{currentmarker}{}%
\end{pgfscope}%
\begin{pgfscope}%
\pgfsys@transformshift{4.127042in}{0.924471in}%
\pgfsys@useobject{currentmarker}{}%
\end{pgfscope}%
\end{pgfscope}%
\end{pgfpicture}%
\makeatother%
\endgroup%

						\end{figcenter}
						\caption{Import time of the \enquote{OSM rural} dataset by tasks.}
					\end{subfigure}
				\end{figcenter}
				\caption{Graph generation times by task for the OSM-based datasets.}
				\label{fig:eval-import-details}
			\end{figure}
			
			In \Cref{fig:eval-import-details} both OSM dataset import times are split into the performed tasks.
			The task with the largest effort in terms of the required time differs from the dataset category.
			
			In the \enquote{OSM city} datasets, the \term*[k-nearest neighbors]{$k$ nearest neighbor} (\term*{kNN}) search is the most time consuming task with a share on the total time of constantly over 60\%.
			The second most time consuming task is the merge operation, where visibility and road edges are merged.
			Together, these two steps are responsible for over 98.1\% of the total graph generation time throughout all \enquote{OSM city} datasets.
			
			Different proportions of the tasks can be seen in the \enquote{OSM rural} category.
			Here, the merge operation is the more time consuming task with a share of at least 51.4\% on the overall graph generation time.
			This effect of a more time-consuming merge operation is further discussed in \Cref{subsubsec:dataset-without-roads-obstacles}.
			The times of the kNN search and the merge operation sum up to at least 96.5\% of the graph generation time.
			
			Even though the order which task is the more time consuming is different, both times of the kNN search and the merge operation are the most significant ones.
			The graph creation task in the \enquote{OSM rural} datasets is the only additional task throughout all tasks in all OSM-based datasets with a share of over 1\%.
			All other tasks have a negligible share on the total graph generation time of below 1\%, which decrease with the size of the dataset.
	
		\subsubsection{Routing}
		
			Routing consists of several tasks with varying degrees of impact on the total routing time.
			The time required for a routing task is influenced by two main factors.
			First, the dataset size is a strong influencing factor but has differently large impacts on the time of the routing tasks.
			Second, the structure of the data, meaning the number, size and distribution of obstacles, also influences the required time for routing.
			In fact, the size of the dataset is the strongest influence on the overall routing time as it affects all tasks during routing.
			\Cref{fig:eval-city-routing-details} and \ref{fig:eval-rural-routing-details} show the routing times from the \enquote{OSM city} and \enquote{OSM rural} datasets as well as details on the separate tasks during routing.
			
			Starting with the total routing times, both categories show different runtime behavior in terms of the route length.
			While the \enquote{OSM city} datasets shows no correlation between the beeline distance and the routing time, the \enquote{OSM rural} dataset shows a negative correlation (longer distance means shorter routing time).
			
			This negative correlation is most prominent in the largest \enquote{OSM rural} dataset between the first and the fifth routing request.
			Since the majority of the time is needed to connect the source and destination locations to the graph, the presence of nearby obstacles has a significant impact on the routing time.
			As illustrated in \Cref{fig:eval-osm-rural-map}, the destination vertex of the first routing request has no nearby obstacles.
			This means no efficient use of shadow areas can be made resulting in more visibility checks and therefore slower runtime.
			The destination vertex of the fifth request has multiple obstacles nearby casting large shadow areas towards many buildings, which then do not need to be further checked.
			As a result, 86 visibility edges are connected to the source vertex of the first and 18 to the source vertex of the fifth request.
			
			This phenomenon of longer routing times for routing between short distant vertices does not appear in the \enquote{OSM city} dataset due to the high density of obstacles enabling an efficient use of shadow areas.
			But the effect is still visible in \Cref{fig:eval-city-routing-details-a} at the requests with a beeline distance between 0.5km and 1km.
			Requests in this range, which have a longer processing time, contain vertices at junctions or wide roads with only a few surrounding obstacles, which yields longer processing times.
			
%			\vspace{3ex}
%			\noindent
			\begin{minipage}{\textwidth}
				\begin{minipage}{0.4\textwidth}
					\centering
					\begin{tabularx}{\textwidth}{p{2cm}|X|X}
						\textbf{Request no.}				& 1			& 5			\\\hline
						\textbf{Beeline\newline distance}	& 100.42 m	& 300.78 m	\\\hline
						\textbf{Routing\newline time}		& 519.5 ms	& 127.19 ms	\\\hline
						\textbf{Time per m}					& 5.17 ms	& 0.42 ms
					\end{tabularx}
				\end{minipage}
				\hfill
				\begin{minipage}{0.56\textwidth}
					\centering
					\includegraphics[width=\textwidth]{images/qgis-osm-rural}
				\end{minipage}
				\\
				\begin{minipage}[t]{0.4\textwidth}
					\captionof{table}[Statistics of routing requests with differently many visibility neighbors.]{Measured values for the first and fifth routing requests illustrated in \Cref{fig:eval-osm-rural-map}.}
				\end{minipage}
				\hfill
				\begin{minipage}[t]{0.56\textwidth}
					\captionof{figure}[Visualization of routing requests with differently many visibility neighbors.]{The first and fifth routing requests (red arrow between source and destination) with the nearby obstacles (gray). The bidirectional visibility edges of the two destinations are shown in blue and green.}
					\label{fig:eval-osm-rural-map}
				\end{minipage}
			\end{minipage}
			
			\clearpage
			\begin{figure}[h!]
				\begin{figcenter}
					\begin{subfigure}[t]{\textwidth}
						\begin{figcenter}
							\begingroup%
\makeatletter%
\begin{pgfpicture}%
\pgfpathrectangle{\pgfpointorigin}{\pgfqpoint{6.077890in}{2.407638in}}%
\pgfusepath{use as bounding box}%
\begin{pgfscope}%
\pgfsetbuttcap%
\pgfsetmiterjoin%
\definecolor{currentfill}{rgb}{1.000000,1.000000,1.000000}%
\pgfsetfillcolor{currentfill}%
\pgfsetlinewidth{0.000000pt}%
\definecolor{currentstroke}{rgb}{1.000000,1.000000,1.000000}%
\pgfsetstrokecolor{currentstroke}%
\pgfsetdash{}{0pt}%
\pgfpathmoveto{\pgfqpoint{0.000000in}{0.000000in}}%
\pgfpathlineto{\pgfqpoint{6.077890in}{0.000000in}}%
\pgfpathlineto{\pgfqpoint{6.077890in}{2.407638in}}%
\pgfpathlineto{\pgfqpoint{0.000000in}{2.407638in}}%
\pgfpathlineto{\pgfqpoint{0.000000in}{0.000000in}}%
\pgfpathclose%
\pgfusepath{fill}%
\end{pgfscope}%
\begin{pgfscope}%
\pgfsetbuttcap%
\pgfsetmiterjoin%
\definecolor{currentfill}{rgb}{1.000000,1.000000,1.000000}%
\pgfsetfillcolor{currentfill}%
\pgfsetlinewidth{0.000000pt}%
\definecolor{currentstroke}{rgb}{0.000000,0.000000,0.000000}%
\pgfsetstrokecolor{currentstroke}%
\pgfsetstrokeopacity{0.000000}%
\pgfsetdash{}{0pt}%
\pgfpathmoveto{\pgfqpoint{0.601779in}{0.451389in}}%
\pgfpathlineto{\pgfqpoint{5.095229in}{0.451389in}}%
\pgfpathlineto{\pgfqpoint{5.095229in}{2.407638in}}%
\pgfpathlineto{\pgfqpoint{0.601779in}{2.407638in}}%
\pgfpathlineto{\pgfqpoint{0.601779in}{0.451389in}}%
\pgfpathclose%
\pgfusepath{fill}%
\end{pgfscope}%
\begin{pgfscope}%
\pgfpathrectangle{\pgfqpoint{0.601779in}{0.451389in}}{\pgfqpoint{4.493449in}{1.956249in}}%
\pgfusepath{clip}%
\pgfsetroundcap%
\pgfsetroundjoin%
\pgfsetlinewidth{1.003750pt}%
\definecolor{currentstroke}{rgb}{0.800000,0.800000,0.800000}%
\pgfsetstrokecolor{currentstroke}%
\pgfsetdash{}{0pt}%
\pgfpathmoveto{\pgfqpoint{0.601779in}{0.451389in}}%
\pgfpathlineto{\pgfqpoint{0.601779in}{2.407638in}}%
\pgfusepath{stroke}%
\end{pgfscope}%
\begin{pgfscope}%
\definecolor{textcolor}{rgb}{0.150000,0.150000,0.150000}%
\pgfsetstrokecolor{textcolor}%
\pgfsetfillcolor{textcolor}%
\pgftext[x=0.601779in,y=0.319444in,,top]{\color{textcolor}\sffamily\fontsize{9.000000}{10.800000}\selectfont 0.0}%
\end{pgfscope}%
\begin{pgfscope}%
\pgfpathrectangle{\pgfqpoint{0.601779in}{0.451389in}}{\pgfqpoint{4.493449in}{1.956249in}}%
\pgfusepath{clip}%
\pgfsetroundcap%
\pgfsetroundjoin%
\pgfsetlinewidth{1.003750pt}%
\definecolor{currentstroke}{rgb}{0.800000,0.800000,0.800000}%
\pgfsetstrokecolor{currentstroke}%
\pgfsetdash{}{0pt}%
\pgfpathmoveto{\pgfqpoint{1.460352in}{0.451389in}}%
\pgfpathlineto{\pgfqpoint{1.460352in}{2.407638in}}%
\pgfusepath{stroke}%
\end{pgfscope}%
\begin{pgfscope}%
\definecolor{textcolor}{rgb}{0.150000,0.150000,0.150000}%
\pgfsetstrokecolor{textcolor}%
\pgfsetfillcolor{textcolor}%
\pgftext[x=1.460352in,y=0.319444in,,top]{\color{textcolor}\sffamily\fontsize{9.000000}{10.800000}\selectfont 0.5}%
\end{pgfscope}%
\begin{pgfscope}%
\pgfpathrectangle{\pgfqpoint{0.601779in}{0.451389in}}{\pgfqpoint{4.493449in}{1.956249in}}%
\pgfusepath{clip}%
\pgfsetroundcap%
\pgfsetroundjoin%
\pgfsetlinewidth{1.003750pt}%
\definecolor{currentstroke}{rgb}{0.800000,0.800000,0.800000}%
\pgfsetstrokecolor{currentstroke}%
\pgfsetdash{}{0pt}%
\pgfpathmoveto{\pgfqpoint{2.318924in}{0.451389in}}%
\pgfpathlineto{\pgfqpoint{2.318924in}{2.407638in}}%
\pgfusepath{stroke}%
\end{pgfscope}%
\begin{pgfscope}%
\definecolor{textcolor}{rgb}{0.150000,0.150000,0.150000}%
\pgfsetstrokecolor{textcolor}%
\pgfsetfillcolor{textcolor}%
\pgftext[x=2.318924in,y=0.319444in,,top]{\color{textcolor}\sffamily\fontsize{9.000000}{10.800000}\selectfont 1.0}%
\end{pgfscope}%
\begin{pgfscope}%
\pgfpathrectangle{\pgfqpoint{0.601779in}{0.451389in}}{\pgfqpoint{4.493449in}{1.956249in}}%
\pgfusepath{clip}%
\pgfsetroundcap%
\pgfsetroundjoin%
\pgfsetlinewidth{1.003750pt}%
\definecolor{currentstroke}{rgb}{0.800000,0.800000,0.800000}%
\pgfsetstrokecolor{currentstroke}%
\pgfsetdash{}{0pt}%
\pgfpathmoveto{\pgfqpoint{3.177497in}{0.451389in}}%
\pgfpathlineto{\pgfqpoint{3.177497in}{2.407638in}}%
\pgfusepath{stroke}%
\end{pgfscope}%
\begin{pgfscope}%
\definecolor{textcolor}{rgb}{0.150000,0.150000,0.150000}%
\pgfsetstrokecolor{textcolor}%
\pgfsetfillcolor{textcolor}%
\pgftext[x=3.177497in,y=0.319444in,,top]{\color{textcolor}\sffamily\fontsize{9.000000}{10.800000}\selectfont 1.5}%
\end{pgfscope}%
\begin{pgfscope}%
\pgfpathrectangle{\pgfqpoint{0.601779in}{0.451389in}}{\pgfqpoint{4.493449in}{1.956249in}}%
\pgfusepath{clip}%
\pgfsetroundcap%
\pgfsetroundjoin%
\pgfsetlinewidth{1.003750pt}%
\definecolor{currentstroke}{rgb}{0.800000,0.800000,0.800000}%
\pgfsetstrokecolor{currentstroke}%
\pgfsetdash{}{0pt}%
\pgfpathmoveto{\pgfqpoint{4.036069in}{0.451389in}}%
\pgfpathlineto{\pgfqpoint{4.036069in}{2.407638in}}%
\pgfusepath{stroke}%
\end{pgfscope}%
\begin{pgfscope}%
\definecolor{textcolor}{rgb}{0.150000,0.150000,0.150000}%
\pgfsetstrokecolor{textcolor}%
\pgfsetfillcolor{textcolor}%
\pgftext[x=4.036069in,y=0.319444in,,top]{\color{textcolor}\sffamily\fontsize{9.000000}{10.800000}\selectfont 2.0}%
\end{pgfscope}%
\begin{pgfscope}%
\pgfpathrectangle{\pgfqpoint{0.601779in}{0.451389in}}{\pgfqpoint{4.493449in}{1.956249in}}%
\pgfusepath{clip}%
\pgfsetroundcap%
\pgfsetroundjoin%
\pgfsetlinewidth{1.003750pt}%
\definecolor{currentstroke}{rgb}{0.800000,0.800000,0.800000}%
\pgfsetstrokecolor{currentstroke}%
\pgfsetdash{}{0pt}%
\pgfpathmoveto{\pgfqpoint{4.894641in}{0.451389in}}%
\pgfpathlineto{\pgfqpoint{4.894641in}{2.407638in}}%
\pgfusepath{stroke}%
\end{pgfscope}%
\begin{pgfscope}%
\definecolor{textcolor}{rgb}{0.150000,0.150000,0.150000}%
\pgfsetstrokecolor{textcolor}%
\pgfsetfillcolor{textcolor}%
\pgftext[x=4.894641in,y=0.319444in,,top]{\color{textcolor}\sffamily\fontsize{9.000000}{10.800000}\selectfont 2.5}%
\end{pgfscope}%
\begin{pgfscope}%
\definecolor{textcolor}{rgb}{0.150000,0.150000,0.150000}%
\pgfsetstrokecolor{textcolor}%
\pgfsetfillcolor{textcolor}%
\pgftext[x=2.848504in,y=0.125000in,,top]{\color{textcolor}\sffamily\fontsize{9.000000}{10.800000}\selectfont Beeline distance in km}%
\end{pgfscope}%
\begin{pgfscope}%
\pgfpathrectangle{\pgfqpoint{0.601779in}{0.451389in}}{\pgfqpoint{4.493449in}{1.956249in}}%
\pgfusepath{clip}%
\pgfsetroundcap%
\pgfsetroundjoin%
\pgfsetlinewidth{1.003750pt}%
\definecolor{currentstroke}{rgb}{0.800000,0.800000,0.800000}%
\pgfsetstrokecolor{currentstroke}%
\pgfsetdash{}{0pt}%
\pgfpathmoveto{\pgfqpoint{0.601779in}{0.451389in}}%
\pgfpathlineto{\pgfqpoint{5.095229in}{0.451389in}}%
\pgfusepath{stroke}%
\end{pgfscope}%
\begin{pgfscope}%
\definecolor{textcolor}{rgb}{0.150000,0.150000,0.150000}%
\pgfsetstrokecolor{textcolor}%
\pgfsetfillcolor{textcolor}%
\pgftext[x=0.400987in, y=0.403903in, left, base]{\color{textcolor}\sffamily\fontsize{9.000000}{10.800000}\selectfont 0}%
\end{pgfscope}%
\begin{pgfscope}%
\pgfpathrectangle{\pgfqpoint{0.601779in}{0.451389in}}{\pgfqpoint{4.493449in}{1.956249in}}%
\pgfusepath{clip}%
\pgfsetroundcap%
\pgfsetroundjoin%
\pgfsetlinewidth{1.003750pt}%
\definecolor{currentstroke}{rgb}{0.800000,0.800000,0.800000}%
\pgfsetstrokecolor{currentstroke}%
\pgfsetdash{}{0pt}%
\pgfpathmoveto{\pgfqpoint{0.601779in}{0.746065in}}%
\pgfpathlineto{\pgfqpoint{5.095229in}{0.746065in}}%
\pgfusepath{stroke}%
\end{pgfscope}%
\begin{pgfscope}%
\definecolor{textcolor}{rgb}{0.150000,0.150000,0.150000}%
\pgfsetstrokecolor{textcolor}%
\pgfsetfillcolor{textcolor}%
\pgftext[x=0.263292in, y=0.698580in, left, base]{\color{textcolor}\sffamily\fontsize{9.000000}{10.800000}\selectfont 250}%
\end{pgfscope}%
\begin{pgfscope}%
\pgfpathrectangle{\pgfqpoint{0.601779in}{0.451389in}}{\pgfqpoint{4.493449in}{1.956249in}}%
\pgfusepath{clip}%
\pgfsetroundcap%
\pgfsetroundjoin%
\pgfsetlinewidth{1.003750pt}%
\definecolor{currentstroke}{rgb}{0.800000,0.800000,0.800000}%
\pgfsetstrokecolor{currentstroke}%
\pgfsetdash{}{0pt}%
\pgfpathmoveto{\pgfqpoint{0.601779in}{1.040741in}}%
\pgfpathlineto{\pgfqpoint{5.095229in}{1.040741in}}%
\pgfusepath{stroke}%
\end{pgfscope}%
\begin{pgfscope}%
\definecolor{textcolor}{rgb}{0.150000,0.150000,0.150000}%
\pgfsetstrokecolor{textcolor}%
\pgfsetfillcolor{textcolor}%
\pgftext[x=0.263292in, y=0.993256in, left, base]{\color{textcolor}\sffamily\fontsize{9.000000}{10.800000}\selectfont 500}%
\end{pgfscope}%
\begin{pgfscope}%
\pgfpathrectangle{\pgfqpoint{0.601779in}{0.451389in}}{\pgfqpoint{4.493449in}{1.956249in}}%
\pgfusepath{clip}%
\pgfsetroundcap%
\pgfsetroundjoin%
\pgfsetlinewidth{1.003750pt}%
\definecolor{currentstroke}{rgb}{0.800000,0.800000,0.800000}%
\pgfsetstrokecolor{currentstroke}%
\pgfsetdash{}{0pt}%
\pgfpathmoveto{\pgfqpoint{0.601779in}{1.335418in}}%
\pgfpathlineto{\pgfqpoint{5.095229in}{1.335418in}}%
\pgfusepath{stroke}%
\end{pgfscope}%
\begin{pgfscope}%
\definecolor{textcolor}{rgb}{0.150000,0.150000,0.150000}%
\pgfsetstrokecolor{textcolor}%
\pgfsetfillcolor{textcolor}%
\pgftext[x=0.263292in, y=1.287933in, left, base]{\color{textcolor}\sffamily\fontsize{9.000000}{10.800000}\selectfont 750}%
\end{pgfscope}%
\begin{pgfscope}%
\pgfpathrectangle{\pgfqpoint{0.601779in}{0.451389in}}{\pgfqpoint{4.493449in}{1.956249in}}%
\pgfusepath{clip}%
\pgfsetroundcap%
\pgfsetroundjoin%
\pgfsetlinewidth{1.003750pt}%
\definecolor{currentstroke}{rgb}{0.800000,0.800000,0.800000}%
\pgfsetstrokecolor{currentstroke}%
\pgfsetdash{}{0pt}%
\pgfpathmoveto{\pgfqpoint{0.601779in}{1.630094in}}%
\pgfpathlineto{\pgfqpoint{5.095229in}{1.630094in}}%
\pgfusepath{stroke}%
\end{pgfscope}%
\begin{pgfscope}%
\definecolor{textcolor}{rgb}{0.150000,0.150000,0.150000}%
\pgfsetstrokecolor{textcolor}%
\pgfsetfillcolor{textcolor}%
\pgftext[x=0.194444in, y=1.582609in, left, base]{\color{textcolor}\sffamily\fontsize{9.000000}{10.800000}\selectfont 1000}%
\end{pgfscope}%
\begin{pgfscope}%
\pgfpathrectangle{\pgfqpoint{0.601779in}{0.451389in}}{\pgfqpoint{4.493449in}{1.956249in}}%
\pgfusepath{clip}%
\pgfsetroundcap%
\pgfsetroundjoin%
\pgfsetlinewidth{1.003750pt}%
\definecolor{currentstroke}{rgb}{0.800000,0.800000,0.800000}%
\pgfsetstrokecolor{currentstroke}%
\pgfsetdash{}{0pt}%
\pgfpathmoveto{\pgfqpoint{0.601779in}{1.924771in}}%
\pgfpathlineto{\pgfqpoint{5.095229in}{1.924771in}}%
\pgfusepath{stroke}%
\end{pgfscope}%
\begin{pgfscope}%
\definecolor{textcolor}{rgb}{0.150000,0.150000,0.150000}%
\pgfsetstrokecolor{textcolor}%
\pgfsetfillcolor{textcolor}%
\pgftext[x=0.194444in, y=1.877285in, left, base]{\color{textcolor}\sffamily\fontsize{9.000000}{10.800000}\selectfont 1250}%
\end{pgfscope}%
\begin{pgfscope}%
\pgfpathrectangle{\pgfqpoint{0.601779in}{0.451389in}}{\pgfqpoint{4.493449in}{1.956249in}}%
\pgfusepath{clip}%
\pgfsetroundcap%
\pgfsetroundjoin%
\pgfsetlinewidth{1.003750pt}%
\definecolor{currentstroke}{rgb}{0.800000,0.800000,0.800000}%
\pgfsetstrokecolor{currentstroke}%
\pgfsetdash{}{0pt}%
\pgfpathmoveto{\pgfqpoint{0.601779in}{2.219447in}}%
\pgfpathlineto{\pgfqpoint{5.095229in}{2.219447in}}%
\pgfusepath{stroke}%
\end{pgfscope}%
\begin{pgfscope}%
\definecolor{textcolor}{rgb}{0.150000,0.150000,0.150000}%
\pgfsetstrokecolor{textcolor}%
\pgfsetfillcolor{textcolor}%
\pgftext[x=0.194444in, y=2.171962in, left, base]{\color{textcolor}\sffamily\fontsize{9.000000}{10.800000}\selectfont 1500}%
\end{pgfscope}%
\begin{pgfscope}%
\definecolor{textcolor}{rgb}{0.150000,0.150000,0.150000}%
\pgfsetstrokecolor{textcolor}%
\pgfsetfillcolor{textcolor}%
\pgftext[x=0.125000in,y=1.429513in,,bottom,rotate=90.000000]{\color{textcolor}\sffamily\fontsize{9.000000}{10.800000}\selectfont Average routing time in ms}%
\end{pgfscope}%
\begin{pgfscope}%
\pgfsetrectcap%
\pgfsetmiterjoin%
\pgfsetlinewidth{1.254687pt}%
\definecolor{currentstroke}{rgb}{0.800000,0.800000,0.800000}%
\pgfsetstrokecolor{currentstroke}%
\pgfsetdash{}{0pt}%
\pgfpathmoveto{\pgfqpoint{0.601779in}{0.451389in}}%
\pgfpathlineto{\pgfqpoint{0.601779in}{2.407638in}}%
\pgfusepath{stroke}%
\end{pgfscope}%
\begin{pgfscope}%
\pgfsetrectcap%
\pgfsetmiterjoin%
\pgfsetlinewidth{1.254687pt}%
\definecolor{currentstroke}{rgb}{0.800000,0.800000,0.800000}%
\pgfsetstrokecolor{currentstroke}%
\pgfsetdash{}{0pt}%
\pgfpathmoveto{\pgfqpoint{5.095229in}{0.451389in}}%
\pgfpathlineto{\pgfqpoint{5.095229in}{2.407638in}}%
\pgfusepath{stroke}%
\end{pgfscope}%
\begin{pgfscope}%
\pgfsetrectcap%
\pgfsetmiterjoin%
\pgfsetlinewidth{1.254687pt}%
\definecolor{currentstroke}{rgb}{0.800000,0.800000,0.800000}%
\pgfsetstrokecolor{currentstroke}%
\pgfsetdash{}{0pt}%
\pgfpathmoveto{\pgfqpoint{0.601779in}{0.451389in}}%
\pgfpathlineto{\pgfqpoint{5.095229in}{0.451389in}}%
\pgfusepath{stroke}%
\end{pgfscope}%
\begin{pgfscope}%
\pgfsetrectcap%
\pgfsetmiterjoin%
\pgfsetlinewidth{1.254687pt}%
\definecolor{currentstroke}{rgb}{0.800000,0.800000,0.800000}%
\pgfsetstrokecolor{currentstroke}%
\pgfsetdash{}{0pt}%
\pgfpathmoveto{\pgfqpoint{0.601779in}{2.407638in}}%
\pgfpathlineto{\pgfqpoint{5.095229in}{2.407638in}}%
\pgfusepath{stroke}%
\end{pgfscope}%
\begin{pgfscope}%
\pgfsetbuttcap%
\pgfsetmiterjoin%
\definecolor{currentfill}{rgb}{1.000000,1.000000,1.000000}%
\pgfsetfillcolor{currentfill}%
\pgfsetfillopacity{0.800000}%
\pgfsetlinewidth{1.003750pt}%
\definecolor{currentstroke}{rgb}{0.800000,0.800000,0.800000}%
\pgfsetstrokecolor{currentstroke}%
\pgfsetstrokeopacity{0.800000}%
\pgfsetdash{}{0pt}%
\pgfpathmoveto{\pgfqpoint{5.295065in}{0.754514in}}%
\pgfpathlineto{\pgfqpoint{6.052890in}{0.754514in}}%
\pgfpathquadraticcurveto{\pgfqpoint{6.077890in}{0.754514in}}{\pgfqpoint{6.077890in}{0.779514in}}%
\pgfpathlineto{\pgfqpoint{6.077890in}{2.079513in}}%
\pgfpathquadraticcurveto{\pgfqpoint{6.077890in}{2.104513in}}{\pgfqpoint{6.052890in}{2.104513in}}%
\pgfpathlineto{\pgfqpoint{5.295065in}{2.104513in}}%
\pgfpathquadraticcurveto{\pgfqpoint{5.270065in}{2.104513in}}{\pgfqpoint{5.270065in}{2.079513in}}%
\pgfpathlineto{\pgfqpoint{5.270065in}{0.779514in}}%
\pgfpathquadraticcurveto{\pgfqpoint{5.270065in}{0.754514in}}{\pgfqpoint{5.295065in}{0.754514in}}%
\pgfpathlineto{\pgfqpoint{5.295065in}{0.754514in}}%
\pgfpathclose%
\pgfusepath{stroke,fill}%
\end{pgfscope}%
\begin{pgfscope}%
\definecolor{textcolor}{rgb}{0.150000,0.150000,0.150000}%
\pgfsetstrokecolor{textcolor}%
\pgfsetfillcolor{textcolor}%
\pgftext[x=5.320065in,y=1.959542in,left,base]{\color{textcolor}\sffamily\fontsize{9.000000}{10.800000}\selectfont Vertex count}%
\end{pgfscope}%
\begin{pgfscope}%
\pgfsetroundcap%
\pgfsetroundjoin%
\pgfsetlinewidth{1.505625pt}%
\definecolor{currentstroke}{rgb}{0.003922,0.450980,0.698039}%
\pgfsetstrokecolor{currentstroke}%
\pgfsetdash{}{0pt}%
\pgfpathmoveto{\pgfqpoint{5.326858in}{1.815792in}}%
\pgfpathlineto{\pgfqpoint{5.451858in}{1.815792in}}%
\pgfpathlineto{\pgfqpoint{5.576858in}{1.815792in}}%
\pgfusepath{stroke}%
\end{pgfscope}%
\begin{pgfscope}%
\definecolor{textcolor}{rgb}{0.150000,0.150000,0.150000}%
\pgfsetstrokecolor{textcolor}%
\pgfsetfillcolor{textcolor}%
\pgftext[x=5.676858in,y=1.772042in,left,base]{\color{textcolor}\sffamily\fontsize{9.000000}{10.800000}\selectfont 7096}%
\end{pgfscope}%
\begin{pgfscope}%
\pgfsetroundcap%
\pgfsetroundjoin%
\pgfsetlinewidth{1.505625pt}%
\definecolor{currentstroke}{rgb}{0.870588,0.560784,0.019608}%
\pgfsetstrokecolor{currentstroke}%
\pgfsetdash{}{0pt}%
\pgfpathmoveto{\pgfqpoint{5.326858in}{1.628292in}}%
\pgfpathlineto{\pgfqpoint{5.451858in}{1.628292in}}%
\pgfpathlineto{\pgfqpoint{5.576858in}{1.628292in}}%
\pgfusepath{stroke}%
\end{pgfscope}%
\begin{pgfscope}%
\definecolor{textcolor}{rgb}{0.150000,0.150000,0.150000}%
\pgfsetstrokecolor{textcolor}%
\pgfsetfillcolor{textcolor}%
\pgftext[x=5.676858in,y=1.584542in,left,base]{\color{textcolor}\sffamily\fontsize{9.000000}{10.800000}\selectfont 12858}%
\end{pgfscope}%
\begin{pgfscope}%
\pgfsetroundcap%
\pgfsetroundjoin%
\pgfsetlinewidth{1.505625pt}%
\definecolor{currentstroke}{rgb}{0.007843,0.619608,0.450980}%
\pgfsetstrokecolor{currentstroke}%
\pgfsetdash{}{0pt}%
\pgfpathmoveto{\pgfqpoint{5.326858in}{1.440792in}}%
\pgfpathlineto{\pgfqpoint{5.451858in}{1.440792in}}%
\pgfpathlineto{\pgfqpoint{5.576858in}{1.440792in}}%
\pgfusepath{stroke}%
\end{pgfscope}%
\begin{pgfscope}%
\definecolor{textcolor}{rgb}{0.150000,0.150000,0.150000}%
\pgfsetstrokecolor{textcolor}%
\pgfsetfillcolor{textcolor}%
\pgftext[x=5.676858in,y=1.397042in,left,base]{\color{textcolor}\sffamily\fontsize{9.000000}{10.800000}\selectfont 17894}%
\end{pgfscope}%
\begin{pgfscope}%
\pgfsetroundcap%
\pgfsetroundjoin%
\pgfsetlinewidth{1.505625pt}%
\definecolor{currentstroke}{rgb}{0.835294,0.368627,0.000000}%
\pgfsetstrokecolor{currentstroke}%
\pgfsetdash{}{0pt}%
\pgfpathmoveto{\pgfqpoint{5.326858in}{1.253293in}}%
\pgfpathlineto{\pgfqpoint{5.451858in}{1.253293in}}%
\pgfpathlineto{\pgfqpoint{5.576858in}{1.253293in}}%
\pgfusepath{stroke}%
\end{pgfscope}%
\begin{pgfscope}%
\definecolor{textcolor}{rgb}{0.150000,0.150000,0.150000}%
\pgfsetstrokecolor{textcolor}%
\pgfsetfillcolor{textcolor}%
\pgftext[x=5.676858in,y=1.209543in,left,base]{\color{textcolor}\sffamily\fontsize{9.000000}{10.800000}\selectfont 22801}%
\end{pgfscope}%
\begin{pgfscope}%
\pgfsetroundcap%
\pgfsetroundjoin%
\pgfsetlinewidth{1.505625pt}%
\definecolor{currentstroke}{rgb}{0.800000,0.470588,0.737255}%
\pgfsetstrokecolor{currentstroke}%
\pgfsetdash{}{0pt}%
\pgfpathmoveto{\pgfqpoint{5.326858in}{1.065793in}}%
\pgfpathlineto{\pgfqpoint{5.451858in}{1.065793in}}%
\pgfpathlineto{\pgfqpoint{5.576858in}{1.065793in}}%
\pgfusepath{stroke}%
\end{pgfscope}%
\begin{pgfscope}%
\definecolor{textcolor}{rgb}{0.150000,0.150000,0.150000}%
\pgfsetstrokecolor{textcolor}%
\pgfsetfillcolor{textcolor}%
\pgftext[x=5.676858in,y=1.022043in,left,base]{\color{textcolor}\sffamily\fontsize{9.000000}{10.800000}\selectfont 34214}%
\end{pgfscope}%
\begin{pgfscope}%
\pgfsetroundcap%
\pgfsetroundjoin%
\pgfsetlinewidth{1.505625pt}%
\definecolor{currentstroke}{rgb}{0.792157,0.568627,0.380392}%
\pgfsetstrokecolor{currentstroke}%
\pgfsetdash{}{0pt}%
\pgfpathmoveto{\pgfqpoint{5.326858in}{0.878293in}}%
\pgfpathlineto{\pgfqpoint{5.451858in}{0.878293in}}%
\pgfpathlineto{\pgfqpoint{5.576858in}{0.878293in}}%
\pgfusepath{stroke}%
\end{pgfscope}%
\begin{pgfscope}%
\definecolor{textcolor}{rgb}{0.150000,0.150000,0.150000}%
\pgfsetstrokecolor{textcolor}%
\pgfsetfillcolor{textcolor}%
\pgftext[x=5.676858in,y=0.834543in,left,base]{\color{textcolor}\sffamily\fontsize{9.000000}{10.800000}\selectfont 45018}%
\end{pgfscope}%
\begin{pgfscope}%
\pgfsetroundcap%
\pgfsetroundjoin%
\pgfsetlinewidth{1.003750pt}%
\definecolor{currentstroke}{rgb}{0.003922,0.450980,0.698039}%
\pgfsetstrokecolor{currentstroke}%
\pgfsetdash{}{0pt}%
\pgfpathmoveto{\pgfqpoint{0.773669in}{0.589481in}}%
\pgfpathlineto{\pgfqpoint{0.859949in}{0.561639in}}%
\pgfpathlineto{\pgfqpoint{0.945274in}{0.585074in}}%
\pgfpathlineto{\pgfqpoint{1.030985in}{0.601652in}}%
\pgfpathlineto{\pgfqpoint{1.116621in}{0.590699in}}%
\pgfpathlineto{\pgfqpoint{1.202341in}{0.536827in}}%
\pgfpathlineto{\pgfqpoint{1.289216in}{0.621530in}}%
\pgfpathlineto{\pgfqpoint{1.373880in}{0.615911in}}%
\pgfpathlineto{\pgfqpoint{1.459905in}{0.600239in}}%
\pgfpathlineto{\pgfqpoint{1.546396in}{0.623906in}}%
\pgfpathlineto{\pgfqpoint{1.631151in}{0.588148in}}%
\pgfusepath{stroke}%
\end{pgfscope}%
\begin{pgfscope}%
\pgfsetbuttcap%
\pgfsetroundjoin%
\definecolor{currentfill}{rgb}{0.003922,0.450980,0.698039}%
\pgfsetfillcolor{currentfill}%
\pgfsetlinewidth{0.752812pt}%
\definecolor{currentstroke}{rgb}{1.000000,1.000000,1.000000}%
\pgfsetstrokecolor{currentstroke}%
\pgfsetdash{}{0pt}%
\pgfsys@defobject{currentmarker}{\pgfqpoint{-0.034722in}{-0.034722in}}{\pgfqpoint{0.034722in}{0.034722in}}{%
\pgfpathmoveto{\pgfqpoint{0.000000in}{-0.034722in}}%
\pgfpathcurveto{\pgfqpoint{0.009208in}{-0.034722in}}{\pgfqpoint{0.018041in}{-0.031064in}}{\pgfqpoint{0.024552in}{-0.024552in}}%
\pgfpathcurveto{\pgfqpoint{0.031064in}{-0.018041in}}{\pgfqpoint{0.034722in}{-0.009208in}}{\pgfqpoint{0.034722in}{0.000000in}}%
\pgfpathcurveto{\pgfqpoint{0.034722in}{0.009208in}}{\pgfqpoint{0.031064in}{0.018041in}}{\pgfqpoint{0.024552in}{0.024552in}}%
\pgfpathcurveto{\pgfqpoint{0.018041in}{0.031064in}}{\pgfqpoint{0.009208in}{0.034722in}}{\pgfqpoint{0.000000in}{0.034722in}}%
\pgfpathcurveto{\pgfqpoint{-0.009208in}{0.034722in}}{\pgfqpoint{-0.018041in}{0.031064in}}{\pgfqpoint{-0.024552in}{0.024552in}}%
\pgfpathcurveto{\pgfqpoint{-0.031064in}{0.018041in}}{\pgfqpoint{-0.034722in}{0.009208in}}{\pgfqpoint{-0.034722in}{0.000000in}}%
\pgfpathcurveto{\pgfqpoint{-0.034722in}{-0.009208in}}{\pgfqpoint{-0.031064in}{-0.018041in}}{\pgfqpoint{-0.024552in}{-0.024552in}}%
\pgfpathcurveto{\pgfqpoint{-0.018041in}{-0.031064in}}{\pgfqpoint{-0.009208in}{-0.034722in}}{\pgfqpoint{0.000000in}{-0.034722in}}%
\pgfpathlineto{\pgfqpoint{0.000000in}{-0.034722in}}%
\pgfpathclose%
\pgfusepath{stroke,fill}%
}%
\begin{pgfscope}%
\pgfsys@transformshift{0.773669in}{0.589481in}%
\pgfsys@useobject{currentmarker}{}%
\end{pgfscope}%
\begin{pgfscope}%
\pgfsys@transformshift{0.859949in}{0.561639in}%
\pgfsys@useobject{currentmarker}{}%
\end{pgfscope}%
\begin{pgfscope}%
\pgfsys@transformshift{0.945274in}{0.585074in}%
\pgfsys@useobject{currentmarker}{}%
\end{pgfscope}%
\begin{pgfscope}%
\pgfsys@transformshift{1.030985in}{0.601652in}%
\pgfsys@useobject{currentmarker}{}%
\end{pgfscope}%
\begin{pgfscope}%
\pgfsys@transformshift{1.116621in}{0.590699in}%
\pgfsys@useobject{currentmarker}{}%
\end{pgfscope}%
\begin{pgfscope}%
\pgfsys@transformshift{1.202341in}{0.536827in}%
\pgfsys@useobject{currentmarker}{}%
\end{pgfscope}%
\begin{pgfscope}%
\pgfsys@transformshift{1.289216in}{0.621530in}%
\pgfsys@useobject{currentmarker}{}%
\end{pgfscope}%
\begin{pgfscope}%
\pgfsys@transformshift{1.373880in}{0.615911in}%
\pgfsys@useobject{currentmarker}{}%
\end{pgfscope}%
\begin{pgfscope}%
\pgfsys@transformshift{1.459905in}{0.600239in}%
\pgfsys@useobject{currentmarker}{}%
\end{pgfscope}%
\begin{pgfscope}%
\pgfsys@transformshift{1.546396in}{0.623906in}%
\pgfsys@useobject{currentmarker}{}%
\end{pgfscope}%
\begin{pgfscope}%
\pgfsys@transformshift{1.631151in}{0.588148in}%
\pgfsys@useobject{currentmarker}{}%
\end{pgfscope}%
\end{pgfscope}%
\begin{pgfscope}%
\pgfsetroundcap%
\pgfsetroundjoin%
\pgfsetlinewidth{1.003750pt}%
\definecolor{currentstroke}{rgb}{0.870588,0.560784,0.019608}%
\pgfsetstrokecolor{currentstroke}%
\pgfsetdash{}{0pt}%
\pgfpathmoveto{\pgfqpoint{0.773669in}{0.684786in}}%
\pgfpathlineto{\pgfqpoint{0.859949in}{0.638408in}}%
\pgfpathlineto{\pgfqpoint{0.945274in}{0.670094in}}%
\pgfpathlineto{\pgfqpoint{1.030985in}{0.682028in}}%
\pgfpathlineto{\pgfqpoint{1.116621in}{0.682021in}}%
\pgfpathlineto{\pgfqpoint{1.202341in}{0.669268in}}%
\pgfpathlineto{\pgfqpoint{1.289216in}{0.767035in}}%
\pgfpathlineto{\pgfqpoint{1.373880in}{0.732647in}}%
\pgfpathlineto{\pgfqpoint{1.459905in}{0.697560in}}%
\pgfpathlineto{\pgfqpoint{1.546396in}{0.762117in}}%
\pgfpathlineto{\pgfqpoint{1.631151in}{0.708791in}}%
\pgfpathlineto{\pgfqpoint{1.717283in}{0.731041in}}%
\pgfpathlineto{\pgfqpoint{1.803300in}{0.792758in}}%
\pgfpathlineto{\pgfqpoint{1.888208in}{0.715685in}}%
\pgfpathlineto{\pgfqpoint{1.973986in}{0.702570in}}%
\pgfpathlineto{\pgfqpoint{2.061275in}{0.666504in}}%
\pgfpathlineto{\pgfqpoint{2.146813in}{0.689132in}}%
\pgfpathlineto{\pgfqpoint{2.232356in}{0.708976in}}%
\pgfpathlineto{\pgfqpoint{2.325145in}{0.659507in}}%
\pgfusepath{stroke}%
\end{pgfscope}%
\begin{pgfscope}%
\pgfsetbuttcap%
\pgfsetroundjoin%
\definecolor{currentfill}{rgb}{0.870588,0.560784,0.019608}%
\pgfsetfillcolor{currentfill}%
\pgfsetlinewidth{0.752812pt}%
\definecolor{currentstroke}{rgb}{1.000000,1.000000,1.000000}%
\pgfsetstrokecolor{currentstroke}%
\pgfsetdash{}{0pt}%
\pgfsys@defobject{currentmarker}{\pgfqpoint{-0.034722in}{-0.034722in}}{\pgfqpoint{0.034722in}{0.034722in}}{%
\pgfpathmoveto{\pgfqpoint{0.000000in}{-0.034722in}}%
\pgfpathcurveto{\pgfqpoint{0.009208in}{-0.034722in}}{\pgfqpoint{0.018041in}{-0.031064in}}{\pgfqpoint{0.024552in}{-0.024552in}}%
\pgfpathcurveto{\pgfqpoint{0.031064in}{-0.018041in}}{\pgfqpoint{0.034722in}{-0.009208in}}{\pgfqpoint{0.034722in}{0.000000in}}%
\pgfpathcurveto{\pgfqpoint{0.034722in}{0.009208in}}{\pgfqpoint{0.031064in}{0.018041in}}{\pgfqpoint{0.024552in}{0.024552in}}%
\pgfpathcurveto{\pgfqpoint{0.018041in}{0.031064in}}{\pgfqpoint{0.009208in}{0.034722in}}{\pgfqpoint{0.000000in}{0.034722in}}%
\pgfpathcurveto{\pgfqpoint{-0.009208in}{0.034722in}}{\pgfqpoint{-0.018041in}{0.031064in}}{\pgfqpoint{-0.024552in}{0.024552in}}%
\pgfpathcurveto{\pgfqpoint{-0.031064in}{0.018041in}}{\pgfqpoint{-0.034722in}{0.009208in}}{\pgfqpoint{-0.034722in}{0.000000in}}%
\pgfpathcurveto{\pgfqpoint{-0.034722in}{-0.009208in}}{\pgfqpoint{-0.031064in}{-0.018041in}}{\pgfqpoint{-0.024552in}{-0.024552in}}%
\pgfpathcurveto{\pgfqpoint{-0.018041in}{-0.031064in}}{\pgfqpoint{-0.009208in}{-0.034722in}}{\pgfqpoint{0.000000in}{-0.034722in}}%
\pgfpathlineto{\pgfqpoint{0.000000in}{-0.034722in}}%
\pgfpathclose%
\pgfusepath{stroke,fill}%
}%
\begin{pgfscope}%
\pgfsys@transformshift{0.773669in}{0.684786in}%
\pgfsys@useobject{currentmarker}{}%
\end{pgfscope}%
\begin{pgfscope}%
\pgfsys@transformshift{0.859949in}{0.638408in}%
\pgfsys@useobject{currentmarker}{}%
\end{pgfscope}%
\begin{pgfscope}%
\pgfsys@transformshift{0.945274in}{0.670094in}%
\pgfsys@useobject{currentmarker}{}%
\end{pgfscope}%
\begin{pgfscope}%
\pgfsys@transformshift{1.030985in}{0.682028in}%
\pgfsys@useobject{currentmarker}{}%
\end{pgfscope}%
\begin{pgfscope}%
\pgfsys@transformshift{1.116621in}{0.682021in}%
\pgfsys@useobject{currentmarker}{}%
\end{pgfscope}%
\begin{pgfscope}%
\pgfsys@transformshift{1.202341in}{0.669268in}%
\pgfsys@useobject{currentmarker}{}%
\end{pgfscope}%
\begin{pgfscope}%
\pgfsys@transformshift{1.289216in}{0.767035in}%
\pgfsys@useobject{currentmarker}{}%
\end{pgfscope}%
\begin{pgfscope}%
\pgfsys@transformshift{1.373880in}{0.732647in}%
\pgfsys@useobject{currentmarker}{}%
\end{pgfscope}%
\begin{pgfscope}%
\pgfsys@transformshift{1.459905in}{0.697560in}%
\pgfsys@useobject{currentmarker}{}%
\end{pgfscope}%
\begin{pgfscope}%
\pgfsys@transformshift{1.546396in}{0.762117in}%
\pgfsys@useobject{currentmarker}{}%
\end{pgfscope}%
\begin{pgfscope}%
\pgfsys@transformshift{1.631151in}{0.708791in}%
\pgfsys@useobject{currentmarker}{}%
\end{pgfscope}%
\begin{pgfscope}%
\pgfsys@transformshift{1.717283in}{0.731041in}%
\pgfsys@useobject{currentmarker}{}%
\end{pgfscope}%
\begin{pgfscope}%
\pgfsys@transformshift{1.803300in}{0.792758in}%
\pgfsys@useobject{currentmarker}{}%
\end{pgfscope}%
\begin{pgfscope}%
\pgfsys@transformshift{1.888208in}{0.715685in}%
\pgfsys@useobject{currentmarker}{}%
\end{pgfscope}%
\begin{pgfscope}%
\pgfsys@transformshift{1.973986in}{0.702570in}%
\pgfsys@useobject{currentmarker}{}%
\end{pgfscope}%
\begin{pgfscope}%
\pgfsys@transformshift{2.061275in}{0.666504in}%
\pgfsys@useobject{currentmarker}{}%
\end{pgfscope}%
\begin{pgfscope}%
\pgfsys@transformshift{2.146813in}{0.689132in}%
\pgfsys@useobject{currentmarker}{}%
\end{pgfscope}%
\begin{pgfscope}%
\pgfsys@transformshift{2.232356in}{0.708976in}%
\pgfsys@useobject{currentmarker}{}%
\end{pgfscope}%
\begin{pgfscope}%
\pgfsys@transformshift{2.325145in}{0.659507in}%
\pgfsys@useobject{currentmarker}{}%
\end{pgfscope}%
\end{pgfscope}%
\begin{pgfscope}%
\pgfsetroundcap%
\pgfsetroundjoin%
\pgfsetlinewidth{1.003750pt}%
\definecolor{currentstroke}{rgb}{0.007843,0.619608,0.450980}%
\pgfsetstrokecolor{currentstroke}%
\pgfsetdash{}{0pt}%
\pgfpathmoveto{\pgfqpoint{0.773669in}{0.889528in}}%
\pgfpathlineto{\pgfqpoint{0.859949in}{0.739647in}}%
\pgfpathlineto{\pgfqpoint{0.945274in}{0.791545in}}%
\pgfpathlineto{\pgfqpoint{1.030985in}{0.822840in}}%
\pgfpathlineto{\pgfqpoint{1.116621in}{0.817002in}}%
\pgfpathlineto{\pgfqpoint{1.202341in}{0.833371in}}%
\pgfpathlineto{\pgfqpoint{1.289216in}{1.001852in}}%
\pgfpathlineto{\pgfqpoint{1.373880in}{0.956574in}}%
\pgfpathlineto{\pgfqpoint{1.459905in}{0.851276in}}%
\pgfpathlineto{\pgfqpoint{1.546396in}{0.984260in}}%
\pgfpathlineto{\pgfqpoint{1.631151in}{0.881118in}}%
\pgfpathlineto{\pgfqpoint{1.717283in}{0.909652in}}%
\pgfpathlineto{\pgfqpoint{1.803300in}{1.110092in}}%
\pgfpathlineto{\pgfqpoint{1.888208in}{0.969226in}}%
\pgfpathlineto{\pgfqpoint{1.973986in}{0.982268in}}%
\pgfpathlineto{\pgfqpoint{2.061275in}{0.897722in}}%
\pgfpathlineto{\pgfqpoint{2.146813in}{0.841341in}}%
\pgfpathlineto{\pgfqpoint{2.232356in}{1.019793in}}%
\pgfpathlineto{\pgfqpoint{2.325145in}{0.949551in}}%
\pgfpathlineto{\pgfqpoint{2.396422in}{0.932488in}}%
\pgfpathlineto{\pgfqpoint{2.487345in}{0.901021in}}%
\pgfpathlineto{\pgfqpoint{2.581905in}{0.825220in}}%
\pgfpathlineto{\pgfqpoint{2.659740in}{0.779935in}}%
\pgfpathlineto{\pgfqpoint{2.737398in}{0.817912in}}%
\pgfpathlineto{\pgfqpoint{2.829387in}{0.810678in}}%
\pgfpathlineto{\pgfqpoint{3.173405in}{0.851916in}}%
\pgfusepath{stroke}%
\end{pgfscope}%
\begin{pgfscope}%
\pgfsetbuttcap%
\pgfsetroundjoin%
\definecolor{currentfill}{rgb}{0.007843,0.619608,0.450980}%
\pgfsetfillcolor{currentfill}%
\pgfsetlinewidth{0.752812pt}%
\definecolor{currentstroke}{rgb}{1.000000,1.000000,1.000000}%
\pgfsetstrokecolor{currentstroke}%
\pgfsetdash{}{0pt}%
\pgfsys@defobject{currentmarker}{\pgfqpoint{-0.034722in}{-0.034722in}}{\pgfqpoint{0.034722in}{0.034722in}}{%
\pgfpathmoveto{\pgfqpoint{0.000000in}{-0.034722in}}%
\pgfpathcurveto{\pgfqpoint{0.009208in}{-0.034722in}}{\pgfqpoint{0.018041in}{-0.031064in}}{\pgfqpoint{0.024552in}{-0.024552in}}%
\pgfpathcurveto{\pgfqpoint{0.031064in}{-0.018041in}}{\pgfqpoint{0.034722in}{-0.009208in}}{\pgfqpoint{0.034722in}{0.000000in}}%
\pgfpathcurveto{\pgfqpoint{0.034722in}{0.009208in}}{\pgfqpoint{0.031064in}{0.018041in}}{\pgfqpoint{0.024552in}{0.024552in}}%
\pgfpathcurveto{\pgfqpoint{0.018041in}{0.031064in}}{\pgfqpoint{0.009208in}{0.034722in}}{\pgfqpoint{0.000000in}{0.034722in}}%
\pgfpathcurveto{\pgfqpoint{-0.009208in}{0.034722in}}{\pgfqpoint{-0.018041in}{0.031064in}}{\pgfqpoint{-0.024552in}{0.024552in}}%
\pgfpathcurveto{\pgfqpoint{-0.031064in}{0.018041in}}{\pgfqpoint{-0.034722in}{0.009208in}}{\pgfqpoint{-0.034722in}{0.000000in}}%
\pgfpathcurveto{\pgfqpoint{-0.034722in}{-0.009208in}}{\pgfqpoint{-0.031064in}{-0.018041in}}{\pgfqpoint{-0.024552in}{-0.024552in}}%
\pgfpathcurveto{\pgfqpoint{-0.018041in}{-0.031064in}}{\pgfqpoint{-0.009208in}{-0.034722in}}{\pgfqpoint{0.000000in}{-0.034722in}}%
\pgfpathlineto{\pgfqpoint{0.000000in}{-0.034722in}}%
\pgfpathclose%
\pgfusepath{stroke,fill}%
}%
\begin{pgfscope}%
\pgfsys@transformshift{0.773669in}{0.889528in}%
\pgfsys@useobject{currentmarker}{}%
\end{pgfscope}%
\begin{pgfscope}%
\pgfsys@transformshift{0.859949in}{0.739647in}%
\pgfsys@useobject{currentmarker}{}%
\end{pgfscope}%
\begin{pgfscope}%
\pgfsys@transformshift{0.945274in}{0.791545in}%
\pgfsys@useobject{currentmarker}{}%
\end{pgfscope}%
\begin{pgfscope}%
\pgfsys@transformshift{1.030985in}{0.822840in}%
\pgfsys@useobject{currentmarker}{}%
\end{pgfscope}%
\begin{pgfscope}%
\pgfsys@transformshift{1.116621in}{0.817002in}%
\pgfsys@useobject{currentmarker}{}%
\end{pgfscope}%
\begin{pgfscope}%
\pgfsys@transformshift{1.202341in}{0.833371in}%
\pgfsys@useobject{currentmarker}{}%
\end{pgfscope}%
\begin{pgfscope}%
\pgfsys@transformshift{1.289216in}{1.001852in}%
\pgfsys@useobject{currentmarker}{}%
\end{pgfscope}%
\begin{pgfscope}%
\pgfsys@transformshift{1.373880in}{0.956574in}%
\pgfsys@useobject{currentmarker}{}%
\end{pgfscope}%
\begin{pgfscope}%
\pgfsys@transformshift{1.459905in}{0.851276in}%
\pgfsys@useobject{currentmarker}{}%
\end{pgfscope}%
\begin{pgfscope}%
\pgfsys@transformshift{1.546396in}{0.984260in}%
\pgfsys@useobject{currentmarker}{}%
\end{pgfscope}%
\begin{pgfscope}%
\pgfsys@transformshift{1.631151in}{0.881118in}%
\pgfsys@useobject{currentmarker}{}%
\end{pgfscope}%
\begin{pgfscope}%
\pgfsys@transformshift{1.717283in}{0.909652in}%
\pgfsys@useobject{currentmarker}{}%
\end{pgfscope}%
\begin{pgfscope}%
\pgfsys@transformshift{1.803300in}{1.110092in}%
\pgfsys@useobject{currentmarker}{}%
\end{pgfscope}%
\begin{pgfscope}%
\pgfsys@transformshift{1.888208in}{0.969226in}%
\pgfsys@useobject{currentmarker}{}%
\end{pgfscope}%
\begin{pgfscope}%
\pgfsys@transformshift{1.973986in}{0.982268in}%
\pgfsys@useobject{currentmarker}{}%
\end{pgfscope}%
\begin{pgfscope}%
\pgfsys@transformshift{2.061275in}{0.897722in}%
\pgfsys@useobject{currentmarker}{}%
\end{pgfscope}%
\begin{pgfscope}%
\pgfsys@transformshift{2.146813in}{0.841341in}%
\pgfsys@useobject{currentmarker}{}%
\end{pgfscope}%
\begin{pgfscope}%
\pgfsys@transformshift{2.232356in}{1.019793in}%
\pgfsys@useobject{currentmarker}{}%
\end{pgfscope}%
\begin{pgfscope}%
\pgfsys@transformshift{2.325145in}{0.949551in}%
\pgfsys@useobject{currentmarker}{}%
\end{pgfscope}%
\begin{pgfscope}%
\pgfsys@transformshift{2.396422in}{0.932488in}%
\pgfsys@useobject{currentmarker}{}%
\end{pgfscope}%
\begin{pgfscope}%
\pgfsys@transformshift{2.487345in}{0.901021in}%
\pgfsys@useobject{currentmarker}{}%
\end{pgfscope}%
\begin{pgfscope}%
\pgfsys@transformshift{2.581905in}{0.825220in}%
\pgfsys@useobject{currentmarker}{}%
\end{pgfscope}%
\begin{pgfscope}%
\pgfsys@transformshift{2.659740in}{0.779935in}%
\pgfsys@useobject{currentmarker}{}%
\end{pgfscope}%
\begin{pgfscope}%
\pgfsys@transformshift{2.737398in}{0.817912in}%
\pgfsys@useobject{currentmarker}{}%
\end{pgfscope}%
\begin{pgfscope}%
\pgfsys@transformshift{2.829387in}{0.810678in}%
\pgfsys@useobject{currentmarker}{}%
\end{pgfscope}%
\begin{pgfscope}%
\pgfsys@transformshift{3.173405in}{0.851916in}%
\pgfsys@useobject{currentmarker}{}%
\end{pgfscope}%
\end{pgfscope}%
\begin{pgfscope}%
\pgfsetroundcap%
\pgfsetroundjoin%
\pgfsetlinewidth{1.003750pt}%
\definecolor{currentstroke}{rgb}{0.835294,0.368627,0.000000}%
\pgfsetstrokecolor{currentstroke}%
\pgfsetdash{}{0pt}%
\pgfpathmoveto{\pgfqpoint{0.773669in}{1.079028in}}%
\pgfpathlineto{\pgfqpoint{0.859949in}{0.889295in}}%
\pgfpathlineto{\pgfqpoint{0.945274in}{0.968824in}}%
\pgfpathlineto{\pgfqpoint{1.030985in}{1.016385in}}%
\pgfpathlineto{\pgfqpoint{1.116621in}{0.985539in}}%
\pgfpathlineto{\pgfqpoint{1.202341in}{0.998000in}}%
\pgfpathlineto{\pgfqpoint{1.289216in}{1.187737in}}%
\pgfpathlineto{\pgfqpoint{1.373880in}{1.176275in}}%
\pgfpathlineto{\pgfqpoint{1.459905in}{1.012306in}}%
\pgfpathlineto{\pgfqpoint{1.546396in}{1.248341in}}%
\pgfpathlineto{\pgfqpoint{1.631151in}{1.054038in}}%
\pgfpathlineto{\pgfqpoint{1.717283in}{1.117672in}}%
\pgfpathlineto{\pgfqpoint{1.803300in}{1.376526in}}%
\pgfpathlineto{\pgfqpoint{1.888208in}{1.159232in}}%
\pgfpathlineto{\pgfqpoint{1.973986in}{1.287958in}}%
\pgfpathlineto{\pgfqpoint{2.061275in}{1.151113in}}%
\pgfpathlineto{\pgfqpoint{2.146813in}{0.993309in}}%
\pgfpathlineto{\pgfqpoint{2.232356in}{1.329906in}}%
\pgfpathlineto{\pgfqpoint{2.325145in}{1.187072in}}%
\pgfpathlineto{\pgfqpoint{2.396422in}{1.158062in}}%
\pgfpathlineto{\pgfqpoint{2.487345in}{1.107290in}}%
\pgfpathlineto{\pgfqpoint{2.581905in}{1.008906in}}%
\pgfpathlineto{\pgfqpoint{2.659740in}{0.939243in}}%
\pgfpathlineto{\pgfqpoint{2.737398in}{0.995147in}}%
\pgfpathlineto{\pgfqpoint{2.829387in}{0.975821in}}%
\pgfpathlineto{\pgfqpoint{3.173405in}{1.048351in}}%
\pgfusepath{stroke}%
\end{pgfscope}%
\begin{pgfscope}%
\pgfsetbuttcap%
\pgfsetroundjoin%
\definecolor{currentfill}{rgb}{0.835294,0.368627,0.000000}%
\pgfsetfillcolor{currentfill}%
\pgfsetlinewidth{0.752812pt}%
\definecolor{currentstroke}{rgb}{1.000000,1.000000,1.000000}%
\pgfsetstrokecolor{currentstroke}%
\pgfsetdash{}{0pt}%
\pgfsys@defobject{currentmarker}{\pgfqpoint{-0.034722in}{-0.034722in}}{\pgfqpoint{0.034722in}{0.034722in}}{%
\pgfpathmoveto{\pgfqpoint{0.000000in}{-0.034722in}}%
\pgfpathcurveto{\pgfqpoint{0.009208in}{-0.034722in}}{\pgfqpoint{0.018041in}{-0.031064in}}{\pgfqpoint{0.024552in}{-0.024552in}}%
\pgfpathcurveto{\pgfqpoint{0.031064in}{-0.018041in}}{\pgfqpoint{0.034722in}{-0.009208in}}{\pgfqpoint{0.034722in}{0.000000in}}%
\pgfpathcurveto{\pgfqpoint{0.034722in}{0.009208in}}{\pgfqpoint{0.031064in}{0.018041in}}{\pgfqpoint{0.024552in}{0.024552in}}%
\pgfpathcurveto{\pgfqpoint{0.018041in}{0.031064in}}{\pgfqpoint{0.009208in}{0.034722in}}{\pgfqpoint{0.000000in}{0.034722in}}%
\pgfpathcurveto{\pgfqpoint{-0.009208in}{0.034722in}}{\pgfqpoint{-0.018041in}{0.031064in}}{\pgfqpoint{-0.024552in}{0.024552in}}%
\pgfpathcurveto{\pgfqpoint{-0.031064in}{0.018041in}}{\pgfqpoint{-0.034722in}{0.009208in}}{\pgfqpoint{-0.034722in}{0.000000in}}%
\pgfpathcurveto{\pgfqpoint{-0.034722in}{-0.009208in}}{\pgfqpoint{-0.031064in}{-0.018041in}}{\pgfqpoint{-0.024552in}{-0.024552in}}%
\pgfpathcurveto{\pgfqpoint{-0.018041in}{-0.031064in}}{\pgfqpoint{-0.009208in}{-0.034722in}}{\pgfqpoint{0.000000in}{-0.034722in}}%
\pgfpathlineto{\pgfqpoint{0.000000in}{-0.034722in}}%
\pgfpathclose%
\pgfusepath{stroke,fill}%
}%
\begin{pgfscope}%
\pgfsys@transformshift{0.773669in}{1.079028in}%
\pgfsys@useobject{currentmarker}{}%
\end{pgfscope}%
\begin{pgfscope}%
\pgfsys@transformshift{0.859949in}{0.889295in}%
\pgfsys@useobject{currentmarker}{}%
\end{pgfscope}%
\begin{pgfscope}%
\pgfsys@transformshift{0.945274in}{0.968824in}%
\pgfsys@useobject{currentmarker}{}%
\end{pgfscope}%
\begin{pgfscope}%
\pgfsys@transformshift{1.030985in}{1.016385in}%
\pgfsys@useobject{currentmarker}{}%
\end{pgfscope}%
\begin{pgfscope}%
\pgfsys@transformshift{1.116621in}{0.985539in}%
\pgfsys@useobject{currentmarker}{}%
\end{pgfscope}%
\begin{pgfscope}%
\pgfsys@transformshift{1.202341in}{0.998000in}%
\pgfsys@useobject{currentmarker}{}%
\end{pgfscope}%
\begin{pgfscope}%
\pgfsys@transformshift{1.289216in}{1.187737in}%
\pgfsys@useobject{currentmarker}{}%
\end{pgfscope}%
\begin{pgfscope}%
\pgfsys@transformshift{1.373880in}{1.176275in}%
\pgfsys@useobject{currentmarker}{}%
\end{pgfscope}%
\begin{pgfscope}%
\pgfsys@transformshift{1.459905in}{1.012306in}%
\pgfsys@useobject{currentmarker}{}%
\end{pgfscope}%
\begin{pgfscope}%
\pgfsys@transformshift{1.546396in}{1.248341in}%
\pgfsys@useobject{currentmarker}{}%
\end{pgfscope}%
\begin{pgfscope}%
\pgfsys@transformshift{1.631151in}{1.054038in}%
\pgfsys@useobject{currentmarker}{}%
\end{pgfscope}%
\begin{pgfscope}%
\pgfsys@transformshift{1.717283in}{1.117672in}%
\pgfsys@useobject{currentmarker}{}%
\end{pgfscope}%
\begin{pgfscope}%
\pgfsys@transformshift{1.803300in}{1.376526in}%
\pgfsys@useobject{currentmarker}{}%
\end{pgfscope}%
\begin{pgfscope}%
\pgfsys@transformshift{1.888208in}{1.159232in}%
\pgfsys@useobject{currentmarker}{}%
\end{pgfscope}%
\begin{pgfscope}%
\pgfsys@transformshift{1.973986in}{1.287958in}%
\pgfsys@useobject{currentmarker}{}%
\end{pgfscope}%
\begin{pgfscope}%
\pgfsys@transformshift{2.061275in}{1.151113in}%
\pgfsys@useobject{currentmarker}{}%
\end{pgfscope}%
\begin{pgfscope}%
\pgfsys@transformshift{2.146813in}{0.993309in}%
\pgfsys@useobject{currentmarker}{}%
\end{pgfscope}%
\begin{pgfscope}%
\pgfsys@transformshift{2.232356in}{1.329906in}%
\pgfsys@useobject{currentmarker}{}%
\end{pgfscope}%
\begin{pgfscope}%
\pgfsys@transformshift{2.325145in}{1.187072in}%
\pgfsys@useobject{currentmarker}{}%
\end{pgfscope}%
\begin{pgfscope}%
\pgfsys@transformshift{2.396422in}{1.158062in}%
\pgfsys@useobject{currentmarker}{}%
\end{pgfscope}%
\begin{pgfscope}%
\pgfsys@transformshift{2.487345in}{1.107290in}%
\pgfsys@useobject{currentmarker}{}%
\end{pgfscope}%
\begin{pgfscope}%
\pgfsys@transformshift{2.581905in}{1.008906in}%
\pgfsys@useobject{currentmarker}{}%
\end{pgfscope}%
\begin{pgfscope}%
\pgfsys@transformshift{2.659740in}{0.939243in}%
\pgfsys@useobject{currentmarker}{}%
\end{pgfscope}%
\begin{pgfscope}%
\pgfsys@transformshift{2.737398in}{0.995147in}%
\pgfsys@useobject{currentmarker}{}%
\end{pgfscope}%
\begin{pgfscope}%
\pgfsys@transformshift{2.829387in}{0.975821in}%
\pgfsys@useobject{currentmarker}{}%
\end{pgfscope}%
\begin{pgfscope}%
\pgfsys@transformshift{3.173405in}{1.048351in}%
\pgfsys@useobject{currentmarker}{}%
\end{pgfscope}%
\end{pgfscope}%
\begin{pgfscope}%
\pgfsetroundcap%
\pgfsetroundjoin%
\pgfsetlinewidth{1.003750pt}%
\definecolor{currentstroke}{rgb}{0.800000,0.470588,0.737255}%
\pgfsetstrokecolor{currentstroke}%
\pgfsetdash{}{0pt}%
\pgfpathmoveto{\pgfqpoint{0.773669in}{1.299161in}}%
\pgfpathlineto{\pgfqpoint{0.859949in}{1.099905in}}%
\pgfpathlineto{\pgfqpoint{0.945274in}{1.178101in}}%
\pgfpathlineto{\pgfqpoint{1.030985in}{1.228193in}}%
\pgfpathlineto{\pgfqpoint{1.116621in}{1.177491in}}%
\pgfpathlineto{\pgfqpoint{1.202341in}{1.161271in}}%
\pgfpathlineto{\pgfqpoint{1.289216in}{1.398795in}}%
\pgfpathlineto{\pgfqpoint{1.373880in}{1.448140in}}%
\pgfpathlineto{\pgfqpoint{1.459905in}{1.231918in}}%
\pgfpathlineto{\pgfqpoint{1.546396in}{1.575234in}}%
\pgfpathlineto{\pgfqpoint{1.631151in}{1.298401in}}%
\pgfpathlineto{\pgfqpoint{1.717283in}{1.378122in}}%
\pgfpathlineto{\pgfqpoint{1.803300in}{1.709150in}}%
\pgfpathlineto{\pgfqpoint{1.888208in}{1.419325in}}%
\pgfpathlineto{\pgfqpoint{1.973986in}{1.744791in}}%
\pgfpathlineto{\pgfqpoint{2.061275in}{1.435122in}}%
\pgfpathlineto{\pgfqpoint{2.146813in}{1.272072in}}%
\pgfpathlineto{\pgfqpoint{2.232356in}{1.728636in}}%
\pgfpathlineto{\pgfqpoint{2.325145in}{1.487832in}}%
\pgfpathlineto{\pgfqpoint{2.396422in}{1.551931in}}%
\pgfpathlineto{\pgfqpoint{2.487345in}{1.389638in}}%
\pgfpathlineto{\pgfqpoint{2.581905in}{1.413774in}}%
\pgfpathlineto{\pgfqpoint{2.659740in}{1.153565in}}%
\pgfpathlineto{\pgfqpoint{2.737398in}{1.253151in}}%
\pgfpathlineto{\pgfqpoint{2.829387in}{1.235487in}}%
\pgfpathlineto{\pgfqpoint{3.173405in}{1.484129in}}%
\pgfpathlineto{\pgfqpoint{3.354136in}{1.410313in}}%
\pgfpathlineto{\pgfqpoint{3.516938in}{1.461747in}}%
\pgfpathlineto{\pgfqpoint{3.686501in}{1.514170in}}%
\pgfpathlineto{\pgfqpoint{3.866734in}{1.715568in}}%
\pgfpathlineto{\pgfqpoint{4.038858in}{1.786189in}}%
\pgfusepath{stroke}%
\end{pgfscope}%
\begin{pgfscope}%
\pgfsetbuttcap%
\pgfsetroundjoin%
\definecolor{currentfill}{rgb}{0.800000,0.470588,0.737255}%
\pgfsetfillcolor{currentfill}%
\pgfsetlinewidth{0.752812pt}%
\definecolor{currentstroke}{rgb}{1.000000,1.000000,1.000000}%
\pgfsetstrokecolor{currentstroke}%
\pgfsetdash{}{0pt}%
\pgfsys@defobject{currentmarker}{\pgfqpoint{-0.034722in}{-0.034722in}}{\pgfqpoint{0.034722in}{0.034722in}}{%
\pgfpathmoveto{\pgfqpoint{0.000000in}{-0.034722in}}%
\pgfpathcurveto{\pgfqpoint{0.009208in}{-0.034722in}}{\pgfqpoint{0.018041in}{-0.031064in}}{\pgfqpoint{0.024552in}{-0.024552in}}%
\pgfpathcurveto{\pgfqpoint{0.031064in}{-0.018041in}}{\pgfqpoint{0.034722in}{-0.009208in}}{\pgfqpoint{0.034722in}{0.000000in}}%
\pgfpathcurveto{\pgfqpoint{0.034722in}{0.009208in}}{\pgfqpoint{0.031064in}{0.018041in}}{\pgfqpoint{0.024552in}{0.024552in}}%
\pgfpathcurveto{\pgfqpoint{0.018041in}{0.031064in}}{\pgfqpoint{0.009208in}{0.034722in}}{\pgfqpoint{0.000000in}{0.034722in}}%
\pgfpathcurveto{\pgfqpoint{-0.009208in}{0.034722in}}{\pgfqpoint{-0.018041in}{0.031064in}}{\pgfqpoint{-0.024552in}{0.024552in}}%
\pgfpathcurveto{\pgfqpoint{-0.031064in}{0.018041in}}{\pgfqpoint{-0.034722in}{0.009208in}}{\pgfqpoint{-0.034722in}{0.000000in}}%
\pgfpathcurveto{\pgfqpoint{-0.034722in}{-0.009208in}}{\pgfqpoint{-0.031064in}{-0.018041in}}{\pgfqpoint{-0.024552in}{-0.024552in}}%
\pgfpathcurveto{\pgfqpoint{-0.018041in}{-0.031064in}}{\pgfqpoint{-0.009208in}{-0.034722in}}{\pgfqpoint{0.000000in}{-0.034722in}}%
\pgfpathlineto{\pgfqpoint{0.000000in}{-0.034722in}}%
\pgfpathclose%
\pgfusepath{stroke,fill}%
}%
\begin{pgfscope}%
\pgfsys@transformshift{0.773669in}{1.299161in}%
\pgfsys@useobject{currentmarker}{}%
\end{pgfscope}%
\begin{pgfscope}%
\pgfsys@transformshift{0.859949in}{1.099905in}%
\pgfsys@useobject{currentmarker}{}%
\end{pgfscope}%
\begin{pgfscope}%
\pgfsys@transformshift{0.945274in}{1.178101in}%
\pgfsys@useobject{currentmarker}{}%
\end{pgfscope}%
\begin{pgfscope}%
\pgfsys@transformshift{1.030985in}{1.228193in}%
\pgfsys@useobject{currentmarker}{}%
\end{pgfscope}%
\begin{pgfscope}%
\pgfsys@transformshift{1.116621in}{1.177491in}%
\pgfsys@useobject{currentmarker}{}%
\end{pgfscope}%
\begin{pgfscope}%
\pgfsys@transformshift{1.202341in}{1.161271in}%
\pgfsys@useobject{currentmarker}{}%
\end{pgfscope}%
\begin{pgfscope}%
\pgfsys@transformshift{1.289216in}{1.398795in}%
\pgfsys@useobject{currentmarker}{}%
\end{pgfscope}%
\begin{pgfscope}%
\pgfsys@transformshift{1.373880in}{1.448140in}%
\pgfsys@useobject{currentmarker}{}%
\end{pgfscope}%
\begin{pgfscope}%
\pgfsys@transformshift{1.459905in}{1.231918in}%
\pgfsys@useobject{currentmarker}{}%
\end{pgfscope}%
\begin{pgfscope}%
\pgfsys@transformshift{1.546396in}{1.575234in}%
\pgfsys@useobject{currentmarker}{}%
\end{pgfscope}%
\begin{pgfscope}%
\pgfsys@transformshift{1.631151in}{1.298401in}%
\pgfsys@useobject{currentmarker}{}%
\end{pgfscope}%
\begin{pgfscope}%
\pgfsys@transformshift{1.717283in}{1.378122in}%
\pgfsys@useobject{currentmarker}{}%
\end{pgfscope}%
\begin{pgfscope}%
\pgfsys@transformshift{1.803300in}{1.709150in}%
\pgfsys@useobject{currentmarker}{}%
\end{pgfscope}%
\begin{pgfscope}%
\pgfsys@transformshift{1.888208in}{1.419325in}%
\pgfsys@useobject{currentmarker}{}%
\end{pgfscope}%
\begin{pgfscope}%
\pgfsys@transformshift{1.973986in}{1.744791in}%
\pgfsys@useobject{currentmarker}{}%
\end{pgfscope}%
\begin{pgfscope}%
\pgfsys@transformshift{2.061275in}{1.435122in}%
\pgfsys@useobject{currentmarker}{}%
\end{pgfscope}%
\begin{pgfscope}%
\pgfsys@transformshift{2.146813in}{1.272072in}%
\pgfsys@useobject{currentmarker}{}%
\end{pgfscope}%
\begin{pgfscope}%
\pgfsys@transformshift{2.232356in}{1.728636in}%
\pgfsys@useobject{currentmarker}{}%
\end{pgfscope}%
\begin{pgfscope}%
\pgfsys@transformshift{2.325145in}{1.487832in}%
\pgfsys@useobject{currentmarker}{}%
\end{pgfscope}%
\begin{pgfscope}%
\pgfsys@transformshift{2.396422in}{1.551931in}%
\pgfsys@useobject{currentmarker}{}%
\end{pgfscope}%
\begin{pgfscope}%
\pgfsys@transformshift{2.487345in}{1.389638in}%
\pgfsys@useobject{currentmarker}{}%
\end{pgfscope}%
\begin{pgfscope}%
\pgfsys@transformshift{2.581905in}{1.413774in}%
\pgfsys@useobject{currentmarker}{}%
\end{pgfscope}%
\begin{pgfscope}%
\pgfsys@transformshift{2.659740in}{1.153565in}%
\pgfsys@useobject{currentmarker}{}%
\end{pgfscope}%
\begin{pgfscope}%
\pgfsys@transformshift{2.737398in}{1.253151in}%
\pgfsys@useobject{currentmarker}{}%
\end{pgfscope}%
\begin{pgfscope}%
\pgfsys@transformshift{2.829387in}{1.235487in}%
\pgfsys@useobject{currentmarker}{}%
\end{pgfscope}%
\begin{pgfscope}%
\pgfsys@transformshift{3.173405in}{1.484129in}%
\pgfsys@useobject{currentmarker}{}%
\end{pgfscope}%
\begin{pgfscope}%
\pgfsys@transformshift{3.354136in}{1.410313in}%
\pgfsys@useobject{currentmarker}{}%
\end{pgfscope}%
\begin{pgfscope}%
\pgfsys@transformshift{3.516938in}{1.461747in}%
\pgfsys@useobject{currentmarker}{}%
\end{pgfscope}%
\begin{pgfscope}%
\pgfsys@transformshift{3.686501in}{1.514170in}%
\pgfsys@useobject{currentmarker}{}%
\end{pgfscope}%
\begin{pgfscope}%
\pgfsys@transformshift{3.866734in}{1.715568in}%
\pgfsys@useobject{currentmarker}{}%
\end{pgfscope}%
\begin{pgfscope}%
\pgfsys@transformshift{4.038858in}{1.786189in}%
\pgfsys@useobject{currentmarker}{}%
\end{pgfscope}%
\end{pgfscope}%
\begin{pgfscope}%
\pgfsetroundcap%
\pgfsetroundjoin%
\pgfsetlinewidth{1.003750pt}%
\definecolor{currentstroke}{rgb}{0.792157,0.568627,0.380392}%
\pgfsetstrokecolor{currentstroke}%
\pgfsetdash{}{0pt}%
\pgfpathmoveto{\pgfqpoint{0.773669in}{1.509365in}}%
\pgfpathlineto{\pgfqpoint{0.859949in}{1.276631in}}%
\pgfpathlineto{\pgfqpoint{0.945274in}{1.371401in}}%
\pgfpathlineto{\pgfqpoint{1.030985in}{1.427379in}}%
\pgfpathlineto{\pgfqpoint{1.116621in}{1.372302in}}%
\pgfpathlineto{\pgfqpoint{1.202341in}{1.341366in}}%
\pgfpathlineto{\pgfqpoint{1.289216in}{1.606382in}}%
\pgfpathlineto{\pgfqpoint{1.373880in}{1.645674in}}%
\pgfpathlineto{\pgfqpoint{1.459905in}{1.431689in}}%
\pgfpathlineto{\pgfqpoint{1.546396in}{1.803497in}}%
\pgfpathlineto{\pgfqpoint{1.631151in}{1.502609in}}%
\pgfpathlineto{\pgfqpoint{1.717283in}{1.597975in}}%
\pgfpathlineto{\pgfqpoint{1.803300in}{2.195449in}}%
\pgfpathlineto{\pgfqpoint{1.888208in}{1.711894in}}%
\pgfpathlineto{\pgfqpoint{1.973986in}{2.160584in}}%
\pgfpathlineto{\pgfqpoint{2.061275in}{1.680490in}}%
\pgfpathlineto{\pgfqpoint{2.146813in}{1.650703in}}%
\pgfpathlineto{\pgfqpoint{2.232356in}{2.231550in}}%
\pgfpathlineto{\pgfqpoint{2.325145in}{1.780259in}}%
\pgfpathlineto{\pgfqpoint{2.396422in}{1.987663in}}%
\pgfpathlineto{\pgfqpoint{2.487345in}{1.738791in}}%
\pgfpathlineto{\pgfqpoint{2.581905in}{1.802513in}}%
\pgfpathlineto{\pgfqpoint{2.659740in}{1.383636in}}%
\pgfpathlineto{\pgfqpoint{2.737398in}{1.726343in}}%
\pgfpathlineto{\pgfqpoint{2.829387in}{1.613232in}}%
\pgfpathlineto{\pgfqpoint{3.173405in}{1.909608in}}%
\pgfpathlineto{\pgfqpoint{3.354136in}{1.933247in}}%
\pgfpathlineto{\pgfqpoint{3.516938in}{1.889352in}}%
\pgfpathlineto{\pgfqpoint{3.686501in}{1.947231in}}%
\pgfpathlineto{\pgfqpoint{3.866734in}{2.225236in}}%
\pgfpathlineto{\pgfqpoint{4.038858in}{2.318552in}}%
\pgfpathlineto{\pgfqpoint{4.199306in}{2.024513in}}%
\pgfpathlineto{\pgfqpoint{4.371578in}{1.897781in}}%
\pgfpathlineto{\pgfqpoint{4.553979in}{1.920362in}}%
\pgfpathlineto{\pgfqpoint{4.889440in}{1.922724in}}%
\pgfusepath{stroke}%
\end{pgfscope}%
\begin{pgfscope}%
\pgfsetbuttcap%
\pgfsetroundjoin%
\definecolor{currentfill}{rgb}{0.792157,0.568627,0.380392}%
\pgfsetfillcolor{currentfill}%
\pgfsetlinewidth{0.752812pt}%
\definecolor{currentstroke}{rgb}{1.000000,1.000000,1.000000}%
\pgfsetstrokecolor{currentstroke}%
\pgfsetdash{}{0pt}%
\pgfsys@defobject{currentmarker}{\pgfqpoint{-0.034722in}{-0.034722in}}{\pgfqpoint{0.034722in}{0.034722in}}{%
\pgfpathmoveto{\pgfqpoint{0.000000in}{-0.034722in}}%
\pgfpathcurveto{\pgfqpoint{0.009208in}{-0.034722in}}{\pgfqpoint{0.018041in}{-0.031064in}}{\pgfqpoint{0.024552in}{-0.024552in}}%
\pgfpathcurveto{\pgfqpoint{0.031064in}{-0.018041in}}{\pgfqpoint{0.034722in}{-0.009208in}}{\pgfqpoint{0.034722in}{0.000000in}}%
\pgfpathcurveto{\pgfqpoint{0.034722in}{0.009208in}}{\pgfqpoint{0.031064in}{0.018041in}}{\pgfqpoint{0.024552in}{0.024552in}}%
\pgfpathcurveto{\pgfqpoint{0.018041in}{0.031064in}}{\pgfqpoint{0.009208in}{0.034722in}}{\pgfqpoint{0.000000in}{0.034722in}}%
\pgfpathcurveto{\pgfqpoint{-0.009208in}{0.034722in}}{\pgfqpoint{-0.018041in}{0.031064in}}{\pgfqpoint{-0.024552in}{0.024552in}}%
\pgfpathcurveto{\pgfqpoint{-0.031064in}{0.018041in}}{\pgfqpoint{-0.034722in}{0.009208in}}{\pgfqpoint{-0.034722in}{0.000000in}}%
\pgfpathcurveto{\pgfqpoint{-0.034722in}{-0.009208in}}{\pgfqpoint{-0.031064in}{-0.018041in}}{\pgfqpoint{-0.024552in}{-0.024552in}}%
\pgfpathcurveto{\pgfqpoint{-0.018041in}{-0.031064in}}{\pgfqpoint{-0.009208in}{-0.034722in}}{\pgfqpoint{0.000000in}{-0.034722in}}%
\pgfpathlineto{\pgfqpoint{0.000000in}{-0.034722in}}%
\pgfpathclose%
\pgfusepath{stroke,fill}%
}%
\begin{pgfscope}%
\pgfsys@transformshift{0.773669in}{1.509365in}%
\pgfsys@useobject{currentmarker}{}%
\end{pgfscope}%
\begin{pgfscope}%
\pgfsys@transformshift{0.859949in}{1.276631in}%
\pgfsys@useobject{currentmarker}{}%
\end{pgfscope}%
\begin{pgfscope}%
\pgfsys@transformshift{0.945274in}{1.371401in}%
\pgfsys@useobject{currentmarker}{}%
\end{pgfscope}%
\begin{pgfscope}%
\pgfsys@transformshift{1.030985in}{1.427379in}%
\pgfsys@useobject{currentmarker}{}%
\end{pgfscope}%
\begin{pgfscope}%
\pgfsys@transformshift{1.116621in}{1.372302in}%
\pgfsys@useobject{currentmarker}{}%
\end{pgfscope}%
\begin{pgfscope}%
\pgfsys@transformshift{1.202341in}{1.341366in}%
\pgfsys@useobject{currentmarker}{}%
\end{pgfscope}%
\begin{pgfscope}%
\pgfsys@transformshift{1.289216in}{1.606382in}%
\pgfsys@useobject{currentmarker}{}%
\end{pgfscope}%
\begin{pgfscope}%
\pgfsys@transformshift{1.373880in}{1.645674in}%
\pgfsys@useobject{currentmarker}{}%
\end{pgfscope}%
\begin{pgfscope}%
\pgfsys@transformshift{1.459905in}{1.431689in}%
\pgfsys@useobject{currentmarker}{}%
\end{pgfscope}%
\begin{pgfscope}%
\pgfsys@transformshift{1.546396in}{1.803497in}%
\pgfsys@useobject{currentmarker}{}%
\end{pgfscope}%
\begin{pgfscope}%
\pgfsys@transformshift{1.631151in}{1.502609in}%
\pgfsys@useobject{currentmarker}{}%
\end{pgfscope}%
\begin{pgfscope}%
\pgfsys@transformshift{1.717283in}{1.597975in}%
\pgfsys@useobject{currentmarker}{}%
\end{pgfscope}%
\begin{pgfscope}%
\pgfsys@transformshift{1.803300in}{2.195449in}%
\pgfsys@useobject{currentmarker}{}%
\end{pgfscope}%
\begin{pgfscope}%
\pgfsys@transformshift{1.888208in}{1.711894in}%
\pgfsys@useobject{currentmarker}{}%
\end{pgfscope}%
\begin{pgfscope}%
\pgfsys@transformshift{1.973986in}{2.160584in}%
\pgfsys@useobject{currentmarker}{}%
\end{pgfscope}%
\begin{pgfscope}%
\pgfsys@transformshift{2.061275in}{1.680490in}%
\pgfsys@useobject{currentmarker}{}%
\end{pgfscope}%
\begin{pgfscope}%
\pgfsys@transformshift{2.146813in}{1.650703in}%
\pgfsys@useobject{currentmarker}{}%
\end{pgfscope}%
\begin{pgfscope}%
\pgfsys@transformshift{2.232356in}{2.231550in}%
\pgfsys@useobject{currentmarker}{}%
\end{pgfscope}%
\begin{pgfscope}%
\pgfsys@transformshift{2.325145in}{1.780259in}%
\pgfsys@useobject{currentmarker}{}%
\end{pgfscope}%
\begin{pgfscope}%
\pgfsys@transformshift{2.396422in}{1.987663in}%
\pgfsys@useobject{currentmarker}{}%
\end{pgfscope}%
\begin{pgfscope}%
\pgfsys@transformshift{2.487345in}{1.738791in}%
\pgfsys@useobject{currentmarker}{}%
\end{pgfscope}%
\begin{pgfscope}%
\pgfsys@transformshift{2.581905in}{1.802513in}%
\pgfsys@useobject{currentmarker}{}%
\end{pgfscope}%
\begin{pgfscope}%
\pgfsys@transformshift{2.659740in}{1.383636in}%
\pgfsys@useobject{currentmarker}{}%
\end{pgfscope}%
\begin{pgfscope}%
\pgfsys@transformshift{2.737398in}{1.726343in}%
\pgfsys@useobject{currentmarker}{}%
\end{pgfscope}%
\begin{pgfscope}%
\pgfsys@transformshift{2.829387in}{1.613232in}%
\pgfsys@useobject{currentmarker}{}%
\end{pgfscope}%
\begin{pgfscope}%
\pgfsys@transformshift{3.173405in}{1.909608in}%
\pgfsys@useobject{currentmarker}{}%
\end{pgfscope}%
\begin{pgfscope}%
\pgfsys@transformshift{3.354136in}{1.933247in}%
\pgfsys@useobject{currentmarker}{}%
\end{pgfscope}%
\begin{pgfscope}%
\pgfsys@transformshift{3.516938in}{1.889352in}%
\pgfsys@useobject{currentmarker}{}%
\end{pgfscope}%
\begin{pgfscope}%
\pgfsys@transformshift{3.686501in}{1.947231in}%
\pgfsys@useobject{currentmarker}{}%
\end{pgfscope}%
\begin{pgfscope}%
\pgfsys@transformshift{3.866734in}{2.225236in}%
\pgfsys@useobject{currentmarker}{}%
\end{pgfscope}%
\begin{pgfscope}%
\pgfsys@transformshift{4.038858in}{2.318552in}%
\pgfsys@useobject{currentmarker}{}%
\end{pgfscope}%
\begin{pgfscope}%
\pgfsys@transformshift{4.199306in}{2.024513in}%
\pgfsys@useobject{currentmarker}{}%
\end{pgfscope}%
\begin{pgfscope}%
\pgfsys@transformshift{4.371578in}{1.897781in}%
\pgfsys@useobject{currentmarker}{}%
\end{pgfscope}%
\begin{pgfscope}%
\pgfsys@transformshift{4.553979in}{1.920362in}%
\pgfsys@useobject{currentmarker}{}%
\end{pgfscope}%
\begin{pgfscope}%
\pgfsys@transformshift{4.889440in}{1.922724in}%
\pgfsys@useobject{currentmarker}{}%
\end{pgfscope}%
\end{pgfscope}%
\end{pgfpicture}%
\makeatother%
\endgroup%

						\end{figcenter}
						\caption{Total routing times of all datasets.}
						\label{fig:eval-city-routing-details-a}
					\end{subfigure}
					\\[3ex]
					\begin{subfigure}[t]{\textwidth}
						\begin{figcenter}
							\begingroup%
\makeatletter%
\begin{pgfpicture}%
\pgfpathrectangle{\pgfpointorigin}{\pgfqpoint{6.079644in}{1.715788in}}%
\pgfusepath{use as bounding box}%
\begin{pgfscope}%
\pgfsetbuttcap%
\pgfsetmiterjoin%
\definecolor{currentfill}{rgb}{1.000000,1.000000,1.000000}%
\pgfsetfillcolor{currentfill}%
\pgfsetlinewidth{0.000000pt}%
\definecolor{currentstroke}{rgb}{1.000000,1.000000,1.000000}%
\pgfsetstrokecolor{currentstroke}%
\pgfsetdash{}{0pt}%
\pgfpathmoveto{\pgfqpoint{0.000000in}{0.000000in}}%
\pgfpathlineto{\pgfqpoint{6.079644in}{0.000000in}}%
\pgfpathlineto{\pgfqpoint{6.079644in}{1.715788in}}%
\pgfpathlineto{\pgfqpoint{0.000000in}{1.715788in}}%
\pgfpathlineto{\pgfqpoint{0.000000in}{0.000000in}}%
\pgfpathclose%
\pgfusepath{fill}%
\end{pgfscope}%
\begin{pgfscope}%
\pgfsetbuttcap%
\pgfsetmiterjoin%
\definecolor{currentfill}{rgb}{1.000000,1.000000,1.000000}%
\pgfsetfillcolor{currentfill}%
\pgfsetlinewidth{0.000000pt}%
\definecolor{currentstroke}{rgb}{0.000000,0.000000,0.000000}%
\pgfsetstrokecolor{currentstroke}%
\pgfsetstrokeopacity{0.000000}%
\pgfsetdash{}{0pt}%
\pgfpathmoveto{\pgfqpoint{0.601779in}{0.451389in}}%
\pgfpathlineto{\pgfqpoint{4.376166in}{0.451389in}}%
\pgfpathlineto{\pgfqpoint{4.376166in}{1.715788in}}%
\pgfpathlineto{\pgfqpoint{0.601779in}{1.715788in}}%
\pgfpathlineto{\pgfqpoint{0.601779in}{0.451389in}}%
\pgfpathclose%
\pgfusepath{fill}%
\end{pgfscope}%
\begin{pgfscope}%
\pgfpathrectangle{\pgfqpoint{0.601779in}{0.451389in}}{\pgfqpoint{3.774387in}{1.264399in}}%
\pgfusepath{clip}%
\pgfsetroundcap%
\pgfsetroundjoin%
\pgfsetlinewidth{1.003750pt}%
\definecolor{currentstroke}{rgb}{0.800000,0.800000,0.800000}%
\pgfsetstrokecolor{currentstroke}%
\pgfsetdash{}{0pt}%
\pgfpathmoveto{\pgfqpoint{0.601779in}{0.451389in}}%
\pgfpathlineto{\pgfqpoint{0.601779in}{1.715788in}}%
\pgfusepath{stroke}%
\end{pgfscope}%
\begin{pgfscope}%
\definecolor{textcolor}{rgb}{0.150000,0.150000,0.150000}%
\pgfsetstrokecolor{textcolor}%
\pgfsetfillcolor{textcolor}%
\pgftext[x=0.601779in,y=0.319444in,,top]{\color{textcolor}\sffamily\fontsize{9.000000}{10.800000}\selectfont 0.0}%
\end{pgfscope}%
\begin{pgfscope}%
\pgfpathrectangle{\pgfqpoint{0.601779in}{0.451389in}}{\pgfqpoint{3.774387in}{1.264399in}}%
\pgfusepath{clip}%
\pgfsetroundcap%
\pgfsetroundjoin%
\pgfsetlinewidth{1.003750pt}%
\definecolor{currentstroke}{rgb}{0.800000,0.800000,0.800000}%
\pgfsetstrokecolor{currentstroke}%
\pgfsetdash{}{0pt}%
\pgfpathmoveto{\pgfqpoint{1.321582in}{0.451389in}}%
\pgfpathlineto{\pgfqpoint{1.321582in}{1.715788in}}%
\pgfusepath{stroke}%
\end{pgfscope}%
\begin{pgfscope}%
\definecolor{textcolor}{rgb}{0.150000,0.150000,0.150000}%
\pgfsetstrokecolor{textcolor}%
\pgfsetfillcolor{textcolor}%
\pgftext[x=1.321582in,y=0.319444in,,top]{\color{textcolor}\sffamily\fontsize{9.000000}{10.800000}\selectfont 0.5}%
\end{pgfscope}%
\begin{pgfscope}%
\pgfpathrectangle{\pgfqpoint{0.601779in}{0.451389in}}{\pgfqpoint{3.774387in}{1.264399in}}%
\pgfusepath{clip}%
\pgfsetroundcap%
\pgfsetroundjoin%
\pgfsetlinewidth{1.003750pt}%
\definecolor{currentstroke}{rgb}{0.800000,0.800000,0.800000}%
\pgfsetstrokecolor{currentstroke}%
\pgfsetdash{}{0pt}%
\pgfpathmoveto{\pgfqpoint{2.041385in}{0.451389in}}%
\pgfpathlineto{\pgfqpoint{2.041385in}{1.715788in}}%
\pgfusepath{stroke}%
\end{pgfscope}%
\begin{pgfscope}%
\definecolor{textcolor}{rgb}{0.150000,0.150000,0.150000}%
\pgfsetstrokecolor{textcolor}%
\pgfsetfillcolor{textcolor}%
\pgftext[x=2.041385in,y=0.319444in,,top]{\color{textcolor}\sffamily\fontsize{9.000000}{10.800000}\selectfont 1.0}%
\end{pgfscope}%
\begin{pgfscope}%
\pgfpathrectangle{\pgfqpoint{0.601779in}{0.451389in}}{\pgfqpoint{3.774387in}{1.264399in}}%
\pgfusepath{clip}%
\pgfsetroundcap%
\pgfsetroundjoin%
\pgfsetlinewidth{1.003750pt}%
\definecolor{currentstroke}{rgb}{0.800000,0.800000,0.800000}%
\pgfsetstrokecolor{currentstroke}%
\pgfsetdash{}{0pt}%
\pgfpathmoveto{\pgfqpoint{2.761188in}{0.451389in}}%
\pgfpathlineto{\pgfqpoint{2.761188in}{1.715788in}}%
\pgfusepath{stroke}%
\end{pgfscope}%
\begin{pgfscope}%
\definecolor{textcolor}{rgb}{0.150000,0.150000,0.150000}%
\pgfsetstrokecolor{textcolor}%
\pgfsetfillcolor{textcolor}%
\pgftext[x=2.761188in,y=0.319444in,,top]{\color{textcolor}\sffamily\fontsize{9.000000}{10.800000}\selectfont 1.5}%
\end{pgfscope}%
\begin{pgfscope}%
\pgfpathrectangle{\pgfqpoint{0.601779in}{0.451389in}}{\pgfqpoint{3.774387in}{1.264399in}}%
\pgfusepath{clip}%
\pgfsetroundcap%
\pgfsetroundjoin%
\pgfsetlinewidth{1.003750pt}%
\definecolor{currentstroke}{rgb}{0.800000,0.800000,0.800000}%
\pgfsetstrokecolor{currentstroke}%
\pgfsetdash{}{0pt}%
\pgfpathmoveto{\pgfqpoint{3.480991in}{0.451389in}}%
\pgfpathlineto{\pgfqpoint{3.480991in}{1.715788in}}%
\pgfusepath{stroke}%
\end{pgfscope}%
\begin{pgfscope}%
\definecolor{textcolor}{rgb}{0.150000,0.150000,0.150000}%
\pgfsetstrokecolor{textcolor}%
\pgfsetfillcolor{textcolor}%
\pgftext[x=3.480991in,y=0.319444in,,top]{\color{textcolor}\sffamily\fontsize{9.000000}{10.800000}\selectfont 2.0}%
\end{pgfscope}%
\begin{pgfscope}%
\pgfpathrectangle{\pgfqpoint{0.601779in}{0.451389in}}{\pgfqpoint{3.774387in}{1.264399in}}%
\pgfusepath{clip}%
\pgfsetroundcap%
\pgfsetroundjoin%
\pgfsetlinewidth{1.003750pt}%
\definecolor{currentstroke}{rgb}{0.800000,0.800000,0.800000}%
\pgfsetstrokecolor{currentstroke}%
\pgfsetdash{}{0pt}%
\pgfpathmoveto{\pgfqpoint{4.200794in}{0.451389in}}%
\pgfpathlineto{\pgfqpoint{4.200794in}{1.715788in}}%
\pgfusepath{stroke}%
\end{pgfscope}%
\begin{pgfscope}%
\definecolor{textcolor}{rgb}{0.150000,0.150000,0.150000}%
\pgfsetstrokecolor{textcolor}%
\pgfsetfillcolor{textcolor}%
\pgftext[x=4.200794in,y=0.319444in,,top]{\color{textcolor}\sffamily\fontsize{9.000000}{10.800000}\selectfont 2.5}%
\end{pgfscope}%
\begin{pgfscope}%
\definecolor{textcolor}{rgb}{0.150000,0.150000,0.150000}%
\pgfsetstrokecolor{textcolor}%
\pgfsetfillcolor{textcolor}%
\pgftext[x=2.488973in,y=0.125000in,,top]{\color{textcolor}\sffamily\fontsize{9.000000}{10.800000}\selectfont Beeline distance in km}%
\end{pgfscope}%
\begin{pgfscope}%
\pgfpathrectangle{\pgfqpoint{0.601779in}{0.451389in}}{\pgfqpoint{3.774387in}{1.264399in}}%
\pgfusepath{clip}%
\pgfsetroundcap%
\pgfsetroundjoin%
\pgfsetlinewidth{1.003750pt}%
\definecolor{currentstroke}{rgb}{0.800000,0.800000,0.800000}%
\pgfsetstrokecolor{currentstroke}%
\pgfsetdash{}{0pt}%
\pgfpathmoveto{\pgfqpoint{0.601779in}{0.451389in}}%
\pgfpathlineto{\pgfqpoint{4.376166in}{0.451389in}}%
\pgfusepath{stroke}%
\end{pgfscope}%
\begin{pgfscope}%
\definecolor{textcolor}{rgb}{0.150000,0.150000,0.150000}%
\pgfsetstrokecolor{textcolor}%
\pgfsetfillcolor{textcolor}%
\pgftext[x=0.400987in, y=0.403903in, left, base]{\color{textcolor}\sffamily\fontsize{9.000000}{10.800000}\selectfont 0}%
\end{pgfscope}%
\begin{pgfscope}%
\pgfpathrectangle{\pgfqpoint{0.601779in}{0.451389in}}{\pgfqpoint{3.774387in}{1.264399in}}%
\pgfusepath{clip}%
\pgfsetroundcap%
\pgfsetroundjoin%
\pgfsetlinewidth{1.003750pt}%
\definecolor{currentstroke}{rgb}{0.800000,0.800000,0.800000}%
\pgfsetstrokecolor{currentstroke}%
\pgfsetdash{}{0pt}%
\pgfpathmoveto{\pgfqpoint{0.601779in}{0.882791in}}%
\pgfpathlineto{\pgfqpoint{4.376166in}{0.882791in}}%
\pgfusepath{stroke}%
\end{pgfscope}%
\begin{pgfscope}%
\definecolor{textcolor}{rgb}{0.150000,0.150000,0.150000}%
\pgfsetstrokecolor{textcolor}%
\pgfsetfillcolor{textcolor}%
\pgftext[x=0.263292in, y=0.835305in, left, base]{\color{textcolor}\sffamily\fontsize{9.000000}{10.800000}\selectfont 500}%
\end{pgfscope}%
\begin{pgfscope}%
\pgfpathrectangle{\pgfqpoint{0.601779in}{0.451389in}}{\pgfqpoint{3.774387in}{1.264399in}}%
\pgfusepath{clip}%
\pgfsetroundcap%
\pgfsetroundjoin%
\pgfsetlinewidth{1.003750pt}%
\definecolor{currentstroke}{rgb}{0.800000,0.800000,0.800000}%
\pgfsetstrokecolor{currentstroke}%
\pgfsetdash{}{0pt}%
\pgfpathmoveto{\pgfqpoint{0.601779in}{1.314193in}}%
\pgfpathlineto{\pgfqpoint{4.376166in}{1.314193in}}%
\pgfusepath{stroke}%
\end{pgfscope}%
\begin{pgfscope}%
\definecolor{textcolor}{rgb}{0.150000,0.150000,0.150000}%
\pgfsetstrokecolor{textcolor}%
\pgfsetfillcolor{textcolor}%
\pgftext[x=0.194444in, y=1.266708in, left, base]{\color{textcolor}\sffamily\fontsize{9.000000}{10.800000}\selectfont 1000}%
\end{pgfscope}%
\begin{pgfscope}%
\definecolor{textcolor}{rgb}{0.150000,0.150000,0.150000}%
\pgfsetstrokecolor{textcolor}%
\pgfsetfillcolor{textcolor}%
\pgftext[x=0.125000in,y=1.083588in,,bottom,rotate=90.000000]{\color{textcolor}\sffamily\fontsize{9.000000}{10.800000}\selectfont Time in ms}%
\end{pgfscope}%
\begin{pgfscope}%
\pgfsetrectcap%
\pgfsetmiterjoin%
\pgfsetlinewidth{1.254687pt}%
\definecolor{currentstroke}{rgb}{0.800000,0.800000,0.800000}%
\pgfsetstrokecolor{currentstroke}%
\pgfsetdash{}{0pt}%
\pgfpathmoveto{\pgfqpoint{0.601779in}{0.451389in}}%
\pgfpathlineto{\pgfqpoint{0.601779in}{1.715788in}}%
\pgfusepath{stroke}%
\end{pgfscope}%
\begin{pgfscope}%
\pgfsetrectcap%
\pgfsetmiterjoin%
\pgfsetlinewidth{1.254687pt}%
\definecolor{currentstroke}{rgb}{0.800000,0.800000,0.800000}%
\pgfsetstrokecolor{currentstroke}%
\pgfsetdash{}{0pt}%
\pgfpathmoveto{\pgfqpoint{4.376166in}{0.451389in}}%
\pgfpathlineto{\pgfqpoint{4.376166in}{1.715788in}}%
\pgfusepath{stroke}%
\end{pgfscope}%
\begin{pgfscope}%
\pgfsetrectcap%
\pgfsetmiterjoin%
\pgfsetlinewidth{1.254687pt}%
\definecolor{currentstroke}{rgb}{0.800000,0.800000,0.800000}%
\pgfsetstrokecolor{currentstroke}%
\pgfsetdash{}{0pt}%
\pgfpathmoveto{\pgfqpoint{0.601779in}{0.451389in}}%
\pgfpathlineto{\pgfqpoint{4.376166in}{0.451389in}}%
\pgfusepath{stroke}%
\end{pgfscope}%
\begin{pgfscope}%
\pgfsetrectcap%
\pgfsetmiterjoin%
\pgfsetlinewidth{1.254687pt}%
\definecolor{currentstroke}{rgb}{0.800000,0.800000,0.800000}%
\pgfsetstrokecolor{currentstroke}%
\pgfsetdash{}{0pt}%
\pgfpathmoveto{\pgfqpoint{0.601779in}{1.715788in}}%
\pgfpathlineto{\pgfqpoint{4.376166in}{1.715788in}}%
\pgfusepath{stroke}%
\end{pgfscope}%
\begin{pgfscope}%
\pgfsetbuttcap%
\pgfsetmiterjoin%
\definecolor{currentfill}{rgb}{1.000000,1.000000,1.000000}%
\pgfsetfillcolor{currentfill}%
\pgfsetfillopacity{0.800000}%
\pgfsetlinewidth{1.003750pt}%
\definecolor{currentstroke}{rgb}{0.800000,0.800000,0.800000}%
\pgfsetstrokecolor{currentstroke}%
\pgfsetstrokeopacity{0.800000}%
\pgfsetdash{}{0pt}%
\pgfpathmoveto{\pgfqpoint{4.558026in}{0.524092in}}%
\pgfpathlineto{\pgfqpoint{6.054644in}{0.524092in}}%
\pgfpathquadraticcurveto{\pgfqpoint{6.079644in}{0.524092in}}{\pgfqpoint{6.079644in}{0.549092in}}%
\pgfpathlineto{\pgfqpoint{6.079644in}{1.618085in}}%
\pgfpathquadraticcurveto{\pgfqpoint{6.079644in}{1.643085in}}{\pgfqpoint{6.054644in}{1.643085in}}%
\pgfpathlineto{\pgfqpoint{4.558026in}{1.643085in}}%
\pgfpathquadraticcurveto{\pgfqpoint{4.533026in}{1.643085in}}{\pgfqpoint{4.533026in}{1.618085in}}%
\pgfpathlineto{\pgfqpoint{4.533026in}{0.549092in}}%
\pgfpathquadraticcurveto{\pgfqpoint{4.533026in}{0.524092in}}{\pgfqpoint{4.558026in}{0.524092in}}%
\pgfpathlineto{\pgfqpoint{4.558026in}{0.524092in}}%
\pgfpathclose%
\pgfusepath{stroke,fill}%
\end{pgfscope}%
\begin{pgfscope}%
\definecolor{textcolor}{rgb}{0.150000,0.150000,0.150000}%
\pgfsetstrokecolor{textcolor}%
\pgfsetfillcolor{textcolor}%
\pgftext[x=5.102935in,y=1.498114in,left,base]{\color{textcolor}\sffamily\fontsize{9.000000}{10.800000}\selectfont Legend}%
\end{pgfscope}%
\begin{pgfscope}%
\pgfsetroundcap%
\pgfsetroundjoin%
\pgfsetlinewidth{1.505625pt}%
\definecolor{currentstroke}{rgb}{0.003922,0.450980,0.698039}%
\pgfsetstrokecolor{currentstroke}%
\pgfsetdash{}{0pt}%
\pgfpathmoveto{\pgfqpoint{4.583026in}{1.354364in}}%
\pgfpathlineto{\pgfqpoint{4.708026in}{1.354364in}}%
\pgfpathlineto{\pgfqpoint{4.833026in}{1.354364in}}%
\pgfusepath{stroke}%
\end{pgfscope}%
\begin{pgfscope}%
\definecolor{textcolor}{rgb}{0.150000,0.150000,0.150000}%
\pgfsetstrokecolor{textcolor}%
\pgfsetfillcolor{textcolor}%
\pgftext[x=4.933026in,y=1.310614in,left,base]{\color{textcolor}\sffamily\fontsize{9.000000}{10.800000}\selectfont Total time}%
\end{pgfscope}%
\begin{pgfscope}%
\pgfsetroundcap%
\pgfsetroundjoin%
\pgfsetlinewidth{1.505625pt}%
\definecolor{currentstroke}{rgb}{0.870588,0.560784,0.019608}%
\pgfsetstrokecolor{currentstroke}%
\pgfsetdash{}{0pt}%
\pgfpathmoveto{\pgfqpoint{4.583026in}{1.166864in}}%
\pgfpathlineto{\pgfqpoint{4.708026in}{1.166864in}}%
\pgfpathlineto{\pgfqpoint{4.833026in}{1.166864in}}%
\pgfusepath{stroke}%
\end{pgfscope}%
\begin{pgfscope}%
\definecolor{textcolor}{rgb}{0.150000,0.150000,0.150000}%
\pgfsetstrokecolor{textcolor}%
\pgfsetfillcolor{textcolor}%
\pgftext[x=4.933026in,y=1.123114in,left,base]{\color{textcolor}\sffamily\fontsize{9.000000}{10.800000}\selectfont A* routing}%
\end{pgfscope}%
\begin{pgfscope}%
\pgfsetroundcap%
\pgfsetroundjoin%
\pgfsetlinewidth{1.505625pt}%
\definecolor{currentstroke}{rgb}{0.007843,0.619608,0.450980}%
\pgfsetstrokecolor{currentstroke}%
\pgfsetdash{}{0pt}%
\pgfpathmoveto{\pgfqpoint{4.583026in}{0.892353in}}%
\pgfpathlineto{\pgfqpoint{4.708026in}{0.892353in}}%
\pgfpathlineto{\pgfqpoint{4.833026in}{0.892353in}}%
\pgfusepath{stroke}%
\end{pgfscope}%
\begin{pgfscope}%
\definecolor{textcolor}{rgb}{0.150000,0.150000,0.150000}%
\pgfsetstrokecolor{textcolor}%
\pgfsetfillcolor{textcolor}%
\pgftext[x=4.933026in, y=0.935615in, left, base]{\color{textcolor}\sffamily\fontsize{9.000000}{10.800000}\selectfont Connect source \&}%
\end{pgfscope}%
\begin{pgfscope}%
\definecolor{textcolor}{rgb}{0.150000,0.150000,0.150000}%
\pgfsetstrokecolor{textcolor}%
\pgfsetfillcolor{textcolor}%
\pgftext[x=4.933026in, y=0.791621in, left, base]{\color{textcolor}\sffamily\fontsize{9.000000}{10.800000}\selectfont destination vertices}%
\end{pgfscope}%
\begin{pgfscope}%
\pgfsetroundcap%
\pgfsetroundjoin%
\pgfsetlinewidth{1.505625pt}%
\definecolor{currentstroke}{rgb}{0.835294,0.368627,0.000000}%
\pgfsetstrokecolor{currentstroke}%
\pgfsetdash{}{0pt}%
\pgfpathmoveto{\pgfqpoint{4.583026in}{0.647871in}}%
\pgfpathlineto{\pgfqpoint{4.708026in}{0.647871in}}%
\pgfpathlineto{\pgfqpoint{4.833026in}{0.647871in}}%
\pgfusepath{stroke}%
\end{pgfscope}%
\begin{pgfscope}%
\definecolor{textcolor}{rgb}{0.150000,0.150000,0.150000}%
\pgfsetstrokecolor{textcolor}%
\pgfsetfillcolor{textcolor}%
\pgftext[x=4.933026in,y=0.604121in,left,base]{\color{textcolor}\sffamily\fontsize{9.000000}{10.800000}\selectfont Restoring graph}%
\end{pgfscope}%
\begin{pgfscope}%
\pgfsetroundcap%
\pgfsetroundjoin%
\pgfsetlinewidth{1.003750pt}%
\definecolor{currentstroke}{rgb}{0.003922,0.450980,0.698039}%
\pgfsetstrokecolor{currentstroke}%
\pgfsetdash{}{0pt}%
\pgfpathmoveto{\pgfqpoint{0.745887in}{1.399808in}}%
\pgfpathlineto{\pgfqpoint{0.818221in}{1.067905in}}%
\pgfpathlineto{\pgfqpoint{0.889755in}{1.187851in}}%
\pgfpathlineto{\pgfqpoint{0.961613in}{1.241861in}}%
\pgfpathlineto{\pgfqpoint{1.033408in}{1.303790in}}%
\pgfpathlineto{\pgfqpoint{1.105273in}{1.247631in}}%
\pgfpathlineto{\pgfqpoint{1.178107in}{1.449202in}}%
\pgfpathlineto{\pgfqpoint{1.249087in}{1.371564in}}%
\pgfpathlineto{\pgfqpoint{1.321208in}{1.141457in}}%
\pgfpathlineto{\pgfqpoint{1.393720in}{1.494318in}}%
\pgfpathlineto{\pgfqpoint{1.464776in}{1.201536in}}%
\pgfpathlineto{\pgfqpoint{1.536986in}{1.307786in}}%
\pgfpathlineto{\pgfqpoint{1.609100in}{1.501788in}}%
\pgfpathlineto{\pgfqpoint{1.680285in}{1.112698in}}%
\pgfpathlineto{\pgfqpoint{1.752199in}{1.531079in}}%
\pgfpathlineto{\pgfqpoint{1.825380in}{1.403397in}}%
\pgfpathlineto{\pgfqpoint{1.897092in}{1.243126in}}%
\pgfpathlineto{\pgfqpoint{1.968809in}{1.498698in}}%
\pgfpathlineto{\pgfqpoint{2.046601in}{1.228767in}}%
\pgfpathlineto{\pgfqpoint{2.106358in}{1.263095in}}%
\pgfpathlineto{\pgfqpoint{2.182585in}{1.212556in}}%
\pgfpathlineto{\pgfqpoint{2.261861in}{1.088173in}}%
\pgfpathlineto{\pgfqpoint{2.320009in}{0.942562in}}%
\pgfpathlineto{\pgfqpoint{2.387763in}{1.001401in}}%
\pgfpathlineto{\pgfqpoint{2.469343in}{1.006110in}}%
\pgfpathlineto{\pgfqpoint{2.757758in}{1.128722in}}%
\pgfpathlineto{\pgfqpoint{2.909278in}{1.238779in}}%
\pgfpathlineto{\pgfqpoint{3.045767in}{1.197697in}}%
\pgfpathlineto{\pgfqpoint{3.187923in}{1.230308in}}%
\pgfpathlineto{\pgfqpoint{3.339026in}{1.485928in}}%
\pgfpathlineto{\pgfqpoint{3.483329in}{1.655729in}}%
\pgfpathlineto{\pgfqpoint{3.617844in}{1.329151in}}%
\pgfpathlineto{\pgfqpoint{3.762273in}{1.181372in}}%
\pgfpathlineto{\pgfqpoint{3.915192in}{1.277693in}}%
\pgfpathlineto{\pgfqpoint{4.196433in}{1.287445in}}%
\pgfusepath{stroke}%
\end{pgfscope}%
\begin{pgfscope}%
\pgfsetbuttcap%
\pgfsetroundjoin%
\definecolor{currentfill}{rgb}{0.003922,0.450980,0.698039}%
\pgfsetfillcolor{currentfill}%
\pgfsetlinewidth{0.752812pt}%
\definecolor{currentstroke}{rgb}{1.000000,1.000000,1.000000}%
\pgfsetstrokecolor{currentstroke}%
\pgfsetdash{}{0pt}%
\pgfsys@defobject{currentmarker}{\pgfqpoint{-0.034722in}{-0.034722in}}{\pgfqpoint{0.034722in}{0.034722in}}{%
\pgfpathmoveto{\pgfqpoint{0.000000in}{-0.034722in}}%
\pgfpathcurveto{\pgfqpoint{0.009208in}{-0.034722in}}{\pgfqpoint{0.018041in}{-0.031064in}}{\pgfqpoint{0.024552in}{-0.024552in}}%
\pgfpathcurveto{\pgfqpoint{0.031064in}{-0.018041in}}{\pgfqpoint{0.034722in}{-0.009208in}}{\pgfqpoint{0.034722in}{0.000000in}}%
\pgfpathcurveto{\pgfqpoint{0.034722in}{0.009208in}}{\pgfqpoint{0.031064in}{0.018041in}}{\pgfqpoint{0.024552in}{0.024552in}}%
\pgfpathcurveto{\pgfqpoint{0.018041in}{0.031064in}}{\pgfqpoint{0.009208in}{0.034722in}}{\pgfqpoint{0.000000in}{0.034722in}}%
\pgfpathcurveto{\pgfqpoint{-0.009208in}{0.034722in}}{\pgfqpoint{-0.018041in}{0.031064in}}{\pgfqpoint{-0.024552in}{0.024552in}}%
\pgfpathcurveto{\pgfqpoint{-0.031064in}{0.018041in}}{\pgfqpoint{-0.034722in}{0.009208in}}{\pgfqpoint{-0.034722in}{0.000000in}}%
\pgfpathcurveto{\pgfqpoint{-0.034722in}{-0.009208in}}{\pgfqpoint{-0.031064in}{-0.018041in}}{\pgfqpoint{-0.024552in}{-0.024552in}}%
\pgfpathcurveto{\pgfqpoint{-0.018041in}{-0.031064in}}{\pgfqpoint{-0.009208in}{-0.034722in}}{\pgfqpoint{0.000000in}{-0.034722in}}%
\pgfpathlineto{\pgfqpoint{0.000000in}{-0.034722in}}%
\pgfpathclose%
\pgfusepath{stroke,fill}%
}%
\begin{pgfscope}%
\pgfsys@transformshift{0.745887in}{1.399808in}%
\pgfsys@useobject{currentmarker}{}%
\end{pgfscope}%
\begin{pgfscope}%
\pgfsys@transformshift{0.818221in}{1.067905in}%
\pgfsys@useobject{currentmarker}{}%
\end{pgfscope}%
\begin{pgfscope}%
\pgfsys@transformshift{0.889755in}{1.187851in}%
\pgfsys@useobject{currentmarker}{}%
\end{pgfscope}%
\begin{pgfscope}%
\pgfsys@transformshift{0.961613in}{1.241861in}%
\pgfsys@useobject{currentmarker}{}%
\end{pgfscope}%
\begin{pgfscope}%
\pgfsys@transformshift{1.033408in}{1.303790in}%
\pgfsys@useobject{currentmarker}{}%
\end{pgfscope}%
\begin{pgfscope}%
\pgfsys@transformshift{1.105273in}{1.247631in}%
\pgfsys@useobject{currentmarker}{}%
\end{pgfscope}%
\begin{pgfscope}%
\pgfsys@transformshift{1.178107in}{1.449202in}%
\pgfsys@useobject{currentmarker}{}%
\end{pgfscope}%
\begin{pgfscope}%
\pgfsys@transformshift{1.249087in}{1.371564in}%
\pgfsys@useobject{currentmarker}{}%
\end{pgfscope}%
\begin{pgfscope}%
\pgfsys@transformshift{1.321208in}{1.141457in}%
\pgfsys@useobject{currentmarker}{}%
\end{pgfscope}%
\begin{pgfscope}%
\pgfsys@transformshift{1.393720in}{1.494318in}%
\pgfsys@useobject{currentmarker}{}%
\end{pgfscope}%
\begin{pgfscope}%
\pgfsys@transformshift{1.464776in}{1.201536in}%
\pgfsys@useobject{currentmarker}{}%
\end{pgfscope}%
\begin{pgfscope}%
\pgfsys@transformshift{1.536986in}{1.307786in}%
\pgfsys@useobject{currentmarker}{}%
\end{pgfscope}%
\begin{pgfscope}%
\pgfsys@transformshift{1.609100in}{1.501788in}%
\pgfsys@useobject{currentmarker}{}%
\end{pgfscope}%
\begin{pgfscope}%
\pgfsys@transformshift{1.680285in}{1.112698in}%
\pgfsys@useobject{currentmarker}{}%
\end{pgfscope}%
\begin{pgfscope}%
\pgfsys@transformshift{1.752199in}{1.531079in}%
\pgfsys@useobject{currentmarker}{}%
\end{pgfscope}%
\begin{pgfscope}%
\pgfsys@transformshift{1.825380in}{1.403397in}%
\pgfsys@useobject{currentmarker}{}%
\end{pgfscope}%
\begin{pgfscope}%
\pgfsys@transformshift{1.897092in}{1.243126in}%
\pgfsys@useobject{currentmarker}{}%
\end{pgfscope}%
\begin{pgfscope}%
\pgfsys@transformshift{1.968809in}{1.498698in}%
\pgfsys@useobject{currentmarker}{}%
\end{pgfscope}%
\begin{pgfscope}%
\pgfsys@transformshift{2.046601in}{1.228767in}%
\pgfsys@useobject{currentmarker}{}%
\end{pgfscope}%
\begin{pgfscope}%
\pgfsys@transformshift{2.106358in}{1.263095in}%
\pgfsys@useobject{currentmarker}{}%
\end{pgfscope}%
\begin{pgfscope}%
\pgfsys@transformshift{2.182585in}{1.212556in}%
\pgfsys@useobject{currentmarker}{}%
\end{pgfscope}%
\begin{pgfscope}%
\pgfsys@transformshift{2.261861in}{1.088173in}%
\pgfsys@useobject{currentmarker}{}%
\end{pgfscope}%
\begin{pgfscope}%
\pgfsys@transformshift{2.320009in}{0.942562in}%
\pgfsys@useobject{currentmarker}{}%
\end{pgfscope}%
\begin{pgfscope}%
\pgfsys@transformshift{2.387763in}{1.001401in}%
\pgfsys@useobject{currentmarker}{}%
\end{pgfscope}%
\begin{pgfscope}%
\pgfsys@transformshift{2.469343in}{1.006110in}%
\pgfsys@useobject{currentmarker}{}%
\end{pgfscope}%
\begin{pgfscope}%
\pgfsys@transformshift{2.757758in}{1.128722in}%
\pgfsys@useobject{currentmarker}{}%
\end{pgfscope}%
\begin{pgfscope}%
\pgfsys@transformshift{2.909278in}{1.238779in}%
\pgfsys@useobject{currentmarker}{}%
\end{pgfscope}%
\begin{pgfscope}%
\pgfsys@transformshift{3.045767in}{1.197697in}%
\pgfsys@useobject{currentmarker}{}%
\end{pgfscope}%
\begin{pgfscope}%
\pgfsys@transformshift{3.187923in}{1.230308in}%
\pgfsys@useobject{currentmarker}{}%
\end{pgfscope}%
\begin{pgfscope}%
\pgfsys@transformshift{3.339026in}{1.485928in}%
\pgfsys@useobject{currentmarker}{}%
\end{pgfscope}%
\begin{pgfscope}%
\pgfsys@transformshift{3.483329in}{1.655729in}%
\pgfsys@useobject{currentmarker}{}%
\end{pgfscope}%
\begin{pgfscope}%
\pgfsys@transformshift{3.617844in}{1.329151in}%
\pgfsys@useobject{currentmarker}{}%
\end{pgfscope}%
\begin{pgfscope}%
\pgfsys@transformshift{3.762273in}{1.181372in}%
\pgfsys@useobject{currentmarker}{}%
\end{pgfscope}%
\begin{pgfscope}%
\pgfsys@transformshift{3.915192in}{1.277693in}%
\pgfsys@useobject{currentmarker}{}%
\end{pgfscope}%
\begin{pgfscope}%
\pgfsys@transformshift{4.196433in}{1.287445in}%
\pgfsys@useobject{currentmarker}{}%
\end{pgfscope}%
\end{pgfscope}%
\begin{pgfscope}%
\pgfsetroundcap%
\pgfsetroundjoin%
\pgfsetlinewidth{1.003750pt}%
\definecolor{currentstroke}{rgb}{0.870588,0.560784,0.019608}%
\pgfsetstrokecolor{currentstroke}%
\pgfsetdash{}{0pt}%
\pgfpathmoveto{\pgfqpoint{0.745887in}{0.454553in}}%
\pgfpathlineto{\pgfqpoint{0.818221in}{0.456478in}}%
\pgfpathlineto{\pgfqpoint{0.889755in}{0.461292in}}%
\pgfpathlineto{\pgfqpoint{0.961613in}{0.458904in}}%
\pgfpathlineto{\pgfqpoint{1.033408in}{0.470614in}}%
\pgfpathlineto{\pgfqpoint{1.105273in}{0.490627in}}%
\pgfpathlineto{\pgfqpoint{1.178107in}{0.491386in}}%
\pgfpathlineto{\pgfqpoint{1.249087in}{0.533988in}}%
\pgfpathlineto{\pgfqpoint{1.321208in}{0.499081in}}%
\pgfpathlineto{\pgfqpoint{1.393720in}{0.585890in}}%
\pgfpathlineto{\pgfqpoint{1.464776in}{0.513908in}}%
\pgfpathlineto{\pgfqpoint{1.536986in}{0.568138in}}%
\pgfpathlineto{\pgfqpoint{1.609100in}{0.603794in}}%
\pgfpathlineto{\pgfqpoint{1.680285in}{0.580867in}}%
\pgfpathlineto{\pgfqpoint{1.752199in}{0.673946in}}%
\pgfpathlineto{\pgfqpoint{1.825380in}{0.516912in}}%
\pgfpathlineto{\pgfqpoint{1.897092in}{0.610586in}}%
\pgfpathlineto{\pgfqpoint{1.968809in}{0.648365in}}%
\pgfpathlineto{\pgfqpoint{2.046601in}{0.585839in}}%
\pgfpathlineto{\pgfqpoint{2.106358in}{0.684408in}}%
\pgfpathlineto{\pgfqpoint{2.182585in}{0.612736in}}%
\pgfpathlineto{\pgfqpoint{2.261861in}{0.700564in}}%
\pgfpathlineto{\pgfqpoint{2.320009in}{0.635603in}}%
\pgfpathlineto{\pgfqpoint{2.387763in}{0.716343in}}%
\pgfpathlineto{\pgfqpoint{2.469343in}{0.657489in}}%
\pgfpathlineto{\pgfqpoint{2.757758in}{0.707235in}}%
\pgfpathlineto{\pgfqpoint{2.909278in}{0.689185in}}%
\pgfpathlineto{\pgfqpoint{3.045767in}{0.694643in}}%
\pgfpathlineto{\pgfqpoint{3.187923in}{0.721569in}}%
\pgfpathlineto{\pgfqpoint{3.339026in}{0.785435in}}%
\pgfpathlineto{\pgfqpoint{3.483329in}{0.813175in}}%
\pgfpathlineto{\pgfqpoint{3.617844in}{0.737642in}}%
\pgfpathlineto{\pgfqpoint{3.762273in}{0.739044in}}%
\pgfpathlineto{\pgfqpoint{3.915192in}{0.792226in}}%
\pgfpathlineto{\pgfqpoint{4.196433in}{0.809924in}}%
\pgfusepath{stroke}%
\end{pgfscope}%
\begin{pgfscope}%
\pgfsetbuttcap%
\pgfsetroundjoin%
\definecolor{currentfill}{rgb}{0.870588,0.560784,0.019608}%
\pgfsetfillcolor{currentfill}%
\pgfsetlinewidth{0.752812pt}%
\definecolor{currentstroke}{rgb}{1.000000,1.000000,1.000000}%
\pgfsetstrokecolor{currentstroke}%
\pgfsetdash{}{0pt}%
\pgfsys@defobject{currentmarker}{\pgfqpoint{-0.034722in}{-0.034722in}}{\pgfqpoint{0.034722in}{0.034722in}}{%
\pgfpathmoveto{\pgfqpoint{0.000000in}{-0.034722in}}%
\pgfpathcurveto{\pgfqpoint{0.009208in}{-0.034722in}}{\pgfqpoint{0.018041in}{-0.031064in}}{\pgfqpoint{0.024552in}{-0.024552in}}%
\pgfpathcurveto{\pgfqpoint{0.031064in}{-0.018041in}}{\pgfqpoint{0.034722in}{-0.009208in}}{\pgfqpoint{0.034722in}{0.000000in}}%
\pgfpathcurveto{\pgfqpoint{0.034722in}{0.009208in}}{\pgfqpoint{0.031064in}{0.018041in}}{\pgfqpoint{0.024552in}{0.024552in}}%
\pgfpathcurveto{\pgfqpoint{0.018041in}{0.031064in}}{\pgfqpoint{0.009208in}{0.034722in}}{\pgfqpoint{0.000000in}{0.034722in}}%
\pgfpathcurveto{\pgfqpoint{-0.009208in}{0.034722in}}{\pgfqpoint{-0.018041in}{0.031064in}}{\pgfqpoint{-0.024552in}{0.024552in}}%
\pgfpathcurveto{\pgfqpoint{-0.031064in}{0.018041in}}{\pgfqpoint{-0.034722in}{0.009208in}}{\pgfqpoint{-0.034722in}{0.000000in}}%
\pgfpathcurveto{\pgfqpoint{-0.034722in}{-0.009208in}}{\pgfqpoint{-0.031064in}{-0.018041in}}{\pgfqpoint{-0.024552in}{-0.024552in}}%
\pgfpathcurveto{\pgfqpoint{-0.018041in}{-0.031064in}}{\pgfqpoint{-0.009208in}{-0.034722in}}{\pgfqpoint{0.000000in}{-0.034722in}}%
\pgfpathlineto{\pgfqpoint{0.000000in}{-0.034722in}}%
\pgfpathclose%
\pgfusepath{stroke,fill}%
}%
\begin{pgfscope}%
\pgfsys@transformshift{0.745887in}{0.454553in}%
\pgfsys@useobject{currentmarker}{}%
\end{pgfscope}%
\begin{pgfscope}%
\pgfsys@transformshift{0.818221in}{0.456478in}%
\pgfsys@useobject{currentmarker}{}%
\end{pgfscope}%
\begin{pgfscope}%
\pgfsys@transformshift{0.889755in}{0.461292in}%
\pgfsys@useobject{currentmarker}{}%
\end{pgfscope}%
\begin{pgfscope}%
\pgfsys@transformshift{0.961613in}{0.458904in}%
\pgfsys@useobject{currentmarker}{}%
\end{pgfscope}%
\begin{pgfscope}%
\pgfsys@transformshift{1.033408in}{0.470614in}%
\pgfsys@useobject{currentmarker}{}%
\end{pgfscope}%
\begin{pgfscope}%
\pgfsys@transformshift{1.105273in}{0.490627in}%
\pgfsys@useobject{currentmarker}{}%
\end{pgfscope}%
\begin{pgfscope}%
\pgfsys@transformshift{1.178107in}{0.491386in}%
\pgfsys@useobject{currentmarker}{}%
\end{pgfscope}%
\begin{pgfscope}%
\pgfsys@transformshift{1.249087in}{0.533988in}%
\pgfsys@useobject{currentmarker}{}%
\end{pgfscope}%
\begin{pgfscope}%
\pgfsys@transformshift{1.321208in}{0.499081in}%
\pgfsys@useobject{currentmarker}{}%
\end{pgfscope}%
\begin{pgfscope}%
\pgfsys@transformshift{1.393720in}{0.585890in}%
\pgfsys@useobject{currentmarker}{}%
\end{pgfscope}%
\begin{pgfscope}%
\pgfsys@transformshift{1.464776in}{0.513908in}%
\pgfsys@useobject{currentmarker}{}%
\end{pgfscope}%
\begin{pgfscope}%
\pgfsys@transformshift{1.536986in}{0.568138in}%
\pgfsys@useobject{currentmarker}{}%
\end{pgfscope}%
\begin{pgfscope}%
\pgfsys@transformshift{1.609100in}{0.603794in}%
\pgfsys@useobject{currentmarker}{}%
\end{pgfscope}%
\begin{pgfscope}%
\pgfsys@transformshift{1.680285in}{0.580867in}%
\pgfsys@useobject{currentmarker}{}%
\end{pgfscope}%
\begin{pgfscope}%
\pgfsys@transformshift{1.752199in}{0.673946in}%
\pgfsys@useobject{currentmarker}{}%
\end{pgfscope}%
\begin{pgfscope}%
\pgfsys@transformshift{1.825380in}{0.516912in}%
\pgfsys@useobject{currentmarker}{}%
\end{pgfscope}%
\begin{pgfscope}%
\pgfsys@transformshift{1.897092in}{0.610586in}%
\pgfsys@useobject{currentmarker}{}%
\end{pgfscope}%
\begin{pgfscope}%
\pgfsys@transformshift{1.968809in}{0.648365in}%
\pgfsys@useobject{currentmarker}{}%
\end{pgfscope}%
\begin{pgfscope}%
\pgfsys@transformshift{2.046601in}{0.585839in}%
\pgfsys@useobject{currentmarker}{}%
\end{pgfscope}%
\begin{pgfscope}%
\pgfsys@transformshift{2.106358in}{0.684408in}%
\pgfsys@useobject{currentmarker}{}%
\end{pgfscope}%
\begin{pgfscope}%
\pgfsys@transformshift{2.182585in}{0.612736in}%
\pgfsys@useobject{currentmarker}{}%
\end{pgfscope}%
\begin{pgfscope}%
\pgfsys@transformshift{2.261861in}{0.700564in}%
\pgfsys@useobject{currentmarker}{}%
\end{pgfscope}%
\begin{pgfscope}%
\pgfsys@transformshift{2.320009in}{0.635603in}%
\pgfsys@useobject{currentmarker}{}%
\end{pgfscope}%
\begin{pgfscope}%
\pgfsys@transformshift{2.387763in}{0.716343in}%
\pgfsys@useobject{currentmarker}{}%
\end{pgfscope}%
\begin{pgfscope}%
\pgfsys@transformshift{2.469343in}{0.657489in}%
\pgfsys@useobject{currentmarker}{}%
\end{pgfscope}%
\begin{pgfscope}%
\pgfsys@transformshift{2.757758in}{0.707235in}%
\pgfsys@useobject{currentmarker}{}%
\end{pgfscope}%
\begin{pgfscope}%
\pgfsys@transformshift{2.909278in}{0.689185in}%
\pgfsys@useobject{currentmarker}{}%
\end{pgfscope}%
\begin{pgfscope}%
\pgfsys@transformshift{3.045767in}{0.694643in}%
\pgfsys@useobject{currentmarker}{}%
\end{pgfscope}%
\begin{pgfscope}%
\pgfsys@transformshift{3.187923in}{0.721569in}%
\pgfsys@useobject{currentmarker}{}%
\end{pgfscope}%
\begin{pgfscope}%
\pgfsys@transformshift{3.339026in}{0.785435in}%
\pgfsys@useobject{currentmarker}{}%
\end{pgfscope}%
\begin{pgfscope}%
\pgfsys@transformshift{3.483329in}{0.813175in}%
\pgfsys@useobject{currentmarker}{}%
\end{pgfscope}%
\begin{pgfscope}%
\pgfsys@transformshift{3.617844in}{0.737642in}%
\pgfsys@useobject{currentmarker}{}%
\end{pgfscope}%
\begin{pgfscope}%
\pgfsys@transformshift{3.762273in}{0.739044in}%
\pgfsys@useobject{currentmarker}{}%
\end{pgfscope}%
\begin{pgfscope}%
\pgfsys@transformshift{3.915192in}{0.792226in}%
\pgfsys@useobject{currentmarker}{}%
\end{pgfscope}%
\begin{pgfscope}%
\pgfsys@transformshift{4.196433in}{0.809924in}%
\pgfsys@useobject{currentmarker}{}%
\end{pgfscope}%
\end{pgfscope}%
\begin{pgfscope}%
\pgfsetroundcap%
\pgfsetroundjoin%
\pgfsetlinewidth{1.003750pt}%
\definecolor{currentstroke}{rgb}{0.007843,0.619608,0.450980}%
\pgfsetstrokecolor{currentstroke}%
\pgfsetdash{}{0pt}%
\pgfpathmoveto{\pgfqpoint{0.745887in}{1.314305in}}%
\pgfpathlineto{\pgfqpoint{0.818221in}{0.982208in}}%
\pgfpathlineto{\pgfqpoint{0.889755in}{1.096568in}}%
\pgfpathlineto{\pgfqpoint{0.961613in}{1.153535in}}%
\pgfpathlineto{\pgfqpoint{1.033408in}{1.203191in}}%
\pgfpathlineto{\pgfqpoint{1.105273in}{1.126855in}}%
\pgfpathlineto{\pgfqpoint{1.178107in}{1.325572in}}%
\pgfpathlineto{\pgfqpoint{1.249087in}{1.206962in}}%
\pgfpathlineto{\pgfqpoint{1.321208in}{1.013727in}}%
\pgfpathlineto{\pgfqpoint{1.393720in}{1.278592in}}%
\pgfpathlineto{\pgfqpoint{1.464776in}{1.058947in}}%
\pgfpathlineto{\pgfqpoint{1.536986in}{1.109882in}}%
\pgfpathlineto{\pgfqpoint{1.609100in}{1.265406in}}%
\pgfpathlineto{\pgfqpoint{1.680285in}{0.901464in}}%
\pgfpathlineto{\pgfqpoint{1.752199in}{1.162374in}}%
\pgfpathlineto{\pgfqpoint{1.825380in}{1.254798in}}%
\pgfpathlineto{\pgfqpoint{1.897092in}{1.002312in}}%
\pgfpathlineto{\pgfqpoint{1.968809in}{1.219277in}}%
\pgfpathlineto{\pgfqpoint{2.046601in}{1.012691in}}%
\pgfpathlineto{\pgfqpoint{2.106358in}{0.948746in}}%
\pgfpathlineto{\pgfqpoint{2.182585in}{0.970088in}}%
\pgfpathlineto{\pgfqpoint{2.261861in}{0.759978in}}%
\pgfpathlineto{\pgfqpoint{2.320009in}{0.679612in}}%
\pgfpathlineto{\pgfqpoint{2.387763in}{0.657735in}}%
\pgfpathlineto{\pgfqpoint{2.469343in}{0.721282in}}%
\pgfpathlineto{\pgfqpoint{2.757758in}{0.793463in}}%
\pgfpathlineto{\pgfqpoint{2.909278in}{0.920784in}}%
\pgfpathlineto{\pgfqpoint{3.045767in}{0.873842in}}%
\pgfpathlineto{\pgfqpoint{3.187923in}{0.829331in}}%
\pgfpathlineto{\pgfqpoint{3.339026in}{1.032349in}}%
\pgfpathlineto{\pgfqpoint{3.483329in}{1.167137in}}%
\pgfpathlineto{\pgfqpoint{3.617844in}{0.902261in}}%
\pgfpathlineto{\pgfqpoint{3.762273in}{0.748821in}}%
\pgfpathlineto{\pgfqpoint{3.915192in}{0.826249in}}%
\pgfpathlineto{\pgfqpoint{4.196433in}{0.807723in}}%
\pgfusepath{stroke}%
\end{pgfscope}%
\begin{pgfscope}%
\pgfsetbuttcap%
\pgfsetroundjoin%
\definecolor{currentfill}{rgb}{0.007843,0.619608,0.450980}%
\pgfsetfillcolor{currentfill}%
\pgfsetlinewidth{0.752812pt}%
\definecolor{currentstroke}{rgb}{1.000000,1.000000,1.000000}%
\pgfsetstrokecolor{currentstroke}%
\pgfsetdash{}{0pt}%
\pgfsys@defobject{currentmarker}{\pgfqpoint{-0.034722in}{-0.034722in}}{\pgfqpoint{0.034722in}{0.034722in}}{%
\pgfpathmoveto{\pgfqpoint{0.000000in}{-0.034722in}}%
\pgfpathcurveto{\pgfqpoint{0.009208in}{-0.034722in}}{\pgfqpoint{0.018041in}{-0.031064in}}{\pgfqpoint{0.024552in}{-0.024552in}}%
\pgfpathcurveto{\pgfqpoint{0.031064in}{-0.018041in}}{\pgfqpoint{0.034722in}{-0.009208in}}{\pgfqpoint{0.034722in}{0.000000in}}%
\pgfpathcurveto{\pgfqpoint{0.034722in}{0.009208in}}{\pgfqpoint{0.031064in}{0.018041in}}{\pgfqpoint{0.024552in}{0.024552in}}%
\pgfpathcurveto{\pgfqpoint{0.018041in}{0.031064in}}{\pgfqpoint{0.009208in}{0.034722in}}{\pgfqpoint{0.000000in}{0.034722in}}%
\pgfpathcurveto{\pgfqpoint{-0.009208in}{0.034722in}}{\pgfqpoint{-0.018041in}{0.031064in}}{\pgfqpoint{-0.024552in}{0.024552in}}%
\pgfpathcurveto{\pgfqpoint{-0.031064in}{0.018041in}}{\pgfqpoint{-0.034722in}{0.009208in}}{\pgfqpoint{-0.034722in}{0.000000in}}%
\pgfpathcurveto{\pgfqpoint{-0.034722in}{-0.009208in}}{\pgfqpoint{-0.031064in}{-0.018041in}}{\pgfqpoint{-0.024552in}{-0.024552in}}%
\pgfpathcurveto{\pgfqpoint{-0.018041in}{-0.031064in}}{\pgfqpoint{-0.009208in}{-0.034722in}}{\pgfqpoint{0.000000in}{-0.034722in}}%
\pgfpathlineto{\pgfqpoint{0.000000in}{-0.034722in}}%
\pgfpathclose%
\pgfusepath{stroke,fill}%
}%
\begin{pgfscope}%
\pgfsys@transformshift{0.745887in}{1.314305in}%
\pgfsys@useobject{currentmarker}{}%
\end{pgfscope}%
\begin{pgfscope}%
\pgfsys@transformshift{0.818221in}{0.982208in}%
\pgfsys@useobject{currentmarker}{}%
\end{pgfscope}%
\begin{pgfscope}%
\pgfsys@transformshift{0.889755in}{1.096568in}%
\pgfsys@useobject{currentmarker}{}%
\end{pgfscope}%
\begin{pgfscope}%
\pgfsys@transformshift{0.961613in}{1.153535in}%
\pgfsys@useobject{currentmarker}{}%
\end{pgfscope}%
\begin{pgfscope}%
\pgfsys@transformshift{1.033408in}{1.203191in}%
\pgfsys@useobject{currentmarker}{}%
\end{pgfscope}%
\begin{pgfscope}%
\pgfsys@transformshift{1.105273in}{1.126855in}%
\pgfsys@useobject{currentmarker}{}%
\end{pgfscope}%
\begin{pgfscope}%
\pgfsys@transformshift{1.178107in}{1.325572in}%
\pgfsys@useobject{currentmarker}{}%
\end{pgfscope}%
\begin{pgfscope}%
\pgfsys@transformshift{1.249087in}{1.206962in}%
\pgfsys@useobject{currentmarker}{}%
\end{pgfscope}%
\begin{pgfscope}%
\pgfsys@transformshift{1.321208in}{1.013727in}%
\pgfsys@useobject{currentmarker}{}%
\end{pgfscope}%
\begin{pgfscope}%
\pgfsys@transformshift{1.393720in}{1.278592in}%
\pgfsys@useobject{currentmarker}{}%
\end{pgfscope}%
\begin{pgfscope}%
\pgfsys@transformshift{1.464776in}{1.058947in}%
\pgfsys@useobject{currentmarker}{}%
\end{pgfscope}%
\begin{pgfscope}%
\pgfsys@transformshift{1.536986in}{1.109882in}%
\pgfsys@useobject{currentmarker}{}%
\end{pgfscope}%
\begin{pgfscope}%
\pgfsys@transformshift{1.609100in}{1.265406in}%
\pgfsys@useobject{currentmarker}{}%
\end{pgfscope}%
\begin{pgfscope}%
\pgfsys@transformshift{1.680285in}{0.901464in}%
\pgfsys@useobject{currentmarker}{}%
\end{pgfscope}%
\begin{pgfscope}%
\pgfsys@transformshift{1.752199in}{1.162374in}%
\pgfsys@useobject{currentmarker}{}%
\end{pgfscope}%
\begin{pgfscope}%
\pgfsys@transformshift{1.825380in}{1.254798in}%
\pgfsys@useobject{currentmarker}{}%
\end{pgfscope}%
\begin{pgfscope}%
\pgfsys@transformshift{1.897092in}{1.002312in}%
\pgfsys@useobject{currentmarker}{}%
\end{pgfscope}%
\begin{pgfscope}%
\pgfsys@transformshift{1.968809in}{1.219277in}%
\pgfsys@useobject{currentmarker}{}%
\end{pgfscope}%
\begin{pgfscope}%
\pgfsys@transformshift{2.046601in}{1.012691in}%
\pgfsys@useobject{currentmarker}{}%
\end{pgfscope}%
\begin{pgfscope}%
\pgfsys@transformshift{2.106358in}{0.948746in}%
\pgfsys@useobject{currentmarker}{}%
\end{pgfscope}%
\begin{pgfscope}%
\pgfsys@transformshift{2.182585in}{0.970088in}%
\pgfsys@useobject{currentmarker}{}%
\end{pgfscope}%
\begin{pgfscope}%
\pgfsys@transformshift{2.261861in}{0.759978in}%
\pgfsys@useobject{currentmarker}{}%
\end{pgfscope}%
\begin{pgfscope}%
\pgfsys@transformshift{2.320009in}{0.679612in}%
\pgfsys@useobject{currentmarker}{}%
\end{pgfscope}%
\begin{pgfscope}%
\pgfsys@transformshift{2.387763in}{0.657735in}%
\pgfsys@useobject{currentmarker}{}%
\end{pgfscope}%
\begin{pgfscope}%
\pgfsys@transformshift{2.469343in}{0.721282in}%
\pgfsys@useobject{currentmarker}{}%
\end{pgfscope}%
\begin{pgfscope}%
\pgfsys@transformshift{2.757758in}{0.793463in}%
\pgfsys@useobject{currentmarker}{}%
\end{pgfscope}%
\begin{pgfscope}%
\pgfsys@transformshift{2.909278in}{0.920784in}%
\pgfsys@useobject{currentmarker}{}%
\end{pgfscope}%
\begin{pgfscope}%
\pgfsys@transformshift{3.045767in}{0.873842in}%
\pgfsys@useobject{currentmarker}{}%
\end{pgfscope}%
\begin{pgfscope}%
\pgfsys@transformshift{3.187923in}{0.829331in}%
\pgfsys@useobject{currentmarker}{}%
\end{pgfscope}%
\begin{pgfscope}%
\pgfsys@transformshift{3.339026in}{1.032349in}%
\pgfsys@useobject{currentmarker}{}%
\end{pgfscope}%
\begin{pgfscope}%
\pgfsys@transformshift{3.483329in}{1.167137in}%
\pgfsys@useobject{currentmarker}{}%
\end{pgfscope}%
\begin{pgfscope}%
\pgfsys@transformshift{3.617844in}{0.902261in}%
\pgfsys@useobject{currentmarker}{}%
\end{pgfscope}%
\begin{pgfscope}%
\pgfsys@transformshift{3.762273in}{0.748821in}%
\pgfsys@useobject{currentmarker}{}%
\end{pgfscope}%
\begin{pgfscope}%
\pgfsys@transformshift{3.915192in}{0.826249in}%
\pgfsys@useobject{currentmarker}{}%
\end{pgfscope}%
\begin{pgfscope}%
\pgfsys@transformshift{4.196433in}{0.807723in}%
\pgfsys@useobject{currentmarker}{}%
\end{pgfscope}%
\end{pgfscope}%
\begin{pgfscope}%
\pgfsetroundcap%
\pgfsetroundjoin%
\pgfsetlinewidth{1.003750pt}%
\definecolor{currentstroke}{rgb}{0.835294,0.368627,0.000000}%
\pgfsetstrokecolor{currentstroke}%
\pgfsetdash{}{0pt}%
\pgfpathmoveto{\pgfqpoint{0.745887in}{0.533676in}}%
\pgfpathlineto{\pgfqpoint{0.818221in}{0.531936in}}%
\pgfpathlineto{\pgfqpoint{0.889755in}{0.532721in}}%
\pgfpathlineto{\pgfqpoint{0.961613in}{0.532100in}}%
\pgfpathlineto{\pgfqpoint{1.033408in}{0.532686in}}%
\pgfpathlineto{\pgfqpoint{1.105273in}{0.532843in}}%
\pgfpathlineto{\pgfqpoint{1.178107in}{0.534880in}}%
\pgfpathlineto{\pgfqpoint{1.249087in}{0.533242in}}%
\pgfpathlineto{\pgfqpoint{1.321208in}{0.531300in}}%
\pgfpathlineto{\pgfqpoint{1.393720in}{0.532464in}}%
\pgfpathlineto{\pgfqpoint{1.464776in}{0.531289in}}%
\pgfpathlineto{\pgfqpoint{1.536986in}{0.532371in}}%
\pgfpathlineto{\pgfqpoint{1.609100in}{0.535081in}}%
\pgfpathlineto{\pgfqpoint{1.680285in}{0.532872in}}%
\pgfpathlineto{\pgfqpoint{1.752199in}{0.597200in}}%
\pgfpathlineto{\pgfqpoint{1.825380in}{0.534241in}}%
\pgfpathlineto{\pgfqpoint{1.897092in}{0.532834in}}%
\pgfpathlineto{\pgfqpoint{1.968809in}{0.533528in}}%
\pgfpathlineto{\pgfqpoint{2.046601in}{0.532795in}}%
\pgfpathlineto{\pgfqpoint{2.106358in}{0.532509in}}%
\pgfpathlineto{\pgfqpoint{2.182585in}{0.532113in}}%
\pgfpathlineto{\pgfqpoint{2.261861in}{0.530178in}}%
\pgfpathlineto{\pgfqpoint{2.320009in}{0.529820in}}%
\pgfpathlineto{\pgfqpoint{2.387763in}{0.529791in}}%
\pgfpathlineto{\pgfqpoint{2.469343in}{0.529748in}}%
\pgfpathlineto{\pgfqpoint{2.757758in}{0.530480in}}%
\pgfpathlineto{\pgfqpoint{2.909278in}{0.531207in}}%
\pgfpathlineto{\pgfqpoint{3.045767in}{0.531570in}}%
\pgfpathlineto{\pgfqpoint{3.187923in}{0.581777in}}%
\pgfpathlineto{\pgfqpoint{3.339026in}{0.560985in}}%
\pgfpathlineto{\pgfqpoint{3.483329in}{0.577611in}}%
\pgfpathlineto{\pgfqpoint{3.617844in}{0.590429in}}%
\pgfpathlineto{\pgfqpoint{3.762273in}{0.595718in}}%
\pgfpathlineto{\pgfqpoint{3.915192in}{0.561421in}}%
\pgfpathlineto{\pgfqpoint{4.196433in}{0.571649in}}%
\pgfusepath{stroke}%
\end{pgfscope}%
\begin{pgfscope}%
\pgfsetbuttcap%
\pgfsetroundjoin%
\definecolor{currentfill}{rgb}{0.835294,0.368627,0.000000}%
\pgfsetfillcolor{currentfill}%
\pgfsetlinewidth{0.752812pt}%
\definecolor{currentstroke}{rgb}{1.000000,1.000000,1.000000}%
\pgfsetstrokecolor{currentstroke}%
\pgfsetdash{}{0pt}%
\pgfsys@defobject{currentmarker}{\pgfqpoint{-0.034722in}{-0.034722in}}{\pgfqpoint{0.034722in}{0.034722in}}{%
\pgfpathmoveto{\pgfqpoint{0.000000in}{-0.034722in}}%
\pgfpathcurveto{\pgfqpoint{0.009208in}{-0.034722in}}{\pgfqpoint{0.018041in}{-0.031064in}}{\pgfqpoint{0.024552in}{-0.024552in}}%
\pgfpathcurveto{\pgfqpoint{0.031064in}{-0.018041in}}{\pgfqpoint{0.034722in}{-0.009208in}}{\pgfqpoint{0.034722in}{0.000000in}}%
\pgfpathcurveto{\pgfqpoint{0.034722in}{0.009208in}}{\pgfqpoint{0.031064in}{0.018041in}}{\pgfqpoint{0.024552in}{0.024552in}}%
\pgfpathcurveto{\pgfqpoint{0.018041in}{0.031064in}}{\pgfqpoint{0.009208in}{0.034722in}}{\pgfqpoint{0.000000in}{0.034722in}}%
\pgfpathcurveto{\pgfqpoint{-0.009208in}{0.034722in}}{\pgfqpoint{-0.018041in}{0.031064in}}{\pgfqpoint{-0.024552in}{0.024552in}}%
\pgfpathcurveto{\pgfqpoint{-0.031064in}{0.018041in}}{\pgfqpoint{-0.034722in}{0.009208in}}{\pgfqpoint{-0.034722in}{0.000000in}}%
\pgfpathcurveto{\pgfqpoint{-0.034722in}{-0.009208in}}{\pgfqpoint{-0.031064in}{-0.018041in}}{\pgfqpoint{-0.024552in}{-0.024552in}}%
\pgfpathcurveto{\pgfqpoint{-0.018041in}{-0.031064in}}{\pgfqpoint{-0.009208in}{-0.034722in}}{\pgfqpoint{0.000000in}{-0.034722in}}%
\pgfpathlineto{\pgfqpoint{0.000000in}{-0.034722in}}%
\pgfpathclose%
\pgfusepath{stroke,fill}%
}%
\begin{pgfscope}%
\pgfsys@transformshift{0.745887in}{0.533676in}%
\pgfsys@useobject{currentmarker}{}%
\end{pgfscope}%
\begin{pgfscope}%
\pgfsys@transformshift{0.818221in}{0.531936in}%
\pgfsys@useobject{currentmarker}{}%
\end{pgfscope}%
\begin{pgfscope}%
\pgfsys@transformshift{0.889755in}{0.532721in}%
\pgfsys@useobject{currentmarker}{}%
\end{pgfscope}%
\begin{pgfscope}%
\pgfsys@transformshift{0.961613in}{0.532100in}%
\pgfsys@useobject{currentmarker}{}%
\end{pgfscope}%
\begin{pgfscope}%
\pgfsys@transformshift{1.033408in}{0.532686in}%
\pgfsys@useobject{currentmarker}{}%
\end{pgfscope}%
\begin{pgfscope}%
\pgfsys@transformshift{1.105273in}{0.532843in}%
\pgfsys@useobject{currentmarker}{}%
\end{pgfscope}%
\begin{pgfscope}%
\pgfsys@transformshift{1.178107in}{0.534880in}%
\pgfsys@useobject{currentmarker}{}%
\end{pgfscope}%
\begin{pgfscope}%
\pgfsys@transformshift{1.249087in}{0.533242in}%
\pgfsys@useobject{currentmarker}{}%
\end{pgfscope}%
\begin{pgfscope}%
\pgfsys@transformshift{1.321208in}{0.531300in}%
\pgfsys@useobject{currentmarker}{}%
\end{pgfscope}%
\begin{pgfscope}%
\pgfsys@transformshift{1.393720in}{0.532464in}%
\pgfsys@useobject{currentmarker}{}%
\end{pgfscope}%
\begin{pgfscope}%
\pgfsys@transformshift{1.464776in}{0.531289in}%
\pgfsys@useobject{currentmarker}{}%
\end{pgfscope}%
\begin{pgfscope}%
\pgfsys@transformshift{1.536986in}{0.532371in}%
\pgfsys@useobject{currentmarker}{}%
\end{pgfscope}%
\begin{pgfscope}%
\pgfsys@transformshift{1.609100in}{0.535081in}%
\pgfsys@useobject{currentmarker}{}%
\end{pgfscope}%
\begin{pgfscope}%
\pgfsys@transformshift{1.680285in}{0.532872in}%
\pgfsys@useobject{currentmarker}{}%
\end{pgfscope}%
\begin{pgfscope}%
\pgfsys@transformshift{1.752199in}{0.597200in}%
\pgfsys@useobject{currentmarker}{}%
\end{pgfscope}%
\begin{pgfscope}%
\pgfsys@transformshift{1.825380in}{0.534241in}%
\pgfsys@useobject{currentmarker}{}%
\end{pgfscope}%
\begin{pgfscope}%
\pgfsys@transformshift{1.897092in}{0.532834in}%
\pgfsys@useobject{currentmarker}{}%
\end{pgfscope}%
\begin{pgfscope}%
\pgfsys@transformshift{1.968809in}{0.533528in}%
\pgfsys@useobject{currentmarker}{}%
\end{pgfscope}%
\begin{pgfscope}%
\pgfsys@transformshift{2.046601in}{0.532795in}%
\pgfsys@useobject{currentmarker}{}%
\end{pgfscope}%
\begin{pgfscope}%
\pgfsys@transformshift{2.106358in}{0.532509in}%
\pgfsys@useobject{currentmarker}{}%
\end{pgfscope}%
\begin{pgfscope}%
\pgfsys@transformshift{2.182585in}{0.532113in}%
\pgfsys@useobject{currentmarker}{}%
\end{pgfscope}%
\begin{pgfscope}%
\pgfsys@transformshift{2.261861in}{0.530178in}%
\pgfsys@useobject{currentmarker}{}%
\end{pgfscope}%
\begin{pgfscope}%
\pgfsys@transformshift{2.320009in}{0.529820in}%
\pgfsys@useobject{currentmarker}{}%
\end{pgfscope}%
\begin{pgfscope}%
\pgfsys@transformshift{2.387763in}{0.529791in}%
\pgfsys@useobject{currentmarker}{}%
\end{pgfscope}%
\begin{pgfscope}%
\pgfsys@transformshift{2.469343in}{0.529748in}%
\pgfsys@useobject{currentmarker}{}%
\end{pgfscope}%
\begin{pgfscope}%
\pgfsys@transformshift{2.757758in}{0.530480in}%
\pgfsys@useobject{currentmarker}{}%
\end{pgfscope}%
\begin{pgfscope}%
\pgfsys@transformshift{2.909278in}{0.531207in}%
\pgfsys@useobject{currentmarker}{}%
\end{pgfscope}%
\begin{pgfscope}%
\pgfsys@transformshift{3.045767in}{0.531570in}%
\pgfsys@useobject{currentmarker}{}%
\end{pgfscope}%
\begin{pgfscope}%
\pgfsys@transformshift{3.187923in}{0.581777in}%
\pgfsys@useobject{currentmarker}{}%
\end{pgfscope}%
\begin{pgfscope}%
\pgfsys@transformshift{3.339026in}{0.560985in}%
\pgfsys@useobject{currentmarker}{}%
\end{pgfscope}%
\begin{pgfscope}%
\pgfsys@transformshift{3.483329in}{0.577611in}%
\pgfsys@useobject{currentmarker}{}%
\end{pgfscope}%
\begin{pgfscope}%
\pgfsys@transformshift{3.617844in}{0.590429in}%
\pgfsys@useobject{currentmarker}{}%
\end{pgfscope}%
\begin{pgfscope}%
\pgfsys@transformshift{3.762273in}{0.595718in}%
\pgfsys@useobject{currentmarker}{}%
\end{pgfscope}%
\begin{pgfscope}%
\pgfsys@transformshift{3.915192in}{0.561421in}%
\pgfsys@useobject{currentmarker}{}%
\end{pgfscope}%
\begin{pgfscope}%
\pgfsys@transformshift{4.196433in}{0.571649in}%
\pgfsys@useobject{currentmarker}{}%
\end{pgfscope}%
\end{pgfscope}%
\end{pgfpicture}%
\makeatother%
\endgroup%

						\end{figcenter}
						\caption{Each task during routing of the largest dataset with 45018 input vertices.}
						\label{fig:eval-city-routing-details-b}
					\end{subfigure}
					\\[3ex]
					\begin{subfigure}[t]{\textwidth}
						\begin{figcenter}
							\begingroup%
\makeatletter%
\begin{pgfpicture}%
\pgfpathrectangle{\pgfpointorigin}{\pgfqpoint{6.079699in}{1.715788in}}%
\pgfusepath{use as bounding box}%
\begin{pgfscope}%
\pgfsetbuttcap%
\pgfsetmiterjoin%
\definecolor{currentfill}{rgb}{1.000000,1.000000,1.000000}%
\pgfsetfillcolor{currentfill}%
\pgfsetlinewidth{0.000000pt}%
\definecolor{currentstroke}{rgb}{1.000000,1.000000,1.000000}%
\pgfsetstrokecolor{currentstroke}%
\pgfsetdash{}{0pt}%
\pgfpathmoveto{\pgfqpoint{0.000000in}{0.000000in}}%
\pgfpathlineto{\pgfqpoint{6.079699in}{0.000000in}}%
\pgfpathlineto{\pgfqpoint{6.079699in}{1.715788in}}%
\pgfpathlineto{\pgfqpoint{0.000000in}{1.715788in}}%
\pgfpathlineto{\pgfqpoint{0.000000in}{0.000000in}}%
\pgfpathclose%
\pgfusepath{fill}%
\end{pgfscope}%
\begin{pgfscope}%
\pgfsetbuttcap%
\pgfsetmiterjoin%
\definecolor{currentfill}{rgb}{1.000000,1.000000,1.000000}%
\pgfsetfillcolor{currentfill}%
\pgfsetlinewidth{0.000000pt}%
\definecolor{currentstroke}{rgb}{0.000000,0.000000,0.000000}%
\pgfsetstrokecolor{currentstroke}%
\pgfsetstrokeopacity{0.000000}%
\pgfsetdash{}{0pt}%
\pgfpathmoveto{\pgfqpoint{0.566440in}{0.451389in}}%
\pgfpathlineto{\pgfqpoint{4.375358in}{0.451389in}}%
\pgfpathlineto{\pgfqpoint{4.375358in}{1.715788in}}%
\pgfpathlineto{\pgfqpoint{0.566440in}{1.715788in}}%
\pgfpathlineto{\pgfqpoint{0.566440in}{0.451389in}}%
\pgfpathclose%
\pgfusepath{fill}%
\end{pgfscope}%
\begin{pgfscope}%
\pgfpathrectangle{\pgfqpoint{0.566440in}{0.451389in}}{\pgfqpoint{3.808918in}{1.264399in}}%
\pgfusepath{clip}%
\pgfsetroundcap%
\pgfsetroundjoin%
\pgfsetlinewidth{1.003750pt}%
\definecolor{currentstroke}{rgb}{0.800000,0.800000,0.800000}%
\pgfsetstrokecolor{currentstroke}%
\pgfsetdash{}{0pt}%
\pgfpathmoveto{\pgfqpoint{0.566440in}{0.451389in}}%
\pgfpathlineto{\pgfqpoint{0.566440in}{1.715788in}}%
\pgfusepath{stroke}%
\end{pgfscope}%
\begin{pgfscope}%
\definecolor{textcolor}{rgb}{0.150000,0.150000,0.150000}%
\pgfsetstrokecolor{textcolor}%
\pgfsetfillcolor{textcolor}%
\pgftext[x=0.566440in,y=0.319444in,,top]{\color{textcolor}\sffamily\fontsize{9.000000}{10.800000}\selectfont 0.0}%
\end{pgfscope}%
\begin{pgfscope}%
\pgfpathrectangle{\pgfqpoint{0.566440in}{0.451389in}}{\pgfqpoint{3.808918in}{1.264399in}}%
\pgfusepath{clip}%
\pgfsetroundcap%
\pgfsetroundjoin%
\pgfsetlinewidth{1.003750pt}%
\definecolor{currentstroke}{rgb}{0.800000,0.800000,0.800000}%
\pgfsetstrokecolor{currentstroke}%
\pgfsetdash{}{0pt}%
\pgfpathmoveto{\pgfqpoint{1.292828in}{0.451389in}}%
\pgfpathlineto{\pgfqpoint{1.292828in}{1.715788in}}%
\pgfusepath{stroke}%
\end{pgfscope}%
\begin{pgfscope}%
\definecolor{textcolor}{rgb}{0.150000,0.150000,0.150000}%
\pgfsetstrokecolor{textcolor}%
\pgfsetfillcolor{textcolor}%
\pgftext[x=1.292828in,y=0.319444in,,top]{\color{textcolor}\sffamily\fontsize{9.000000}{10.800000}\selectfont 0.5}%
\end{pgfscope}%
\begin{pgfscope}%
\pgfpathrectangle{\pgfqpoint{0.566440in}{0.451389in}}{\pgfqpoint{3.808918in}{1.264399in}}%
\pgfusepath{clip}%
\pgfsetroundcap%
\pgfsetroundjoin%
\pgfsetlinewidth{1.003750pt}%
\definecolor{currentstroke}{rgb}{0.800000,0.800000,0.800000}%
\pgfsetstrokecolor{currentstroke}%
\pgfsetdash{}{0pt}%
\pgfpathmoveto{\pgfqpoint{2.019217in}{0.451389in}}%
\pgfpathlineto{\pgfqpoint{2.019217in}{1.715788in}}%
\pgfusepath{stroke}%
\end{pgfscope}%
\begin{pgfscope}%
\definecolor{textcolor}{rgb}{0.150000,0.150000,0.150000}%
\pgfsetstrokecolor{textcolor}%
\pgfsetfillcolor{textcolor}%
\pgftext[x=2.019217in,y=0.319444in,,top]{\color{textcolor}\sffamily\fontsize{9.000000}{10.800000}\selectfont 1.0}%
\end{pgfscope}%
\begin{pgfscope}%
\pgfpathrectangle{\pgfqpoint{0.566440in}{0.451389in}}{\pgfqpoint{3.808918in}{1.264399in}}%
\pgfusepath{clip}%
\pgfsetroundcap%
\pgfsetroundjoin%
\pgfsetlinewidth{1.003750pt}%
\definecolor{currentstroke}{rgb}{0.800000,0.800000,0.800000}%
\pgfsetstrokecolor{currentstroke}%
\pgfsetdash{}{0pt}%
\pgfpathmoveto{\pgfqpoint{2.745605in}{0.451389in}}%
\pgfpathlineto{\pgfqpoint{2.745605in}{1.715788in}}%
\pgfusepath{stroke}%
\end{pgfscope}%
\begin{pgfscope}%
\definecolor{textcolor}{rgb}{0.150000,0.150000,0.150000}%
\pgfsetstrokecolor{textcolor}%
\pgfsetfillcolor{textcolor}%
\pgftext[x=2.745605in,y=0.319444in,,top]{\color{textcolor}\sffamily\fontsize{9.000000}{10.800000}\selectfont 1.5}%
\end{pgfscope}%
\begin{pgfscope}%
\pgfpathrectangle{\pgfqpoint{0.566440in}{0.451389in}}{\pgfqpoint{3.808918in}{1.264399in}}%
\pgfusepath{clip}%
\pgfsetroundcap%
\pgfsetroundjoin%
\pgfsetlinewidth{1.003750pt}%
\definecolor{currentstroke}{rgb}{0.800000,0.800000,0.800000}%
\pgfsetstrokecolor{currentstroke}%
\pgfsetdash{}{0pt}%
\pgfpathmoveto{\pgfqpoint{3.471993in}{0.451389in}}%
\pgfpathlineto{\pgfqpoint{3.471993in}{1.715788in}}%
\pgfusepath{stroke}%
\end{pgfscope}%
\begin{pgfscope}%
\definecolor{textcolor}{rgb}{0.150000,0.150000,0.150000}%
\pgfsetstrokecolor{textcolor}%
\pgfsetfillcolor{textcolor}%
\pgftext[x=3.471993in,y=0.319444in,,top]{\color{textcolor}\sffamily\fontsize{9.000000}{10.800000}\selectfont 2.0}%
\end{pgfscope}%
\begin{pgfscope}%
\pgfpathrectangle{\pgfqpoint{0.566440in}{0.451389in}}{\pgfqpoint{3.808918in}{1.264399in}}%
\pgfusepath{clip}%
\pgfsetroundcap%
\pgfsetroundjoin%
\pgfsetlinewidth{1.003750pt}%
\definecolor{currentstroke}{rgb}{0.800000,0.800000,0.800000}%
\pgfsetstrokecolor{currentstroke}%
\pgfsetdash{}{0pt}%
\pgfpathmoveto{\pgfqpoint{4.198381in}{0.451389in}}%
\pgfpathlineto{\pgfqpoint{4.198381in}{1.715788in}}%
\pgfusepath{stroke}%
\end{pgfscope}%
\begin{pgfscope}%
\definecolor{textcolor}{rgb}{0.150000,0.150000,0.150000}%
\pgfsetstrokecolor{textcolor}%
\pgfsetfillcolor{textcolor}%
\pgftext[x=4.198381in,y=0.319444in,,top]{\color{textcolor}\sffamily\fontsize{9.000000}{10.800000}\selectfont 2.5}%
\end{pgfscope}%
\begin{pgfscope}%
\definecolor{textcolor}{rgb}{0.150000,0.150000,0.150000}%
\pgfsetstrokecolor{textcolor}%
\pgfsetfillcolor{textcolor}%
\pgftext[x=2.470899in,y=0.125000in,,top]{\color{textcolor}\sffamily\fontsize{9.000000}{10.800000}\selectfont Beeline distance in km}%
\end{pgfscope}%
\begin{pgfscope}%
\pgfpathrectangle{\pgfqpoint{0.566440in}{0.451389in}}{\pgfqpoint{3.808918in}{1.264399in}}%
\pgfusepath{clip}%
\pgfsetroundcap%
\pgfsetroundjoin%
\pgfsetlinewidth{1.003750pt}%
\definecolor{currentstroke}{rgb}{0.800000,0.800000,0.800000}%
\pgfsetstrokecolor{currentstroke}%
\pgfsetdash{}{0pt}%
\pgfpathmoveto{\pgfqpoint{0.566440in}{0.451389in}}%
\pgfpathlineto{\pgfqpoint{4.375358in}{0.451389in}}%
\pgfusepath{stroke}%
\end{pgfscope}%
\begin{pgfscope}%
\definecolor{textcolor}{rgb}{0.150000,0.150000,0.150000}%
\pgfsetstrokecolor{textcolor}%
\pgfsetfillcolor{textcolor}%
\pgftext[x=0.194444in, y=0.403903in, left, base]{\color{textcolor}\sffamily\fontsize{9.000000}{10.800000}\selectfont 0.00}%
\end{pgfscope}%
\begin{pgfscope}%
\pgfpathrectangle{\pgfqpoint{0.566440in}{0.451389in}}{\pgfqpoint{3.808918in}{1.264399in}}%
\pgfusepath{clip}%
\pgfsetroundcap%
\pgfsetroundjoin%
\pgfsetlinewidth{1.003750pt}%
\definecolor{currentstroke}{rgb}{0.800000,0.800000,0.800000}%
\pgfsetstrokecolor{currentstroke}%
\pgfsetdash{}{0pt}%
\pgfpathmoveto{\pgfqpoint{0.566440in}{0.752494in}}%
\pgfpathlineto{\pgfqpoint{4.375358in}{0.752494in}}%
\pgfusepath{stroke}%
\end{pgfscope}%
\begin{pgfscope}%
\definecolor{textcolor}{rgb}{0.150000,0.150000,0.150000}%
\pgfsetstrokecolor{textcolor}%
\pgfsetfillcolor{textcolor}%
\pgftext[x=0.194444in, y=0.705009in, left, base]{\color{textcolor}\sffamily\fontsize{9.000000}{10.800000}\selectfont 0.25}%
\end{pgfscope}%
\begin{pgfscope}%
\pgfpathrectangle{\pgfqpoint{0.566440in}{0.451389in}}{\pgfqpoint{3.808918in}{1.264399in}}%
\pgfusepath{clip}%
\pgfsetroundcap%
\pgfsetroundjoin%
\pgfsetlinewidth{1.003750pt}%
\definecolor{currentstroke}{rgb}{0.800000,0.800000,0.800000}%
\pgfsetstrokecolor{currentstroke}%
\pgfsetdash{}{0pt}%
\pgfpathmoveto{\pgfqpoint{0.566440in}{1.053600in}}%
\pgfpathlineto{\pgfqpoint{4.375358in}{1.053600in}}%
\pgfusepath{stroke}%
\end{pgfscope}%
\begin{pgfscope}%
\definecolor{textcolor}{rgb}{0.150000,0.150000,0.150000}%
\pgfsetstrokecolor{textcolor}%
\pgfsetfillcolor{textcolor}%
\pgftext[x=0.194444in, y=1.006115in, left, base]{\color{textcolor}\sffamily\fontsize{9.000000}{10.800000}\selectfont 0.50}%
\end{pgfscope}%
\begin{pgfscope}%
\pgfpathrectangle{\pgfqpoint{0.566440in}{0.451389in}}{\pgfqpoint{3.808918in}{1.264399in}}%
\pgfusepath{clip}%
\pgfsetroundcap%
\pgfsetroundjoin%
\pgfsetlinewidth{1.003750pt}%
\definecolor{currentstroke}{rgb}{0.800000,0.800000,0.800000}%
\pgfsetstrokecolor{currentstroke}%
\pgfsetdash{}{0pt}%
\pgfpathmoveto{\pgfqpoint{0.566440in}{1.354706in}}%
\pgfpathlineto{\pgfqpoint{4.375358in}{1.354706in}}%
\pgfusepath{stroke}%
\end{pgfscope}%
\begin{pgfscope}%
\definecolor{textcolor}{rgb}{0.150000,0.150000,0.150000}%
\pgfsetstrokecolor{textcolor}%
\pgfsetfillcolor{textcolor}%
\pgftext[x=0.194444in, y=1.307221in, left, base]{\color{textcolor}\sffamily\fontsize{9.000000}{10.800000}\selectfont 0.75}%
\end{pgfscope}%
\begin{pgfscope}%
\pgfpathrectangle{\pgfqpoint{0.566440in}{0.451389in}}{\pgfqpoint{3.808918in}{1.264399in}}%
\pgfusepath{clip}%
\pgfsetroundcap%
\pgfsetroundjoin%
\pgfsetlinewidth{1.003750pt}%
\definecolor{currentstroke}{rgb}{0.800000,0.800000,0.800000}%
\pgfsetstrokecolor{currentstroke}%
\pgfsetdash{}{0pt}%
\pgfpathmoveto{\pgfqpoint{0.566440in}{1.655812in}}%
\pgfpathlineto{\pgfqpoint{4.375358in}{1.655812in}}%
\pgfusepath{stroke}%
\end{pgfscope}%
\begin{pgfscope}%
\definecolor{textcolor}{rgb}{0.150000,0.150000,0.150000}%
\pgfsetstrokecolor{textcolor}%
\pgfsetfillcolor{textcolor}%
\pgftext[x=0.194444in, y=1.608327in, left, base]{\color{textcolor}\sffamily\fontsize{9.000000}{10.800000}\selectfont 1.00}%
\end{pgfscope}%
\begin{pgfscope}%
\definecolor{textcolor}{rgb}{0.150000,0.150000,0.150000}%
\pgfsetstrokecolor{textcolor}%
\pgfsetfillcolor{textcolor}%
\pgftext[x=0.125000in,y=1.083588in,,bottom,rotate=90.000000]{\color{textcolor}\sffamily\fontsize{9.000000}{10.800000}\selectfont Share of total time}%
\end{pgfscope}%
\begin{pgfscope}%
\pgfsetrectcap%
\pgfsetmiterjoin%
\pgfsetlinewidth{1.254687pt}%
\definecolor{currentstroke}{rgb}{0.800000,0.800000,0.800000}%
\pgfsetstrokecolor{currentstroke}%
\pgfsetdash{}{0pt}%
\pgfpathmoveto{\pgfqpoint{0.566440in}{0.451389in}}%
\pgfpathlineto{\pgfqpoint{0.566440in}{1.715788in}}%
\pgfusepath{stroke}%
\end{pgfscope}%
\begin{pgfscope}%
\pgfsetrectcap%
\pgfsetmiterjoin%
\pgfsetlinewidth{1.254687pt}%
\definecolor{currentstroke}{rgb}{0.800000,0.800000,0.800000}%
\pgfsetstrokecolor{currentstroke}%
\pgfsetdash{}{0pt}%
\pgfpathmoveto{\pgfqpoint{4.375358in}{0.451389in}}%
\pgfpathlineto{\pgfqpoint{4.375358in}{1.715788in}}%
\pgfusepath{stroke}%
\end{pgfscope}%
\begin{pgfscope}%
\pgfsetrectcap%
\pgfsetmiterjoin%
\pgfsetlinewidth{1.254687pt}%
\definecolor{currentstroke}{rgb}{0.800000,0.800000,0.800000}%
\pgfsetstrokecolor{currentstroke}%
\pgfsetdash{}{0pt}%
\pgfpathmoveto{\pgfqpoint{0.566440in}{0.451389in}}%
\pgfpathlineto{\pgfqpoint{4.375358in}{0.451389in}}%
\pgfusepath{stroke}%
\end{pgfscope}%
\begin{pgfscope}%
\pgfsetrectcap%
\pgfsetmiterjoin%
\pgfsetlinewidth{1.254687pt}%
\definecolor{currentstroke}{rgb}{0.800000,0.800000,0.800000}%
\pgfsetstrokecolor{currentstroke}%
\pgfsetdash{}{0pt}%
\pgfpathmoveto{\pgfqpoint{0.566440in}{1.715788in}}%
\pgfpathlineto{\pgfqpoint{4.375358in}{1.715788in}}%
\pgfusepath{stroke}%
\end{pgfscope}%
\begin{pgfscope}%
\pgfsetbuttcap%
\pgfsetmiterjoin%
\definecolor{currentfill}{rgb}{1.000000,1.000000,1.000000}%
\pgfsetfillcolor{currentfill}%
\pgfsetfillopacity{0.800000}%
\pgfsetlinewidth{1.003750pt}%
\definecolor{currentstroke}{rgb}{0.800000,0.800000,0.800000}%
\pgfsetstrokecolor{currentstroke}%
\pgfsetstrokeopacity{0.800000}%
\pgfsetdash{}{0pt}%
\pgfpathmoveto{\pgfqpoint{4.558081in}{0.524092in}}%
\pgfpathlineto{\pgfqpoint{6.054699in}{0.524092in}}%
\pgfpathquadraticcurveto{\pgfqpoint{6.079699in}{0.524092in}}{\pgfqpoint{6.079699in}{0.549092in}}%
\pgfpathlineto{\pgfqpoint{6.079699in}{1.618085in}}%
\pgfpathquadraticcurveto{\pgfqpoint{6.079699in}{1.643085in}}{\pgfqpoint{6.054699in}{1.643085in}}%
\pgfpathlineto{\pgfqpoint{4.558081in}{1.643085in}}%
\pgfpathquadraticcurveto{\pgfqpoint{4.533081in}{1.643085in}}{\pgfqpoint{4.533081in}{1.618085in}}%
\pgfpathlineto{\pgfqpoint{4.533081in}{0.549092in}}%
\pgfpathquadraticcurveto{\pgfqpoint{4.533081in}{0.524092in}}{\pgfqpoint{4.558081in}{0.524092in}}%
\pgfpathlineto{\pgfqpoint{4.558081in}{0.524092in}}%
\pgfpathclose%
\pgfusepath{stroke,fill}%
\end{pgfscope}%
\begin{pgfscope}%
\definecolor{textcolor}{rgb}{0.150000,0.150000,0.150000}%
\pgfsetstrokecolor{textcolor}%
\pgfsetfillcolor{textcolor}%
\pgftext[x=5.102991in,y=1.498114in,left,base]{\color{textcolor}\sffamily\fontsize{9.000000}{10.800000}\selectfont Legend}%
\end{pgfscope}%
\begin{pgfscope}%
\pgfsetroundcap%
\pgfsetroundjoin%
\pgfsetlinewidth{1.505625pt}%
\definecolor{currentstroke}{rgb}{0.003922,0.450980,0.698039}%
\pgfsetstrokecolor{currentstroke}%
\pgfsetdash{}{0pt}%
\pgfpathmoveto{\pgfqpoint{4.583081in}{1.354364in}}%
\pgfpathlineto{\pgfqpoint{4.708081in}{1.354364in}}%
\pgfpathlineto{\pgfqpoint{4.833081in}{1.354364in}}%
\pgfusepath{stroke}%
\end{pgfscope}%
\begin{pgfscope}%
\definecolor{textcolor}{rgb}{0.150000,0.150000,0.150000}%
\pgfsetstrokecolor{textcolor}%
\pgfsetfillcolor{textcolor}%
\pgftext[x=4.933081in,y=1.310614in,left,base]{\color{textcolor}\sffamily\fontsize{9.000000}{10.800000}\selectfont Total time}%
\end{pgfscope}%
\begin{pgfscope}%
\pgfsetroundcap%
\pgfsetroundjoin%
\pgfsetlinewidth{1.505625pt}%
\definecolor{currentstroke}{rgb}{0.870588,0.560784,0.019608}%
\pgfsetstrokecolor{currentstroke}%
\pgfsetdash{}{0pt}%
\pgfpathmoveto{\pgfqpoint{4.583081in}{1.166864in}}%
\pgfpathlineto{\pgfqpoint{4.708081in}{1.166864in}}%
\pgfpathlineto{\pgfqpoint{4.833081in}{1.166864in}}%
\pgfusepath{stroke}%
\end{pgfscope}%
\begin{pgfscope}%
\definecolor{textcolor}{rgb}{0.150000,0.150000,0.150000}%
\pgfsetstrokecolor{textcolor}%
\pgfsetfillcolor{textcolor}%
\pgftext[x=4.933081in,y=1.123114in,left,base]{\color{textcolor}\sffamily\fontsize{9.000000}{10.800000}\selectfont A* routing}%
\end{pgfscope}%
\begin{pgfscope}%
\pgfsetroundcap%
\pgfsetroundjoin%
\pgfsetlinewidth{1.505625pt}%
\definecolor{currentstroke}{rgb}{0.007843,0.619608,0.450980}%
\pgfsetstrokecolor{currentstroke}%
\pgfsetdash{}{0pt}%
\pgfpathmoveto{\pgfqpoint{4.583081in}{0.892353in}}%
\pgfpathlineto{\pgfqpoint{4.708081in}{0.892353in}}%
\pgfpathlineto{\pgfqpoint{4.833081in}{0.892353in}}%
\pgfusepath{stroke}%
\end{pgfscope}%
\begin{pgfscope}%
\definecolor{textcolor}{rgb}{0.150000,0.150000,0.150000}%
\pgfsetstrokecolor{textcolor}%
\pgfsetfillcolor{textcolor}%
\pgftext[x=4.933081in, y=0.935615in, left, base]{\color{textcolor}\sffamily\fontsize{9.000000}{10.800000}\selectfont Connect source \&}%
\end{pgfscope}%
\begin{pgfscope}%
\definecolor{textcolor}{rgb}{0.150000,0.150000,0.150000}%
\pgfsetstrokecolor{textcolor}%
\pgfsetfillcolor{textcolor}%
\pgftext[x=4.933081in, y=0.791621in, left, base]{\color{textcolor}\sffamily\fontsize{9.000000}{10.800000}\selectfont destination vertices}%
\end{pgfscope}%
\begin{pgfscope}%
\pgfsetroundcap%
\pgfsetroundjoin%
\pgfsetlinewidth{1.505625pt}%
\definecolor{currentstroke}{rgb}{0.835294,0.368627,0.000000}%
\pgfsetstrokecolor{currentstroke}%
\pgfsetdash{}{0pt}%
\pgfpathmoveto{\pgfqpoint{4.583081in}{0.647871in}}%
\pgfpathlineto{\pgfqpoint{4.708081in}{0.647871in}}%
\pgfpathlineto{\pgfqpoint{4.833081in}{0.647871in}}%
\pgfusepath{stroke}%
\end{pgfscope}%
\begin{pgfscope}%
\definecolor{textcolor}{rgb}{0.150000,0.150000,0.150000}%
\pgfsetstrokecolor{textcolor}%
\pgfsetfillcolor{textcolor}%
\pgftext[x=4.933081in,y=0.604121in,left,base]{\color{textcolor}\sffamily\fontsize{9.000000}{10.800000}\selectfont Restoring graph}%
\end{pgfscope}%
\begin{pgfscope}%
\pgfsetroundcap%
\pgfsetroundjoin%
\pgfsetlinewidth{1.003750pt}%
\definecolor{currentstroke}{rgb}{0.003922,0.450980,0.698039}%
\pgfsetstrokecolor{currentstroke}%
\pgfsetdash{}{0pt}%
\pgfpathmoveto{\pgfqpoint{0.711866in}{1.655812in}}%
\pgfpathlineto{\pgfqpoint{0.784862in}{1.655812in}}%
\pgfpathlineto{\pgfqpoint{0.857051in}{1.655812in}}%
\pgfpathlineto{\pgfqpoint{0.929566in}{1.655812in}}%
\pgfpathlineto{\pgfqpoint{1.002017in}{1.655812in}}%
\pgfpathlineto{\pgfqpoint{1.074540in}{1.655812in}}%
\pgfpathlineto{\pgfqpoint{1.148040in}{1.655812in}}%
\pgfpathlineto{\pgfqpoint{1.219669in}{1.655812in}}%
\pgfpathlineto{\pgfqpoint{1.292451in}{1.655812in}}%
\pgfpathlineto{\pgfqpoint{1.365625in}{1.655812in}}%
\pgfpathlineto{\pgfqpoint{1.437332in}{1.655812in}}%
\pgfpathlineto{\pgfqpoint{1.510203in}{1.655812in}}%
\pgfpathlineto{\pgfqpoint{1.582977in}{1.655812in}}%
\pgfpathlineto{\pgfqpoint{1.654812in}{1.655812in}}%
\pgfpathlineto{\pgfqpoint{1.727384in}{1.655812in}}%
\pgfpathlineto{\pgfqpoint{1.801235in}{1.655812in}}%
\pgfpathlineto{\pgfqpoint{1.873603in}{1.655812in}}%
\pgfpathlineto{\pgfqpoint{1.945976in}{1.655812in}}%
\pgfpathlineto{\pgfqpoint{2.024480in}{1.655812in}}%
\pgfpathlineto{\pgfqpoint{2.084783in}{1.655812in}}%
\pgfpathlineto{\pgfqpoint{2.161708in}{1.655812in}}%
\pgfpathlineto{\pgfqpoint{2.241709in}{1.655812in}}%
\pgfpathlineto{\pgfqpoint{2.307561in}{1.655812in}}%
\pgfpathlineto{\pgfqpoint{2.373263in}{1.655812in}}%
\pgfpathlineto{\pgfqpoint{2.451089in}{1.655812in}}%
\pgfpathlineto{\pgfqpoint{2.742143in}{1.655812in}}%
\pgfpathlineto{\pgfqpoint{2.895049in}{1.655812in}}%
\pgfpathlineto{\pgfqpoint{3.032787in}{1.655812in}}%
\pgfpathlineto{\pgfqpoint{3.176244in}{1.655812in}}%
\pgfpathlineto{\pgfqpoint{3.328729in}{1.655812in}}%
\pgfpathlineto{\pgfqpoint{3.474352in}{1.655812in}}%
\pgfpathlineto{\pgfqpoint{3.610098in}{1.655812in}}%
\pgfpathlineto{\pgfqpoint{3.755848in}{1.655812in}}%
\pgfpathlineto{\pgfqpoint{3.910166in}{1.655812in}}%
\pgfpathlineto{\pgfqpoint{4.193981in}{1.655812in}}%
\pgfusepath{stroke}%
\end{pgfscope}%
\begin{pgfscope}%
\pgfsetbuttcap%
\pgfsetroundjoin%
\definecolor{currentfill}{rgb}{0.003922,0.450980,0.698039}%
\pgfsetfillcolor{currentfill}%
\pgfsetlinewidth{0.752812pt}%
\definecolor{currentstroke}{rgb}{1.000000,1.000000,1.000000}%
\pgfsetstrokecolor{currentstroke}%
\pgfsetdash{}{0pt}%
\pgfsys@defobject{currentmarker}{\pgfqpoint{-0.034722in}{-0.034722in}}{\pgfqpoint{0.034722in}{0.034722in}}{%
\pgfpathmoveto{\pgfqpoint{0.000000in}{-0.034722in}}%
\pgfpathcurveto{\pgfqpoint{0.009208in}{-0.034722in}}{\pgfqpoint{0.018041in}{-0.031064in}}{\pgfqpoint{0.024552in}{-0.024552in}}%
\pgfpathcurveto{\pgfqpoint{0.031064in}{-0.018041in}}{\pgfqpoint{0.034722in}{-0.009208in}}{\pgfqpoint{0.034722in}{0.000000in}}%
\pgfpathcurveto{\pgfqpoint{0.034722in}{0.009208in}}{\pgfqpoint{0.031064in}{0.018041in}}{\pgfqpoint{0.024552in}{0.024552in}}%
\pgfpathcurveto{\pgfqpoint{0.018041in}{0.031064in}}{\pgfqpoint{0.009208in}{0.034722in}}{\pgfqpoint{0.000000in}{0.034722in}}%
\pgfpathcurveto{\pgfqpoint{-0.009208in}{0.034722in}}{\pgfqpoint{-0.018041in}{0.031064in}}{\pgfqpoint{-0.024552in}{0.024552in}}%
\pgfpathcurveto{\pgfqpoint{-0.031064in}{0.018041in}}{\pgfqpoint{-0.034722in}{0.009208in}}{\pgfqpoint{-0.034722in}{0.000000in}}%
\pgfpathcurveto{\pgfqpoint{-0.034722in}{-0.009208in}}{\pgfqpoint{-0.031064in}{-0.018041in}}{\pgfqpoint{-0.024552in}{-0.024552in}}%
\pgfpathcurveto{\pgfqpoint{-0.018041in}{-0.031064in}}{\pgfqpoint{-0.009208in}{-0.034722in}}{\pgfqpoint{0.000000in}{-0.034722in}}%
\pgfpathlineto{\pgfqpoint{0.000000in}{-0.034722in}}%
\pgfpathclose%
\pgfusepath{stroke,fill}%
}%
\begin{pgfscope}%
\pgfsys@transformshift{0.711866in}{1.655812in}%
\pgfsys@useobject{currentmarker}{}%
\end{pgfscope}%
\begin{pgfscope}%
\pgfsys@transformshift{0.784862in}{1.655812in}%
\pgfsys@useobject{currentmarker}{}%
\end{pgfscope}%
\begin{pgfscope}%
\pgfsys@transformshift{0.857051in}{1.655812in}%
\pgfsys@useobject{currentmarker}{}%
\end{pgfscope}%
\begin{pgfscope}%
\pgfsys@transformshift{0.929566in}{1.655812in}%
\pgfsys@useobject{currentmarker}{}%
\end{pgfscope}%
\begin{pgfscope}%
\pgfsys@transformshift{1.002017in}{1.655812in}%
\pgfsys@useobject{currentmarker}{}%
\end{pgfscope}%
\begin{pgfscope}%
\pgfsys@transformshift{1.074540in}{1.655812in}%
\pgfsys@useobject{currentmarker}{}%
\end{pgfscope}%
\begin{pgfscope}%
\pgfsys@transformshift{1.148040in}{1.655812in}%
\pgfsys@useobject{currentmarker}{}%
\end{pgfscope}%
\begin{pgfscope}%
\pgfsys@transformshift{1.219669in}{1.655812in}%
\pgfsys@useobject{currentmarker}{}%
\end{pgfscope}%
\begin{pgfscope}%
\pgfsys@transformshift{1.292451in}{1.655812in}%
\pgfsys@useobject{currentmarker}{}%
\end{pgfscope}%
\begin{pgfscope}%
\pgfsys@transformshift{1.365625in}{1.655812in}%
\pgfsys@useobject{currentmarker}{}%
\end{pgfscope}%
\begin{pgfscope}%
\pgfsys@transformshift{1.437332in}{1.655812in}%
\pgfsys@useobject{currentmarker}{}%
\end{pgfscope}%
\begin{pgfscope}%
\pgfsys@transformshift{1.510203in}{1.655812in}%
\pgfsys@useobject{currentmarker}{}%
\end{pgfscope}%
\begin{pgfscope}%
\pgfsys@transformshift{1.582977in}{1.655812in}%
\pgfsys@useobject{currentmarker}{}%
\end{pgfscope}%
\begin{pgfscope}%
\pgfsys@transformshift{1.654812in}{1.655812in}%
\pgfsys@useobject{currentmarker}{}%
\end{pgfscope}%
\begin{pgfscope}%
\pgfsys@transformshift{1.727384in}{1.655812in}%
\pgfsys@useobject{currentmarker}{}%
\end{pgfscope}%
\begin{pgfscope}%
\pgfsys@transformshift{1.801235in}{1.655812in}%
\pgfsys@useobject{currentmarker}{}%
\end{pgfscope}%
\begin{pgfscope}%
\pgfsys@transformshift{1.873603in}{1.655812in}%
\pgfsys@useobject{currentmarker}{}%
\end{pgfscope}%
\begin{pgfscope}%
\pgfsys@transformshift{1.945976in}{1.655812in}%
\pgfsys@useobject{currentmarker}{}%
\end{pgfscope}%
\begin{pgfscope}%
\pgfsys@transformshift{2.024480in}{1.655812in}%
\pgfsys@useobject{currentmarker}{}%
\end{pgfscope}%
\begin{pgfscope}%
\pgfsys@transformshift{2.084783in}{1.655812in}%
\pgfsys@useobject{currentmarker}{}%
\end{pgfscope}%
\begin{pgfscope}%
\pgfsys@transformshift{2.161708in}{1.655812in}%
\pgfsys@useobject{currentmarker}{}%
\end{pgfscope}%
\begin{pgfscope}%
\pgfsys@transformshift{2.241709in}{1.655812in}%
\pgfsys@useobject{currentmarker}{}%
\end{pgfscope}%
\begin{pgfscope}%
\pgfsys@transformshift{2.307561in}{1.655812in}%
\pgfsys@useobject{currentmarker}{}%
\end{pgfscope}%
\begin{pgfscope}%
\pgfsys@transformshift{2.373263in}{1.655812in}%
\pgfsys@useobject{currentmarker}{}%
\end{pgfscope}%
\begin{pgfscope}%
\pgfsys@transformshift{2.451089in}{1.655812in}%
\pgfsys@useobject{currentmarker}{}%
\end{pgfscope}%
\begin{pgfscope}%
\pgfsys@transformshift{2.742143in}{1.655812in}%
\pgfsys@useobject{currentmarker}{}%
\end{pgfscope}%
\begin{pgfscope}%
\pgfsys@transformshift{2.895049in}{1.655812in}%
\pgfsys@useobject{currentmarker}{}%
\end{pgfscope}%
\begin{pgfscope}%
\pgfsys@transformshift{3.032787in}{1.655812in}%
\pgfsys@useobject{currentmarker}{}%
\end{pgfscope}%
\begin{pgfscope}%
\pgfsys@transformshift{3.176244in}{1.655812in}%
\pgfsys@useobject{currentmarker}{}%
\end{pgfscope}%
\begin{pgfscope}%
\pgfsys@transformshift{3.328729in}{1.655812in}%
\pgfsys@useobject{currentmarker}{}%
\end{pgfscope}%
\begin{pgfscope}%
\pgfsys@transformshift{3.474352in}{1.655812in}%
\pgfsys@useobject{currentmarker}{}%
\end{pgfscope}%
\begin{pgfscope}%
\pgfsys@transformshift{3.610098in}{1.655812in}%
\pgfsys@useobject{currentmarker}{}%
\end{pgfscope}%
\begin{pgfscope}%
\pgfsys@transformshift{3.755848in}{1.655812in}%
\pgfsys@useobject{currentmarker}{}%
\end{pgfscope}%
\begin{pgfscope}%
\pgfsys@transformshift{3.910166in}{1.655812in}%
\pgfsys@useobject{currentmarker}{}%
\end{pgfscope}%
\begin{pgfscope}%
\pgfsys@transformshift{4.193981in}{1.655812in}%
\pgfsys@useobject{currentmarker}{}%
\end{pgfscope}%
\end{pgfscope}%
\begin{pgfscope}%
\pgfsetroundcap%
\pgfsetroundjoin%
\pgfsetlinewidth{1.003750pt}%
\definecolor{currentstroke}{rgb}{0.870588,0.560784,0.019608}%
\pgfsetstrokecolor{currentstroke}%
\pgfsetdash{}{0pt}%
\pgfpathmoveto{\pgfqpoint{0.711866in}{0.456298in}}%
\pgfpathlineto{\pgfqpoint{0.784862in}{0.462034in}}%
\pgfpathlineto{\pgfqpoint{0.857051in}{0.470194in}}%
\pgfpathlineto{\pgfqpoint{0.929566in}{0.464279in}}%
\pgfpathlineto{\pgfqpoint{1.002017in}{0.502538in}}%
\pgfpathlineto{\pgfqpoint{1.074540in}{0.557992in}}%
\pgfpathlineto{\pgfqpoint{1.148040in}{0.515243in}}%
\pgfpathlineto{\pgfqpoint{1.219669in}{0.617383in}}%
\pgfpathlineto{\pgfqpoint{1.292451in}{0.547007in}}%
\pgfpathlineto{\pgfqpoint{1.365625in}{0.702549in}}%
\pgfpathlineto{\pgfqpoint{1.437332in}{0.580472in}}%
\pgfpathlineto{\pgfqpoint{1.510203in}{0.685444in}}%
\pgfpathlineto{\pgfqpoint{1.582977in}{0.663020in}}%
\pgfpathlineto{\pgfqpoint{1.654812in}{0.706354in}}%
\pgfpathlineto{\pgfqpoint{1.727384in}{0.847406in}}%
\pgfpathlineto{\pgfqpoint{1.801235in}{0.565727in}}%
\pgfpathlineto{\pgfqpoint{1.873603in}{0.785407in}}%
\pgfpathlineto{\pgfqpoint{1.945976in}{0.801729in}}%
\pgfpathlineto{\pgfqpoint{2.024480in}{0.688089in}}%
\pgfpathlineto{\pgfqpoint{2.084783in}{0.925939in}}%
\pgfpathlineto{\pgfqpoint{2.161708in}{0.727304in}}%
\pgfpathlineto{\pgfqpoint{2.241709in}{1.016883in}}%
\pgfpathlineto{\pgfqpoint{2.307561in}{0.828020in}}%
\pgfpathlineto{\pgfqpoint{2.373263in}{1.014660in}}%
\pgfpathlineto{\pgfqpoint{2.451089in}{0.853883in}}%
\pgfpathlineto{\pgfqpoint{2.742143in}{0.983537in}}%
\pgfpathlineto{\pgfqpoint{2.895049in}{0.912029in}}%
\pgfpathlineto{\pgfqpoint{3.032787in}{0.948922in}}%
\pgfpathlineto{\pgfqpoint{3.176244in}{0.985992in}}%
\pgfpathlineto{\pgfqpoint{3.328729in}{0.899788in}}%
\pgfpathlineto{\pgfqpoint{3.474352in}{0.893265in}}%
\pgfpathlineto{\pgfqpoint{3.610098in}{0.958626in}}%
\pgfpathlineto{\pgfqpoint{3.755848in}{1.002423in}}%
\pgfpathlineto{\pgfqpoint{3.910166in}{1.006947in}}%
\pgfpathlineto{\pgfqpoint{4.193981in}{1.012664in}}%
\pgfusepath{stroke}%
\end{pgfscope}%
\begin{pgfscope}%
\pgfsetbuttcap%
\pgfsetroundjoin%
\definecolor{currentfill}{rgb}{0.870588,0.560784,0.019608}%
\pgfsetfillcolor{currentfill}%
\pgfsetlinewidth{0.752812pt}%
\definecolor{currentstroke}{rgb}{1.000000,1.000000,1.000000}%
\pgfsetstrokecolor{currentstroke}%
\pgfsetdash{}{0pt}%
\pgfsys@defobject{currentmarker}{\pgfqpoint{-0.034722in}{-0.034722in}}{\pgfqpoint{0.034722in}{0.034722in}}{%
\pgfpathmoveto{\pgfqpoint{0.000000in}{-0.034722in}}%
\pgfpathcurveto{\pgfqpoint{0.009208in}{-0.034722in}}{\pgfqpoint{0.018041in}{-0.031064in}}{\pgfqpoint{0.024552in}{-0.024552in}}%
\pgfpathcurveto{\pgfqpoint{0.031064in}{-0.018041in}}{\pgfqpoint{0.034722in}{-0.009208in}}{\pgfqpoint{0.034722in}{0.000000in}}%
\pgfpathcurveto{\pgfqpoint{0.034722in}{0.009208in}}{\pgfqpoint{0.031064in}{0.018041in}}{\pgfqpoint{0.024552in}{0.024552in}}%
\pgfpathcurveto{\pgfqpoint{0.018041in}{0.031064in}}{\pgfqpoint{0.009208in}{0.034722in}}{\pgfqpoint{0.000000in}{0.034722in}}%
\pgfpathcurveto{\pgfqpoint{-0.009208in}{0.034722in}}{\pgfqpoint{-0.018041in}{0.031064in}}{\pgfqpoint{-0.024552in}{0.024552in}}%
\pgfpathcurveto{\pgfqpoint{-0.031064in}{0.018041in}}{\pgfqpoint{-0.034722in}{0.009208in}}{\pgfqpoint{-0.034722in}{0.000000in}}%
\pgfpathcurveto{\pgfqpoint{-0.034722in}{-0.009208in}}{\pgfqpoint{-0.031064in}{-0.018041in}}{\pgfqpoint{-0.024552in}{-0.024552in}}%
\pgfpathcurveto{\pgfqpoint{-0.018041in}{-0.031064in}}{\pgfqpoint{-0.009208in}{-0.034722in}}{\pgfqpoint{0.000000in}{-0.034722in}}%
\pgfpathlineto{\pgfqpoint{0.000000in}{-0.034722in}}%
\pgfpathclose%
\pgfusepath{stroke,fill}%
}%
\begin{pgfscope}%
\pgfsys@transformshift{0.711866in}{0.456298in}%
\pgfsys@useobject{currentmarker}{}%
\end{pgfscope}%
\begin{pgfscope}%
\pgfsys@transformshift{0.784862in}{0.462034in}%
\pgfsys@useobject{currentmarker}{}%
\end{pgfscope}%
\begin{pgfscope}%
\pgfsys@transformshift{0.857051in}{0.470194in}%
\pgfsys@useobject{currentmarker}{}%
\end{pgfscope}%
\begin{pgfscope}%
\pgfsys@transformshift{0.929566in}{0.464279in}%
\pgfsys@useobject{currentmarker}{}%
\end{pgfscope}%
\begin{pgfscope}%
\pgfsys@transformshift{1.002017in}{0.502538in}%
\pgfsys@useobject{currentmarker}{}%
\end{pgfscope}%
\begin{pgfscope}%
\pgfsys@transformshift{1.074540in}{0.557992in}%
\pgfsys@useobject{currentmarker}{}%
\end{pgfscope}%
\begin{pgfscope}%
\pgfsys@transformshift{1.148040in}{0.515243in}%
\pgfsys@useobject{currentmarker}{}%
\end{pgfscope}%
\begin{pgfscope}%
\pgfsys@transformshift{1.219669in}{0.617383in}%
\pgfsys@useobject{currentmarker}{}%
\end{pgfscope}%
\begin{pgfscope}%
\pgfsys@transformshift{1.292451in}{0.547007in}%
\pgfsys@useobject{currentmarker}{}%
\end{pgfscope}%
\begin{pgfscope}%
\pgfsys@transformshift{1.365625in}{0.702549in}%
\pgfsys@useobject{currentmarker}{}%
\end{pgfscope}%
\begin{pgfscope}%
\pgfsys@transformshift{1.437332in}{0.580472in}%
\pgfsys@useobject{currentmarker}{}%
\end{pgfscope}%
\begin{pgfscope}%
\pgfsys@transformshift{1.510203in}{0.685444in}%
\pgfsys@useobject{currentmarker}{}%
\end{pgfscope}%
\begin{pgfscope}%
\pgfsys@transformshift{1.582977in}{0.663020in}%
\pgfsys@useobject{currentmarker}{}%
\end{pgfscope}%
\begin{pgfscope}%
\pgfsys@transformshift{1.654812in}{0.706354in}%
\pgfsys@useobject{currentmarker}{}%
\end{pgfscope}%
\begin{pgfscope}%
\pgfsys@transformshift{1.727384in}{0.847406in}%
\pgfsys@useobject{currentmarker}{}%
\end{pgfscope}%
\begin{pgfscope}%
\pgfsys@transformshift{1.801235in}{0.565727in}%
\pgfsys@useobject{currentmarker}{}%
\end{pgfscope}%
\begin{pgfscope}%
\pgfsys@transformshift{1.873603in}{0.785407in}%
\pgfsys@useobject{currentmarker}{}%
\end{pgfscope}%
\begin{pgfscope}%
\pgfsys@transformshift{1.945976in}{0.801729in}%
\pgfsys@useobject{currentmarker}{}%
\end{pgfscope}%
\begin{pgfscope}%
\pgfsys@transformshift{2.024480in}{0.688089in}%
\pgfsys@useobject{currentmarker}{}%
\end{pgfscope}%
\begin{pgfscope}%
\pgfsys@transformshift{2.084783in}{0.925939in}%
\pgfsys@useobject{currentmarker}{}%
\end{pgfscope}%
\begin{pgfscope}%
\pgfsys@transformshift{2.161708in}{0.727304in}%
\pgfsys@useobject{currentmarker}{}%
\end{pgfscope}%
\begin{pgfscope}%
\pgfsys@transformshift{2.241709in}{1.016883in}%
\pgfsys@useobject{currentmarker}{}%
\end{pgfscope}%
\begin{pgfscope}%
\pgfsys@transformshift{2.307561in}{0.828020in}%
\pgfsys@useobject{currentmarker}{}%
\end{pgfscope}%
\begin{pgfscope}%
\pgfsys@transformshift{2.373263in}{1.014660in}%
\pgfsys@useobject{currentmarker}{}%
\end{pgfscope}%
\begin{pgfscope}%
\pgfsys@transformshift{2.451089in}{0.853883in}%
\pgfsys@useobject{currentmarker}{}%
\end{pgfscope}%
\begin{pgfscope}%
\pgfsys@transformshift{2.742143in}{0.983537in}%
\pgfsys@useobject{currentmarker}{}%
\end{pgfscope}%
\begin{pgfscope}%
\pgfsys@transformshift{2.895049in}{0.912029in}%
\pgfsys@useobject{currentmarker}{}%
\end{pgfscope}%
\begin{pgfscope}%
\pgfsys@transformshift{3.032787in}{0.948922in}%
\pgfsys@useobject{currentmarker}{}%
\end{pgfscope}%
\begin{pgfscope}%
\pgfsys@transformshift{3.176244in}{0.985992in}%
\pgfsys@useobject{currentmarker}{}%
\end{pgfscope}%
\begin{pgfscope}%
\pgfsys@transformshift{3.328729in}{0.899788in}%
\pgfsys@useobject{currentmarker}{}%
\end{pgfscope}%
\begin{pgfscope}%
\pgfsys@transformshift{3.474352in}{0.893265in}%
\pgfsys@useobject{currentmarker}{}%
\end{pgfscope}%
\begin{pgfscope}%
\pgfsys@transformshift{3.610098in}{0.958626in}%
\pgfsys@useobject{currentmarker}{}%
\end{pgfscope}%
\begin{pgfscope}%
\pgfsys@transformshift{3.755848in}{1.002423in}%
\pgfsys@useobject{currentmarker}{}%
\end{pgfscope}%
\begin{pgfscope}%
\pgfsys@transformshift{3.910166in}{1.006947in}%
\pgfsys@useobject{currentmarker}{}%
\end{pgfscope}%
\begin{pgfscope}%
\pgfsys@transformshift{4.193981in}{1.012664in}%
\pgfsys@useobject{currentmarker}{}%
\end{pgfscope}%
\end{pgfscope}%
\begin{pgfscope}%
\pgfsetroundcap%
\pgfsetroundjoin%
\pgfsetlinewidth{1.003750pt}%
\definecolor{currentstroke}{rgb}{0.007843,0.619608,0.450980}%
\pgfsetstrokecolor{currentstroke}%
\pgfsetdash{}{0pt}%
\pgfpathmoveto{\pgfqpoint{0.711866in}{1.469398in}}%
\pgfpathlineto{\pgfqpoint{0.784862in}{1.414431in}}%
\pgfpathlineto{\pgfqpoint{0.857051in}{1.429583in}}%
\pgfpathlineto{\pgfqpoint{0.929566in}{1.447289in}}%
\pgfpathlineto{\pgfqpoint{1.002017in}{1.398216in}}%
\pgfpathlineto{\pgfqpoint{1.074540in}{1.335088in}}%
\pgfpathlineto{\pgfqpoint{1.148040in}{1.425740in}}%
\pgfpathlineto{\pgfqpoint{1.219669in}{1.329309in}}%
\pgfpathlineto{\pgfqpoint{1.292451in}{1.366249in}}%
\pgfpathlineto{\pgfqpoint{1.365625in}{1.263837in}}%
\pgfpathlineto{\pgfqpoint{1.437332in}{1.345548in}}%
\pgfpathlineto{\pgfqpoint{1.510203in}{1.254375in}}%
\pgfpathlineto{\pgfqpoint{1.582977in}{1.225045in}}%
\pgfpathlineto{\pgfqpoint{1.654812in}{1.247182in}}%
\pgfpathlineto{\pgfqpoint{1.727384in}{1.096626in}}%
\pgfpathlineto{\pgfqpoint{1.801235in}{1.384939in}}%
\pgfpathlineto{\pgfqpoint{1.873603in}{1.055179in}}%
\pgfpathlineto{\pgfqpoint{1.945976in}{1.149361in}}%
\pgfpathlineto{\pgfqpoint{2.024480in}{1.275008in}}%
\pgfpathlineto{\pgfqpoint{2.084783in}{1.002848in}}%
\pgfpathlineto{\pgfqpoint{2.161708in}{1.156930in}}%
\pgfpathlineto{\pgfqpoint{2.241709in}{0.888707in}}%
\pgfpathlineto{\pgfqpoint{2.307561in}{1.076300in}}%
\pgfpathlineto{\pgfqpoint{2.373263in}{0.877041in}}%
\pgfpathlineto{\pgfqpoint{2.451089in}{0.936364in}}%
\pgfpathlineto{\pgfqpoint{2.742143in}{0.930813in}}%
\pgfpathlineto{\pgfqpoint{2.895049in}{0.992833in}}%
\pgfpathlineto{\pgfqpoint{3.032787in}{0.941075in}}%
\pgfpathlineto{\pgfqpoint{3.176244in}{0.932682in}}%
\pgfpathlineto{\pgfqpoint{3.328729in}{1.032645in}}%
\pgfpathlineto{\pgfqpoint{3.474352in}{1.057376in}}%
\pgfpathlineto{\pgfqpoint{3.610098in}{0.943449in}}%
\pgfpathlineto{\pgfqpoint{3.755848in}{0.885922in}}%
\pgfpathlineto{\pgfqpoint{3.910166in}{0.906757in}}%
\pgfpathlineto{\pgfqpoint{4.193981in}{0.899047in}}%
\pgfusepath{stroke}%
\end{pgfscope}%
\begin{pgfscope}%
\pgfsetbuttcap%
\pgfsetroundjoin%
\definecolor{currentfill}{rgb}{0.007843,0.619608,0.450980}%
\pgfsetfillcolor{currentfill}%
\pgfsetlinewidth{0.752812pt}%
\definecolor{currentstroke}{rgb}{1.000000,1.000000,1.000000}%
\pgfsetstrokecolor{currentstroke}%
\pgfsetdash{}{0pt}%
\pgfsys@defobject{currentmarker}{\pgfqpoint{-0.034722in}{-0.034722in}}{\pgfqpoint{0.034722in}{0.034722in}}{%
\pgfpathmoveto{\pgfqpoint{0.000000in}{-0.034722in}}%
\pgfpathcurveto{\pgfqpoint{0.009208in}{-0.034722in}}{\pgfqpoint{0.018041in}{-0.031064in}}{\pgfqpoint{0.024552in}{-0.024552in}}%
\pgfpathcurveto{\pgfqpoint{0.031064in}{-0.018041in}}{\pgfqpoint{0.034722in}{-0.009208in}}{\pgfqpoint{0.034722in}{0.000000in}}%
\pgfpathcurveto{\pgfqpoint{0.034722in}{0.009208in}}{\pgfqpoint{0.031064in}{0.018041in}}{\pgfqpoint{0.024552in}{0.024552in}}%
\pgfpathcurveto{\pgfqpoint{0.018041in}{0.031064in}}{\pgfqpoint{0.009208in}{0.034722in}}{\pgfqpoint{0.000000in}{0.034722in}}%
\pgfpathcurveto{\pgfqpoint{-0.009208in}{0.034722in}}{\pgfqpoint{-0.018041in}{0.031064in}}{\pgfqpoint{-0.024552in}{0.024552in}}%
\pgfpathcurveto{\pgfqpoint{-0.031064in}{0.018041in}}{\pgfqpoint{-0.034722in}{0.009208in}}{\pgfqpoint{-0.034722in}{0.000000in}}%
\pgfpathcurveto{\pgfqpoint{-0.034722in}{-0.009208in}}{\pgfqpoint{-0.031064in}{-0.018041in}}{\pgfqpoint{-0.024552in}{-0.024552in}}%
\pgfpathcurveto{\pgfqpoint{-0.018041in}{-0.031064in}}{\pgfqpoint{-0.009208in}{-0.034722in}}{\pgfqpoint{0.000000in}{-0.034722in}}%
\pgfpathlineto{\pgfqpoint{0.000000in}{-0.034722in}}%
\pgfpathclose%
\pgfusepath{stroke,fill}%
}%
\begin{pgfscope}%
\pgfsys@transformshift{0.711866in}{1.469398in}%
\pgfsys@useobject{currentmarker}{}%
\end{pgfscope}%
\begin{pgfscope}%
\pgfsys@transformshift{0.784862in}{1.414431in}%
\pgfsys@useobject{currentmarker}{}%
\end{pgfscope}%
\begin{pgfscope}%
\pgfsys@transformshift{0.857051in}{1.429583in}%
\pgfsys@useobject{currentmarker}{}%
\end{pgfscope}%
\begin{pgfscope}%
\pgfsys@transformshift{0.929566in}{1.447289in}%
\pgfsys@useobject{currentmarker}{}%
\end{pgfscope}%
\begin{pgfscope}%
\pgfsys@transformshift{1.002017in}{1.398216in}%
\pgfsys@useobject{currentmarker}{}%
\end{pgfscope}%
\begin{pgfscope}%
\pgfsys@transformshift{1.074540in}{1.335088in}%
\pgfsys@useobject{currentmarker}{}%
\end{pgfscope}%
\begin{pgfscope}%
\pgfsys@transformshift{1.148040in}{1.425740in}%
\pgfsys@useobject{currentmarker}{}%
\end{pgfscope}%
\begin{pgfscope}%
\pgfsys@transformshift{1.219669in}{1.329309in}%
\pgfsys@useobject{currentmarker}{}%
\end{pgfscope}%
\begin{pgfscope}%
\pgfsys@transformshift{1.292451in}{1.366249in}%
\pgfsys@useobject{currentmarker}{}%
\end{pgfscope}%
\begin{pgfscope}%
\pgfsys@transformshift{1.365625in}{1.263837in}%
\pgfsys@useobject{currentmarker}{}%
\end{pgfscope}%
\begin{pgfscope}%
\pgfsys@transformshift{1.437332in}{1.345548in}%
\pgfsys@useobject{currentmarker}{}%
\end{pgfscope}%
\begin{pgfscope}%
\pgfsys@transformshift{1.510203in}{1.254375in}%
\pgfsys@useobject{currentmarker}{}%
\end{pgfscope}%
\begin{pgfscope}%
\pgfsys@transformshift{1.582977in}{1.225045in}%
\pgfsys@useobject{currentmarker}{}%
\end{pgfscope}%
\begin{pgfscope}%
\pgfsys@transformshift{1.654812in}{1.247182in}%
\pgfsys@useobject{currentmarker}{}%
\end{pgfscope}%
\begin{pgfscope}%
\pgfsys@transformshift{1.727384in}{1.096626in}%
\pgfsys@useobject{currentmarker}{}%
\end{pgfscope}%
\begin{pgfscope}%
\pgfsys@transformshift{1.801235in}{1.384939in}%
\pgfsys@useobject{currentmarker}{}%
\end{pgfscope}%
\begin{pgfscope}%
\pgfsys@transformshift{1.873603in}{1.055179in}%
\pgfsys@useobject{currentmarker}{}%
\end{pgfscope}%
\begin{pgfscope}%
\pgfsys@transformshift{1.945976in}{1.149361in}%
\pgfsys@useobject{currentmarker}{}%
\end{pgfscope}%
\begin{pgfscope}%
\pgfsys@transformshift{2.024480in}{1.275008in}%
\pgfsys@useobject{currentmarker}{}%
\end{pgfscope}%
\begin{pgfscope}%
\pgfsys@transformshift{2.084783in}{1.002848in}%
\pgfsys@useobject{currentmarker}{}%
\end{pgfscope}%
\begin{pgfscope}%
\pgfsys@transformshift{2.161708in}{1.156930in}%
\pgfsys@useobject{currentmarker}{}%
\end{pgfscope}%
\begin{pgfscope}%
\pgfsys@transformshift{2.241709in}{0.888707in}%
\pgfsys@useobject{currentmarker}{}%
\end{pgfscope}%
\begin{pgfscope}%
\pgfsys@transformshift{2.307561in}{1.076300in}%
\pgfsys@useobject{currentmarker}{}%
\end{pgfscope}%
\begin{pgfscope}%
\pgfsys@transformshift{2.373263in}{0.877041in}%
\pgfsys@useobject{currentmarker}{}%
\end{pgfscope}%
\begin{pgfscope}%
\pgfsys@transformshift{2.451089in}{0.936364in}%
\pgfsys@useobject{currentmarker}{}%
\end{pgfscope}%
\begin{pgfscope}%
\pgfsys@transformshift{2.742143in}{0.930813in}%
\pgfsys@useobject{currentmarker}{}%
\end{pgfscope}%
\begin{pgfscope}%
\pgfsys@transformshift{2.895049in}{0.992833in}%
\pgfsys@useobject{currentmarker}{}%
\end{pgfscope}%
\begin{pgfscope}%
\pgfsys@transformshift{3.032787in}{0.941075in}%
\pgfsys@useobject{currentmarker}{}%
\end{pgfscope}%
\begin{pgfscope}%
\pgfsys@transformshift{3.176244in}{0.932682in}%
\pgfsys@useobject{currentmarker}{}%
\end{pgfscope}%
\begin{pgfscope}%
\pgfsys@transformshift{3.328729in}{1.032645in}%
\pgfsys@useobject{currentmarker}{}%
\end{pgfscope}%
\begin{pgfscope}%
\pgfsys@transformshift{3.474352in}{1.057376in}%
\pgfsys@useobject{currentmarker}{}%
\end{pgfscope}%
\begin{pgfscope}%
\pgfsys@transformshift{3.610098in}{0.943449in}%
\pgfsys@useobject{currentmarker}{}%
\end{pgfscope}%
\begin{pgfscope}%
\pgfsys@transformshift{3.755848in}{0.885922in}%
\pgfsys@useobject{currentmarker}{}%
\end{pgfscope}%
\begin{pgfscope}%
\pgfsys@transformshift{3.910166in}{0.906757in}%
\pgfsys@useobject{currentmarker}{}%
\end{pgfscope}%
\begin{pgfscope}%
\pgfsys@transformshift{4.193981in}{0.899047in}%
\pgfsys@useobject{currentmarker}{}%
\end{pgfscope}%
\end{pgfscope}%
\begin{pgfscope}%
\pgfsetroundcap%
\pgfsetroundjoin%
\pgfsetlinewidth{1.003750pt}%
\definecolor{currentstroke}{rgb}{0.835294,0.368627,0.000000}%
\pgfsetstrokecolor{currentstroke}%
\pgfsetdash{}{0pt}%
\pgfpathmoveto{\pgfqpoint{0.711866in}{0.632823in}}%
\pgfpathlineto{\pgfqpoint{0.784862in}{0.682023in}}%
\pgfpathlineto{\pgfqpoint{0.857051in}{0.658730in}}%
\pgfpathlineto{\pgfqpoint{0.929566in}{0.646861in}}%
\pgfpathlineto{\pgfqpoint{1.002017in}{0.657669in}}%
\pgfpathlineto{\pgfqpoint{1.074540in}{0.665342in}}%
\pgfpathlineto{\pgfqpoint{1.148040in}{0.617415in}}%
\pgfpathlineto{\pgfqpoint{1.219669in}{0.611647in}}%
\pgfpathlineto{\pgfqpoint{1.292451in}{0.645097in}}%
\pgfpathlineto{\pgfqpoint{1.365625in}{0.592035in}}%
\pgfpathlineto{\pgfqpoint{1.437332in}{0.632287in}}%
\pgfpathlineto{\pgfqpoint{1.510203in}{0.618513in}}%
\pgfpathlineto{\pgfqpoint{1.582977in}{0.656734in}}%
\pgfpathlineto{\pgfqpoint{1.654812in}{0.604645in}}%
\pgfpathlineto{\pgfqpoint{1.727384in}{0.614189in}}%
\pgfpathlineto{\pgfqpoint{1.801235in}{0.607573in}}%
\pgfpathlineto{\pgfqpoint{1.873603in}{0.717727in}}%
\pgfpathlineto{\pgfqpoint{1.945976in}{0.607197in}}%
\pgfpathlineto{\pgfqpoint{2.024480in}{0.595189in}}%
\pgfpathlineto{\pgfqpoint{2.084783in}{0.629524in}}%
\pgfpathlineto{\pgfqpoint{2.161708in}{0.673769in}}%
\pgfpathlineto{\pgfqpoint{2.241709in}{0.652669in}}%
\pgfpathlineto{\pgfqpoint{2.307561in}{0.653697in}}%
\pgfpathlineto{\pgfqpoint{2.373263in}{0.666439in}}%
\pgfpathlineto{\pgfqpoint{2.451089in}{0.767680in}}%
\pgfpathlineto{\pgfqpoint{2.742143in}{0.643826in}}%
\pgfpathlineto{\pgfqpoint{2.895049in}{0.653247in}}%
\pgfpathlineto{\pgfqpoint{3.032787in}{0.668011in}}%
\pgfpathlineto{\pgfqpoint{3.176244in}{0.639312in}}%
\pgfpathlineto{\pgfqpoint{3.328729in}{0.610778in}}%
\pgfpathlineto{\pgfqpoint{3.474352in}{0.607256in}}%
\pgfpathlineto{\pgfqpoint{3.610098in}{0.654601in}}%
\pgfpathlineto{\pgfqpoint{3.755848in}{0.669387in}}%
\pgfpathlineto{\pgfqpoint{3.910166in}{0.644007in}}%
\pgfpathlineto{\pgfqpoint{4.193981in}{0.645586in}}%
\pgfusepath{stroke}%
\end{pgfscope}%
\begin{pgfscope}%
\pgfsetbuttcap%
\pgfsetroundjoin%
\definecolor{currentfill}{rgb}{0.835294,0.368627,0.000000}%
\pgfsetfillcolor{currentfill}%
\pgfsetlinewidth{0.752812pt}%
\definecolor{currentstroke}{rgb}{1.000000,1.000000,1.000000}%
\pgfsetstrokecolor{currentstroke}%
\pgfsetdash{}{0pt}%
\pgfsys@defobject{currentmarker}{\pgfqpoint{-0.034722in}{-0.034722in}}{\pgfqpoint{0.034722in}{0.034722in}}{%
\pgfpathmoveto{\pgfqpoint{0.000000in}{-0.034722in}}%
\pgfpathcurveto{\pgfqpoint{0.009208in}{-0.034722in}}{\pgfqpoint{0.018041in}{-0.031064in}}{\pgfqpoint{0.024552in}{-0.024552in}}%
\pgfpathcurveto{\pgfqpoint{0.031064in}{-0.018041in}}{\pgfqpoint{0.034722in}{-0.009208in}}{\pgfqpoint{0.034722in}{0.000000in}}%
\pgfpathcurveto{\pgfqpoint{0.034722in}{0.009208in}}{\pgfqpoint{0.031064in}{0.018041in}}{\pgfqpoint{0.024552in}{0.024552in}}%
\pgfpathcurveto{\pgfqpoint{0.018041in}{0.031064in}}{\pgfqpoint{0.009208in}{0.034722in}}{\pgfqpoint{0.000000in}{0.034722in}}%
\pgfpathcurveto{\pgfqpoint{-0.009208in}{0.034722in}}{\pgfqpoint{-0.018041in}{0.031064in}}{\pgfqpoint{-0.024552in}{0.024552in}}%
\pgfpathcurveto{\pgfqpoint{-0.031064in}{0.018041in}}{\pgfqpoint{-0.034722in}{0.009208in}}{\pgfqpoint{-0.034722in}{0.000000in}}%
\pgfpathcurveto{\pgfqpoint{-0.034722in}{-0.009208in}}{\pgfqpoint{-0.031064in}{-0.018041in}}{\pgfqpoint{-0.024552in}{-0.024552in}}%
\pgfpathcurveto{\pgfqpoint{-0.018041in}{-0.031064in}}{\pgfqpoint{-0.009208in}{-0.034722in}}{\pgfqpoint{0.000000in}{-0.034722in}}%
\pgfpathlineto{\pgfqpoint{0.000000in}{-0.034722in}}%
\pgfpathclose%
\pgfusepath{stroke,fill}%
}%
\begin{pgfscope}%
\pgfsys@transformshift{0.711866in}{0.632823in}%
\pgfsys@useobject{currentmarker}{}%
\end{pgfscope}%
\begin{pgfscope}%
\pgfsys@transformshift{0.784862in}{0.682023in}%
\pgfsys@useobject{currentmarker}{}%
\end{pgfscope}%
\begin{pgfscope}%
\pgfsys@transformshift{0.857051in}{0.658730in}%
\pgfsys@useobject{currentmarker}{}%
\end{pgfscope}%
\begin{pgfscope}%
\pgfsys@transformshift{0.929566in}{0.646861in}%
\pgfsys@useobject{currentmarker}{}%
\end{pgfscope}%
\begin{pgfscope}%
\pgfsys@transformshift{1.002017in}{0.657669in}%
\pgfsys@useobject{currentmarker}{}%
\end{pgfscope}%
\begin{pgfscope}%
\pgfsys@transformshift{1.074540in}{0.665342in}%
\pgfsys@useobject{currentmarker}{}%
\end{pgfscope}%
\begin{pgfscope}%
\pgfsys@transformshift{1.148040in}{0.617415in}%
\pgfsys@useobject{currentmarker}{}%
\end{pgfscope}%
\begin{pgfscope}%
\pgfsys@transformshift{1.219669in}{0.611647in}%
\pgfsys@useobject{currentmarker}{}%
\end{pgfscope}%
\begin{pgfscope}%
\pgfsys@transformshift{1.292451in}{0.645097in}%
\pgfsys@useobject{currentmarker}{}%
\end{pgfscope}%
\begin{pgfscope}%
\pgfsys@transformshift{1.365625in}{0.592035in}%
\pgfsys@useobject{currentmarker}{}%
\end{pgfscope}%
\begin{pgfscope}%
\pgfsys@transformshift{1.437332in}{0.632287in}%
\pgfsys@useobject{currentmarker}{}%
\end{pgfscope}%
\begin{pgfscope}%
\pgfsys@transformshift{1.510203in}{0.618513in}%
\pgfsys@useobject{currentmarker}{}%
\end{pgfscope}%
\begin{pgfscope}%
\pgfsys@transformshift{1.582977in}{0.656734in}%
\pgfsys@useobject{currentmarker}{}%
\end{pgfscope}%
\begin{pgfscope}%
\pgfsys@transformshift{1.654812in}{0.604645in}%
\pgfsys@useobject{currentmarker}{}%
\end{pgfscope}%
\begin{pgfscope}%
\pgfsys@transformshift{1.727384in}{0.614189in}%
\pgfsys@useobject{currentmarker}{}%
\end{pgfscope}%
\begin{pgfscope}%
\pgfsys@transformshift{1.801235in}{0.607573in}%
\pgfsys@useobject{currentmarker}{}%
\end{pgfscope}%
\begin{pgfscope}%
\pgfsys@transformshift{1.873603in}{0.717727in}%
\pgfsys@useobject{currentmarker}{}%
\end{pgfscope}%
\begin{pgfscope}%
\pgfsys@transformshift{1.945976in}{0.607197in}%
\pgfsys@useobject{currentmarker}{}%
\end{pgfscope}%
\begin{pgfscope}%
\pgfsys@transformshift{2.024480in}{0.595189in}%
\pgfsys@useobject{currentmarker}{}%
\end{pgfscope}%
\begin{pgfscope}%
\pgfsys@transformshift{2.084783in}{0.629524in}%
\pgfsys@useobject{currentmarker}{}%
\end{pgfscope}%
\begin{pgfscope}%
\pgfsys@transformshift{2.161708in}{0.673769in}%
\pgfsys@useobject{currentmarker}{}%
\end{pgfscope}%
\begin{pgfscope}%
\pgfsys@transformshift{2.241709in}{0.652669in}%
\pgfsys@useobject{currentmarker}{}%
\end{pgfscope}%
\begin{pgfscope}%
\pgfsys@transformshift{2.307561in}{0.653697in}%
\pgfsys@useobject{currentmarker}{}%
\end{pgfscope}%
\begin{pgfscope}%
\pgfsys@transformshift{2.373263in}{0.666439in}%
\pgfsys@useobject{currentmarker}{}%
\end{pgfscope}%
\begin{pgfscope}%
\pgfsys@transformshift{2.451089in}{0.767680in}%
\pgfsys@useobject{currentmarker}{}%
\end{pgfscope}%
\begin{pgfscope}%
\pgfsys@transformshift{2.742143in}{0.643826in}%
\pgfsys@useobject{currentmarker}{}%
\end{pgfscope}%
\begin{pgfscope}%
\pgfsys@transformshift{2.895049in}{0.653247in}%
\pgfsys@useobject{currentmarker}{}%
\end{pgfscope}%
\begin{pgfscope}%
\pgfsys@transformshift{3.032787in}{0.668011in}%
\pgfsys@useobject{currentmarker}{}%
\end{pgfscope}%
\begin{pgfscope}%
\pgfsys@transformshift{3.176244in}{0.639312in}%
\pgfsys@useobject{currentmarker}{}%
\end{pgfscope}%
\begin{pgfscope}%
\pgfsys@transformshift{3.328729in}{0.610778in}%
\pgfsys@useobject{currentmarker}{}%
\end{pgfscope}%
\begin{pgfscope}%
\pgfsys@transformshift{3.474352in}{0.607256in}%
\pgfsys@useobject{currentmarker}{}%
\end{pgfscope}%
\begin{pgfscope}%
\pgfsys@transformshift{3.610098in}{0.654601in}%
\pgfsys@useobject{currentmarker}{}%
\end{pgfscope}%
\begin{pgfscope}%
\pgfsys@transformshift{3.755848in}{0.669387in}%
\pgfsys@useobject{currentmarker}{}%
\end{pgfscope}%
\begin{pgfscope}%
\pgfsys@transformshift{3.910166in}{0.644007in}%
\pgfsys@useobject{currentmarker}{}%
\end{pgfscope}%
\begin{pgfscope}%
\pgfsys@transformshift{4.193981in}{0.645586in}%
\pgfsys@useobject{currentmarker}{}%
\end{pgfscope}%
\end{pgfscope}%
\end{pgfpicture}%
\makeatother%
\endgroup%

						\end{figcenter}
						\caption{Same as \Cref{fig:eval-city-routing-details-b} but showing the relative shares on the total time.}
					\end{subfigure}
					\\[3ex]
					\begin{subfigure}[t]{\textwidth}
						\begin{figcenter}
							%% Creator: Matplotlib, PGF backend
%%
%% To include the figure in your LaTeX document, write
%%   \input{<filename>.pgf}
%%
%% Make sure the required packages are loaded in your preamble
%%   \usepackage{pgf}
%%
%% Also ensure that all the required font packages are loaded; for instance,
%% the lmodern package is sometimes necessary when using math font.
%%   \usepackage{lmodern}
%%
%% Figures using additional raster images can only be included by \input if
%% they are in the same directory as the main LaTeX file. For loading figures
%% from other directories you can use the `import` package
%%   \usepackage{import}
%%
%% and then include the figures with
%%   \import{<path to file>}{<filename>.pgf}
%%
%% Matplotlib used the following preamble
%%   
%%   \usepackage{fontspec}
%%   \setmainfont{DejaVuSerif.ttf}[Path=\detokenize{/home/hauke/.local/lib/python3.11/site-packages/matplotlib/mpl-data/fonts/ttf/}]
%%   \setsansfont{DroidSans.ttf}[Path=\detokenize{/usr/share/fonts/droid/}]
%%   \setmonofont{DejaVuSansMono.ttf}[Path=\detokenize{/home/hauke/.local/lib/python3.11/site-packages/matplotlib/mpl-data/fonts/ttf/}]
%%   \makeatletter\@ifpackageloaded{underscore}{}{\usepackage[strings]{underscore}}\makeatother
%%
\begingroup%
\makeatletter%
\begin{pgfpicture}%
\pgfpathrectangle{\pgfpointorigin}{\pgfqpoint{5.719729in}{1.717241in}}%
\pgfusepath{use as bounding box, clip}%
\begin{pgfscope}%
\pgfsetbuttcap%
\pgfsetmiterjoin%
\definecolor{currentfill}{rgb}{1.000000,1.000000,1.000000}%
\pgfsetfillcolor{currentfill}%
\pgfsetlinewidth{0.000000pt}%
\definecolor{currentstroke}{rgb}{1.000000,1.000000,1.000000}%
\pgfsetstrokecolor{currentstroke}%
\pgfsetdash{}{0pt}%
\pgfpathmoveto{\pgfqpoint{0.000000in}{0.000000in}}%
\pgfpathlineto{\pgfqpoint{5.719729in}{0.000000in}}%
\pgfpathlineto{\pgfqpoint{5.719729in}{1.717241in}}%
\pgfpathlineto{\pgfqpoint{0.000000in}{1.717241in}}%
\pgfpathlineto{\pgfqpoint{0.000000in}{0.000000in}}%
\pgfpathclose%
\pgfusepath{fill}%
\end{pgfscope}%
\begin{pgfscope}%
\pgfsetbuttcap%
\pgfsetmiterjoin%
\definecolor{currentfill}{rgb}{1.000000,1.000000,1.000000}%
\pgfsetfillcolor{currentfill}%
\pgfsetlinewidth{0.000000pt}%
\definecolor{currentstroke}{rgb}{0.000000,0.000000,0.000000}%
\pgfsetstrokecolor{currentstroke}%
\pgfsetstrokeopacity{0.000000}%
\pgfsetdash{}{0pt}%
\pgfpathmoveto{\pgfqpoint{0.532932in}{0.451389in}}%
\pgfpathlineto{\pgfqpoint{4.023350in}{0.451389in}}%
\pgfpathlineto{\pgfqpoint{4.023350in}{1.682563in}}%
\pgfpathlineto{\pgfqpoint{0.532932in}{1.682563in}}%
\pgfpathlineto{\pgfqpoint{0.532932in}{0.451389in}}%
\pgfpathclose%
\pgfusepath{fill}%
\end{pgfscope}%
\begin{pgfscope}%
\pgfpathrectangle{\pgfqpoint{0.532932in}{0.451389in}}{\pgfqpoint{3.490418in}{1.231174in}}%
\pgfusepath{clip}%
\pgfsetroundcap%
\pgfsetroundjoin%
\pgfsetlinewidth{1.003750pt}%
\definecolor{currentstroke}{rgb}{0.800000,0.800000,0.800000}%
\pgfsetstrokecolor{currentstroke}%
\pgfsetdash{}{0pt}%
\pgfpathmoveto{\pgfqpoint{0.532932in}{0.451389in}}%
\pgfpathlineto{\pgfqpoint{0.532932in}{1.682563in}}%
\pgfusepath{stroke}%
\end{pgfscope}%
\begin{pgfscope}%
\definecolor{textcolor}{rgb}{0.150000,0.150000,0.150000}%
\pgfsetstrokecolor{textcolor}%
\pgfsetfillcolor{textcolor}%
\pgftext[x=0.532932in,y=0.319444in,,top]{\color{textcolor}\sffamily\fontsize{9.000000}{10.800000}\selectfont 0}%
\end{pgfscope}%
\begin{pgfscope}%
\pgfpathrectangle{\pgfqpoint{0.532932in}{0.451389in}}{\pgfqpoint{3.490418in}{1.231174in}}%
\pgfusepath{clip}%
\pgfsetroundcap%
\pgfsetroundjoin%
\pgfsetlinewidth{1.003750pt}%
\definecolor{currentstroke}{rgb}{0.800000,0.800000,0.800000}%
\pgfsetstrokecolor{currentstroke}%
\pgfsetdash{}{0pt}%
\pgfpathmoveto{\pgfqpoint{1.083536in}{0.451389in}}%
\pgfpathlineto{\pgfqpoint{1.083536in}{1.682563in}}%
\pgfusepath{stroke}%
\end{pgfscope}%
\begin{pgfscope}%
\definecolor{textcolor}{rgb}{0.150000,0.150000,0.150000}%
\pgfsetstrokecolor{textcolor}%
\pgfsetfillcolor{textcolor}%
\pgftext[x=1.083536in,y=0.319444in,,top]{\color{textcolor}\sffamily\fontsize{9.000000}{10.800000}\selectfont 1000}%
\end{pgfscope}%
\begin{pgfscope}%
\pgfpathrectangle{\pgfqpoint{0.532932in}{0.451389in}}{\pgfqpoint{3.490418in}{1.231174in}}%
\pgfusepath{clip}%
\pgfsetroundcap%
\pgfsetroundjoin%
\pgfsetlinewidth{1.003750pt}%
\definecolor{currentstroke}{rgb}{0.800000,0.800000,0.800000}%
\pgfsetstrokecolor{currentstroke}%
\pgfsetdash{}{0pt}%
\pgfpathmoveto{\pgfqpoint{1.634140in}{0.451389in}}%
\pgfpathlineto{\pgfqpoint{1.634140in}{1.682563in}}%
\pgfusepath{stroke}%
\end{pgfscope}%
\begin{pgfscope}%
\definecolor{textcolor}{rgb}{0.150000,0.150000,0.150000}%
\pgfsetstrokecolor{textcolor}%
\pgfsetfillcolor{textcolor}%
\pgftext[x=1.634140in,y=0.319444in,,top]{\color{textcolor}\sffamily\fontsize{9.000000}{10.800000}\selectfont 2000}%
\end{pgfscope}%
\begin{pgfscope}%
\pgfpathrectangle{\pgfqpoint{0.532932in}{0.451389in}}{\pgfqpoint{3.490418in}{1.231174in}}%
\pgfusepath{clip}%
\pgfsetroundcap%
\pgfsetroundjoin%
\pgfsetlinewidth{1.003750pt}%
\definecolor{currentstroke}{rgb}{0.800000,0.800000,0.800000}%
\pgfsetstrokecolor{currentstroke}%
\pgfsetdash{}{0pt}%
\pgfpathmoveto{\pgfqpoint{2.184745in}{0.451389in}}%
\pgfpathlineto{\pgfqpoint{2.184745in}{1.682563in}}%
\pgfusepath{stroke}%
\end{pgfscope}%
\begin{pgfscope}%
\definecolor{textcolor}{rgb}{0.150000,0.150000,0.150000}%
\pgfsetstrokecolor{textcolor}%
\pgfsetfillcolor{textcolor}%
\pgftext[x=2.184745in,y=0.319444in,,top]{\color{textcolor}\sffamily\fontsize{9.000000}{10.800000}\selectfont 3000}%
\end{pgfscope}%
\begin{pgfscope}%
\pgfpathrectangle{\pgfqpoint{0.532932in}{0.451389in}}{\pgfqpoint{3.490418in}{1.231174in}}%
\pgfusepath{clip}%
\pgfsetroundcap%
\pgfsetroundjoin%
\pgfsetlinewidth{1.003750pt}%
\definecolor{currentstroke}{rgb}{0.800000,0.800000,0.800000}%
\pgfsetstrokecolor{currentstroke}%
\pgfsetdash{}{0pt}%
\pgfpathmoveto{\pgfqpoint{2.735349in}{0.451389in}}%
\pgfpathlineto{\pgfqpoint{2.735349in}{1.682563in}}%
\pgfusepath{stroke}%
\end{pgfscope}%
\begin{pgfscope}%
\definecolor{textcolor}{rgb}{0.150000,0.150000,0.150000}%
\pgfsetstrokecolor{textcolor}%
\pgfsetfillcolor{textcolor}%
\pgftext[x=2.735349in,y=0.319444in,,top]{\color{textcolor}\sffamily\fontsize{9.000000}{10.800000}\selectfont 4000}%
\end{pgfscope}%
\begin{pgfscope}%
\pgfpathrectangle{\pgfqpoint{0.532932in}{0.451389in}}{\pgfqpoint{3.490418in}{1.231174in}}%
\pgfusepath{clip}%
\pgfsetroundcap%
\pgfsetroundjoin%
\pgfsetlinewidth{1.003750pt}%
\definecolor{currentstroke}{rgb}{0.800000,0.800000,0.800000}%
\pgfsetstrokecolor{currentstroke}%
\pgfsetdash{}{0pt}%
\pgfpathmoveto{\pgfqpoint{3.285953in}{0.451389in}}%
\pgfpathlineto{\pgfqpoint{3.285953in}{1.682563in}}%
\pgfusepath{stroke}%
\end{pgfscope}%
\begin{pgfscope}%
\definecolor{textcolor}{rgb}{0.150000,0.150000,0.150000}%
\pgfsetstrokecolor{textcolor}%
\pgfsetfillcolor{textcolor}%
\pgftext[x=3.285953in,y=0.319444in,,top]{\color{textcolor}\sffamily\fontsize{9.000000}{10.800000}\selectfont 5000}%
\end{pgfscope}%
\begin{pgfscope}%
\pgfpathrectangle{\pgfqpoint{0.532932in}{0.451389in}}{\pgfqpoint{3.490418in}{1.231174in}}%
\pgfusepath{clip}%
\pgfsetroundcap%
\pgfsetroundjoin%
\pgfsetlinewidth{1.003750pt}%
\definecolor{currentstroke}{rgb}{0.800000,0.800000,0.800000}%
\pgfsetstrokecolor{currentstroke}%
\pgfsetdash{}{0pt}%
\pgfpathmoveto{\pgfqpoint{3.836557in}{0.451389in}}%
\pgfpathlineto{\pgfqpoint{3.836557in}{1.682563in}}%
\pgfusepath{stroke}%
\end{pgfscope}%
\begin{pgfscope}%
\definecolor{textcolor}{rgb}{0.150000,0.150000,0.150000}%
\pgfsetstrokecolor{textcolor}%
\pgfsetfillcolor{textcolor}%
\pgftext[x=3.836557in,y=0.319444in,,top]{\color{textcolor}\sffamily\fontsize{9.000000}{10.800000}\selectfont 6000}%
\end{pgfscope}%
\begin{pgfscope}%
\definecolor{textcolor}{rgb}{0.150000,0.150000,0.150000}%
\pgfsetstrokecolor{textcolor}%
\pgfsetfillcolor{textcolor}%
\pgftext[x=2.278141in,y=0.125000in,,top]{\color{textcolor}\sffamily\fontsize{9.000000}{10.800000}\selectfont Input obstacle vertices}%
\end{pgfscope}%
\begin{pgfscope}%
\pgfpathrectangle{\pgfqpoint{0.532932in}{0.451389in}}{\pgfqpoint{3.490418in}{1.231174in}}%
\pgfusepath{clip}%
\pgfsetroundcap%
\pgfsetroundjoin%
\pgfsetlinewidth{1.003750pt}%
\definecolor{currentstroke}{rgb}{0.800000,0.800000,0.800000}%
\pgfsetstrokecolor{currentstroke}%
\pgfsetdash{}{0pt}%
\pgfpathmoveto{\pgfqpoint{0.532932in}{0.451389in}}%
\pgfpathlineto{\pgfqpoint{4.023350in}{0.451389in}}%
\pgfusepath{stroke}%
\end{pgfscope}%
\begin{pgfscope}%
\definecolor{textcolor}{rgb}{0.150000,0.150000,0.150000}%
\pgfsetstrokecolor{textcolor}%
\pgfsetfillcolor{textcolor}%
\pgftext[x=0.332140in, y=0.403903in, left, base]{\color{textcolor}\sffamily\fontsize{9.000000}{10.800000}\selectfont 0}%
\end{pgfscope}%
\begin{pgfscope}%
\pgfpathrectangle{\pgfqpoint{0.532932in}{0.451389in}}{\pgfqpoint{3.490418in}{1.231174in}}%
\pgfusepath{clip}%
\pgfsetroundcap%
\pgfsetroundjoin%
\pgfsetlinewidth{1.003750pt}%
\definecolor{currentstroke}{rgb}{0.800000,0.800000,0.800000}%
\pgfsetstrokecolor{currentstroke}%
\pgfsetdash{}{0pt}%
\pgfpathmoveto{\pgfqpoint{0.532932in}{0.857511in}}%
\pgfpathlineto{\pgfqpoint{4.023350in}{0.857511in}}%
\pgfusepath{stroke}%
\end{pgfscope}%
\begin{pgfscope}%
\definecolor{textcolor}{rgb}{0.150000,0.150000,0.150000}%
\pgfsetstrokecolor{textcolor}%
\pgfsetfillcolor{textcolor}%
\pgftext[x=0.194444in, y=0.810026in, left, base]{\color{textcolor}\sffamily\fontsize{9.000000}{10.800000}\selectfont 200}%
\end{pgfscope}%
\begin{pgfscope}%
\pgfpathrectangle{\pgfqpoint{0.532932in}{0.451389in}}{\pgfqpoint{3.490418in}{1.231174in}}%
\pgfusepath{clip}%
\pgfsetroundcap%
\pgfsetroundjoin%
\pgfsetlinewidth{1.003750pt}%
\definecolor{currentstroke}{rgb}{0.800000,0.800000,0.800000}%
\pgfsetstrokecolor{currentstroke}%
\pgfsetdash{}{0pt}%
\pgfpathmoveto{\pgfqpoint{0.532932in}{1.263633in}}%
\pgfpathlineto{\pgfqpoint{4.023350in}{1.263633in}}%
\pgfusepath{stroke}%
\end{pgfscope}%
\begin{pgfscope}%
\definecolor{textcolor}{rgb}{0.150000,0.150000,0.150000}%
\pgfsetstrokecolor{textcolor}%
\pgfsetfillcolor{textcolor}%
\pgftext[x=0.194444in, y=1.216148in, left, base]{\color{textcolor}\sffamily\fontsize{9.000000}{10.800000}\selectfont 400}%
\end{pgfscope}%
\begin{pgfscope}%
\pgfpathrectangle{\pgfqpoint{0.532932in}{0.451389in}}{\pgfqpoint{3.490418in}{1.231174in}}%
\pgfusepath{clip}%
\pgfsetroundcap%
\pgfsetroundjoin%
\pgfsetlinewidth{1.003750pt}%
\definecolor{currentstroke}{rgb}{0.800000,0.800000,0.800000}%
\pgfsetstrokecolor{currentstroke}%
\pgfsetdash{}{0pt}%
\pgfpathmoveto{\pgfqpoint{0.532932in}{1.669756in}}%
\pgfpathlineto{\pgfqpoint{4.023350in}{1.669756in}}%
\pgfusepath{stroke}%
\end{pgfscope}%
\begin{pgfscope}%
\definecolor{textcolor}{rgb}{0.150000,0.150000,0.150000}%
\pgfsetstrokecolor{textcolor}%
\pgfsetfillcolor{textcolor}%
\pgftext[x=0.194444in, y=1.622270in, left, base]{\color{textcolor}\sffamily\fontsize{9.000000}{10.800000}\selectfont 600}%
\end{pgfscope}%
\begin{pgfscope}%
\definecolor{textcolor}{rgb}{0.150000,0.150000,0.150000}%
\pgfsetstrokecolor{textcolor}%
\pgfsetfillcolor{textcolor}%
\pgftext[x=0.125000in,y=1.066976in,,bottom,rotate=90.000000]{\color{textcolor}\sffamily\fontsize{9.000000}{10.800000}\selectfont Time in ms}%
\end{pgfscope}%
\begin{pgfscope}%
\pgfsetrectcap%
\pgfsetmiterjoin%
\pgfsetlinewidth{1.254687pt}%
\definecolor{currentstroke}{rgb}{0.800000,0.800000,0.800000}%
\pgfsetstrokecolor{currentstroke}%
\pgfsetdash{}{0pt}%
\pgfpathmoveto{\pgfqpoint{0.532932in}{0.451389in}}%
\pgfpathlineto{\pgfqpoint{0.532932in}{1.682563in}}%
\pgfusepath{stroke}%
\end{pgfscope}%
\begin{pgfscope}%
\pgfsetrectcap%
\pgfsetmiterjoin%
\pgfsetlinewidth{1.254687pt}%
\definecolor{currentstroke}{rgb}{0.800000,0.800000,0.800000}%
\pgfsetstrokecolor{currentstroke}%
\pgfsetdash{}{0pt}%
\pgfpathmoveto{\pgfqpoint{4.023350in}{0.451389in}}%
\pgfpathlineto{\pgfqpoint{4.023350in}{1.682563in}}%
\pgfusepath{stroke}%
\end{pgfscope}%
\begin{pgfscope}%
\pgfsetrectcap%
\pgfsetmiterjoin%
\pgfsetlinewidth{1.254687pt}%
\definecolor{currentstroke}{rgb}{0.800000,0.800000,0.800000}%
\pgfsetstrokecolor{currentstroke}%
\pgfsetdash{}{0pt}%
\pgfpathmoveto{\pgfqpoint{0.532932in}{0.451389in}}%
\pgfpathlineto{\pgfqpoint{4.023350in}{0.451389in}}%
\pgfusepath{stroke}%
\end{pgfscope}%
\begin{pgfscope}%
\pgfsetrectcap%
\pgfsetmiterjoin%
\pgfsetlinewidth{1.254687pt}%
\definecolor{currentstroke}{rgb}{0.800000,0.800000,0.800000}%
\pgfsetstrokecolor{currentstroke}%
\pgfsetdash{}{0pt}%
\pgfpathmoveto{\pgfqpoint{0.532932in}{1.682563in}}%
\pgfpathlineto{\pgfqpoint{4.023350in}{1.682563in}}%
\pgfusepath{stroke}%
\end{pgfscope}%
\begin{pgfscope}%
\pgfsetbuttcap%
\pgfsetmiterjoin%
\definecolor{currentfill}{rgb}{1.000000,1.000000,1.000000}%
\pgfsetfillcolor{currentfill}%
\pgfsetfillopacity{0.800000}%
\pgfsetlinewidth{1.003750pt}%
\definecolor{currentstroke}{rgb}{0.800000,0.800000,0.800000}%
\pgfsetstrokecolor{currentstroke}%
\pgfsetstrokeopacity{0.800000}%
\pgfsetdash{}{0pt}%
\pgfpathmoveto{\pgfqpoint{4.198110in}{0.507479in}}%
\pgfpathlineto{\pgfqpoint{5.694729in}{0.507479in}}%
\pgfpathquadraticcurveto{\pgfqpoint{5.719729in}{0.507479in}}{\pgfqpoint{5.719729in}{0.532479in}}%
\pgfpathlineto{\pgfqpoint{5.719729in}{1.601472in}}%
\pgfpathquadraticcurveto{\pgfqpoint{5.719729in}{1.626472in}}{\pgfqpoint{5.694729in}{1.626472in}}%
\pgfpathlineto{\pgfqpoint{4.198110in}{1.626472in}}%
\pgfpathquadraticcurveto{\pgfqpoint{4.173110in}{1.626472in}}{\pgfqpoint{4.173110in}{1.601472in}}%
\pgfpathlineto{\pgfqpoint{4.173110in}{0.532479in}}%
\pgfpathquadraticcurveto{\pgfqpoint{4.173110in}{0.507479in}}{\pgfqpoint{4.198110in}{0.507479in}}%
\pgfpathlineto{\pgfqpoint{4.198110in}{0.507479in}}%
\pgfpathclose%
\pgfusepath{stroke,fill}%
\end{pgfscope}%
\begin{pgfscope}%
\definecolor{textcolor}{rgb}{0.150000,0.150000,0.150000}%
\pgfsetstrokecolor{textcolor}%
\pgfsetfillcolor{textcolor}%
\pgftext[x=4.743020in,y=1.481502in,left,base]{\color{textcolor}\sffamily\fontsize{9.000000}{10.800000}\selectfont Legend}%
\end{pgfscope}%
\begin{pgfscope}%
\pgfsetroundcap%
\pgfsetroundjoin%
\pgfsetlinewidth{1.505625pt}%
\definecolor{currentstroke}{rgb}{0.003922,0.450980,0.698039}%
\pgfsetstrokecolor{currentstroke}%
\pgfsetdash{}{0pt}%
\pgfpathmoveto{\pgfqpoint{4.223110in}{1.337752in}}%
\pgfpathlineto{\pgfqpoint{4.348110in}{1.337752in}}%
\pgfpathlineto{\pgfqpoint{4.473110in}{1.337752in}}%
\pgfusepath{stroke}%
\end{pgfscope}%
\begin{pgfscope}%
\definecolor{textcolor}{rgb}{0.150000,0.150000,0.150000}%
\pgfsetstrokecolor{textcolor}%
\pgfsetfillcolor{textcolor}%
\pgftext[x=4.573110in,y=1.294002in,left,base]{\color{textcolor}\sffamily\fontsize{9.000000}{10.800000}\selectfont Total time}%
\end{pgfscope}%
\begin{pgfscope}%
\pgfsetroundcap%
\pgfsetroundjoin%
\pgfsetlinewidth{1.505625pt}%
\definecolor{currentstroke}{rgb}{0.870588,0.560784,0.019608}%
\pgfsetstrokecolor{currentstroke}%
\pgfsetdash{}{0pt}%
\pgfpathmoveto{\pgfqpoint{4.223110in}{1.150252in}}%
\pgfpathlineto{\pgfqpoint{4.348110in}{1.150252in}}%
\pgfpathlineto{\pgfqpoint{4.473110in}{1.150252in}}%
\pgfusepath{stroke}%
\end{pgfscope}%
\begin{pgfscope}%
\definecolor{textcolor}{rgb}{0.150000,0.150000,0.150000}%
\pgfsetstrokecolor{textcolor}%
\pgfsetfillcolor{textcolor}%
\pgftext[x=4.573110in,y=1.106502in,left,base]{\color{textcolor}\sffamily\fontsize{9.000000}{10.800000}\selectfont A* routing}%
\end{pgfscope}%
\begin{pgfscope}%
\pgfsetroundcap%
\pgfsetroundjoin%
\pgfsetlinewidth{1.505625pt}%
\definecolor{currentstroke}{rgb}{0.007843,0.619608,0.450980}%
\pgfsetstrokecolor{currentstroke}%
\pgfsetdash{}{0pt}%
\pgfpathmoveto{\pgfqpoint{4.223110in}{0.875740in}}%
\pgfpathlineto{\pgfqpoint{4.348110in}{0.875740in}}%
\pgfpathlineto{\pgfqpoint{4.473110in}{0.875740in}}%
\pgfusepath{stroke}%
\end{pgfscope}%
\begin{pgfscope}%
\definecolor{textcolor}{rgb}{0.150000,0.150000,0.150000}%
\pgfsetstrokecolor{textcolor}%
\pgfsetfillcolor{textcolor}%
\pgftext[x=4.573110in, y=0.919002in, left, base]{\color{textcolor}\sffamily\fontsize{9.000000}{10.800000}\selectfont Connect source \&}%
\end{pgfscope}%
\begin{pgfscope}%
\definecolor{textcolor}{rgb}{0.150000,0.150000,0.150000}%
\pgfsetstrokecolor{textcolor}%
\pgfsetfillcolor{textcolor}%
\pgftext[x=4.573110in, y=0.775008in, left, base]{\color{textcolor}\sffamily\fontsize{9.000000}{10.800000}\selectfont destination vertices}%
\end{pgfscope}%
\begin{pgfscope}%
\pgfsetroundcap%
\pgfsetroundjoin%
\pgfsetlinewidth{1.505625pt}%
\definecolor{currentstroke}{rgb}{0.835294,0.368627,0.000000}%
\pgfsetstrokecolor{currentstroke}%
\pgfsetdash{}{0pt}%
\pgfpathmoveto{\pgfqpoint{4.223110in}{0.631258in}}%
\pgfpathlineto{\pgfqpoint{4.348110in}{0.631258in}}%
\pgfpathlineto{\pgfqpoint{4.473110in}{0.631258in}}%
\pgfusepath{stroke}%
\end{pgfscope}%
\begin{pgfscope}%
\definecolor{textcolor}{rgb}{0.150000,0.150000,0.150000}%
\pgfsetstrokecolor{textcolor}%
\pgfsetfillcolor{textcolor}%
\pgftext[x=4.573110in,y=0.587508in,left,base]{\color{textcolor}\sffamily\fontsize{9.000000}{10.800000}\selectfont Restoring graph}%
\end{pgfscope}%
\begin{pgfscope}%
\pgfsetroundcap%
\pgfsetroundjoin%
\pgfsetlinewidth{1.003750pt}%
\definecolor{currentstroke}{rgb}{0.003922,0.450980,0.698039}%
\pgfsetstrokecolor{currentstroke}%
\pgfsetdash{}{0pt}%
\pgfpathmoveto{\pgfqpoint{0.736655in}{0.810679in}}%
\pgfpathlineto{\pgfqpoint{1.036735in}{1.177166in}}%
\pgfpathlineto{\pgfqpoint{1.386919in}{1.332674in}}%
\pgfpathlineto{\pgfqpoint{1.780050in}{1.379498in}}%
\pgfpathlineto{\pgfqpoint{2.954489in}{1.553607in}}%
\pgfpathlineto{\pgfqpoint{3.866840in}{1.624439in}}%
\pgfusepath{stroke}%
\end{pgfscope}%
\begin{pgfscope}%
\pgfsetbuttcap%
\pgfsetroundjoin%
\definecolor{currentfill}{rgb}{0.003922,0.450980,0.698039}%
\pgfsetfillcolor{currentfill}%
\pgfsetlinewidth{0.752812pt}%
\definecolor{currentstroke}{rgb}{1.000000,1.000000,1.000000}%
\pgfsetstrokecolor{currentstroke}%
\pgfsetdash{}{0pt}%
\pgfsys@defobject{currentmarker}{\pgfqpoint{-0.034722in}{-0.034722in}}{\pgfqpoint{0.034722in}{0.034722in}}{%
\pgfpathmoveto{\pgfqpoint{0.000000in}{-0.034722in}}%
\pgfpathcurveto{\pgfqpoint{0.009208in}{-0.034722in}}{\pgfqpoint{0.018041in}{-0.031064in}}{\pgfqpoint{0.024552in}{-0.024552in}}%
\pgfpathcurveto{\pgfqpoint{0.031064in}{-0.018041in}}{\pgfqpoint{0.034722in}{-0.009208in}}{\pgfqpoint{0.034722in}{0.000000in}}%
\pgfpathcurveto{\pgfqpoint{0.034722in}{0.009208in}}{\pgfqpoint{0.031064in}{0.018041in}}{\pgfqpoint{0.024552in}{0.024552in}}%
\pgfpathcurveto{\pgfqpoint{0.018041in}{0.031064in}}{\pgfqpoint{0.009208in}{0.034722in}}{\pgfqpoint{0.000000in}{0.034722in}}%
\pgfpathcurveto{\pgfqpoint{-0.009208in}{0.034722in}}{\pgfqpoint{-0.018041in}{0.031064in}}{\pgfqpoint{-0.024552in}{0.024552in}}%
\pgfpathcurveto{\pgfqpoint{-0.031064in}{0.018041in}}{\pgfqpoint{-0.034722in}{0.009208in}}{\pgfqpoint{-0.034722in}{0.000000in}}%
\pgfpathcurveto{\pgfqpoint{-0.034722in}{-0.009208in}}{\pgfqpoint{-0.031064in}{-0.018041in}}{\pgfqpoint{-0.024552in}{-0.024552in}}%
\pgfpathcurveto{\pgfqpoint{-0.018041in}{-0.031064in}}{\pgfqpoint{-0.009208in}{-0.034722in}}{\pgfqpoint{0.000000in}{-0.034722in}}%
\pgfpathlineto{\pgfqpoint{0.000000in}{-0.034722in}}%
\pgfpathclose%
\pgfusepath{stroke,fill}%
}%
\begin{pgfscope}%
\pgfsys@transformshift{0.736655in}{0.810679in}%
\pgfsys@useobject{currentmarker}{}%
\end{pgfscope}%
\begin{pgfscope}%
\pgfsys@transformshift{1.036735in}{1.177166in}%
\pgfsys@useobject{currentmarker}{}%
\end{pgfscope}%
\begin{pgfscope}%
\pgfsys@transformshift{1.386919in}{1.332674in}%
\pgfsys@useobject{currentmarker}{}%
\end{pgfscope}%
\begin{pgfscope}%
\pgfsys@transformshift{1.780050in}{1.379498in}%
\pgfsys@useobject{currentmarker}{}%
\end{pgfscope}%
\begin{pgfscope}%
\pgfsys@transformshift{2.954489in}{1.553607in}%
\pgfsys@useobject{currentmarker}{}%
\end{pgfscope}%
\begin{pgfscope}%
\pgfsys@transformshift{3.866840in}{1.624439in}%
\pgfsys@useobject{currentmarker}{}%
\end{pgfscope}%
\end{pgfscope}%
\begin{pgfscope}%
\pgfsetroundcap%
\pgfsetroundjoin%
\pgfsetlinewidth{1.003750pt}%
\definecolor{currentstroke}{rgb}{0.870588,0.560784,0.019608}%
\pgfsetstrokecolor{currentstroke}%
\pgfsetdash{}{0pt}%
\pgfpathmoveto{\pgfqpoint{0.736655in}{0.469579in}}%
\pgfpathlineto{\pgfqpoint{1.036735in}{0.490874in}}%
\pgfpathlineto{\pgfqpoint{1.386919in}{0.497589in}}%
\pgfpathlineto{\pgfqpoint{1.780050in}{0.500972in}}%
\pgfpathlineto{\pgfqpoint{2.954489in}{0.502552in}}%
\pgfpathlineto{\pgfqpoint{3.866840in}{0.502211in}}%
\pgfusepath{stroke}%
\end{pgfscope}%
\begin{pgfscope}%
\pgfsetbuttcap%
\pgfsetroundjoin%
\definecolor{currentfill}{rgb}{0.870588,0.560784,0.019608}%
\pgfsetfillcolor{currentfill}%
\pgfsetlinewidth{0.752812pt}%
\definecolor{currentstroke}{rgb}{1.000000,1.000000,1.000000}%
\pgfsetstrokecolor{currentstroke}%
\pgfsetdash{}{0pt}%
\pgfsys@defobject{currentmarker}{\pgfqpoint{-0.034722in}{-0.034722in}}{\pgfqpoint{0.034722in}{0.034722in}}{%
\pgfpathmoveto{\pgfqpoint{0.000000in}{-0.034722in}}%
\pgfpathcurveto{\pgfqpoint{0.009208in}{-0.034722in}}{\pgfqpoint{0.018041in}{-0.031064in}}{\pgfqpoint{0.024552in}{-0.024552in}}%
\pgfpathcurveto{\pgfqpoint{0.031064in}{-0.018041in}}{\pgfqpoint{0.034722in}{-0.009208in}}{\pgfqpoint{0.034722in}{0.000000in}}%
\pgfpathcurveto{\pgfqpoint{0.034722in}{0.009208in}}{\pgfqpoint{0.031064in}{0.018041in}}{\pgfqpoint{0.024552in}{0.024552in}}%
\pgfpathcurveto{\pgfqpoint{0.018041in}{0.031064in}}{\pgfqpoint{0.009208in}{0.034722in}}{\pgfqpoint{0.000000in}{0.034722in}}%
\pgfpathcurveto{\pgfqpoint{-0.009208in}{0.034722in}}{\pgfqpoint{-0.018041in}{0.031064in}}{\pgfqpoint{-0.024552in}{0.024552in}}%
\pgfpathcurveto{\pgfqpoint{-0.031064in}{0.018041in}}{\pgfqpoint{-0.034722in}{0.009208in}}{\pgfqpoint{-0.034722in}{0.000000in}}%
\pgfpathcurveto{\pgfqpoint{-0.034722in}{-0.009208in}}{\pgfqpoint{-0.031064in}{-0.018041in}}{\pgfqpoint{-0.024552in}{-0.024552in}}%
\pgfpathcurveto{\pgfqpoint{-0.018041in}{-0.031064in}}{\pgfqpoint{-0.009208in}{-0.034722in}}{\pgfqpoint{0.000000in}{-0.034722in}}%
\pgfpathlineto{\pgfqpoint{0.000000in}{-0.034722in}}%
\pgfpathclose%
\pgfusepath{stroke,fill}%
}%
\begin{pgfscope}%
\pgfsys@transformshift{0.736655in}{0.469579in}%
\pgfsys@useobject{currentmarker}{}%
\end{pgfscope}%
\begin{pgfscope}%
\pgfsys@transformshift{1.036735in}{0.490874in}%
\pgfsys@useobject{currentmarker}{}%
\end{pgfscope}%
\begin{pgfscope}%
\pgfsys@transformshift{1.386919in}{0.497589in}%
\pgfsys@useobject{currentmarker}{}%
\end{pgfscope}%
\begin{pgfscope}%
\pgfsys@transformshift{1.780050in}{0.500972in}%
\pgfsys@useobject{currentmarker}{}%
\end{pgfscope}%
\begin{pgfscope}%
\pgfsys@transformshift{2.954489in}{0.502552in}%
\pgfsys@useobject{currentmarker}{}%
\end{pgfscope}%
\begin{pgfscope}%
\pgfsys@transformshift{3.866840in}{0.502211in}%
\pgfsys@useobject{currentmarker}{}%
\end{pgfscope}%
\end{pgfscope}%
\begin{pgfscope}%
\pgfsetroundcap%
\pgfsetroundjoin%
\pgfsetlinewidth{1.003750pt}%
\definecolor{currentstroke}{rgb}{0.007843,0.619608,0.450980}%
\pgfsetstrokecolor{currentstroke}%
\pgfsetdash{}{0pt}%
\pgfpathmoveto{\pgfqpoint{0.736655in}{0.781838in}}%
\pgfpathlineto{\pgfqpoint{1.036735in}{1.119295in}}%
\pgfpathlineto{\pgfqpoint{1.386919in}{1.265759in}}%
\pgfpathlineto{\pgfqpoint{1.780050in}{1.305494in}}%
\pgfpathlineto{\pgfqpoint{2.954489in}{1.467208in}}%
\pgfpathlineto{\pgfqpoint{3.866840in}{1.531416in}}%
\pgfusepath{stroke}%
\end{pgfscope}%
\begin{pgfscope}%
\pgfsetbuttcap%
\pgfsetroundjoin%
\definecolor{currentfill}{rgb}{0.007843,0.619608,0.450980}%
\pgfsetfillcolor{currentfill}%
\pgfsetlinewidth{0.752812pt}%
\definecolor{currentstroke}{rgb}{1.000000,1.000000,1.000000}%
\pgfsetstrokecolor{currentstroke}%
\pgfsetdash{}{0pt}%
\pgfsys@defobject{currentmarker}{\pgfqpoint{-0.034722in}{-0.034722in}}{\pgfqpoint{0.034722in}{0.034722in}}{%
\pgfpathmoveto{\pgfqpoint{0.000000in}{-0.034722in}}%
\pgfpathcurveto{\pgfqpoint{0.009208in}{-0.034722in}}{\pgfqpoint{0.018041in}{-0.031064in}}{\pgfqpoint{0.024552in}{-0.024552in}}%
\pgfpathcurveto{\pgfqpoint{0.031064in}{-0.018041in}}{\pgfqpoint{0.034722in}{-0.009208in}}{\pgfqpoint{0.034722in}{0.000000in}}%
\pgfpathcurveto{\pgfqpoint{0.034722in}{0.009208in}}{\pgfqpoint{0.031064in}{0.018041in}}{\pgfqpoint{0.024552in}{0.024552in}}%
\pgfpathcurveto{\pgfqpoint{0.018041in}{0.031064in}}{\pgfqpoint{0.009208in}{0.034722in}}{\pgfqpoint{0.000000in}{0.034722in}}%
\pgfpathcurveto{\pgfqpoint{-0.009208in}{0.034722in}}{\pgfqpoint{-0.018041in}{0.031064in}}{\pgfqpoint{-0.024552in}{0.024552in}}%
\pgfpathcurveto{\pgfqpoint{-0.031064in}{0.018041in}}{\pgfqpoint{-0.034722in}{0.009208in}}{\pgfqpoint{-0.034722in}{0.000000in}}%
\pgfpathcurveto{\pgfqpoint{-0.034722in}{-0.009208in}}{\pgfqpoint{-0.031064in}{-0.018041in}}{\pgfqpoint{-0.024552in}{-0.024552in}}%
\pgfpathcurveto{\pgfqpoint{-0.018041in}{-0.031064in}}{\pgfqpoint{-0.009208in}{-0.034722in}}{\pgfqpoint{0.000000in}{-0.034722in}}%
\pgfpathlineto{\pgfqpoint{0.000000in}{-0.034722in}}%
\pgfpathclose%
\pgfusepath{stroke,fill}%
}%
\begin{pgfscope}%
\pgfsys@transformshift{0.736655in}{0.781838in}%
\pgfsys@useobject{currentmarker}{}%
\end{pgfscope}%
\begin{pgfscope}%
\pgfsys@transformshift{1.036735in}{1.119295in}%
\pgfsys@useobject{currentmarker}{}%
\end{pgfscope}%
\begin{pgfscope}%
\pgfsys@transformshift{1.386919in}{1.265759in}%
\pgfsys@useobject{currentmarker}{}%
\end{pgfscope}%
\begin{pgfscope}%
\pgfsys@transformshift{1.780050in}{1.305494in}%
\pgfsys@useobject{currentmarker}{}%
\end{pgfscope}%
\begin{pgfscope}%
\pgfsys@transformshift{2.954489in}{1.467208in}%
\pgfsys@useobject{currentmarker}{}%
\end{pgfscope}%
\begin{pgfscope}%
\pgfsys@transformshift{3.866840in}{1.531416in}%
\pgfsys@useobject{currentmarker}{}%
\end{pgfscope}%
\end{pgfscope}%
\begin{pgfscope}%
\pgfsetroundcap%
\pgfsetroundjoin%
\pgfsetlinewidth{1.003750pt}%
\definecolor{currentstroke}{rgb}{0.835294,0.368627,0.000000}%
\pgfsetstrokecolor{currentstroke}%
\pgfsetdash{}{0pt}%
\pgfpathmoveto{\pgfqpoint{0.736655in}{0.461958in}}%
\pgfpathlineto{\pgfqpoint{1.036735in}{0.469709in}}%
\pgfpathlineto{\pgfqpoint{1.386919in}{0.472032in}}%
\pgfpathlineto{\pgfqpoint{1.780050in}{0.475732in}}%
\pgfpathlineto{\pgfqpoint{2.954489in}{0.486545in}}%
\pgfpathlineto{\pgfqpoint{3.866840in}{0.493504in}}%
\pgfusepath{stroke}%
\end{pgfscope}%
\begin{pgfscope}%
\pgfsetbuttcap%
\pgfsetroundjoin%
\definecolor{currentfill}{rgb}{0.835294,0.368627,0.000000}%
\pgfsetfillcolor{currentfill}%
\pgfsetlinewidth{0.752812pt}%
\definecolor{currentstroke}{rgb}{1.000000,1.000000,1.000000}%
\pgfsetstrokecolor{currentstroke}%
\pgfsetdash{}{0pt}%
\pgfsys@defobject{currentmarker}{\pgfqpoint{-0.034722in}{-0.034722in}}{\pgfqpoint{0.034722in}{0.034722in}}{%
\pgfpathmoveto{\pgfqpoint{0.000000in}{-0.034722in}}%
\pgfpathcurveto{\pgfqpoint{0.009208in}{-0.034722in}}{\pgfqpoint{0.018041in}{-0.031064in}}{\pgfqpoint{0.024552in}{-0.024552in}}%
\pgfpathcurveto{\pgfqpoint{0.031064in}{-0.018041in}}{\pgfqpoint{0.034722in}{-0.009208in}}{\pgfqpoint{0.034722in}{0.000000in}}%
\pgfpathcurveto{\pgfqpoint{0.034722in}{0.009208in}}{\pgfqpoint{0.031064in}{0.018041in}}{\pgfqpoint{0.024552in}{0.024552in}}%
\pgfpathcurveto{\pgfqpoint{0.018041in}{0.031064in}}{\pgfqpoint{0.009208in}{0.034722in}}{\pgfqpoint{0.000000in}{0.034722in}}%
\pgfpathcurveto{\pgfqpoint{-0.009208in}{0.034722in}}{\pgfqpoint{-0.018041in}{0.031064in}}{\pgfqpoint{-0.024552in}{0.024552in}}%
\pgfpathcurveto{\pgfqpoint{-0.031064in}{0.018041in}}{\pgfqpoint{-0.034722in}{0.009208in}}{\pgfqpoint{-0.034722in}{0.000000in}}%
\pgfpathcurveto{\pgfqpoint{-0.034722in}{-0.009208in}}{\pgfqpoint{-0.031064in}{-0.018041in}}{\pgfqpoint{-0.024552in}{-0.024552in}}%
\pgfpathcurveto{\pgfqpoint{-0.018041in}{-0.031064in}}{\pgfqpoint{-0.009208in}{-0.034722in}}{\pgfqpoint{0.000000in}{-0.034722in}}%
\pgfpathlineto{\pgfqpoint{0.000000in}{-0.034722in}}%
\pgfpathclose%
\pgfusepath{stroke,fill}%
}%
\begin{pgfscope}%
\pgfsys@transformshift{0.736655in}{0.461958in}%
\pgfsys@useobject{currentmarker}{}%
\end{pgfscope}%
\begin{pgfscope}%
\pgfsys@transformshift{1.036735in}{0.469709in}%
\pgfsys@useobject{currentmarker}{}%
\end{pgfscope}%
\begin{pgfscope}%
\pgfsys@transformshift{1.386919in}{0.472032in}%
\pgfsys@useobject{currentmarker}{}%
\end{pgfscope}%
\begin{pgfscope}%
\pgfsys@transformshift{1.780050in}{0.475732in}%
\pgfsys@useobject{currentmarker}{}%
\end{pgfscope}%
\begin{pgfscope}%
\pgfsys@transformshift{2.954489in}{0.486545in}%
\pgfsys@useobject{currentmarker}{}%
\end{pgfscope}%
\begin{pgfscope}%
\pgfsys@transformshift{3.866840in}{0.493504in}%
\pgfsys@useobject{currentmarker}{}%
\end{pgfscope}%
\end{pgfscope}%
\end{pgfpicture}%
\makeatother%
\endgroup%

						\end{figcenter}
						\caption{Routing between the same waypoints appearing in all datasets (distance between the waypoints: 600 m).}
					\end{subfigure}
				\end{figcenter}
				\caption{Routing time statistics of the \enquote{OSM city} datasets.}
				\label{fig:eval-city-routing-details}
			\end{figure}
			
			\clearpage
			\begin{figure}[h!]
				\begin{figcenter}
					\begin{subfigure}[t]{\textwidth}
						\begin{figcenter}
							\begingroup%
\makeatletter%
\begin{pgfpicture}%
\pgfpathrectangle{\pgfpointorigin}{\pgfqpoint{6.077890in}{2.407638in}}%
\pgfusepath{use as bounding box}%
\begin{pgfscope}%
\pgfsetbuttcap%
\pgfsetmiterjoin%
\definecolor{currentfill}{rgb}{1.000000,1.000000,1.000000}%
\pgfsetfillcolor{currentfill}%
\pgfsetlinewidth{0.000000pt}%
\definecolor{currentstroke}{rgb}{1.000000,1.000000,1.000000}%
\pgfsetstrokecolor{currentstroke}%
\pgfsetdash{}{0pt}%
\pgfpathmoveto{\pgfqpoint{0.000000in}{0.000000in}}%
\pgfpathlineto{\pgfqpoint{6.077890in}{0.000000in}}%
\pgfpathlineto{\pgfqpoint{6.077890in}{2.407638in}}%
\pgfpathlineto{\pgfqpoint{0.000000in}{2.407638in}}%
\pgfpathlineto{\pgfqpoint{0.000000in}{0.000000in}}%
\pgfpathclose%
\pgfusepath{fill}%
\end{pgfscope}%
\begin{pgfscope}%
\pgfsetbuttcap%
\pgfsetmiterjoin%
\definecolor{currentfill}{rgb}{1.000000,1.000000,1.000000}%
\pgfsetfillcolor{currentfill}%
\pgfsetlinewidth{0.000000pt}%
\definecolor{currentstroke}{rgb}{0.000000,0.000000,0.000000}%
\pgfsetstrokecolor{currentstroke}%
\pgfsetstrokeopacity{0.000000}%
\pgfsetdash{}{0pt}%
\pgfpathmoveto{\pgfqpoint{0.601779in}{0.451389in}}%
\pgfpathlineto{\pgfqpoint{5.095229in}{0.451389in}}%
\pgfpathlineto{\pgfqpoint{5.095229in}{2.407638in}}%
\pgfpathlineto{\pgfqpoint{0.601779in}{2.407638in}}%
\pgfpathlineto{\pgfqpoint{0.601779in}{0.451389in}}%
\pgfpathclose%
\pgfusepath{fill}%
\end{pgfscope}%
\begin{pgfscope}%
\pgfpathrectangle{\pgfqpoint{0.601779in}{0.451389in}}{\pgfqpoint{4.493449in}{1.956249in}}%
\pgfusepath{clip}%
\pgfsetroundcap%
\pgfsetroundjoin%
\pgfsetlinewidth{1.003750pt}%
\definecolor{currentstroke}{rgb}{0.800000,0.800000,0.800000}%
\pgfsetstrokecolor{currentstroke}%
\pgfsetdash{}{0pt}%
\pgfpathmoveto{\pgfqpoint{0.601779in}{0.451389in}}%
\pgfpathlineto{\pgfqpoint{0.601779in}{2.407638in}}%
\pgfusepath{stroke}%
\end{pgfscope}%
\begin{pgfscope}%
\definecolor{textcolor}{rgb}{0.150000,0.150000,0.150000}%
\pgfsetstrokecolor{textcolor}%
\pgfsetfillcolor{textcolor}%
\pgftext[x=0.601779in,y=0.319444in,,top]{\color{textcolor}\sffamily\fontsize{9.000000}{10.800000}\selectfont 0.0}%
\end{pgfscope}%
\begin{pgfscope}%
\pgfpathrectangle{\pgfqpoint{0.601779in}{0.451389in}}{\pgfqpoint{4.493449in}{1.956249in}}%
\pgfusepath{clip}%
\pgfsetroundcap%
\pgfsetroundjoin%
\pgfsetlinewidth{1.003750pt}%
\definecolor{currentstroke}{rgb}{0.800000,0.800000,0.800000}%
\pgfsetstrokecolor{currentstroke}%
\pgfsetdash{}{0pt}%
\pgfpathmoveto{\pgfqpoint{1.460352in}{0.451389in}}%
\pgfpathlineto{\pgfqpoint{1.460352in}{2.407638in}}%
\pgfusepath{stroke}%
\end{pgfscope}%
\begin{pgfscope}%
\definecolor{textcolor}{rgb}{0.150000,0.150000,0.150000}%
\pgfsetstrokecolor{textcolor}%
\pgfsetfillcolor{textcolor}%
\pgftext[x=1.460352in,y=0.319444in,,top]{\color{textcolor}\sffamily\fontsize{9.000000}{10.800000}\selectfont 0.5}%
\end{pgfscope}%
\begin{pgfscope}%
\pgfpathrectangle{\pgfqpoint{0.601779in}{0.451389in}}{\pgfqpoint{4.493449in}{1.956249in}}%
\pgfusepath{clip}%
\pgfsetroundcap%
\pgfsetroundjoin%
\pgfsetlinewidth{1.003750pt}%
\definecolor{currentstroke}{rgb}{0.800000,0.800000,0.800000}%
\pgfsetstrokecolor{currentstroke}%
\pgfsetdash{}{0pt}%
\pgfpathmoveto{\pgfqpoint{2.318924in}{0.451389in}}%
\pgfpathlineto{\pgfqpoint{2.318924in}{2.407638in}}%
\pgfusepath{stroke}%
\end{pgfscope}%
\begin{pgfscope}%
\definecolor{textcolor}{rgb}{0.150000,0.150000,0.150000}%
\pgfsetstrokecolor{textcolor}%
\pgfsetfillcolor{textcolor}%
\pgftext[x=2.318924in,y=0.319444in,,top]{\color{textcolor}\sffamily\fontsize{9.000000}{10.800000}\selectfont 1.0}%
\end{pgfscope}%
\begin{pgfscope}%
\pgfpathrectangle{\pgfqpoint{0.601779in}{0.451389in}}{\pgfqpoint{4.493449in}{1.956249in}}%
\pgfusepath{clip}%
\pgfsetroundcap%
\pgfsetroundjoin%
\pgfsetlinewidth{1.003750pt}%
\definecolor{currentstroke}{rgb}{0.800000,0.800000,0.800000}%
\pgfsetstrokecolor{currentstroke}%
\pgfsetdash{}{0pt}%
\pgfpathmoveto{\pgfqpoint{3.177497in}{0.451389in}}%
\pgfpathlineto{\pgfqpoint{3.177497in}{2.407638in}}%
\pgfusepath{stroke}%
\end{pgfscope}%
\begin{pgfscope}%
\definecolor{textcolor}{rgb}{0.150000,0.150000,0.150000}%
\pgfsetstrokecolor{textcolor}%
\pgfsetfillcolor{textcolor}%
\pgftext[x=3.177497in,y=0.319444in,,top]{\color{textcolor}\sffamily\fontsize{9.000000}{10.800000}\selectfont 1.5}%
\end{pgfscope}%
\begin{pgfscope}%
\pgfpathrectangle{\pgfqpoint{0.601779in}{0.451389in}}{\pgfqpoint{4.493449in}{1.956249in}}%
\pgfusepath{clip}%
\pgfsetroundcap%
\pgfsetroundjoin%
\pgfsetlinewidth{1.003750pt}%
\definecolor{currentstroke}{rgb}{0.800000,0.800000,0.800000}%
\pgfsetstrokecolor{currentstroke}%
\pgfsetdash{}{0pt}%
\pgfpathmoveto{\pgfqpoint{4.036069in}{0.451389in}}%
\pgfpathlineto{\pgfqpoint{4.036069in}{2.407638in}}%
\pgfusepath{stroke}%
\end{pgfscope}%
\begin{pgfscope}%
\definecolor{textcolor}{rgb}{0.150000,0.150000,0.150000}%
\pgfsetstrokecolor{textcolor}%
\pgfsetfillcolor{textcolor}%
\pgftext[x=4.036069in,y=0.319444in,,top]{\color{textcolor}\sffamily\fontsize{9.000000}{10.800000}\selectfont 2.0}%
\end{pgfscope}%
\begin{pgfscope}%
\pgfpathrectangle{\pgfqpoint{0.601779in}{0.451389in}}{\pgfqpoint{4.493449in}{1.956249in}}%
\pgfusepath{clip}%
\pgfsetroundcap%
\pgfsetroundjoin%
\pgfsetlinewidth{1.003750pt}%
\definecolor{currentstroke}{rgb}{0.800000,0.800000,0.800000}%
\pgfsetstrokecolor{currentstroke}%
\pgfsetdash{}{0pt}%
\pgfpathmoveto{\pgfqpoint{4.894641in}{0.451389in}}%
\pgfpathlineto{\pgfqpoint{4.894641in}{2.407638in}}%
\pgfusepath{stroke}%
\end{pgfscope}%
\begin{pgfscope}%
\definecolor{textcolor}{rgb}{0.150000,0.150000,0.150000}%
\pgfsetstrokecolor{textcolor}%
\pgfsetfillcolor{textcolor}%
\pgftext[x=4.894641in,y=0.319444in,,top]{\color{textcolor}\sffamily\fontsize{9.000000}{10.800000}\selectfont 2.5}%
\end{pgfscope}%
\begin{pgfscope}%
\definecolor{textcolor}{rgb}{0.150000,0.150000,0.150000}%
\pgfsetstrokecolor{textcolor}%
\pgfsetfillcolor{textcolor}%
\pgftext[x=2.848504in,y=0.125000in,,top]{\color{textcolor}\sffamily\fontsize{9.000000}{10.800000}\selectfont Beeline distance in km}%
\end{pgfscope}%
\begin{pgfscope}%
\pgfpathrectangle{\pgfqpoint{0.601779in}{0.451389in}}{\pgfqpoint{4.493449in}{1.956249in}}%
\pgfusepath{clip}%
\pgfsetroundcap%
\pgfsetroundjoin%
\pgfsetlinewidth{1.003750pt}%
\definecolor{currentstroke}{rgb}{0.800000,0.800000,0.800000}%
\pgfsetstrokecolor{currentstroke}%
\pgfsetdash{}{0pt}%
\pgfpathmoveto{\pgfqpoint{0.601779in}{0.451389in}}%
\pgfpathlineto{\pgfqpoint{5.095229in}{0.451389in}}%
\pgfusepath{stroke}%
\end{pgfscope}%
\begin{pgfscope}%
\definecolor{textcolor}{rgb}{0.150000,0.150000,0.150000}%
\pgfsetstrokecolor{textcolor}%
\pgfsetfillcolor{textcolor}%
\pgftext[x=0.400987in, y=0.403903in, left, base]{\color{textcolor}\sffamily\fontsize{9.000000}{10.800000}\selectfont 0}%
\end{pgfscope}%
\begin{pgfscope}%
\pgfpathrectangle{\pgfqpoint{0.601779in}{0.451389in}}{\pgfqpoint{4.493449in}{1.956249in}}%
\pgfusepath{clip}%
\pgfsetroundcap%
\pgfsetroundjoin%
\pgfsetlinewidth{1.003750pt}%
\definecolor{currentstroke}{rgb}{0.800000,0.800000,0.800000}%
\pgfsetstrokecolor{currentstroke}%
\pgfsetdash{}{0pt}%
\pgfpathmoveto{\pgfqpoint{0.601779in}{0.746065in}}%
\pgfpathlineto{\pgfqpoint{5.095229in}{0.746065in}}%
\pgfusepath{stroke}%
\end{pgfscope}%
\begin{pgfscope}%
\definecolor{textcolor}{rgb}{0.150000,0.150000,0.150000}%
\pgfsetstrokecolor{textcolor}%
\pgfsetfillcolor{textcolor}%
\pgftext[x=0.263292in, y=0.698580in, left, base]{\color{textcolor}\sffamily\fontsize{9.000000}{10.800000}\selectfont 250}%
\end{pgfscope}%
\begin{pgfscope}%
\pgfpathrectangle{\pgfqpoint{0.601779in}{0.451389in}}{\pgfqpoint{4.493449in}{1.956249in}}%
\pgfusepath{clip}%
\pgfsetroundcap%
\pgfsetroundjoin%
\pgfsetlinewidth{1.003750pt}%
\definecolor{currentstroke}{rgb}{0.800000,0.800000,0.800000}%
\pgfsetstrokecolor{currentstroke}%
\pgfsetdash{}{0pt}%
\pgfpathmoveto{\pgfqpoint{0.601779in}{1.040741in}}%
\pgfpathlineto{\pgfqpoint{5.095229in}{1.040741in}}%
\pgfusepath{stroke}%
\end{pgfscope}%
\begin{pgfscope}%
\definecolor{textcolor}{rgb}{0.150000,0.150000,0.150000}%
\pgfsetstrokecolor{textcolor}%
\pgfsetfillcolor{textcolor}%
\pgftext[x=0.263292in, y=0.993256in, left, base]{\color{textcolor}\sffamily\fontsize{9.000000}{10.800000}\selectfont 500}%
\end{pgfscope}%
\begin{pgfscope}%
\pgfpathrectangle{\pgfqpoint{0.601779in}{0.451389in}}{\pgfqpoint{4.493449in}{1.956249in}}%
\pgfusepath{clip}%
\pgfsetroundcap%
\pgfsetroundjoin%
\pgfsetlinewidth{1.003750pt}%
\definecolor{currentstroke}{rgb}{0.800000,0.800000,0.800000}%
\pgfsetstrokecolor{currentstroke}%
\pgfsetdash{}{0pt}%
\pgfpathmoveto{\pgfqpoint{0.601779in}{1.335418in}}%
\pgfpathlineto{\pgfqpoint{5.095229in}{1.335418in}}%
\pgfusepath{stroke}%
\end{pgfscope}%
\begin{pgfscope}%
\definecolor{textcolor}{rgb}{0.150000,0.150000,0.150000}%
\pgfsetstrokecolor{textcolor}%
\pgfsetfillcolor{textcolor}%
\pgftext[x=0.263292in, y=1.287933in, left, base]{\color{textcolor}\sffamily\fontsize{9.000000}{10.800000}\selectfont 750}%
\end{pgfscope}%
\begin{pgfscope}%
\pgfpathrectangle{\pgfqpoint{0.601779in}{0.451389in}}{\pgfqpoint{4.493449in}{1.956249in}}%
\pgfusepath{clip}%
\pgfsetroundcap%
\pgfsetroundjoin%
\pgfsetlinewidth{1.003750pt}%
\definecolor{currentstroke}{rgb}{0.800000,0.800000,0.800000}%
\pgfsetstrokecolor{currentstroke}%
\pgfsetdash{}{0pt}%
\pgfpathmoveto{\pgfqpoint{0.601779in}{1.630094in}}%
\pgfpathlineto{\pgfqpoint{5.095229in}{1.630094in}}%
\pgfusepath{stroke}%
\end{pgfscope}%
\begin{pgfscope}%
\definecolor{textcolor}{rgb}{0.150000,0.150000,0.150000}%
\pgfsetstrokecolor{textcolor}%
\pgfsetfillcolor{textcolor}%
\pgftext[x=0.194444in, y=1.582609in, left, base]{\color{textcolor}\sffamily\fontsize{9.000000}{10.800000}\selectfont 1000}%
\end{pgfscope}%
\begin{pgfscope}%
\pgfpathrectangle{\pgfqpoint{0.601779in}{0.451389in}}{\pgfqpoint{4.493449in}{1.956249in}}%
\pgfusepath{clip}%
\pgfsetroundcap%
\pgfsetroundjoin%
\pgfsetlinewidth{1.003750pt}%
\definecolor{currentstroke}{rgb}{0.800000,0.800000,0.800000}%
\pgfsetstrokecolor{currentstroke}%
\pgfsetdash{}{0pt}%
\pgfpathmoveto{\pgfqpoint{0.601779in}{1.924771in}}%
\pgfpathlineto{\pgfqpoint{5.095229in}{1.924771in}}%
\pgfusepath{stroke}%
\end{pgfscope}%
\begin{pgfscope}%
\definecolor{textcolor}{rgb}{0.150000,0.150000,0.150000}%
\pgfsetstrokecolor{textcolor}%
\pgfsetfillcolor{textcolor}%
\pgftext[x=0.194444in, y=1.877285in, left, base]{\color{textcolor}\sffamily\fontsize{9.000000}{10.800000}\selectfont 1250}%
\end{pgfscope}%
\begin{pgfscope}%
\pgfpathrectangle{\pgfqpoint{0.601779in}{0.451389in}}{\pgfqpoint{4.493449in}{1.956249in}}%
\pgfusepath{clip}%
\pgfsetroundcap%
\pgfsetroundjoin%
\pgfsetlinewidth{1.003750pt}%
\definecolor{currentstroke}{rgb}{0.800000,0.800000,0.800000}%
\pgfsetstrokecolor{currentstroke}%
\pgfsetdash{}{0pt}%
\pgfpathmoveto{\pgfqpoint{0.601779in}{2.219447in}}%
\pgfpathlineto{\pgfqpoint{5.095229in}{2.219447in}}%
\pgfusepath{stroke}%
\end{pgfscope}%
\begin{pgfscope}%
\definecolor{textcolor}{rgb}{0.150000,0.150000,0.150000}%
\pgfsetstrokecolor{textcolor}%
\pgfsetfillcolor{textcolor}%
\pgftext[x=0.194444in, y=2.171962in, left, base]{\color{textcolor}\sffamily\fontsize{9.000000}{10.800000}\selectfont 1500}%
\end{pgfscope}%
\begin{pgfscope}%
\definecolor{textcolor}{rgb}{0.150000,0.150000,0.150000}%
\pgfsetstrokecolor{textcolor}%
\pgfsetfillcolor{textcolor}%
\pgftext[x=0.125000in,y=1.429513in,,bottom,rotate=90.000000]{\color{textcolor}\sffamily\fontsize{9.000000}{10.800000}\selectfont Average routing time in ms}%
\end{pgfscope}%
\begin{pgfscope}%
\pgfsetrectcap%
\pgfsetmiterjoin%
\pgfsetlinewidth{1.254687pt}%
\definecolor{currentstroke}{rgb}{0.800000,0.800000,0.800000}%
\pgfsetstrokecolor{currentstroke}%
\pgfsetdash{}{0pt}%
\pgfpathmoveto{\pgfqpoint{0.601779in}{0.451389in}}%
\pgfpathlineto{\pgfqpoint{0.601779in}{2.407638in}}%
\pgfusepath{stroke}%
\end{pgfscope}%
\begin{pgfscope}%
\pgfsetrectcap%
\pgfsetmiterjoin%
\pgfsetlinewidth{1.254687pt}%
\definecolor{currentstroke}{rgb}{0.800000,0.800000,0.800000}%
\pgfsetstrokecolor{currentstroke}%
\pgfsetdash{}{0pt}%
\pgfpathmoveto{\pgfqpoint{5.095229in}{0.451389in}}%
\pgfpathlineto{\pgfqpoint{5.095229in}{2.407638in}}%
\pgfusepath{stroke}%
\end{pgfscope}%
\begin{pgfscope}%
\pgfsetrectcap%
\pgfsetmiterjoin%
\pgfsetlinewidth{1.254687pt}%
\definecolor{currentstroke}{rgb}{0.800000,0.800000,0.800000}%
\pgfsetstrokecolor{currentstroke}%
\pgfsetdash{}{0pt}%
\pgfpathmoveto{\pgfqpoint{0.601779in}{0.451389in}}%
\pgfpathlineto{\pgfqpoint{5.095229in}{0.451389in}}%
\pgfusepath{stroke}%
\end{pgfscope}%
\begin{pgfscope}%
\pgfsetrectcap%
\pgfsetmiterjoin%
\pgfsetlinewidth{1.254687pt}%
\definecolor{currentstroke}{rgb}{0.800000,0.800000,0.800000}%
\pgfsetstrokecolor{currentstroke}%
\pgfsetdash{}{0pt}%
\pgfpathmoveto{\pgfqpoint{0.601779in}{2.407638in}}%
\pgfpathlineto{\pgfqpoint{5.095229in}{2.407638in}}%
\pgfusepath{stroke}%
\end{pgfscope}%
\begin{pgfscope}%
\pgfsetbuttcap%
\pgfsetmiterjoin%
\definecolor{currentfill}{rgb}{1.000000,1.000000,1.000000}%
\pgfsetfillcolor{currentfill}%
\pgfsetfillopacity{0.800000}%
\pgfsetlinewidth{1.003750pt}%
\definecolor{currentstroke}{rgb}{0.800000,0.800000,0.800000}%
\pgfsetstrokecolor{currentstroke}%
\pgfsetstrokeopacity{0.800000}%
\pgfsetdash{}{0pt}%
\pgfpathmoveto{\pgfqpoint{5.295065in}{0.754514in}}%
\pgfpathlineto{\pgfqpoint{6.052890in}{0.754514in}}%
\pgfpathquadraticcurveto{\pgfqpoint{6.077890in}{0.754514in}}{\pgfqpoint{6.077890in}{0.779514in}}%
\pgfpathlineto{\pgfqpoint{6.077890in}{2.079513in}}%
\pgfpathquadraticcurveto{\pgfqpoint{6.077890in}{2.104513in}}{\pgfqpoint{6.052890in}{2.104513in}}%
\pgfpathlineto{\pgfqpoint{5.295065in}{2.104513in}}%
\pgfpathquadraticcurveto{\pgfqpoint{5.270065in}{2.104513in}}{\pgfqpoint{5.270065in}{2.079513in}}%
\pgfpathlineto{\pgfqpoint{5.270065in}{0.779514in}}%
\pgfpathquadraticcurveto{\pgfqpoint{5.270065in}{0.754514in}}{\pgfqpoint{5.295065in}{0.754514in}}%
\pgfpathlineto{\pgfqpoint{5.295065in}{0.754514in}}%
\pgfpathclose%
\pgfusepath{stroke,fill}%
\end{pgfscope}%
\begin{pgfscope}%
\definecolor{textcolor}{rgb}{0.150000,0.150000,0.150000}%
\pgfsetstrokecolor{textcolor}%
\pgfsetfillcolor{textcolor}%
\pgftext[x=5.320065in,y=1.959542in,left,base]{\color{textcolor}\sffamily\fontsize{9.000000}{10.800000}\selectfont Vertex count}%
\end{pgfscope}%
\begin{pgfscope}%
\pgfsetroundcap%
\pgfsetroundjoin%
\pgfsetlinewidth{1.505625pt}%
\definecolor{currentstroke}{rgb}{0.003922,0.450980,0.698039}%
\pgfsetstrokecolor{currentstroke}%
\pgfsetdash{}{0pt}%
\pgfpathmoveto{\pgfqpoint{5.326858in}{1.815792in}}%
\pgfpathlineto{\pgfqpoint{5.451858in}{1.815792in}}%
\pgfpathlineto{\pgfqpoint{5.576858in}{1.815792in}}%
\pgfusepath{stroke}%
\end{pgfscope}%
\begin{pgfscope}%
\definecolor{textcolor}{rgb}{0.150000,0.150000,0.150000}%
\pgfsetstrokecolor{textcolor}%
\pgfsetfillcolor{textcolor}%
\pgftext[x=5.676858in,y=1.772042in,left,base]{\color{textcolor}\sffamily\fontsize{9.000000}{10.800000}\selectfont 7096}%
\end{pgfscope}%
\begin{pgfscope}%
\pgfsetroundcap%
\pgfsetroundjoin%
\pgfsetlinewidth{1.505625pt}%
\definecolor{currentstroke}{rgb}{0.870588,0.560784,0.019608}%
\pgfsetstrokecolor{currentstroke}%
\pgfsetdash{}{0pt}%
\pgfpathmoveto{\pgfqpoint{5.326858in}{1.628292in}}%
\pgfpathlineto{\pgfqpoint{5.451858in}{1.628292in}}%
\pgfpathlineto{\pgfqpoint{5.576858in}{1.628292in}}%
\pgfusepath{stroke}%
\end{pgfscope}%
\begin{pgfscope}%
\definecolor{textcolor}{rgb}{0.150000,0.150000,0.150000}%
\pgfsetstrokecolor{textcolor}%
\pgfsetfillcolor{textcolor}%
\pgftext[x=5.676858in,y=1.584542in,left,base]{\color{textcolor}\sffamily\fontsize{9.000000}{10.800000}\selectfont 12858}%
\end{pgfscope}%
\begin{pgfscope}%
\pgfsetroundcap%
\pgfsetroundjoin%
\pgfsetlinewidth{1.505625pt}%
\definecolor{currentstroke}{rgb}{0.007843,0.619608,0.450980}%
\pgfsetstrokecolor{currentstroke}%
\pgfsetdash{}{0pt}%
\pgfpathmoveto{\pgfqpoint{5.326858in}{1.440792in}}%
\pgfpathlineto{\pgfqpoint{5.451858in}{1.440792in}}%
\pgfpathlineto{\pgfqpoint{5.576858in}{1.440792in}}%
\pgfusepath{stroke}%
\end{pgfscope}%
\begin{pgfscope}%
\definecolor{textcolor}{rgb}{0.150000,0.150000,0.150000}%
\pgfsetstrokecolor{textcolor}%
\pgfsetfillcolor{textcolor}%
\pgftext[x=5.676858in,y=1.397042in,left,base]{\color{textcolor}\sffamily\fontsize{9.000000}{10.800000}\selectfont 17894}%
\end{pgfscope}%
\begin{pgfscope}%
\pgfsetroundcap%
\pgfsetroundjoin%
\pgfsetlinewidth{1.505625pt}%
\definecolor{currentstroke}{rgb}{0.835294,0.368627,0.000000}%
\pgfsetstrokecolor{currentstroke}%
\pgfsetdash{}{0pt}%
\pgfpathmoveto{\pgfqpoint{5.326858in}{1.253293in}}%
\pgfpathlineto{\pgfqpoint{5.451858in}{1.253293in}}%
\pgfpathlineto{\pgfqpoint{5.576858in}{1.253293in}}%
\pgfusepath{stroke}%
\end{pgfscope}%
\begin{pgfscope}%
\definecolor{textcolor}{rgb}{0.150000,0.150000,0.150000}%
\pgfsetstrokecolor{textcolor}%
\pgfsetfillcolor{textcolor}%
\pgftext[x=5.676858in,y=1.209543in,left,base]{\color{textcolor}\sffamily\fontsize{9.000000}{10.800000}\selectfont 22801}%
\end{pgfscope}%
\begin{pgfscope}%
\pgfsetroundcap%
\pgfsetroundjoin%
\pgfsetlinewidth{1.505625pt}%
\definecolor{currentstroke}{rgb}{0.800000,0.470588,0.737255}%
\pgfsetstrokecolor{currentstroke}%
\pgfsetdash{}{0pt}%
\pgfpathmoveto{\pgfqpoint{5.326858in}{1.065793in}}%
\pgfpathlineto{\pgfqpoint{5.451858in}{1.065793in}}%
\pgfpathlineto{\pgfqpoint{5.576858in}{1.065793in}}%
\pgfusepath{stroke}%
\end{pgfscope}%
\begin{pgfscope}%
\definecolor{textcolor}{rgb}{0.150000,0.150000,0.150000}%
\pgfsetstrokecolor{textcolor}%
\pgfsetfillcolor{textcolor}%
\pgftext[x=5.676858in,y=1.022043in,left,base]{\color{textcolor}\sffamily\fontsize{9.000000}{10.800000}\selectfont 34214}%
\end{pgfscope}%
\begin{pgfscope}%
\pgfsetroundcap%
\pgfsetroundjoin%
\pgfsetlinewidth{1.505625pt}%
\definecolor{currentstroke}{rgb}{0.792157,0.568627,0.380392}%
\pgfsetstrokecolor{currentstroke}%
\pgfsetdash{}{0pt}%
\pgfpathmoveto{\pgfqpoint{5.326858in}{0.878293in}}%
\pgfpathlineto{\pgfqpoint{5.451858in}{0.878293in}}%
\pgfpathlineto{\pgfqpoint{5.576858in}{0.878293in}}%
\pgfusepath{stroke}%
\end{pgfscope}%
\begin{pgfscope}%
\definecolor{textcolor}{rgb}{0.150000,0.150000,0.150000}%
\pgfsetstrokecolor{textcolor}%
\pgfsetfillcolor{textcolor}%
\pgftext[x=5.676858in,y=0.834543in,left,base]{\color{textcolor}\sffamily\fontsize{9.000000}{10.800000}\selectfont 45018}%
\end{pgfscope}%
\begin{pgfscope}%
\pgfsetroundcap%
\pgfsetroundjoin%
\pgfsetlinewidth{1.003750pt}%
\definecolor{currentstroke}{rgb}{0.003922,0.450980,0.698039}%
\pgfsetstrokecolor{currentstroke}%
\pgfsetdash{}{0pt}%
\pgfpathmoveto{\pgfqpoint{0.773669in}{0.589481in}}%
\pgfpathlineto{\pgfqpoint{0.859949in}{0.561639in}}%
\pgfpathlineto{\pgfqpoint{0.945274in}{0.585074in}}%
\pgfpathlineto{\pgfqpoint{1.030985in}{0.601652in}}%
\pgfpathlineto{\pgfqpoint{1.116621in}{0.590699in}}%
\pgfpathlineto{\pgfqpoint{1.202341in}{0.536827in}}%
\pgfpathlineto{\pgfqpoint{1.289216in}{0.621530in}}%
\pgfpathlineto{\pgfqpoint{1.373880in}{0.615911in}}%
\pgfpathlineto{\pgfqpoint{1.459905in}{0.600239in}}%
\pgfpathlineto{\pgfqpoint{1.546396in}{0.623906in}}%
\pgfpathlineto{\pgfqpoint{1.631151in}{0.588148in}}%
\pgfusepath{stroke}%
\end{pgfscope}%
\begin{pgfscope}%
\pgfsetbuttcap%
\pgfsetroundjoin%
\definecolor{currentfill}{rgb}{0.003922,0.450980,0.698039}%
\pgfsetfillcolor{currentfill}%
\pgfsetlinewidth{0.752812pt}%
\definecolor{currentstroke}{rgb}{1.000000,1.000000,1.000000}%
\pgfsetstrokecolor{currentstroke}%
\pgfsetdash{}{0pt}%
\pgfsys@defobject{currentmarker}{\pgfqpoint{-0.034722in}{-0.034722in}}{\pgfqpoint{0.034722in}{0.034722in}}{%
\pgfpathmoveto{\pgfqpoint{0.000000in}{-0.034722in}}%
\pgfpathcurveto{\pgfqpoint{0.009208in}{-0.034722in}}{\pgfqpoint{0.018041in}{-0.031064in}}{\pgfqpoint{0.024552in}{-0.024552in}}%
\pgfpathcurveto{\pgfqpoint{0.031064in}{-0.018041in}}{\pgfqpoint{0.034722in}{-0.009208in}}{\pgfqpoint{0.034722in}{0.000000in}}%
\pgfpathcurveto{\pgfqpoint{0.034722in}{0.009208in}}{\pgfqpoint{0.031064in}{0.018041in}}{\pgfqpoint{0.024552in}{0.024552in}}%
\pgfpathcurveto{\pgfqpoint{0.018041in}{0.031064in}}{\pgfqpoint{0.009208in}{0.034722in}}{\pgfqpoint{0.000000in}{0.034722in}}%
\pgfpathcurveto{\pgfqpoint{-0.009208in}{0.034722in}}{\pgfqpoint{-0.018041in}{0.031064in}}{\pgfqpoint{-0.024552in}{0.024552in}}%
\pgfpathcurveto{\pgfqpoint{-0.031064in}{0.018041in}}{\pgfqpoint{-0.034722in}{0.009208in}}{\pgfqpoint{-0.034722in}{0.000000in}}%
\pgfpathcurveto{\pgfqpoint{-0.034722in}{-0.009208in}}{\pgfqpoint{-0.031064in}{-0.018041in}}{\pgfqpoint{-0.024552in}{-0.024552in}}%
\pgfpathcurveto{\pgfqpoint{-0.018041in}{-0.031064in}}{\pgfqpoint{-0.009208in}{-0.034722in}}{\pgfqpoint{0.000000in}{-0.034722in}}%
\pgfpathlineto{\pgfqpoint{0.000000in}{-0.034722in}}%
\pgfpathclose%
\pgfusepath{stroke,fill}%
}%
\begin{pgfscope}%
\pgfsys@transformshift{0.773669in}{0.589481in}%
\pgfsys@useobject{currentmarker}{}%
\end{pgfscope}%
\begin{pgfscope}%
\pgfsys@transformshift{0.859949in}{0.561639in}%
\pgfsys@useobject{currentmarker}{}%
\end{pgfscope}%
\begin{pgfscope}%
\pgfsys@transformshift{0.945274in}{0.585074in}%
\pgfsys@useobject{currentmarker}{}%
\end{pgfscope}%
\begin{pgfscope}%
\pgfsys@transformshift{1.030985in}{0.601652in}%
\pgfsys@useobject{currentmarker}{}%
\end{pgfscope}%
\begin{pgfscope}%
\pgfsys@transformshift{1.116621in}{0.590699in}%
\pgfsys@useobject{currentmarker}{}%
\end{pgfscope}%
\begin{pgfscope}%
\pgfsys@transformshift{1.202341in}{0.536827in}%
\pgfsys@useobject{currentmarker}{}%
\end{pgfscope}%
\begin{pgfscope}%
\pgfsys@transformshift{1.289216in}{0.621530in}%
\pgfsys@useobject{currentmarker}{}%
\end{pgfscope}%
\begin{pgfscope}%
\pgfsys@transformshift{1.373880in}{0.615911in}%
\pgfsys@useobject{currentmarker}{}%
\end{pgfscope}%
\begin{pgfscope}%
\pgfsys@transformshift{1.459905in}{0.600239in}%
\pgfsys@useobject{currentmarker}{}%
\end{pgfscope}%
\begin{pgfscope}%
\pgfsys@transformshift{1.546396in}{0.623906in}%
\pgfsys@useobject{currentmarker}{}%
\end{pgfscope}%
\begin{pgfscope}%
\pgfsys@transformshift{1.631151in}{0.588148in}%
\pgfsys@useobject{currentmarker}{}%
\end{pgfscope}%
\end{pgfscope}%
\begin{pgfscope}%
\pgfsetroundcap%
\pgfsetroundjoin%
\pgfsetlinewidth{1.003750pt}%
\definecolor{currentstroke}{rgb}{0.870588,0.560784,0.019608}%
\pgfsetstrokecolor{currentstroke}%
\pgfsetdash{}{0pt}%
\pgfpathmoveto{\pgfqpoint{0.773669in}{0.684786in}}%
\pgfpathlineto{\pgfqpoint{0.859949in}{0.638408in}}%
\pgfpathlineto{\pgfqpoint{0.945274in}{0.670094in}}%
\pgfpathlineto{\pgfqpoint{1.030985in}{0.682028in}}%
\pgfpathlineto{\pgfqpoint{1.116621in}{0.682021in}}%
\pgfpathlineto{\pgfqpoint{1.202341in}{0.669268in}}%
\pgfpathlineto{\pgfqpoint{1.289216in}{0.767035in}}%
\pgfpathlineto{\pgfqpoint{1.373880in}{0.732647in}}%
\pgfpathlineto{\pgfqpoint{1.459905in}{0.697560in}}%
\pgfpathlineto{\pgfqpoint{1.546396in}{0.762117in}}%
\pgfpathlineto{\pgfqpoint{1.631151in}{0.708791in}}%
\pgfpathlineto{\pgfqpoint{1.717283in}{0.731041in}}%
\pgfpathlineto{\pgfqpoint{1.803300in}{0.792758in}}%
\pgfpathlineto{\pgfqpoint{1.888208in}{0.715685in}}%
\pgfpathlineto{\pgfqpoint{1.973986in}{0.702570in}}%
\pgfpathlineto{\pgfqpoint{2.061275in}{0.666504in}}%
\pgfpathlineto{\pgfqpoint{2.146813in}{0.689132in}}%
\pgfpathlineto{\pgfqpoint{2.232356in}{0.708976in}}%
\pgfpathlineto{\pgfqpoint{2.325145in}{0.659507in}}%
\pgfusepath{stroke}%
\end{pgfscope}%
\begin{pgfscope}%
\pgfsetbuttcap%
\pgfsetroundjoin%
\definecolor{currentfill}{rgb}{0.870588,0.560784,0.019608}%
\pgfsetfillcolor{currentfill}%
\pgfsetlinewidth{0.752812pt}%
\definecolor{currentstroke}{rgb}{1.000000,1.000000,1.000000}%
\pgfsetstrokecolor{currentstroke}%
\pgfsetdash{}{0pt}%
\pgfsys@defobject{currentmarker}{\pgfqpoint{-0.034722in}{-0.034722in}}{\pgfqpoint{0.034722in}{0.034722in}}{%
\pgfpathmoveto{\pgfqpoint{0.000000in}{-0.034722in}}%
\pgfpathcurveto{\pgfqpoint{0.009208in}{-0.034722in}}{\pgfqpoint{0.018041in}{-0.031064in}}{\pgfqpoint{0.024552in}{-0.024552in}}%
\pgfpathcurveto{\pgfqpoint{0.031064in}{-0.018041in}}{\pgfqpoint{0.034722in}{-0.009208in}}{\pgfqpoint{0.034722in}{0.000000in}}%
\pgfpathcurveto{\pgfqpoint{0.034722in}{0.009208in}}{\pgfqpoint{0.031064in}{0.018041in}}{\pgfqpoint{0.024552in}{0.024552in}}%
\pgfpathcurveto{\pgfqpoint{0.018041in}{0.031064in}}{\pgfqpoint{0.009208in}{0.034722in}}{\pgfqpoint{0.000000in}{0.034722in}}%
\pgfpathcurveto{\pgfqpoint{-0.009208in}{0.034722in}}{\pgfqpoint{-0.018041in}{0.031064in}}{\pgfqpoint{-0.024552in}{0.024552in}}%
\pgfpathcurveto{\pgfqpoint{-0.031064in}{0.018041in}}{\pgfqpoint{-0.034722in}{0.009208in}}{\pgfqpoint{-0.034722in}{0.000000in}}%
\pgfpathcurveto{\pgfqpoint{-0.034722in}{-0.009208in}}{\pgfqpoint{-0.031064in}{-0.018041in}}{\pgfqpoint{-0.024552in}{-0.024552in}}%
\pgfpathcurveto{\pgfqpoint{-0.018041in}{-0.031064in}}{\pgfqpoint{-0.009208in}{-0.034722in}}{\pgfqpoint{0.000000in}{-0.034722in}}%
\pgfpathlineto{\pgfqpoint{0.000000in}{-0.034722in}}%
\pgfpathclose%
\pgfusepath{stroke,fill}%
}%
\begin{pgfscope}%
\pgfsys@transformshift{0.773669in}{0.684786in}%
\pgfsys@useobject{currentmarker}{}%
\end{pgfscope}%
\begin{pgfscope}%
\pgfsys@transformshift{0.859949in}{0.638408in}%
\pgfsys@useobject{currentmarker}{}%
\end{pgfscope}%
\begin{pgfscope}%
\pgfsys@transformshift{0.945274in}{0.670094in}%
\pgfsys@useobject{currentmarker}{}%
\end{pgfscope}%
\begin{pgfscope}%
\pgfsys@transformshift{1.030985in}{0.682028in}%
\pgfsys@useobject{currentmarker}{}%
\end{pgfscope}%
\begin{pgfscope}%
\pgfsys@transformshift{1.116621in}{0.682021in}%
\pgfsys@useobject{currentmarker}{}%
\end{pgfscope}%
\begin{pgfscope}%
\pgfsys@transformshift{1.202341in}{0.669268in}%
\pgfsys@useobject{currentmarker}{}%
\end{pgfscope}%
\begin{pgfscope}%
\pgfsys@transformshift{1.289216in}{0.767035in}%
\pgfsys@useobject{currentmarker}{}%
\end{pgfscope}%
\begin{pgfscope}%
\pgfsys@transformshift{1.373880in}{0.732647in}%
\pgfsys@useobject{currentmarker}{}%
\end{pgfscope}%
\begin{pgfscope}%
\pgfsys@transformshift{1.459905in}{0.697560in}%
\pgfsys@useobject{currentmarker}{}%
\end{pgfscope}%
\begin{pgfscope}%
\pgfsys@transformshift{1.546396in}{0.762117in}%
\pgfsys@useobject{currentmarker}{}%
\end{pgfscope}%
\begin{pgfscope}%
\pgfsys@transformshift{1.631151in}{0.708791in}%
\pgfsys@useobject{currentmarker}{}%
\end{pgfscope}%
\begin{pgfscope}%
\pgfsys@transformshift{1.717283in}{0.731041in}%
\pgfsys@useobject{currentmarker}{}%
\end{pgfscope}%
\begin{pgfscope}%
\pgfsys@transformshift{1.803300in}{0.792758in}%
\pgfsys@useobject{currentmarker}{}%
\end{pgfscope}%
\begin{pgfscope}%
\pgfsys@transformshift{1.888208in}{0.715685in}%
\pgfsys@useobject{currentmarker}{}%
\end{pgfscope}%
\begin{pgfscope}%
\pgfsys@transformshift{1.973986in}{0.702570in}%
\pgfsys@useobject{currentmarker}{}%
\end{pgfscope}%
\begin{pgfscope}%
\pgfsys@transformshift{2.061275in}{0.666504in}%
\pgfsys@useobject{currentmarker}{}%
\end{pgfscope}%
\begin{pgfscope}%
\pgfsys@transformshift{2.146813in}{0.689132in}%
\pgfsys@useobject{currentmarker}{}%
\end{pgfscope}%
\begin{pgfscope}%
\pgfsys@transformshift{2.232356in}{0.708976in}%
\pgfsys@useobject{currentmarker}{}%
\end{pgfscope}%
\begin{pgfscope}%
\pgfsys@transformshift{2.325145in}{0.659507in}%
\pgfsys@useobject{currentmarker}{}%
\end{pgfscope}%
\end{pgfscope}%
\begin{pgfscope}%
\pgfsetroundcap%
\pgfsetroundjoin%
\pgfsetlinewidth{1.003750pt}%
\definecolor{currentstroke}{rgb}{0.007843,0.619608,0.450980}%
\pgfsetstrokecolor{currentstroke}%
\pgfsetdash{}{0pt}%
\pgfpathmoveto{\pgfqpoint{0.773669in}{0.889528in}}%
\pgfpathlineto{\pgfqpoint{0.859949in}{0.739647in}}%
\pgfpathlineto{\pgfqpoint{0.945274in}{0.791545in}}%
\pgfpathlineto{\pgfqpoint{1.030985in}{0.822840in}}%
\pgfpathlineto{\pgfqpoint{1.116621in}{0.817002in}}%
\pgfpathlineto{\pgfqpoint{1.202341in}{0.833371in}}%
\pgfpathlineto{\pgfqpoint{1.289216in}{1.001852in}}%
\pgfpathlineto{\pgfqpoint{1.373880in}{0.956574in}}%
\pgfpathlineto{\pgfqpoint{1.459905in}{0.851276in}}%
\pgfpathlineto{\pgfqpoint{1.546396in}{0.984260in}}%
\pgfpathlineto{\pgfqpoint{1.631151in}{0.881118in}}%
\pgfpathlineto{\pgfqpoint{1.717283in}{0.909652in}}%
\pgfpathlineto{\pgfqpoint{1.803300in}{1.110092in}}%
\pgfpathlineto{\pgfqpoint{1.888208in}{0.969226in}}%
\pgfpathlineto{\pgfqpoint{1.973986in}{0.982268in}}%
\pgfpathlineto{\pgfqpoint{2.061275in}{0.897722in}}%
\pgfpathlineto{\pgfqpoint{2.146813in}{0.841341in}}%
\pgfpathlineto{\pgfqpoint{2.232356in}{1.019793in}}%
\pgfpathlineto{\pgfqpoint{2.325145in}{0.949551in}}%
\pgfpathlineto{\pgfqpoint{2.396422in}{0.932488in}}%
\pgfpathlineto{\pgfqpoint{2.487345in}{0.901021in}}%
\pgfpathlineto{\pgfqpoint{2.581905in}{0.825220in}}%
\pgfpathlineto{\pgfqpoint{2.659740in}{0.779935in}}%
\pgfpathlineto{\pgfqpoint{2.737398in}{0.817912in}}%
\pgfpathlineto{\pgfqpoint{2.829387in}{0.810678in}}%
\pgfpathlineto{\pgfqpoint{3.173405in}{0.851916in}}%
\pgfusepath{stroke}%
\end{pgfscope}%
\begin{pgfscope}%
\pgfsetbuttcap%
\pgfsetroundjoin%
\definecolor{currentfill}{rgb}{0.007843,0.619608,0.450980}%
\pgfsetfillcolor{currentfill}%
\pgfsetlinewidth{0.752812pt}%
\definecolor{currentstroke}{rgb}{1.000000,1.000000,1.000000}%
\pgfsetstrokecolor{currentstroke}%
\pgfsetdash{}{0pt}%
\pgfsys@defobject{currentmarker}{\pgfqpoint{-0.034722in}{-0.034722in}}{\pgfqpoint{0.034722in}{0.034722in}}{%
\pgfpathmoveto{\pgfqpoint{0.000000in}{-0.034722in}}%
\pgfpathcurveto{\pgfqpoint{0.009208in}{-0.034722in}}{\pgfqpoint{0.018041in}{-0.031064in}}{\pgfqpoint{0.024552in}{-0.024552in}}%
\pgfpathcurveto{\pgfqpoint{0.031064in}{-0.018041in}}{\pgfqpoint{0.034722in}{-0.009208in}}{\pgfqpoint{0.034722in}{0.000000in}}%
\pgfpathcurveto{\pgfqpoint{0.034722in}{0.009208in}}{\pgfqpoint{0.031064in}{0.018041in}}{\pgfqpoint{0.024552in}{0.024552in}}%
\pgfpathcurveto{\pgfqpoint{0.018041in}{0.031064in}}{\pgfqpoint{0.009208in}{0.034722in}}{\pgfqpoint{0.000000in}{0.034722in}}%
\pgfpathcurveto{\pgfqpoint{-0.009208in}{0.034722in}}{\pgfqpoint{-0.018041in}{0.031064in}}{\pgfqpoint{-0.024552in}{0.024552in}}%
\pgfpathcurveto{\pgfqpoint{-0.031064in}{0.018041in}}{\pgfqpoint{-0.034722in}{0.009208in}}{\pgfqpoint{-0.034722in}{0.000000in}}%
\pgfpathcurveto{\pgfqpoint{-0.034722in}{-0.009208in}}{\pgfqpoint{-0.031064in}{-0.018041in}}{\pgfqpoint{-0.024552in}{-0.024552in}}%
\pgfpathcurveto{\pgfqpoint{-0.018041in}{-0.031064in}}{\pgfqpoint{-0.009208in}{-0.034722in}}{\pgfqpoint{0.000000in}{-0.034722in}}%
\pgfpathlineto{\pgfqpoint{0.000000in}{-0.034722in}}%
\pgfpathclose%
\pgfusepath{stroke,fill}%
}%
\begin{pgfscope}%
\pgfsys@transformshift{0.773669in}{0.889528in}%
\pgfsys@useobject{currentmarker}{}%
\end{pgfscope}%
\begin{pgfscope}%
\pgfsys@transformshift{0.859949in}{0.739647in}%
\pgfsys@useobject{currentmarker}{}%
\end{pgfscope}%
\begin{pgfscope}%
\pgfsys@transformshift{0.945274in}{0.791545in}%
\pgfsys@useobject{currentmarker}{}%
\end{pgfscope}%
\begin{pgfscope}%
\pgfsys@transformshift{1.030985in}{0.822840in}%
\pgfsys@useobject{currentmarker}{}%
\end{pgfscope}%
\begin{pgfscope}%
\pgfsys@transformshift{1.116621in}{0.817002in}%
\pgfsys@useobject{currentmarker}{}%
\end{pgfscope}%
\begin{pgfscope}%
\pgfsys@transformshift{1.202341in}{0.833371in}%
\pgfsys@useobject{currentmarker}{}%
\end{pgfscope}%
\begin{pgfscope}%
\pgfsys@transformshift{1.289216in}{1.001852in}%
\pgfsys@useobject{currentmarker}{}%
\end{pgfscope}%
\begin{pgfscope}%
\pgfsys@transformshift{1.373880in}{0.956574in}%
\pgfsys@useobject{currentmarker}{}%
\end{pgfscope}%
\begin{pgfscope}%
\pgfsys@transformshift{1.459905in}{0.851276in}%
\pgfsys@useobject{currentmarker}{}%
\end{pgfscope}%
\begin{pgfscope}%
\pgfsys@transformshift{1.546396in}{0.984260in}%
\pgfsys@useobject{currentmarker}{}%
\end{pgfscope}%
\begin{pgfscope}%
\pgfsys@transformshift{1.631151in}{0.881118in}%
\pgfsys@useobject{currentmarker}{}%
\end{pgfscope}%
\begin{pgfscope}%
\pgfsys@transformshift{1.717283in}{0.909652in}%
\pgfsys@useobject{currentmarker}{}%
\end{pgfscope}%
\begin{pgfscope}%
\pgfsys@transformshift{1.803300in}{1.110092in}%
\pgfsys@useobject{currentmarker}{}%
\end{pgfscope}%
\begin{pgfscope}%
\pgfsys@transformshift{1.888208in}{0.969226in}%
\pgfsys@useobject{currentmarker}{}%
\end{pgfscope}%
\begin{pgfscope}%
\pgfsys@transformshift{1.973986in}{0.982268in}%
\pgfsys@useobject{currentmarker}{}%
\end{pgfscope}%
\begin{pgfscope}%
\pgfsys@transformshift{2.061275in}{0.897722in}%
\pgfsys@useobject{currentmarker}{}%
\end{pgfscope}%
\begin{pgfscope}%
\pgfsys@transformshift{2.146813in}{0.841341in}%
\pgfsys@useobject{currentmarker}{}%
\end{pgfscope}%
\begin{pgfscope}%
\pgfsys@transformshift{2.232356in}{1.019793in}%
\pgfsys@useobject{currentmarker}{}%
\end{pgfscope}%
\begin{pgfscope}%
\pgfsys@transformshift{2.325145in}{0.949551in}%
\pgfsys@useobject{currentmarker}{}%
\end{pgfscope}%
\begin{pgfscope}%
\pgfsys@transformshift{2.396422in}{0.932488in}%
\pgfsys@useobject{currentmarker}{}%
\end{pgfscope}%
\begin{pgfscope}%
\pgfsys@transformshift{2.487345in}{0.901021in}%
\pgfsys@useobject{currentmarker}{}%
\end{pgfscope}%
\begin{pgfscope}%
\pgfsys@transformshift{2.581905in}{0.825220in}%
\pgfsys@useobject{currentmarker}{}%
\end{pgfscope}%
\begin{pgfscope}%
\pgfsys@transformshift{2.659740in}{0.779935in}%
\pgfsys@useobject{currentmarker}{}%
\end{pgfscope}%
\begin{pgfscope}%
\pgfsys@transformshift{2.737398in}{0.817912in}%
\pgfsys@useobject{currentmarker}{}%
\end{pgfscope}%
\begin{pgfscope}%
\pgfsys@transformshift{2.829387in}{0.810678in}%
\pgfsys@useobject{currentmarker}{}%
\end{pgfscope}%
\begin{pgfscope}%
\pgfsys@transformshift{3.173405in}{0.851916in}%
\pgfsys@useobject{currentmarker}{}%
\end{pgfscope}%
\end{pgfscope}%
\begin{pgfscope}%
\pgfsetroundcap%
\pgfsetroundjoin%
\pgfsetlinewidth{1.003750pt}%
\definecolor{currentstroke}{rgb}{0.835294,0.368627,0.000000}%
\pgfsetstrokecolor{currentstroke}%
\pgfsetdash{}{0pt}%
\pgfpathmoveto{\pgfqpoint{0.773669in}{1.079028in}}%
\pgfpathlineto{\pgfqpoint{0.859949in}{0.889295in}}%
\pgfpathlineto{\pgfqpoint{0.945274in}{0.968824in}}%
\pgfpathlineto{\pgfqpoint{1.030985in}{1.016385in}}%
\pgfpathlineto{\pgfqpoint{1.116621in}{0.985539in}}%
\pgfpathlineto{\pgfqpoint{1.202341in}{0.998000in}}%
\pgfpathlineto{\pgfqpoint{1.289216in}{1.187737in}}%
\pgfpathlineto{\pgfqpoint{1.373880in}{1.176275in}}%
\pgfpathlineto{\pgfqpoint{1.459905in}{1.012306in}}%
\pgfpathlineto{\pgfqpoint{1.546396in}{1.248341in}}%
\pgfpathlineto{\pgfqpoint{1.631151in}{1.054038in}}%
\pgfpathlineto{\pgfqpoint{1.717283in}{1.117672in}}%
\pgfpathlineto{\pgfqpoint{1.803300in}{1.376526in}}%
\pgfpathlineto{\pgfqpoint{1.888208in}{1.159232in}}%
\pgfpathlineto{\pgfqpoint{1.973986in}{1.287958in}}%
\pgfpathlineto{\pgfqpoint{2.061275in}{1.151113in}}%
\pgfpathlineto{\pgfqpoint{2.146813in}{0.993309in}}%
\pgfpathlineto{\pgfqpoint{2.232356in}{1.329906in}}%
\pgfpathlineto{\pgfqpoint{2.325145in}{1.187072in}}%
\pgfpathlineto{\pgfqpoint{2.396422in}{1.158062in}}%
\pgfpathlineto{\pgfqpoint{2.487345in}{1.107290in}}%
\pgfpathlineto{\pgfqpoint{2.581905in}{1.008906in}}%
\pgfpathlineto{\pgfqpoint{2.659740in}{0.939243in}}%
\pgfpathlineto{\pgfqpoint{2.737398in}{0.995147in}}%
\pgfpathlineto{\pgfqpoint{2.829387in}{0.975821in}}%
\pgfpathlineto{\pgfqpoint{3.173405in}{1.048351in}}%
\pgfusepath{stroke}%
\end{pgfscope}%
\begin{pgfscope}%
\pgfsetbuttcap%
\pgfsetroundjoin%
\definecolor{currentfill}{rgb}{0.835294,0.368627,0.000000}%
\pgfsetfillcolor{currentfill}%
\pgfsetlinewidth{0.752812pt}%
\definecolor{currentstroke}{rgb}{1.000000,1.000000,1.000000}%
\pgfsetstrokecolor{currentstroke}%
\pgfsetdash{}{0pt}%
\pgfsys@defobject{currentmarker}{\pgfqpoint{-0.034722in}{-0.034722in}}{\pgfqpoint{0.034722in}{0.034722in}}{%
\pgfpathmoveto{\pgfqpoint{0.000000in}{-0.034722in}}%
\pgfpathcurveto{\pgfqpoint{0.009208in}{-0.034722in}}{\pgfqpoint{0.018041in}{-0.031064in}}{\pgfqpoint{0.024552in}{-0.024552in}}%
\pgfpathcurveto{\pgfqpoint{0.031064in}{-0.018041in}}{\pgfqpoint{0.034722in}{-0.009208in}}{\pgfqpoint{0.034722in}{0.000000in}}%
\pgfpathcurveto{\pgfqpoint{0.034722in}{0.009208in}}{\pgfqpoint{0.031064in}{0.018041in}}{\pgfqpoint{0.024552in}{0.024552in}}%
\pgfpathcurveto{\pgfqpoint{0.018041in}{0.031064in}}{\pgfqpoint{0.009208in}{0.034722in}}{\pgfqpoint{0.000000in}{0.034722in}}%
\pgfpathcurveto{\pgfqpoint{-0.009208in}{0.034722in}}{\pgfqpoint{-0.018041in}{0.031064in}}{\pgfqpoint{-0.024552in}{0.024552in}}%
\pgfpathcurveto{\pgfqpoint{-0.031064in}{0.018041in}}{\pgfqpoint{-0.034722in}{0.009208in}}{\pgfqpoint{-0.034722in}{0.000000in}}%
\pgfpathcurveto{\pgfqpoint{-0.034722in}{-0.009208in}}{\pgfqpoint{-0.031064in}{-0.018041in}}{\pgfqpoint{-0.024552in}{-0.024552in}}%
\pgfpathcurveto{\pgfqpoint{-0.018041in}{-0.031064in}}{\pgfqpoint{-0.009208in}{-0.034722in}}{\pgfqpoint{0.000000in}{-0.034722in}}%
\pgfpathlineto{\pgfqpoint{0.000000in}{-0.034722in}}%
\pgfpathclose%
\pgfusepath{stroke,fill}%
}%
\begin{pgfscope}%
\pgfsys@transformshift{0.773669in}{1.079028in}%
\pgfsys@useobject{currentmarker}{}%
\end{pgfscope}%
\begin{pgfscope}%
\pgfsys@transformshift{0.859949in}{0.889295in}%
\pgfsys@useobject{currentmarker}{}%
\end{pgfscope}%
\begin{pgfscope}%
\pgfsys@transformshift{0.945274in}{0.968824in}%
\pgfsys@useobject{currentmarker}{}%
\end{pgfscope}%
\begin{pgfscope}%
\pgfsys@transformshift{1.030985in}{1.016385in}%
\pgfsys@useobject{currentmarker}{}%
\end{pgfscope}%
\begin{pgfscope}%
\pgfsys@transformshift{1.116621in}{0.985539in}%
\pgfsys@useobject{currentmarker}{}%
\end{pgfscope}%
\begin{pgfscope}%
\pgfsys@transformshift{1.202341in}{0.998000in}%
\pgfsys@useobject{currentmarker}{}%
\end{pgfscope}%
\begin{pgfscope}%
\pgfsys@transformshift{1.289216in}{1.187737in}%
\pgfsys@useobject{currentmarker}{}%
\end{pgfscope}%
\begin{pgfscope}%
\pgfsys@transformshift{1.373880in}{1.176275in}%
\pgfsys@useobject{currentmarker}{}%
\end{pgfscope}%
\begin{pgfscope}%
\pgfsys@transformshift{1.459905in}{1.012306in}%
\pgfsys@useobject{currentmarker}{}%
\end{pgfscope}%
\begin{pgfscope}%
\pgfsys@transformshift{1.546396in}{1.248341in}%
\pgfsys@useobject{currentmarker}{}%
\end{pgfscope}%
\begin{pgfscope}%
\pgfsys@transformshift{1.631151in}{1.054038in}%
\pgfsys@useobject{currentmarker}{}%
\end{pgfscope}%
\begin{pgfscope}%
\pgfsys@transformshift{1.717283in}{1.117672in}%
\pgfsys@useobject{currentmarker}{}%
\end{pgfscope}%
\begin{pgfscope}%
\pgfsys@transformshift{1.803300in}{1.376526in}%
\pgfsys@useobject{currentmarker}{}%
\end{pgfscope}%
\begin{pgfscope}%
\pgfsys@transformshift{1.888208in}{1.159232in}%
\pgfsys@useobject{currentmarker}{}%
\end{pgfscope}%
\begin{pgfscope}%
\pgfsys@transformshift{1.973986in}{1.287958in}%
\pgfsys@useobject{currentmarker}{}%
\end{pgfscope}%
\begin{pgfscope}%
\pgfsys@transformshift{2.061275in}{1.151113in}%
\pgfsys@useobject{currentmarker}{}%
\end{pgfscope}%
\begin{pgfscope}%
\pgfsys@transformshift{2.146813in}{0.993309in}%
\pgfsys@useobject{currentmarker}{}%
\end{pgfscope}%
\begin{pgfscope}%
\pgfsys@transformshift{2.232356in}{1.329906in}%
\pgfsys@useobject{currentmarker}{}%
\end{pgfscope}%
\begin{pgfscope}%
\pgfsys@transformshift{2.325145in}{1.187072in}%
\pgfsys@useobject{currentmarker}{}%
\end{pgfscope}%
\begin{pgfscope}%
\pgfsys@transformshift{2.396422in}{1.158062in}%
\pgfsys@useobject{currentmarker}{}%
\end{pgfscope}%
\begin{pgfscope}%
\pgfsys@transformshift{2.487345in}{1.107290in}%
\pgfsys@useobject{currentmarker}{}%
\end{pgfscope}%
\begin{pgfscope}%
\pgfsys@transformshift{2.581905in}{1.008906in}%
\pgfsys@useobject{currentmarker}{}%
\end{pgfscope}%
\begin{pgfscope}%
\pgfsys@transformshift{2.659740in}{0.939243in}%
\pgfsys@useobject{currentmarker}{}%
\end{pgfscope}%
\begin{pgfscope}%
\pgfsys@transformshift{2.737398in}{0.995147in}%
\pgfsys@useobject{currentmarker}{}%
\end{pgfscope}%
\begin{pgfscope}%
\pgfsys@transformshift{2.829387in}{0.975821in}%
\pgfsys@useobject{currentmarker}{}%
\end{pgfscope}%
\begin{pgfscope}%
\pgfsys@transformshift{3.173405in}{1.048351in}%
\pgfsys@useobject{currentmarker}{}%
\end{pgfscope}%
\end{pgfscope}%
\begin{pgfscope}%
\pgfsetroundcap%
\pgfsetroundjoin%
\pgfsetlinewidth{1.003750pt}%
\definecolor{currentstroke}{rgb}{0.800000,0.470588,0.737255}%
\pgfsetstrokecolor{currentstroke}%
\pgfsetdash{}{0pt}%
\pgfpathmoveto{\pgfqpoint{0.773669in}{1.299161in}}%
\pgfpathlineto{\pgfqpoint{0.859949in}{1.099905in}}%
\pgfpathlineto{\pgfqpoint{0.945274in}{1.178101in}}%
\pgfpathlineto{\pgfqpoint{1.030985in}{1.228193in}}%
\pgfpathlineto{\pgfqpoint{1.116621in}{1.177491in}}%
\pgfpathlineto{\pgfqpoint{1.202341in}{1.161271in}}%
\pgfpathlineto{\pgfqpoint{1.289216in}{1.398795in}}%
\pgfpathlineto{\pgfqpoint{1.373880in}{1.448140in}}%
\pgfpathlineto{\pgfqpoint{1.459905in}{1.231918in}}%
\pgfpathlineto{\pgfqpoint{1.546396in}{1.575234in}}%
\pgfpathlineto{\pgfqpoint{1.631151in}{1.298401in}}%
\pgfpathlineto{\pgfqpoint{1.717283in}{1.378122in}}%
\pgfpathlineto{\pgfqpoint{1.803300in}{1.709150in}}%
\pgfpathlineto{\pgfqpoint{1.888208in}{1.419325in}}%
\pgfpathlineto{\pgfqpoint{1.973986in}{1.744791in}}%
\pgfpathlineto{\pgfqpoint{2.061275in}{1.435122in}}%
\pgfpathlineto{\pgfqpoint{2.146813in}{1.272072in}}%
\pgfpathlineto{\pgfqpoint{2.232356in}{1.728636in}}%
\pgfpathlineto{\pgfqpoint{2.325145in}{1.487832in}}%
\pgfpathlineto{\pgfqpoint{2.396422in}{1.551931in}}%
\pgfpathlineto{\pgfqpoint{2.487345in}{1.389638in}}%
\pgfpathlineto{\pgfqpoint{2.581905in}{1.413774in}}%
\pgfpathlineto{\pgfqpoint{2.659740in}{1.153565in}}%
\pgfpathlineto{\pgfqpoint{2.737398in}{1.253151in}}%
\pgfpathlineto{\pgfqpoint{2.829387in}{1.235487in}}%
\pgfpathlineto{\pgfqpoint{3.173405in}{1.484129in}}%
\pgfpathlineto{\pgfqpoint{3.354136in}{1.410313in}}%
\pgfpathlineto{\pgfqpoint{3.516938in}{1.461747in}}%
\pgfpathlineto{\pgfqpoint{3.686501in}{1.514170in}}%
\pgfpathlineto{\pgfqpoint{3.866734in}{1.715568in}}%
\pgfpathlineto{\pgfqpoint{4.038858in}{1.786189in}}%
\pgfusepath{stroke}%
\end{pgfscope}%
\begin{pgfscope}%
\pgfsetbuttcap%
\pgfsetroundjoin%
\definecolor{currentfill}{rgb}{0.800000,0.470588,0.737255}%
\pgfsetfillcolor{currentfill}%
\pgfsetlinewidth{0.752812pt}%
\definecolor{currentstroke}{rgb}{1.000000,1.000000,1.000000}%
\pgfsetstrokecolor{currentstroke}%
\pgfsetdash{}{0pt}%
\pgfsys@defobject{currentmarker}{\pgfqpoint{-0.034722in}{-0.034722in}}{\pgfqpoint{0.034722in}{0.034722in}}{%
\pgfpathmoveto{\pgfqpoint{0.000000in}{-0.034722in}}%
\pgfpathcurveto{\pgfqpoint{0.009208in}{-0.034722in}}{\pgfqpoint{0.018041in}{-0.031064in}}{\pgfqpoint{0.024552in}{-0.024552in}}%
\pgfpathcurveto{\pgfqpoint{0.031064in}{-0.018041in}}{\pgfqpoint{0.034722in}{-0.009208in}}{\pgfqpoint{0.034722in}{0.000000in}}%
\pgfpathcurveto{\pgfqpoint{0.034722in}{0.009208in}}{\pgfqpoint{0.031064in}{0.018041in}}{\pgfqpoint{0.024552in}{0.024552in}}%
\pgfpathcurveto{\pgfqpoint{0.018041in}{0.031064in}}{\pgfqpoint{0.009208in}{0.034722in}}{\pgfqpoint{0.000000in}{0.034722in}}%
\pgfpathcurveto{\pgfqpoint{-0.009208in}{0.034722in}}{\pgfqpoint{-0.018041in}{0.031064in}}{\pgfqpoint{-0.024552in}{0.024552in}}%
\pgfpathcurveto{\pgfqpoint{-0.031064in}{0.018041in}}{\pgfqpoint{-0.034722in}{0.009208in}}{\pgfqpoint{-0.034722in}{0.000000in}}%
\pgfpathcurveto{\pgfqpoint{-0.034722in}{-0.009208in}}{\pgfqpoint{-0.031064in}{-0.018041in}}{\pgfqpoint{-0.024552in}{-0.024552in}}%
\pgfpathcurveto{\pgfqpoint{-0.018041in}{-0.031064in}}{\pgfqpoint{-0.009208in}{-0.034722in}}{\pgfqpoint{0.000000in}{-0.034722in}}%
\pgfpathlineto{\pgfqpoint{0.000000in}{-0.034722in}}%
\pgfpathclose%
\pgfusepath{stroke,fill}%
}%
\begin{pgfscope}%
\pgfsys@transformshift{0.773669in}{1.299161in}%
\pgfsys@useobject{currentmarker}{}%
\end{pgfscope}%
\begin{pgfscope}%
\pgfsys@transformshift{0.859949in}{1.099905in}%
\pgfsys@useobject{currentmarker}{}%
\end{pgfscope}%
\begin{pgfscope}%
\pgfsys@transformshift{0.945274in}{1.178101in}%
\pgfsys@useobject{currentmarker}{}%
\end{pgfscope}%
\begin{pgfscope}%
\pgfsys@transformshift{1.030985in}{1.228193in}%
\pgfsys@useobject{currentmarker}{}%
\end{pgfscope}%
\begin{pgfscope}%
\pgfsys@transformshift{1.116621in}{1.177491in}%
\pgfsys@useobject{currentmarker}{}%
\end{pgfscope}%
\begin{pgfscope}%
\pgfsys@transformshift{1.202341in}{1.161271in}%
\pgfsys@useobject{currentmarker}{}%
\end{pgfscope}%
\begin{pgfscope}%
\pgfsys@transformshift{1.289216in}{1.398795in}%
\pgfsys@useobject{currentmarker}{}%
\end{pgfscope}%
\begin{pgfscope}%
\pgfsys@transformshift{1.373880in}{1.448140in}%
\pgfsys@useobject{currentmarker}{}%
\end{pgfscope}%
\begin{pgfscope}%
\pgfsys@transformshift{1.459905in}{1.231918in}%
\pgfsys@useobject{currentmarker}{}%
\end{pgfscope}%
\begin{pgfscope}%
\pgfsys@transformshift{1.546396in}{1.575234in}%
\pgfsys@useobject{currentmarker}{}%
\end{pgfscope}%
\begin{pgfscope}%
\pgfsys@transformshift{1.631151in}{1.298401in}%
\pgfsys@useobject{currentmarker}{}%
\end{pgfscope}%
\begin{pgfscope}%
\pgfsys@transformshift{1.717283in}{1.378122in}%
\pgfsys@useobject{currentmarker}{}%
\end{pgfscope}%
\begin{pgfscope}%
\pgfsys@transformshift{1.803300in}{1.709150in}%
\pgfsys@useobject{currentmarker}{}%
\end{pgfscope}%
\begin{pgfscope}%
\pgfsys@transformshift{1.888208in}{1.419325in}%
\pgfsys@useobject{currentmarker}{}%
\end{pgfscope}%
\begin{pgfscope}%
\pgfsys@transformshift{1.973986in}{1.744791in}%
\pgfsys@useobject{currentmarker}{}%
\end{pgfscope}%
\begin{pgfscope}%
\pgfsys@transformshift{2.061275in}{1.435122in}%
\pgfsys@useobject{currentmarker}{}%
\end{pgfscope}%
\begin{pgfscope}%
\pgfsys@transformshift{2.146813in}{1.272072in}%
\pgfsys@useobject{currentmarker}{}%
\end{pgfscope}%
\begin{pgfscope}%
\pgfsys@transformshift{2.232356in}{1.728636in}%
\pgfsys@useobject{currentmarker}{}%
\end{pgfscope}%
\begin{pgfscope}%
\pgfsys@transformshift{2.325145in}{1.487832in}%
\pgfsys@useobject{currentmarker}{}%
\end{pgfscope}%
\begin{pgfscope}%
\pgfsys@transformshift{2.396422in}{1.551931in}%
\pgfsys@useobject{currentmarker}{}%
\end{pgfscope}%
\begin{pgfscope}%
\pgfsys@transformshift{2.487345in}{1.389638in}%
\pgfsys@useobject{currentmarker}{}%
\end{pgfscope}%
\begin{pgfscope}%
\pgfsys@transformshift{2.581905in}{1.413774in}%
\pgfsys@useobject{currentmarker}{}%
\end{pgfscope}%
\begin{pgfscope}%
\pgfsys@transformshift{2.659740in}{1.153565in}%
\pgfsys@useobject{currentmarker}{}%
\end{pgfscope}%
\begin{pgfscope}%
\pgfsys@transformshift{2.737398in}{1.253151in}%
\pgfsys@useobject{currentmarker}{}%
\end{pgfscope}%
\begin{pgfscope}%
\pgfsys@transformshift{2.829387in}{1.235487in}%
\pgfsys@useobject{currentmarker}{}%
\end{pgfscope}%
\begin{pgfscope}%
\pgfsys@transformshift{3.173405in}{1.484129in}%
\pgfsys@useobject{currentmarker}{}%
\end{pgfscope}%
\begin{pgfscope}%
\pgfsys@transformshift{3.354136in}{1.410313in}%
\pgfsys@useobject{currentmarker}{}%
\end{pgfscope}%
\begin{pgfscope}%
\pgfsys@transformshift{3.516938in}{1.461747in}%
\pgfsys@useobject{currentmarker}{}%
\end{pgfscope}%
\begin{pgfscope}%
\pgfsys@transformshift{3.686501in}{1.514170in}%
\pgfsys@useobject{currentmarker}{}%
\end{pgfscope}%
\begin{pgfscope}%
\pgfsys@transformshift{3.866734in}{1.715568in}%
\pgfsys@useobject{currentmarker}{}%
\end{pgfscope}%
\begin{pgfscope}%
\pgfsys@transformshift{4.038858in}{1.786189in}%
\pgfsys@useobject{currentmarker}{}%
\end{pgfscope}%
\end{pgfscope}%
\begin{pgfscope}%
\pgfsetroundcap%
\pgfsetroundjoin%
\pgfsetlinewidth{1.003750pt}%
\definecolor{currentstroke}{rgb}{0.792157,0.568627,0.380392}%
\pgfsetstrokecolor{currentstroke}%
\pgfsetdash{}{0pt}%
\pgfpathmoveto{\pgfqpoint{0.773669in}{1.509365in}}%
\pgfpathlineto{\pgfqpoint{0.859949in}{1.276631in}}%
\pgfpathlineto{\pgfqpoint{0.945274in}{1.371401in}}%
\pgfpathlineto{\pgfqpoint{1.030985in}{1.427379in}}%
\pgfpathlineto{\pgfqpoint{1.116621in}{1.372302in}}%
\pgfpathlineto{\pgfqpoint{1.202341in}{1.341366in}}%
\pgfpathlineto{\pgfqpoint{1.289216in}{1.606382in}}%
\pgfpathlineto{\pgfqpoint{1.373880in}{1.645674in}}%
\pgfpathlineto{\pgfqpoint{1.459905in}{1.431689in}}%
\pgfpathlineto{\pgfqpoint{1.546396in}{1.803497in}}%
\pgfpathlineto{\pgfqpoint{1.631151in}{1.502609in}}%
\pgfpathlineto{\pgfqpoint{1.717283in}{1.597975in}}%
\pgfpathlineto{\pgfqpoint{1.803300in}{2.195449in}}%
\pgfpathlineto{\pgfqpoint{1.888208in}{1.711894in}}%
\pgfpathlineto{\pgfqpoint{1.973986in}{2.160584in}}%
\pgfpathlineto{\pgfqpoint{2.061275in}{1.680490in}}%
\pgfpathlineto{\pgfqpoint{2.146813in}{1.650703in}}%
\pgfpathlineto{\pgfqpoint{2.232356in}{2.231550in}}%
\pgfpathlineto{\pgfqpoint{2.325145in}{1.780259in}}%
\pgfpathlineto{\pgfqpoint{2.396422in}{1.987663in}}%
\pgfpathlineto{\pgfqpoint{2.487345in}{1.738791in}}%
\pgfpathlineto{\pgfqpoint{2.581905in}{1.802513in}}%
\pgfpathlineto{\pgfqpoint{2.659740in}{1.383636in}}%
\pgfpathlineto{\pgfqpoint{2.737398in}{1.726343in}}%
\pgfpathlineto{\pgfqpoint{2.829387in}{1.613232in}}%
\pgfpathlineto{\pgfqpoint{3.173405in}{1.909608in}}%
\pgfpathlineto{\pgfqpoint{3.354136in}{1.933247in}}%
\pgfpathlineto{\pgfqpoint{3.516938in}{1.889352in}}%
\pgfpathlineto{\pgfqpoint{3.686501in}{1.947231in}}%
\pgfpathlineto{\pgfqpoint{3.866734in}{2.225236in}}%
\pgfpathlineto{\pgfqpoint{4.038858in}{2.318552in}}%
\pgfpathlineto{\pgfqpoint{4.199306in}{2.024513in}}%
\pgfpathlineto{\pgfqpoint{4.371578in}{1.897781in}}%
\pgfpathlineto{\pgfqpoint{4.553979in}{1.920362in}}%
\pgfpathlineto{\pgfqpoint{4.889440in}{1.922724in}}%
\pgfusepath{stroke}%
\end{pgfscope}%
\begin{pgfscope}%
\pgfsetbuttcap%
\pgfsetroundjoin%
\definecolor{currentfill}{rgb}{0.792157,0.568627,0.380392}%
\pgfsetfillcolor{currentfill}%
\pgfsetlinewidth{0.752812pt}%
\definecolor{currentstroke}{rgb}{1.000000,1.000000,1.000000}%
\pgfsetstrokecolor{currentstroke}%
\pgfsetdash{}{0pt}%
\pgfsys@defobject{currentmarker}{\pgfqpoint{-0.034722in}{-0.034722in}}{\pgfqpoint{0.034722in}{0.034722in}}{%
\pgfpathmoveto{\pgfqpoint{0.000000in}{-0.034722in}}%
\pgfpathcurveto{\pgfqpoint{0.009208in}{-0.034722in}}{\pgfqpoint{0.018041in}{-0.031064in}}{\pgfqpoint{0.024552in}{-0.024552in}}%
\pgfpathcurveto{\pgfqpoint{0.031064in}{-0.018041in}}{\pgfqpoint{0.034722in}{-0.009208in}}{\pgfqpoint{0.034722in}{0.000000in}}%
\pgfpathcurveto{\pgfqpoint{0.034722in}{0.009208in}}{\pgfqpoint{0.031064in}{0.018041in}}{\pgfqpoint{0.024552in}{0.024552in}}%
\pgfpathcurveto{\pgfqpoint{0.018041in}{0.031064in}}{\pgfqpoint{0.009208in}{0.034722in}}{\pgfqpoint{0.000000in}{0.034722in}}%
\pgfpathcurveto{\pgfqpoint{-0.009208in}{0.034722in}}{\pgfqpoint{-0.018041in}{0.031064in}}{\pgfqpoint{-0.024552in}{0.024552in}}%
\pgfpathcurveto{\pgfqpoint{-0.031064in}{0.018041in}}{\pgfqpoint{-0.034722in}{0.009208in}}{\pgfqpoint{-0.034722in}{0.000000in}}%
\pgfpathcurveto{\pgfqpoint{-0.034722in}{-0.009208in}}{\pgfqpoint{-0.031064in}{-0.018041in}}{\pgfqpoint{-0.024552in}{-0.024552in}}%
\pgfpathcurveto{\pgfqpoint{-0.018041in}{-0.031064in}}{\pgfqpoint{-0.009208in}{-0.034722in}}{\pgfqpoint{0.000000in}{-0.034722in}}%
\pgfpathlineto{\pgfqpoint{0.000000in}{-0.034722in}}%
\pgfpathclose%
\pgfusepath{stroke,fill}%
}%
\begin{pgfscope}%
\pgfsys@transformshift{0.773669in}{1.509365in}%
\pgfsys@useobject{currentmarker}{}%
\end{pgfscope}%
\begin{pgfscope}%
\pgfsys@transformshift{0.859949in}{1.276631in}%
\pgfsys@useobject{currentmarker}{}%
\end{pgfscope}%
\begin{pgfscope}%
\pgfsys@transformshift{0.945274in}{1.371401in}%
\pgfsys@useobject{currentmarker}{}%
\end{pgfscope}%
\begin{pgfscope}%
\pgfsys@transformshift{1.030985in}{1.427379in}%
\pgfsys@useobject{currentmarker}{}%
\end{pgfscope}%
\begin{pgfscope}%
\pgfsys@transformshift{1.116621in}{1.372302in}%
\pgfsys@useobject{currentmarker}{}%
\end{pgfscope}%
\begin{pgfscope}%
\pgfsys@transformshift{1.202341in}{1.341366in}%
\pgfsys@useobject{currentmarker}{}%
\end{pgfscope}%
\begin{pgfscope}%
\pgfsys@transformshift{1.289216in}{1.606382in}%
\pgfsys@useobject{currentmarker}{}%
\end{pgfscope}%
\begin{pgfscope}%
\pgfsys@transformshift{1.373880in}{1.645674in}%
\pgfsys@useobject{currentmarker}{}%
\end{pgfscope}%
\begin{pgfscope}%
\pgfsys@transformshift{1.459905in}{1.431689in}%
\pgfsys@useobject{currentmarker}{}%
\end{pgfscope}%
\begin{pgfscope}%
\pgfsys@transformshift{1.546396in}{1.803497in}%
\pgfsys@useobject{currentmarker}{}%
\end{pgfscope}%
\begin{pgfscope}%
\pgfsys@transformshift{1.631151in}{1.502609in}%
\pgfsys@useobject{currentmarker}{}%
\end{pgfscope}%
\begin{pgfscope}%
\pgfsys@transformshift{1.717283in}{1.597975in}%
\pgfsys@useobject{currentmarker}{}%
\end{pgfscope}%
\begin{pgfscope}%
\pgfsys@transformshift{1.803300in}{2.195449in}%
\pgfsys@useobject{currentmarker}{}%
\end{pgfscope}%
\begin{pgfscope}%
\pgfsys@transformshift{1.888208in}{1.711894in}%
\pgfsys@useobject{currentmarker}{}%
\end{pgfscope}%
\begin{pgfscope}%
\pgfsys@transformshift{1.973986in}{2.160584in}%
\pgfsys@useobject{currentmarker}{}%
\end{pgfscope}%
\begin{pgfscope}%
\pgfsys@transformshift{2.061275in}{1.680490in}%
\pgfsys@useobject{currentmarker}{}%
\end{pgfscope}%
\begin{pgfscope}%
\pgfsys@transformshift{2.146813in}{1.650703in}%
\pgfsys@useobject{currentmarker}{}%
\end{pgfscope}%
\begin{pgfscope}%
\pgfsys@transformshift{2.232356in}{2.231550in}%
\pgfsys@useobject{currentmarker}{}%
\end{pgfscope}%
\begin{pgfscope}%
\pgfsys@transformshift{2.325145in}{1.780259in}%
\pgfsys@useobject{currentmarker}{}%
\end{pgfscope}%
\begin{pgfscope}%
\pgfsys@transformshift{2.396422in}{1.987663in}%
\pgfsys@useobject{currentmarker}{}%
\end{pgfscope}%
\begin{pgfscope}%
\pgfsys@transformshift{2.487345in}{1.738791in}%
\pgfsys@useobject{currentmarker}{}%
\end{pgfscope}%
\begin{pgfscope}%
\pgfsys@transformshift{2.581905in}{1.802513in}%
\pgfsys@useobject{currentmarker}{}%
\end{pgfscope}%
\begin{pgfscope}%
\pgfsys@transformshift{2.659740in}{1.383636in}%
\pgfsys@useobject{currentmarker}{}%
\end{pgfscope}%
\begin{pgfscope}%
\pgfsys@transformshift{2.737398in}{1.726343in}%
\pgfsys@useobject{currentmarker}{}%
\end{pgfscope}%
\begin{pgfscope}%
\pgfsys@transformshift{2.829387in}{1.613232in}%
\pgfsys@useobject{currentmarker}{}%
\end{pgfscope}%
\begin{pgfscope}%
\pgfsys@transformshift{3.173405in}{1.909608in}%
\pgfsys@useobject{currentmarker}{}%
\end{pgfscope}%
\begin{pgfscope}%
\pgfsys@transformshift{3.354136in}{1.933247in}%
\pgfsys@useobject{currentmarker}{}%
\end{pgfscope}%
\begin{pgfscope}%
\pgfsys@transformshift{3.516938in}{1.889352in}%
\pgfsys@useobject{currentmarker}{}%
\end{pgfscope}%
\begin{pgfscope}%
\pgfsys@transformshift{3.686501in}{1.947231in}%
\pgfsys@useobject{currentmarker}{}%
\end{pgfscope}%
\begin{pgfscope}%
\pgfsys@transformshift{3.866734in}{2.225236in}%
\pgfsys@useobject{currentmarker}{}%
\end{pgfscope}%
\begin{pgfscope}%
\pgfsys@transformshift{4.038858in}{2.318552in}%
\pgfsys@useobject{currentmarker}{}%
\end{pgfscope}%
\begin{pgfscope}%
\pgfsys@transformshift{4.199306in}{2.024513in}%
\pgfsys@useobject{currentmarker}{}%
\end{pgfscope}%
\begin{pgfscope}%
\pgfsys@transformshift{4.371578in}{1.897781in}%
\pgfsys@useobject{currentmarker}{}%
\end{pgfscope}%
\begin{pgfscope}%
\pgfsys@transformshift{4.553979in}{1.920362in}%
\pgfsys@useobject{currentmarker}{}%
\end{pgfscope}%
\begin{pgfscope}%
\pgfsys@transformshift{4.889440in}{1.922724in}%
\pgfsys@useobject{currentmarker}{}%
\end{pgfscope}%
\end{pgfscope}%
\end{pgfpicture}%
\makeatother%
\endgroup%

						\end{figcenter}
						\caption{Total routing times of all datasets.}
					\end{subfigure}
					\\[3ex]
					\begin{subfigure}[t]{\textwidth}
						\begin{figcenter}
							\begingroup%
\makeatletter%
\begin{pgfpicture}%
\pgfpathrectangle{\pgfpointorigin}{\pgfqpoint{6.080545in}{1.715788in}}%
\pgfusepath{use as bounding box}%
\begin{pgfscope}%
\pgfsetbuttcap%
\pgfsetmiterjoin%
\definecolor{currentfill}{rgb}{1.000000,1.000000,1.000000}%
\pgfsetfillcolor{currentfill}%
\pgfsetlinewidth{0.000000pt}%
\definecolor{currentstroke}{rgb}{1.000000,1.000000,1.000000}%
\pgfsetstrokecolor{currentstroke}%
\pgfsetdash{}{0pt}%
\pgfpathmoveto{\pgfqpoint{0.000000in}{0.000000in}}%
\pgfpathlineto{\pgfqpoint{6.080545in}{0.000000in}}%
\pgfpathlineto{\pgfqpoint{6.080545in}{1.715788in}}%
\pgfpathlineto{\pgfqpoint{0.000000in}{1.715788in}}%
\pgfpathlineto{\pgfqpoint{0.000000in}{0.000000in}}%
\pgfpathclose%
\pgfusepath{fill}%
\end{pgfscope}%
\begin{pgfscope}%
\pgfsetbuttcap%
\pgfsetmiterjoin%
\definecolor{currentfill}{rgb}{1.000000,1.000000,1.000000}%
\pgfsetfillcolor{currentfill}%
\pgfsetlinewidth{0.000000pt}%
\definecolor{currentstroke}{rgb}{0.000000,0.000000,0.000000}%
\pgfsetstrokecolor{currentstroke}%
\pgfsetstrokeopacity{0.000000}%
\pgfsetdash{}{0pt}%
\pgfpathmoveto{\pgfqpoint{0.532932in}{0.451389in}}%
\pgfpathlineto{\pgfqpoint{4.375365in}{0.451389in}}%
\pgfpathlineto{\pgfqpoint{4.375365in}{1.715788in}}%
\pgfpathlineto{\pgfqpoint{0.532932in}{1.715788in}}%
\pgfpathlineto{\pgfqpoint{0.532932in}{0.451389in}}%
\pgfpathclose%
\pgfusepath{fill}%
\end{pgfscope}%
\begin{pgfscope}%
\pgfpathrectangle{\pgfqpoint{0.532932in}{0.451389in}}{\pgfqpoint{3.842434in}{1.264399in}}%
\pgfusepath{clip}%
\pgfsetroundcap%
\pgfsetroundjoin%
\pgfsetlinewidth{1.003750pt}%
\definecolor{currentstroke}{rgb}{0.800000,0.800000,0.800000}%
\pgfsetstrokecolor{currentstroke}%
\pgfsetdash{}{0pt}%
\pgfpathmoveto{\pgfqpoint{0.532932in}{0.451389in}}%
\pgfpathlineto{\pgfqpoint{0.532932in}{1.715788in}}%
\pgfusepath{stroke}%
\end{pgfscope}%
\begin{pgfscope}%
\definecolor{textcolor}{rgb}{0.150000,0.150000,0.150000}%
\pgfsetstrokecolor{textcolor}%
\pgfsetfillcolor{textcolor}%
\pgftext[x=0.532932in,y=0.319444in,,top]{\color{textcolor}\sffamily\fontsize{9.000000}{10.800000}\selectfont 0.0}%
\end{pgfscope}%
\begin{pgfscope}%
\pgfpathrectangle{\pgfqpoint{0.532932in}{0.451389in}}{\pgfqpoint{3.842434in}{1.264399in}}%
\pgfusepath{clip}%
\pgfsetroundcap%
\pgfsetroundjoin%
\pgfsetlinewidth{1.003750pt}%
\definecolor{currentstroke}{rgb}{0.800000,0.800000,0.800000}%
\pgfsetstrokecolor{currentstroke}%
\pgfsetdash{}{0pt}%
\pgfpathmoveto{\pgfqpoint{1.263388in}{0.451389in}}%
\pgfpathlineto{\pgfqpoint{1.263388in}{1.715788in}}%
\pgfusepath{stroke}%
\end{pgfscope}%
\begin{pgfscope}%
\definecolor{textcolor}{rgb}{0.150000,0.150000,0.150000}%
\pgfsetstrokecolor{textcolor}%
\pgfsetfillcolor{textcolor}%
\pgftext[x=1.263388in,y=0.319444in,,top]{\color{textcolor}\sffamily\fontsize{9.000000}{10.800000}\selectfont 0.5}%
\end{pgfscope}%
\begin{pgfscope}%
\pgfpathrectangle{\pgfqpoint{0.532932in}{0.451389in}}{\pgfqpoint{3.842434in}{1.264399in}}%
\pgfusepath{clip}%
\pgfsetroundcap%
\pgfsetroundjoin%
\pgfsetlinewidth{1.003750pt}%
\definecolor{currentstroke}{rgb}{0.800000,0.800000,0.800000}%
\pgfsetstrokecolor{currentstroke}%
\pgfsetdash{}{0pt}%
\pgfpathmoveto{\pgfqpoint{1.993844in}{0.451389in}}%
\pgfpathlineto{\pgfqpoint{1.993844in}{1.715788in}}%
\pgfusepath{stroke}%
\end{pgfscope}%
\begin{pgfscope}%
\definecolor{textcolor}{rgb}{0.150000,0.150000,0.150000}%
\pgfsetstrokecolor{textcolor}%
\pgfsetfillcolor{textcolor}%
\pgftext[x=1.993844in,y=0.319444in,,top]{\color{textcolor}\sffamily\fontsize{9.000000}{10.800000}\selectfont 1.0}%
\end{pgfscope}%
\begin{pgfscope}%
\pgfpathrectangle{\pgfqpoint{0.532932in}{0.451389in}}{\pgfqpoint{3.842434in}{1.264399in}}%
\pgfusepath{clip}%
\pgfsetroundcap%
\pgfsetroundjoin%
\pgfsetlinewidth{1.003750pt}%
\definecolor{currentstroke}{rgb}{0.800000,0.800000,0.800000}%
\pgfsetstrokecolor{currentstroke}%
\pgfsetdash{}{0pt}%
\pgfpathmoveto{\pgfqpoint{2.724300in}{0.451389in}}%
\pgfpathlineto{\pgfqpoint{2.724300in}{1.715788in}}%
\pgfusepath{stroke}%
\end{pgfscope}%
\begin{pgfscope}%
\definecolor{textcolor}{rgb}{0.150000,0.150000,0.150000}%
\pgfsetstrokecolor{textcolor}%
\pgfsetfillcolor{textcolor}%
\pgftext[x=2.724300in,y=0.319444in,,top]{\color{textcolor}\sffamily\fontsize{9.000000}{10.800000}\selectfont 1.5}%
\end{pgfscope}%
\begin{pgfscope}%
\pgfpathrectangle{\pgfqpoint{0.532932in}{0.451389in}}{\pgfqpoint{3.842434in}{1.264399in}}%
\pgfusepath{clip}%
\pgfsetroundcap%
\pgfsetroundjoin%
\pgfsetlinewidth{1.003750pt}%
\definecolor{currentstroke}{rgb}{0.800000,0.800000,0.800000}%
\pgfsetstrokecolor{currentstroke}%
\pgfsetdash{}{0pt}%
\pgfpathmoveto{\pgfqpoint{3.454756in}{0.451389in}}%
\pgfpathlineto{\pgfqpoint{3.454756in}{1.715788in}}%
\pgfusepath{stroke}%
\end{pgfscope}%
\begin{pgfscope}%
\definecolor{textcolor}{rgb}{0.150000,0.150000,0.150000}%
\pgfsetstrokecolor{textcolor}%
\pgfsetfillcolor{textcolor}%
\pgftext[x=3.454756in,y=0.319444in,,top]{\color{textcolor}\sffamily\fontsize{9.000000}{10.800000}\selectfont 2.0}%
\end{pgfscope}%
\begin{pgfscope}%
\pgfpathrectangle{\pgfqpoint{0.532932in}{0.451389in}}{\pgfqpoint{3.842434in}{1.264399in}}%
\pgfusepath{clip}%
\pgfsetroundcap%
\pgfsetroundjoin%
\pgfsetlinewidth{1.003750pt}%
\definecolor{currentstroke}{rgb}{0.800000,0.800000,0.800000}%
\pgfsetstrokecolor{currentstroke}%
\pgfsetdash{}{0pt}%
\pgfpathmoveto{\pgfqpoint{4.185212in}{0.451389in}}%
\pgfpathlineto{\pgfqpoint{4.185212in}{1.715788in}}%
\pgfusepath{stroke}%
\end{pgfscope}%
\begin{pgfscope}%
\definecolor{textcolor}{rgb}{0.150000,0.150000,0.150000}%
\pgfsetstrokecolor{textcolor}%
\pgfsetfillcolor{textcolor}%
\pgftext[x=4.185212in,y=0.319444in,,top]{\color{textcolor}\sffamily\fontsize{9.000000}{10.800000}\selectfont 2.5}%
\end{pgfscope}%
\begin{pgfscope}%
\definecolor{textcolor}{rgb}{0.150000,0.150000,0.150000}%
\pgfsetstrokecolor{textcolor}%
\pgfsetfillcolor{textcolor}%
\pgftext[x=2.454149in,y=0.125000in,,top]{\color{textcolor}\sffamily\fontsize{9.000000}{10.800000}\selectfont Beeline distance in km}%
\end{pgfscope}%
\begin{pgfscope}%
\pgfpathrectangle{\pgfqpoint{0.532932in}{0.451389in}}{\pgfqpoint{3.842434in}{1.264399in}}%
\pgfusepath{clip}%
\pgfsetroundcap%
\pgfsetroundjoin%
\pgfsetlinewidth{1.003750pt}%
\definecolor{currentstroke}{rgb}{0.800000,0.800000,0.800000}%
\pgfsetstrokecolor{currentstroke}%
\pgfsetdash{}{0pt}%
\pgfpathmoveto{\pgfqpoint{0.532932in}{0.451389in}}%
\pgfpathlineto{\pgfqpoint{4.375365in}{0.451389in}}%
\pgfusepath{stroke}%
\end{pgfscope}%
\begin{pgfscope}%
\definecolor{textcolor}{rgb}{0.150000,0.150000,0.150000}%
\pgfsetstrokecolor{textcolor}%
\pgfsetfillcolor{textcolor}%
\pgftext[x=0.332140in, y=0.403903in, left, base]{\color{textcolor}\sffamily\fontsize{9.000000}{10.800000}\selectfont 0}%
\end{pgfscope}%
\begin{pgfscope}%
\pgfpathrectangle{\pgfqpoint{0.532932in}{0.451389in}}{\pgfqpoint{3.842434in}{1.264399in}}%
\pgfusepath{clip}%
\pgfsetroundcap%
\pgfsetroundjoin%
\pgfsetlinewidth{1.003750pt}%
\definecolor{currentstroke}{rgb}{0.800000,0.800000,0.800000}%
\pgfsetstrokecolor{currentstroke}%
\pgfsetdash{}{0pt}%
\pgfpathmoveto{\pgfqpoint{0.532932in}{0.915031in}}%
\pgfpathlineto{\pgfqpoint{4.375365in}{0.915031in}}%
\pgfusepath{stroke}%
\end{pgfscope}%
\begin{pgfscope}%
\definecolor{textcolor}{rgb}{0.150000,0.150000,0.150000}%
\pgfsetstrokecolor{textcolor}%
\pgfsetfillcolor{textcolor}%
\pgftext[x=0.194444in, y=0.867546in, left, base]{\color{textcolor}\sffamily\fontsize{9.000000}{10.800000}\selectfont 200}%
\end{pgfscope}%
\begin{pgfscope}%
\pgfpathrectangle{\pgfqpoint{0.532932in}{0.451389in}}{\pgfqpoint{3.842434in}{1.264399in}}%
\pgfusepath{clip}%
\pgfsetroundcap%
\pgfsetroundjoin%
\pgfsetlinewidth{1.003750pt}%
\definecolor{currentstroke}{rgb}{0.800000,0.800000,0.800000}%
\pgfsetstrokecolor{currentstroke}%
\pgfsetdash{}{0pt}%
\pgfpathmoveto{\pgfqpoint{0.532932in}{1.378673in}}%
\pgfpathlineto{\pgfqpoint{4.375365in}{1.378673in}}%
\pgfusepath{stroke}%
\end{pgfscope}%
\begin{pgfscope}%
\definecolor{textcolor}{rgb}{0.150000,0.150000,0.150000}%
\pgfsetstrokecolor{textcolor}%
\pgfsetfillcolor{textcolor}%
\pgftext[x=0.194444in, y=1.331188in, left, base]{\color{textcolor}\sffamily\fontsize{9.000000}{10.800000}\selectfont 400}%
\end{pgfscope}%
\begin{pgfscope}%
\definecolor{textcolor}{rgb}{0.150000,0.150000,0.150000}%
\pgfsetstrokecolor{textcolor}%
\pgfsetfillcolor{textcolor}%
\pgftext[x=0.125000in,y=1.083588in,,bottom,rotate=90.000000]{\color{textcolor}\sffamily\fontsize{9.000000}{10.800000}\selectfont Time in ms}%
\end{pgfscope}%
\begin{pgfscope}%
\pgfsetrectcap%
\pgfsetmiterjoin%
\pgfsetlinewidth{1.254687pt}%
\definecolor{currentstroke}{rgb}{0.800000,0.800000,0.800000}%
\pgfsetstrokecolor{currentstroke}%
\pgfsetdash{}{0pt}%
\pgfpathmoveto{\pgfqpoint{0.532932in}{0.451389in}}%
\pgfpathlineto{\pgfqpoint{0.532932in}{1.715788in}}%
\pgfusepath{stroke}%
\end{pgfscope}%
\begin{pgfscope}%
\pgfsetrectcap%
\pgfsetmiterjoin%
\pgfsetlinewidth{1.254687pt}%
\definecolor{currentstroke}{rgb}{0.800000,0.800000,0.800000}%
\pgfsetstrokecolor{currentstroke}%
\pgfsetdash{}{0pt}%
\pgfpathmoveto{\pgfqpoint{4.375365in}{0.451389in}}%
\pgfpathlineto{\pgfqpoint{4.375365in}{1.715788in}}%
\pgfusepath{stroke}%
\end{pgfscope}%
\begin{pgfscope}%
\pgfsetrectcap%
\pgfsetmiterjoin%
\pgfsetlinewidth{1.254687pt}%
\definecolor{currentstroke}{rgb}{0.800000,0.800000,0.800000}%
\pgfsetstrokecolor{currentstroke}%
\pgfsetdash{}{0pt}%
\pgfpathmoveto{\pgfqpoint{0.532932in}{0.451389in}}%
\pgfpathlineto{\pgfqpoint{4.375365in}{0.451389in}}%
\pgfusepath{stroke}%
\end{pgfscope}%
\begin{pgfscope}%
\pgfsetrectcap%
\pgfsetmiterjoin%
\pgfsetlinewidth{1.254687pt}%
\definecolor{currentstroke}{rgb}{0.800000,0.800000,0.800000}%
\pgfsetstrokecolor{currentstroke}%
\pgfsetdash{}{0pt}%
\pgfpathmoveto{\pgfqpoint{0.532932in}{1.715788in}}%
\pgfpathlineto{\pgfqpoint{4.375365in}{1.715788in}}%
\pgfusepath{stroke}%
\end{pgfscope}%
\begin{pgfscope}%
\pgfsetbuttcap%
\pgfsetmiterjoin%
\definecolor{currentfill}{rgb}{1.000000,1.000000,1.000000}%
\pgfsetfillcolor{currentfill}%
\pgfsetfillopacity{0.800000}%
\pgfsetlinewidth{1.003750pt}%
\definecolor{currentstroke}{rgb}{0.800000,0.800000,0.800000}%
\pgfsetstrokecolor{currentstroke}%
\pgfsetstrokeopacity{0.800000}%
\pgfsetdash{}{0pt}%
\pgfpathmoveto{\pgfqpoint{4.558926in}{0.524092in}}%
\pgfpathlineto{\pgfqpoint{6.055545in}{0.524092in}}%
\pgfpathquadraticcurveto{\pgfqpoint{6.080545in}{0.524092in}}{\pgfqpoint{6.080545in}{0.549092in}}%
\pgfpathlineto{\pgfqpoint{6.080545in}{1.618085in}}%
\pgfpathquadraticcurveto{\pgfqpoint{6.080545in}{1.643085in}}{\pgfqpoint{6.055545in}{1.643085in}}%
\pgfpathlineto{\pgfqpoint{4.558926in}{1.643085in}}%
\pgfpathquadraticcurveto{\pgfqpoint{4.533926in}{1.643085in}}{\pgfqpoint{4.533926in}{1.618085in}}%
\pgfpathlineto{\pgfqpoint{4.533926in}{0.549092in}}%
\pgfpathquadraticcurveto{\pgfqpoint{4.533926in}{0.524092in}}{\pgfqpoint{4.558926in}{0.524092in}}%
\pgfpathlineto{\pgfqpoint{4.558926in}{0.524092in}}%
\pgfpathclose%
\pgfusepath{stroke,fill}%
\end{pgfscope}%
\begin{pgfscope}%
\definecolor{textcolor}{rgb}{0.150000,0.150000,0.150000}%
\pgfsetstrokecolor{textcolor}%
\pgfsetfillcolor{textcolor}%
\pgftext[x=5.103836in,y=1.498114in,left,base]{\color{textcolor}\sffamily\fontsize{9.000000}{10.800000}\selectfont Legend}%
\end{pgfscope}%
\begin{pgfscope}%
\pgfsetroundcap%
\pgfsetroundjoin%
\pgfsetlinewidth{1.505625pt}%
\definecolor{currentstroke}{rgb}{0.003922,0.450980,0.698039}%
\pgfsetstrokecolor{currentstroke}%
\pgfsetdash{}{0pt}%
\pgfpathmoveto{\pgfqpoint{4.583926in}{1.354364in}}%
\pgfpathlineto{\pgfqpoint{4.708926in}{1.354364in}}%
\pgfpathlineto{\pgfqpoint{4.833926in}{1.354364in}}%
\pgfusepath{stroke}%
\end{pgfscope}%
\begin{pgfscope}%
\definecolor{textcolor}{rgb}{0.150000,0.150000,0.150000}%
\pgfsetstrokecolor{textcolor}%
\pgfsetfillcolor{textcolor}%
\pgftext[x=4.933926in,y=1.310614in,left,base]{\color{textcolor}\sffamily\fontsize{9.000000}{10.800000}\selectfont Total time}%
\end{pgfscope}%
\begin{pgfscope}%
\pgfsetroundcap%
\pgfsetroundjoin%
\pgfsetlinewidth{1.505625pt}%
\definecolor{currentstroke}{rgb}{0.870588,0.560784,0.019608}%
\pgfsetstrokecolor{currentstroke}%
\pgfsetdash{}{0pt}%
\pgfpathmoveto{\pgfqpoint{4.583926in}{1.166864in}}%
\pgfpathlineto{\pgfqpoint{4.708926in}{1.166864in}}%
\pgfpathlineto{\pgfqpoint{4.833926in}{1.166864in}}%
\pgfusepath{stroke}%
\end{pgfscope}%
\begin{pgfscope}%
\definecolor{textcolor}{rgb}{0.150000,0.150000,0.150000}%
\pgfsetstrokecolor{textcolor}%
\pgfsetfillcolor{textcolor}%
\pgftext[x=4.933926in,y=1.123114in,left,base]{\color{textcolor}\sffamily\fontsize{9.000000}{10.800000}\selectfont A* routing}%
\end{pgfscope}%
\begin{pgfscope}%
\pgfsetroundcap%
\pgfsetroundjoin%
\pgfsetlinewidth{1.505625pt}%
\definecolor{currentstroke}{rgb}{0.007843,0.619608,0.450980}%
\pgfsetstrokecolor{currentstroke}%
\pgfsetdash{}{0pt}%
\pgfpathmoveto{\pgfqpoint{4.583926in}{0.892353in}}%
\pgfpathlineto{\pgfqpoint{4.708926in}{0.892353in}}%
\pgfpathlineto{\pgfqpoint{4.833926in}{0.892353in}}%
\pgfusepath{stroke}%
\end{pgfscope}%
\begin{pgfscope}%
\definecolor{textcolor}{rgb}{0.150000,0.150000,0.150000}%
\pgfsetstrokecolor{textcolor}%
\pgfsetfillcolor{textcolor}%
\pgftext[x=4.933926in, y=0.935615in, left, base]{\color{textcolor}\sffamily\fontsize{9.000000}{10.800000}\selectfont Connect source \&}%
\end{pgfscope}%
\begin{pgfscope}%
\definecolor{textcolor}{rgb}{0.150000,0.150000,0.150000}%
\pgfsetstrokecolor{textcolor}%
\pgfsetfillcolor{textcolor}%
\pgftext[x=4.933926in, y=0.791621in, left, base]{\color{textcolor}\sffamily\fontsize{9.000000}{10.800000}\selectfont destination vertices}%
\end{pgfscope}%
\begin{pgfscope}%
\pgfsetroundcap%
\pgfsetroundjoin%
\pgfsetlinewidth{1.505625pt}%
\definecolor{currentstroke}{rgb}{0.835294,0.368627,0.000000}%
\pgfsetstrokecolor{currentstroke}%
\pgfsetdash{}{0pt}%
\pgfpathmoveto{\pgfqpoint{4.583926in}{0.647871in}}%
\pgfpathlineto{\pgfqpoint{4.708926in}{0.647871in}}%
\pgfpathlineto{\pgfqpoint{4.833926in}{0.647871in}}%
\pgfusepath{stroke}%
\end{pgfscope}%
\begin{pgfscope}%
\definecolor{textcolor}{rgb}{0.150000,0.150000,0.150000}%
\pgfsetstrokecolor{textcolor}%
\pgfsetfillcolor{textcolor}%
\pgftext[x=4.933926in,y=0.604121in,left,base]{\color{textcolor}\sffamily\fontsize{9.000000}{10.800000}\selectfont Restoring graph}%
\end{pgfscope}%
\begin{pgfscope}%
\pgfsetroundcap%
\pgfsetroundjoin%
\pgfsetlinewidth{1.003750pt}%
\definecolor{currentstroke}{rgb}{0.003922,0.450980,0.698039}%
\pgfsetstrokecolor{currentstroke}%
\pgfsetdash{}{0pt}%
\pgfpathmoveto{\pgfqpoint{0.679638in}{1.655693in}}%
\pgfpathlineto{\pgfqpoint{0.753275in}{1.295097in}}%
\pgfpathlineto{\pgfqpoint{0.826099in}{1.225648in}}%
\pgfpathlineto{\pgfqpoint{0.899253in}{1.119894in}}%
\pgfpathlineto{\pgfqpoint{0.972342in}{0.746251in}}%
\pgfpathlineto{\pgfqpoint{1.045503in}{0.845628in}}%
\pgfpathlineto{\pgfqpoint{1.119649in}{1.212745in}}%
\pgfpathlineto{\pgfqpoint{1.191908in}{1.140176in}}%
\pgfpathlineto{\pgfqpoint{1.265331in}{0.974903in}}%
\pgfpathlineto{\pgfqpoint{1.339149in}{1.122439in}}%
\pgfpathlineto{\pgfqpoint{1.411486in}{1.397249in}}%
\pgfpathlineto{\pgfqpoint{1.484998in}{1.180362in}}%
\pgfpathlineto{\pgfqpoint{1.558413in}{0.851952in}}%
\pgfpathlineto{\pgfqpoint{1.630880in}{1.151138in}}%
\pgfpathlineto{\pgfqpoint{1.704091in}{1.185446in}}%
\pgfpathlineto{\pgfqpoint{1.778592in}{0.761109in}}%
\pgfpathlineto{\pgfqpoint{1.851598in}{0.989552in}}%
\pgfpathlineto{\pgfqpoint{1.924607in}{0.935619in}}%
\pgfpathlineto{\pgfqpoint{2.003801in}{0.733362in}}%
\pgfpathlineto{\pgfqpoint{2.064635in}{0.797020in}}%
\pgfpathlineto{\pgfqpoint{2.142236in}{0.717478in}}%
\pgfpathlineto{\pgfqpoint{2.222941in}{0.824238in}}%
\pgfpathlineto{\pgfqpoint{2.289373in}{0.791276in}}%
\pgfpathlineto{\pgfqpoint{2.355652in}{1.272981in}}%
\pgfpathlineto{\pgfqpoint{2.434164in}{1.136508in}}%
\pgfpathlineto{\pgfqpoint{2.727780in}{0.927809in}}%
\pgfpathlineto{\pgfqpoint{2.882032in}{0.886770in}}%
\pgfpathlineto{\pgfqpoint{3.020982in}{0.748639in}}%
\pgfpathlineto{\pgfqpoint{3.165700in}{0.892978in}}%
\pgfpathlineto{\pgfqpoint{3.319528in}{0.860442in}}%
\pgfpathlineto{\pgfqpoint{3.466433in}{0.768198in}}%
\pgfpathlineto{\pgfqpoint{3.603372in}{0.768529in}}%
\pgfpathlineto{\pgfqpoint{3.750405in}{0.769025in}}%
\pgfpathlineto{\pgfqpoint{3.906080in}{0.838333in}}%
\pgfpathlineto{\pgfqpoint{4.192392in}{0.877923in}}%
\pgfusepath{stroke}%
\end{pgfscope}%
\begin{pgfscope}%
\pgfsetbuttcap%
\pgfsetroundjoin%
\definecolor{currentfill}{rgb}{0.003922,0.450980,0.698039}%
\pgfsetfillcolor{currentfill}%
\pgfsetlinewidth{0.752812pt}%
\definecolor{currentstroke}{rgb}{1.000000,1.000000,1.000000}%
\pgfsetstrokecolor{currentstroke}%
\pgfsetdash{}{0pt}%
\pgfsys@defobject{currentmarker}{\pgfqpoint{-0.034722in}{-0.034722in}}{\pgfqpoint{0.034722in}{0.034722in}}{%
\pgfpathmoveto{\pgfqpoint{0.000000in}{-0.034722in}}%
\pgfpathcurveto{\pgfqpoint{0.009208in}{-0.034722in}}{\pgfqpoint{0.018041in}{-0.031064in}}{\pgfqpoint{0.024552in}{-0.024552in}}%
\pgfpathcurveto{\pgfqpoint{0.031064in}{-0.018041in}}{\pgfqpoint{0.034722in}{-0.009208in}}{\pgfqpoint{0.034722in}{0.000000in}}%
\pgfpathcurveto{\pgfqpoint{0.034722in}{0.009208in}}{\pgfqpoint{0.031064in}{0.018041in}}{\pgfqpoint{0.024552in}{0.024552in}}%
\pgfpathcurveto{\pgfqpoint{0.018041in}{0.031064in}}{\pgfqpoint{0.009208in}{0.034722in}}{\pgfqpoint{0.000000in}{0.034722in}}%
\pgfpathcurveto{\pgfqpoint{-0.009208in}{0.034722in}}{\pgfqpoint{-0.018041in}{0.031064in}}{\pgfqpoint{-0.024552in}{0.024552in}}%
\pgfpathcurveto{\pgfqpoint{-0.031064in}{0.018041in}}{\pgfqpoint{-0.034722in}{0.009208in}}{\pgfqpoint{-0.034722in}{0.000000in}}%
\pgfpathcurveto{\pgfqpoint{-0.034722in}{-0.009208in}}{\pgfqpoint{-0.031064in}{-0.018041in}}{\pgfqpoint{-0.024552in}{-0.024552in}}%
\pgfpathcurveto{\pgfqpoint{-0.018041in}{-0.031064in}}{\pgfqpoint{-0.009208in}{-0.034722in}}{\pgfqpoint{0.000000in}{-0.034722in}}%
\pgfpathlineto{\pgfqpoint{0.000000in}{-0.034722in}}%
\pgfpathclose%
\pgfusepath{stroke,fill}%
}%
\begin{pgfscope}%
\pgfsys@transformshift{0.679638in}{1.655693in}%
\pgfsys@useobject{currentmarker}{}%
\end{pgfscope}%
\begin{pgfscope}%
\pgfsys@transformshift{0.753275in}{1.295097in}%
\pgfsys@useobject{currentmarker}{}%
\end{pgfscope}%
\begin{pgfscope}%
\pgfsys@transformshift{0.826099in}{1.225648in}%
\pgfsys@useobject{currentmarker}{}%
\end{pgfscope}%
\begin{pgfscope}%
\pgfsys@transformshift{0.899253in}{1.119894in}%
\pgfsys@useobject{currentmarker}{}%
\end{pgfscope}%
\begin{pgfscope}%
\pgfsys@transformshift{0.972342in}{0.746251in}%
\pgfsys@useobject{currentmarker}{}%
\end{pgfscope}%
\begin{pgfscope}%
\pgfsys@transformshift{1.045503in}{0.845628in}%
\pgfsys@useobject{currentmarker}{}%
\end{pgfscope}%
\begin{pgfscope}%
\pgfsys@transformshift{1.119649in}{1.212745in}%
\pgfsys@useobject{currentmarker}{}%
\end{pgfscope}%
\begin{pgfscope}%
\pgfsys@transformshift{1.191908in}{1.140176in}%
\pgfsys@useobject{currentmarker}{}%
\end{pgfscope}%
\begin{pgfscope}%
\pgfsys@transformshift{1.265331in}{0.974903in}%
\pgfsys@useobject{currentmarker}{}%
\end{pgfscope}%
\begin{pgfscope}%
\pgfsys@transformshift{1.339149in}{1.122439in}%
\pgfsys@useobject{currentmarker}{}%
\end{pgfscope}%
\begin{pgfscope}%
\pgfsys@transformshift{1.411486in}{1.397249in}%
\pgfsys@useobject{currentmarker}{}%
\end{pgfscope}%
\begin{pgfscope}%
\pgfsys@transformshift{1.484998in}{1.180362in}%
\pgfsys@useobject{currentmarker}{}%
\end{pgfscope}%
\begin{pgfscope}%
\pgfsys@transformshift{1.558413in}{0.851952in}%
\pgfsys@useobject{currentmarker}{}%
\end{pgfscope}%
\begin{pgfscope}%
\pgfsys@transformshift{1.630880in}{1.151138in}%
\pgfsys@useobject{currentmarker}{}%
\end{pgfscope}%
\begin{pgfscope}%
\pgfsys@transformshift{1.704091in}{1.185446in}%
\pgfsys@useobject{currentmarker}{}%
\end{pgfscope}%
\begin{pgfscope}%
\pgfsys@transformshift{1.778592in}{0.761109in}%
\pgfsys@useobject{currentmarker}{}%
\end{pgfscope}%
\begin{pgfscope}%
\pgfsys@transformshift{1.851598in}{0.989552in}%
\pgfsys@useobject{currentmarker}{}%
\end{pgfscope}%
\begin{pgfscope}%
\pgfsys@transformshift{1.924607in}{0.935619in}%
\pgfsys@useobject{currentmarker}{}%
\end{pgfscope}%
\begin{pgfscope}%
\pgfsys@transformshift{2.003801in}{0.733362in}%
\pgfsys@useobject{currentmarker}{}%
\end{pgfscope}%
\begin{pgfscope}%
\pgfsys@transformshift{2.064635in}{0.797020in}%
\pgfsys@useobject{currentmarker}{}%
\end{pgfscope}%
\begin{pgfscope}%
\pgfsys@transformshift{2.142236in}{0.717478in}%
\pgfsys@useobject{currentmarker}{}%
\end{pgfscope}%
\begin{pgfscope}%
\pgfsys@transformshift{2.222941in}{0.824238in}%
\pgfsys@useobject{currentmarker}{}%
\end{pgfscope}%
\begin{pgfscope}%
\pgfsys@transformshift{2.289373in}{0.791276in}%
\pgfsys@useobject{currentmarker}{}%
\end{pgfscope}%
\begin{pgfscope}%
\pgfsys@transformshift{2.355652in}{1.272981in}%
\pgfsys@useobject{currentmarker}{}%
\end{pgfscope}%
\begin{pgfscope}%
\pgfsys@transformshift{2.434164in}{1.136508in}%
\pgfsys@useobject{currentmarker}{}%
\end{pgfscope}%
\begin{pgfscope}%
\pgfsys@transformshift{2.727780in}{0.927809in}%
\pgfsys@useobject{currentmarker}{}%
\end{pgfscope}%
\begin{pgfscope}%
\pgfsys@transformshift{2.882032in}{0.886770in}%
\pgfsys@useobject{currentmarker}{}%
\end{pgfscope}%
\begin{pgfscope}%
\pgfsys@transformshift{3.020982in}{0.748639in}%
\pgfsys@useobject{currentmarker}{}%
\end{pgfscope}%
\begin{pgfscope}%
\pgfsys@transformshift{3.165700in}{0.892978in}%
\pgfsys@useobject{currentmarker}{}%
\end{pgfscope}%
\begin{pgfscope}%
\pgfsys@transformshift{3.319528in}{0.860442in}%
\pgfsys@useobject{currentmarker}{}%
\end{pgfscope}%
\begin{pgfscope}%
\pgfsys@transformshift{3.466433in}{0.768198in}%
\pgfsys@useobject{currentmarker}{}%
\end{pgfscope}%
\begin{pgfscope}%
\pgfsys@transformshift{3.603372in}{0.768529in}%
\pgfsys@useobject{currentmarker}{}%
\end{pgfscope}%
\begin{pgfscope}%
\pgfsys@transformshift{3.750405in}{0.769025in}%
\pgfsys@useobject{currentmarker}{}%
\end{pgfscope}%
\begin{pgfscope}%
\pgfsys@transformshift{3.906080in}{0.838333in}%
\pgfsys@useobject{currentmarker}{}%
\end{pgfscope}%
\begin{pgfscope}%
\pgfsys@transformshift{4.192392in}{0.877923in}%
\pgfsys@useobject{currentmarker}{}%
\end{pgfscope}%
\end{pgfscope}%
\begin{pgfscope}%
\pgfsetroundcap%
\pgfsetroundjoin%
\pgfsetlinewidth{1.003750pt}%
\definecolor{currentstroke}{rgb}{0.870588,0.560784,0.019608}%
\pgfsetstrokecolor{currentstroke}%
\pgfsetdash{}{0pt}%
\pgfpathmoveto{\pgfqpoint{0.679638in}{0.453790in}}%
\pgfpathlineto{\pgfqpoint{0.753275in}{0.463932in}}%
\pgfpathlineto{\pgfqpoint{0.826099in}{0.463017in}}%
\pgfpathlineto{\pgfqpoint{0.899253in}{0.457928in}}%
\pgfpathlineto{\pgfqpoint{0.972342in}{0.459688in}}%
\pgfpathlineto{\pgfqpoint{1.045503in}{0.465156in}}%
\pgfpathlineto{\pgfqpoint{1.119649in}{0.476356in}}%
\pgfpathlineto{\pgfqpoint{1.191908in}{0.459878in}}%
\pgfpathlineto{\pgfqpoint{1.265331in}{0.480357in}}%
\pgfpathlineto{\pgfqpoint{1.339149in}{0.474638in}}%
\pgfpathlineto{\pgfqpoint{1.411486in}{0.494134in}}%
\pgfpathlineto{\pgfqpoint{1.484998in}{0.485851in}}%
\pgfpathlineto{\pgfqpoint{1.558413in}{0.579525in}}%
\pgfpathlineto{\pgfqpoint{1.630880in}{0.580970in}}%
\pgfpathlineto{\pgfqpoint{1.704091in}{0.488100in}}%
\pgfpathlineto{\pgfqpoint{1.778592in}{0.484300in}}%
\pgfpathlineto{\pgfqpoint{1.851598in}{0.588135in}}%
\pgfpathlineto{\pgfqpoint{1.924607in}{0.497813in}}%
\pgfpathlineto{\pgfqpoint{2.003801in}{0.497196in}}%
\pgfpathlineto{\pgfqpoint{2.064635in}{0.599958in}}%
\pgfpathlineto{\pgfqpoint{2.142236in}{0.544639in}}%
\pgfpathlineto{\pgfqpoint{2.222941in}{0.618773in}}%
\pgfpathlineto{\pgfqpoint{2.289373in}{0.512944in}}%
\pgfpathlineto{\pgfqpoint{2.355652in}{0.602344in}}%
\pgfpathlineto{\pgfqpoint{2.434164in}{0.508579in}}%
\pgfpathlineto{\pgfqpoint{2.727780in}{0.596043in}}%
\pgfpathlineto{\pgfqpoint{2.882032in}{0.560588in}}%
\pgfpathlineto{\pgfqpoint{3.020982in}{0.568887in}}%
\pgfpathlineto{\pgfqpoint{3.165700in}{0.598011in}}%
\pgfpathlineto{\pgfqpoint{3.319528in}{0.594385in}}%
\pgfpathlineto{\pgfqpoint{3.466433in}{0.611477in}}%
\pgfpathlineto{\pgfqpoint{3.603372in}{0.607119in}}%
\pgfpathlineto{\pgfqpoint{3.750405in}{0.600902in}}%
\pgfpathlineto{\pgfqpoint{3.906080in}{0.608422in}}%
\pgfpathlineto{\pgfqpoint{4.192392in}{0.607722in}}%
\pgfusepath{stroke}%
\end{pgfscope}%
\begin{pgfscope}%
\pgfsetbuttcap%
\pgfsetroundjoin%
\definecolor{currentfill}{rgb}{0.870588,0.560784,0.019608}%
\pgfsetfillcolor{currentfill}%
\pgfsetlinewidth{0.752812pt}%
\definecolor{currentstroke}{rgb}{1.000000,1.000000,1.000000}%
\pgfsetstrokecolor{currentstroke}%
\pgfsetdash{}{0pt}%
\pgfsys@defobject{currentmarker}{\pgfqpoint{-0.034722in}{-0.034722in}}{\pgfqpoint{0.034722in}{0.034722in}}{%
\pgfpathmoveto{\pgfqpoint{0.000000in}{-0.034722in}}%
\pgfpathcurveto{\pgfqpoint{0.009208in}{-0.034722in}}{\pgfqpoint{0.018041in}{-0.031064in}}{\pgfqpoint{0.024552in}{-0.024552in}}%
\pgfpathcurveto{\pgfqpoint{0.031064in}{-0.018041in}}{\pgfqpoint{0.034722in}{-0.009208in}}{\pgfqpoint{0.034722in}{0.000000in}}%
\pgfpathcurveto{\pgfqpoint{0.034722in}{0.009208in}}{\pgfqpoint{0.031064in}{0.018041in}}{\pgfqpoint{0.024552in}{0.024552in}}%
\pgfpathcurveto{\pgfqpoint{0.018041in}{0.031064in}}{\pgfqpoint{0.009208in}{0.034722in}}{\pgfqpoint{0.000000in}{0.034722in}}%
\pgfpathcurveto{\pgfqpoint{-0.009208in}{0.034722in}}{\pgfqpoint{-0.018041in}{0.031064in}}{\pgfqpoint{-0.024552in}{0.024552in}}%
\pgfpathcurveto{\pgfqpoint{-0.031064in}{0.018041in}}{\pgfqpoint{-0.034722in}{0.009208in}}{\pgfqpoint{-0.034722in}{0.000000in}}%
\pgfpathcurveto{\pgfqpoint{-0.034722in}{-0.009208in}}{\pgfqpoint{-0.031064in}{-0.018041in}}{\pgfqpoint{-0.024552in}{-0.024552in}}%
\pgfpathcurveto{\pgfqpoint{-0.018041in}{-0.031064in}}{\pgfqpoint{-0.009208in}{-0.034722in}}{\pgfqpoint{0.000000in}{-0.034722in}}%
\pgfpathlineto{\pgfqpoint{0.000000in}{-0.034722in}}%
\pgfpathclose%
\pgfusepath{stroke,fill}%
}%
\begin{pgfscope}%
\pgfsys@transformshift{0.679638in}{0.453790in}%
\pgfsys@useobject{currentmarker}{}%
\end{pgfscope}%
\begin{pgfscope}%
\pgfsys@transformshift{0.753275in}{0.463932in}%
\pgfsys@useobject{currentmarker}{}%
\end{pgfscope}%
\begin{pgfscope}%
\pgfsys@transformshift{0.826099in}{0.463017in}%
\pgfsys@useobject{currentmarker}{}%
\end{pgfscope}%
\begin{pgfscope}%
\pgfsys@transformshift{0.899253in}{0.457928in}%
\pgfsys@useobject{currentmarker}{}%
\end{pgfscope}%
\begin{pgfscope}%
\pgfsys@transformshift{0.972342in}{0.459688in}%
\pgfsys@useobject{currentmarker}{}%
\end{pgfscope}%
\begin{pgfscope}%
\pgfsys@transformshift{1.045503in}{0.465156in}%
\pgfsys@useobject{currentmarker}{}%
\end{pgfscope}%
\begin{pgfscope}%
\pgfsys@transformshift{1.119649in}{0.476356in}%
\pgfsys@useobject{currentmarker}{}%
\end{pgfscope}%
\begin{pgfscope}%
\pgfsys@transformshift{1.191908in}{0.459878in}%
\pgfsys@useobject{currentmarker}{}%
\end{pgfscope}%
\begin{pgfscope}%
\pgfsys@transformshift{1.265331in}{0.480357in}%
\pgfsys@useobject{currentmarker}{}%
\end{pgfscope}%
\begin{pgfscope}%
\pgfsys@transformshift{1.339149in}{0.474638in}%
\pgfsys@useobject{currentmarker}{}%
\end{pgfscope}%
\begin{pgfscope}%
\pgfsys@transformshift{1.411486in}{0.494134in}%
\pgfsys@useobject{currentmarker}{}%
\end{pgfscope}%
\begin{pgfscope}%
\pgfsys@transformshift{1.484998in}{0.485851in}%
\pgfsys@useobject{currentmarker}{}%
\end{pgfscope}%
\begin{pgfscope}%
\pgfsys@transformshift{1.558413in}{0.579525in}%
\pgfsys@useobject{currentmarker}{}%
\end{pgfscope}%
\begin{pgfscope}%
\pgfsys@transformshift{1.630880in}{0.580970in}%
\pgfsys@useobject{currentmarker}{}%
\end{pgfscope}%
\begin{pgfscope}%
\pgfsys@transformshift{1.704091in}{0.488100in}%
\pgfsys@useobject{currentmarker}{}%
\end{pgfscope}%
\begin{pgfscope}%
\pgfsys@transformshift{1.778592in}{0.484300in}%
\pgfsys@useobject{currentmarker}{}%
\end{pgfscope}%
\begin{pgfscope}%
\pgfsys@transformshift{1.851598in}{0.588135in}%
\pgfsys@useobject{currentmarker}{}%
\end{pgfscope}%
\begin{pgfscope}%
\pgfsys@transformshift{1.924607in}{0.497813in}%
\pgfsys@useobject{currentmarker}{}%
\end{pgfscope}%
\begin{pgfscope}%
\pgfsys@transformshift{2.003801in}{0.497196in}%
\pgfsys@useobject{currentmarker}{}%
\end{pgfscope}%
\begin{pgfscope}%
\pgfsys@transformshift{2.064635in}{0.599958in}%
\pgfsys@useobject{currentmarker}{}%
\end{pgfscope}%
\begin{pgfscope}%
\pgfsys@transformshift{2.142236in}{0.544639in}%
\pgfsys@useobject{currentmarker}{}%
\end{pgfscope}%
\begin{pgfscope}%
\pgfsys@transformshift{2.222941in}{0.618773in}%
\pgfsys@useobject{currentmarker}{}%
\end{pgfscope}%
\begin{pgfscope}%
\pgfsys@transformshift{2.289373in}{0.512944in}%
\pgfsys@useobject{currentmarker}{}%
\end{pgfscope}%
\begin{pgfscope}%
\pgfsys@transformshift{2.355652in}{0.602344in}%
\pgfsys@useobject{currentmarker}{}%
\end{pgfscope}%
\begin{pgfscope}%
\pgfsys@transformshift{2.434164in}{0.508579in}%
\pgfsys@useobject{currentmarker}{}%
\end{pgfscope}%
\begin{pgfscope}%
\pgfsys@transformshift{2.727780in}{0.596043in}%
\pgfsys@useobject{currentmarker}{}%
\end{pgfscope}%
\begin{pgfscope}%
\pgfsys@transformshift{2.882032in}{0.560588in}%
\pgfsys@useobject{currentmarker}{}%
\end{pgfscope}%
\begin{pgfscope}%
\pgfsys@transformshift{3.020982in}{0.568887in}%
\pgfsys@useobject{currentmarker}{}%
\end{pgfscope}%
\begin{pgfscope}%
\pgfsys@transformshift{3.165700in}{0.598011in}%
\pgfsys@useobject{currentmarker}{}%
\end{pgfscope}%
\begin{pgfscope}%
\pgfsys@transformshift{3.319528in}{0.594385in}%
\pgfsys@useobject{currentmarker}{}%
\end{pgfscope}%
\begin{pgfscope}%
\pgfsys@transformshift{3.466433in}{0.611477in}%
\pgfsys@useobject{currentmarker}{}%
\end{pgfscope}%
\begin{pgfscope}%
\pgfsys@transformshift{3.603372in}{0.607119in}%
\pgfsys@useobject{currentmarker}{}%
\end{pgfscope}%
\begin{pgfscope}%
\pgfsys@transformshift{3.750405in}{0.600902in}%
\pgfsys@useobject{currentmarker}{}%
\end{pgfscope}%
\begin{pgfscope}%
\pgfsys@transformshift{3.906080in}{0.608422in}%
\pgfsys@useobject{currentmarker}{}%
\end{pgfscope}%
\begin{pgfscope}%
\pgfsys@transformshift{4.192392in}{0.607722in}%
\pgfsys@useobject{currentmarker}{}%
\end{pgfscope}%
\end{pgfscope}%
\begin{pgfscope}%
\pgfsetroundcap%
\pgfsetroundjoin%
\pgfsetlinewidth{1.003750pt}%
\definecolor{currentstroke}{rgb}{0.007843,0.619608,0.450980}%
\pgfsetstrokecolor{currentstroke}%
\pgfsetdash{}{0pt}%
\pgfpathmoveto{\pgfqpoint{0.679638in}{1.606234in}}%
\pgfpathlineto{\pgfqpoint{0.753275in}{1.239687in}}%
\pgfpathlineto{\pgfqpoint{0.826099in}{1.173725in}}%
\pgfpathlineto{\pgfqpoint{0.899253in}{1.076079in}}%
\pgfpathlineto{\pgfqpoint{0.972342in}{0.705747in}}%
\pgfpathlineto{\pgfqpoint{1.045503in}{0.799756in}}%
\pgfpathlineto{\pgfqpoint{1.119649in}{1.149546in}}%
\pgfpathlineto{\pgfqpoint{1.191908in}{1.092036in}}%
\pgfpathlineto{\pgfqpoint{1.265331in}{0.908415in}}%
\pgfpathlineto{\pgfqpoint{1.339149in}{1.061347in}}%
\pgfpathlineto{\pgfqpoint{1.411486in}{1.311665in}}%
\pgfpathlineto{\pgfqpoint{1.484998in}{1.104570in}}%
\pgfpathlineto{\pgfqpoint{1.558413in}{0.690969in}}%
\pgfpathlineto{\pgfqpoint{1.630880in}{0.985690in}}%
\pgfpathlineto{\pgfqpoint{1.704091in}{1.108328in}}%
\pgfpathlineto{\pgfqpoint{1.778592in}{0.694805in}}%
\pgfpathlineto{\pgfqpoint{1.851598in}{0.822050in}}%
\pgfpathlineto{\pgfqpoint{1.924607in}{0.857284in}}%
\pgfpathlineto{\pgfqpoint{2.003801in}{0.649976in}}%
\pgfpathlineto{\pgfqpoint{2.064635in}{0.617099in}}%
\pgfpathlineto{\pgfqpoint{2.142236in}{0.592403in}}%
\pgfpathlineto{\pgfqpoint{2.222941in}{0.619197in}}%
\pgfpathlineto{\pgfqpoint{2.289373in}{0.694557in}}%
\pgfpathlineto{\pgfqpoint{2.355652in}{1.080581in}}%
\pgfpathlineto{\pgfqpoint{2.434164in}{1.040838in}}%
\pgfpathlineto{\pgfqpoint{2.727780in}{0.750809in}}%
\pgfpathlineto{\pgfqpoint{2.882032in}{0.745152in}}%
\pgfpathlineto{\pgfqpoint{3.020982in}{0.600960in}}%
\pgfpathlineto{\pgfqpoint{3.165700in}{0.714051in}}%
\pgfpathlineto{\pgfqpoint{3.319528in}{0.683952in}}%
\pgfpathlineto{\pgfqpoint{3.466433in}{0.576041in}}%
\pgfpathlineto{\pgfqpoint{3.603372in}{0.580158in}}%
\pgfpathlineto{\pgfqpoint{3.750405in}{0.590076in}}%
\pgfpathlineto{\pgfqpoint{3.906080in}{0.647822in}}%
\pgfpathlineto{\pgfqpoint{4.192392in}{0.685567in}}%
\pgfusepath{stroke}%
\end{pgfscope}%
\begin{pgfscope}%
\pgfsetbuttcap%
\pgfsetroundjoin%
\definecolor{currentfill}{rgb}{0.007843,0.619608,0.450980}%
\pgfsetfillcolor{currentfill}%
\pgfsetlinewidth{0.752812pt}%
\definecolor{currentstroke}{rgb}{1.000000,1.000000,1.000000}%
\pgfsetstrokecolor{currentstroke}%
\pgfsetdash{}{0pt}%
\pgfsys@defobject{currentmarker}{\pgfqpoint{-0.034722in}{-0.034722in}}{\pgfqpoint{0.034722in}{0.034722in}}{%
\pgfpathmoveto{\pgfqpoint{0.000000in}{-0.034722in}}%
\pgfpathcurveto{\pgfqpoint{0.009208in}{-0.034722in}}{\pgfqpoint{0.018041in}{-0.031064in}}{\pgfqpoint{0.024552in}{-0.024552in}}%
\pgfpathcurveto{\pgfqpoint{0.031064in}{-0.018041in}}{\pgfqpoint{0.034722in}{-0.009208in}}{\pgfqpoint{0.034722in}{0.000000in}}%
\pgfpathcurveto{\pgfqpoint{0.034722in}{0.009208in}}{\pgfqpoint{0.031064in}{0.018041in}}{\pgfqpoint{0.024552in}{0.024552in}}%
\pgfpathcurveto{\pgfqpoint{0.018041in}{0.031064in}}{\pgfqpoint{0.009208in}{0.034722in}}{\pgfqpoint{0.000000in}{0.034722in}}%
\pgfpathcurveto{\pgfqpoint{-0.009208in}{0.034722in}}{\pgfqpoint{-0.018041in}{0.031064in}}{\pgfqpoint{-0.024552in}{0.024552in}}%
\pgfpathcurveto{\pgfqpoint{-0.031064in}{0.018041in}}{\pgfqpoint{-0.034722in}{0.009208in}}{\pgfqpoint{-0.034722in}{0.000000in}}%
\pgfpathcurveto{\pgfqpoint{-0.034722in}{-0.009208in}}{\pgfqpoint{-0.031064in}{-0.018041in}}{\pgfqpoint{-0.024552in}{-0.024552in}}%
\pgfpathcurveto{\pgfqpoint{-0.018041in}{-0.031064in}}{\pgfqpoint{-0.009208in}{-0.034722in}}{\pgfqpoint{0.000000in}{-0.034722in}}%
\pgfpathlineto{\pgfqpoint{0.000000in}{-0.034722in}}%
\pgfpathclose%
\pgfusepath{stroke,fill}%
}%
\begin{pgfscope}%
\pgfsys@transformshift{0.679638in}{1.606234in}%
\pgfsys@useobject{currentmarker}{}%
\end{pgfscope}%
\begin{pgfscope}%
\pgfsys@transformshift{0.753275in}{1.239687in}%
\pgfsys@useobject{currentmarker}{}%
\end{pgfscope}%
\begin{pgfscope}%
\pgfsys@transformshift{0.826099in}{1.173725in}%
\pgfsys@useobject{currentmarker}{}%
\end{pgfscope}%
\begin{pgfscope}%
\pgfsys@transformshift{0.899253in}{1.076079in}%
\pgfsys@useobject{currentmarker}{}%
\end{pgfscope}%
\begin{pgfscope}%
\pgfsys@transformshift{0.972342in}{0.705747in}%
\pgfsys@useobject{currentmarker}{}%
\end{pgfscope}%
\begin{pgfscope}%
\pgfsys@transformshift{1.045503in}{0.799756in}%
\pgfsys@useobject{currentmarker}{}%
\end{pgfscope}%
\begin{pgfscope}%
\pgfsys@transformshift{1.119649in}{1.149546in}%
\pgfsys@useobject{currentmarker}{}%
\end{pgfscope}%
\begin{pgfscope}%
\pgfsys@transformshift{1.191908in}{1.092036in}%
\pgfsys@useobject{currentmarker}{}%
\end{pgfscope}%
\begin{pgfscope}%
\pgfsys@transformshift{1.265331in}{0.908415in}%
\pgfsys@useobject{currentmarker}{}%
\end{pgfscope}%
\begin{pgfscope}%
\pgfsys@transformshift{1.339149in}{1.061347in}%
\pgfsys@useobject{currentmarker}{}%
\end{pgfscope}%
\begin{pgfscope}%
\pgfsys@transformshift{1.411486in}{1.311665in}%
\pgfsys@useobject{currentmarker}{}%
\end{pgfscope}%
\begin{pgfscope}%
\pgfsys@transformshift{1.484998in}{1.104570in}%
\pgfsys@useobject{currentmarker}{}%
\end{pgfscope}%
\begin{pgfscope}%
\pgfsys@transformshift{1.558413in}{0.690969in}%
\pgfsys@useobject{currentmarker}{}%
\end{pgfscope}%
\begin{pgfscope}%
\pgfsys@transformshift{1.630880in}{0.985690in}%
\pgfsys@useobject{currentmarker}{}%
\end{pgfscope}%
\begin{pgfscope}%
\pgfsys@transformshift{1.704091in}{1.108328in}%
\pgfsys@useobject{currentmarker}{}%
\end{pgfscope}%
\begin{pgfscope}%
\pgfsys@transformshift{1.778592in}{0.694805in}%
\pgfsys@useobject{currentmarker}{}%
\end{pgfscope}%
\begin{pgfscope}%
\pgfsys@transformshift{1.851598in}{0.822050in}%
\pgfsys@useobject{currentmarker}{}%
\end{pgfscope}%
\begin{pgfscope}%
\pgfsys@transformshift{1.924607in}{0.857284in}%
\pgfsys@useobject{currentmarker}{}%
\end{pgfscope}%
\begin{pgfscope}%
\pgfsys@transformshift{2.003801in}{0.649976in}%
\pgfsys@useobject{currentmarker}{}%
\end{pgfscope}%
\begin{pgfscope}%
\pgfsys@transformshift{2.064635in}{0.617099in}%
\pgfsys@useobject{currentmarker}{}%
\end{pgfscope}%
\begin{pgfscope}%
\pgfsys@transformshift{2.142236in}{0.592403in}%
\pgfsys@useobject{currentmarker}{}%
\end{pgfscope}%
\begin{pgfscope}%
\pgfsys@transformshift{2.222941in}{0.619197in}%
\pgfsys@useobject{currentmarker}{}%
\end{pgfscope}%
\begin{pgfscope}%
\pgfsys@transformshift{2.289373in}{0.694557in}%
\pgfsys@useobject{currentmarker}{}%
\end{pgfscope}%
\begin{pgfscope}%
\pgfsys@transformshift{2.355652in}{1.080581in}%
\pgfsys@useobject{currentmarker}{}%
\end{pgfscope}%
\begin{pgfscope}%
\pgfsys@transformshift{2.434164in}{1.040838in}%
\pgfsys@useobject{currentmarker}{}%
\end{pgfscope}%
\begin{pgfscope}%
\pgfsys@transformshift{2.727780in}{0.750809in}%
\pgfsys@useobject{currentmarker}{}%
\end{pgfscope}%
\begin{pgfscope}%
\pgfsys@transformshift{2.882032in}{0.745152in}%
\pgfsys@useobject{currentmarker}{}%
\end{pgfscope}%
\begin{pgfscope}%
\pgfsys@transformshift{3.020982in}{0.600960in}%
\pgfsys@useobject{currentmarker}{}%
\end{pgfscope}%
\begin{pgfscope}%
\pgfsys@transformshift{3.165700in}{0.714051in}%
\pgfsys@useobject{currentmarker}{}%
\end{pgfscope}%
\begin{pgfscope}%
\pgfsys@transformshift{3.319528in}{0.683952in}%
\pgfsys@useobject{currentmarker}{}%
\end{pgfscope}%
\begin{pgfscope}%
\pgfsys@transformshift{3.466433in}{0.576041in}%
\pgfsys@useobject{currentmarker}{}%
\end{pgfscope}%
\begin{pgfscope}%
\pgfsys@transformshift{3.603372in}{0.580158in}%
\pgfsys@useobject{currentmarker}{}%
\end{pgfscope}%
\begin{pgfscope}%
\pgfsys@transformshift{3.750405in}{0.590076in}%
\pgfsys@useobject{currentmarker}{}%
\end{pgfscope}%
\begin{pgfscope}%
\pgfsys@transformshift{3.906080in}{0.647822in}%
\pgfsys@useobject{currentmarker}{}%
\end{pgfscope}%
\begin{pgfscope}%
\pgfsys@transformshift{4.192392in}{0.685567in}%
\pgfsys@useobject{currentmarker}{}%
\end{pgfscope}%
\end{pgfscope}%
\begin{pgfscope}%
\pgfsetroundcap%
\pgfsetroundjoin%
\pgfsetlinewidth{1.003750pt}%
\definecolor{currentstroke}{rgb}{0.835294,0.368627,0.000000}%
\pgfsetstrokecolor{currentstroke}%
\pgfsetdash{}{0pt}%
\pgfpathmoveto{\pgfqpoint{0.679638in}{0.498379in}}%
\pgfpathlineto{\pgfqpoint{0.753275in}{0.494162in}}%
\pgfpathlineto{\pgfqpoint{0.826099in}{0.491419in}}%
\pgfpathlineto{\pgfqpoint{0.899253in}{0.488575in}}%
\pgfpathlineto{\pgfqpoint{0.972342in}{0.483517in}}%
\pgfpathlineto{\pgfqpoint{1.045503in}{0.483415in}}%
\pgfpathlineto{\pgfqpoint{1.119649in}{0.489532in}}%
\pgfpathlineto{\pgfqpoint{1.191908in}{0.490872in}}%
\pgfpathlineto{\pgfqpoint{1.265331in}{0.488747in}}%
\pgfpathlineto{\pgfqpoint{1.339149in}{0.488872in}}%
\pgfpathlineto{\pgfqpoint{1.411486in}{0.494134in}}%
\pgfpathlineto{\pgfqpoint{1.484998in}{0.492579in}}%
\pgfpathlineto{\pgfqpoint{1.558413in}{0.484143in}}%
\pgfpathlineto{\pgfqpoint{1.630880in}{0.487149in}}%
\pgfpathlineto{\pgfqpoint{1.704091in}{0.491681in}}%
\pgfpathlineto{\pgfqpoint{1.778592in}{0.484669in}}%
\pgfpathlineto{\pgfqpoint{1.851598in}{0.482024in}}%
\pgfpathlineto{\pgfqpoint{1.924607in}{0.482545in}}%
\pgfpathlineto{\pgfqpoint{2.003801in}{0.488814in}}%
\pgfpathlineto{\pgfqpoint{2.064635in}{0.482487in}}%
\pgfpathlineto{\pgfqpoint{2.142236in}{0.482154in}}%
\pgfpathlineto{\pgfqpoint{2.222941in}{0.487715in}}%
\pgfpathlineto{\pgfqpoint{2.289373in}{0.485434in}}%
\pgfpathlineto{\pgfqpoint{2.355652in}{0.492495in}}%
\pgfpathlineto{\pgfqpoint{2.434164in}{0.489570in}}%
\pgfpathlineto{\pgfqpoint{2.727780in}{0.482696in}}%
\pgfpathlineto{\pgfqpoint{2.882032in}{0.482810in}}%
\pgfpathlineto{\pgfqpoint{3.020982in}{0.481303in}}%
\pgfpathlineto{\pgfqpoint{3.165700in}{0.482682in}}%
\pgfpathlineto{\pgfqpoint{3.319528in}{0.483848in}}%
\pgfpathlineto{\pgfqpoint{3.466433in}{0.483218in}}%
\pgfpathlineto{\pgfqpoint{3.603372in}{0.481354in}}%
\pgfpathlineto{\pgfqpoint{3.750405in}{0.480600in}}%
\pgfpathlineto{\pgfqpoint{3.906080in}{0.484588in}}%
\pgfpathlineto{\pgfqpoint{4.192392in}{0.485241in}}%
\pgfusepath{stroke}%
\end{pgfscope}%
\begin{pgfscope}%
\pgfsetbuttcap%
\pgfsetroundjoin%
\definecolor{currentfill}{rgb}{0.835294,0.368627,0.000000}%
\pgfsetfillcolor{currentfill}%
\pgfsetlinewidth{0.752812pt}%
\definecolor{currentstroke}{rgb}{1.000000,1.000000,1.000000}%
\pgfsetstrokecolor{currentstroke}%
\pgfsetdash{}{0pt}%
\pgfsys@defobject{currentmarker}{\pgfqpoint{-0.034722in}{-0.034722in}}{\pgfqpoint{0.034722in}{0.034722in}}{%
\pgfpathmoveto{\pgfqpoint{0.000000in}{-0.034722in}}%
\pgfpathcurveto{\pgfqpoint{0.009208in}{-0.034722in}}{\pgfqpoint{0.018041in}{-0.031064in}}{\pgfqpoint{0.024552in}{-0.024552in}}%
\pgfpathcurveto{\pgfqpoint{0.031064in}{-0.018041in}}{\pgfqpoint{0.034722in}{-0.009208in}}{\pgfqpoint{0.034722in}{0.000000in}}%
\pgfpathcurveto{\pgfqpoint{0.034722in}{0.009208in}}{\pgfqpoint{0.031064in}{0.018041in}}{\pgfqpoint{0.024552in}{0.024552in}}%
\pgfpathcurveto{\pgfqpoint{0.018041in}{0.031064in}}{\pgfqpoint{0.009208in}{0.034722in}}{\pgfqpoint{0.000000in}{0.034722in}}%
\pgfpathcurveto{\pgfqpoint{-0.009208in}{0.034722in}}{\pgfqpoint{-0.018041in}{0.031064in}}{\pgfqpoint{-0.024552in}{0.024552in}}%
\pgfpathcurveto{\pgfqpoint{-0.031064in}{0.018041in}}{\pgfqpoint{-0.034722in}{0.009208in}}{\pgfqpoint{-0.034722in}{0.000000in}}%
\pgfpathcurveto{\pgfqpoint{-0.034722in}{-0.009208in}}{\pgfqpoint{-0.031064in}{-0.018041in}}{\pgfqpoint{-0.024552in}{-0.024552in}}%
\pgfpathcurveto{\pgfqpoint{-0.018041in}{-0.031064in}}{\pgfqpoint{-0.009208in}{-0.034722in}}{\pgfqpoint{0.000000in}{-0.034722in}}%
\pgfpathlineto{\pgfqpoint{0.000000in}{-0.034722in}}%
\pgfpathclose%
\pgfusepath{stroke,fill}%
}%
\begin{pgfscope}%
\pgfsys@transformshift{0.679638in}{0.498379in}%
\pgfsys@useobject{currentmarker}{}%
\end{pgfscope}%
\begin{pgfscope}%
\pgfsys@transformshift{0.753275in}{0.494162in}%
\pgfsys@useobject{currentmarker}{}%
\end{pgfscope}%
\begin{pgfscope}%
\pgfsys@transformshift{0.826099in}{0.491419in}%
\pgfsys@useobject{currentmarker}{}%
\end{pgfscope}%
\begin{pgfscope}%
\pgfsys@transformshift{0.899253in}{0.488575in}%
\pgfsys@useobject{currentmarker}{}%
\end{pgfscope}%
\begin{pgfscope}%
\pgfsys@transformshift{0.972342in}{0.483517in}%
\pgfsys@useobject{currentmarker}{}%
\end{pgfscope}%
\begin{pgfscope}%
\pgfsys@transformshift{1.045503in}{0.483415in}%
\pgfsys@useobject{currentmarker}{}%
\end{pgfscope}%
\begin{pgfscope}%
\pgfsys@transformshift{1.119649in}{0.489532in}%
\pgfsys@useobject{currentmarker}{}%
\end{pgfscope}%
\begin{pgfscope}%
\pgfsys@transformshift{1.191908in}{0.490872in}%
\pgfsys@useobject{currentmarker}{}%
\end{pgfscope}%
\begin{pgfscope}%
\pgfsys@transformshift{1.265331in}{0.488747in}%
\pgfsys@useobject{currentmarker}{}%
\end{pgfscope}%
\begin{pgfscope}%
\pgfsys@transformshift{1.339149in}{0.488872in}%
\pgfsys@useobject{currentmarker}{}%
\end{pgfscope}%
\begin{pgfscope}%
\pgfsys@transformshift{1.411486in}{0.494134in}%
\pgfsys@useobject{currentmarker}{}%
\end{pgfscope}%
\begin{pgfscope}%
\pgfsys@transformshift{1.484998in}{0.492579in}%
\pgfsys@useobject{currentmarker}{}%
\end{pgfscope}%
\begin{pgfscope}%
\pgfsys@transformshift{1.558413in}{0.484143in}%
\pgfsys@useobject{currentmarker}{}%
\end{pgfscope}%
\begin{pgfscope}%
\pgfsys@transformshift{1.630880in}{0.487149in}%
\pgfsys@useobject{currentmarker}{}%
\end{pgfscope}%
\begin{pgfscope}%
\pgfsys@transformshift{1.704091in}{0.491681in}%
\pgfsys@useobject{currentmarker}{}%
\end{pgfscope}%
\begin{pgfscope}%
\pgfsys@transformshift{1.778592in}{0.484669in}%
\pgfsys@useobject{currentmarker}{}%
\end{pgfscope}%
\begin{pgfscope}%
\pgfsys@transformshift{1.851598in}{0.482024in}%
\pgfsys@useobject{currentmarker}{}%
\end{pgfscope}%
\begin{pgfscope}%
\pgfsys@transformshift{1.924607in}{0.482545in}%
\pgfsys@useobject{currentmarker}{}%
\end{pgfscope}%
\begin{pgfscope}%
\pgfsys@transformshift{2.003801in}{0.488814in}%
\pgfsys@useobject{currentmarker}{}%
\end{pgfscope}%
\begin{pgfscope}%
\pgfsys@transformshift{2.064635in}{0.482487in}%
\pgfsys@useobject{currentmarker}{}%
\end{pgfscope}%
\begin{pgfscope}%
\pgfsys@transformshift{2.142236in}{0.482154in}%
\pgfsys@useobject{currentmarker}{}%
\end{pgfscope}%
\begin{pgfscope}%
\pgfsys@transformshift{2.222941in}{0.487715in}%
\pgfsys@useobject{currentmarker}{}%
\end{pgfscope}%
\begin{pgfscope}%
\pgfsys@transformshift{2.289373in}{0.485434in}%
\pgfsys@useobject{currentmarker}{}%
\end{pgfscope}%
\begin{pgfscope}%
\pgfsys@transformshift{2.355652in}{0.492495in}%
\pgfsys@useobject{currentmarker}{}%
\end{pgfscope}%
\begin{pgfscope}%
\pgfsys@transformshift{2.434164in}{0.489570in}%
\pgfsys@useobject{currentmarker}{}%
\end{pgfscope}%
\begin{pgfscope}%
\pgfsys@transformshift{2.727780in}{0.482696in}%
\pgfsys@useobject{currentmarker}{}%
\end{pgfscope}%
\begin{pgfscope}%
\pgfsys@transformshift{2.882032in}{0.482810in}%
\pgfsys@useobject{currentmarker}{}%
\end{pgfscope}%
\begin{pgfscope}%
\pgfsys@transformshift{3.020982in}{0.481303in}%
\pgfsys@useobject{currentmarker}{}%
\end{pgfscope}%
\begin{pgfscope}%
\pgfsys@transformshift{3.165700in}{0.482682in}%
\pgfsys@useobject{currentmarker}{}%
\end{pgfscope}%
\begin{pgfscope}%
\pgfsys@transformshift{3.319528in}{0.483848in}%
\pgfsys@useobject{currentmarker}{}%
\end{pgfscope}%
\begin{pgfscope}%
\pgfsys@transformshift{3.466433in}{0.483218in}%
\pgfsys@useobject{currentmarker}{}%
\end{pgfscope}%
\begin{pgfscope}%
\pgfsys@transformshift{3.603372in}{0.481354in}%
\pgfsys@useobject{currentmarker}{}%
\end{pgfscope}%
\begin{pgfscope}%
\pgfsys@transformshift{3.750405in}{0.480600in}%
\pgfsys@useobject{currentmarker}{}%
\end{pgfscope}%
\begin{pgfscope}%
\pgfsys@transformshift{3.906080in}{0.484588in}%
\pgfsys@useobject{currentmarker}{}%
\end{pgfscope}%
\begin{pgfscope}%
\pgfsys@transformshift{4.192392in}{0.485241in}%
\pgfsys@useobject{currentmarker}{}%
\end{pgfscope}%
\end{pgfscope}%
\end{pgfpicture}%
\makeatother%
\endgroup%

						\end{figcenter}
						\caption{Detailed routing times of the largest dataset with 6055 input vertices.}
						\label{fig:eval-rural-routing-details-b}
					\end{subfigure}
					\\[3ex]
					\begin{subfigure}[t]{\textwidth}
						\begin{figcenter}
							%% Creator: Matplotlib, PGF backend
%%
%% To include the figure in your LaTeX document, write
%%   \input{<filename>.pgf}
%%
%% Make sure the required packages are loaded in your preamble
%%   \usepackage{pgf}
%%
%% Also ensure that all the required font packages are loaded; for instance,
%% the lmodern package is sometimes necessary when using math font.
%%   \usepackage{lmodern}
%%
%% Figures using additional raster images can only be included by \input if
%% they are in the same directory as the main LaTeX file. For loading figures
%% from other directories you can use the `import` package
%%   \usepackage{import}
%%
%% and then include the figures with
%%   \import{<path to file>}{<filename>.pgf}
%%
%% Matplotlib used the following preamble
%%   
%%   \usepackage{fontspec}
%%   \setmainfont{DejaVuSerif.ttf}[Path=\detokenize{/usr/lib/python3.11/site-packages/matplotlib/mpl-data/fonts/ttf/}]
%%   \setsansfont{DroidSans.ttf}[Path=\detokenize{/usr/share/fonts/droid/}]
%%   \setmonofont{DejaVuSansMono.ttf}[Path=\detokenize{/usr/lib/python3.11/site-packages/matplotlib/mpl-data/fonts/ttf/}]
%%   \makeatletter\@ifpackageloaded{underscore}{}{\usepackage[strings]{underscore}}\makeatother
%%
\begingroup%
\makeatletter%
\begin{pgfpicture}%
\pgfpathrectangle{\pgfpointorigin}{\pgfqpoint{5.721286in}{1.715550in}}%
\pgfusepath{use as bounding box, clip}%
\begin{pgfscope}%
\pgfsetbuttcap%
\pgfsetmiterjoin%
\definecolor{currentfill}{rgb}{1.000000,1.000000,1.000000}%
\pgfsetfillcolor{currentfill}%
\pgfsetlinewidth{0.000000pt}%
\definecolor{currentstroke}{rgb}{1.000000,1.000000,1.000000}%
\pgfsetstrokecolor{currentstroke}%
\pgfsetdash{}{0pt}%
\pgfpathmoveto{\pgfqpoint{0.000000in}{0.000000in}}%
\pgfpathlineto{\pgfqpoint{5.721286in}{0.000000in}}%
\pgfpathlineto{\pgfqpoint{5.721286in}{1.715550in}}%
\pgfpathlineto{\pgfqpoint{0.000000in}{1.715550in}}%
\pgfpathlineto{\pgfqpoint{0.000000in}{0.000000in}}%
\pgfpathclose%
\pgfusepath{fill}%
\end{pgfscope}%
\begin{pgfscope}%
\pgfsetbuttcap%
\pgfsetmiterjoin%
\definecolor{currentfill}{rgb}{1.000000,1.000000,1.000000}%
\pgfsetfillcolor{currentfill}%
\pgfsetlinewidth{0.000000pt}%
\definecolor{currentstroke}{rgb}{0.000000,0.000000,0.000000}%
\pgfsetstrokecolor{currentstroke}%
\pgfsetstrokeopacity{0.000000}%
\pgfsetdash{}{0pt}%
\pgfpathmoveto{\pgfqpoint{0.566440in}{0.451389in}}%
\pgfpathlineto{\pgfqpoint{4.025686in}{0.451389in}}%
\pgfpathlineto{\pgfqpoint{4.025686in}{1.715550in}}%
\pgfpathlineto{\pgfqpoint{0.566440in}{1.715550in}}%
\pgfpathlineto{\pgfqpoint{0.566440in}{0.451389in}}%
\pgfpathclose%
\pgfusepath{fill}%
\end{pgfscope}%
\begin{pgfscope}%
\pgfpathrectangle{\pgfqpoint{0.566440in}{0.451389in}}{\pgfqpoint{3.459246in}{1.264161in}}%
\pgfusepath{clip}%
\pgfsetroundcap%
\pgfsetroundjoin%
\pgfsetlinewidth{1.003750pt}%
\definecolor{currentstroke}{rgb}{0.800000,0.800000,0.800000}%
\pgfsetstrokecolor{currentstroke}%
\pgfsetdash{}{0pt}%
\pgfpathmoveto{\pgfqpoint{0.566440in}{0.451389in}}%
\pgfpathlineto{\pgfqpoint{0.566440in}{1.715550in}}%
\pgfusepath{stroke}%
\end{pgfscope}%
\begin{pgfscope}%
\definecolor{textcolor}{rgb}{0.150000,0.150000,0.150000}%
\pgfsetstrokecolor{textcolor}%
\pgfsetfillcolor{textcolor}%
\pgftext[x=0.566440in,y=0.319444in,,top]{\color{textcolor}\sffamily\fontsize{9.000000}{10.800000}\selectfont 0.0}%
\end{pgfscope}%
\begin{pgfscope}%
\pgfpathrectangle{\pgfqpoint{0.566440in}{0.451389in}}{\pgfqpoint{3.459246in}{1.264161in}}%
\pgfusepath{clip}%
\pgfsetroundcap%
\pgfsetroundjoin%
\pgfsetlinewidth{1.003750pt}%
\definecolor{currentstroke}{rgb}{0.800000,0.800000,0.800000}%
\pgfsetstrokecolor{currentstroke}%
\pgfsetdash{}{0pt}%
\pgfpathmoveto{\pgfqpoint{1.224051in}{0.451389in}}%
\pgfpathlineto{\pgfqpoint{1.224051in}{1.715550in}}%
\pgfusepath{stroke}%
\end{pgfscope}%
\begin{pgfscope}%
\definecolor{textcolor}{rgb}{0.150000,0.150000,0.150000}%
\pgfsetstrokecolor{textcolor}%
\pgfsetfillcolor{textcolor}%
\pgftext[x=1.224051in,y=0.319444in,,top]{\color{textcolor}\sffamily\fontsize{9.000000}{10.800000}\selectfont 0.5}%
\end{pgfscope}%
\begin{pgfscope}%
\pgfpathrectangle{\pgfqpoint{0.566440in}{0.451389in}}{\pgfqpoint{3.459246in}{1.264161in}}%
\pgfusepath{clip}%
\pgfsetroundcap%
\pgfsetroundjoin%
\pgfsetlinewidth{1.003750pt}%
\definecolor{currentstroke}{rgb}{0.800000,0.800000,0.800000}%
\pgfsetstrokecolor{currentstroke}%
\pgfsetdash{}{0pt}%
\pgfpathmoveto{\pgfqpoint{1.881662in}{0.451389in}}%
\pgfpathlineto{\pgfqpoint{1.881662in}{1.715550in}}%
\pgfusepath{stroke}%
\end{pgfscope}%
\begin{pgfscope}%
\definecolor{textcolor}{rgb}{0.150000,0.150000,0.150000}%
\pgfsetstrokecolor{textcolor}%
\pgfsetfillcolor{textcolor}%
\pgftext[x=1.881662in,y=0.319444in,,top]{\color{textcolor}\sffamily\fontsize{9.000000}{10.800000}\selectfont 1.0}%
\end{pgfscope}%
\begin{pgfscope}%
\pgfpathrectangle{\pgfqpoint{0.566440in}{0.451389in}}{\pgfqpoint{3.459246in}{1.264161in}}%
\pgfusepath{clip}%
\pgfsetroundcap%
\pgfsetroundjoin%
\pgfsetlinewidth{1.003750pt}%
\definecolor{currentstroke}{rgb}{0.800000,0.800000,0.800000}%
\pgfsetstrokecolor{currentstroke}%
\pgfsetdash{}{0pt}%
\pgfpathmoveto{\pgfqpoint{2.539273in}{0.451389in}}%
\pgfpathlineto{\pgfqpoint{2.539273in}{1.715550in}}%
\pgfusepath{stroke}%
\end{pgfscope}%
\begin{pgfscope}%
\definecolor{textcolor}{rgb}{0.150000,0.150000,0.150000}%
\pgfsetstrokecolor{textcolor}%
\pgfsetfillcolor{textcolor}%
\pgftext[x=2.539273in,y=0.319444in,,top]{\color{textcolor}\sffamily\fontsize{9.000000}{10.800000}\selectfont 1.5}%
\end{pgfscope}%
\begin{pgfscope}%
\pgfpathrectangle{\pgfqpoint{0.566440in}{0.451389in}}{\pgfqpoint{3.459246in}{1.264161in}}%
\pgfusepath{clip}%
\pgfsetroundcap%
\pgfsetroundjoin%
\pgfsetlinewidth{1.003750pt}%
\definecolor{currentstroke}{rgb}{0.800000,0.800000,0.800000}%
\pgfsetstrokecolor{currentstroke}%
\pgfsetdash{}{0pt}%
\pgfpathmoveto{\pgfqpoint{3.196885in}{0.451389in}}%
\pgfpathlineto{\pgfqpoint{3.196885in}{1.715550in}}%
\pgfusepath{stroke}%
\end{pgfscope}%
\begin{pgfscope}%
\definecolor{textcolor}{rgb}{0.150000,0.150000,0.150000}%
\pgfsetstrokecolor{textcolor}%
\pgfsetfillcolor{textcolor}%
\pgftext[x=3.196885in,y=0.319444in,,top]{\color{textcolor}\sffamily\fontsize{9.000000}{10.800000}\selectfont 2.0}%
\end{pgfscope}%
\begin{pgfscope}%
\pgfpathrectangle{\pgfqpoint{0.566440in}{0.451389in}}{\pgfqpoint{3.459246in}{1.264161in}}%
\pgfusepath{clip}%
\pgfsetroundcap%
\pgfsetroundjoin%
\pgfsetlinewidth{1.003750pt}%
\definecolor{currentstroke}{rgb}{0.800000,0.800000,0.800000}%
\pgfsetstrokecolor{currentstroke}%
\pgfsetdash{}{0pt}%
\pgfpathmoveto{\pgfqpoint{3.854496in}{0.451389in}}%
\pgfpathlineto{\pgfqpoint{3.854496in}{1.715550in}}%
\pgfusepath{stroke}%
\end{pgfscope}%
\begin{pgfscope}%
\definecolor{textcolor}{rgb}{0.150000,0.150000,0.150000}%
\pgfsetstrokecolor{textcolor}%
\pgfsetfillcolor{textcolor}%
\pgftext[x=3.854496in,y=0.319444in,,top]{\color{textcolor}\sffamily\fontsize{9.000000}{10.800000}\selectfont 2.5}%
\end{pgfscope}%
\begin{pgfscope}%
\definecolor{textcolor}{rgb}{0.150000,0.150000,0.150000}%
\pgfsetstrokecolor{textcolor}%
\pgfsetfillcolor{textcolor}%
\pgftext[x=2.296063in,y=0.125000in,,top]{\color{textcolor}\sffamily\fontsize{9.000000}{10.800000}\selectfont Beeline distance in km}%
\end{pgfscope}%
\begin{pgfscope}%
\pgfpathrectangle{\pgfqpoint{0.566440in}{0.451389in}}{\pgfqpoint{3.459246in}{1.264161in}}%
\pgfusepath{clip}%
\pgfsetroundcap%
\pgfsetroundjoin%
\pgfsetlinewidth{1.003750pt}%
\definecolor{currentstroke}{rgb}{0.800000,0.800000,0.800000}%
\pgfsetstrokecolor{currentstroke}%
\pgfsetdash{}{0pt}%
\pgfpathmoveto{\pgfqpoint{0.566440in}{0.451389in}}%
\pgfpathlineto{\pgfqpoint{4.025686in}{0.451389in}}%
\pgfusepath{stroke}%
\end{pgfscope}%
\begin{pgfscope}%
\definecolor{textcolor}{rgb}{0.150000,0.150000,0.150000}%
\pgfsetstrokecolor{textcolor}%
\pgfsetfillcolor{textcolor}%
\pgftext[x=0.194444in, y=0.403903in, left, base]{\color{textcolor}\sffamily\fontsize{9.000000}{10.800000}\selectfont 0.00}%
\end{pgfscope}%
\begin{pgfscope}%
\pgfpathrectangle{\pgfqpoint{0.566440in}{0.451389in}}{\pgfqpoint{3.459246in}{1.264161in}}%
\pgfusepath{clip}%
\pgfsetroundcap%
\pgfsetroundjoin%
\pgfsetlinewidth{1.003750pt}%
\definecolor{currentstroke}{rgb}{0.800000,0.800000,0.800000}%
\pgfsetstrokecolor{currentstroke}%
\pgfsetdash{}{0pt}%
\pgfpathmoveto{\pgfqpoint{0.566440in}{0.752420in}}%
\pgfpathlineto{\pgfqpoint{4.025686in}{0.752420in}}%
\pgfusepath{stroke}%
\end{pgfscope}%
\begin{pgfscope}%
\definecolor{textcolor}{rgb}{0.150000,0.150000,0.150000}%
\pgfsetstrokecolor{textcolor}%
\pgfsetfillcolor{textcolor}%
\pgftext[x=0.194444in, y=0.704935in, left, base]{\color{textcolor}\sffamily\fontsize{9.000000}{10.800000}\selectfont 0.25}%
\end{pgfscope}%
\begin{pgfscope}%
\pgfpathrectangle{\pgfqpoint{0.566440in}{0.451389in}}{\pgfqpoint{3.459246in}{1.264161in}}%
\pgfusepath{clip}%
\pgfsetroundcap%
\pgfsetroundjoin%
\pgfsetlinewidth{1.003750pt}%
\definecolor{currentstroke}{rgb}{0.800000,0.800000,0.800000}%
\pgfsetstrokecolor{currentstroke}%
\pgfsetdash{}{0pt}%
\pgfpathmoveto{\pgfqpoint{0.566440in}{1.053451in}}%
\pgfpathlineto{\pgfqpoint{4.025686in}{1.053451in}}%
\pgfusepath{stroke}%
\end{pgfscope}%
\begin{pgfscope}%
\definecolor{textcolor}{rgb}{0.150000,0.150000,0.150000}%
\pgfsetstrokecolor{textcolor}%
\pgfsetfillcolor{textcolor}%
\pgftext[x=0.194444in, y=1.005966in, left, base]{\color{textcolor}\sffamily\fontsize{9.000000}{10.800000}\selectfont 0.50}%
\end{pgfscope}%
\begin{pgfscope}%
\pgfpathrectangle{\pgfqpoint{0.566440in}{0.451389in}}{\pgfqpoint{3.459246in}{1.264161in}}%
\pgfusepath{clip}%
\pgfsetroundcap%
\pgfsetroundjoin%
\pgfsetlinewidth{1.003750pt}%
\definecolor{currentstroke}{rgb}{0.800000,0.800000,0.800000}%
\pgfsetstrokecolor{currentstroke}%
\pgfsetdash{}{0pt}%
\pgfpathmoveto{\pgfqpoint{0.566440in}{1.354482in}}%
\pgfpathlineto{\pgfqpoint{4.025686in}{1.354482in}}%
\pgfusepath{stroke}%
\end{pgfscope}%
\begin{pgfscope}%
\definecolor{textcolor}{rgb}{0.150000,0.150000,0.150000}%
\pgfsetstrokecolor{textcolor}%
\pgfsetfillcolor{textcolor}%
\pgftext[x=0.194444in, y=1.306997in, left, base]{\color{textcolor}\sffamily\fontsize{9.000000}{10.800000}\selectfont 0.75}%
\end{pgfscope}%
\begin{pgfscope}%
\pgfpathrectangle{\pgfqpoint{0.566440in}{0.451389in}}{\pgfqpoint{3.459246in}{1.264161in}}%
\pgfusepath{clip}%
\pgfsetroundcap%
\pgfsetroundjoin%
\pgfsetlinewidth{1.003750pt}%
\definecolor{currentstroke}{rgb}{0.800000,0.800000,0.800000}%
\pgfsetstrokecolor{currentstroke}%
\pgfsetdash{}{0pt}%
\pgfpathmoveto{\pgfqpoint{0.566440in}{1.655514in}}%
\pgfpathlineto{\pgfqpoint{4.025686in}{1.655514in}}%
\pgfusepath{stroke}%
\end{pgfscope}%
\begin{pgfscope}%
\definecolor{textcolor}{rgb}{0.150000,0.150000,0.150000}%
\pgfsetstrokecolor{textcolor}%
\pgfsetfillcolor{textcolor}%
\pgftext[x=0.194444in, y=1.608028in, left, base]{\color{textcolor}\sffamily\fontsize{9.000000}{10.800000}\selectfont 1.00}%
\end{pgfscope}%
\begin{pgfscope}%
\definecolor{textcolor}{rgb}{0.150000,0.150000,0.150000}%
\pgfsetstrokecolor{textcolor}%
\pgfsetfillcolor{textcolor}%
\pgftext[x=0.125000in,y=1.083469in,,bottom,rotate=90.000000]{\color{textcolor}\sffamily\fontsize{9.000000}{10.800000}\selectfont Share of total time}%
\end{pgfscope}%
\begin{pgfscope}%
\pgfsetrectcap%
\pgfsetmiterjoin%
\pgfsetlinewidth{1.254687pt}%
\definecolor{currentstroke}{rgb}{0.800000,0.800000,0.800000}%
\pgfsetstrokecolor{currentstroke}%
\pgfsetdash{}{0pt}%
\pgfpathmoveto{\pgfqpoint{0.566440in}{0.451389in}}%
\pgfpathlineto{\pgfqpoint{0.566440in}{1.715550in}}%
\pgfusepath{stroke}%
\end{pgfscope}%
\begin{pgfscope}%
\pgfsetrectcap%
\pgfsetmiterjoin%
\pgfsetlinewidth{1.254687pt}%
\definecolor{currentstroke}{rgb}{0.800000,0.800000,0.800000}%
\pgfsetstrokecolor{currentstroke}%
\pgfsetdash{}{0pt}%
\pgfpathmoveto{\pgfqpoint{4.025686in}{0.451389in}}%
\pgfpathlineto{\pgfqpoint{4.025686in}{1.715550in}}%
\pgfusepath{stroke}%
\end{pgfscope}%
\begin{pgfscope}%
\pgfsetrectcap%
\pgfsetmiterjoin%
\pgfsetlinewidth{1.254687pt}%
\definecolor{currentstroke}{rgb}{0.800000,0.800000,0.800000}%
\pgfsetstrokecolor{currentstroke}%
\pgfsetdash{}{0pt}%
\pgfpathmoveto{\pgfqpoint{0.566440in}{0.451389in}}%
\pgfpathlineto{\pgfqpoint{4.025686in}{0.451389in}}%
\pgfusepath{stroke}%
\end{pgfscope}%
\begin{pgfscope}%
\pgfsetrectcap%
\pgfsetmiterjoin%
\pgfsetlinewidth{1.254687pt}%
\definecolor{currentstroke}{rgb}{0.800000,0.800000,0.800000}%
\pgfsetstrokecolor{currentstroke}%
\pgfsetdash{}{0pt}%
\pgfpathmoveto{\pgfqpoint{0.566440in}{1.715550in}}%
\pgfpathlineto{\pgfqpoint{4.025686in}{1.715550in}}%
\pgfusepath{stroke}%
\end{pgfscope}%
\begin{pgfscope}%
\pgfsetbuttcap%
\pgfsetmiterjoin%
\definecolor{currentfill}{rgb}{1.000000,1.000000,1.000000}%
\pgfsetfillcolor{currentfill}%
\pgfsetfillopacity{0.800000}%
\pgfsetlinewidth{1.003750pt}%
\definecolor{currentstroke}{rgb}{0.800000,0.800000,0.800000}%
\pgfsetstrokecolor{currentstroke}%
\pgfsetstrokeopacity{0.800000}%
\pgfsetdash{}{0pt}%
\pgfpathmoveto{\pgfqpoint{4.199667in}{0.523973in}}%
\pgfpathlineto{\pgfqpoint{5.696286in}{0.523973in}}%
\pgfpathquadraticcurveto{\pgfqpoint{5.721286in}{0.523973in}}{\pgfqpoint{5.721286in}{0.548973in}}%
\pgfpathlineto{\pgfqpoint{5.721286in}{1.617966in}}%
\pgfpathquadraticcurveto{\pgfqpoint{5.721286in}{1.642966in}}{\pgfqpoint{5.696286in}{1.642966in}}%
\pgfpathlineto{\pgfqpoint{4.199667in}{1.642966in}}%
\pgfpathquadraticcurveto{\pgfqpoint{4.174667in}{1.642966in}}{\pgfqpoint{4.174667in}{1.617966in}}%
\pgfpathlineto{\pgfqpoint{4.174667in}{0.548973in}}%
\pgfpathquadraticcurveto{\pgfqpoint{4.174667in}{0.523973in}}{\pgfqpoint{4.199667in}{0.523973in}}%
\pgfpathlineto{\pgfqpoint{4.199667in}{0.523973in}}%
\pgfpathclose%
\pgfusepath{stroke,fill}%
\end{pgfscope}%
\begin{pgfscope}%
\definecolor{textcolor}{rgb}{0.150000,0.150000,0.150000}%
\pgfsetstrokecolor{textcolor}%
\pgfsetfillcolor{textcolor}%
\pgftext[x=4.744577in,y=1.497995in,left,base]{\color{textcolor}\sffamily\fontsize{9.000000}{10.800000}\selectfont Legend}%
\end{pgfscope}%
\begin{pgfscope}%
\pgfsetroundcap%
\pgfsetroundjoin%
\pgfsetlinewidth{1.505625pt}%
\definecolor{currentstroke}{rgb}{0.003922,0.450980,0.698039}%
\pgfsetstrokecolor{currentstroke}%
\pgfsetdash{}{0pt}%
\pgfpathmoveto{\pgfqpoint{4.224667in}{1.354245in}}%
\pgfpathlineto{\pgfqpoint{4.349667in}{1.354245in}}%
\pgfpathlineto{\pgfqpoint{4.474667in}{1.354245in}}%
\pgfusepath{stroke}%
\end{pgfscope}%
\begin{pgfscope}%
\definecolor{textcolor}{rgb}{0.150000,0.150000,0.150000}%
\pgfsetstrokecolor{textcolor}%
\pgfsetfillcolor{textcolor}%
\pgftext[x=4.574667in,y=1.310495in,left,base]{\color{textcolor}\sffamily\fontsize{9.000000}{10.800000}\selectfont Total time}%
\end{pgfscope}%
\begin{pgfscope}%
\pgfsetroundcap%
\pgfsetroundjoin%
\pgfsetlinewidth{1.505625pt}%
\definecolor{currentstroke}{rgb}{0.870588,0.560784,0.019608}%
\pgfsetstrokecolor{currentstroke}%
\pgfsetdash{}{0pt}%
\pgfpathmoveto{\pgfqpoint{4.224667in}{1.166745in}}%
\pgfpathlineto{\pgfqpoint{4.349667in}{1.166745in}}%
\pgfpathlineto{\pgfqpoint{4.474667in}{1.166745in}}%
\pgfusepath{stroke}%
\end{pgfscope}%
\begin{pgfscope}%
\definecolor{textcolor}{rgb}{0.150000,0.150000,0.150000}%
\pgfsetstrokecolor{textcolor}%
\pgfsetfillcolor{textcolor}%
\pgftext[x=4.574667in,y=1.122995in,left,base]{\color{textcolor}\sffamily\fontsize{9.000000}{10.800000}\selectfont A* routing}%
\end{pgfscope}%
\begin{pgfscope}%
\pgfsetroundcap%
\pgfsetroundjoin%
\pgfsetlinewidth{1.505625pt}%
\definecolor{currentstroke}{rgb}{0.007843,0.619608,0.450980}%
\pgfsetstrokecolor{currentstroke}%
\pgfsetdash{}{0pt}%
\pgfpathmoveto{\pgfqpoint{4.224667in}{0.892234in}}%
\pgfpathlineto{\pgfqpoint{4.349667in}{0.892234in}}%
\pgfpathlineto{\pgfqpoint{4.474667in}{0.892234in}}%
\pgfusepath{stroke}%
\end{pgfscope}%
\begin{pgfscope}%
\definecolor{textcolor}{rgb}{0.150000,0.150000,0.150000}%
\pgfsetstrokecolor{textcolor}%
\pgfsetfillcolor{textcolor}%
\pgftext[x=4.574667in, y=0.935496in, left, base]{\color{textcolor}\sffamily\fontsize{9.000000}{10.800000}\selectfont Connect source \&}%
\end{pgfscope}%
\begin{pgfscope}%
\definecolor{textcolor}{rgb}{0.150000,0.150000,0.150000}%
\pgfsetstrokecolor{textcolor}%
\pgfsetfillcolor{textcolor}%
\pgftext[x=4.574667in, y=0.791502in, left, base]{\color{textcolor}\sffamily\fontsize{9.000000}{10.800000}\selectfont destination vertices}%
\end{pgfscope}%
\begin{pgfscope}%
\pgfsetroundcap%
\pgfsetroundjoin%
\pgfsetlinewidth{1.505625pt}%
\definecolor{currentstroke}{rgb}{0.835294,0.368627,0.000000}%
\pgfsetstrokecolor{currentstroke}%
\pgfsetdash{}{0pt}%
\pgfpathmoveto{\pgfqpoint{4.224667in}{0.647752in}}%
\pgfpathlineto{\pgfqpoint{4.349667in}{0.647752in}}%
\pgfpathlineto{\pgfqpoint{4.474667in}{0.647752in}}%
\pgfusepath{stroke}%
\end{pgfscope}%
\begin{pgfscope}%
\definecolor{textcolor}{rgb}{0.150000,0.150000,0.150000}%
\pgfsetstrokecolor{textcolor}%
\pgfsetfillcolor{textcolor}%
\pgftext[x=4.574667in,y=0.604002in,left,base]{\color{textcolor}\sffamily\fontsize{9.000000}{10.800000}\selectfont Restoring graph}%
\end{pgfscope}%
\begin{pgfscope}%
\pgfsetroundcap%
\pgfsetroundjoin%
\pgfsetlinewidth{1.003750pt}%
\definecolor{currentstroke}{rgb}{0.003922,0.450980,0.698039}%
\pgfsetstrokecolor{currentstroke}%
\pgfsetdash{}{0pt}%
\pgfpathmoveto{\pgfqpoint{0.698516in}{1.655514in}}%
\pgfpathlineto{\pgfqpoint{0.764810in}{1.655514in}}%
\pgfpathlineto{\pgfqpoint{0.830371in}{1.655514in}}%
\pgfpathlineto{\pgfqpoint{0.896229in}{1.655514in}}%
\pgfpathlineto{\pgfqpoint{0.962030in}{1.655514in}}%
\pgfpathlineto{\pgfqpoint{1.027895in}{1.655514in}}%
\pgfpathlineto{\pgfqpoint{1.094646in}{1.655514in}}%
\pgfpathlineto{\pgfqpoint{1.159700in}{1.655514in}}%
\pgfpathlineto{\pgfqpoint{1.225800in}{1.655514in}}%
\pgfpathlineto{\pgfqpoint{1.292257in}{1.655514in}}%
\pgfpathlineto{\pgfqpoint{1.357380in}{1.655514in}}%
\pgfpathlineto{\pgfqpoint{1.423561in}{1.655514in}}%
\pgfpathlineto{\pgfqpoint{1.489655in}{1.655514in}}%
\pgfpathlineto{\pgfqpoint{1.554895in}{1.655514in}}%
\pgfpathlineto{\pgfqpoint{1.620805in}{1.655514in}}%
\pgfpathlineto{\pgfqpoint{1.687876in}{1.655514in}}%
\pgfpathlineto{\pgfqpoint{1.753602in}{1.655514in}}%
\pgfpathlineto{\pgfqpoint{1.819330in}{1.655514in}}%
\pgfpathlineto{\pgfqpoint{1.890627in}{1.655514in}}%
\pgfpathlineto{\pgfqpoint{1.945394in}{1.655514in}}%
\pgfpathlineto{\pgfqpoint{2.015256in}{1.655514in}}%
\pgfpathlineto{\pgfqpoint{2.087913in}{1.655514in}}%
\pgfpathlineto{\pgfqpoint{2.147720in}{1.655514in}}%
\pgfpathlineto{\pgfqpoint{2.207389in}{1.655514in}}%
\pgfpathlineto{\pgfqpoint{2.278072in}{1.655514in}}%
\pgfpathlineto{\pgfqpoint{2.542406in}{1.655514in}}%
\pgfpathlineto{\pgfqpoint{2.681275in}{1.655514in}}%
\pgfpathlineto{\pgfqpoint{2.806369in}{1.655514in}}%
\pgfpathlineto{\pgfqpoint{2.936655in}{1.655514in}}%
\pgfpathlineto{\pgfqpoint{3.075142in}{1.655514in}}%
\pgfpathlineto{\pgfqpoint{3.207397in}{1.655514in}}%
\pgfpathlineto{\pgfqpoint{3.330680in}{1.655514in}}%
\pgfpathlineto{\pgfqpoint{3.463050in}{1.655514in}}%
\pgfpathlineto{\pgfqpoint{3.603200in}{1.655514in}}%
\pgfpathlineto{\pgfqpoint{3.860960in}{1.655514in}}%
\pgfusepath{stroke}%
\end{pgfscope}%
\begin{pgfscope}%
\pgfsetbuttcap%
\pgfsetroundjoin%
\definecolor{currentfill}{rgb}{0.003922,0.450980,0.698039}%
\pgfsetfillcolor{currentfill}%
\pgfsetlinewidth{0.752812pt}%
\definecolor{currentstroke}{rgb}{1.000000,1.000000,1.000000}%
\pgfsetstrokecolor{currentstroke}%
\pgfsetdash{}{0pt}%
\pgfsys@defobject{currentmarker}{\pgfqpoint{-0.034722in}{-0.034722in}}{\pgfqpoint{0.034722in}{0.034722in}}{%
\pgfpathmoveto{\pgfqpoint{0.000000in}{-0.034722in}}%
\pgfpathcurveto{\pgfqpoint{0.009208in}{-0.034722in}}{\pgfqpoint{0.018041in}{-0.031064in}}{\pgfqpoint{0.024552in}{-0.024552in}}%
\pgfpathcurveto{\pgfqpoint{0.031064in}{-0.018041in}}{\pgfqpoint{0.034722in}{-0.009208in}}{\pgfqpoint{0.034722in}{0.000000in}}%
\pgfpathcurveto{\pgfqpoint{0.034722in}{0.009208in}}{\pgfqpoint{0.031064in}{0.018041in}}{\pgfqpoint{0.024552in}{0.024552in}}%
\pgfpathcurveto{\pgfqpoint{0.018041in}{0.031064in}}{\pgfqpoint{0.009208in}{0.034722in}}{\pgfqpoint{0.000000in}{0.034722in}}%
\pgfpathcurveto{\pgfqpoint{-0.009208in}{0.034722in}}{\pgfqpoint{-0.018041in}{0.031064in}}{\pgfqpoint{-0.024552in}{0.024552in}}%
\pgfpathcurveto{\pgfqpoint{-0.031064in}{0.018041in}}{\pgfqpoint{-0.034722in}{0.009208in}}{\pgfqpoint{-0.034722in}{0.000000in}}%
\pgfpathcurveto{\pgfqpoint{-0.034722in}{-0.009208in}}{\pgfqpoint{-0.031064in}{-0.018041in}}{\pgfqpoint{-0.024552in}{-0.024552in}}%
\pgfpathcurveto{\pgfqpoint{-0.018041in}{-0.031064in}}{\pgfqpoint{-0.009208in}{-0.034722in}}{\pgfqpoint{0.000000in}{-0.034722in}}%
\pgfpathlineto{\pgfqpoint{0.000000in}{-0.034722in}}%
\pgfpathclose%
\pgfusepath{stroke,fill}%
}%
\begin{pgfscope}%
\pgfsys@transformshift{0.698516in}{1.655514in}%
\pgfsys@useobject{currentmarker}{}%
\end{pgfscope}%
\begin{pgfscope}%
\pgfsys@transformshift{0.764810in}{1.655514in}%
\pgfsys@useobject{currentmarker}{}%
\end{pgfscope}%
\begin{pgfscope}%
\pgfsys@transformshift{0.830371in}{1.655514in}%
\pgfsys@useobject{currentmarker}{}%
\end{pgfscope}%
\begin{pgfscope}%
\pgfsys@transformshift{0.896229in}{1.655514in}%
\pgfsys@useobject{currentmarker}{}%
\end{pgfscope}%
\begin{pgfscope}%
\pgfsys@transformshift{0.962030in}{1.655514in}%
\pgfsys@useobject{currentmarker}{}%
\end{pgfscope}%
\begin{pgfscope}%
\pgfsys@transformshift{1.027895in}{1.655514in}%
\pgfsys@useobject{currentmarker}{}%
\end{pgfscope}%
\begin{pgfscope}%
\pgfsys@transformshift{1.094646in}{1.655514in}%
\pgfsys@useobject{currentmarker}{}%
\end{pgfscope}%
\begin{pgfscope}%
\pgfsys@transformshift{1.159700in}{1.655514in}%
\pgfsys@useobject{currentmarker}{}%
\end{pgfscope}%
\begin{pgfscope}%
\pgfsys@transformshift{1.225800in}{1.655514in}%
\pgfsys@useobject{currentmarker}{}%
\end{pgfscope}%
\begin{pgfscope}%
\pgfsys@transformshift{1.292257in}{1.655514in}%
\pgfsys@useobject{currentmarker}{}%
\end{pgfscope}%
\begin{pgfscope}%
\pgfsys@transformshift{1.357380in}{1.655514in}%
\pgfsys@useobject{currentmarker}{}%
\end{pgfscope}%
\begin{pgfscope}%
\pgfsys@transformshift{1.423561in}{1.655514in}%
\pgfsys@useobject{currentmarker}{}%
\end{pgfscope}%
\begin{pgfscope}%
\pgfsys@transformshift{1.489655in}{1.655514in}%
\pgfsys@useobject{currentmarker}{}%
\end{pgfscope}%
\begin{pgfscope}%
\pgfsys@transformshift{1.554895in}{1.655514in}%
\pgfsys@useobject{currentmarker}{}%
\end{pgfscope}%
\begin{pgfscope}%
\pgfsys@transformshift{1.620805in}{1.655514in}%
\pgfsys@useobject{currentmarker}{}%
\end{pgfscope}%
\begin{pgfscope}%
\pgfsys@transformshift{1.687876in}{1.655514in}%
\pgfsys@useobject{currentmarker}{}%
\end{pgfscope}%
\begin{pgfscope}%
\pgfsys@transformshift{1.753602in}{1.655514in}%
\pgfsys@useobject{currentmarker}{}%
\end{pgfscope}%
\begin{pgfscope}%
\pgfsys@transformshift{1.819330in}{1.655514in}%
\pgfsys@useobject{currentmarker}{}%
\end{pgfscope}%
\begin{pgfscope}%
\pgfsys@transformshift{1.890627in}{1.655514in}%
\pgfsys@useobject{currentmarker}{}%
\end{pgfscope}%
\begin{pgfscope}%
\pgfsys@transformshift{1.945394in}{1.655514in}%
\pgfsys@useobject{currentmarker}{}%
\end{pgfscope}%
\begin{pgfscope}%
\pgfsys@transformshift{2.015256in}{1.655514in}%
\pgfsys@useobject{currentmarker}{}%
\end{pgfscope}%
\begin{pgfscope}%
\pgfsys@transformshift{2.087913in}{1.655514in}%
\pgfsys@useobject{currentmarker}{}%
\end{pgfscope}%
\begin{pgfscope}%
\pgfsys@transformshift{2.147720in}{1.655514in}%
\pgfsys@useobject{currentmarker}{}%
\end{pgfscope}%
\begin{pgfscope}%
\pgfsys@transformshift{2.207389in}{1.655514in}%
\pgfsys@useobject{currentmarker}{}%
\end{pgfscope}%
\begin{pgfscope}%
\pgfsys@transformshift{2.278072in}{1.655514in}%
\pgfsys@useobject{currentmarker}{}%
\end{pgfscope}%
\begin{pgfscope}%
\pgfsys@transformshift{2.542406in}{1.655514in}%
\pgfsys@useobject{currentmarker}{}%
\end{pgfscope}%
\begin{pgfscope}%
\pgfsys@transformshift{2.681275in}{1.655514in}%
\pgfsys@useobject{currentmarker}{}%
\end{pgfscope}%
\begin{pgfscope}%
\pgfsys@transformshift{2.806369in}{1.655514in}%
\pgfsys@useobject{currentmarker}{}%
\end{pgfscope}%
\begin{pgfscope}%
\pgfsys@transformshift{2.936655in}{1.655514in}%
\pgfsys@useobject{currentmarker}{}%
\end{pgfscope}%
\begin{pgfscope}%
\pgfsys@transformshift{3.075142in}{1.655514in}%
\pgfsys@useobject{currentmarker}{}%
\end{pgfscope}%
\begin{pgfscope}%
\pgfsys@transformshift{3.207397in}{1.655514in}%
\pgfsys@useobject{currentmarker}{}%
\end{pgfscope}%
\begin{pgfscope}%
\pgfsys@transformshift{3.330680in}{1.655514in}%
\pgfsys@useobject{currentmarker}{}%
\end{pgfscope}%
\begin{pgfscope}%
\pgfsys@transformshift{3.463050in}{1.655514in}%
\pgfsys@useobject{currentmarker}{}%
\end{pgfscope}%
\begin{pgfscope}%
\pgfsys@transformshift{3.603200in}{1.655514in}%
\pgfsys@useobject{currentmarker}{}%
\end{pgfscope}%
\begin{pgfscope}%
\pgfsys@transformshift{3.860960in}{1.655514in}%
\pgfsys@useobject{currentmarker}{}%
\end{pgfscope}%
\end{pgfscope}%
\begin{pgfscope}%
\pgfsetroundcap%
\pgfsetroundjoin%
\pgfsetlinewidth{1.003750pt}%
\definecolor{currentstroke}{rgb}{0.870588,0.560784,0.019608}%
\pgfsetstrokecolor{currentstroke}%
\pgfsetdash{}{0pt}%
\pgfpathmoveto{\pgfqpoint{0.698516in}{0.454794in}}%
\pgfpathlineto{\pgfqpoint{0.764810in}{0.473025in}}%
\pgfpathlineto{\pgfqpoint{0.830371in}{0.470568in}}%
\pgfpathlineto{\pgfqpoint{0.896229in}{0.465852in}}%
\pgfpathlineto{\pgfqpoint{0.962030in}{0.490371in}}%
\pgfpathlineto{\pgfqpoint{1.027895in}{0.505473in}}%
\pgfpathlineto{\pgfqpoint{1.094646in}{0.506651in}}%
\pgfpathlineto{\pgfqpoint{1.159700in}{0.474141in}}%
\pgfpathlineto{\pgfqpoint{1.225800in}{0.536196in}}%
\pgfpathlineto{\pgfqpoint{1.292257in}{0.506440in}}%
\pgfpathlineto{\pgfqpoint{1.357380in}{0.529374in}}%
\pgfpathlineto{\pgfqpoint{1.423561in}{0.534069in}}%
\pgfpathlineto{\pgfqpoint{1.489655in}{0.829231in}}%
\pgfpathlineto{\pgfqpoint{1.554895in}{0.748048in}}%
\pgfpathlineto{\pgfqpoint{1.620805in}{0.539823in}}%
\pgfpathlineto{\pgfqpoint{1.687876in}{0.615146in}}%
\pgfpathlineto{\pgfqpoint{1.753602in}{0.840773in}}%
\pgfpathlineto{\pgfqpoint{1.819330in}{0.607762in}}%
\pgfpathlineto{\pgfqpoint{1.890627in}{0.685128in}}%
\pgfpathlineto{\pgfqpoint{1.945394in}{1.031772in}}%
\pgfpathlineto{\pgfqpoint{2.015256in}{0.862555in}}%
\pgfpathlineto{\pgfqpoint{2.087913in}{0.972747in}}%
\pgfpathlineto{\pgfqpoint{2.147720in}{0.710296in}}%
\pgfpathlineto{\pgfqpoint{2.207389in}{0.711988in}}%
\pgfpathlineto{\pgfqpoint{2.278072in}{0.598818in}}%
\pgfpathlineto{\pgfqpoint{2.542406in}{0.855910in}}%
\pgfpathlineto{\pgfqpoint{2.681275in}{0.818129in}}%
\pgfpathlineto{\pgfqpoint{2.806369in}{0.952176in}}%
\pgfpathlineto{\pgfqpoint{2.936655in}{0.887555in}}%
\pgfpathlineto{\pgfqpoint{3.075142in}{0.903915in}}%
\pgfpathlineto{\pgfqpoint{3.207397in}{1.102252in}}%
\pgfpathlineto{\pgfqpoint{3.330680in}{1.091737in}}%
\pgfpathlineto{\pgfqpoint{3.463050in}{1.049525in}}%
\pgfpathlineto{\pgfqpoint{3.603200in}{0.993273in}}%
\pgfpathlineto{\pgfqpoint{3.860960in}{0.960924in}}%
\pgfusepath{stroke}%
\end{pgfscope}%
\begin{pgfscope}%
\pgfsetbuttcap%
\pgfsetroundjoin%
\definecolor{currentfill}{rgb}{0.870588,0.560784,0.019608}%
\pgfsetfillcolor{currentfill}%
\pgfsetlinewidth{0.752812pt}%
\definecolor{currentstroke}{rgb}{1.000000,1.000000,1.000000}%
\pgfsetstrokecolor{currentstroke}%
\pgfsetdash{}{0pt}%
\pgfsys@defobject{currentmarker}{\pgfqpoint{-0.034722in}{-0.034722in}}{\pgfqpoint{0.034722in}{0.034722in}}{%
\pgfpathmoveto{\pgfqpoint{0.000000in}{-0.034722in}}%
\pgfpathcurveto{\pgfqpoint{0.009208in}{-0.034722in}}{\pgfqpoint{0.018041in}{-0.031064in}}{\pgfqpoint{0.024552in}{-0.024552in}}%
\pgfpathcurveto{\pgfqpoint{0.031064in}{-0.018041in}}{\pgfqpoint{0.034722in}{-0.009208in}}{\pgfqpoint{0.034722in}{0.000000in}}%
\pgfpathcurveto{\pgfqpoint{0.034722in}{0.009208in}}{\pgfqpoint{0.031064in}{0.018041in}}{\pgfqpoint{0.024552in}{0.024552in}}%
\pgfpathcurveto{\pgfqpoint{0.018041in}{0.031064in}}{\pgfqpoint{0.009208in}{0.034722in}}{\pgfqpoint{0.000000in}{0.034722in}}%
\pgfpathcurveto{\pgfqpoint{-0.009208in}{0.034722in}}{\pgfqpoint{-0.018041in}{0.031064in}}{\pgfqpoint{-0.024552in}{0.024552in}}%
\pgfpathcurveto{\pgfqpoint{-0.031064in}{0.018041in}}{\pgfqpoint{-0.034722in}{0.009208in}}{\pgfqpoint{-0.034722in}{0.000000in}}%
\pgfpathcurveto{\pgfqpoint{-0.034722in}{-0.009208in}}{\pgfqpoint{-0.031064in}{-0.018041in}}{\pgfqpoint{-0.024552in}{-0.024552in}}%
\pgfpathcurveto{\pgfqpoint{-0.018041in}{-0.031064in}}{\pgfqpoint{-0.009208in}{-0.034722in}}{\pgfqpoint{0.000000in}{-0.034722in}}%
\pgfpathlineto{\pgfqpoint{0.000000in}{-0.034722in}}%
\pgfpathclose%
\pgfusepath{stroke,fill}%
}%
\begin{pgfscope}%
\pgfsys@transformshift{0.698516in}{0.454794in}%
\pgfsys@useobject{currentmarker}{}%
\end{pgfscope}%
\begin{pgfscope}%
\pgfsys@transformshift{0.764810in}{0.473025in}%
\pgfsys@useobject{currentmarker}{}%
\end{pgfscope}%
\begin{pgfscope}%
\pgfsys@transformshift{0.830371in}{0.470568in}%
\pgfsys@useobject{currentmarker}{}%
\end{pgfscope}%
\begin{pgfscope}%
\pgfsys@transformshift{0.896229in}{0.465852in}%
\pgfsys@useobject{currentmarker}{}%
\end{pgfscope}%
\begin{pgfscope}%
\pgfsys@transformshift{0.962030in}{0.490371in}%
\pgfsys@useobject{currentmarker}{}%
\end{pgfscope}%
\begin{pgfscope}%
\pgfsys@transformshift{1.027895in}{0.505473in}%
\pgfsys@useobject{currentmarker}{}%
\end{pgfscope}%
\begin{pgfscope}%
\pgfsys@transformshift{1.094646in}{0.506651in}%
\pgfsys@useobject{currentmarker}{}%
\end{pgfscope}%
\begin{pgfscope}%
\pgfsys@transformshift{1.159700in}{0.474141in}%
\pgfsys@useobject{currentmarker}{}%
\end{pgfscope}%
\begin{pgfscope}%
\pgfsys@transformshift{1.225800in}{0.536196in}%
\pgfsys@useobject{currentmarker}{}%
\end{pgfscope}%
\begin{pgfscope}%
\pgfsys@transformshift{1.292257in}{0.506440in}%
\pgfsys@useobject{currentmarker}{}%
\end{pgfscope}%
\begin{pgfscope}%
\pgfsys@transformshift{1.357380in}{0.529374in}%
\pgfsys@useobject{currentmarker}{}%
\end{pgfscope}%
\begin{pgfscope}%
\pgfsys@transformshift{1.423561in}{0.534069in}%
\pgfsys@useobject{currentmarker}{}%
\end{pgfscope}%
\begin{pgfscope}%
\pgfsys@transformshift{1.489655in}{0.829231in}%
\pgfsys@useobject{currentmarker}{}%
\end{pgfscope}%
\begin{pgfscope}%
\pgfsys@transformshift{1.554895in}{0.748048in}%
\pgfsys@useobject{currentmarker}{}%
\end{pgfscope}%
\begin{pgfscope}%
\pgfsys@transformshift{1.620805in}{0.539823in}%
\pgfsys@useobject{currentmarker}{}%
\end{pgfscope}%
\begin{pgfscope}%
\pgfsys@transformshift{1.687876in}{0.615146in}%
\pgfsys@useobject{currentmarker}{}%
\end{pgfscope}%
\begin{pgfscope}%
\pgfsys@transformshift{1.753602in}{0.840773in}%
\pgfsys@useobject{currentmarker}{}%
\end{pgfscope}%
\begin{pgfscope}%
\pgfsys@transformshift{1.819330in}{0.607762in}%
\pgfsys@useobject{currentmarker}{}%
\end{pgfscope}%
\begin{pgfscope}%
\pgfsys@transformshift{1.890627in}{0.685128in}%
\pgfsys@useobject{currentmarker}{}%
\end{pgfscope}%
\begin{pgfscope}%
\pgfsys@transformshift{1.945394in}{1.031772in}%
\pgfsys@useobject{currentmarker}{}%
\end{pgfscope}%
\begin{pgfscope}%
\pgfsys@transformshift{2.015256in}{0.862555in}%
\pgfsys@useobject{currentmarker}{}%
\end{pgfscope}%
\begin{pgfscope}%
\pgfsys@transformshift{2.087913in}{0.972747in}%
\pgfsys@useobject{currentmarker}{}%
\end{pgfscope}%
\begin{pgfscope}%
\pgfsys@transformshift{2.147720in}{0.710296in}%
\pgfsys@useobject{currentmarker}{}%
\end{pgfscope}%
\begin{pgfscope}%
\pgfsys@transformshift{2.207389in}{0.711988in}%
\pgfsys@useobject{currentmarker}{}%
\end{pgfscope}%
\begin{pgfscope}%
\pgfsys@transformshift{2.278072in}{0.598818in}%
\pgfsys@useobject{currentmarker}{}%
\end{pgfscope}%
\begin{pgfscope}%
\pgfsys@transformshift{2.542406in}{0.855910in}%
\pgfsys@useobject{currentmarker}{}%
\end{pgfscope}%
\begin{pgfscope}%
\pgfsys@transformshift{2.681275in}{0.818129in}%
\pgfsys@useobject{currentmarker}{}%
\end{pgfscope}%
\begin{pgfscope}%
\pgfsys@transformshift{2.806369in}{0.952176in}%
\pgfsys@useobject{currentmarker}{}%
\end{pgfscope}%
\begin{pgfscope}%
\pgfsys@transformshift{2.936655in}{0.887555in}%
\pgfsys@useobject{currentmarker}{}%
\end{pgfscope}%
\begin{pgfscope}%
\pgfsys@transformshift{3.075142in}{0.903915in}%
\pgfsys@useobject{currentmarker}{}%
\end{pgfscope}%
\begin{pgfscope}%
\pgfsys@transformshift{3.207397in}{1.102252in}%
\pgfsys@useobject{currentmarker}{}%
\end{pgfscope}%
\begin{pgfscope}%
\pgfsys@transformshift{3.330680in}{1.091737in}%
\pgfsys@useobject{currentmarker}{}%
\end{pgfscope}%
\begin{pgfscope}%
\pgfsys@transformshift{3.463050in}{1.049525in}%
\pgfsys@useobject{currentmarker}{}%
\end{pgfscope}%
\begin{pgfscope}%
\pgfsys@transformshift{3.603200in}{0.993273in}%
\pgfsys@useobject{currentmarker}{}%
\end{pgfscope}%
\begin{pgfscope}%
\pgfsys@transformshift{3.860960in}{0.960924in}%
\pgfsys@useobject{currentmarker}{}%
\end{pgfscope}%
\end{pgfscope}%
\begin{pgfscope}%
\pgfsetroundcap%
\pgfsetroundjoin%
\pgfsetlinewidth{1.003750pt}%
\definecolor{currentstroke}{rgb}{0.007843,0.619608,0.450980}%
\pgfsetstrokecolor{currentstroke}%
\pgfsetdash{}{0pt}%
\pgfpathmoveto{\pgfqpoint{0.698516in}{1.595097in}}%
\pgfpathlineto{\pgfqpoint{0.764810in}{1.561508in}}%
\pgfpathlineto{\pgfqpoint{0.830371in}{1.564396in}}%
\pgfpathlineto{\pgfqpoint{0.896229in}{1.558440in}}%
\pgfpathlineto{\pgfqpoint{0.962030in}{1.466196in}}%
\pgfpathlineto{\pgfqpoint{1.027895in}{1.477595in}}%
\pgfpathlineto{\pgfqpoint{1.094646in}{1.521755in}}%
\pgfpathlineto{\pgfqpoint{1.159700in}{1.541744in}}%
\pgfpathlineto{\pgfqpoint{1.225800in}{1.463702in}}%
\pgfpathlineto{\pgfqpoint{1.292257in}{1.514131in}}%
\pgfpathlineto{\pgfqpoint{1.357380in}{1.506878in}}%
\pgfpathlineto{\pgfqpoint{1.423561in}{1.487093in}}%
\pgfpathlineto{\pgfqpoint{1.489655in}{1.170585in}}%
\pgfpathlineto{\pgfqpoint{1.554895in}{1.278825in}}%
\pgfpathlineto{\pgfqpoint{1.620805in}{1.485391in}}%
\pgfpathlineto{\pgfqpoint{1.687876in}{1.343074in}}%
\pgfpathlineto{\pgfqpoint{1.753602in}{1.173293in}}%
\pgfpathlineto{\pgfqpoint{1.819330in}{1.390480in}}%
\pgfpathlineto{\pgfqpoint{1.890627in}{1.250429in}}%
\pgfpathlineto{\pgfqpoint{1.945394in}{0.946184in}}%
\pgfpathlineto{\pgfqpoint{2.015256in}{1.100291in}}%
\pgfpathlineto{\pgfqpoint{2.087913in}{1.013688in}}%
\pgfpathlineto{\pgfqpoint{2.147720in}{1.260252in}}%
\pgfpathlineto{\pgfqpoint{2.207389in}{1.321496in}}%
\pgfpathlineto{\pgfqpoint{2.278072in}{1.416024in}}%
\pgfpathlineto{\pgfqpoint{2.542406in}{1.158281in}}%
\pgfpathlineto{\pgfqpoint{2.681275in}{1.193551in}}%
\pgfpathlineto{\pgfqpoint{2.806369in}{1.027010in}}%
\pgfpathlineto{\pgfqpoint{2.936655in}{1.129067in}}%
\pgfpathlineto{\pgfqpoint{3.075142in}{1.105495in}}%
\pgfpathlineto{\pgfqpoint{3.207397in}{0.878188in}}%
\pgfpathlineto{\pgfqpoint{3.330680in}{0.882548in}}%
\pgfpathlineto{\pgfqpoint{3.463050in}{0.940385in}}%
\pgfpathlineto{\pgfqpoint{3.603200in}{1.003149in}}%
\pgfpathlineto{\pgfqpoint{3.860960in}{1.034927in}}%
\pgfusepath{stroke}%
\end{pgfscope}%
\begin{pgfscope}%
\pgfsetbuttcap%
\pgfsetroundjoin%
\definecolor{currentfill}{rgb}{0.007843,0.619608,0.450980}%
\pgfsetfillcolor{currentfill}%
\pgfsetlinewidth{0.752812pt}%
\definecolor{currentstroke}{rgb}{1.000000,1.000000,1.000000}%
\pgfsetstrokecolor{currentstroke}%
\pgfsetdash{}{0pt}%
\pgfsys@defobject{currentmarker}{\pgfqpoint{-0.034722in}{-0.034722in}}{\pgfqpoint{0.034722in}{0.034722in}}{%
\pgfpathmoveto{\pgfqpoint{0.000000in}{-0.034722in}}%
\pgfpathcurveto{\pgfqpoint{0.009208in}{-0.034722in}}{\pgfqpoint{0.018041in}{-0.031064in}}{\pgfqpoint{0.024552in}{-0.024552in}}%
\pgfpathcurveto{\pgfqpoint{0.031064in}{-0.018041in}}{\pgfqpoint{0.034722in}{-0.009208in}}{\pgfqpoint{0.034722in}{0.000000in}}%
\pgfpathcurveto{\pgfqpoint{0.034722in}{0.009208in}}{\pgfqpoint{0.031064in}{0.018041in}}{\pgfqpoint{0.024552in}{0.024552in}}%
\pgfpathcurveto{\pgfqpoint{0.018041in}{0.031064in}}{\pgfqpoint{0.009208in}{0.034722in}}{\pgfqpoint{0.000000in}{0.034722in}}%
\pgfpathcurveto{\pgfqpoint{-0.009208in}{0.034722in}}{\pgfqpoint{-0.018041in}{0.031064in}}{\pgfqpoint{-0.024552in}{0.024552in}}%
\pgfpathcurveto{\pgfqpoint{-0.031064in}{0.018041in}}{\pgfqpoint{-0.034722in}{0.009208in}}{\pgfqpoint{-0.034722in}{0.000000in}}%
\pgfpathcurveto{\pgfqpoint{-0.034722in}{-0.009208in}}{\pgfqpoint{-0.031064in}{-0.018041in}}{\pgfqpoint{-0.024552in}{-0.024552in}}%
\pgfpathcurveto{\pgfqpoint{-0.018041in}{-0.031064in}}{\pgfqpoint{-0.009208in}{-0.034722in}}{\pgfqpoint{0.000000in}{-0.034722in}}%
\pgfpathlineto{\pgfqpoint{0.000000in}{-0.034722in}}%
\pgfpathclose%
\pgfusepath{stroke,fill}%
}%
\begin{pgfscope}%
\pgfsys@transformshift{0.698516in}{1.595097in}%
\pgfsys@useobject{currentmarker}{}%
\end{pgfscope}%
\begin{pgfscope}%
\pgfsys@transformshift{0.764810in}{1.561508in}%
\pgfsys@useobject{currentmarker}{}%
\end{pgfscope}%
\begin{pgfscope}%
\pgfsys@transformshift{0.830371in}{1.564396in}%
\pgfsys@useobject{currentmarker}{}%
\end{pgfscope}%
\begin{pgfscope}%
\pgfsys@transformshift{0.896229in}{1.558440in}%
\pgfsys@useobject{currentmarker}{}%
\end{pgfscope}%
\begin{pgfscope}%
\pgfsys@transformshift{0.962030in}{1.466196in}%
\pgfsys@useobject{currentmarker}{}%
\end{pgfscope}%
\begin{pgfscope}%
\pgfsys@transformshift{1.027895in}{1.477595in}%
\pgfsys@useobject{currentmarker}{}%
\end{pgfscope}%
\begin{pgfscope}%
\pgfsys@transformshift{1.094646in}{1.521755in}%
\pgfsys@useobject{currentmarker}{}%
\end{pgfscope}%
\begin{pgfscope}%
\pgfsys@transformshift{1.159700in}{1.541744in}%
\pgfsys@useobject{currentmarker}{}%
\end{pgfscope}%
\begin{pgfscope}%
\pgfsys@transformshift{1.225800in}{1.463702in}%
\pgfsys@useobject{currentmarker}{}%
\end{pgfscope}%
\begin{pgfscope}%
\pgfsys@transformshift{1.292257in}{1.514131in}%
\pgfsys@useobject{currentmarker}{}%
\end{pgfscope}%
\begin{pgfscope}%
\pgfsys@transformshift{1.357380in}{1.506878in}%
\pgfsys@useobject{currentmarker}{}%
\end{pgfscope}%
\begin{pgfscope}%
\pgfsys@transformshift{1.423561in}{1.487093in}%
\pgfsys@useobject{currentmarker}{}%
\end{pgfscope}%
\begin{pgfscope}%
\pgfsys@transformshift{1.489655in}{1.170585in}%
\pgfsys@useobject{currentmarker}{}%
\end{pgfscope}%
\begin{pgfscope}%
\pgfsys@transformshift{1.554895in}{1.278825in}%
\pgfsys@useobject{currentmarker}{}%
\end{pgfscope}%
\begin{pgfscope}%
\pgfsys@transformshift{1.620805in}{1.485391in}%
\pgfsys@useobject{currentmarker}{}%
\end{pgfscope}%
\begin{pgfscope}%
\pgfsys@transformshift{1.687876in}{1.343074in}%
\pgfsys@useobject{currentmarker}{}%
\end{pgfscope}%
\begin{pgfscope}%
\pgfsys@transformshift{1.753602in}{1.173293in}%
\pgfsys@useobject{currentmarker}{}%
\end{pgfscope}%
\begin{pgfscope}%
\pgfsys@transformshift{1.819330in}{1.390480in}%
\pgfsys@useobject{currentmarker}{}%
\end{pgfscope}%
\begin{pgfscope}%
\pgfsys@transformshift{1.890627in}{1.250429in}%
\pgfsys@useobject{currentmarker}{}%
\end{pgfscope}%
\begin{pgfscope}%
\pgfsys@transformshift{1.945394in}{0.946184in}%
\pgfsys@useobject{currentmarker}{}%
\end{pgfscope}%
\begin{pgfscope}%
\pgfsys@transformshift{2.015256in}{1.100291in}%
\pgfsys@useobject{currentmarker}{}%
\end{pgfscope}%
\begin{pgfscope}%
\pgfsys@transformshift{2.087913in}{1.013688in}%
\pgfsys@useobject{currentmarker}{}%
\end{pgfscope}%
\begin{pgfscope}%
\pgfsys@transformshift{2.147720in}{1.260252in}%
\pgfsys@useobject{currentmarker}{}%
\end{pgfscope}%
\begin{pgfscope}%
\pgfsys@transformshift{2.207389in}{1.321496in}%
\pgfsys@useobject{currentmarker}{}%
\end{pgfscope}%
\begin{pgfscope}%
\pgfsys@transformshift{2.278072in}{1.416024in}%
\pgfsys@useobject{currentmarker}{}%
\end{pgfscope}%
\begin{pgfscope}%
\pgfsys@transformshift{2.542406in}{1.158281in}%
\pgfsys@useobject{currentmarker}{}%
\end{pgfscope}%
\begin{pgfscope}%
\pgfsys@transformshift{2.681275in}{1.193551in}%
\pgfsys@useobject{currentmarker}{}%
\end{pgfscope}%
\begin{pgfscope}%
\pgfsys@transformshift{2.806369in}{1.027010in}%
\pgfsys@useobject{currentmarker}{}%
\end{pgfscope}%
\begin{pgfscope}%
\pgfsys@transformshift{2.936655in}{1.129067in}%
\pgfsys@useobject{currentmarker}{}%
\end{pgfscope}%
\begin{pgfscope}%
\pgfsys@transformshift{3.075142in}{1.105495in}%
\pgfsys@useobject{currentmarker}{}%
\end{pgfscope}%
\begin{pgfscope}%
\pgfsys@transformshift{3.207397in}{0.878188in}%
\pgfsys@useobject{currentmarker}{}%
\end{pgfscope}%
\begin{pgfscope}%
\pgfsys@transformshift{3.330680in}{0.882548in}%
\pgfsys@useobject{currentmarker}{}%
\end{pgfscope}%
\begin{pgfscope}%
\pgfsys@transformshift{3.463050in}{0.940385in}%
\pgfsys@useobject{currentmarker}{}%
\end{pgfscope}%
\begin{pgfscope}%
\pgfsys@transformshift{3.603200in}{1.003149in}%
\pgfsys@useobject{currentmarker}{}%
\end{pgfscope}%
\begin{pgfscope}%
\pgfsys@transformshift{3.860960in}{1.034927in}%
\pgfsys@useobject{currentmarker}{}%
\end{pgfscope}%
\end{pgfscope}%
\begin{pgfscope}%
\pgfsetroundcap%
\pgfsetroundjoin%
\pgfsetlinewidth{1.003750pt}%
\definecolor{currentstroke}{rgb}{0.835294,0.368627,0.000000}%
\pgfsetstrokecolor{currentstroke}%
\pgfsetdash{}{0pt}%
\pgfpathmoveto{\pgfqpoint{0.698516in}{0.508312in}}%
\pgfpathlineto{\pgfqpoint{0.764810in}{0.523625in}}%
\pgfpathlineto{\pgfqpoint{0.830371in}{0.522902in}}%
\pgfpathlineto{\pgfqpoint{0.896229in}{0.533831in}}%
\pgfpathlineto{\pgfqpoint{0.962030in}{0.601410in}}%
\pgfpathlineto{\pgfqpoint{1.027895in}{0.574993in}}%
\pgfpathlineto{\pgfqpoint{1.094646in}{0.529742in}}%
\pgfpathlineto{\pgfqpoint{1.159700in}{0.542001in}}%
\pgfpathlineto{\pgfqpoint{1.225800in}{0.557910in}}%
\pgfpathlineto{\pgfqpoint{1.292257in}{0.536786in}}%
\pgfpathlineto{\pgfqpoint{1.357380in}{0.521909in}}%
\pgfpathlineto{\pgfqpoint{1.423561in}{0.536831in}}%
\pgfpathlineto{\pgfqpoint{1.489655in}{0.558217in}}%
\pgfpathlineto{\pgfqpoint{1.554895in}{0.531238in}}%
\pgfpathlineto{\pgfqpoint{1.620805in}{0.532880in}}%
\pgfpathlineto{\pgfqpoint{1.687876in}{0.599610in}}%
\pgfpathlineto{\pgfqpoint{1.753602in}{0.543882in}}%
\pgfpathlineto{\pgfqpoint{1.819330in}{0.557807in}}%
\pgfpathlineto{\pgfqpoint{1.890627in}{0.622114in}}%
\pgfpathlineto{\pgfqpoint{1.945394in}{0.579724in}}%
\pgfpathlineto{\pgfqpoint{2.015256in}{0.595055in}}%
\pgfpathlineto{\pgfqpoint{2.087913in}{0.568046in}}%
\pgfpathlineto{\pgfqpoint{2.147720in}{0.583776in}}%
\pgfpathlineto{\pgfqpoint{2.207389in}{0.524151in}}%
\pgfpathlineto{\pgfqpoint{2.278072in}{0.542665in}}%
\pgfpathlineto{\pgfqpoint{2.542406in}{0.541297in}}%
\pgfpathlineto{\pgfqpoint{2.681275in}{0.543843in}}%
\pgfpathlineto{\pgfqpoint{2.806369in}{0.577909in}}%
\pgfpathlineto{\pgfqpoint{2.936655in}{0.539108in}}%
\pgfpathlineto{\pgfqpoint{3.075142in}{0.545798in}}%
\pgfpathlineto{\pgfqpoint{3.207397in}{0.576773in}}%
\pgfpathlineto{\pgfqpoint{3.330680in}{0.574398in}}%
\pgfpathlineto{\pgfqpoint{3.463050in}{0.567425in}}%
\pgfpathlineto{\pgfqpoint{3.603200in}{0.560885in}}%
\pgfpathlineto{\pgfqpoint{3.860960in}{0.555961in}}%
\pgfusepath{stroke}%
\end{pgfscope}%
\begin{pgfscope}%
\pgfsetbuttcap%
\pgfsetroundjoin%
\definecolor{currentfill}{rgb}{0.835294,0.368627,0.000000}%
\pgfsetfillcolor{currentfill}%
\pgfsetlinewidth{0.752812pt}%
\definecolor{currentstroke}{rgb}{1.000000,1.000000,1.000000}%
\pgfsetstrokecolor{currentstroke}%
\pgfsetdash{}{0pt}%
\pgfsys@defobject{currentmarker}{\pgfqpoint{-0.034722in}{-0.034722in}}{\pgfqpoint{0.034722in}{0.034722in}}{%
\pgfpathmoveto{\pgfqpoint{0.000000in}{-0.034722in}}%
\pgfpathcurveto{\pgfqpoint{0.009208in}{-0.034722in}}{\pgfqpoint{0.018041in}{-0.031064in}}{\pgfqpoint{0.024552in}{-0.024552in}}%
\pgfpathcurveto{\pgfqpoint{0.031064in}{-0.018041in}}{\pgfqpoint{0.034722in}{-0.009208in}}{\pgfqpoint{0.034722in}{0.000000in}}%
\pgfpathcurveto{\pgfqpoint{0.034722in}{0.009208in}}{\pgfqpoint{0.031064in}{0.018041in}}{\pgfqpoint{0.024552in}{0.024552in}}%
\pgfpathcurveto{\pgfqpoint{0.018041in}{0.031064in}}{\pgfqpoint{0.009208in}{0.034722in}}{\pgfqpoint{0.000000in}{0.034722in}}%
\pgfpathcurveto{\pgfqpoint{-0.009208in}{0.034722in}}{\pgfqpoint{-0.018041in}{0.031064in}}{\pgfqpoint{-0.024552in}{0.024552in}}%
\pgfpathcurveto{\pgfqpoint{-0.031064in}{0.018041in}}{\pgfqpoint{-0.034722in}{0.009208in}}{\pgfqpoint{-0.034722in}{0.000000in}}%
\pgfpathcurveto{\pgfqpoint{-0.034722in}{-0.009208in}}{\pgfqpoint{-0.031064in}{-0.018041in}}{\pgfqpoint{-0.024552in}{-0.024552in}}%
\pgfpathcurveto{\pgfqpoint{-0.018041in}{-0.031064in}}{\pgfqpoint{-0.009208in}{-0.034722in}}{\pgfqpoint{0.000000in}{-0.034722in}}%
\pgfpathlineto{\pgfqpoint{0.000000in}{-0.034722in}}%
\pgfpathclose%
\pgfusepath{stroke,fill}%
}%
\begin{pgfscope}%
\pgfsys@transformshift{0.698516in}{0.508312in}%
\pgfsys@useobject{currentmarker}{}%
\end{pgfscope}%
\begin{pgfscope}%
\pgfsys@transformshift{0.764810in}{0.523625in}%
\pgfsys@useobject{currentmarker}{}%
\end{pgfscope}%
\begin{pgfscope}%
\pgfsys@transformshift{0.830371in}{0.522902in}%
\pgfsys@useobject{currentmarker}{}%
\end{pgfscope}%
\begin{pgfscope}%
\pgfsys@transformshift{0.896229in}{0.533831in}%
\pgfsys@useobject{currentmarker}{}%
\end{pgfscope}%
\begin{pgfscope}%
\pgfsys@transformshift{0.962030in}{0.601410in}%
\pgfsys@useobject{currentmarker}{}%
\end{pgfscope}%
\begin{pgfscope}%
\pgfsys@transformshift{1.027895in}{0.574993in}%
\pgfsys@useobject{currentmarker}{}%
\end{pgfscope}%
\begin{pgfscope}%
\pgfsys@transformshift{1.094646in}{0.529742in}%
\pgfsys@useobject{currentmarker}{}%
\end{pgfscope}%
\begin{pgfscope}%
\pgfsys@transformshift{1.159700in}{0.542001in}%
\pgfsys@useobject{currentmarker}{}%
\end{pgfscope}%
\begin{pgfscope}%
\pgfsys@transformshift{1.225800in}{0.557910in}%
\pgfsys@useobject{currentmarker}{}%
\end{pgfscope}%
\begin{pgfscope}%
\pgfsys@transformshift{1.292257in}{0.536786in}%
\pgfsys@useobject{currentmarker}{}%
\end{pgfscope}%
\begin{pgfscope}%
\pgfsys@transformshift{1.357380in}{0.521909in}%
\pgfsys@useobject{currentmarker}{}%
\end{pgfscope}%
\begin{pgfscope}%
\pgfsys@transformshift{1.423561in}{0.536831in}%
\pgfsys@useobject{currentmarker}{}%
\end{pgfscope}%
\begin{pgfscope}%
\pgfsys@transformshift{1.489655in}{0.558217in}%
\pgfsys@useobject{currentmarker}{}%
\end{pgfscope}%
\begin{pgfscope}%
\pgfsys@transformshift{1.554895in}{0.531238in}%
\pgfsys@useobject{currentmarker}{}%
\end{pgfscope}%
\begin{pgfscope}%
\pgfsys@transformshift{1.620805in}{0.532880in}%
\pgfsys@useobject{currentmarker}{}%
\end{pgfscope}%
\begin{pgfscope}%
\pgfsys@transformshift{1.687876in}{0.599610in}%
\pgfsys@useobject{currentmarker}{}%
\end{pgfscope}%
\begin{pgfscope}%
\pgfsys@transformshift{1.753602in}{0.543882in}%
\pgfsys@useobject{currentmarker}{}%
\end{pgfscope}%
\begin{pgfscope}%
\pgfsys@transformshift{1.819330in}{0.557807in}%
\pgfsys@useobject{currentmarker}{}%
\end{pgfscope}%
\begin{pgfscope}%
\pgfsys@transformshift{1.890627in}{0.622114in}%
\pgfsys@useobject{currentmarker}{}%
\end{pgfscope}%
\begin{pgfscope}%
\pgfsys@transformshift{1.945394in}{0.579724in}%
\pgfsys@useobject{currentmarker}{}%
\end{pgfscope}%
\begin{pgfscope}%
\pgfsys@transformshift{2.015256in}{0.595055in}%
\pgfsys@useobject{currentmarker}{}%
\end{pgfscope}%
\begin{pgfscope}%
\pgfsys@transformshift{2.087913in}{0.568046in}%
\pgfsys@useobject{currentmarker}{}%
\end{pgfscope}%
\begin{pgfscope}%
\pgfsys@transformshift{2.147720in}{0.583776in}%
\pgfsys@useobject{currentmarker}{}%
\end{pgfscope}%
\begin{pgfscope}%
\pgfsys@transformshift{2.207389in}{0.524151in}%
\pgfsys@useobject{currentmarker}{}%
\end{pgfscope}%
\begin{pgfscope}%
\pgfsys@transformshift{2.278072in}{0.542665in}%
\pgfsys@useobject{currentmarker}{}%
\end{pgfscope}%
\begin{pgfscope}%
\pgfsys@transformshift{2.542406in}{0.541297in}%
\pgfsys@useobject{currentmarker}{}%
\end{pgfscope}%
\begin{pgfscope}%
\pgfsys@transformshift{2.681275in}{0.543843in}%
\pgfsys@useobject{currentmarker}{}%
\end{pgfscope}%
\begin{pgfscope}%
\pgfsys@transformshift{2.806369in}{0.577909in}%
\pgfsys@useobject{currentmarker}{}%
\end{pgfscope}%
\begin{pgfscope}%
\pgfsys@transformshift{2.936655in}{0.539108in}%
\pgfsys@useobject{currentmarker}{}%
\end{pgfscope}%
\begin{pgfscope}%
\pgfsys@transformshift{3.075142in}{0.545798in}%
\pgfsys@useobject{currentmarker}{}%
\end{pgfscope}%
\begin{pgfscope}%
\pgfsys@transformshift{3.207397in}{0.576773in}%
\pgfsys@useobject{currentmarker}{}%
\end{pgfscope}%
\begin{pgfscope}%
\pgfsys@transformshift{3.330680in}{0.574398in}%
\pgfsys@useobject{currentmarker}{}%
\end{pgfscope}%
\begin{pgfscope}%
\pgfsys@transformshift{3.463050in}{0.567425in}%
\pgfsys@useobject{currentmarker}{}%
\end{pgfscope}%
\begin{pgfscope}%
\pgfsys@transformshift{3.603200in}{0.560885in}%
\pgfsys@useobject{currentmarker}{}%
\end{pgfscope}%
\begin{pgfscope}%
\pgfsys@transformshift{3.860960in}{0.555961in}%
\pgfsys@useobject{currentmarker}{}%
\end{pgfscope}%
\end{pgfscope}%
\end{pgfpicture}%
\makeatother%
\endgroup%

						\end{figcenter}
						\caption{Same as \Cref{fig:eval-rural-routing-details-b} but showing the relative shares on the total time.}
					\end{subfigure}
					\\[3ex]
					\begin{subfigure}[t]{\textwidth}
						\begin{figcenter}
							%% Creator: Matplotlib, PGF backend
%%
%% To include the figure in your LaTeX document, write
%%   \input{<filename>.pgf}
%%
%% Make sure the required packages are loaded in your preamble
%%   \usepackage{pgf}
%%
%% Also ensure that all the required font packages are loaded; for instance,
%% the lmodern package is sometimes necessary when using math font.
%%   \usepackage{lmodern}
%%
%% Figures using additional raster images can only be included by \input if
%% they are in the same directory as the main LaTeX file. For loading figures
%% from other directories you can use the `import` package
%%   \usepackage{import}
%%
%% and then include the figures with
%%   \import{<path to file>}{<filename>.pgf}
%%
%% Matplotlib used the following preamble
%%   
%%   \usepackage{fontspec}
%%   \setmainfont{DejaVuSerif.ttf}[Path=\detokenize{/home/hauke/.local/lib/python3.11/site-packages/matplotlib/mpl-data/fonts/ttf/}]
%%   \setsansfont{DroidSans.ttf}[Path=\detokenize{/usr/share/fonts/droid/}]
%%   \setmonofont{DejaVuSansMono.ttf}[Path=\detokenize{/home/hauke/.local/lib/python3.11/site-packages/matplotlib/mpl-data/fonts/ttf/}]
%%   \makeatletter\@ifpackageloaded{underscore}{}{\usepackage[strings]{underscore}}\makeatother
%%
\begingroup%
\makeatletter%
\begin{pgfpicture}%
\pgfpathrectangle{\pgfpointorigin}{\pgfqpoint{5.719729in}{1.717241in}}%
\pgfusepath{use as bounding box, clip}%
\begin{pgfscope}%
\pgfsetbuttcap%
\pgfsetmiterjoin%
\definecolor{currentfill}{rgb}{1.000000,1.000000,1.000000}%
\pgfsetfillcolor{currentfill}%
\pgfsetlinewidth{0.000000pt}%
\definecolor{currentstroke}{rgb}{1.000000,1.000000,1.000000}%
\pgfsetstrokecolor{currentstroke}%
\pgfsetdash{}{0pt}%
\pgfpathmoveto{\pgfqpoint{0.000000in}{0.000000in}}%
\pgfpathlineto{\pgfqpoint{5.719729in}{0.000000in}}%
\pgfpathlineto{\pgfqpoint{5.719729in}{1.717241in}}%
\pgfpathlineto{\pgfqpoint{0.000000in}{1.717241in}}%
\pgfpathlineto{\pgfqpoint{0.000000in}{0.000000in}}%
\pgfpathclose%
\pgfusepath{fill}%
\end{pgfscope}%
\begin{pgfscope}%
\pgfsetbuttcap%
\pgfsetmiterjoin%
\definecolor{currentfill}{rgb}{1.000000,1.000000,1.000000}%
\pgfsetfillcolor{currentfill}%
\pgfsetlinewidth{0.000000pt}%
\definecolor{currentstroke}{rgb}{0.000000,0.000000,0.000000}%
\pgfsetstrokecolor{currentstroke}%
\pgfsetstrokeopacity{0.000000}%
\pgfsetdash{}{0pt}%
\pgfpathmoveto{\pgfqpoint{0.532932in}{0.451389in}}%
\pgfpathlineto{\pgfqpoint{4.023350in}{0.451389in}}%
\pgfpathlineto{\pgfqpoint{4.023350in}{1.682563in}}%
\pgfpathlineto{\pgfqpoint{0.532932in}{1.682563in}}%
\pgfpathlineto{\pgfqpoint{0.532932in}{0.451389in}}%
\pgfpathclose%
\pgfusepath{fill}%
\end{pgfscope}%
\begin{pgfscope}%
\pgfpathrectangle{\pgfqpoint{0.532932in}{0.451389in}}{\pgfqpoint{3.490418in}{1.231174in}}%
\pgfusepath{clip}%
\pgfsetroundcap%
\pgfsetroundjoin%
\pgfsetlinewidth{1.003750pt}%
\definecolor{currentstroke}{rgb}{0.800000,0.800000,0.800000}%
\pgfsetstrokecolor{currentstroke}%
\pgfsetdash{}{0pt}%
\pgfpathmoveto{\pgfqpoint{0.532932in}{0.451389in}}%
\pgfpathlineto{\pgfqpoint{0.532932in}{1.682563in}}%
\pgfusepath{stroke}%
\end{pgfscope}%
\begin{pgfscope}%
\definecolor{textcolor}{rgb}{0.150000,0.150000,0.150000}%
\pgfsetstrokecolor{textcolor}%
\pgfsetfillcolor{textcolor}%
\pgftext[x=0.532932in,y=0.319444in,,top]{\color{textcolor}\sffamily\fontsize{9.000000}{10.800000}\selectfont 0}%
\end{pgfscope}%
\begin{pgfscope}%
\pgfpathrectangle{\pgfqpoint{0.532932in}{0.451389in}}{\pgfqpoint{3.490418in}{1.231174in}}%
\pgfusepath{clip}%
\pgfsetroundcap%
\pgfsetroundjoin%
\pgfsetlinewidth{1.003750pt}%
\definecolor{currentstroke}{rgb}{0.800000,0.800000,0.800000}%
\pgfsetstrokecolor{currentstroke}%
\pgfsetdash{}{0pt}%
\pgfpathmoveto{\pgfqpoint{1.083536in}{0.451389in}}%
\pgfpathlineto{\pgfqpoint{1.083536in}{1.682563in}}%
\pgfusepath{stroke}%
\end{pgfscope}%
\begin{pgfscope}%
\definecolor{textcolor}{rgb}{0.150000,0.150000,0.150000}%
\pgfsetstrokecolor{textcolor}%
\pgfsetfillcolor{textcolor}%
\pgftext[x=1.083536in,y=0.319444in,,top]{\color{textcolor}\sffamily\fontsize{9.000000}{10.800000}\selectfont 1000}%
\end{pgfscope}%
\begin{pgfscope}%
\pgfpathrectangle{\pgfqpoint{0.532932in}{0.451389in}}{\pgfqpoint{3.490418in}{1.231174in}}%
\pgfusepath{clip}%
\pgfsetroundcap%
\pgfsetroundjoin%
\pgfsetlinewidth{1.003750pt}%
\definecolor{currentstroke}{rgb}{0.800000,0.800000,0.800000}%
\pgfsetstrokecolor{currentstroke}%
\pgfsetdash{}{0pt}%
\pgfpathmoveto{\pgfqpoint{1.634140in}{0.451389in}}%
\pgfpathlineto{\pgfqpoint{1.634140in}{1.682563in}}%
\pgfusepath{stroke}%
\end{pgfscope}%
\begin{pgfscope}%
\definecolor{textcolor}{rgb}{0.150000,0.150000,0.150000}%
\pgfsetstrokecolor{textcolor}%
\pgfsetfillcolor{textcolor}%
\pgftext[x=1.634140in,y=0.319444in,,top]{\color{textcolor}\sffamily\fontsize{9.000000}{10.800000}\selectfont 2000}%
\end{pgfscope}%
\begin{pgfscope}%
\pgfpathrectangle{\pgfqpoint{0.532932in}{0.451389in}}{\pgfqpoint{3.490418in}{1.231174in}}%
\pgfusepath{clip}%
\pgfsetroundcap%
\pgfsetroundjoin%
\pgfsetlinewidth{1.003750pt}%
\definecolor{currentstroke}{rgb}{0.800000,0.800000,0.800000}%
\pgfsetstrokecolor{currentstroke}%
\pgfsetdash{}{0pt}%
\pgfpathmoveto{\pgfqpoint{2.184745in}{0.451389in}}%
\pgfpathlineto{\pgfqpoint{2.184745in}{1.682563in}}%
\pgfusepath{stroke}%
\end{pgfscope}%
\begin{pgfscope}%
\definecolor{textcolor}{rgb}{0.150000,0.150000,0.150000}%
\pgfsetstrokecolor{textcolor}%
\pgfsetfillcolor{textcolor}%
\pgftext[x=2.184745in,y=0.319444in,,top]{\color{textcolor}\sffamily\fontsize{9.000000}{10.800000}\selectfont 3000}%
\end{pgfscope}%
\begin{pgfscope}%
\pgfpathrectangle{\pgfqpoint{0.532932in}{0.451389in}}{\pgfqpoint{3.490418in}{1.231174in}}%
\pgfusepath{clip}%
\pgfsetroundcap%
\pgfsetroundjoin%
\pgfsetlinewidth{1.003750pt}%
\definecolor{currentstroke}{rgb}{0.800000,0.800000,0.800000}%
\pgfsetstrokecolor{currentstroke}%
\pgfsetdash{}{0pt}%
\pgfpathmoveto{\pgfqpoint{2.735349in}{0.451389in}}%
\pgfpathlineto{\pgfqpoint{2.735349in}{1.682563in}}%
\pgfusepath{stroke}%
\end{pgfscope}%
\begin{pgfscope}%
\definecolor{textcolor}{rgb}{0.150000,0.150000,0.150000}%
\pgfsetstrokecolor{textcolor}%
\pgfsetfillcolor{textcolor}%
\pgftext[x=2.735349in,y=0.319444in,,top]{\color{textcolor}\sffamily\fontsize{9.000000}{10.800000}\selectfont 4000}%
\end{pgfscope}%
\begin{pgfscope}%
\pgfpathrectangle{\pgfqpoint{0.532932in}{0.451389in}}{\pgfqpoint{3.490418in}{1.231174in}}%
\pgfusepath{clip}%
\pgfsetroundcap%
\pgfsetroundjoin%
\pgfsetlinewidth{1.003750pt}%
\definecolor{currentstroke}{rgb}{0.800000,0.800000,0.800000}%
\pgfsetstrokecolor{currentstroke}%
\pgfsetdash{}{0pt}%
\pgfpathmoveto{\pgfqpoint{3.285953in}{0.451389in}}%
\pgfpathlineto{\pgfqpoint{3.285953in}{1.682563in}}%
\pgfusepath{stroke}%
\end{pgfscope}%
\begin{pgfscope}%
\definecolor{textcolor}{rgb}{0.150000,0.150000,0.150000}%
\pgfsetstrokecolor{textcolor}%
\pgfsetfillcolor{textcolor}%
\pgftext[x=3.285953in,y=0.319444in,,top]{\color{textcolor}\sffamily\fontsize{9.000000}{10.800000}\selectfont 5000}%
\end{pgfscope}%
\begin{pgfscope}%
\pgfpathrectangle{\pgfqpoint{0.532932in}{0.451389in}}{\pgfqpoint{3.490418in}{1.231174in}}%
\pgfusepath{clip}%
\pgfsetroundcap%
\pgfsetroundjoin%
\pgfsetlinewidth{1.003750pt}%
\definecolor{currentstroke}{rgb}{0.800000,0.800000,0.800000}%
\pgfsetstrokecolor{currentstroke}%
\pgfsetdash{}{0pt}%
\pgfpathmoveto{\pgfqpoint{3.836557in}{0.451389in}}%
\pgfpathlineto{\pgfqpoint{3.836557in}{1.682563in}}%
\pgfusepath{stroke}%
\end{pgfscope}%
\begin{pgfscope}%
\definecolor{textcolor}{rgb}{0.150000,0.150000,0.150000}%
\pgfsetstrokecolor{textcolor}%
\pgfsetfillcolor{textcolor}%
\pgftext[x=3.836557in,y=0.319444in,,top]{\color{textcolor}\sffamily\fontsize{9.000000}{10.800000}\selectfont 6000}%
\end{pgfscope}%
\begin{pgfscope}%
\definecolor{textcolor}{rgb}{0.150000,0.150000,0.150000}%
\pgfsetstrokecolor{textcolor}%
\pgfsetfillcolor{textcolor}%
\pgftext[x=2.278141in,y=0.125000in,,top]{\color{textcolor}\sffamily\fontsize{9.000000}{10.800000}\selectfont Input obstacle vertices}%
\end{pgfscope}%
\begin{pgfscope}%
\pgfpathrectangle{\pgfqpoint{0.532932in}{0.451389in}}{\pgfqpoint{3.490418in}{1.231174in}}%
\pgfusepath{clip}%
\pgfsetroundcap%
\pgfsetroundjoin%
\pgfsetlinewidth{1.003750pt}%
\definecolor{currentstroke}{rgb}{0.800000,0.800000,0.800000}%
\pgfsetstrokecolor{currentstroke}%
\pgfsetdash{}{0pt}%
\pgfpathmoveto{\pgfqpoint{0.532932in}{0.451389in}}%
\pgfpathlineto{\pgfqpoint{4.023350in}{0.451389in}}%
\pgfusepath{stroke}%
\end{pgfscope}%
\begin{pgfscope}%
\definecolor{textcolor}{rgb}{0.150000,0.150000,0.150000}%
\pgfsetstrokecolor{textcolor}%
\pgfsetfillcolor{textcolor}%
\pgftext[x=0.332140in, y=0.403903in, left, base]{\color{textcolor}\sffamily\fontsize{9.000000}{10.800000}\selectfont 0}%
\end{pgfscope}%
\begin{pgfscope}%
\pgfpathrectangle{\pgfqpoint{0.532932in}{0.451389in}}{\pgfqpoint{3.490418in}{1.231174in}}%
\pgfusepath{clip}%
\pgfsetroundcap%
\pgfsetroundjoin%
\pgfsetlinewidth{1.003750pt}%
\definecolor{currentstroke}{rgb}{0.800000,0.800000,0.800000}%
\pgfsetstrokecolor{currentstroke}%
\pgfsetdash{}{0pt}%
\pgfpathmoveto{\pgfqpoint{0.532932in}{0.857511in}}%
\pgfpathlineto{\pgfqpoint{4.023350in}{0.857511in}}%
\pgfusepath{stroke}%
\end{pgfscope}%
\begin{pgfscope}%
\definecolor{textcolor}{rgb}{0.150000,0.150000,0.150000}%
\pgfsetstrokecolor{textcolor}%
\pgfsetfillcolor{textcolor}%
\pgftext[x=0.194444in, y=0.810026in, left, base]{\color{textcolor}\sffamily\fontsize{9.000000}{10.800000}\selectfont 200}%
\end{pgfscope}%
\begin{pgfscope}%
\pgfpathrectangle{\pgfqpoint{0.532932in}{0.451389in}}{\pgfqpoint{3.490418in}{1.231174in}}%
\pgfusepath{clip}%
\pgfsetroundcap%
\pgfsetroundjoin%
\pgfsetlinewidth{1.003750pt}%
\definecolor{currentstroke}{rgb}{0.800000,0.800000,0.800000}%
\pgfsetstrokecolor{currentstroke}%
\pgfsetdash{}{0pt}%
\pgfpathmoveto{\pgfqpoint{0.532932in}{1.263633in}}%
\pgfpathlineto{\pgfqpoint{4.023350in}{1.263633in}}%
\pgfusepath{stroke}%
\end{pgfscope}%
\begin{pgfscope}%
\definecolor{textcolor}{rgb}{0.150000,0.150000,0.150000}%
\pgfsetstrokecolor{textcolor}%
\pgfsetfillcolor{textcolor}%
\pgftext[x=0.194444in, y=1.216148in, left, base]{\color{textcolor}\sffamily\fontsize{9.000000}{10.800000}\selectfont 400}%
\end{pgfscope}%
\begin{pgfscope}%
\pgfpathrectangle{\pgfqpoint{0.532932in}{0.451389in}}{\pgfqpoint{3.490418in}{1.231174in}}%
\pgfusepath{clip}%
\pgfsetroundcap%
\pgfsetroundjoin%
\pgfsetlinewidth{1.003750pt}%
\definecolor{currentstroke}{rgb}{0.800000,0.800000,0.800000}%
\pgfsetstrokecolor{currentstroke}%
\pgfsetdash{}{0pt}%
\pgfpathmoveto{\pgfqpoint{0.532932in}{1.669756in}}%
\pgfpathlineto{\pgfqpoint{4.023350in}{1.669756in}}%
\pgfusepath{stroke}%
\end{pgfscope}%
\begin{pgfscope}%
\definecolor{textcolor}{rgb}{0.150000,0.150000,0.150000}%
\pgfsetstrokecolor{textcolor}%
\pgfsetfillcolor{textcolor}%
\pgftext[x=0.194444in, y=1.622270in, left, base]{\color{textcolor}\sffamily\fontsize{9.000000}{10.800000}\selectfont 600}%
\end{pgfscope}%
\begin{pgfscope}%
\definecolor{textcolor}{rgb}{0.150000,0.150000,0.150000}%
\pgfsetstrokecolor{textcolor}%
\pgfsetfillcolor{textcolor}%
\pgftext[x=0.125000in,y=1.066976in,,bottom,rotate=90.000000]{\color{textcolor}\sffamily\fontsize{9.000000}{10.800000}\selectfont Time in ms}%
\end{pgfscope}%
\begin{pgfscope}%
\pgfsetrectcap%
\pgfsetmiterjoin%
\pgfsetlinewidth{1.254687pt}%
\definecolor{currentstroke}{rgb}{0.800000,0.800000,0.800000}%
\pgfsetstrokecolor{currentstroke}%
\pgfsetdash{}{0pt}%
\pgfpathmoveto{\pgfqpoint{0.532932in}{0.451389in}}%
\pgfpathlineto{\pgfqpoint{0.532932in}{1.682563in}}%
\pgfusepath{stroke}%
\end{pgfscope}%
\begin{pgfscope}%
\pgfsetrectcap%
\pgfsetmiterjoin%
\pgfsetlinewidth{1.254687pt}%
\definecolor{currentstroke}{rgb}{0.800000,0.800000,0.800000}%
\pgfsetstrokecolor{currentstroke}%
\pgfsetdash{}{0pt}%
\pgfpathmoveto{\pgfqpoint{4.023350in}{0.451389in}}%
\pgfpathlineto{\pgfqpoint{4.023350in}{1.682563in}}%
\pgfusepath{stroke}%
\end{pgfscope}%
\begin{pgfscope}%
\pgfsetrectcap%
\pgfsetmiterjoin%
\pgfsetlinewidth{1.254687pt}%
\definecolor{currentstroke}{rgb}{0.800000,0.800000,0.800000}%
\pgfsetstrokecolor{currentstroke}%
\pgfsetdash{}{0pt}%
\pgfpathmoveto{\pgfqpoint{0.532932in}{0.451389in}}%
\pgfpathlineto{\pgfqpoint{4.023350in}{0.451389in}}%
\pgfusepath{stroke}%
\end{pgfscope}%
\begin{pgfscope}%
\pgfsetrectcap%
\pgfsetmiterjoin%
\pgfsetlinewidth{1.254687pt}%
\definecolor{currentstroke}{rgb}{0.800000,0.800000,0.800000}%
\pgfsetstrokecolor{currentstroke}%
\pgfsetdash{}{0pt}%
\pgfpathmoveto{\pgfqpoint{0.532932in}{1.682563in}}%
\pgfpathlineto{\pgfqpoint{4.023350in}{1.682563in}}%
\pgfusepath{stroke}%
\end{pgfscope}%
\begin{pgfscope}%
\pgfsetbuttcap%
\pgfsetmiterjoin%
\definecolor{currentfill}{rgb}{1.000000,1.000000,1.000000}%
\pgfsetfillcolor{currentfill}%
\pgfsetfillopacity{0.800000}%
\pgfsetlinewidth{1.003750pt}%
\definecolor{currentstroke}{rgb}{0.800000,0.800000,0.800000}%
\pgfsetstrokecolor{currentstroke}%
\pgfsetstrokeopacity{0.800000}%
\pgfsetdash{}{0pt}%
\pgfpathmoveto{\pgfqpoint{4.198110in}{0.507479in}}%
\pgfpathlineto{\pgfqpoint{5.694729in}{0.507479in}}%
\pgfpathquadraticcurveto{\pgfqpoint{5.719729in}{0.507479in}}{\pgfqpoint{5.719729in}{0.532479in}}%
\pgfpathlineto{\pgfqpoint{5.719729in}{1.601472in}}%
\pgfpathquadraticcurveto{\pgfqpoint{5.719729in}{1.626472in}}{\pgfqpoint{5.694729in}{1.626472in}}%
\pgfpathlineto{\pgfqpoint{4.198110in}{1.626472in}}%
\pgfpathquadraticcurveto{\pgfqpoint{4.173110in}{1.626472in}}{\pgfqpoint{4.173110in}{1.601472in}}%
\pgfpathlineto{\pgfqpoint{4.173110in}{0.532479in}}%
\pgfpathquadraticcurveto{\pgfqpoint{4.173110in}{0.507479in}}{\pgfqpoint{4.198110in}{0.507479in}}%
\pgfpathlineto{\pgfqpoint{4.198110in}{0.507479in}}%
\pgfpathclose%
\pgfusepath{stroke,fill}%
\end{pgfscope}%
\begin{pgfscope}%
\definecolor{textcolor}{rgb}{0.150000,0.150000,0.150000}%
\pgfsetstrokecolor{textcolor}%
\pgfsetfillcolor{textcolor}%
\pgftext[x=4.743020in,y=1.481502in,left,base]{\color{textcolor}\sffamily\fontsize{9.000000}{10.800000}\selectfont Legend}%
\end{pgfscope}%
\begin{pgfscope}%
\pgfsetroundcap%
\pgfsetroundjoin%
\pgfsetlinewidth{1.505625pt}%
\definecolor{currentstroke}{rgb}{0.003922,0.450980,0.698039}%
\pgfsetstrokecolor{currentstroke}%
\pgfsetdash{}{0pt}%
\pgfpathmoveto{\pgfqpoint{4.223110in}{1.337752in}}%
\pgfpathlineto{\pgfqpoint{4.348110in}{1.337752in}}%
\pgfpathlineto{\pgfqpoint{4.473110in}{1.337752in}}%
\pgfusepath{stroke}%
\end{pgfscope}%
\begin{pgfscope}%
\definecolor{textcolor}{rgb}{0.150000,0.150000,0.150000}%
\pgfsetstrokecolor{textcolor}%
\pgfsetfillcolor{textcolor}%
\pgftext[x=4.573110in,y=1.294002in,left,base]{\color{textcolor}\sffamily\fontsize{9.000000}{10.800000}\selectfont Total time}%
\end{pgfscope}%
\begin{pgfscope}%
\pgfsetroundcap%
\pgfsetroundjoin%
\pgfsetlinewidth{1.505625pt}%
\definecolor{currentstroke}{rgb}{0.870588,0.560784,0.019608}%
\pgfsetstrokecolor{currentstroke}%
\pgfsetdash{}{0pt}%
\pgfpathmoveto{\pgfqpoint{4.223110in}{1.150252in}}%
\pgfpathlineto{\pgfqpoint{4.348110in}{1.150252in}}%
\pgfpathlineto{\pgfqpoint{4.473110in}{1.150252in}}%
\pgfusepath{stroke}%
\end{pgfscope}%
\begin{pgfscope}%
\definecolor{textcolor}{rgb}{0.150000,0.150000,0.150000}%
\pgfsetstrokecolor{textcolor}%
\pgfsetfillcolor{textcolor}%
\pgftext[x=4.573110in,y=1.106502in,left,base]{\color{textcolor}\sffamily\fontsize{9.000000}{10.800000}\selectfont A* routing}%
\end{pgfscope}%
\begin{pgfscope}%
\pgfsetroundcap%
\pgfsetroundjoin%
\pgfsetlinewidth{1.505625pt}%
\definecolor{currentstroke}{rgb}{0.007843,0.619608,0.450980}%
\pgfsetstrokecolor{currentstroke}%
\pgfsetdash{}{0pt}%
\pgfpathmoveto{\pgfqpoint{4.223110in}{0.875740in}}%
\pgfpathlineto{\pgfqpoint{4.348110in}{0.875740in}}%
\pgfpathlineto{\pgfqpoint{4.473110in}{0.875740in}}%
\pgfusepath{stroke}%
\end{pgfscope}%
\begin{pgfscope}%
\definecolor{textcolor}{rgb}{0.150000,0.150000,0.150000}%
\pgfsetstrokecolor{textcolor}%
\pgfsetfillcolor{textcolor}%
\pgftext[x=4.573110in, y=0.919002in, left, base]{\color{textcolor}\sffamily\fontsize{9.000000}{10.800000}\selectfont Connect source \&}%
\end{pgfscope}%
\begin{pgfscope}%
\definecolor{textcolor}{rgb}{0.150000,0.150000,0.150000}%
\pgfsetstrokecolor{textcolor}%
\pgfsetfillcolor{textcolor}%
\pgftext[x=4.573110in, y=0.775008in, left, base]{\color{textcolor}\sffamily\fontsize{9.000000}{10.800000}\selectfont destination vertices}%
\end{pgfscope}%
\begin{pgfscope}%
\pgfsetroundcap%
\pgfsetroundjoin%
\pgfsetlinewidth{1.505625pt}%
\definecolor{currentstroke}{rgb}{0.835294,0.368627,0.000000}%
\pgfsetstrokecolor{currentstroke}%
\pgfsetdash{}{0pt}%
\pgfpathmoveto{\pgfqpoint{4.223110in}{0.631258in}}%
\pgfpathlineto{\pgfqpoint{4.348110in}{0.631258in}}%
\pgfpathlineto{\pgfqpoint{4.473110in}{0.631258in}}%
\pgfusepath{stroke}%
\end{pgfscope}%
\begin{pgfscope}%
\definecolor{textcolor}{rgb}{0.150000,0.150000,0.150000}%
\pgfsetstrokecolor{textcolor}%
\pgfsetfillcolor{textcolor}%
\pgftext[x=4.573110in,y=0.587508in,left,base]{\color{textcolor}\sffamily\fontsize{9.000000}{10.800000}\selectfont Restoring graph}%
\end{pgfscope}%
\begin{pgfscope}%
\pgfsetroundcap%
\pgfsetroundjoin%
\pgfsetlinewidth{1.003750pt}%
\definecolor{currentstroke}{rgb}{0.003922,0.450980,0.698039}%
\pgfsetstrokecolor{currentstroke}%
\pgfsetdash{}{0pt}%
\pgfpathmoveto{\pgfqpoint{0.736655in}{0.810679in}}%
\pgfpathlineto{\pgfqpoint{1.036735in}{1.177166in}}%
\pgfpathlineto{\pgfqpoint{1.386919in}{1.332674in}}%
\pgfpathlineto{\pgfqpoint{1.780050in}{1.379498in}}%
\pgfpathlineto{\pgfqpoint{2.954489in}{1.553607in}}%
\pgfpathlineto{\pgfqpoint{3.866840in}{1.624439in}}%
\pgfusepath{stroke}%
\end{pgfscope}%
\begin{pgfscope}%
\pgfsetbuttcap%
\pgfsetroundjoin%
\definecolor{currentfill}{rgb}{0.003922,0.450980,0.698039}%
\pgfsetfillcolor{currentfill}%
\pgfsetlinewidth{0.752812pt}%
\definecolor{currentstroke}{rgb}{1.000000,1.000000,1.000000}%
\pgfsetstrokecolor{currentstroke}%
\pgfsetdash{}{0pt}%
\pgfsys@defobject{currentmarker}{\pgfqpoint{-0.034722in}{-0.034722in}}{\pgfqpoint{0.034722in}{0.034722in}}{%
\pgfpathmoveto{\pgfqpoint{0.000000in}{-0.034722in}}%
\pgfpathcurveto{\pgfqpoint{0.009208in}{-0.034722in}}{\pgfqpoint{0.018041in}{-0.031064in}}{\pgfqpoint{0.024552in}{-0.024552in}}%
\pgfpathcurveto{\pgfqpoint{0.031064in}{-0.018041in}}{\pgfqpoint{0.034722in}{-0.009208in}}{\pgfqpoint{0.034722in}{0.000000in}}%
\pgfpathcurveto{\pgfqpoint{0.034722in}{0.009208in}}{\pgfqpoint{0.031064in}{0.018041in}}{\pgfqpoint{0.024552in}{0.024552in}}%
\pgfpathcurveto{\pgfqpoint{0.018041in}{0.031064in}}{\pgfqpoint{0.009208in}{0.034722in}}{\pgfqpoint{0.000000in}{0.034722in}}%
\pgfpathcurveto{\pgfqpoint{-0.009208in}{0.034722in}}{\pgfqpoint{-0.018041in}{0.031064in}}{\pgfqpoint{-0.024552in}{0.024552in}}%
\pgfpathcurveto{\pgfqpoint{-0.031064in}{0.018041in}}{\pgfqpoint{-0.034722in}{0.009208in}}{\pgfqpoint{-0.034722in}{0.000000in}}%
\pgfpathcurveto{\pgfqpoint{-0.034722in}{-0.009208in}}{\pgfqpoint{-0.031064in}{-0.018041in}}{\pgfqpoint{-0.024552in}{-0.024552in}}%
\pgfpathcurveto{\pgfqpoint{-0.018041in}{-0.031064in}}{\pgfqpoint{-0.009208in}{-0.034722in}}{\pgfqpoint{0.000000in}{-0.034722in}}%
\pgfpathlineto{\pgfqpoint{0.000000in}{-0.034722in}}%
\pgfpathclose%
\pgfusepath{stroke,fill}%
}%
\begin{pgfscope}%
\pgfsys@transformshift{0.736655in}{0.810679in}%
\pgfsys@useobject{currentmarker}{}%
\end{pgfscope}%
\begin{pgfscope}%
\pgfsys@transformshift{1.036735in}{1.177166in}%
\pgfsys@useobject{currentmarker}{}%
\end{pgfscope}%
\begin{pgfscope}%
\pgfsys@transformshift{1.386919in}{1.332674in}%
\pgfsys@useobject{currentmarker}{}%
\end{pgfscope}%
\begin{pgfscope}%
\pgfsys@transformshift{1.780050in}{1.379498in}%
\pgfsys@useobject{currentmarker}{}%
\end{pgfscope}%
\begin{pgfscope}%
\pgfsys@transformshift{2.954489in}{1.553607in}%
\pgfsys@useobject{currentmarker}{}%
\end{pgfscope}%
\begin{pgfscope}%
\pgfsys@transformshift{3.866840in}{1.624439in}%
\pgfsys@useobject{currentmarker}{}%
\end{pgfscope}%
\end{pgfscope}%
\begin{pgfscope}%
\pgfsetroundcap%
\pgfsetroundjoin%
\pgfsetlinewidth{1.003750pt}%
\definecolor{currentstroke}{rgb}{0.870588,0.560784,0.019608}%
\pgfsetstrokecolor{currentstroke}%
\pgfsetdash{}{0pt}%
\pgfpathmoveto{\pgfqpoint{0.736655in}{0.469579in}}%
\pgfpathlineto{\pgfqpoint{1.036735in}{0.490874in}}%
\pgfpathlineto{\pgfqpoint{1.386919in}{0.497589in}}%
\pgfpathlineto{\pgfqpoint{1.780050in}{0.500972in}}%
\pgfpathlineto{\pgfqpoint{2.954489in}{0.502552in}}%
\pgfpathlineto{\pgfqpoint{3.866840in}{0.502211in}}%
\pgfusepath{stroke}%
\end{pgfscope}%
\begin{pgfscope}%
\pgfsetbuttcap%
\pgfsetroundjoin%
\definecolor{currentfill}{rgb}{0.870588,0.560784,0.019608}%
\pgfsetfillcolor{currentfill}%
\pgfsetlinewidth{0.752812pt}%
\definecolor{currentstroke}{rgb}{1.000000,1.000000,1.000000}%
\pgfsetstrokecolor{currentstroke}%
\pgfsetdash{}{0pt}%
\pgfsys@defobject{currentmarker}{\pgfqpoint{-0.034722in}{-0.034722in}}{\pgfqpoint{0.034722in}{0.034722in}}{%
\pgfpathmoveto{\pgfqpoint{0.000000in}{-0.034722in}}%
\pgfpathcurveto{\pgfqpoint{0.009208in}{-0.034722in}}{\pgfqpoint{0.018041in}{-0.031064in}}{\pgfqpoint{0.024552in}{-0.024552in}}%
\pgfpathcurveto{\pgfqpoint{0.031064in}{-0.018041in}}{\pgfqpoint{0.034722in}{-0.009208in}}{\pgfqpoint{0.034722in}{0.000000in}}%
\pgfpathcurveto{\pgfqpoint{0.034722in}{0.009208in}}{\pgfqpoint{0.031064in}{0.018041in}}{\pgfqpoint{0.024552in}{0.024552in}}%
\pgfpathcurveto{\pgfqpoint{0.018041in}{0.031064in}}{\pgfqpoint{0.009208in}{0.034722in}}{\pgfqpoint{0.000000in}{0.034722in}}%
\pgfpathcurveto{\pgfqpoint{-0.009208in}{0.034722in}}{\pgfqpoint{-0.018041in}{0.031064in}}{\pgfqpoint{-0.024552in}{0.024552in}}%
\pgfpathcurveto{\pgfqpoint{-0.031064in}{0.018041in}}{\pgfqpoint{-0.034722in}{0.009208in}}{\pgfqpoint{-0.034722in}{0.000000in}}%
\pgfpathcurveto{\pgfqpoint{-0.034722in}{-0.009208in}}{\pgfqpoint{-0.031064in}{-0.018041in}}{\pgfqpoint{-0.024552in}{-0.024552in}}%
\pgfpathcurveto{\pgfqpoint{-0.018041in}{-0.031064in}}{\pgfqpoint{-0.009208in}{-0.034722in}}{\pgfqpoint{0.000000in}{-0.034722in}}%
\pgfpathlineto{\pgfqpoint{0.000000in}{-0.034722in}}%
\pgfpathclose%
\pgfusepath{stroke,fill}%
}%
\begin{pgfscope}%
\pgfsys@transformshift{0.736655in}{0.469579in}%
\pgfsys@useobject{currentmarker}{}%
\end{pgfscope}%
\begin{pgfscope}%
\pgfsys@transformshift{1.036735in}{0.490874in}%
\pgfsys@useobject{currentmarker}{}%
\end{pgfscope}%
\begin{pgfscope}%
\pgfsys@transformshift{1.386919in}{0.497589in}%
\pgfsys@useobject{currentmarker}{}%
\end{pgfscope}%
\begin{pgfscope}%
\pgfsys@transformshift{1.780050in}{0.500972in}%
\pgfsys@useobject{currentmarker}{}%
\end{pgfscope}%
\begin{pgfscope}%
\pgfsys@transformshift{2.954489in}{0.502552in}%
\pgfsys@useobject{currentmarker}{}%
\end{pgfscope}%
\begin{pgfscope}%
\pgfsys@transformshift{3.866840in}{0.502211in}%
\pgfsys@useobject{currentmarker}{}%
\end{pgfscope}%
\end{pgfscope}%
\begin{pgfscope}%
\pgfsetroundcap%
\pgfsetroundjoin%
\pgfsetlinewidth{1.003750pt}%
\definecolor{currentstroke}{rgb}{0.007843,0.619608,0.450980}%
\pgfsetstrokecolor{currentstroke}%
\pgfsetdash{}{0pt}%
\pgfpathmoveto{\pgfqpoint{0.736655in}{0.781838in}}%
\pgfpathlineto{\pgfqpoint{1.036735in}{1.119295in}}%
\pgfpathlineto{\pgfqpoint{1.386919in}{1.265759in}}%
\pgfpathlineto{\pgfqpoint{1.780050in}{1.305494in}}%
\pgfpathlineto{\pgfqpoint{2.954489in}{1.467208in}}%
\pgfpathlineto{\pgfqpoint{3.866840in}{1.531416in}}%
\pgfusepath{stroke}%
\end{pgfscope}%
\begin{pgfscope}%
\pgfsetbuttcap%
\pgfsetroundjoin%
\definecolor{currentfill}{rgb}{0.007843,0.619608,0.450980}%
\pgfsetfillcolor{currentfill}%
\pgfsetlinewidth{0.752812pt}%
\definecolor{currentstroke}{rgb}{1.000000,1.000000,1.000000}%
\pgfsetstrokecolor{currentstroke}%
\pgfsetdash{}{0pt}%
\pgfsys@defobject{currentmarker}{\pgfqpoint{-0.034722in}{-0.034722in}}{\pgfqpoint{0.034722in}{0.034722in}}{%
\pgfpathmoveto{\pgfqpoint{0.000000in}{-0.034722in}}%
\pgfpathcurveto{\pgfqpoint{0.009208in}{-0.034722in}}{\pgfqpoint{0.018041in}{-0.031064in}}{\pgfqpoint{0.024552in}{-0.024552in}}%
\pgfpathcurveto{\pgfqpoint{0.031064in}{-0.018041in}}{\pgfqpoint{0.034722in}{-0.009208in}}{\pgfqpoint{0.034722in}{0.000000in}}%
\pgfpathcurveto{\pgfqpoint{0.034722in}{0.009208in}}{\pgfqpoint{0.031064in}{0.018041in}}{\pgfqpoint{0.024552in}{0.024552in}}%
\pgfpathcurveto{\pgfqpoint{0.018041in}{0.031064in}}{\pgfqpoint{0.009208in}{0.034722in}}{\pgfqpoint{0.000000in}{0.034722in}}%
\pgfpathcurveto{\pgfqpoint{-0.009208in}{0.034722in}}{\pgfqpoint{-0.018041in}{0.031064in}}{\pgfqpoint{-0.024552in}{0.024552in}}%
\pgfpathcurveto{\pgfqpoint{-0.031064in}{0.018041in}}{\pgfqpoint{-0.034722in}{0.009208in}}{\pgfqpoint{-0.034722in}{0.000000in}}%
\pgfpathcurveto{\pgfqpoint{-0.034722in}{-0.009208in}}{\pgfqpoint{-0.031064in}{-0.018041in}}{\pgfqpoint{-0.024552in}{-0.024552in}}%
\pgfpathcurveto{\pgfqpoint{-0.018041in}{-0.031064in}}{\pgfqpoint{-0.009208in}{-0.034722in}}{\pgfqpoint{0.000000in}{-0.034722in}}%
\pgfpathlineto{\pgfqpoint{0.000000in}{-0.034722in}}%
\pgfpathclose%
\pgfusepath{stroke,fill}%
}%
\begin{pgfscope}%
\pgfsys@transformshift{0.736655in}{0.781838in}%
\pgfsys@useobject{currentmarker}{}%
\end{pgfscope}%
\begin{pgfscope}%
\pgfsys@transformshift{1.036735in}{1.119295in}%
\pgfsys@useobject{currentmarker}{}%
\end{pgfscope}%
\begin{pgfscope}%
\pgfsys@transformshift{1.386919in}{1.265759in}%
\pgfsys@useobject{currentmarker}{}%
\end{pgfscope}%
\begin{pgfscope}%
\pgfsys@transformshift{1.780050in}{1.305494in}%
\pgfsys@useobject{currentmarker}{}%
\end{pgfscope}%
\begin{pgfscope}%
\pgfsys@transformshift{2.954489in}{1.467208in}%
\pgfsys@useobject{currentmarker}{}%
\end{pgfscope}%
\begin{pgfscope}%
\pgfsys@transformshift{3.866840in}{1.531416in}%
\pgfsys@useobject{currentmarker}{}%
\end{pgfscope}%
\end{pgfscope}%
\begin{pgfscope}%
\pgfsetroundcap%
\pgfsetroundjoin%
\pgfsetlinewidth{1.003750pt}%
\definecolor{currentstroke}{rgb}{0.835294,0.368627,0.000000}%
\pgfsetstrokecolor{currentstroke}%
\pgfsetdash{}{0pt}%
\pgfpathmoveto{\pgfqpoint{0.736655in}{0.461958in}}%
\pgfpathlineto{\pgfqpoint{1.036735in}{0.469709in}}%
\pgfpathlineto{\pgfqpoint{1.386919in}{0.472032in}}%
\pgfpathlineto{\pgfqpoint{1.780050in}{0.475732in}}%
\pgfpathlineto{\pgfqpoint{2.954489in}{0.486545in}}%
\pgfpathlineto{\pgfqpoint{3.866840in}{0.493504in}}%
\pgfusepath{stroke}%
\end{pgfscope}%
\begin{pgfscope}%
\pgfsetbuttcap%
\pgfsetroundjoin%
\definecolor{currentfill}{rgb}{0.835294,0.368627,0.000000}%
\pgfsetfillcolor{currentfill}%
\pgfsetlinewidth{0.752812pt}%
\definecolor{currentstroke}{rgb}{1.000000,1.000000,1.000000}%
\pgfsetstrokecolor{currentstroke}%
\pgfsetdash{}{0pt}%
\pgfsys@defobject{currentmarker}{\pgfqpoint{-0.034722in}{-0.034722in}}{\pgfqpoint{0.034722in}{0.034722in}}{%
\pgfpathmoveto{\pgfqpoint{0.000000in}{-0.034722in}}%
\pgfpathcurveto{\pgfqpoint{0.009208in}{-0.034722in}}{\pgfqpoint{0.018041in}{-0.031064in}}{\pgfqpoint{0.024552in}{-0.024552in}}%
\pgfpathcurveto{\pgfqpoint{0.031064in}{-0.018041in}}{\pgfqpoint{0.034722in}{-0.009208in}}{\pgfqpoint{0.034722in}{0.000000in}}%
\pgfpathcurveto{\pgfqpoint{0.034722in}{0.009208in}}{\pgfqpoint{0.031064in}{0.018041in}}{\pgfqpoint{0.024552in}{0.024552in}}%
\pgfpathcurveto{\pgfqpoint{0.018041in}{0.031064in}}{\pgfqpoint{0.009208in}{0.034722in}}{\pgfqpoint{0.000000in}{0.034722in}}%
\pgfpathcurveto{\pgfqpoint{-0.009208in}{0.034722in}}{\pgfqpoint{-0.018041in}{0.031064in}}{\pgfqpoint{-0.024552in}{0.024552in}}%
\pgfpathcurveto{\pgfqpoint{-0.031064in}{0.018041in}}{\pgfqpoint{-0.034722in}{0.009208in}}{\pgfqpoint{-0.034722in}{0.000000in}}%
\pgfpathcurveto{\pgfqpoint{-0.034722in}{-0.009208in}}{\pgfqpoint{-0.031064in}{-0.018041in}}{\pgfqpoint{-0.024552in}{-0.024552in}}%
\pgfpathcurveto{\pgfqpoint{-0.018041in}{-0.031064in}}{\pgfqpoint{-0.009208in}{-0.034722in}}{\pgfqpoint{0.000000in}{-0.034722in}}%
\pgfpathlineto{\pgfqpoint{0.000000in}{-0.034722in}}%
\pgfpathclose%
\pgfusepath{stroke,fill}%
}%
\begin{pgfscope}%
\pgfsys@transformshift{0.736655in}{0.461958in}%
\pgfsys@useobject{currentmarker}{}%
\end{pgfscope}%
\begin{pgfscope}%
\pgfsys@transformshift{1.036735in}{0.469709in}%
\pgfsys@useobject{currentmarker}{}%
\end{pgfscope}%
\begin{pgfscope}%
\pgfsys@transformshift{1.386919in}{0.472032in}%
\pgfsys@useobject{currentmarker}{}%
\end{pgfscope}%
\begin{pgfscope}%
\pgfsys@transformshift{1.780050in}{0.475732in}%
\pgfsys@useobject{currentmarker}{}%
\end{pgfscope}%
\begin{pgfscope}%
\pgfsys@transformshift{2.954489in}{0.486545in}%
\pgfsys@useobject{currentmarker}{}%
\end{pgfscope}%
\begin{pgfscope}%
\pgfsys@transformshift{3.866840in}{0.493504in}%
\pgfsys@useobject{currentmarker}{}%
\end{pgfscope}%
\end{pgfscope}%
\end{pgfpicture}%
\makeatother%
\endgroup%

						\end{figcenter}
						\caption{Routing between the same waypoints appearing in all datasets (distance between the waypoints: 600 m).}
					\end{subfigure}
				\end{figcenter}
				\caption{Routing time statistics of the \enquote{OSM rural} datasets.}
				\label{fig:eval-rural-routing-details}
			\end{figure}
			
		\subsubsection{Dataset without roads or obstacles}
		\label{subsubsec:dataset-without-roads-obstacles}
		
			The normal OSM-based datasets contain both, roads and obstacles.
			However, datasets containing only roads or only obstacles are thinkable and can also be used with the hybrid routing algorithm.
			For this evaluation, the 4km\textsuperscript{2} datasets of the \enquote{OSM city} and \enquote{OSM rural} categories were filtered yielding three datasets for each category:
			The normal dataset, one without roads and one without obstacles.
			In the dataset without roads, all road edges were removed which do not cross buildings, meaning building passages remained within this dataset to ensure the reachability of backyards.
			Influences of the two filters are discussed in the following.
			
			\begin{figure}[h!]
				\begin{figcenter}
					\begin{tabularx}{0.95\textwidth}{p{2.85cm}RRR>{\raggedleft\arraybackslash}p{2.25cm}R}
\toprule
\textbf{Operation}	& \textbf{Normal}	& \textbf{No roads}	& \textbf{Decrease compared to normal}	& \textbf{No obstacles}	& \textbf{Decrease compared to normal}	\\
\midrule
kNN search			& 315.86 s			& 288.69 s			&  8.60\%								&   0.01 ms				& 99.99\%								\\
Create graph		&   0.69 s			&   0.68 s			&  0.84\%								&   0.01 ms				& 99.99\%								\\
Get obstacles		&   0.30 s			&   0.29 s			&  3.28\%								&   9.54 ms				& 96.82\%								\\
Merge road edges	& 120.13 s			&  30.79 s			& 74.37\%								& 484.55 ms				& 99.60\%								\\
Add POI attributes	&   0.06 s			&   0.03 s			& 53.81\%								&   7.63 ms				& 85.97\%								\\
\midrule
Total time			& 437.34 s			& 320.57 s			& 26.70\%								& 525.61 ms				& 99.88\%								\\
\bottomrule
					\end{tabularx}
					\captionof{table}[Measurements for the 4km\textsuperscript{2} \enquote{OSM city} normal, no roads and no obstacles datasets.]{}
				\end{figcenter}
				\vspace{3ex}
				\begin{figcenter}
					\begingroup%
\makeatletter%
\begin{pgfpicture}%
\pgfpathrectangle{\pgfpointorigin}{\pgfqpoint{6.101690in}{1.716449in}}%
\pgfusepath{use as bounding box}%
\begin{pgfscope}%
\pgfsetbuttcap%
\pgfsetmiterjoin%
\definecolor{currentfill}{rgb}{1.000000,1.000000,1.000000}%
\pgfsetfillcolor{currentfill}%
\pgfsetlinewidth{0.000000pt}%
\definecolor{currentstroke}{rgb}{1.000000,1.000000,1.000000}%
\pgfsetstrokecolor{currentstroke}%
\pgfsetdash{}{0pt}%
\pgfpathmoveto{\pgfqpoint{0.000000in}{0.000000in}}%
\pgfpathlineto{\pgfqpoint{6.101690in}{0.000000in}}%
\pgfpathlineto{\pgfqpoint{6.101690in}{1.716449in}}%
\pgfpathlineto{\pgfqpoint{0.000000in}{1.716449in}}%
\pgfpathlineto{\pgfqpoint{0.000000in}{0.000000in}}%
\pgfpathclose%
\pgfusepath{fill}%
\end{pgfscope}%
\begin{pgfscope}%
\pgfsetbuttcap%
\pgfsetmiterjoin%
\definecolor{currentfill}{rgb}{1.000000,1.000000,1.000000}%
\pgfsetfillcolor{currentfill}%
\pgfsetlinewidth{0.000000pt}%
\definecolor{currentstroke}{rgb}{0.000000,0.000000,0.000000}%
\pgfsetstrokecolor{currentstroke}%
\pgfsetstrokeopacity{0.000000}%
\pgfsetdash{}{0pt}%
\pgfpathmoveto{\pgfqpoint{0.592976in}{0.400938in}}%
\pgfpathlineto{\pgfqpoint{4.765159in}{0.400938in}}%
\pgfpathlineto{\pgfqpoint{4.765159in}{1.716449in}}%
\pgfpathlineto{\pgfqpoint{0.592976in}{1.716449in}}%
\pgfpathlineto{\pgfqpoint{0.592976in}{0.400938in}}%
\pgfpathclose%
\pgfusepath{fill}%
\end{pgfscope}%
\begin{pgfscope}%
\definecolor{textcolor}{rgb}{0.150000,0.150000,0.150000}%
\pgfsetstrokecolor{textcolor}%
\pgfsetfillcolor{textcolor}%
\pgftext[x=0.940658in,y=0.268994in,,top]{\color{textcolor}\sffamily\fontsize{9.000000}{10.800000}\selectfont Total time}%
\end{pgfscope}%
\begin{pgfscope}%
\definecolor{textcolor}{rgb}{0.150000,0.150000,0.150000}%
\pgfsetstrokecolor{textcolor}%
\pgfsetfillcolor{textcolor}%
\pgftext[x=1.636022in,y=0.268994in,,top]{\color{textcolor}\sffamily\fontsize{9.000000}{10.800000}\selectfont kNN search}%
\end{pgfscope}%
\begin{pgfscope}%
\definecolor{textcolor}{rgb}{0.150000,0.150000,0.150000}%
\pgfsetstrokecolor{textcolor}%
\pgfsetfillcolor{textcolor}%
\pgftext[x=2.147517in, y=0.174023in, left, base]{\color{textcolor}\sffamily\fontsize{9.000000}{10.800000}\selectfont Create}%
\end{pgfscope}%
\begin{pgfscope}%
\definecolor{textcolor}{rgb}{0.150000,0.150000,0.150000}%
\pgfsetstrokecolor{textcolor}%
\pgfsetfillcolor{textcolor}%
\pgftext[x=2.168086in, y=0.030029in, left, base]{\color{textcolor}\sffamily\fontsize{9.000000}{10.800000}\selectfont graph}%
\end{pgfscope}%
\begin{pgfscope}%
\definecolor{textcolor}{rgb}{0.150000,0.150000,0.150000}%
\pgfsetstrokecolor{textcolor}%
\pgfsetfillcolor{textcolor}%
\pgftext[x=2.929002in, y=0.174023in, left, base]{\color{textcolor}\sffamily\fontsize{9.000000}{10.800000}\selectfont Get}%
\end{pgfscope}%
\begin{pgfscope}%
\definecolor{textcolor}{rgb}{0.150000,0.150000,0.150000}%
\pgfsetstrokecolor{textcolor}%
\pgfsetfillcolor{textcolor}%
\pgftext[x=2.764756in, y=0.030029in, left, base]{\color{textcolor}\sffamily\fontsize{9.000000}{10.800000}\selectfont obstacles}%
\end{pgfscope}%
\begin{pgfscope}%
\definecolor{textcolor}{rgb}{0.150000,0.150000,0.150000}%
\pgfsetstrokecolor{textcolor}%
\pgfsetfillcolor{textcolor}%
\pgftext[x=3.397101in, y=0.174023in, left, base]{\color{textcolor}\sffamily\fontsize{9.000000}{10.800000}\selectfont Merge road}%
\end{pgfscope}%
\begin{pgfscope}%
\definecolor{textcolor}{rgb}{0.150000,0.150000,0.150000}%
\pgfsetstrokecolor{textcolor}%
\pgfsetfillcolor{textcolor}%
\pgftext[x=3.558020in, y=0.030029in, left, base]{\color{textcolor}\sffamily\fontsize{9.000000}{10.800000}\selectfont edges}%
\end{pgfscope}%
\begin{pgfscope}%
\definecolor{textcolor}{rgb}{0.150000,0.150000,0.150000}%
\pgfsetstrokecolor{textcolor}%
\pgfsetfillcolor{textcolor}%
\pgftext[x=4.187039in, y=0.174023in, left, base]{\color{textcolor}\sffamily\fontsize{9.000000}{10.800000}\selectfont Add POI}%
\end{pgfscope}%
\begin{pgfscope}%
\definecolor{textcolor}{rgb}{0.150000,0.150000,0.150000}%
\pgfsetstrokecolor{textcolor}%
\pgfsetfillcolor{textcolor}%
\pgftext[x=4.143979in, y=0.030029in, left, base]{\color{textcolor}\sffamily\fontsize{9.000000}{10.800000}\selectfont attributes}%
\end{pgfscope}%
\begin{pgfscope}%
\pgfpathrectangle{\pgfqpoint{0.592976in}{0.400938in}}{\pgfqpoint{4.172183in}{1.315510in}}%
\pgfusepath{clip}%
\pgfsetroundcap%
\pgfsetroundjoin%
\pgfsetlinewidth{1.003750pt}%
\definecolor{currentstroke}{rgb}{0.800000,0.800000,0.800000}%
\pgfsetstrokecolor{currentstroke}%
\pgfsetdash{}{0pt}%
\pgfpathmoveto{\pgfqpoint{0.592976in}{0.609452in}}%
\pgfpathlineto{\pgfqpoint{4.765159in}{0.609452in}}%
\pgfusepath{stroke}%
\end{pgfscope}%
\begin{pgfscope}%
\definecolor{textcolor}{rgb}{0.150000,0.150000,0.150000}%
\pgfsetstrokecolor{textcolor}%
\pgfsetfillcolor{textcolor}%
\pgftext[x=0.194444in, y=0.561967in, left, base]{\color{textcolor}\sffamily\fontsize{9.000000}{10.800000}\selectfont \(\displaystyle {10^{-4}}\)}%
\end{pgfscope}%
\begin{pgfscope}%
\pgfpathrectangle{\pgfqpoint{0.592976in}{0.400938in}}{\pgfqpoint{4.172183in}{1.315510in}}%
\pgfusepath{clip}%
\pgfsetroundcap%
\pgfsetroundjoin%
\pgfsetlinewidth{1.003750pt}%
\definecolor{currentstroke}{rgb}{0.800000,0.800000,0.800000}%
\pgfsetstrokecolor{currentstroke}%
\pgfsetdash{}{0pt}%
\pgfpathmoveto{\pgfqpoint{0.592976in}{0.924835in}}%
\pgfpathlineto{\pgfqpoint{4.765159in}{0.924835in}}%
\pgfusepath{stroke}%
\end{pgfscope}%
\begin{pgfscope}%
\definecolor{textcolor}{rgb}{0.150000,0.150000,0.150000}%
\pgfsetstrokecolor{textcolor}%
\pgfsetfillcolor{textcolor}%
\pgftext[x=0.194444in, y=0.877350in, left, base]{\color{textcolor}\sffamily\fontsize{9.000000}{10.800000}\selectfont \(\displaystyle {10^{-2}}\)}%
\end{pgfscope}%
\begin{pgfscope}%
\pgfpathrectangle{\pgfqpoint{0.592976in}{0.400938in}}{\pgfqpoint{4.172183in}{1.315510in}}%
\pgfusepath{clip}%
\pgfsetroundcap%
\pgfsetroundjoin%
\pgfsetlinewidth{1.003750pt}%
\definecolor{currentstroke}{rgb}{0.800000,0.800000,0.800000}%
\pgfsetstrokecolor{currentstroke}%
\pgfsetdash{}{0pt}%
\pgfpathmoveto{\pgfqpoint{0.592976in}{1.240218in}}%
\pgfpathlineto{\pgfqpoint{4.765159in}{1.240218in}}%
\pgfusepath{stroke}%
\end{pgfscope}%
\begin{pgfscope}%
\definecolor{textcolor}{rgb}{0.150000,0.150000,0.150000}%
\pgfsetstrokecolor{textcolor}%
\pgfsetfillcolor{textcolor}%
\pgftext[x=0.274690in, y=1.192732in, left, base]{\color{textcolor}\sffamily\fontsize{9.000000}{10.800000}\selectfont \(\displaystyle {10^{0}}\)}%
\end{pgfscope}%
\begin{pgfscope}%
\pgfpathrectangle{\pgfqpoint{0.592976in}{0.400938in}}{\pgfqpoint{4.172183in}{1.315510in}}%
\pgfusepath{clip}%
\pgfsetroundcap%
\pgfsetroundjoin%
\pgfsetlinewidth{1.003750pt}%
\definecolor{currentstroke}{rgb}{0.800000,0.800000,0.800000}%
\pgfsetstrokecolor{currentstroke}%
\pgfsetdash{}{0pt}%
\pgfpathmoveto{\pgfqpoint{0.592976in}{1.555601in}}%
\pgfpathlineto{\pgfqpoint{4.765159in}{1.555601in}}%
\pgfusepath{stroke}%
\end{pgfscope}%
\begin{pgfscope}%
\definecolor{textcolor}{rgb}{0.150000,0.150000,0.150000}%
\pgfsetstrokecolor{textcolor}%
\pgfsetfillcolor{textcolor}%
\pgftext[x=0.274690in, y=1.508115in, left, base]{\color{textcolor}\sffamily\fontsize{9.000000}{10.800000}\selectfont \(\displaystyle {10^{2}}\)}%
\end{pgfscope}%
\begin{pgfscope}%
\definecolor{textcolor}{rgb}{0.150000,0.150000,0.150000}%
\pgfsetstrokecolor{textcolor}%
\pgfsetfillcolor{textcolor}%
\pgftext[x=0.125000in,y=1.058693in,,bottom,rotate=90.000000]{\color{textcolor}\sffamily\fontsize{9.000000}{10.800000}\selectfont Time in s}%
\end{pgfscope}%
\begin{pgfscope}%
\pgfpathrectangle{\pgfqpoint{0.592976in}{0.400938in}}{\pgfqpoint{4.172183in}{1.315510in}}%
\pgfusepath{clip}%
\pgfsetbuttcap%
\pgfsetmiterjoin%
\definecolor{currentfill}{rgb}{0.349020,0.490196,0.749020}%
\pgfsetfillcolor{currentfill}%
\pgfsetlinewidth{1.003750pt}%
\definecolor{currentstroke}{rgb}{1.000000,1.000000,1.000000}%
\pgfsetstrokecolor{currentstroke}%
\pgfsetdash{}{0pt}%
\pgfpathmoveto{\pgfqpoint{0.662512in}{-156.451171in}}%
\pgfpathlineto{\pgfqpoint{0.847943in}{-156.451171in}}%
\pgfpathlineto{\pgfqpoint{0.847943in}{1.656653in}}%
\pgfpathlineto{\pgfqpoint{0.662512in}{1.656653in}}%
\pgfpathlineto{\pgfqpoint{0.662512in}{-156.451171in}}%
\pgfpathclose%
\pgfusepath{stroke,fill}%
\end{pgfscope}%
\begin{pgfscope}%
\pgfpathrectangle{\pgfqpoint{0.592976in}{0.400938in}}{\pgfqpoint{4.172183in}{1.315510in}}%
\pgfusepath{clip}%
\pgfsetbuttcap%
\pgfsetmiterjoin%
\definecolor{currentfill}{rgb}{0.349020,0.490196,0.749020}%
\pgfsetfillcolor{currentfill}%
\pgfsetlinewidth{1.003750pt}%
\definecolor{currentstroke}{rgb}{1.000000,1.000000,1.000000}%
\pgfsetstrokecolor{currentstroke}%
\pgfsetdash{}{0pt}%
\pgfpathmoveto{\pgfqpoint{1.357876in}{-156.451171in}}%
\pgfpathlineto{\pgfqpoint{1.543307in}{-156.451171in}}%
\pgfpathlineto{\pgfqpoint{1.543307in}{1.634366in}}%
\pgfpathlineto{\pgfqpoint{1.357876in}{1.634366in}}%
\pgfpathlineto{\pgfqpoint{1.357876in}{-156.451171in}}%
\pgfpathclose%
\pgfusepath{stroke,fill}%
\end{pgfscope}%
\begin{pgfscope}%
\pgfpathrectangle{\pgfqpoint{0.592976in}{0.400938in}}{\pgfqpoint{4.172183in}{1.315510in}}%
\pgfusepath{clip}%
\pgfsetbuttcap%
\pgfsetmiterjoin%
\definecolor{currentfill}{rgb}{0.349020,0.490196,0.749020}%
\pgfsetfillcolor{currentfill}%
\pgfsetlinewidth{1.003750pt}%
\definecolor{currentstroke}{rgb}{1.000000,1.000000,1.000000}%
\pgfsetstrokecolor{currentstroke}%
\pgfsetdash{}{0pt}%
\pgfpathmoveto{\pgfqpoint{2.053240in}{-156.451171in}}%
\pgfpathlineto{\pgfqpoint{2.238670in}{-156.451171in}}%
\pgfpathlineto{\pgfqpoint{2.238670in}{1.214516in}}%
\pgfpathlineto{\pgfqpoint{2.053240in}{1.214516in}}%
\pgfpathlineto{\pgfqpoint{2.053240in}{-156.451171in}}%
\pgfpathclose%
\pgfusepath{stroke,fill}%
\end{pgfscope}%
\begin{pgfscope}%
\pgfpathrectangle{\pgfqpoint{0.592976in}{0.400938in}}{\pgfqpoint{4.172183in}{1.315510in}}%
\pgfusepath{clip}%
\pgfsetbuttcap%
\pgfsetmiterjoin%
\definecolor{currentfill}{rgb}{0.349020,0.490196,0.749020}%
\pgfsetfillcolor{currentfill}%
\pgfsetlinewidth{1.003750pt}%
\definecolor{currentstroke}{rgb}{1.000000,1.000000,1.000000}%
\pgfsetstrokecolor{currentstroke}%
\pgfsetdash{}{0pt}%
\pgfpathmoveto{\pgfqpoint{2.748604in}{-156.451171in}}%
\pgfpathlineto{\pgfqpoint{2.934034in}{-156.451171in}}%
\pgfpathlineto{\pgfqpoint{2.934034in}{1.157810in}}%
\pgfpathlineto{\pgfqpoint{2.748604in}{1.157810in}}%
\pgfpathlineto{\pgfqpoint{2.748604in}{-156.451171in}}%
\pgfpathclose%
\pgfusepath{stroke,fill}%
\end{pgfscope}%
\begin{pgfscope}%
\pgfpathrectangle{\pgfqpoint{0.592976in}{0.400938in}}{\pgfqpoint{4.172183in}{1.315510in}}%
\pgfusepath{clip}%
\pgfsetbuttcap%
\pgfsetmiterjoin%
\definecolor{currentfill}{rgb}{0.349020,0.490196,0.749020}%
\pgfsetfillcolor{currentfill}%
\pgfsetlinewidth{1.003750pt}%
\definecolor{currentstroke}{rgb}{1.000000,1.000000,1.000000}%
\pgfsetstrokecolor{currentstroke}%
\pgfsetdash{}{0pt}%
\pgfpathmoveto{\pgfqpoint{3.443968in}{-156.451171in}}%
\pgfpathlineto{\pgfqpoint{3.629398in}{-156.451171in}}%
\pgfpathlineto{\pgfqpoint{3.629398in}{1.568158in}}%
\pgfpathlineto{\pgfqpoint{3.443968in}{1.568158in}}%
\pgfpathlineto{\pgfqpoint{3.443968in}{-156.451171in}}%
\pgfpathclose%
\pgfusepath{stroke,fill}%
\end{pgfscope}%
\begin{pgfscope}%
\pgfpathrectangle{\pgfqpoint{0.592976in}{0.400938in}}{\pgfqpoint{4.172183in}{1.315510in}}%
\pgfusepath{clip}%
\pgfsetbuttcap%
\pgfsetmiterjoin%
\definecolor{currentfill}{rgb}{0.349020,0.490196,0.749020}%
\pgfsetfillcolor{currentfill}%
\pgfsetlinewidth{1.003750pt}%
\definecolor{currentstroke}{rgb}{1.000000,1.000000,1.000000}%
\pgfsetstrokecolor{currentstroke}%
\pgfsetdash{}{0pt}%
\pgfpathmoveto{\pgfqpoint{4.139332in}{-156.451171in}}%
\pgfpathlineto{\pgfqpoint{4.324762in}{-156.451171in}}%
\pgfpathlineto{\pgfqpoint{4.324762in}{1.040758in}}%
\pgfpathlineto{\pgfqpoint{4.139332in}{1.040758in}}%
\pgfpathlineto{\pgfqpoint{4.139332in}{-156.451171in}}%
\pgfpathclose%
\pgfusepath{stroke,fill}%
\end{pgfscope}%
\begin{pgfscope}%
\pgfpathrectangle{\pgfqpoint{0.592976in}{0.400938in}}{\pgfqpoint{4.172183in}{1.315510in}}%
\pgfusepath{clip}%
\pgfsetbuttcap%
\pgfsetmiterjoin%
\definecolor{currentfill}{rgb}{0.852941,0.544118,0.370588}%
\pgfsetfillcolor{currentfill}%
\pgfsetlinewidth{1.003750pt}%
\definecolor{currentstroke}{rgb}{1.000000,1.000000,1.000000}%
\pgfsetstrokecolor{currentstroke}%
\pgfsetdash{}{0pt}%
\pgfpathmoveto{\pgfqpoint{0.847943in}{-156.451171in}}%
\pgfpathlineto{\pgfqpoint{1.033373in}{-156.451171in}}%
\pgfpathlineto{\pgfqpoint{1.033373in}{1.635380in}}%
\pgfpathlineto{\pgfqpoint{0.847943in}{1.635380in}}%
\pgfpathlineto{\pgfqpoint{0.847943in}{-156.451171in}}%
\pgfpathclose%
\pgfusepath{stroke,fill}%
\end{pgfscope}%
\begin{pgfscope}%
\pgfpathrectangle{\pgfqpoint{0.592976in}{0.400938in}}{\pgfqpoint{4.172183in}{1.315510in}}%
\pgfusepath{clip}%
\pgfsetbuttcap%
\pgfsetmiterjoin%
\definecolor{currentfill}{rgb}{0.852941,0.544118,0.370588}%
\pgfsetfillcolor{currentfill}%
\pgfsetlinewidth{1.003750pt}%
\definecolor{currentstroke}{rgb}{1.000000,1.000000,1.000000}%
\pgfsetstrokecolor{currentstroke}%
\pgfsetdash{}{0pt}%
\pgfpathmoveto{\pgfqpoint{1.543307in}{-156.451171in}}%
\pgfpathlineto{\pgfqpoint{1.728737in}{-156.451171in}}%
\pgfpathlineto{\pgfqpoint{1.728737in}{1.628207in}}%
\pgfpathlineto{\pgfqpoint{1.543307in}{1.628207in}}%
\pgfpathlineto{\pgfqpoint{1.543307in}{-156.451171in}}%
\pgfpathclose%
\pgfusepath{stroke,fill}%
\end{pgfscope}%
\begin{pgfscope}%
\pgfpathrectangle{\pgfqpoint{0.592976in}{0.400938in}}{\pgfqpoint{4.172183in}{1.315510in}}%
\pgfusepath{clip}%
\pgfsetbuttcap%
\pgfsetmiterjoin%
\definecolor{currentfill}{rgb}{0.852941,0.544118,0.370588}%
\pgfsetfillcolor{currentfill}%
\pgfsetlinewidth{1.003750pt}%
\definecolor{currentstroke}{rgb}{1.000000,1.000000,1.000000}%
\pgfsetstrokecolor{currentstroke}%
\pgfsetdash{}{0pt}%
\pgfpathmoveto{\pgfqpoint{2.238670in}{-156.451171in}}%
\pgfpathlineto{\pgfqpoint{2.424101in}{-156.451171in}}%
\pgfpathlineto{\pgfqpoint{2.424101in}{1.213935in}}%
\pgfpathlineto{\pgfqpoint{2.238670in}{1.213935in}}%
\pgfpathlineto{\pgfqpoint{2.238670in}{-156.451171in}}%
\pgfpathclose%
\pgfusepath{stroke,fill}%
\end{pgfscope}%
\begin{pgfscope}%
\pgfpathrectangle{\pgfqpoint{0.592976in}{0.400938in}}{\pgfqpoint{4.172183in}{1.315510in}}%
\pgfusepath{clip}%
\pgfsetbuttcap%
\pgfsetmiterjoin%
\definecolor{currentfill}{rgb}{0.852941,0.544118,0.370588}%
\pgfsetfillcolor{currentfill}%
\pgfsetlinewidth{1.003750pt}%
\definecolor{currentstroke}{rgb}{1.000000,1.000000,1.000000}%
\pgfsetstrokecolor{currentstroke}%
\pgfsetdash{}{0pt}%
\pgfpathmoveto{\pgfqpoint{2.934034in}{-156.451171in}}%
\pgfpathlineto{\pgfqpoint{3.119465in}{-156.451171in}}%
\pgfpathlineto{\pgfqpoint{3.119465in}{1.155527in}}%
\pgfpathlineto{\pgfqpoint{2.934034in}{1.155527in}}%
\pgfpathlineto{\pgfqpoint{2.934034in}{-156.451171in}}%
\pgfpathclose%
\pgfusepath{stroke,fill}%
\end{pgfscope}%
\begin{pgfscope}%
\pgfpathrectangle{\pgfqpoint{0.592976in}{0.400938in}}{\pgfqpoint{4.172183in}{1.315510in}}%
\pgfusepath{clip}%
\pgfsetbuttcap%
\pgfsetmiterjoin%
\definecolor{currentfill}{rgb}{0.852941,0.544118,0.370588}%
\pgfsetfillcolor{currentfill}%
\pgfsetlinewidth{1.003750pt}%
\definecolor{currentstroke}{rgb}{1.000000,1.000000,1.000000}%
\pgfsetstrokecolor{currentstroke}%
\pgfsetdash{}{0pt}%
\pgfpathmoveto{\pgfqpoint{3.629398in}{-156.451171in}}%
\pgfpathlineto{\pgfqpoint{3.814829in}{-156.451171in}}%
\pgfpathlineto{\pgfqpoint{3.814829in}{1.474926in}}%
\pgfpathlineto{\pgfqpoint{3.629398in}{1.474926in}}%
\pgfpathlineto{\pgfqpoint{3.629398in}{-156.451171in}}%
\pgfpathclose%
\pgfusepath{stroke,fill}%
\end{pgfscope}%
\begin{pgfscope}%
\pgfpathrectangle{\pgfqpoint{0.592976in}{0.400938in}}{\pgfqpoint{4.172183in}{1.315510in}}%
\pgfusepath{clip}%
\pgfsetbuttcap%
\pgfsetmiterjoin%
\definecolor{currentfill}{rgb}{0.852941,0.544118,0.370588}%
\pgfsetfillcolor{currentfill}%
\pgfsetlinewidth{1.003750pt}%
\definecolor{currentstroke}{rgb}{1.000000,1.000000,1.000000}%
\pgfsetstrokecolor{currentstroke}%
\pgfsetdash{}{0pt}%
\pgfpathmoveto{\pgfqpoint{4.324762in}{-156.451171in}}%
\pgfpathlineto{\pgfqpoint{4.510193in}{-156.451171in}}%
\pgfpathlineto{\pgfqpoint{4.510193in}{0.987860in}}%
\pgfpathlineto{\pgfqpoint{4.324762in}{0.987860in}}%
\pgfpathlineto{\pgfqpoint{4.324762in}{-156.451171in}}%
\pgfpathclose%
\pgfusepath{stroke,fill}%
\end{pgfscope}%
\begin{pgfscope}%
\pgfpathrectangle{\pgfqpoint{0.592976in}{0.400938in}}{\pgfqpoint{4.172183in}{1.315510in}}%
\pgfusepath{clip}%
\pgfsetbuttcap%
\pgfsetmiterjoin%
\definecolor{currentfill}{rgb}{0.460784,0.749020,0.443137}%
\pgfsetfillcolor{currentfill}%
\pgfsetlinewidth{1.003750pt}%
\definecolor{currentstroke}{rgb}{1.000000,1.000000,1.000000}%
\pgfsetstrokecolor{currentstroke}%
\pgfsetdash{}{0pt}%
\pgfpathmoveto{\pgfqpoint{1.033373in}{-156.451171in}}%
\pgfpathlineto{\pgfqpoint{1.218803in}{-156.451171in}}%
\pgfpathlineto{\pgfqpoint{1.218803in}{1.196169in}}%
\pgfpathlineto{\pgfqpoint{1.033373in}{1.196169in}}%
\pgfpathlineto{\pgfqpoint{1.033373in}{-156.451171in}}%
\pgfpathclose%
\pgfusepath{stroke,fill}%
\end{pgfscope}%
\begin{pgfscope}%
\pgfpathrectangle{\pgfqpoint{0.592976in}{0.400938in}}{\pgfqpoint{4.172183in}{1.315510in}}%
\pgfusepath{clip}%
\pgfsetbuttcap%
\pgfsetmiterjoin%
\definecolor{currentfill}{rgb}{0.460784,0.749020,0.443137}%
\pgfsetfillcolor{currentfill}%
\pgfsetlinewidth{1.003750pt}%
\definecolor{currentstroke}{rgb}{1.000000,1.000000,1.000000}%
\pgfsetstrokecolor{currentstroke}%
\pgfsetdash{}{0pt}%
\pgfpathmoveto{\pgfqpoint{1.728737in}{-156.451171in}}%
\pgfpathlineto{\pgfqpoint{1.914167in}{-156.451171in}}%
\pgfpathlineto{\pgfqpoint{1.914167in}{0.460734in}}%
\pgfpathlineto{\pgfqpoint{1.728737in}{0.460734in}}%
\pgfpathlineto{\pgfqpoint{1.728737in}{-156.451171in}}%
\pgfpathclose%
\pgfusepath{stroke,fill}%
\end{pgfscope}%
\begin{pgfscope}%
\pgfpathrectangle{\pgfqpoint{0.592976in}{0.400938in}}{\pgfqpoint{4.172183in}{1.315510in}}%
\pgfusepath{clip}%
\pgfsetbuttcap%
\pgfsetmiterjoin%
\definecolor{currentfill}{rgb}{0.460784,0.749020,0.443137}%
\pgfsetfillcolor{currentfill}%
\pgfsetlinewidth{1.003750pt}%
\definecolor{currentstroke}{rgb}{1.000000,1.000000,1.000000}%
\pgfsetstrokecolor{currentstroke}%
\pgfsetdash{}{0pt}%
\pgfpathmoveto{\pgfqpoint{2.424101in}{-156.451171in}}%
\pgfpathlineto{\pgfqpoint{2.609531in}{-156.451171in}}%
\pgfpathlineto{\pgfqpoint{2.609531in}{0.473819in}}%
\pgfpathlineto{\pgfqpoint{2.424101in}{0.473819in}}%
\pgfpathlineto{\pgfqpoint{2.424101in}{-156.451171in}}%
\pgfpathclose%
\pgfusepath{stroke,fill}%
\end{pgfscope}%
\begin{pgfscope}%
\pgfpathrectangle{\pgfqpoint{0.592976in}{0.400938in}}{\pgfqpoint{4.172183in}{1.315510in}}%
\pgfusepath{clip}%
\pgfsetbuttcap%
\pgfsetmiterjoin%
\definecolor{currentfill}{rgb}{0.460784,0.749020,0.443137}%
\pgfsetfillcolor{currentfill}%
\pgfsetlinewidth{1.003750pt}%
\definecolor{currentstroke}{rgb}{1.000000,1.000000,1.000000}%
\pgfsetstrokecolor{currentstroke}%
\pgfsetdash{}{0pt}%
\pgfpathmoveto{\pgfqpoint{3.119465in}{-156.451171in}}%
\pgfpathlineto{\pgfqpoint{3.304895in}{-156.451171in}}%
\pgfpathlineto{\pgfqpoint{3.304895in}{0.921623in}}%
\pgfpathlineto{\pgfqpoint{3.119465in}{0.921623in}}%
\pgfpathlineto{\pgfqpoint{3.119465in}{-156.451171in}}%
\pgfpathclose%
\pgfusepath{stroke,fill}%
\end{pgfscope}%
\begin{pgfscope}%
\pgfpathrectangle{\pgfqpoint{0.592976in}{0.400938in}}{\pgfqpoint{4.172183in}{1.315510in}}%
\pgfusepath{clip}%
\pgfsetbuttcap%
\pgfsetmiterjoin%
\definecolor{currentfill}{rgb}{0.460784,0.749020,0.443137}%
\pgfsetfillcolor{currentfill}%
\pgfsetlinewidth{1.003750pt}%
\definecolor{currentstroke}{rgb}{1.000000,1.000000,1.000000}%
\pgfsetstrokecolor{currentstroke}%
\pgfsetdash{}{0pt}%
\pgfpathmoveto{\pgfqpoint{3.814829in}{-156.451171in}}%
\pgfpathlineto{\pgfqpoint{4.000259in}{-156.451171in}}%
\pgfpathlineto{\pgfqpoint{4.000259in}{1.190598in}}%
\pgfpathlineto{\pgfqpoint{3.814829in}{1.190598in}}%
\pgfpathlineto{\pgfqpoint{3.814829in}{-156.451171in}}%
\pgfpathclose%
\pgfusepath{stroke,fill}%
\end{pgfscope}%
\begin{pgfscope}%
\pgfpathrectangle{\pgfqpoint{0.592976in}{0.400938in}}{\pgfqpoint{4.172183in}{1.315510in}}%
\pgfusepath{clip}%
\pgfsetbuttcap%
\pgfsetmiterjoin%
\definecolor{currentfill}{rgb}{0.460784,0.749020,0.443137}%
\pgfsetfillcolor{currentfill}%
\pgfsetlinewidth{1.003750pt}%
\definecolor{currentstroke}{rgb}{1.000000,1.000000,1.000000}%
\pgfsetstrokecolor{currentstroke}%
\pgfsetdash{}{0pt}%
\pgfpathmoveto{\pgfqpoint{4.510193in}{-156.451171in}}%
\pgfpathlineto{\pgfqpoint{4.695623in}{-156.451171in}}%
\pgfpathlineto{\pgfqpoint{4.695623in}{0.906274in}}%
\pgfpathlineto{\pgfqpoint{4.510193in}{0.906274in}}%
\pgfpathlineto{\pgfqpoint{4.510193in}{-156.451171in}}%
\pgfpathclose%
\pgfusepath{stroke,fill}%
\end{pgfscope}%
\begin{pgfscope}%
\pgfsetrectcap%
\pgfsetmiterjoin%
\pgfsetlinewidth{1.254687pt}%
\definecolor{currentstroke}{rgb}{0.800000,0.800000,0.800000}%
\pgfsetstrokecolor{currentstroke}%
\pgfsetdash{}{0pt}%
\pgfpathmoveto{\pgfqpoint{0.592976in}{0.400938in}}%
\pgfpathlineto{\pgfqpoint{0.592976in}{1.716449in}}%
\pgfusepath{stroke}%
\end{pgfscope}%
\begin{pgfscope}%
\pgfsetrectcap%
\pgfsetmiterjoin%
\pgfsetlinewidth{1.254687pt}%
\definecolor{currentstroke}{rgb}{0.800000,0.800000,0.800000}%
\pgfsetstrokecolor{currentstroke}%
\pgfsetdash{}{0pt}%
\pgfpathmoveto{\pgfqpoint{4.765159in}{0.400938in}}%
\pgfpathlineto{\pgfqpoint{4.765159in}{1.716449in}}%
\pgfusepath{stroke}%
\end{pgfscope}%
\begin{pgfscope}%
\pgfsetrectcap%
\pgfsetmiterjoin%
\pgfsetlinewidth{1.254687pt}%
\definecolor{currentstroke}{rgb}{0.800000,0.800000,0.800000}%
\pgfsetstrokecolor{currentstroke}%
\pgfsetdash{}{0pt}%
\pgfpathmoveto{\pgfqpoint{0.592976in}{0.400938in}}%
\pgfpathlineto{\pgfqpoint{4.765159in}{0.400938in}}%
\pgfusepath{stroke}%
\end{pgfscope}%
\begin{pgfscope}%
\pgfsetrectcap%
\pgfsetmiterjoin%
\pgfsetlinewidth{1.254687pt}%
\definecolor{currentstroke}{rgb}{0.800000,0.800000,0.800000}%
\pgfsetstrokecolor{currentstroke}%
\pgfsetdash{}{0pt}%
\pgfpathmoveto{\pgfqpoint{0.592976in}{1.716449in}}%
\pgfpathlineto{\pgfqpoint{4.765159in}{1.716449in}}%
\pgfusepath{stroke}%
\end{pgfscope}%
\begin{pgfscope}%
\pgfsetbuttcap%
\pgfsetmiterjoin%
\definecolor{currentfill}{rgb}{1.000000,1.000000,1.000000}%
\pgfsetfillcolor{currentfill}%
\pgfsetfillopacity{0.800000}%
\pgfsetlinewidth{1.003750pt}%
\definecolor{currentstroke}{rgb}{0.800000,0.800000,0.800000}%
\pgfsetstrokecolor{currentstroke}%
\pgfsetstrokeopacity{0.800000}%
\pgfsetdash{}{0pt}%
\pgfpathmoveto{\pgfqpoint{4.956964in}{0.664944in}}%
\pgfpathlineto{\pgfqpoint{6.076690in}{0.664944in}}%
\pgfpathquadraticcurveto{\pgfqpoint{6.101690in}{0.664944in}}{\pgfqpoint{6.101690in}{0.689944in}}%
\pgfpathlineto{\pgfqpoint{6.101690in}{1.427443in}}%
\pgfpathquadraticcurveto{\pgfqpoint{6.101690in}{1.452443in}}{\pgfqpoint{6.076690in}{1.452443in}}%
\pgfpathlineto{\pgfqpoint{4.956964in}{1.452443in}}%
\pgfpathquadraticcurveto{\pgfqpoint{4.931964in}{1.452443in}}{\pgfqpoint{4.931964in}{1.427443in}}%
\pgfpathlineto{\pgfqpoint{4.931964in}{0.689944in}}%
\pgfpathquadraticcurveto{\pgfqpoint{4.931964in}{0.664944in}}{\pgfqpoint{4.956964in}{0.664944in}}%
\pgfpathlineto{\pgfqpoint{4.956964in}{0.664944in}}%
\pgfpathclose%
\pgfusepath{stroke,fill}%
\end{pgfscope}%
\begin{pgfscope}%
\definecolor{textcolor}{rgb}{0.150000,0.150000,0.150000}%
\pgfsetstrokecolor{textcolor}%
\pgfsetfillcolor{textcolor}%
\pgftext[x=5.303723in,y=1.307472in,left,base]{\color{textcolor}\sffamily\fontsize{9.000000}{10.800000}\selectfont Dataset}%
\end{pgfscope}%
\begin{pgfscope}%
\pgfsetbuttcap%
\pgfsetmiterjoin%
\definecolor{currentfill}{rgb}{0.349020,0.490196,0.749020}%
\pgfsetfillcolor{currentfill}%
\pgfsetlinewidth{1.003750pt}%
\definecolor{currentstroke}{rgb}{1.000000,1.000000,1.000000}%
\pgfsetstrokecolor{currentstroke}%
\pgfsetdash{}{0pt}%
\pgfpathmoveto{\pgfqpoint{4.981964in}{1.119973in}}%
\pgfpathlineto{\pgfqpoint{5.231964in}{1.119973in}}%
\pgfpathlineto{\pgfqpoint{5.231964in}{1.207473in}}%
\pgfpathlineto{\pgfqpoint{4.981964in}{1.207473in}}%
\pgfpathlineto{\pgfqpoint{4.981964in}{1.119973in}}%
\pgfpathclose%
\pgfusepath{stroke,fill}%
\end{pgfscope}%
\begin{pgfscope}%
\definecolor{textcolor}{rgb}{0.150000,0.150000,0.150000}%
\pgfsetstrokecolor{textcolor}%
\pgfsetfillcolor{textcolor}%
\pgftext[x=5.331964in,y=1.119973in,left,base]{\color{textcolor}\sffamily\fontsize{9.000000}{10.800000}\selectfont Normal}%
\end{pgfscope}%
\begin{pgfscope}%
\pgfsetbuttcap%
\pgfsetmiterjoin%
\definecolor{currentfill}{rgb}{0.852941,0.544118,0.370588}%
\pgfsetfillcolor{currentfill}%
\pgfsetlinewidth{1.003750pt}%
\definecolor{currentstroke}{rgb}{1.000000,1.000000,1.000000}%
\pgfsetstrokecolor{currentstroke}%
\pgfsetdash{}{0pt}%
\pgfpathmoveto{\pgfqpoint{4.981964in}{0.932473in}}%
\pgfpathlineto{\pgfqpoint{5.231964in}{0.932473in}}%
\pgfpathlineto{\pgfqpoint{5.231964in}{1.019973in}}%
\pgfpathlineto{\pgfqpoint{4.981964in}{1.019973in}}%
\pgfpathlineto{\pgfqpoint{4.981964in}{0.932473in}}%
\pgfpathclose%
\pgfusepath{stroke,fill}%
\end{pgfscope}%
\begin{pgfscope}%
\definecolor{textcolor}{rgb}{0.150000,0.150000,0.150000}%
\pgfsetstrokecolor{textcolor}%
\pgfsetfillcolor{textcolor}%
\pgftext[x=5.331964in,y=0.932473in,left,base]{\color{textcolor}\sffamily\fontsize{9.000000}{10.800000}\selectfont No roads}%
\end{pgfscope}%
\begin{pgfscope}%
\pgfsetbuttcap%
\pgfsetmiterjoin%
\definecolor{currentfill}{rgb}{0.460784,0.749020,0.443137}%
\pgfsetfillcolor{currentfill}%
\pgfsetlinewidth{1.003750pt}%
\definecolor{currentstroke}{rgb}{1.000000,1.000000,1.000000}%
\pgfsetstrokecolor{currentstroke}%
\pgfsetdash{}{0pt}%
\pgfpathmoveto{\pgfqpoint{4.981964in}{0.744973in}}%
\pgfpathlineto{\pgfqpoint{5.231964in}{0.744973in}}%
\pgfpathlineto{\pgfqpoint{5.231964in}{0.832473in}}%
\pgfpathlineto{\pgfqpoint{4.981964in}{0.832473in}}%
\pgfpathlineto{\pgfqpoint{4.981964in}{0.744973in}}%
\pgfpathclose%
\pgfusepath{stroke,fill}%
\end{pgfscope}%
\begin{pgfscope}%
\definecolor{textcolor}{rgb}{0.150000,0.150000,0.150000}%
\pgfsetstrokecolor{textcolor}%
\pgfsetfillcolor{textcolor}%
\pgftext[x=5.331964in,y=0.744973in,left,base]{\color{textcolor}\sffamily\fontsize{9.000000}{10.800000}\selectfont No obstacles}%
\end{pgfscope}%
\end{pgfpicture}%
\makeatother%
\endgroup%

				\end{figcenter}
				\caption[Graph generation time comparison of normal, no-road and no-obstacle \enquote{OSM city} datasets.]{Comparison of the normal 4km\textsuperscript{2} \enquote{OSM city} dataset (blue) with the same dataset but without roads (orange) and without obstacles (green).}
				\label{fig:eval-import-osm-no-roads-obstacles-city}
			\end{figure}
			
			\noindent
			\Cref{fig:eval-import-osm-no-roads-obstacles-city} lists and illustrates the results for the \enquote{OSM city} datasets.
			Some noteworthy insights can be inferred from these data:
			\begin{itemize}
				\item The kNN search and visibility graph creation times were significantly reduced for the dataset without obstacles, which is expected, since both operations take only obstacles into account.
				\item The merge operation for the dataset without roads took longer compared to the dataset without obstacles.
				This is due to the 175 remaining building passages within the \enquote{no roads} dataset, which means this operation still merged road edges and therefore required some time.
				However, compared to the normal dataset, the merge operation of the \enquote{no road} dataset only required 25.63\% of the time, which does not match the number of remaining road edges of 4.82\% (175 edges) compared to the normal dataset (3629 edges).\\
				Because building passages consist of short segments, their start and end vertices (dead-end vertices) need to be connected to the rest of the graph.
				This requires the creation of visibility edges, which is an expensive operation.
				The normal dataset contains 742 and the dataset without roads (and only building passages) still 346 such dead-end vertices, which yields the relatively long merge operation time.
				\item In the \enquote{no obstacles} dataset the merge operation took 0.4\% of the normal time. It is a low but expected value, because no visibility edges exist and therefore no merge of edges took place.
			\end{itemize}
			
			\begin{figure}[h!]
				\begin{figcenter}
					\begin{tabularx}{0.95\textwidth}{p{2.85cm}>{\raggedleft\arraybackslash}p{1.95cm}RR>{\raggedleft\arraybackslash}p{2.15cm}R}
\toprule
\textbf{Operation}	& \textbf{Normal}	& \textbf{No roads}	& \textbf{Decrease compared to normal}	& \textbf{No obstacles}	& \textbf{Decrease compared to normal}	\\
\midrule
kNN search			&  5,957.97 ms		& 5,924.27 ms		&   0.57\%								&  0.01 ms				& 99.99\%								\\
Create graph		&    191.68 ms		&   210.40 ms		&  -9.76\%								&  0.01 ms				& 99.99\%								\\
Get obstacles		&     42.53 ms		&    40.76 ms		&   4.15\%								&  0.78 ms				& 98.16\%								\\
Merge road edges	&  6,599.78 ms		&     5.67 ms		&  99.91\%								& 32.79 ms				& 99.50\%								\\
Add POI attributes	&      4.83 ms		&     1.53 ms		&  68.21\%								&  0.42 ms				& 91.26\%								\\
\midrule
Total time			& 12,853.51 ms		& 6,201.96 ms		&  51.75\%								& 37.19 ms				& 99.71\%								\\
\bottomrule
					\end{tabularx}
					\captionof{table}{Measurements for the 4km\textsuperscript{2} \enquote{OSM rural} normal, no roads and no obstacles datasets.}
				\end{figcenter}
				\vspace{3ex}
				\begin{figcenter}
					\begingroup%
\makeatletter%
\begin{pgfpicture}%
\pgfpathrectangle{\pgfpointorigin}{\pgfqpoint{6.101690in}{1.716449in}}%
\pgfusepath{use as bounding box}%
\begin{pgfscope}%
\pgfsetbuttcap%
\pgfsetmiterjoin%
\definecolor{currentfill}{rgb}{1.000000,1.000000,1.000000}%
\pgfsetfillcolor{currentfill}%
\pgfsetlinewidth{0.000000pt}%
\definecolor{currentstroke}{rgb}{1.000000,1.000000,1.000000}%
\pgfsetstrokecolor{currentstroke}%
\pgfsetdash{}{0pt}%
\pgfpathmoveto{\pgfqpoint{0.000000in}{0.000000in}}%
\pgfpathlineto{\pgfqpoint{6.101690in}{0.000000in}}%
\pgfpathlineto{\pgfqpoint{6.101690in}{1.716449in}}%
\pgfpathlineto{\pgfqpoint{0.000000in}{1.716449in}}%
\pgfpathlineto{\pgfqpoint{0.000000in}{0.000000in}}%
\pgfpathclose%
\pgfusepath{fill}%
\end{pgfscope}%
\begin{pgfscope}%
\pgfsetbuttcap%
\pgfsetmiterjoin%
\definecolor{currentfill}{rgb}{1.000000,1.000000,1.000000}%
\pgfsetfillcolor{currentfill}%
\pgfsetlinewidth{0.000000pt}%
\definecolor{currentstroke}{rgb}{0.000000,0.000000,0.000000}%
\pgfsetstrokecolor{currentstroke}%
\pgfsetstrokeopacity{0.000000}%
\pgfsetdash{}{0pt}%
\pgfpathmoveto{\pgfqpoint{0.592976in}{0.400938in}}%
\pgfpathlineto{\pgfqpoint{4.765159in}{0.400938in}}%
\pgfpathlineto{\pgfqpoint{4.765159in}{1.716449in}}%
\pgfpathlineto{\pgfqpoint{0.592976in}{1.716449in}}%
\pgfpathlineto{\pgfqpoint{0.592976in}{0.400938in}}%
\pgfpathclose%
\pgfusepath{fill}%
\end{pgfscope}%
\begin{pgfscope}%
\definecolor{textcolor}{rgb}{0.150000,0.150000,0.150000}%
\pgfsetstrokecolor{textcolor}%
\pgfsetfillcolor{textcolor}%
\pgftext[x=0.940658in,y=0.268994in,,top]{\color{textcolor}\sffamily\fontsize{9.000000}{10.800000}\selectfont Total time}%
\end{pgfscope}%
\begin{pgfscope}%
\definecolor{textcolor}{rgb}{0.150000,0.150000,0.150000}%
\pgfsetstrokecolor{textcolor}%
\pgfsetfillcolor{textcolor}%
\pgftext[x=1.636022in,y=0.268994in,,top]{\color{textcolor}\sffamily\fontsize{9.000000}{10.800000}\selectfont kNN search}%
\end{pgfscope}%
\begin{pgfscope}%
\definecolor{textcolor}{rgb}{0.150000,0.150000,0.150000}%
\pgfsetstrokecolor{textcolor}%
\pgfsetfillcolor{textcolor}%
\pgftext[x=2.147517in, y=0.174023in, left, base]{\color{textcolor}\sffamily\fontsize{9.000000}{10.800000}\selectfont Create}%
\end{pgfscope}%
\begin{pgfscope}%
\definecolor{textcolor}{rgb}{0.150000,0.150000,0.150000}%
\pgfsetstrokecolor{textcolor}%
\pgfsetfillcolor{textcolor}%
\pgftext[x=2.168086in, y=0.030029in, left, base]{\color{textcolor}\sffamily\fontsize{9.000000}{10.800000}\selectfont graph}%
\end{pgfscope}%
\begin{pgfscope}%
\definecolor{textcolor}{rgb}{0.150000,0.150000,0.150000}%
\pgfsetstrokecolor{textcolor}%
\pgfsetfillcolor{textcolor}%
\pgftext[x=2.929002in, y=0.174023in, left, base]{\color{textcolor}\sffamily\fontsize{9.000000}{10.800000}\selectfont Get}%
\end{pgfscope}%
\begin{pgfscope}%
\definecolor{textcolor}{rgb}{0.150000,0.150000,0.150000}%
\pgfsetstrokecolor{textcolor}%
\pgfsetfillcolor{textcolor}%
\pgftext[x=2.764756in, y=0.030029in, left, base]{\color{textcolor}\sffamily\fontsize{9.000000}{10.800000}\selectfont obstacles}%
\end{pgfscope}%
\begin{pgfscope}%
\definecolor{textcolor}{rgb}{0.150000,0.150000,0.150000}%
\pgfsetstrokecolor{textcolor}%
\pgfsetfillcolor{textcolor}%
\pgftext[x=3.397101in, y=0.174023in, left, base]{\color{textcolor}\sffamily\fontsize{9.000000}{10.800000}\selectfont Merge road}%
\end{pgfscope}%
\begin{pgfscope}%
\definecolor{textcolor}{rgb}{0.150000,0.150000,0.150000}%
\pgfsetstrokecolor{textcolor}%
\pgfsetfillcolor{textcolor}%
\pgftext[x=3.558020in, y=0.030029in, left, base]{\color{textcolor}\sffamily\fontsize{9.000000}{10.800000}\selectfont edges}%
\end{pgfscope}%
\begin{pgfscope}%
\definecolor{textcolor}{rgb}{0.150000,0.150000,0.150000}%
\pgfsetstrokecolor{textcolor}%
\pgfsetfillcolor{textcolor}%
\pgftext[x=4.187039in, y=0.174023in, left, base]{\color{textcolor}\sffamily\fontsize{9.000000}{10.800000}\selectfont Add POI}%
\end{pgfscope}%
\begin{pgfscope}%
\definecolor{textcolor}{rgb}{0.150000,0.150000,0.150000}%
\pgfsetstrokecolor{textcolor}%
\pgfsetfillcolor{textcolor}%
\pgftext[x=4.143979in, y=0.030029in, left, base]{\color{textcolor}\sffamily\fontsize{9.000000}{10.800000}\selectfont attributes}%
\end{pgfscope}%
\begin{pgfscope}%
\pgfpathrectangle{\pgfqpoint{0.592976in}{0.400938in}}{\pgfqpoint{4.172183in}{1.315510in}}%
\pgfusepath{clip}%
\pgfsetroundcap%
\pgfsetroundjoin%
\pgfsetlinewidth{1.003750pt}%
\definecolor{currentstroke}{rgb}{0.800000,0.800000,0.800000}%
\pgfsetstrokecolor{currentstroke}%
\pgfsetdash{}{0pt}%
\pgfpathmoveto{\pgfqpoint{0.592976in}{0.609452in}}%
\pgfpathlineto{\pgfqpoint{4.765159in}{0.609452in}}%
\pgfusepath{stroke}%
\end{pgfscope}%
\begin{pgfscope}%
\definecolor{textcolor}{rgb}{0.150000,0.150000,0.150000}%
\pgfsetstrokecolor{textcolor}%
\pgfsetfillcolor{textcolor}%
\pgftext[x=0.194444in, y=0.561967in, left, base]{\color{textcolor}\sffamily\fontsize{9.000000}{10.800000}\selectfont \(\displaystyle {10^{-4}}\)}%
\end{pgfscope}%
\begin{pgfscope}%
\pgfpathrectangle{\pgfqpoint{0.592976in}{0.400938in}}{\pgfqpoint{4.172183in}{1.315510in}}%
\pgfusepath{clip}%
\pgfsetroundcap%
\pgfsetroundjoin%
\pgfsetlinewidth{1.003750pt}%
\definecolor{currentstroke}{rgb}{0.800000,0.800000,0.800000}%
\pgfsetstrokecolor{currentstroke}%
\pgfsetdash{}{0pt}%
\pgfpathmoveto{\pgfqpoint{0.592976in}{0.924835in}}%
\pgfpathlineto{\pgfqpoint{4.765159in}{0.924835in}}%
\pgfusepath{stroke}%
\end{pgfscope}%
\begin{pgfscope}%
\definecolor{textcolor}{rgb}{0.150000,0.150000,0.150000}%
\pgfsetstrokecolor{textcolor}%
\pgfsetfillcolor{textcolor}%
\pgftext[x=0.194444in, y=0.877350in, left, base]{\color{textcolor}\sffamily\fontsize{9.000000}{10.800000}\selectfont \(\displaystyle {10^{-2}}\)}%
\end{pgfscope}%
\begin{pgfscope}%
\pgfpathrectangle{\pgfqpoint{0.592976in}{0.400938in}}{\pgfqpoint{4.172183in}{1.315510in}}%
\pgfusepath{clip}%
\pgfsetroundcap%
\pgfsetroundjoin%
\pgfsetlinewidth{1.003750pt}%
\definecolor{currentstroke}{rgb}{0.800000,0.800000,0.800000}%
\pgfsetstrokecolor{currentstroke}%
\pgfsetdash{}{0pt}%
\pgfpathmoveto{\pgfqpoint{0.592976in}{1.240218in}}%
\pgfpathlineto{\pgfqpoint{4.765159in}{1.240218in}}%
\pgfusepath{stroke}%
\end{pgfscope}%
\begin{pgfscope}%
\definecolor{textcolor}{rgb}{0.150000,0.150000,0.150000}%
\pgfsetstrokecolor{textcolor}%
\pgfsetfillcolor{textcolor}%
\pgftext[x=0.274690in, y=1.192732in, left, base]{\color{textcolor}\sffamily\fontsize{9.000000}{10.800000}\selectfont \(\displaystyle {10^{0}}\)}%
\end{pgfscope}%
\begin{pgfscope}%
\pgfpathrectangle{\pgfqpoint{0.592976in}{0.400938in}}{\pgfqpoint{4.172183in}{1.315510in}}%
\pgfusepath{clip}%
\pgfsetroundcap%
\pgfsetroundjoin%
\pgfsetlinewidth{1.003750pt}%
\definecolor{currentstroke}{rgb}{0.800000,0.800000,0.800000}%
\pgfsetstrokecolor{currentstroke}%
\pgfsetdash{}{0pt}%
\pgfpathmoveto{\pgfqpoint{0.592976in}{1.555601in}}%
\pgfpathlineto{\pgfqpoint{4.765159in}{1.555601in}}%
\pgfusepath{stroke}%
\end{pgfscope}%
\begin{pgfscope}%
\definecolor{textcolor}{rgb}{0.150000,0.150000,0.150000}%
\pgfsetstrokecolor{textcolor}%
\pgfsetfillcolor{textcolor}%
\pgftext[x=0.274690in, y=1.508115in, left, base]{\color{textcolor}\sffamily\fontsize{9.000000}{10.800000}\selectfont \(\displaystyle {10^{2}}\)}%
\end{pgfscope}%
\begin{pgfscope}%
\definecolor{textcolor}{rgb}{0.150000,0.150000,0.150000}%
\pgfsetstrokecolor{textcolor}%
\pgfsetfillcolor{textcolor}%
\pgftext[x=0.125000in,y=1.058693in,,bottom,rotate=90.000000]{\color{textcolor}\sffamily\fontsize{9.000000}{10.800000}\selectfont Time in s}%
\end{pgfscope}%
\begin{pgfscope}%
\pgfpathrectangle{\pgfqpoint{0.592976in}{0.400938in}}{\pgfqpoint{4.172183in}{1.315510in}}%
\pgfusepath{clip}%
\pgfsetbuttcap%
\pgfsetmiterjoin%
\definecolor{currentfill}{rgb}{0.349020,0.490196,0.749020}%
\pgfsetfillcolor{currentfill}%
\pgfsetlinewidth{1.003750pt}%
\definecolor{currentstroke}{rgb}{1.000000,1.000000,1.000000}%
\pgfsetstrokecolor{currentstroke}%
\pgfsetdash{}{0pt}%
\pgfpathmoveto{\pgfqpoint{0.662512in}{-156.451171in}}%
\pgfpathlineto{\pgfqpoint{0.847943in}{-156.451171in}}%
\pgfpathlineto{\pgfqpoint{0.847943in}{1.656653in}}%
\pgfpathlineto{\pgfqpoint{0.662512in}{1.656653in}}%
\pgfpathlineto{\pgfqpoint{0.662512in}{-156.451171in}}%
\pgfpathclose%
\pgfusepath{stroke,fill}%
\end{pgfscope}%
\begin{pgfscope}%
\pgfpathrectangle{\pgfqpoint{0.592976in}{0.400938in}}{\pgfqpoint{4.172183in}{1.315510in}}%
\pgfusepath{clip}%
\pgfsetbuttcap%
\pgfsetmiterjoin%
\definecolor{currentfill}{rgb}{0.349020,0.490196,0.749020}%
\pgfsetfillcolor{currentfill}%
\pgfsetlinewidth{1.003750pt}%
\definecolor{currentstroke}{rgb}{1.000000,1.000000,1.000000}%
\pgfsetstrokecolor{currentstroke}%
\pgfsetdash{}{0pt}%
\pgfpathmoveto{\pgfqpoint{1.357876in}{-156.451171in}}%
\pgfpathlineto{\pgfqpoint{1.543307in}{-156.451171in}}%
\pgfpathlineto{\pgfqpoint{1.543307in}{1.634366in}}%
\pgfpathlineto{\pgfqpoint{1.357876in}{1.634366in}}%
\pgfpathlineto{\pgfqpoint{1.357876in}{-156.451171in}}%
\pgfpathclose%
\pgfusepath{stroke,fill}%
\end{pgfscope}%
\begin{pgfscope}%
\pgfpathrectangle{\pgfqpoint{0.592976in}{0.400938in}}{\pgfqpoint{4.172183in}{1.315510in}}%
\pgfusepath{clip}%
\pgfsetbuttcap%
\pgfsetmiterjoin%
\definecolor{currentfill}{rgb}{0.349020,0.490196,0.749020}%
\pgfsetfillcolor{currentfill}%
\pgfsetlinewidth{1.003750pt}%
\definecolor{currentstroke}{rgb}{1.000000,1.000000,1.000000}%
\pgfsetstrokecolor{currentstroke}%
\pgfsetdash{}{0pt}%
\pgfpathmoveto{\pgfqpoint{2.053240in}{-156.451171in}}%
\pgfpathlineto{\pgfqpoint{2.238670in}{-156.451171in}}%
\pgfpathlineto{\pgfqpoint{2.238670in}{1.214516in}}%
\pgfpathlineto{\pgfqpoint{2.053240in}{1.214516in}}%
\pgfpathlineto{\pgfqpoint{2.053240in}{-156.451171in}}%
\pgfpathclose%
\pgfusepath{stroke,fill}%
\end{pgfscope}%
\begin{pgfscope}%
\pgfpathrectangle{\pgfqpoint{0.592976in}{0.400938in}}{\pgfqpoint{4.172183in}{1.315510in}}%
\pgfusepath{clip}%
\pgfsetbuttcap%
\pgfsetmiterjoin%
\definecolor{currentfill}{rgb}{0.349020,0.490196,0.749020}%
\pgfsetfillcolor{currentfill}%
\pgfsetlinewidth{1.003750pt}%
\definecolor{currentstroke}{rgb}{1.000000,1.000000,1.000000}%
\pgfsetstrokecolor{currentstroke}%
\pgfsetdash{}{0pt}%
\pgfpathmoveto{\pgfqpoint{2.748604in}{-156.451171in}}%
\pgfpathlineto{\pgfqpoint{2.934034in}{-156.451171in}}%
\pgfpathlineto{\pgfqpoint{2.934034in}{1.157810in}}%
\pgfpathlineto{\pgfqpoint{2.748604in}{1.157810in}}%
\pgfpathlineto{\pgfqpoint{2.748604in}{-156.451171in}}%
\pgfpathclose%
\pgfusepath{stroke,fill}%
\end{pgfscope}%
\begin{pgfscope}%
\pgfpathrectangle{\pgfqpoint{0.592976in}{0.400938in}}{\pgfqpoint{4.172183in}{1.315510in}}%
\pgfusepath{clip}%
\pgfsetbuttcap%
\pgfsetmiterjoin%
\definecolor{currentfill}{rgb}{0.349020,0.490196,0.749020}%
\pgfsetfillcolor{currentfill}%
\pgfsetlinewidth{1.003750pt}%
\definecolor{currentstroke}{rgb}{1.000000,1.000000,1.000000}%
\pgfsetstrokecolor{currentstroke}%
\pgfsetdash{}{0pt}%
\pgfpathmoveto{\pgfqpoint{3.443968in}{-156.451171in}}%
\pgfpathlineto{\pgfqpoint{3.629398in}{-156.451171in}}%
\pgfpathlineto{\pgfqpoint{3.629398in}{1.568158in}}%
\pgfpathlineto{\pgfqpoint{3.443968in}{1.568158in}}%
\pgfpathlineto{\pgfqpoint{3.443968in}{-156.451171in}}%
\pgfpathclose%
\pgfusepath{stroke,fill}%
\end{pgfscope}%
\begin{pgfscope}%
\pgfpathrectangle{\pgfqpoint{0.592976in}{0.400938in}}{\pgfqpoint{4.172183in}{1.315510in}}%
\pgfusepath{clip}%
\pgfsetbuttcap%
\pgfsetmiterjoin%
\definecolor{currentfill}{rgb}{0.349020,0.490196,0.749020}%
\pgfsetfillcolor{currentfill}%
\pgfsetlinewidth{1.003750pt}%
\definecolor{currentstroke}{rgb}{1.000000,1.000000,1.000000}%
\pgfsetstrokecolor{currentstroke}%
\pgfsetdash{}{0pt}%
\pgfpathmoveto{\pgfqpoint{4.139332in}{-156.451171in}}%
\pgfpathlineto{\pgfqpoint{4.324762in}{-156.451171in}}%
\pgfpathlineto{\pgfqpoint{4.324762in}{1.040758in}}%
\pgfpathlineto{\pgfqpoint{4.139332in}{1.040758in}}%
\pgfpathlineto{\pgfqpoint{4.139332in}{-156.451171in}}%
\pgfpathclose%
\pgfusepath{stroke,fill}%
\end{pgfscope}%
\begin{pgfscope}%
\pgfpathrectangle{\pgfqpoint{0.592976in}{0.400938in}}{\pgfqpoint{4.172183in}{1.315510in}}%
\pgfusepath{clip}%
\pgfsetbuttcap%
\pgfsetmiterjoin%
\definecolor{currentfill}{rgb}{0.852941,0.544118,0.370588}%
\pgfsetfillcolor{currentfill}%
\pgfsetlinewidth{1.003750pt}%
\definecolor{currentstroke}{rgb}{1.000000,1.000000,1.000000}%
\pgfsetstrokecolor{currentstroke}%
\pgfsetdash{}{0pt}%
\pgfpathmoveto{\pgfqpoint{0.847943in}{-156.451171in}}%
\pgfpathlineto{\pgfqpoint{1.033373in}{-156.451171in}}%
\pgfpathlineto{\pgfqpoint{1.033373in}{1.635380in}}%
\pgfpathlineto{\pgfqpoint{0.847943in}{1.635380in}}%
\pgfpathlineto{\pgfqpoint{0.847943in}{-156.451171in}}%
\pgfpathclose%
\pgfusepath{stroke,fill}%
\end{pgfscope}%
\begin{pgfscope}%
\pgfpathrectangle{\pgfqpoint{0.592976in}{0.400938in}}{\pgfqpoint{4.172183in}{1.315510in}}%
\pgfusepath{clip}%
\pgfsetbuttcap%
\pgfsetmiterjoin%
\definecolor{currentfill}{rgb}{0.852941,0.544118,0.370588}%
\pgfsetfillcolor{currentfill}%
\pgfsetlinewidth{1.003750pt}%
\definecolor{currentstroke}{rgb}{1.000000,1.000000,1.000000}%
\pgfsetstrokecolor{currentstroke}%
\pgfsetdash{}{0pt}%
\pgfpathmoveto{\pgfqpoint{1.543307in}{-156.451171in}}%
\pgfpathlineto{\pgfqpoint{1.728737in}{-156.451171in}}%
\pgfpathlineto{\pgfqpoint{1.728737in}{1.628207in}}%
\pgfpathlineto{\pgfqpoint{1.543307in}{1.628207in}}%
\pgfpathlineto{\pgfqpoint{1.543307in}{-156.451171in}}%
\pgfpathclose%
\pgfusepath{stroke,fill}%
\end{pgfscope}%
\begin{pgfscope}%
\pgfpathrectangle{\pgfqpoint{0.592976in}{0.400938in}}{\pgfqpoint{4.172183in}{1.315510in}}%
\pgfusepath{clip}%
\pgfsetbuttcap%
\pgfsetmiterjoin%
\definecolor{currentfill}{rgb}{0.852941,0.544118,0.370588}%
\pgfsetfillcolor{currentfill}%
\pgfsetlinewidth{1.003750pt}%
\definecolor{currentstroke}{rgb}{1.000000,1.000000,1.000000}%
\pgfsetstrokecolor{currentstroke}%
\pgfsetdash{}{0pt}%
\pgfpathmoveto{\pgfqpoint{2.238670in}{-156.451171in}}%
\pgfpathlineto{\pgfqpoint{2.424101in}{-156.451171in}}%
\pgfpathlineto{\pgfqpoint{2.424101in}{1.213935in}}%
\pgfpathlineto{\pgfqpoint{2.238670in}{1.213935in}}%
\pgfpathlineto{\pgfqpoint{2.238670in}{-156.451171in}}%
\pgfpathclose%
\pgfusepath{stroke,fill}%
\end{pgfscope}%
\begin{pgfscope}%
\pgfpathrectangle{\pgfqpoint{0.592976in}{0.400938in}}{\pgfqpoint{4.172183in}{1.315510in}}%
\pgfusepath{clip}%
\pgfsetbuttcap%
\pgfsetmiterjoin%
\definecolor{currentfill}{rgb}{0.852941,0.544118,0.370588}%
\pgfsetfillcolor{currentfill}%
\pgfsetlinewidth{1.003750pt}%
\definecolor{currentstroke}{rgb}{1.000000,1.000000,1.000000}%
\pgfsetstrokecolor{currentstroke}%
\pgfsetdash{}{0pt}%
\pgfpathmoveto{\pgfqpoint{2.934034in}{-156.451171in}}%
\pgfpathlineto{\pgfqpoint{3.119465in}{-156.451171in}}%
\pgfpathlineto{\pgfqpoint{3.119465in}{1.155527in}}%
\pgfpathlineto{\pgfqpoint{2.934034in}{1.155527in}}%
\pgfpathlineto{\pgfqpoint{2.934034in}{-156.451171in}}%
\pgfpathclose%
\pgfusepath{stroke,fill}%
\end{pgfscope}%
\begin{pgfscope}%
\pgfpathrectangle{\pgfqpoint{0.592976in}{0.400938in}}{\pgfqpoint{4.172183in}{1.315510in}}%
\pgfusepath{clip}%
\pgfsetbuttcap%
\pgfsetmiterjoin%
\definecolor{currentfill}{rgb}{0.852941,0.544118,0.370588}%
\pgfsetfillcolor{currentfill}%
\pgfsetlinewidth{1.003750pt}%
\definecolor{currentstroke}{rgb}{1.000000,1.000000,1.000000}%
\pgfsetstrokecolor{currentstroke}%
\pgfsetdash{}{0pt}%
\pgfpathmoveto{\pgfqpoint{3.629398in}{-156.451171in}}%
\pgfpathlineto{\pgfqpoint{3.814829in}{-156.451171in}}%
\pgfpathlineto{\pgfqpoint{3.814829in}{1.474926in}}%
\pgfpathlineto{\pgfqpoint{3.629398in}{1.474926in}}%
\pgfpathlineto{\pgfqpoint{3.629398in}{-156.451171in}}%
\pgfpathclose%
\pgfusepath{stroke,fill}%
\end{pgfscope}%
\begin{pgfscope}%
\pgfpathrectangle{\pgfqpoint{0.592976in}{0.400938in}}{\pgfqpoint{4.172183in}{1.315510in}}%
\pgfusepath{clip}%
\pgfsetbuttcap%
\pgfsetmiterjoin%
\definecolor{currentfill}{rgb}{0.852941,0.544118,0.370588}%
\pgfsetfillcolor{currentfill}%
\pgfsetlinewidth{1.003750pt}%
\definecolor{currentstroke}{rgb}{1.000000,1.000000,1.000000}%
\pgfsetstrokecolor{currentstroke}%
\pgfsetdash{}{0pt}%
\pgfpathmoveto{\pgfqpoint{4.324762in}{-156.451171in}}%
\pgfpathlineto{\pgfqpoint{4.510193in}{-156.451171in}}%
\pgfpathlineto{\pgfqpoint{4.510193in}{0.987860in}}%
\pgfpathlineto{\pgfqpoint{4.324762in}{0.987860in}}%
\pgfpathlineto{\pgfqpoint{4.324762in}{-156.451171in}}%
\pgfpathclose%
\pgfusepath{stroke,fill}%
\end{pgfscope}%
\begin{pgfscope}%
\pgfpathrectangle{\pgfqpoint{0.592976in}{0.400938in}}{\pgfqpoint{4.172183in}{1.315510in}}%
\pgfusepath{clip}%
\pgfsetbuttcap%
\pgfsetmiterjoin%
\definecolor{currentfill}{rgb}{0.460784,0.749020,0.443137}%
\pgfsetfillcolor{currentfill}%
\pgfsetlinewidth{1.003750pt}%
\definecolor{currentstroke}{rgb}{1.000000,1.000000,1.000000}%
\pgfsetstrokecolor{currentstroke}%
\pgfsetdash{}{0pt}%
\pgfpathmoveto{\pgfqpoint{1.033373in}{-156.451171in}}%
\pgfpathlineto{\pgfqpoint{1.218803in}{-156.451171in}}%
\pgfpathlineto{\pgfqpoint{1.218803in}{1.196169in}}%
\pgfpathlineto{\pgfqpoint{1.033373in}{1.196169in}}%
\pgfpathlineto{\pgfqpoint{1.033373in}{-156.451171in}}%
\pgfpathclose%
\pgfusepath{stroke,fill}%
\end{pgfscope}%
\begin{pgfscope}%
\pgfpathrectangle{\pgfqpoint{0.592976in}{0.400938in}}{\pgfqpoint{4.172183in}{1.315510in}}%
\pgfusepath{clip}%
\pgfsetbuttcap%
\pgfsetmiterjoin%
\definecolor{currentfill}{rgb}{0.460784,0.749020,0.443137}%
\pgfsetfillcolor{currentfill}%
\pgfsetlinewidth{1.003750pt}%
\definecolor{currentstroke}{rgb}{1.000000,1.000000,1.000000}%
\pgfsetstrokecolor{currentstroke}%
\pgfsetdash{}{0pt}%
\pgfpathmoveto{\pgfqpoint{1.728737in}{-156.451171in}}%
\pgfpathlineto{\pgfqpoint{1.914167in}{-156.451171in}}%
\pgfpathlineto{\pgfqpoint{1.914167in}{0.460734in}}%
\pgfpathlineto{\pgfqpoint{1.728737in}{0.460734in}}%
\pgfpathlineto{\pgfqpoint{1.728737in}{-156.451171in}}%
\pgfpathclose%
\pgfusepath{stroke,fill}%
\end{pgfscope}%
\begin{pgfscope}%
\pgfpathrectangle{\pgfqpoint{0.592976in}{0.400938in}}{\pgfqpoint{4.172183in}{1.315510in}}%
\pgfusepath{clip}%
\pgfsetbuttcap%
\pgfsetmiterjoin%
\definecolor{currentfill}{rgb}{0.460784,0.749020,0.443137}%
\pgfsetfillcolor{currentfill}%
\pgfsetlinewidth{1.003750pt}%
\definecolor{currentstroke}{rgb}{1.000000,1.000000,1.000000}%
\pgfsetstrokecolor{currentstroke}%
\pgfsetdash{}{0pt}%
\pgfpathmoveto{\pgfqpoint{2.424101in}{-156.451171in}}%
\pgfpathlineto{\pgfqpoint{2.609531in}{-156.451171in}}%
\pgfpathlineto{\pgfqpoint{2.609531in}{0.473819in}}%
\pgfpathlineto{\pgfqpoint{2.424101in}{0.473819in}}%
\pgfpathlineto{\pgfqpoint{2.424101in}{-156.451171in}}%
\pgfpathclose%
\pgfusepath{stroke,fill}%
\end{pgfscope}%
\begin{pgfscope}%
\pgfpathrectangle{\pgfqpoint{0.592976in}{0.400938in}}{\pgfqpoint{4.172183in}{1.315510in}}%
\pgfusepath{clip}%
\pgfsetbuttcap%
\pgfsetmiterjoin%
\definecolor{currentfill}{rgb}{0.460784,0.749020,0.443137}%
\pgfsetfillcolor{currentfill}%
\pgfsetlinewidth{1.003750pt}%
\definecolor{currentstroke}{rgb}{1.000000,1.000000,1.000000}%
\pgfsetstrokecolor{currentstroke}%
\pgfsetdash{}{0pt}%
\pgfpathmoveto{\pgfqpoint{3.119465in}{-156.451171in}}%
\pgfpathlineto{\pgfqpoint{3.304895in}{-156.451171in}}%
\pgfpathlineto{\pgfqpoint{3.304895in}{0.921623in}}%
\pgfpathlineto{\pgfqpoint{3.119465in}{0.921623in}}%
\pgfpathlineto{\pgfqpoint{3.119465in}{-156.451171in}}%
\pgfpathclose%
\pgfusepath{stroke,fill}%
\end{pgfscope}%
\begin{pgfscope}%
\pgfpathrectangle{\pgfqpoint{0.592976in}{0.400938in}}{\pgfqpoint{4.172183in}{1.315510in}}%
\pgfusepath{clip}%
\pgfsetbuttcap%
\pgfsetmiterjoin%
\definecolor{currentfill}{rgb}{0.460784,0.749020,0.443137}%
\pgfsetfillcolor{currentfill}%
\pgfsetlinewidth{1.003750pt}%
\definecolor{currentstroke}{rgb}{1.000000,1.000000,1.000000}%
\pgfsetstrokecolor{currentstroke}%
\pgfsetdash{}{0pt}%
\pgfpathmoveto{\pgfqpoint{3.814829in}{-156.451171in}}%
\pgfpathlineto{\pgfqpoint{4.000259in}{-156.451171in}}%
\pgfpathlineto{\pgfqpoint{4.000259in}{1.190598in}}%
\pgfpathlineto{\pgfqpoint{3.814829in}{1.190598in}}%
\pgfpathlineto{\pgfqpoint{3.814829in}{-156.451171in}}%
\pgfpathclose%
\pgfusepath{stroke,fill}%
\end{pgfscope}%
\begin{pgfscope}%
\pgfpathrectangle{\pgfqpoint{0.592976in}{0.400938in}}{\pgfqpoint{4.172183in}{1.315510in}}%
\pgfusepath{clip}%
\pgfsetbuttcap%
\pgfsetmiterjoin%
\definecolor{currentfill}{rgb}{0.460784,0.749020,0.443137}%
\pgfsetfillcolor{currentfill}%
\pgfsetlinewidth{1.003750pt}%
\definecolor{currentstroke}{rgb}{1.000000,1.000000,1.000000}%
\pgfsetstrokecolor{currentstroke}%
\pgfsetdash{}{0pt}%
\pgfpathmoveto{\pgfqpoint{4.510193in}{-156.451171in}}%
\pgfpathlineto{\pgfqpoint{4.695623in}{-156.451171in}}%
\pgfpathlineto{\pgfqpoint{4.695623in}{0.906274in}}%
\pgfpathlineto{\pgfqpoint{4.510193in}{0.906274in}}%
\pgfpathlineto{\pgfqpoint{4.510193in}{-156.451171in}}%
\pgfpathclose%
\pgfusepath{stroke,fill}%
\end{pgfscope}%
\begin{pgfscope}%
\pgfsetrectcap%
\pgfsetmiterjoin%
\pgfsetlinewidth{1.254687pt}%
\definecolor{currentstroke}{rgb}{0.800000,0.800000,0.800000}%
\pgfsetstrokecolor{currentstroke}%
\pgfsetdash{}{0pt}%
\pgfpathmoveto{\pgfqpoint{0.592976in}{0.400938in}}%
\pgfpathlineto{\pgfqpoint{0.592976in}{1.716449in}}%
\pgfusepath{stroke}%
\end{pgfscope}%
\begin{pgfscope}%
\pgfsetrectcap%
\pgfsetmiterjoin%
\pgfsetlinewidth{1.254687pt}%
\definecolor{currentstroke}{rgb}{0.800000,0.800000,0.800000}%
\pgfsetstrokecolor{currentstroke}%
\pgfsetdash{}{0pt}%
\pgfpathmoveto{\pgfqpoint{4.765159in}{0.400938in}}%
\pgfpathlineto{\pgfqpoint{4.765159in}{1.716449in}}%
\pgfusepath{stroke}%
\end{pgfscope}%
\begin{pgfscope}%
\pgfsetrectcap%
\pgfsetmiterjoin%
\pgfsetlinewidth{1.254687pt}%
\definecolor{currentstroke}{rgb}{0.800000,0.800000,0.800000}%
\pgfsetstrokecolor{currentstroke}%
\pgfsetdash{}{0pt}%
\pgfpathmoveto{\pgfqpoint{0.592976in}{0.400938in}}%
\pgfpathlineto{\pgfqpoint{4.765159in}{0.400938in}}%
\pgfusepath{stroke}%
\end{pgfscope}%
\begin{pgfscope}%
\pgfsetrectcap%
\pgfsetmiterjoin%
\pgfsetlinewidth{1.254687pt}%
\definecolor{currentstroke}{rgb}{0.800000,0.800000,0.800000}%
\pgfsetstrokecolor{currentstroke}%
\pgfsetdash{}{0pt}%
\pgfpathmoveto{\pgfqpoint{0.592976in}{1.716449in}}%
\pgfpathlineto{\pgfqpoint{4.765159in}{1.716449in}}%
\pgfusepath{stroke}%
\end{pgfscope}%
\begin{pgfscope}%
\pgfsetbuttcap%
\pgfsetmiterjoin%
\definecolor{currentfill}{rgb}{1.000000,1.000000,1.000000}%
\pgfsetfillcolor{currentfill}%
\pgfsetfillopacity{0.800000}%
\pgfsetlinewidth{1.003750pt}%
\definecolor{currentstroke}{rgb}{0.800000,0.800000,0.800000}%
\pgfsetstrokecolor{currentstroke}%
\pgfsetstrokeopacity{0.800000}%
\pgfsetdash{}{0pt}%
\pgfpathmoveto{\pgfqpoint{4.956964in}{0.664944in}}%
\pgfpathlineto{\pgfqpoint{6.076690in}{0.664944in}}%
\pgfpathquadraticcurveto{\pgfqpoint{6.101690in}{0.664944in}}{\pgfqpoint{6.101690in}{0.689944in}}%
\pgfpathlineto{\pgfqpoint{6.101690in}{1.427443in}}%
\pgfpathquadraticcurveto{\pgfqpoint{6.101690in}{1.452443in}}{\pgfqpoint{6.076690in}{1.452443in}}%
\pgfpathlineto{\pgfqpoint{4.956964in}{1.452443in}}%
\pgfpathquadraticcurveto{\pgfqpoint{4.931964in}{1.452443in}}{\pgfqpoint{4.931964in}{1.427443in}}%
\pgfpathlineto{\pgfqpoint{4.931964in}{0.689944in}}%
\pgfpathquadraticcurveto{\pgfqpoint{4.931964in}{0.664944in}}{\pgfqpoint{4.956964in}{0.664944in}}%
\pgfpathlineto{\pgfqpoint{4.956964in}{0.664944in}}%
\pgfpathclose%
\pgfusepath{stroke,fill}%
\end{pgfscope}%
\begin{pgfscope}%
\definecolor{textcolor}{rgb}{0.150000,0.150000,0.150000}%
\pgfsetstrokecolor{textcolor}%
\pgfsetfillcolor{textcolor}%
\pgftext[x=5.303723in,y=1.307472in,left,base]{\color{textcolor}\sffamily\fontsize{9.000000}{10.800000}\selectfont Dataset}%
\end{pgfscope}%
\begin{pgfscope}%
\pgfsetbuttcap%
\pgfsetmiterjoin%
\definecolor{currentfill}{rgb}{0.349020,0.490196,0.749020}%
\pgfsetfillcolor{currentfill}%
\pgfsetlinewidth{1.003750pt}%
\definecolor{currentstroke}{rgb}{1.000000,1.000000,1.000000}%
\pgfsetstrokecolor{currentstroke}%
\pgfsetdash{}{0pt}%
\pgfpathmoveto{\pgfqpoint{4.981964in}{1.119973in}}%
\pgfpathlineto{\pgfqpoint{5.231964in}{1.119973in}}%
\pgfpathlineto{\pgfqpoint{5.231964in}{1.207473in}}%
\pgfpathlineto{\pgfqpoint{4.981964in}{1.207473in}}%
\pgfpathlineto{\pgfqpoint{4.981964in}{1.119973in}}%
\pgfpathclose%
\pgfusepath{stroke,fill}%
\end{pgfscope}%
\begin{pgfscope}%
\definecolor{textcolor}{rgb}{0.150000,0.150000,0.150000}%
\pgfsetstrokecolor{textcolor}%
\pgfsetfillcolor{textcolor}%
\pgftext[x=5.331964in,y=1.119973in,left,base]{\color{textcolor}\sffamily\fontsize{9.000000}{10.800000}\selectfont Normal}%
\end{pgfscope}%
\begin{pgfscope}%
\pgfsetbuttcap%
\pgfsetmiterjoin%
\definecolor{currentfill}{rgb}{0.852941,0.544118,0.370588}%
\pgfsetfillcolor{currentfill}%
\pgfsetlinewidth{1.003750pt}%
\definecolor{currentstroke}{rgb}{1.000000,1.000000,1.000000}%
\pgfsetstrokecolor{currentstroke}%
\pgfsetdash{}{0pt}%
\pgfpathmoveto{\pgfqpoint{4.981964in}{0.932473in}}%
\pgfpathlineto{\pgfqpoint{5.231964in}{0.932473in}}%
\pgfpathlineto{\pgfqpoint{5.231964in}{1.019973in}}%
\pgfpathlineto{\pgfqpoint{4.981964in}{1.019973in}}%
\pgfpathlineto{\pgfqpoint{4.981964in}{0.932473in}}%
\pgfpathclose%
\pgfusepath{stroke,fill}%
\end{pgfscope}%
\begin{pgfscope}%
\definecolor{textcolor}{rgb}{0.150000,0.150000,0.150000}%
\pgfsetstrokecolor{textcolor}%
\pgfsetfillcolor{textcolor}%
\pgftext[x=5.331964in,y=0.932473in,left,base]{\color{textcolor}\sffamily\fontsize{9.000000}{10.800000}\selectfont No roads}%
\end{pgfscope}%
\begin{pgfscope}%
\pgfsetbuttcap%
\pgfsetmiterjoin%
\definecolor{currentfill}{rgb}{0.460784,0.749020,0.443137}%
\pgfsetfillcolor{currentfill}%
\pgfsetlinewidth{1.003750pt}%
\definecolor{currentstroke}{rgb}{1.000000,1.000000,1.000000}%
\pgfsetstrokecolor{currentstroke}%
\pgfsetdash{}{0pt}%
\pgfpathmoveto{\pgfqpoint{4.981964in}{0.744973in}}%
\pgfpathlineto{\pgfqpoint{5.231964in}{0.744973in}}%
\pgfpathlineto{\pgfqpoint{5.231964in}{0.832473in}}%
\pgfpathlineto{\pgfqpoint{4.981964in}{0.832473in}}%
\pgfpathlineto{\pgfqpoint{4.981964in}{0.744973in}}%
\pgfpathclose%
\pgfusepath{stroke,fill}%
\end{pgfscope}%
\begin{pgfscope}%
\definecolor{textcolor}{rgb}{0.150000,0.150000,0.150000}%
\pgfsetstrokecolor{textcolor}%
\pgfsetfillcolor{textcolor}%
\pgftext[x=5.331964in,y=0.744973in,left,base]{\color{textcolor}\sffamily\fontsize{9.000000}{10.800000}\selectfont No obstacles}%
\end{pgfscope}%
\end{pgfpicture}%
\makeatother%
\endgroup%

				\end{figcenter}
				\caption[Graph generation time comparison of normal, no-road and no-obstacle \enquote{OSM rural} datasets.]{Comparison of the normal 4km\textsuperscript{2} \enquote{OSM rural} dataset (blue) with the same dataset but without roads (orange) and without obstacles (green).}
				\label{fig:eval-import-osm-no-roads-obstacles-rural}
			\end{figure}
			
			\noindent
			\Cref{fig:eval-import-osm-no-roads-obstacles-rural} lists and illustrates the results for the \enquote{OSM rural} datasets.
			There are some differences to the results of the \enquote{OSM city} dataset:
			\begin{itemize}
				\item The rural dataset contains no building passages and therefore shows the expected reduction in time for the merge operation.
				\item A comparison of the kNN search and road merge operations of the normal dataset shows that the merge operation actually took 10.77\% longer than the kNN search.
				Results of the normal \enquote{OSM city} dataset shows the opposite relation with 61.97\% less time required for the merge operation.
				A possible explanation is the ratio of road to visibility edges, which is 1:33 for the city dataset (33 visibility edges per road edge) and 1:264 for the rural dataset.
				This means that more intersections per road edges were processed using the rural dataset, which was actually the case:
				Using the city dataset the number of road edges increased due to splitting at intersections by a factor of 66 and within the rural dataset by a factor of 221.
				Therefore, the time spent on one road edge was significantly higher within the rural dataset and likely lead to the significant increase in time required for the merge operation.
				\item The graph creation time using the \enquote{no roads} dataset is especially noteworthy since it shows an \emph{increase} in processing time compared to the normal dataset.
				Because no algorithmic reason exists that would result in an increase of the graph generation time and because the measurements contains outliers, no further investigation took place to find an explanation for this increase.
			\end{itemize}
			Apart these differences and unique characteristics, both dataset categories show similar results for the kNN search, graph generation and overall processing time.
			First of all, removing all obstacles eliminates the main complexity and therefore significantly decreases the processing time.
			Decreasing their amount will therefore significantly reduce the processing time due to the overall quadratic runtime complexity of the kNN search.
			Second, removing roads does not affect the kNN search and graph creation but the total processing time decreases due to the faster merge operation.
			Third, even though it is not a significant part of the processing time in the first place, getting obstacles and adding attributes to POIs is also reduced when removing obstacles or roads.
	
	\subsection{Pattern-based datasets}
	
		\subsubsection{Import and graph generation}
		
			The pattern-based datasets contain repeating patterns of obstacles and thus have a much more regular structure compared to OSM-based datasets.
			This is also reflected in the results of the graph generation containing less deviations as shown in \Cref{fig:eval-import-pattern-abs}.
			Due to the absence of roads, the overall graph generation time is mainly determined by the kNN search.
			Only smaller datasets of up to an input vertex count of about 3,500 (depending on the dataset category) show a combined time of all other tasks of up to 5\% on the total graph generation time.
			All datasets show the already mentioned quadratic runtime complexity of the kNN search and no anomalies were observers during the overall graph generation.
			Detailed absolute and relative times for the maze dataset can be seen in \Cref{fig:eval-import-pattern-maze-abs-rel}.
			
			\begin{figure}[h]
				\begin{figcenter}
					\begingroup%
\makeatletter%
\begin{pgfpicture}%
\pgfpathrectangle{\pgfpointorigin}{\pgfqpoint{6.086490in}{2.410942in}}%
\pgfusepath{use as bounding box}%
\begin{pgfscope}%
\pgfsetbuttcap%
\pgfsetmiterjoin%
\definecolor{currentfill}{rgb}{1.000000,1.000000,1.000000}%
\pgfsetfillcolor{currentfill}%
\pgfsetlinewidth{0.000000pt}%
\definecolor{currentstroke}{rgb}{1.000000,1.000000,1.000000}%
\pgfsetstrokecolor{currentstroke}%
\pgfsetdash{}{0pt}%
\pgfpathmoveto{\pgfqpoint{0.000000in}{0.000000in}}%
\pgfpathlineto{\pgfqpoint{6.086490in}{0.000000in}}%
\pgfpathlineto{\pgfqpoint{6.086490in}{2.410942in}}%
\pgfpathlineto{\pgfqpoint{0.000000in}{2.410942in}}%
\pgfpathlineto{\pgfqpoint{0.000000in}{0.000000in}}%
\pgfpathclose%
\pgfusepath{fill}%
\end{pgfscope}%
\begin{pgfscope}%
\pgfsetbuttcap%
\pgfsetmiterjoin%
\definecolor{currentfill}{rgb}{1.000000,1.000000,1.000000}%
\pgfsetfillcolor{currentfill}%
\pgfsetlinewidth{0.000000pt}%
\definecolor{currentstroke}{rgb}{0.000000,0.000000,0.000000}%
\pgfsetstrokecolor{currentstroke}%
\pgfsetstrokeopacity{0.000000}%
\pgfsetdash{}{0pt}%
\pgfpathmoveto{\pgfqpoint{0.532932in}{0.451389in}}%
\pgfpathlineto{\pgfqpoint{2.036571in}{0.451389in}}%
\pgfpathlineto{\pgfqpoint{2.036571in}{2.204860in}}%
\pgfpathlineto{\pgfqpoint{0.532932in}{2.204860in}}%
\pgfpathlineto{\pgfqpoint{0.532932in}{0.451389in}}%
\pgfpathclose%
\pgfusepath{fill}%
\end{pgfscope}%
\begin{pgfscope}%
\pgfpathrectangle{\pgfqpoint{0.532932in}{0.451389in}}{\pgfqpoint{1.503640in}{1.753471in}}%
\pgfusepath{clip}%
\pgfsetroundcap%
\pgfsetroundjoin%
\pgfsetlinewidth{1.003750pt}%
\definecolor{currentstroke}{rgb}{0.800000,0.800000,0.800000}%
\pgfsetstrokecolor{currentstroke}%
\pgfsetdash{}{0pt}%
\pgfpathmoveto{\pgfqpoint{0.532932in}{0.451389in}}%
\pgfpathlineto{\pgfqpoint{0.532932in}{2.204860in}}%
\pgfusepath{stroke}%
\end{pgfscope}%
\begin{pgfscope}%
\definecolor{textcolor}{rgb}{0.150000,0.150000,0.150000}%
\pgfsetstrokecolor{textcolor}%
\pgfsetfillcolor{textcolor}%
\pgftext[x=0.532932in,y=0.319444in,,top]{\color{textcolor}\sffamily\fontsize{9.000000}{10.800000}\selectfont 0k}%
\end{pgfscope}%
\begin{pgfscope}%
\pgfpathrectangle{\pgfqpoint{0.532932in}{0.451389in}}{\pgfqpoint{1.503640in}{1.753471in}}%
\pgfusepath{clip}%
\pgfsetroundcap%
\pgfsetroundjoin%
\pgfsetlinewidth{1.003750pt}%
\definecolor{currentstroke}{rgb}{0.800000,0.800000,0.800000}%
\pgfsetstrokecolor{currentstroke}%
\pgfsetdash{}{0pt}%
\pgfpathmoveto{\pgfqpoint{0.988580in}{0.451389in}}%
\pgfpathlineto{\pgfqpoint{0.988580in}{2.204860in}}%
\pgfusepath{stroke}%
\end{pgfscope}%
\begin{pgfscope}%
\definecolor{textcolor}{rgb}{0.150000,0.150000,0.150000}%
\pgfsetstrokecolor{textcolor}%
\pgfsetfillcolor{textcolor}%
\pgftext[x=0.988580in,y=0.319444in,,top]{\color{textcolor}\sffamily\fontsize{9.000000}{10.800000}\selectfont 10k}%
\end{pgfscope}%
\begin{pgfscope}%
\pgfpathrectangle{\pgfqpoint{0.532932in}{0.451389in}}{\pgfqpoint{1.503640in}{1.753471in}}%
\pgfusepath{clip}%
\pgfsetroundcap%
\pgfsetroundjoin%
\pgfsetlinewidth{1.003750pt}%
\definecolor{currentstroke}{rgb}{0.800000,0.800000,0.800000}%
\pgfsetstrokecolor{currentstroke}%
\pgfsetdash{}{0pt}%
\pgfpathmoveto{\pgfqpoint{1.444229in}{0.451389in}}%
\pgfpathlineto{\pgfqpoint{1.444229in}{2.204860in}}%
\pgfusepath{stroke}%
\end{pgfscope}%
\begin{pgfscope}%
\definecolor{textcolor}{rgb}{0.150000,0.150000,0.150000}%
\pgfsetstrokecolor{textcolor}%
\pgfsetfillcolor{textcolor}%
\pgftext[x=1.444229in,y=0.319444in,,top]{\color{textcolor}\sffamily\fontsize{9.000000}{10.800000}\selectfont 20k}%
\end{pgfscope}%
\begin{pgfscope}%
\pgfpathrectangle{\pgfqpoint{0.532932in}{0.451389in}}{\pgfqpoint{1.503640in}{1.753471in}}%
\pgfusepath{clip}%
\pgfsetroundcap%
\pgfsetroundjoin%
\pgfsetlinewidth{1.003750pt}%
\definecolor{currentstroke}{rgb}{0.800000,0.800000,0.800000}%
\pgfsetstrokecolor{currentstroke}%
\pgfsetdash{}{0pt}%
\pgfpathmoveto{\pgfqpoint{1.899877in}{0.451389in}}%
\pgfpathlineto{\pgfqpoint{1.899877in}{2.204860in}}%
\pgfusepath{stroke}%
\end{pgfscope}%
\begin{pgfscope}%
\definecolor{textcolor}{rgb}{0.150000,0.150000,0.150000}%
\pgfsetstrokecolor{textcolor}%
\pgfsetfillcolor{textcolor}%
\pgftext[x=1.899877in,y=0.319444in,,top]{\color{textcolor}\sffamily\fontsize{9.000000}{10.800000}\selectfont 30k}%
\end{pgfscope}%
\begin{pgfscope}%
\definecolor{textcolor}{rgb}{0.150000,0.150000,0.150000}%
\pgfsetstrokecolor{textcolor}%
\pgfsetfillcolor{textcolor}%
\pgftext[x=1.284752in,y=0.125000in,,top]{\color{textcolor}\sffamily\fontsize{9.000000}{10.800000}\selectfont Input obstacle vertices}%
\end{pgfscope}%
\begin{pgfscope}%
\pgfpathrectangle{\pgfqpoint{0.532932in}{0.451389in}}{\pgfqpoint{1.503640in}{1.753471in}}%
\pgfusepath{clip}%
\pgfsetroundcap%
\pgfsetroundjoin%
\pgfsetlinewidth{1.003750pt}%
\definecolor{currentstroke}{rgb}{0.800000,0.800000,0.800000}%
\pgfsetstrokecolor{currentstroke}%
\pgfsetdash{}{0pt}%
\pgfpathmoveto{\pgfqpoint{0.532932in}{0.451389in}}%
\pgfpathlineto{\pgfqpoint{2.036571in}{0.451389in}}%
\pgfusepath{stroke}%
\end{pgfscope}%
\begin{pgfscope}%
\definecolor{textcolor}{rgb}{0.150000,0.150000,0.150000}%
\pgfsetstrokecolor{textcolor}%
\pgfsetfillcolor{textcolor}%
\pgftext[x=0.332140in, y=0.403903in, left, base]{\color{textcolor}\sffamily\fontsize{9.000000}{10.800000}\selectfont 0}%
\end{pgfscope}%
\begin{pgfscope}%
\pgfpathrectangle{\pgfqpoint{0.532932in}{0.451389in}}{\pgfqpoint{1.503640in}{1.753471in}}%
\pgfusepath{clip}%
\pgfsetroundcap%
\pgfsetroundjoin%
\pgfsetlinewidth{1.003750pt}%
\definecolor{currentstroke}{rgb}{0.800000,0.800000,0.800000}%
\pgfsetstrokecolor{currentstroke}%
\pgfsetdash{}{0pt}%
\pgfpathmoveto{\pgfqpoint{0.532932in}{0.824468in}}%
\pgfpathlineto{\pgfqpoint{2.036571in}{0.824468in}}%
\pgfusepath{stroke}%
\end{pgfscope}%
\begin{pgfscope}%
\definecolor{textcolor}{rgb}{0.150000,0.150000,0.150000}%
\pgfsetstrokecolor{textcolor}%
\pgfsetfillcolor{textcolor}%
\pgftext[x=0.263292in, y=0.776982in, left, base]{\color{textcolor}\sffamily\fontsize{9.000000}{10.800000}\selectfont 50}%
\end{pgfscope}%
\begin{pgfscope}%
\pgfpathrectangle{\pgfqpoint{0.532932in}{0.451389in}}{\pgfqpoint{1.503640in}{1.753471in}}%
\pgfusepath{clip}%
\pgfsetroundcap%
\pgfsetroundjoin%
\pgfsetlinewidth{1.003750pt}%
\definecolor{currentstroke}{rgb}{0.800000,0.800000,0.800000}%
\pgfsetstrokecolor{currentstroke}%
\pgfsetdash{}{0pt}%
\pgfpathmoveto{\pgfqpoint{0.532932in}{1.197547in}}%
\pgfpathlineto{\pgfqpoint{2.036571in}{1.197547in}}%
\pgfusepath{stroke}%
\end{pgfscope}%
\begin{pgfscope}%
\definecolor{textcolor}{rgb}{0.150000,0.150000,0.150000}%
\pgfsetstrokecolor{textcolor}%
\pgfsetfillcolor{textcolor}%
\pgftext[x=0.194444in, y=1.150061in, left, base]{\color{textcolor}\sffamily\fontsize{9.000000}{10.800000}\selectfont 100}%
\end{pgfscope}%
\begin{pgfscope}%
\pgfpathrectangle{\pgfqpoint{0.532932in}{0.451389in}}{\pgfqpoint{1.503640in}{1.753471in}}%
\pgfusepath{clip}%
\pgfsetroundcap%
\pgfsetroundjoin%
\pgfsetlinewidth{1.003750pt}%
\definecolor{currentstroke}{rgb}{0.800000,0.800000,0.800000}%
\pgfsetstrokecolor{currentstroke}%
\pgfsetdash{}{0pt}%
\pgfpathmoveto{\pgfqpoint{0.532932in}{1.570626in}}%
\pgfpathlineto{\pgfqpoint{2.036571in}{1.570626in}}%
\pgfusepath{stroke}%
\end{pgfscope}%
\begin{pgfscope}%
\definecolor{textcolor}{rgb}{0.150000,0.150000,0.150000}%
\pgfsetstrokecolor{textcolor}%
\pgfsetfillcolor{textcolor}%
\pgftext[x=0.194444in, y=1.523140in, left, base]{\color{textcolor}\sffamily\fontsize{9.000000}{10.800000}\selectfont 150}%
\end{pgfscope}%
\begin{pgfscope}%
\pgfpathrectangle{\pgfqpoint{0.532932in}{0.451389in}}{\pgfqpoint{1.503640in}{1.753471in}}%
\pgfusepath{clip}%
\pgfsetroundcap%
\pgfsetroundjoin%
\pgfsetlinewidth{1.003750pt}%
\definecolor{currentstroke}{rgb}{0.800000,0.800000,0.800000}%
\pgfsetstrokecolor{currentstroke}%
\pgfsetdash{}{0pt}%
\pgfpathmoveto{\pgfqpoint{0.532932in}{1.943705in}}%
\pgfpathlineto{\pgfqpoint{2.036571in}{1.943705in}}%
\pgfusepath{stroke}%
\end{pgfscope}%
\begin{pgfscope}%
\definecolor{textcolor}{rgb}{0.150000,0.150000,0.150000}%
\pgfsetstrokecolor{textcolor}%
\pgfsetfillcolor{textcolor}%
\pgftext[x=0.194444in, y=1.896219in, left, base]{\color{textcolor}\sffamily\fontsize{9.000000}{10.800000}\selectfont 200}%
\end{pgfscope}%
\begin{pgfscope}%
\definecolor{textcolor}{rgb}{0.150000,0.150000,0.150000}%
\pgfsetstrokecolor{textcolor}%
\pgfsetfillcolor{textcolor}%
\pgftext[x=0.125000in,y=1.328124in,,bottom,rotate=90.000000]{\color{textcolor}\sffamily\fontsize{9.000000}{10.800000}\selectfont Time in s}%
\end{pgfscope}%
\begin{pgfscope}%
\pgfpathrectangle{\pgfqpoint{0.532932in}{0.451389in}}{\pgfqpoint{1.503640in}{1.753471in}}%
\pgfusepath{clip}%
\pgfsetbuttcap%
\pgfsetroundjoin%
\definecolor{currentfill}{rgb}{0.003922,0.450980,0.698039}%
\pgfsetfillcolor{currentfill}%
\pgfsetfillopacity{0.200000}%
\pgfsetlinewidth{1.003750pt}%
\definecolor{currentstroke}{rgb}{0.003922,0.450980,0.698039}%
\pgfsetstrokecolor{currentstroke}%
\pgfsetstrokeopacity{0.200000}%
\pgfsetdash{}{0pt}%
\pgfsys@defobject{currentmarker}{\pgfqpoint{0.536349in}{0.451456in}}{\pgfqpoint{1.899877in}{1.648125in}}{%
\pgfpathmoveto{\pgfqpoint{0.536349in}{0.451476in}}%
\pgfpathlineto{\pgfqpoint{0.536349in}{0.451456in}}%
\pgfpathlineto{\pgfqpoint{0.546601in}{0.451661in}}%
\pgfpathlineto{\pgfqpoint{0.563688in}{0.452258in}}%
\pgfpathlineto{\pgfqpoint{0.587610in}{0.453791in}}%
\pgfpathlineto{\pgfqpoint{0.618366in}{0.457195in}}%
\pgfpathlineto{\pgfqpoint{0.655957in}{0.461782in}}%
\pgfpathlineto{\pgfqpoint{0.700383in}{0.470361in}}%
\pgfpathlineto{\pgfqpoint{0.751643in}{0.481466in}}%
\pgfpathlineto{\pgfqpoint{0.809738in}{0.497854in}}%
\pgfpathlineto{\pgfqpoint{0.874668in}{0.530202in}}%
\pgfpathlineto{\pgfqpoint{1.025032in}{0.593663in}}%
\pgfpathlineto{\pgfqpoint{1.202735in}{0.723497in}}%
\pgfpathlineto{\pgfqpoint{1.407777in}{0.885636in}}%
\pgfpathlineto{\pgfqpoint{1.640157in}{1.203238in}}%
\pgfpathlineto{\pgfqpoint{1.899877in}{1.634338in}}%
\pgfpathlineto{\pgfqpoint{1.899877in}{1.648125in}}%
\pgfpathlineto{\pgfqpoint{1.899877in}{1.648125in}}%
\pgfpathlineto{\pgfqpoint{1.640157in}{1.215321in}}%
\pgfpathlineto{\pgfqpoint{1.407777in}{0.896266in}}%
\pgfpathlineto{\pgfqpoint{1.202735in}{0.725732in}}%
\pgfpathlineto{\pgfqpoint{1.025032in}{0.596078in}}%
\pgfpathlineto{\pgfqpoint{0.874668in}{0.531213in}}%
\pgfpathlineto{\pgfqpoint{0.809738in}{0.498050in}}%
\pgfpathlineto{\pgfqpoint{0.751643in}{0.481634in}}%
\pgfpathlineto{\pgfqpoint{0.700383in}{0.470412in}}%
\pgfpathlineto{\pgfqpoint{0.655957in}{0.461820in}}%
\pgfpathlineto{\pgfqpoint{0.618366in}{0.457225in}}%
\pgfpathlineto{\pgfqpoint{0.587610in}{0.453813in}}%
\pgfpathlineto{\pgfqpoint{0.563688in}{0.452781in}}%
\pgfpathlineto{\pgfqpoint{0.546601in}{0.451716in}}%
\pgfpathlineto{\pgfqpoint{0.536349in}{0.451476in}}%
\pgfpathlineto{\pgfqpoint{0.536349in}{0.451476in}}%
\pgfpathclose%
\pgfusepath{stroke,fill}%
}%
\begin{pgfscope}%
\pgfsys@transformshift{0.000000in}{0.000000in}%
\pgfsys@useobject{currentmarker}{}%
\end{pgfscope}%
\end{pgfscope}%
\begin{pgfscope}%
\pgfsetrectcap%
\pgfsetmiterjoin%
\pgfsetlinewidth{1.254687pt}%
\definecolor{currentstroke}{rgb}{0.800000,0.800000,0.800000}%
\pgfsetstrokecolor{currentstroke}%
\pgfsetdash{}{0pt}%
\pgfpathmoveto{\pgfqpoint{0.532932in}{0.451389in}}%
\pgfpathlineto{\pgfqpoint{0.532932in}{2.204860in}}%
\pgfusepath{stroke}%
\end{pgfscope}%
\begin{pgfscope}%
\pgfsetrectcap%
\pgfsetmiterjoin%
\pgfsetlinewidth{1.254687pt}%
\definecolor{currentstroke}{rgb}{0.800000,0.800000,0.800000}%
\pgfsetstrokecolor{currentstroke}%
\pgfsetdash{}{0pt}%
\pgfpathmoveto{\pgfqpoint{2.036571in}{0.451389in}}%
\pgfpathlineto{\pgfqpoint{2.036571in}{2.204860in}}%
\pgfusepath{stroke}%
\end{pgfscope}%
\begin{pgfscope}%
\pgfsetrectcap%
\pgfsetmiterjoin%
\pgfsetlinewidth{1.254687pt}%
\definecolor{currentstroke}{rgb}{0.800000,0.800000,0.800000}%
\pgfsetstrokecolor{currentstroke}%
\pgfsetdash{}{0pt}%
\pgfpathmoveto{\pgfqpoint{0.532932in}{0.451389in}}%
\pgfpathlineto{\pgfqpoint{2.036571in}{0.451389in}}%
\pgfusepath{stroke}%
\end{pgfscope}%
\begin{pgfscope}%
\pgfsetrectcap%
\pgfsetmiterjoin%
\pgfsetlinewidth{1.254687pt}%
\definecolor{currentstroke}{rgb}{0.800000,0.800000,0.800000}%
\pgfsetstrokecolor{currentstroke}%
\pgfsetdash{}{0pt}%
\pgfpathmoveto{\pgfqpoint{0.532932in}{2.204860in}}%
\pgfpathlineto{\pgfqpoint{2.036571in}{2.204860in}}%
\pgfusepath{stroke}%
\end{pgfscope}%
\begin{pgfscope}%
\definecolor{textcolor}{rgb}{0.150000,0.150000,0.150000}%
\pgfsetstrokecolor{textcolor}%
\pgfsetfillcolor{textcolor}%
\pgftext[x=1.284752in,y=2.315971in,,base]{\color{textcolor}\sffamily\fontsize{9.000000}{10.800000}\selectfont Rectangles}%
\end{pgfscope}%
\begin{pgfscope}%
\pgfsetroundcap%
\pgfsetroundjoin%
\pgfsetlinewidth{1.003750pt}%
\definecolor{currentstroke}{rgb}{0.003922,0.450980,0.698039}%
\pgfsetstrokecolor{currentstroke}%
\pgfsetdash{}{0pt}%
\pgfpathmoveto{\pgfqpoint{0.536349in}{0.451462in}}%
\pgfpathlineto{\pgfqpoint{0.546601in}{0.451686in}}%
\pgfpathlineto{\pgfqpoint{0.563688in}{0.452410in}}%
\pgfpathlineto{\pgfqpoint{0.587610in}{0.453799in}}%
\pgfpathlineto{\pgfqpoint{0.618366in}{0.457214in}}%
\pgfpathlineto{\pgfqpoint{0.655957in}{0.461802in}}%
\pgfpathlineto{\pgfqpoint{0.700383in}{0.470382in}}%
\pgfpathlineto{\pgfqpoint{0.751643in}{0.481524in}}%
\pgfpathlineto{\pgfqpoint{0.809738in}{0.497976in}}%
\pgfpathlineto{\pgfqpoint{0.874668in}{0.530767in}}%
\pgfpathlineto{\pgfqpoint{1.025032in}{0.594735in}}%
\pgfpathlineto{\pgfqpoint{1.202735in}{0.724301in}}%
\pgfpathlineto{\pgfqpoint{1.407777in}{0.890702in}}%
\pgfpathlineto{\pgfqpoint{1.640157in}{1.209848in}}%
\pgfpathlineto{\pgfqpoint{1.899877in}{1.640341in}}%
\pgfusepath{stroke}%
\end{pgfscope}%
\begin{pgfscope}%
\pgfsetbuttcap%
\pgfsetroundjoin%
\definecolor{currentfill}{rgb}{0.003922,0.450980,0.698039}%
\pgfsetfillcolor{currentfill}%
\pgfsetlinewidth{0.752812pt}%
\definecolor{currentstroke}{rgb}{1.000000,1.000000,1.000000}%
\pgfsetstrokecolor{currentstroke}%
\pgfsetdash{}{0pt}%
\pgfsys@defobject{currentmarker}{\pgfqpoint{-0.034722in}{-0.034722in}}{\pgfqpoint{0.034722in}{0.034722in}}{%
\pgfpathmoveto{\pgfqpoint{0.000000in}{-0.034722in}}%
\pgfpathcurveto{\pgfqpoint{0.009208in}{-0.034722in}}{\pgfqpoint{0.018041in}{-0.031064in}}{\pgfqpoint{0.024552in}{-0.024552in}}%
\pgfpathcurveto{\pgfqpoint{0.031064in}{-0.018041in}}{\pgfqpoint{0.034722in}{-0.009208in}}{\pgfqpoint{0.034722in}{0.000000in}}%
\pgfpathcurveto{\pgfqpoint{0.034722in}{0.009208in}}{\pgfqpoint{0.031064in}{0.018041in}}{\pgfqpoint{0.024552in}{0.024552in}}%
\pgfpathcurveto{\pgfqpoint{0.018041in}{0.031064in}}{\pgfqpoint{0.009208in}{0.034722in}}{\pgfqpoint{0.000000in}{0.034722in}}%
\pgfpathcurveto{\pgfqpoint{-0.009208in}{0.034722in}}{\pgfqpoint{-0.018041in}{0.031064in}}{\pgfqpoint{-0.024552in}{0.024552in}}%
\pgfpathcurveto{\pgfqpoint{-0.031064in}{0.018041in}}{\pgfqpoint{-0.034722in}{0.009208in}}{\pgfqpoint{-0.034722in}{0.000000in}}%
\pgfpathcurveto{\pgfqpoint{-0.034722in}{-0.009208in}}{\pgfqpoint{-0.031064in}{-0.018041in}}{\pgfqpoint{-0.024552in}{-0.024552in}}%
\pgfpathcurveto{\pgfqpoint{-0.018041in}{-0.031064in}}{\pgfqpoint{-0.009208in}{-0.034722in}}{\pgfqpoint{0.000000in}{-0.034722in}}%
\pgfpathlineto{\pgfqpoint{0.000000in}{-0.034722in}}%
\pgfpathclose%
\pgfusepath{stroke,fill}%
}%
\begin{pgfscope}%
\pgfsys@transformshift{0.536349in}{0.451462in}%
\pgfsys@useobject{currentmarker}{}%
\end{pgfscope}%
\begin{pgfscope}%
\pgfsys@transformshift{0.546601in}{0.451686in}%
\pgfsys@useobject{currentmarker}{}%
\end{pgfscope}%
\begin{pgfscope}%
\pgfsys@transformshift{0.563688in}{0.452410in}%
\pgfsys@useobject{currentmarker}{}%
\end{pgfscope}%
\begin{pgfscope}%
\pgfsys@transformshift{0.587610in}{0.453799in}%
\pgfsys@useobject{currentmarker}{}%
\end{pgfscope}%
\begin{pgfscope}%
\pgfsys@transformshift{0.618366in}{0.457214in}%
\pgfsys@useobject{currentmarker}{}%
\end{pgfscope}%
\begin{pgfscope}%
\pgfsys@transformshift{0.655957in}{0.461802in}%
\pgfsys@useobject{currentmarker}{}%
\end{pgfscope}%
\begin{pgfscope}%
\pgfsys@transformshift{0.700383in}{0.470382in}%
\pgfsys@useobject{currentmarker}{}%
\end{pgfscope}%
\begin{pgfscope}%
\pgfsys@transformshift{0.751643in}{0.481524in}%
\pgfsys@useobject{currentmarker}{}%
\end{pgfscope}%
\begin{pgfscope}%
\pgfsys@transformshift{0.809738in}{0.497976in}%
\pgfsys@useobject{currentmarker}{}%
\end{pgfscope}%
\begin{pgfscope}%
\pgfsys@transformshift{0.874668in}{0.530767in}%
\pgfsys@useobject{currentmarker}{}%
\end{pgfscope}%
\begin{pgfscope}%
\pgfsys@transformshift{1.025032in}{0.594735in}%
\pgfsys@useobject{currentmarker}{}%
\end{pgfscope}%
\begin{pgfscope}%
\pgfsys@transformshift{1.202735in}{0.724301in}%
\pgfsys@useobject{currentmarker}{}%
\end{pgfscope}%
\begin{pgfscope}%
\pgfsys@transformshift{1.407777in}{0.890702in}%
\pgfsys@useobject{currentmarker}{}%
\end{pgfscope}%
\begin{pgfscope}%
\pgfsys@transformshift{1.640157in}{1.209848in}%
\pgfsys@useobject{currentmarker}{}%
\end{pgfscope}%
\begin{pgfscope}%
\pgfsys@transformshift{1.899877in}{1.640341in}%
\pgfsys@useobject{currentmarker}{}%
\end{pgfscope}%
\end{pgfscope}%
\begin{pgfscope}%
\pgfsetbuttcap%
\pgfsetmiterjoin%
\definecolor{currentfill}{rgb}{1.000000,1.000000,1.000000}%
\pgfsetfillcolor{currentfill}%
\pgfsetlinewidth{0.000000pt}%
\definecolor{currentstroke}{rgb}{0.000000,0.000000,0.000000}%
\pgfsetstrokecolor{currentstroke}%
\pgfsetstrokeopacity{0.000000}%
\pgfsetdash{}{0pt}%
\pgfpathmoveto{\pgfqpoint{2.557891in}{0.451389in}}%
\pgfpathlineto{\pgfqpoint{4.061531in}{0.451389in}}%
\pgfpathlineto{\pgfqpoint{4.061531in}{2.204860in}}%
\pgfpathlineto{\pgfqpoint{2.557891in}{2.204860in}}%
\pgfpathlineto{\pgfqpoint{2.557891in}{0.451389in}}%
\pgfpathclose%
\pgfusepath{fill}%
\end{pgfscope}%
\begin{pgfscope}%
\pgfpathrectangle{\pgfqpoint{2.557891in}{0.451389in}}{\pgfqpoint{1.503640in}{1.753471in}}%
\pgfusepath{clip}%
\pgfsetroundcap%
\pgfsetroundjoin%
\pgfsetlinewidth{1.003750pt}%
\definecolor{currentstroke}{rgb}{0.800000,0.800000,0.800000}%
\pgfsetstrokecolor{currentstroke}%
\pgfsetdash{}{0pt}%
\pgfpathmoveto{\pgfqpoint{2.557891in}{0.451389in}}%
\pgfpathlineto{\pgfqpoint{2.557891in}{2.204860in}}%
\pgfusepath{stroke}%
\end{pgfscope}%
\begin{pgfscope}%
\definecolor{textcolor}{rgb}{0.150000,0.150000,0.150000}%
\pgfsetstrokecolor{textcolor}%
\pgfsetfillcolor{textcolor}%
\pgftext[x=2.557891in,y=0.319444in,,top]{\color{textcolor}\sffamily\fontsize{9.000000}{10.800000}\selectfont 0k}%
\end{pgfscope}%
\begin{pgfscope}%
\pgfpathrectangle{\pgfqpoint{2.557891in}{0.451389in}}{\pgfqpoint{1.503640in}{1.753471in}}%
\pgfusepath{clip}%
\pgfsetroundcap%
\pgfsetroundjoin%
\pgfsetlinewidth{1.003750pt}%
\definecolor{currentstroke}{rgb}{0.800000,0.800000,0.800000}%
\pgfsetstrokecolor{currentstroke}%
\pgfsetdash{}{0pt}%
\pgfpathmoveto{\pgfqpoint{3.013539in}{0.451389in}}%
\pgfpathlineto{\pgfqpoint{3.013539in}{2.204860in}}%
\pgfusepath{stroke}%
\end{pgfscope}%
\begin{pgfscope}%
\definecolor{textcolor}{rgb}{0.150000,0.150000,0.150000}%
\pgfsetstrokecolor{textcolor}%
\pgfsetfillcolor{textcolor}%
\pgftext[x=3.013539in,y=0.319444in,,top]{\color{textcolor}\sffamily\fontsize{9.000000}{10.800000}\selectfont 10k}%
\end{pgfscope}%
\begin{pgfscope}%
\pgfpathrectangle{\pgfqpoint{2.557891in}{0.451389in}}{\pgfqpoint{1.503640in}{1.753471in}}%
\pgfusepath{clip}%
\pgfsetroundcap%
\pgfsetroundjoin%
\pgfsetlinewidth{1.003750pt}%
\definecolor{currentstroke}{rgb}{0.800000,0.800000,0.800000}%
\pgfsetstrokecolor{currentstroke}%
\pgfsetdash{}{0pt}%
\pgfpathmoveto{\pgfqpoint{3.469188in}{0.451389in}}%
\pgfpathlineto{\pgfqpoint{3.469188in}{2.204860in}}%
\pgfusepath{stroke}%
\end{pgfscope}%
\begin{pgfscope}%
\definecolor{textcolor}{rgb}{0.150000,0.150000,0.150000}%
\pgfsetstrokecolor{textcolor}%
\pgfsetfillcolor{textcolor}%
\pgftext[x=3.469188in,y=0.319444in,,top]{\color{textcolor}\sffamily\fontsize{9.000000}{10.800000}\selectfont 20k}%
\end{pgfscope}%
\begin{pgfscope}%
\pgfpathrectangle{\pgfqpoint{2.557891in}{0.451389in}}{\pgfqpoint{1.503640in}{1.753471in}}%
\pgfusepath{clip}%
\pgfsetroundcap%
\pgfsetroundjoin%
\pgfsetlinewidth{1.003750pt}%
\definecolor{currentstroke}{rgb}{0.800000,0.800000,0.800000}%
\pgfsetstrokecolor{currentstroke}%
\pgfsetdash{}{0pt}%
\pgfpathmoveto{\pgfqpoint{3.924836in}{0.451389in}}%
\pgfpathlineto{\pgfqpoint{3.924836in}{2.204860in}}%
\pgfusepath{stroke}%
\end{pgfscope}%
\begin{pgfscope}%
\definecolor{textcolor}{rgb}{0.150000,0.150000,0.150000}%
\pgfsetstrokecolor{textcolor}%
\pgfsetfillcolor{textcolor}%
\pgftext[x=3.924836in,y=0.319444in,,top]{\color{textcolor}\sffamily\fontsize{9.000000}{10.800000}\selectfont 30k}%
\end{pgfscope}%
\begin{pgfscope}%
\definecolor{textcolor}{rgb}{0.150000,0.150000,0.150000}%
\pgfsetstrokecolor{textcolor}%
\pgfsetfillcolor{textcolor}%
\pgftext[x=3.309711in,y=0.125000in,,top]{\color{textcolor}\sffamily\fontsize{9.000000}{10.800000}\selectfont Input obstacle vertices}%
\end{pgfscope}%
\begin{pgfscope}%
\pgfpathrectangle{\pgfqpoint{2.557891in}{0.451389in}}{\pgfqpoint{1.503640in}{1.753471in}}%
\pgfusepath{clip}%
\pgfsetroundcap%
\pgfsetroundjoin%
\pgfsetlinewidth{1.003750pt}%
\definecolor{currentstroke}{rgb}{0.800000,0.800000,0.800000}%
\pgfsetstrokecolor{currentstroke}%
\pgfsetdash{}{0pt}%
\pgfpathmoveto{\pgfqpoint{2.557891in}{0.451389in}}%
\pgfpathlineto{\pgfqpoint{4.061531in}{0.451389in}}%
\pgfusepath{stroke}%
\end{pgfscope}%
\begin{pgfscope}%
\definecolor{textcolor}{rgb}{0.150000,0.150000,0.150000}%
\pgfsetstrokecolor{textcolor}%
\pgfsetfillcolor{textcolor}%
\pgftext[x=2.357099in, y=0.403903in, left, base]{\color{textcolor}\sffamily\fontsize{9.000000}{10.800000}\selectfont 0}%
\end{pgfscope}%
\begin{pgfscope}%
\pgfpathrectangle{\pgfqpoint{2.557891in}{0.451389in}}{\pgfqpoint{1.503640in}{1.753471in}}%
\pgfusepath{clip}%
\pgfsetroundcap%
\pgfsetroundjoin%
\pgfsetlinewidth{1.003750pt}%
\definecolor{currentstroke}{rgb}{0.800000,0.800000,0.800000}%
\pgfsetstrokecolor{currentstroke}%
\pgfsetdash{}{0pt}%
\pgfpathmoveto{\pgfqpoint{2.557891in}{0.824468in}}%
\pgfpathlineto{\pgfqpoint{4.061531in}{0.824468in}}%
\pgfusepath{stroke}%
\end{pgfscope}%
\begin{pgfscope}%
\definecolor{textcolor}{rgb}{0.150000,0.150000,0.150000}%
\pgfsetstrokecolor{textcolor}%
\pgfsetfillcolor{textcolor}%
\pgftext[x=2.288251in, y=0.776982in, left, base]{\color{textcolor}\sffamily\fontsize{9.000000}{10.800000}\selectfont 50}%
\end{pgfscope}%
\begin{pgfscope}%
\pgfpathrectangle{\pgfqpoint{2.557891in}{0.451389in}}{\pgfqpoint{1.503640in}{1.753471in}}%
\pgfusepath{clip}%
\pgfsetroundcap%
\pgfsetroundjoin%
\pgfsetlinewidth{1.003750pt}%
\definecolor{currentstroke}{rgb}{0.800000,0.800000,0.800000}%
\pgfsetstrokecolor{currentstroke}%
\pgfsetdash{}{0pt}%
\pgfpathmoveto{\pgfqpoint{2.557891in}{1.197547in}}%
\pgfpathlineto{\pgfqpoint{4.061531in}{1.197547in}}%
\pgfusepath{stroke}%
\end{pgfscope}%
\begin{pgfscope}%
\definecolor{textcolor}{rgb}{0.150000,0.150000,0.150000}%
\pgfsetstrokecolor{textcolor}%
\pgfsetfillcolor{textcolor}%
\pgftext[x=2.219403in, y=1.150061in, left, base]{\color{textcolor}\sffamily\fontsize{9.000000}{10.800000}\selectfont 100}%
\end{pgfscope}%
\begin{pgfscope}%
\pgfpathrectangle{\pgfqpoint{2.557891in}{0.451389in}}{\pgfqpoint{1.503640in}{1.753471in}}%
\pgfusepath{clip}%
\pgfsetroundcap%
\pgfsetroundjoin%
\pgfsetlinewidth{1.003750pt}%
\definecolor{currentstroke}{rgb}{0.800000,0.800000,0.800000}%
\pgfsetstrokecolor{currentstroke}%
\pgfsetdash{}{0pt}%
\pgfpathmoveto{\pgfqpoint{2.557891in}{1.570626in}}%
\pgfpathlineto{\pgfqpoint{4.061531in}{1.570626in}}%
\pgfusepath{stroke}%
\end{pgfscope}%
\begin{pgfscope}%
\definecolor{textcolor}{rgb}{0.150000,0.150000,0.150000}%
\pgfsetstrokecolor{textcolor}%
\pgfsetfillcolor{textcolor}%
\pgftext[x=2.219403in, y=1.523140in, left, base]{\color{textcolor}\sffamily\fontsize{9.000000}{10.800000}\selectfont 150}%
\end{pgfscope}%
\begin{pgfscope}%
\pgfpathrectangle{\pgfqpoint{2.557891in}{0.451389in}}{\pgfqpoint{1.503640in}{1.753471in}}%
\pgfusepath{clip}%
\pgfsetroundcap%
\pgfsetroundjoin%
\pgfsetlinewidth{1.003750pt}%
\definecolor{currentstroke}{rgb}{0.800000,0.800000,0.800000}%
\pgfsetstrokecolor{currentstroke}%
\pgfsetdash{}{0pt}%
\pgfpathmoveto{\pgfqpoint{2.557891in}{1.943705in}}%
\pgfpathlineto{\pgfqpoint{4.061531in}{1.943705in}}%
\pgfusepath{stroke}%
\end{pgfscope}%
\begin{pgfscope}%
\definecolor{textcolor}{rgb}{0.150000,0.150000,0.150000}%
\pgfsetstrokecolor{textcolor}%
\pgfsetfillcolor{textcolor}%
\pgftext[x=2.219403in, y=1.896219in, left, base]{\color{textcolor}\sffamily\fontsize{9.000000}{10.800000}\selectfont 200}%
\end{pgfscope}%
\begin{pgfscope}%
\pgfpathrectangle{\pgfqpoint{2.557891in}{0.451389in}}{\pgfqpoint{1.503640in}{1.753471in}}%
\pgfusepath{clip}%
\pgfsetbuttcap%
\pgfsetroundjoin%
\definecolor{currentfill}{rgb}{0.003922,0.450980,0.698039}%
\pgfsetfillcolor{currentfill}%
\pgfsetfillopacity{0.200000}%
\pgfsetlinewidth{1.003750pt}%
\definecolor{currentstroke}{rgb}{0.003922,0.450980,0.698039}%
\pgfsetstrokecolor{currentstroke}%
\pgfsetstrokeopacity{0.200000}%
\pgfsetdash{}{0pt}%
\pgfsys@defobject{currentmarker}{\pgfqpoint{2.560351in}{0.451431in}}{\pgfqpoint{3.975140in}{1.990221in}}{%
\pgfpathmoveto{\pgfqpoint{2.560351in}{0.451454in}}%
\pgfpathlineto{\pgfqpoint{2.560351in}{0.451431in}}%
\pgfpathlineto{\pgfqpoint{2.567733in}{0.451694in}}%
\pgfpathlineto{\pgfqpoint{2.597259in}{0.452761in}}%
\pgfpathlineto{\pgfqpoint{2.646469in}{0.457901in}}%
\pgfpathlineto{\pgfqpoint{2.715363in}{0.469204in}}%
\pgfpathlineto{\pgfqpoint{2.803941in}{0.492226in}}%
\pgfpathlineto{\pgfqpoint{2.912203in}{0.539191in}}%
\pgfpathlineto{\pgfqpoint{3.040149in}{0.608914in}}%
\pgfpathlineto{\pgfqpoint{3.187779in}{0.716234in}}%
\pgfpathlineto{\pgfqpoint{3.355093in}{0.901256in}}%
\pgfpathlineto{\pgfqpoint{3.542091in}{1.138402in}}%
\pgfpathlineto{\pgfqpoint{3.748774in}{1.496831in}}%
\pgfpathlineto{\pgfqpoint{3.975140in}{1.979625in}}%
\pgfpathlineto{\pgfqpoint{3.975140in}{1.990221in}}%
\pgfpathlineto{\pgfqpoint{3.975140in}{1.990221in}}%
\pgfpathlineto{\pgfqpoint{3.748774in}{1.514818in}}%
\pgfpathlineto{\pgfqpoint{3.542091in}{1.163350in}}%
\pgfpathlineto{\pgfqpoint{3.355093in}{0.903846in}}%
\pgfpathlineto{\pgfqpoint{3.187779in}{0.718113in}}%
\pgfpathlineto{\pgfqpoint{3.040149in}{0.610609in}}%
\pgfpathlineto{\pgfqpoint{2.912203in}{0.539755in}}%
\pgfpathlineto{\pgfqpoint{2.803941in}{0.492476in}}%
\pgfpathlineto{\pgfqpoint{2.715363in}{0.469305in}}%
\pgfpathlineto{\pgfqpoint{2.646469in}{0.457928in}}%
\pgfpathlineto{\pgfqpoint{2.597259in}{0.452781in}}%
\pgfpathlineto{\pgfqpoint{2.567733in}{0.451731in}}%
\pgfpathlineto{\pgfqpoint{2.560351in}{0.451454in}}%
\pgfpathlineto{\pgfqpoint{2.560351in}{0.451454in}}%
\pgfpathclose%
\pgfusepath{stroke,fill}%
}%
\begin{pgfscope}%
\pgfsys@transformshift{0.000000in}{0.000000in}%
\pgfsys@useobject{currentmarker}{}%
\end{pgfscope}%
\end{pgfscope}%
\begin{pgfscope}%
\pgfsetrectcap%
\pgfsetmiterjoin%
\pgfsetlinewidth{1.254687pt}%
\definecolor{currentstroke}{rgb}{0.800000,0.800000,0.800000}%
\pgfsetstrokecolor{currentstroke}%
\pgfsetdash{}{0pt}%
\pgfpathmoveto{\pgfqpoint{2.557891in}{0.451389in}}%
\pgfpathlineto{\pgfqpoint{2.557891in}{2.204860in}}%
\pgfusepath{stroke}%
\end{pgfscope}%
\begin{pgfscope}%
\pgfsetrectcap%
\pgfsetmiterjoin%
\pgfsetlinewidth{1.254687pt}%
\definecolor{currentstroke}{rgb}{0.800000,0.800000,0.800000}%
\pgfsetstrokecolor{currentstroke}%
\pgfsetdash{}{0pt}%
\pgfpathmoveto{\pgfqpoint{4.061531in}{0.451389in}}%
\pgfpathlineto{\pgfqpoint{4.061531in}{2.204860in}}%
\pgfusepath{stroke}%
\end{pgfscope}%
\begin{pgfscope}%
\pgfsetrectcap%
\pgfsetmiterjoin%
\pgfsetlinewidth{1.254687pt}%
\definecolor{currentstroke}{rgb}{0.800000,0.800000,0.800000}%
\pgfsetstrokecolor{currentstroke}%
\pgfsetdash{}{0pt}%
\pgfpathmoveto{\pgfqpoint{2.557891in}{0.451389in}}%
\pgfpathlineto{\pgfqpoint{4.061531in}{0.451389in}}%
\pgfusepath{stroke}%
\end{pgfscope}%
\begin{pgfscope}%
\pgfsetrectcap%
\pgfsetmiterjoin%
\pgfsetlinewidth{1.254687pt}%
\definecolor{currentstroke}{rgb}{0.800000,0.800000,0.800000}%
\pgfsetstrokecolor{currentstroke}%
\pgfsetdash{}{0pt}%
\pgfpathmoveto{\pgfqpoint{2.557891in}{2.204860in}}%
\pgfpathlineto{\pgfqpoint{4.061531in}{2.204860in}}%
\pgfusepath{stroke}%
\end{pgfscope}%
\begin{pgfscope}%
\definecolor{textcolor}{rgb}{0.150000,0.150000,0.150000}%
\pgfsetstrokecolor{textcolor}%
\pgfsetfillcolor{textcolor}%
\pgftext[x=3.309711in,y=2.315971in,,base]{\color{textcolor}\sffamily\fontsize{9.000000}{10.800000}\selectfont Maze}%
\end{pgfscope}%
\begin{pgfscope}%
\pgfsetroundcap%
\pgfsetroundjoin%
\pgfsetlinewidth{1.003750pt}%
\definecolor{currentstroke}{rgb}{0.003922,0.450980,0.698039}%
\pgfsetstrokecolor{currentstroke}%
\pgfsetdash{}{0pt}%
\pgfpathmoveto{\pgfqpoint{2.560351in}{0.451437in}}%
\pgfpathlineto{\pgfqpoint{2.567733in}{0.451709in}}%
\pgfpathlineto{\pgfqpoint{2.597259in}{0.452768in}}%
\pgfpathlineto{\pgfqpoint{2.646469in}{0.457915in}}%
\pgfpathlineto{\pgfqpoint{2.715363in}{0.469249in}}%
\pgfpathlineto{\pgfqpoint{2.803941in}{0.492348in}}%
\pgfpathlineto{\pgfqpoint{2.912203in}{0.539399in}}%
\pgfpathlineto{\pgfqpoint{3.040149in}{0.609661in}}%
\pgfpathlineto{\pgfqpoint{3.187779in}{0.717273in}}%
\pgfpathlineto{\pgfqpoint{3.355093in}{0.902677in}}%
\pgfpathlineto{\pgfqpoint{3.542091in}{1.148594in}}%
\pgfpathlineto{\pgfqpoint{3.748774in}{1.504044in}}%
\pgfpathlineto{\pgfqpoint{3.975140in}{1.984821in}}%
\pgfusepath{stroke}%
\end{pgfscope}%
\begin{pgfscope}%
\pgfsetbuttcap%
\pgfsetroundjoin%
\definecolor{currentfill}{rgb}{0.003922,0.450980,0.698039}%
\pgfsetfillcolor{currentfill}%
\pgfsetlinewidth{0.752812pt}%
\definecolor{currentstroke}{rgb}{1.000000,1.000000,1.000000}%
\pgfsetstrokecolor{currentstroke}%
\pgfsetdash{}{0pt}%
\pgfsys@defobject{currentmarker}{\pgfqpoint{-0.034722in}{-0.034722in}}{\pgfqpoint{0.034722in}{0.034722in}}{%
\pgfpathmoveto{\pgfqpoint{0.000000in}{-0.034722in}}%
\pgfpathcurveto{\pgfqpoint{0.009208in}{-0.034722in}}{\pgfqpoint{0.018041in}{-0.031064in}}{\pgfqpoint{0.024552in}{-0.024552in}}%
\pgfpathcurveto{\pgfqpoint{0.031064in}{-0.018041in}}{\pgfqpoint{0.034722in}{-0.009208in}}{\pgfqpoint{0.034722in}{0.000000in}}%
\pgfpathcurveto{\pgfqpoint{0.034722in}{0.009208in}}{\pgfqpoint{0.031064in}{0.018041in}}{\pgfqpoint{0.024552in}{0.024552in}}%
\pgfpathcurveto{\pgfqpoint{0.018041in}{0.031064in}}{\pgfqpoint{0.009208in}{0.034722in}}{\pgfqpoint{0.000000in}{0.034722in}}%
\pgfpathcurveto{\pgfqpoint{-0.009208in}{0.034722in}}{\pgfqpoint{-0.018041in}{0.031064in}}{\pgfqpoint{-0.024552in}{0.024552in}}%
\pgfpathcurveto{\pgfqpoint{-0.031064in}{0.018041in}}{\pgfqpoint{-0.034722in}{0.009208in}}{\pgfqpoint{-0.034722in}{0.000000in}}%
\pgfpathcurveto{\pgfqpoint{-0.034722in}{-0.009208in}}{\pgfqpoint{-0.031064in}{-0.018041in}}{\pgfqpoint{-0.024552in}{-0.024552in}}%
\pgfpathcurveto{\pgfqpoint{-0.018041in}{-0.031064in}}{\pgfqpoint{-0.009208in}{-0.034722in}}{\pgfqpoint{0.000000in}{-0.034722in}}%
\pgfpathlineto{\pgfqpoint{0.000000in}{-0.034722in}}%
\pgfpathclose%
\pgfusepath{stroke,fill}%
}%
\begin{pgfscope}%
\pgfsys@transformshift{2.560351in}{0.451437in}%
\pgfsys@useobject{currentmarker}{}%
\end{pgfscope}%
\begin{pgfscope}%
\pgfsys@transformshift{2.567733in}{0.451709in}%
\pgfsys@useobject{currentmarker}{}%
\end{pgfscope}%
\begin{pgfscope}%
\pgfsys@transformshift{2.597259in}{0.452768in}%
\pgfsys@useobject{currentmarker}{}%
\end{pgfscope}%
\begin{pgfscope}%
\pgfsys@transformshift{2.646469in}{0.457915in}%
\pgfsys@useobject{currentmarker}{}%
\end{pgfscope}%
\begin{pgfscope}%
\pgfsys@transformshift{2.715363in}{0.469249in}%
\pgfsys@useobject{currentmarker}{}%
\end{pgfscope}%
\begin{pgfscope}%
\pgfsys@transformshift{2.803941in}{0.492348in}%
\pgfsys@useobject{currentmarker}{}%
\end{pgfscope}%
\begin{pgfscope}%
\pgfsys@transformshift{2.912203in}{0.539399in}%
\pgfsys@useobject{currentmarker}{}%
\end{pgfscope}%
\begin{pgfscope}%
\pgfsys@transformshift{3.040149in}{0.609661in}%
\pgfsys@useobject{currentmarker}{}%
\end{pgfscope}%
\begin{pgfscope}%
\pgfsys@transformshift{3.187779in}{0.717273in}%
\pgfsys@useobject{currentmarker}{}%
\end{pgfscope}%
\begin{pgfscope}%
\pgfsys@transformshift{3.355093in}{0.902677in}%
\pgfsys@useobject{currentmarker}{}%
\end{pgfscope}%
\begin{pgfscope}%
\pgfsys@transformshift{3.542091in}{1.148594in}%
\pgfsys@useobject{currentmarker}{}%
\end{pgfscope}%
\begin{pgfscope}%
\pgfsys@transformshift{3.748774in}{1.504044in}%
\pgfsys@useobject{currentmarker}{}%
\end{pgfscope}%
\begin{pgfscope}%
\pgfsys@transformshift{3.975140in}{1.984821in}%
\pgfsys@useobject{currentmarker}{}%
\end{pgfscope}%
\end{pgfscope}%
\begin{pgfscope}%
\pgfsetbuttcap%
\pgfsetmiterjoin%
\definecolor{currentfill}{rgb}{1.000000,1.000000,1.000000}%
\pgfsetfillcolor{currentfill}%
\pgfsetlinewidth{0.000000pt}%
\definecolor{currentstroke}{rgb}{0.000000,0.000000,0.000000}%
\pgfsetstrokecolor{currentstroke}%
\pgfsetstrokeopacity{0.000000}%
\pgfsetdash{}{0pt}%
\pgfpathmoveto{\pgfqpoint{4.582850in}{0.451389in}}%
\pgfpathlineto{\pgfqpoint{6.086490in}{0.451389in}}%
\pgfpathlineto{\pgfqpoint{6.086490in}{2.204860in}}%
\pgfpathlineto{\pgfqpoint{4.582850in}{2.204860in}}%
\pgfpathlineto{\pgfqpoint{4.582850in}{0.451389in}}%
\pgfpathclose%
\pgfusepath{fill}%
\end{pgfscope}%
\begin{pgfscope}%
\pgfpathrectangle{\pgfqpoint{4.582850in}{0.451389in}}{\pgfqpoint{1.503640in}{1.753471in}}%
\pgfusepath{clip}%
\pgfsetroundcap%
\pgfsetroundjoin%
\pgfsetlinewidth{1.003750pt}%
\definecolor{currentstroke}{rgb}{0.800000,0.800000,0.800000}%
\pgfsetstrokecolor{currentstroke}%
\pgfsetdash{}{0pt}%
\pgfpathmoveto{\pgfqpoint{4.582850in}{0.451389in}}%
\pgfpathlineto{\pgfqpoint{4.582850in}{2.204860in}}%
\pgfusepath{stroke}%
\end{pgfscope}%
\begin{pgfscope}%
\definecolor{textcolor}{rgb}{0.150000,0.150000,0.150000}%
\pgfsetstrokecolor{textcolor}%
\pgfsetfillcolor{textcolor}%
\pgftext[x=4.582850in,y=0.319444in,,top]{\color{textcolor}\sffamily\fontsize{9.000000}{10.800000}\selectfont 0k}%
\end{pgfscope}%
\begin{pgfscope}%
\pgfpathrectangle{\pgfqpoint{4.582850in}{0.451389in}}{\pgfqpoint{1.503640in}{1.753471in}}%
\pgfusepath{clip}%
\pgfsetroundcap%
\pgfsetroundjoin%
\pgfsetlinewidth{1.003750pt}%
\definecolor{currentstroke}{rgb}{0.800000,0.800000,0.800000}%
\pgfsetstrokecolor{currentstroke}%
\pgfsetdash{}{0pt}%
\pgfpathmoveto{\pgfqpoint{4.905535in}{0.451389in}}%
\pgfpathlineto{\pgfqpoint{4.905535in}{2.204860in}}%
\pgfusepath{stroke}%
\end{pgfscope}%
\begin{pgfscope}%
\definecolor{textcolor}{rgb}{0.150000,0.150000,0.150000}%
\pgfsetstrokecolor{textcolor}%
\pgfsetfillcolor{textcolor}%
\pgftext[x=4.905535in,y=0.319444in,,top]{\color{textcolor}\sffamily\fontsize{9.000000}{10.800000}\selectfont 10k}%
\end{pgfscope}%
\begin{pgfscope}%
\pgfpathrectangle{\pgfqpoint{4.582850in}{0.451389in}}{\pgfqpoint{1.503640in}{1.753471in}}%
\pgfusepath{clip}%
\pgfsetroundcap%
\pgfsetroundjoin%
\pgfsetlinewidth{1.003750pt}%
\definecolor{currentstroke}{rgb}{0.800000,0.800000,0.800000}%
\pgfsetstrokecolor{currentstroke}%
\pgfsetdash{}{0pt}%
\pgfpathmoveto{\pgfqpoint{5.228219in}{0.451389in}}%
\pgfpathlineto{\pgfqpoint{5.228219in}{2.204860in}}%
\pgfusepath{stroke}%
\end{pgfscope}%
\begin{pgfscope}%
\definecolor{textcolor}{rgb}{0.150000,0.150000,0.150000}%
\pgfsetstrokecolor{textcolor}%
\pgfsetfillcolor{textcolor}%
\pgftext[x=5.228219in,y=0.319444in,,top]{\color{textcolor}\sffamily\fontsize{9.000000}{10.800000}\selectfont 20k}%
\end{pgfscope}%
\begin{pgfscope}%
\pgfpathrectangle{\pgfqpoint{4.582850in}{0.451389in}}{\pgfqpoint{1.503640in}{1.753471in}}%
\pgfusepath{clip}%
\pgfsetroundcap%
\pgfsetroundjoin%
\pgfsetlinewidth{1.003750pt}%
\definecolor{currentstroke}{rgb}{0.800000,0.800000,0.800000}%
\pgfsetstrokecolor{currentstroke}%
\pgfsetdash{}{0pt}%
\pgfpathmoveto{\pgfqpoint{5.550904in}{0.451389in}}%
\pgfpathlineto{\pgfqpoint{5.550904in}{2.204860in}}%
\pgfusepath{stroke}%
\end{pgfscope}%
\begin{pgfscope}%
\definecolor{textcolor}{rgb}{0.150000,0.150000,0.150000}%
\pgfsetstrokecolor{textcolor}%
\pgfsetfillcolor{textcolor}%
\pgftext[x=5.550904in,y=0.319444in,,top]{\color{textcolor}\sffamily\fontsize{9.000000}{10.800000}\selectfont 30k}%
\end{pgfscope}%
\begin{pgfscope}%
\pgfpathrectangle{\pgfqpoint{4.582850in}{0.451389in}}{\pgfqpoint{1.503640in}{1.753471in}}%
\pgfusepath{clip}%
\pgfsetroundcap%
\pgfsetroundjoin%
\pgfsetlinewidth{1.003750pt}%
\definecolor{currentstroke}{rgb}{0.800000,0.800000,0.800000}%
\pgfsetstrokecolor{currentstroke}%
\pgfsetdash{}{0pt}%
\pgfpathmoveto{\pgfqpoint{5.873589in}{0.451389in}}%
\pgfpathlineto{\pgfqpoint{5.873589in}{2.204860in}}%
\pgfusepath{stroke}%
\end{pgfscope}%
\begin{pgfscope}%
\definecolor{textcolor}{rgb}{0.150000,0.150000,0.150000}%
\pgfsetstrokecolor{textcolor}%
\pgfsetfillcolor{textcolor}%
\pgftext[x=5.873589in,y=0.319444in,,top]{\color{textcolor}\sffamily\fontsize{9.000000}{10.800000}\selectfont 40k}%
\end{pgfscope}%
\begin{pgfscope}%
\definecolor{textcolor}{rgb}{0.150000,0.150000,0.150000}%
\pgfsetstrokecolor{textcolor}%
\pgfsetfillcolor{textcolor}%
\pgftext[x=5.334670in,y=0.125000in,,top]{\color{textcolor}\sffamily\fontsize{9.000000}{10.800000}\selectfont Input obstacle vertices}%
\end{pgfscope}%
\begin{pgfscope}%
\pgfpathrectangle{\pgfqpoint{4.582850in}{0.451389in}}{\pgfqpoint{1.503640in}{1.753471in}}%
\pgfusepath{clip}%
\pgfsetroundcap%
\pgfsetroundjoin%
\pgfsetlinewidth{1.003750pt}%
\definecolor{currentstroke}{rgb}{0.800000,0.800000,0.800000}%
\pgfsetstrokecolor{currentstroke}%
\pgfsetdash{}{0pt}%
\pgfpathmoveto{\pgfqpoint{4.582850in}{0.451389in}}%
\pgfpathlineto{\pgfqpoint{6.086490in}{0.451389in}}%
\pgfusepath{stroke}%
\end{pgfscope}%
\begin{pgfscope}%
\definecolor{textcolor}{rgb}{0.150000,0.150000,0.150000}%
\pgfsetstrokecolor{textcolor}%
\pgfsetfillcolor{textcolor}%
\pgftext[x=4.382058in, y=0.403903in, left, base]{\color{textcolor}\sffamily\fontsize{9.000000}{10.800000}\selectfont 0}%
\end{pgfscope}%
\begin{pgfscope}%
\pgfpathrectangle{\pgfqpoint{4.582850in}{0.451389in}}{\pgfqpoint{1.503640in}{1.753471in}}%
\pgfusepath{clip}%
\pgfsetroundcap%
\pgfsetroundjoin%
\pgfsetlinewidth{1.003750pt}%
\definecolor{currentstroke}{rgb}{0.800000,0.800000,0.800000}%
\pgfsetstrokecolor{currentstroke}%
\pgfsetdash{}{0pt}%
\pgfpathmoveto{\pgfqpoint{4.582850in}{0.846054in}}%
\pgfpathlineto{\pgfqpoint{6.086490in}{0.846054in}}%
\pgfusepath{stroke}%
\end{pgfscope}%
\begin{pgfscope}%
\definecolor{textcolor}{rgb}{0.150000,0.150000,0.150000}%
\pgfsetstrokecolor{textcolor}%
\pgfsetfillcolor{textcolor}%
\pgftext[x=4.244362in, y=0.798568in, left, base]{\color{textcolor}\sffamily\fontsize{9.000000}{10.800000}\selectfont 200}%
\end{pgfscope}%
\begin{pgfscope}%
\pgfpathrectangle{\pgfqpoint{4.582850in}{0.451389in}}{\pgfqpoint{1.503640in}{1.753471in}}%
\pgfusepath{clip}%
\pgfsetroundcap%
\pgfsetroundjoin%
\pgfsetlinewidth{1.003750pt}%
\definecolor{currentstroke}{rgb}{0.800000,0.800000,0.800000}%
\pgfsetstrokecolor{currentstroke}%
\pgfsetdash{}{0pt}%
\pgfpathmoveto{\pgfqpoint{4.582850in}{1.240719in}}%
\pgfpathlineto{\pgfqpoint{6.086490in}{1.240719in}}%
\pgfusepath{stroke}%
\end{pgfscope}%
\begin{pgfscope}%
\definecolor{textcolor}{rgb}{0.150000,0.150000,0.150000}%
\pgfsetstrokecolor{textcolor}%
\pgfsetfillcolor{textcolor}%
\pgftext[x=4.244362in, y=1.193233in, left, base]{\color{textcolor}\sffamily\fontsize{9.000000}{10.800000}\selectfont 400}%
\end{pgfscope}%
\begin{pgfscope}%
\pgfpathrectangle{\pgfqpoint{4.582850in}{0.451389in}}{\pgfqpoint{1.503640in}{1.753471in}}%
\pgfusepath{clip}%
\pgfsetroundcap%
\pgfsetroundjoin%
\pgfsetlinewidth{1.003750pt}%
\definecolor{currentstroke}{rgb}{0.800000,0.800000,0.800000}%
\pgfsetstrokecolor{currentstroke}%
\pgfsetdash{}{0pt}%
\pgfpathmoveto{\pgfqpoint{4.582850in}{1.635384in}}%
\pgfpathlineto{\pgfqpoint{6.086490in}{1.635384in}}%
\pgfusepath{stroke}%
\end{pgfscope}%
\begin{pgfscope}%
\definecolor{textcolor}{rgb}{0.150000,0.150000,0.150000}%
\pgfsetstrokecolor{textcolor}%
\pgfsetfillcolor{textcolor}%
\pgftext[x=4.244362in, y=1.587898in, left, base]{\color{textcolor}\sffamily\fontsize{9.000000}{10.800000}\selectfont 600}%
\end{pgfscope}%
\begin{pgfscope}%
\pgfpathrectangle{\pgfqpoint{4.582850in}{0.451389in}}{\pgfqpoint{1.503640in}{1.753471in}}%
\pgfusepath{clip}%
\pgfsetroundcap%
\pgfsetroundjoin%
\pgfsetlinewidth{1.003750pt}%
\definecolor{currentstroke}{rgb}{0.800000,0.800000,0.800000}%
\pgfsetstrokecolor{currentstroke}%
\pgfsetdash{}{0pt}%
\pgfpathmoveto{\pgfqpoint{4.582850in}{2.030049in}}%
\pgfpathlineto{\pgfqpoint{6.086490in}{2.030049in}}%
\pgfusepath{stroke}%
\end{pgfscope}%
\begin{pgfscope}%
\definecolor{textcolor}{rgb}{0.150000,0.150000,0.150000}%
\pgfsetstrokecolor{textcolor}%
\pgfsetfillcolor{textcolor}%
\pgftext[x=4.244362in, y=1.982563in, left, base]{\color{textcolor}\sffamily\fontsize{9.000000}{10.800000}\selectfont 800}%
\end{pgfscope}%
\begin{pgfscope}%
\pgfpathrectangle{\pgfqpoint{4.582850in}{0.451389in}}{\pgfqpoint{1.503640in}{1.753471in}}%
\pgfusepath{clip}%
\pgfsetbuttcap%
\pgfsetroundjoin%
\definecolor{currentfill}{rgb}{0.003922,0.450980,0.698039}%
\pgfsetfillcolor{currentfill}%
\pgfsetfillopacity{0.200000}%
\pgfsetlinewidth{1.003750pt}%
\definecolor{currentstroke}{rgb}{0.003922,0.450980,0.698039}%
\pgfsetstrokecolor{currentstroke}%
\pgfsetstrokeopacity{0.200000}%
\pgfsetdash{}{0pt}%
\pgfsys@defobject{currentmarker}{\pgfqpoint{4.597177in}{0.451572in}}{\pgfqpoint{6.015570in}{2.121370in}}{%
\pgfpathmoveto{\pgfqpoint{4.597177in}{0.451580in}}%
\pgfpathlineto{\pgfqpoint{4.597177in}{0.451572in}}%
\pgfpathlineto{\pgfqpoint{4.640159in}{0.453559in}}%
\pgfpathlineto{\pgfqpoint{4.711795in}{0.460684in}}%
\pgfpathlineto{\pgfqpoint{4.812085in}{0.479438in}}%
\pgfpathlineto{\pgfqpoint{4.941030in}{0.521597in}}%
\pgfpathlineto{\pgfqpoint{5.098629in}{0.591563in}}%
\pgfpathlineto{\pgfqpoint{5.284883in}{0.709910in}}%
\pgfpathlineto{\pgfqpoint{5.499791in}{0.998498in}}%
\pgfpathlineto{\pgfqpoint{5.743353in}{1.397735in}}%
\pgfpathlineto{\pgfqpoint{6.015570in}{2.074418in}}%
\pgfpathlineto{\pgfqpoint{6.015570in}{2.121370in}}%
\pgfpathlineto{\pgfqpoint{6.015570in}{2.121370in}}%
\pgfpathlineto{\pgfqpoint{5.743353in}{1.463055in}}%
\pgfpathlineto{\pgfqpoint{5.499791in}{1.077770in}}%
\pgfpathlineto{\pgfqpoint{5.284883in}{0.716404in}}%
\pgfpathlineto{\pgfqpoint{5.098629in}{0.593351in}}%
\pgfpathlineto{\pgfqpoint{4.941030in}{0.523923in}}%
\pgfpathlineto{\pgfqpoint{4.812085in}{0.479752in}}%
\pgfpathlineto{\pgfqpoint{4.711795in}{0.460738in}}%
\pgfpathlineto{\pgfqpoint{4.640159in}{0.453567in}}%
\pgfpathlineto{\pgfqpoint{4.597177in}{0.451580in}}%
\pgfpathlineto{\pgfqpoint{4.597177in}{0.451580in}}%
\pgfpathclose%
\pgfusepath{stroke,fill}%
}%
\begin{pgfscope}%
\pgfsys@transformshift{0.000000in}{0.000000in}%
\pgfsys@useobject{currentmarker}{}%
\end{pgfscope}%
\end{pgfscope}%
\begin{pgfscope}%
\pgfsetrectcap%
\pgfsetmiterjoin%
\pgfsetlinewidth{1.254687pt}%
\definecolor{currentstroke}{rgb}{0.800000,0.800000,0.800000}%
\pgfsetstrokecolor{currentstroke}%
\pgfsetdash{}{0pt}%
\pgfpathmoveto{\pgfqpoint{4.582850in}{0.451389in}}%
\pgfpathlineto{\pgfqpoint{4.582850in}{2.204860in}}%
\pgfusepath{stroke}%
\end{pgfscope}%
\begin{pgfscope}%
\pgfsetrectcap%
\pgfsetmiterjoin%
\pgfsetlinewidth{1.254687pt}%
\definecolor{currentstroke}{rgb}{0.800000,0.800000,0.800000}%
\pgfsetstrokecolor{currentstroke}%
\pgfsetdash{}{0pt}%
\pgfpathmoveto{\pgfqpoint{6.086490in}{0.451389in}}%
\pgfpathlineto{\pgfqpoint{6.086490in}{2.204860in}}%
\pgfusepath{stroke}%
\end{pgfscope}%
\begin{pgfscope}%
\pgfsetrectcap%
\pgfsetmiterjoin%
\pgfsetlinewidth{1.254687pt}%
\definecolor{currentstroke}{rgb}{0.800000,0.800000,0.800000}%
\pgfsetstrokecolor{currentstroke}%
\pgfsetdash{}{0pt}%
\pgfpathmoveto{\pgfqpoint{4.582850in}{0.451389in}}%
\pgfpathlineto{\pgfqpoint{6.086490in}{0.451389in}}%
\pgfusepath{stroke}%
\end{pgfscope}%
\begin{pgfscope}%
\pgfsetrectcap%
\pgfsetmiterjoin%
\pgfsetlinewidth{1.254687pt}%
\definecolor{currentstroke}{rgb}{0.800000,0.800000,0.800000}%
\pgfsetstrokecolor{currentstroke}%
\pgfsetdash{}{0pt}%
\pgfpathmoveto{\pgfqpoint{4.582850in}{2.204860in}}%
\pgfpathlineto{\pgfqpoint{6.086490in}{2.204860in}}%
\pgfusepath{stroke}%
\end{pgfscope}%
\begin{pgfscope}%
\definecolor{textcolor}{rgb}{0.150000,0.150000,0.150000}%
\pgfsetstrokecolor{textcolor}%
\pgfsetfillcolor{textcolor}%
\pgftext[x=5.334670in,y=2.315971in,,base]{\color{textcolor}\sffamily\fontsize{9.000000}{10.800000}\selectfont Circles}%
\end{pgfscope}%
\begin{pgfscope}%
\pgfsetroundcap%
\pgfsetroundjoin%
\pgfsetlinewidth{1.003750pt}%
\definecolor{currentstroke}{rgb}{0.003922,0.450980,0.698039}%
\pgfsetstrokecolor{currentstroke}%
\pgfsetdash{}{0pt}%
\pgfpathmoveto{\pgfqpoint{4.597177in}{0.451575in}}%
\pgfpathlineto{\pgfqpoint{4.640159in}{0.453563in}}%
\pgfpathlineto{\pgfqpoint{4.711795in}{0.460712in}}%
\pgfpathlineto{\pgfqpoint{4.812085in}{0.479566in}}%
\pgfpathlineto{\pgfqpoint{4.941030in}{0.523222in}}%
\pgfpathlineto{\pgfqpoint{5.098629in}{0.592497in}}%
\pgfpathlineto{\pgfqpoint{5.284883in}{0.713360in}}%
\pgfpathlineto{\pgfqpoint{5.499791in}{1.055155in}}%
\pgfpathlineto{\pgfqpoint{5.743353in}{1.422895in}}%
\pgfpathlineto{\pgfqpoint{6.015570in}{2.105667in}}%
\pgfusepath{stroke}%
\end{pgfscope}%
\begin{pgfscope}%
\pgfsetbuttcap%
\pgfsetroundjoin%
\definecolor{currentfill}{rgb}{0.003922,0.450980,0.698039}%
\pgfsetfillcolor{currentfill}%
\pgfsetlinewidth{0.752812pt}%
\definecolor{currentstroke}{rgb}{1.000000,1.000000,1.000000}%
\pgfsetstrokecolor{currentstroke}%
\pgfsetdash{}{0pt}%
\pgfsys@defobject{currentmarker}{\pgfqpoint{-0.034722in}{-0.034722in}}{\pgfqpoint{0.034722in}{0.034722in}}{%
\pgfpathmoveto{\pgfqpoint{0.000000in}{-0.034722in}}%
\pgfpathcurveto{\pgfqpoint{0.009208in}{-0.034722in}}{\pgfqpoint{0.018041in}{-0.031064in}}{\pgfqpoint{0.024552in}{-0.024552in}}%
\pgfpathcurveto{\pgfqpoint{0.031064in}{-0.018041in}}{\pgfqpoint{0.034722in}{-0.009208in}}{\pgfqpoint{0.034722in}{0.000000in}}%
\pgfpathcurveto{\pgfqpoint{0.034722in}{0.009208in}}{\pgfqpoint{0.031064in}{0.018041in}}{\pgfqpoint{0.024552in}{0.024552in}}%
\pgfpathcurveto{\pgfqpoint{0.018041in}{0.031064in}}{\pgfqpoint{0.009208in}{0.034722in}}{\pgfqpoint{0.000000in}{0.034722in}}%
\pgfpathcurveto{\pgfqpoint{-0.009208in}{0.034722in}}{\pgfqpoint{-0.018041in}{0.031064in}}{\pgfqpoint{-0.024552in}{0.024552in}}%
\pgfpathcurveto{\pgfqpoint{-0.031064in}{0.018041in}}{\pgfqpoint{-0.034722in}{0.009208in}}{\pgfqpoint{-0.034722in}{0.000000in}}%
\pgfpathcurveto{\pgfqpoint{-0.034722in}{-0.009208in}}{\pgfqpoint{-0.031064in}{-0.018041in}}{\pgfqpoint{-0.024552in}{-0.024552in}}%
\pgfpathcurveto{\pgfqpoint{-0.018041in}{-0.031064in}}{\pgfqpoint{-0.009208in}{-0.034722in}}{\pgfqpoint{0.000000in}{-0.034722in}}%
\pgfpathlineto{\pgfqpoint{0.000000in}{-0.034722in}}%
\pgfpathclose%
\pgfusepath{stroke,fill}%
}%
\begin{pgfscope}%
\pgfsys@transformshift{4.597177in}{0.451575in}%
\pgfsys@useobject{currentmarker}{}%
\end{pgfscope}%
\begin{pgfscope}%
\pgfsys@transformshift{4.640159in}{0.453563in}%
\pgfsys@useobject{currentmarker}{}%
\end{pgfscope}%
\begin{pgfscope}%
\pgfsys@transformshift{4.711795in}{0.460712in}%
\pgfsys@useobject{currentmarker}{}%
\end{pgfscope}%
\begin{pgfscope}%
\pgfsys@transformshift{4.812085in}{0.479566in}%
\pgfsys@useobject{currentmarker}{}%
\end{pgfscope}%
\begin{pgfscope}%
\pgfsys@transformshift{4.941030in}{0.523222in}%
\pgfsys@useobject{currentmarker}{}%
\end{pgfscope}%
\begin{pgfscope}%
\pgfsys@transformshift{5.098629in}{0.592497in}%
\pgfsys@useobject{currentmarker}{}%
\end{pgfscope}%
\begin{pgfscope}%
\pgfsys@transformshift{5.284883in}{0.713360in}%
\pgfsys@useobject{currentmarker}{}%
\end{pgfscope}%
\begin{pgfscope}%
\pgfsys@transformshift{5.499791in}{1.055155in}%
\pgfsys@useobject{currentmarker}{}%
\end{pgfscope}%
\begin{pgfscope}%
\pgfsys@transformshift{5.743353in}{1.422895in}%
\pgfsys@useobject{currentmarker}{}%
\end{pgfscope}%
\begin{pgfscope}%
\pgfsys@transformshift{6.015570in}{2.105667in}%
\pgfsys@useobject{currentmarker}{}%
\end{pgfscope}%
\end{pgfscope}%
\end{pgfpicture}%
\makeatother%
\endgroup%

				\end{figcenter}
				\caption{Total graph generation times for all three pattern-based dataset categories.}
				\label{fig:eval-import-pattern-abs}
			\end{figure}
			
			\begin{figure}[h]
				\begin{figcenter}
					\begingroup%
\makeatletter%
\begin{pgfpicture}%
\pgfpathrectangle{\pgfpointorigin}{\pgfqpoint{6.054442in}{2.407638in}}%
\pgfusepath{use as bounding box}%
\begin{pgfscope}%
\pgfsetbuttcap%
\pgfsetmiterjoin%
\definecolor{currentfill}{rgb}{1.000000,1.000000,1.000000}%
\pgfsetfillcolor{currentfill}%
\pgfsetlinewidth{0.000000pt}%
\definecolor{currentstroke}{rgb}{1.000000,1.000000,1.000000}%
\pgfsetstrokecolor{currentstroke}%
\pgfsetdash{}{0pt}%
\pgfpathmoveto{\pgfqpoint{0.000000in}{0.000000in}}%
\pgfpathlineto{\pgfqpoint{6.054442in}{0.000000in}}%
\pgfpathlineto{\pgfqpoint{6.054442in}{2.407638in}}%
\pgfpathlineto{\pgfqpoint{0.000000in}{2.407638in}}%
\pgfpathlineto{\pgfqpoint{0.000000in}{0.000000in}}%
\pgfpathclose%
\pgfusepath{fill}%
\end{pgfscope}%
\begin{pgfscope}%
\pgfsetbuttcap%
\pgfsetmiterjoin%
\definecolor{currentfill}{rgb}{1.000000,1.000000,1.000000}%
\pgfsetfillcolor{currentfill}%
\pgfsetlinewidth{0.000000pt}%
\definecolor{currentstroke}{rgb}{0.000000,0.000000,0.000000}%
\pgfsetstrokecolor{currentstroke}%
\pgfsetstrokeopacity{0.000000}%
\pgfsetdash{}{0pt}%
\pgfpathmoveto{\pgfqpoint{0.592976in}{0.451389in}}%
\pgfpathlineto{\pgfqpoint{2.272388in}{0.451389in}}%
\pgfpathlineto{\pgfqpoint{2.272388in}{2.407638in}}%
\pgfpathlineto{\pgfqpoint{0.592976in}{2.407638in}}%
\pgfpathlineto{\pgfqpoint{0.592976in}{0.451389in}}%
\pgfpathclose%
\pgfusepath{fill}%
\end{pgfscope}%
\begin{pgfscope}%
\pgfpathrectangle{\pgfqpoint{0.592976in}{0.451389in}}{\pgfqpoint{1.679412in}{1.956249in}}%
\pgfusepath{clip}%
\pgfsetroundcap%
\pgfsetroundjoin%
\pgfsetlinewidth{1.003750pt}%
\definecolor{currentstroke}{rgb}{0.800000,0.800000,0.800000}%
\pgfsetstrokecolor{currentstroke}%
\pgfsetdash{}{0pt}%
\pgfpathmoveto{\pgfqpoint{0.653891in}{0.451389in}}%
\pgfpathlineto{\pgfqpoint{0.653891in}{2.407638in}}%
\pgfusepath{stroke}%
\end{pgfscope}%
\begin{pgfscope}%
\definecolor{textcolor}{rgb}{0.150000,0.150000,0.150000}%
\pgfsetstrokecolor{textcolor}%
\pgfsetfillcolor{textcolor}%
\pgftext[x=0.653891in,y=0.319444in,,top]{\color{textcolor}\sffamily\fontsize{9.000000}{10.800000}\selectfont 0k}%
\end{pgfscope}%
\begin{pgfscope}%
\pgfpathrectangle{\pgfqpoint{0.592976in}{0.451389in}}{\pgfqpoint{1.679412in}{1.956249in}}%
\pgfusepath{clip}%
\pgfsetroundcap%
\pgfsetroundjoin%
\pgfsetlinewidth{1.003750pt}%
\definecolor{currentstroke}{rgb}{0.800000,0.800000,0.800000}%
\pgfsetstrokecolor{currentstroke}%
\pgfsetdash{}{0pt}%
\pgfpathmoveto{\pgfqpoint{1.001224in}{0.451389in}}%
\pgfpathlineto{\pgfqpoint{1.001224in}{2.407638in}}%
\pgfusepath{stroke}%
\end{pgfscope}%
\begin{pgfscope}%
\definecolor{textcolor}{rgb}{0.150000,0.150000,0.150000}%
\pgfsetstrokecolor{textcolor}%
\pgfsetfillcolor{textcolor}%
\pgftext[x=1.001224in,y=0.319444in,,top]{\color{textcolor}\sffamily\fontsize{9.000000}{10.800000}\selectfont 10k}%
\end{pgfscope}%
\begin{pgfscope}%
\pgfpathrectangle{\pgfqpoint{0.592976in}{0.451389in}}{\pgfqpoint{1.679412in}{1.956249in}}%
\pgfusepath{clip}%
\pgfsetroundcap%
\pgfsetroundjoin%
\pgfsetlinewidth{1.003750pt}%
\definecolor{currentstroke}{rgb}{0.800000,0.800000,0.800000}%
\pgfsetstrokecolor{currentstroke}%
\pgfsetdash{}{0pt}%
\pgfpathmoveto{\pgfqpoint{1.348558in}{0.451389in}}%
\pgfpathlineto{\pgfqpoint{1.348558in}{2.407638in}}%
\pgfusepath{stroke}%
\end{pgfscope}%
\begin{pgfscope}%
\definecolor{textcolor}{rgb}{0.150000,0.150000,0.150000}%
\pgfsetstrokecolor{textcolor}%
\pgfsetfillcolor{textcolor}%
\pgftext[x=1.348558in,y=0.319444in,,top]{\color{textcolor}\sffamily\fontsize{9.000000}{10.800000}\selectfont 20k}%
\end{pgfscope}%
\begin{pgfscope}%
\pgfpathrectangle{\pgfqpoint{0.592976in}{0.451389in}}{\pgfqpoint{1.679412in}{1.956249in}}%
\pgfusepath{clip}%
\pgfsetroundcap%
\pgfsetroundjoin%
\pgfsetlinewidth{1.003750pt}%
\definecolor{currentstroke}{rgb}{0.800000,0.800000,0.800000}%
\pgfsetstrokecolor{currentstroke}%
\pgfsetdash{}{0pt}%
\pgfpathmoveto{\pgfqpoint{1.695891in}{0.451389in}}%
\pgfpathlineto{\pgfqpoint{1.695891in}{2.407638in}}%
\pgfusepath{stroke}%
\end{pgfscope}%
\begin{pgfscope}%
\definecolor{textcolor}{rgb}{0.150000,0.150000,0.150000}%
\pgfsetstrokecolor{textcolor}%
\pgfsetfillcolor{textcolor}%
\pgftext[x=1.695891in,y=0.319444in,,top]{\color{textcolor}\sffamily\fontsize{9.000000}{10.800000}\selectfont 30k}%
\end{pgfscope}%
\begin{pgfscope}%
\pgfpathrectangle{\pgfqpoint{0.592976in}{0.451389in}}{\pgfqpoint{1.679412in}{1.956249in}}%
\pgfusepath{clip}%
\pgfsetroundcap%
\pgfsetroundjoin%
\pgfsetlinewidth{1.003750pt}%
\definecolor{currentstroke}{rgb}{0.800000,0.800000,0.800000}%
\pgfsetstrokecolor{currentstroke}%
\pgfsetdash{}{0pt}%
\pgfpathmoveto{\pgfqpoint{2.043224in}{0.451389in}}%
\pgfpathlineto{\pgfqpoint{2.043224in}{2.407638in}}%
\pgfusepath{stroke}%
\end{pgfscope}%
\begin{pgfscope}%
\definecolor{textcolor}{rgb}{0.150000,0.150000,0.150000}%
\pgfsetstrokecolor{textcolor}%
\pgfsetfillcolor{textcolor}%
\pgftext[x=2.043224in,y=0.319444in,,top]{\color{textcolor}\sffamily\fontsize{9.000000}{10.800000}\selectfont 40k}%
\end{pgfscope}%
\begin{pgfscope}%
\definecolor{textcolor}{rgb}{0.150000,0.150000,0.150000}%
\pgfsetstrokecolor{textcolor}%
\pgfsetfillcolor{textcolor}%
\pgftext[x=1.432682in,y=0.125000in,,top]{\color{textcolor}\sffamily\fontsize{9.000000}{10.800000}\selectfont Input obstacle vertices}%
\end{pgfscope}%
\begin{pgfscope}%
\pgfpathrectangle{\pgfqpoint{0.592976in}{0.451389in}}{\pgfqpoint{1.679412in}{1.956249in}}%
\pgfusepath{clip}%
\pgfsetroundcap%
\pgfsetroundjoin%
\pgfsetlinewidth{1.003750pt}%
\definecolor{currentstroke}{rgb}{0.800000,0.800000,0.800000}%
\pgfsetstrokecolor{currentstroke}%
\pgfsetdash{}{0pt}%
\pgfpathmoveto{\pgfqpoint{0.592976in}{0.755672in}}%
\pgfpathlineto{\pgfqpoint{2.272388in}{0.755672in}}%
\pgfusepath{stroke}%
\end{pgfscope}%
\begin{pgfscope}%
\definecolor{textcolor}{rgb}{0.150000,0.150000,0.150000}%
\pgfsetstrokecolor{textcolor}%
\pgfsetfillcolor{textcolor}%
\pgftext[x=0.194444in, y=0.708187in, left, base]{\color{textcolor}\sffamily\fontsize{9.000000}{10.800000}\selectfont \(\displaystyle {10^{-4}}\)}%
\end{pgfscope}%
\begin{pgfscope}%
\pgfpathrectangle{\pgfqpoint{0.592976in}{0.451389in}}{\pgfqpoint{1.679412in}{1.956249in}}%
\pgfusepath{clip}%
\pgfsetroundcap%
\pgfsetroundjoin%
\pgfsetlinewidth{1.003750pt}%
\definecolor{currentstroke}{rgb}{0.800000,0.800000,0.800000}%
\pgfsetstrokecolor{currentstroke}%
\pgfsetdash{}{0pt}%
\pgfpathmoveto{\pgfqpoint{0.592976in}{1.208696in}}%
\pgfpathlineto{\pgfqpoint{2.272388in}{1.208696in}}%
\pgfusepath{stroke}%
\end{pgfscope}%
\begin{pgfscope}%
\definecolor{textcolor}{rgb}{0.150000,0.150000,0.150000}%
\pgfsetstrokecolor{textcolor}%
\pgfsetfillcolor{textcolor}%
\pgftext[x=0.194444in, y=1.161210in, left, base]{\color{textcolor}\sffamily\fontsize{9.000000}{10.800000}\selectfont \(\displaystyle {10^{-2}}\)}%
\end{pgfscope}%
\begin{pgfscope}%
\pgfpathrectangle{\pgfqpoint{0.592976in}{0.451389in}}{\pgfqpoint{1.679412in}{1.956249in}}%
\pgfusepath{clip}%
\pgfsetroundcap%
\pgfsetroundjoin%
\pgfsetlinewidth{1.003750pt}%
\definecolor{currentstroke}{rgb}{0.800000,0.800000,0.800000}%
\pgfsetstrokecolor{currentstroke}%
\pgfsetdash{}{0pt}%
\pgfpathmoveto{\pgfqpoint{0.592976in}{1.661719in}}%
\pgfpathlineto{\pgfqpoint{2.272388in}{1.661719in}}%
\pgfusepath{stroke}%
\end{pgfscope}%
\begin{pgfscope}%
\definecolor{textcolor}{rgb}{0.150000,0.150000,0.150000}%
\pgfsetstrokecolor{textcolor}%
\pgfsetfillcolor{textcolor}%
\pgftext[x=0.274690in, y=1.614234in, left, base]{\color{textcolor}\sffamily\fontsize{9.000000}{10.800000}\selectfont \(\displaystyle {10^{0}}\)}%
\end{pgfscope}%
\begin{pgfscope}%
\pgfpathrectangle{\pgfqpoint{0.592976in}{0.451389in}}{\pgfqpoint{1.679412in}{1.956249in}}%
\pgfusepath{clip}%
\pgfsetroundcap%
\pgfsetroundjoin%
\pgfsetlinewidth{1.003750pt}%
\definecolor{currentstroke}{rgb}{0.800000,0.800000,0.800000}%
\pgfsetstrokecolor{currentstroke}%
\pgfsetdash{}{0pt}%
\pgfpathmoveto{\pgfqpoint{0.592976in}{2.114743in}}%
\pgfpathlineto{\pgfqpoint{2.272388in}{2.114743in}}%
\pgfusepath{stroke}%
\end{pgfscope}%
\begin{pgfscope}%
\definecolor{textcolor}{rgb}{0.150000,0.150000,0.150000}%
\pgfsetstrokecolor{textcolor}%
\pgfsetfillcolor{textcolor}%
\pgftext[x=0.274690in, y=2.067257in, left, base]{\color{textcolor}\sffamily\fontsize{9.000000}{10.800000}\selectfont \(\displaystyle {10^{2}}\)}%
\end{pgfscope}%
\begin{pgfscope}%
\definecolor{textcolor}{rgb}{0.150000,0.150000,0.150000}%
\pgfsetstrokecolor{textcolor}%
\pgfsetfillcolor{textcolor}%
\pgftext[x=0.125000in,y=1.429513in,,bottom,rotate=90.000000]{\color{textcolor}\sffamily\fontsize{9.000000}{10.800000}\selectfont Time in s}%
\end{pgfscope}%
\begin{pgfscope}%
\pgfpathrectangle{\pgfqpoint{0.592976in}{0.451389in}}{\pgfqpoint{1.679412in}{1.956249in}}%
\pgfusepath{clip}%
\pgfsetbuttcap%
\pgfsetroundjoin%
\definecolor{currentfill}{rgb}{0.003922,0.450980,0.698039}%
\pgfsetfillcolor{currentfill}%
\pgfsetfillopacity{0.200000}%
\pgfsetlinewidth{1.003750pt}%
\definecolor{currentstroke}{rgb}{0.003922,0.450980,0.698039}%
\pgfsetstrokecolor{currentstroke}%
\pgfsetstrokeopacity{0.200000}%
\pgfsetdash{}{0pt}%
\pgfsys@defobject{currentmarker}{\pgfqpoint{0.669313in}{1.412599in}}{\pgfqpoint{2.196051in}{2.318717in}}{%
\pgfpathmoveto{\pgfqpoint{0.669313in}{1.424224in}}%
\pgfpathlineto{\pgfqpoint{0.669313in}{1.412599in}}%
\pgfpathlineto{\pgfqpoint{0.715578in}{1.668299in}}%
\pgfpathlineto{\pgfqpoint{0.792686in}{1.817149in}}%
\pgfpathlineto{\pgfqpoint{0.900637in}{1.922725in}}%
\pgfpathlineto{\pgfqpoint{1.039431in}{2.007139in}}%
\pgfpathlineto{\pgfqpoint{1.209069in}{2.088879in}}%
\pgfpathlineto{\pgfqpoint{1.409550in}{2.144111in}}%
\pgfpathlineto{\pgfqpoint{1.640874in}{2.212234in}}%
\pgfpathlineto{\pgfqpoint{1.903041in}{2.253748in}}%
\pgfpathlineto{\pgfqpoint{2.196051in}{2.317437in}}%
\pgfpathlineto{\pgfqpoint{2.196051in}{2.318717in}}%
\pgfpathlineto{\pgfqpoint{2.196051in}{2.318717in}}%
\pgfpathlineto{\pgfqpoint{1.903041in}{2.254422in}}%
\pgfpathlineto{\pgfqpoint{1.640874in}{2.215435in}}%
\pgfpathlineto{\pgfqpoint{1.409550in}{2.145461in}}%
\pgfpathlineto{\pgfqpoint{1.209069in}{2.090386in}}%
\pgfpathlineto{\pgfqpoint{1.039431in}{2.008834in}}%
\pgfpathlineto{\pgfqpoint{0.900637in}{1.923188in}}%
\pgfpathlineto{\pgfqpoint{0.792686in}{1.819576in}}%
\pgfpathlineto{\pgfqpoint{0.715578in}{1.669556in}}%
\pgfpathlineto{\pgfqpoint{0.669313in}{1.424224in}}%
\pgfpathlineto{\pgfqpoint{0.669313in}{1.424224in}}%
\pgfpathclose%
\pgfusepath{stroke,fill}%
}%
\begin{pgfscope}%
\pgfsys@transformshift{0.000000in}{0.000000in}%
\pgfsys@useobject{currentmarker}{}%
\end{pgfscope}%
\end{pgfscope}%
\begin{pgfscope}%
\pgfpathrectangle{\pgfqpoint{0.592976in}{0.451389in}}{\pgfqpoint{1.679412in}{1.956249in}}%
\pgfusepath{clip}%
\pgfsetbuttcap%
\pgfsetroundjoin%
\definecolor{currentfill}{rgb}{0.870588,0.560784,0.019608}%
\pgfsetfillcolor{currentfill}%
\pgfsetfillopacity{0.200000}%
\pgfsetlinewidth{1.003750pt}%
\definecolor{currentstroke}{rgb}{0.870588,0.560784,0.019608}%
\pgfsetstrokecolor{currentstroke}%
\pgfsetstrokeopacity{0.200000}%
\pgfsetdash{}{0pt}%
\pgfsys@defobject{currentmarker}{\pgfqpoint{0.669313in}{1.403320in}}{\pgfqpoint{2.196051in}{2.318547in}}{%
\pgfpathmoveto{\pgfqpoint{0.669313in}{1.412098in}}%
\pgfpathlineto{\pgfqpoint{0.669313in}{1.403320in}}%
\pgfpathlineto{\pgfqpoint{0.715578in}{1.664951in}}%
\pgfpathlineto{\pgfqpoint{0.792686in}{1.814932in}}%
\pgfpathlineto{\pgfqpoint{0.900637in}{1.921390in}}%
\pgfpathlineto{\pgfqpoint{1.039431in}{2.006017in}}%
\pgfpathlineto{\pgfqpoint{1.209069in}{2.088170in}}%
\pgfpathlineto{\pgfqpoint{1.409550in}{2.143559in}}%
\pgfpathlineto{\pgfqpoint{1.640874in}{2.211900in}}%
\pgfpathlineto{\pgfqpoint{1.903041in}{2.253486in}}%
\pgfpathlineto{\pgfqpoint{2.196051in}{2.317267in}}%
\pgfpathlineto{\pgfqpoint{2.196051in}{2.318547in}}%
\pgfpathlineto{\pgfqpoint{2.196051in}{2.318547in}}%
\pgfpathlineto{\pgfqpoint{1.903041in}{2.254163in}}%
\pgfpathlineto{\pgfqpoint{1.640874in}{2.215129in}}%
\pgfpathlineto{\pgfqpoint{1.409550in}{2.144908in}}%
\pgfpathlineto{\pgfqpoint{1.209069in}{2.089696in}}%
\pgfpathlineto{\pgfqpoint{1.039431in}{2.007853in}}%
\pgfpathlineto{\pgfqpoint{0.900637in}{1.921858in}}%
\pgfpathlineto{\pgfqpoint{0.792686in}{1.817463in}}%
\pgfpathlineto{\pgfqpoint{0.715578in}{1.665986in}}%
\pgfpathlineto{\pgfqpoint{0.669313in}{1.412098in}}%
\pgfpathlineto{\pgfqpoint{0.669313in}{1.412098in}}%
\pgfpathclose%
\pgfusepath{stroke,fill}%
}%
\begin{pgfscope}%
\pgfsys@transformshift{0.000000in}{0.000000in}%
\pgfsys@useobject{currentmarker}{}%
\end{pgfscope}%
\end{pgfscope}%
\begin{pgfscope}%
\pgfpathrectangle{\pgfqpoint{0.592976in}{0.451389in}}{\pgfqpoint{1.679412in}{1.956249in}}%
\pgfusepath{clip}%
\pgfsetbuttcap%
\pgfsetroundjoin%
\definecolor{currentfill}{rgb}{0.007843,0.619608,0.450980}%
\pgfsetfillcolor{currentfill}%
\pgfsetfillopacity{0.200000}%
\pgfsetlinewidth{1.003750pt}%
\definecolor{currentstroke}{rgb}{0.007843,0.619608,0.450980}%
\pgfsetstrokecolor{currentstroke}%
\pgfsetstrokeopacity{0.200000}%
\pgfsetdash{}{0pt}%
\pgfsys@defobject{currentmarker}{\pgfqpoint{0.669313in}{1.117698in}}{\pgfqpoint{2.196051in}{1.666814in}}{%
\pgfpathmoveto{\pgfqpoint{0.669313in}{1.124253in}}%
\pgfpathlineto{\pgfqpoint{0.669313in}{1.117698in}}%
\pgfpathlineto{\pgfqpoint{0.715578in}{1.292578in}}%
\pgfpathlineto{\pgfqpoint{0.792686in}{1.399924in}}%
\pgfpathlineto{\pgfqpoint{0.900637in}{1.462932in}}%
\pgfpathlineto{\pgfqpoint{1.039431in}{1.514824in}}%
\pgfpathlineto{\pgfqpoint{1.209069in}{1.573443in}}%
\pgfpathlineto{\pgfqpoint{1.409550in}{1.610249in}}%
\pgfpathlineto{\pgfqpoint{1.640874in}{1.618005in}}%
\pgfpathlineto{\pgfqpoint{1.903041in}{1.641818in}}%
\pgfpathlineto{\pgfqpoint{2.196051in}{1.664257in}}%
\pgfpathlineto{\pgfqpoint{2.196051in}{1.666814in}}%
\pgfpathlineto{\pgfqpoint{2.196051in}{1.666814in}}%
\pgfpathlineto{\pgfqpoint{1.903041in}{1.643643in}}%
\pgfpathlineto{\pgfqpoint{1.640874in}{1.632661in}}%
\pgfpathlineto{\pgfqpoint{1.409550in}{1.613566in}}%
\pgfpathlineto{\pgfqpoint{1.209069in}{1.582635in}}%
\pgfpathlineto{\pgfqpoint{1.039431in}{1.542598in}}%
\pgfpathlineto{\pgfqpoint{0.900637in}{1.467030in}}%
\pgfpathlineto{\pgfqpoint{0.792686in}{1.406784in}}%
\pgfpathlineto{\pgfqpoint{0.715578in}{1.296908in}}%
\pgfpathlineto{\pgfqpoint{0.669313in}{1.124253in}}%
\pgfpathlineto{\pgfqpoint{0.669313in}{1.124253in}}%
\pgfpathclose%
\pgfusepath{stroke,fill}%
}%
\begin{pgfscope}%
\pgfsys@transformshift{0.000000in}{0.000000in}%
\pgfsys@useobject{currentmarker}{}%
\end{pgfscope}%
\end{pgfscope}%
\begin{pgfscope}%
\pgfpathrectangle{\pgfqpoint{0.592976in}{0.451389in}}{\pgfqpoint{1.679412in}{1.956249in}}%
\pgfusepath{clip}%
\pgfsetbuttcap%
\pgfsetroundjoin%
\definecolor{currentfill}{rgb}{0.835294,0.368627,0.000000}%
\pgfsetfillcolor{currentfill}%
\pgfsetfillopacity{0.200000}%
\pgfsetlinewidth{1.003750pt}%
\definecolor{currentstroke}{rgb}{0.835294,0.368627,0.000000}%
\pgfsetstrokecolor{currentstroke}%
\pgfsetstrokeopacity{0.200000}%
\pgfsetdash{}{0pt}%
\pgfsys@defobject{currentmarker}{\pgfqpoint{0.669313in}{1.083020in}}{\pgfqpoint{2.196051in}{1.536461in}}{%
\pgfpathmoveto{\pgfqpoint{0.669313in}{1.098023in}}%
\pgfpathlineto{\pgfqpoint{0.669313in}{1.083020in}}%
\pgfpathlineto{\pgfqpoint{0.715578in}{1.212277in}}%
\pgfpathlineto{\pgfqpoint{0.792686in}{1.293535in}}%
\pgfpathlineto{\pgfqpoint{0.900637in}{1.348261in}}%
\pgfpathlineto{\pgfqpoint{1.039431in}{1.399682in}}%
\pgfpathlineto{\pgfqpoint{1.209069in}{1.433877in}}%
\pgfpathlineto{\pgfqpoint{1.409550in}{1.463362in}}%
\pgfpathlineto{\pgfqpoint{1.640874in}{1.486606in}}%
\pgfpathlineto{\pgfqpoint{1.903041in}{1.513695in}}%
\pgfpathlineto{\pgfqpoint{2.196051in}{1.531792in}}%
\pgfpathlineto{\pgfqpoint{2.196051in}{1.536461in}}%
\pgfpathlineto{\pgfqpoint{2.196051in}{1.536461in}}%
\pgfpathlineto{\pgfqpoint{1.903041in}{1.521165in}}%
\pgfpathlineto{\pgfqpoint{1.640874in}{1.493553in}}%
\pgfpathlineto{\pgfqpoint{1.409550in}{1.472340in}}%
\pgfpathlineto{\pgfqpoint{1.209069in}{1.454400in}}%
\pgfpathlineto{\pgfqpoint{1.039431in}{1.425373in}}%
\pgfpathlineto{\pgfqpoint{0.900637in}{1.371684in}}%
\pgfpathlineto{\pgfqpoint{0.792686in}{1.332531in}}%
\pgfpathlineto{\pgfqpoint{0.715578in}{1.216949in}}%
\pgfpathlineto{\pgfqpoint{0.669313in}{1.098023in}}%
\pgfpathlineto{\pgfqpoint{0.669313in}{1.098023in}}%
\pgfpathclose%
\pgfusepath{stroke,fill}%
}%
\begin{pgfscope}%
\pgfsys@transformshift{0.000000in}{0.000000in}%
\pgfsys@useobject{currentmarker}{}%
\end{pgfscope}%
\end{pgfscope}%
\begin{pgfscope}%
\pgfpathrectangle{\pgfqpoint{0.592976in}{0.451389in}}{\pgfqpoint{1.679412in}{1.956249in}}%
\pgfusepath{clip}%
\pgfsetbuttcap%
\pgfsetroundjoin%
\definecolor{currentfill}{rgb}{0.800000,0.470588,0.737255}%
\pgfsetfillcolor{currentfill}%
\pgfsetfillopacity{0.200000}%
\pgfsetlinewidth{1.003750pt}%
\definecolor{currentstroke}{rgb}{0.800000,0.470588,0.737255}%
\pgfsetstrokecolor{currentstroke}%
\pgfsetstrokeopacity{0.200000}%
\pgfsetdash{}{0pt}%
\pgfsys@defobject{currentmarker}{\pgfqpoint{0.669313in}{0.623911in}}{\pgfqpoint{2.196051in}{1.153430in}}{%
\pgfpathmoveto{\pgfqpoint{0.669313in}{0.651835in}}%
\pgfpathlineto{\pgfqpoint{0.669313in}{0.623911in}}%
\pgfpathlineto{\pgfqpoint{0.715578in}{0.624659in}}%
\pgfpathlineto{\pgfqpoint{0.792686in}{0.651835in}}%
\pgfpathlineto{\pgfqpoint{0.900637in}{0.677991in}}%
\pgfpathlineto{\pgfqpoint{1.039431in}{0.741292in}}%
\pgfpathlineto{\pgfqpoint{1.209069in}{0.787071in}}%
\pgfpathlineto{\pgfqpoint{1.409550in}{0.832787in}}%
\pgfpathlineto{\pgfqpoint{1.640874in}{0.811060in}}%
\pgfpathlineto{\pgfqpoint{1.903041in}{0.852870in}}%
\pgfpathlineto{\pgfqpoint{2.196051in}{0.870708in}}%
\pgfpathlineto{\pgfqpoint{2.196051in}{0.964256in}}%
\pgfpathlineto{\pgfqpoint{2.196051in}{0.964256in}}%
\pgfpathlineto{\pgfqpoint{1.903041in}{1.017328in}}%
\pgfpathlineto{\pgfqpoint{1.640874in}{1.005743in}}%
\pgfpathlineto{\pgfqpoint{1.409550in}{0.927712in}}%
\pgfpathlineto{\pgfqpoint{1.209069in}{1.153430in}}%
\pgfpathlineto{\pgfqpoint{1.039431in}{0.900071in}}%
\pgfpathlineto{\pgfqpoint{0.900637in}{0.884854in}}%
\pgfpathlineto{\pgfqpoint{0.792686in}{0.894090in}}%
\pgfpathlineto{\pgfqpoint{0.715578in}{0.661005in}}%
\pgfpathlineto{\pgfqpoint{0.669313in}{0.651835in}}%
\pgfpathlineto{\pgfqpoint{0.669313in}{0.651835in}}%
\pgfpathclose%
\pgfusepath{stroke,fill}%
}%
\begin{pgfscope}%
\pgfsys@transformshift{0.000000in}{0.000000in}%
\pgfsys@useobject{currentmarker}{}%
\end{pgfscope}%
\end{pgfscope}%
\begin{pgfscope}%
\pgfpathrectangle{\pgfqpoint{0.592976in}{0.451389in}}{\pgfqpoint{1.679412in}{1.956249in}}%
\pgfusepath{clip}%
\pgfsetbuttcap%
\pgfsetroundjoin%
\definecolor{currentfill}{rgb}{0.792157,0.568627,0.380392}%
\pgfsetfillcolor{currentfill}%
\pgfsetfillopacity{0.200000}%
\pgfsetlinewidth{1.003750pt}%
\definecolor{currentstroke}{rgb}{0.792157,0.568627,0.380392}%
\pgfsetstrokecolor{currentstroke}%
\pgfsetstrokeopacity{0.200000}%
\pgfsetdash{}{0pt}%
\pgfsys@defobject{currentmarker}{\pgfqpoint{0.669313in}{0.540309in}}{\pgfqpoint{2.196051in}{1.025233in}}{%
\pgfpathmoveto{\pgfqpoint{0.669313in}{0.603079in}}%
\pgfpathlineto{\pgfqpoint{0.669313in}{0.540309in}}%
\pgfpathlineto{\pgfqpoint{0.715578in}{0.596359in}}%
\pgfpathlineto{\pgfqpoint{0.792686in}{0.639182in}}%
\pgfpathlineto{\pgfqpoint{0.900637in}{0.718884in}}%
\pgfpathlineto{\pgfqpoint{1.039431in}{0.778407in}}%
\pgfpathlineto{\pgfqpoint{1.209069in}{0.801167in}}%
\pgfpathlineto{\pgfqpoint{1.409550in}{0.843421in}}%
\pgfpathlineto{\pgfqpoint{1.640874in}{0.838035in}}%
\pgfpathlineto{\pgfqpoint{1.903041in}{0.863022in}}%
\pgfpathlineto{\pgfqpoint{2.196051in}{0.884536in}}%
\pgfpathlineto{\pgfqpoint{2.196051in}{1.025233in}}%
\pgfpathlineto{\pgfqpoint{2.196051in}{1.025233in}}%
\pgfpathlineto{\pgfqpoint{1.903041in}{0.894090in}}%
\pgfpathlineto{\pgfqpoint{1.640874in}{0.876750in}}%
\pgfpathlineto{\pgfqpoint{1.409550in}{0.859730in}}%
\pgfpathlineto{\pgfqpoint{1.209069in}{0.851839in}}%
\pgfpathlineto{\pgfqpoint{1.039431in}{0.834567in}}%
\pgfpathlineto{\pgfqpoint{0.900637in}{0.821167in}}%
\pgfpathlineto{\pgfqpoint{0.792686in}{0.691344in}}%
\pgfpathlineto{\pgfqpoint{0.715578in}{0.667482in}}%
\pgfpathlineto{\pgfqpoint{0.669313in}{0.603079in}}%
\pgfpathlineto{\pgfqpoint{0.669313in}{0.603079in}}%
\pgfpathclose%
\pgfusepath{stroke,fill}%
}%
\begin{pgfscope}%
\pgfsys@transformshift{0.000000in}{0.000000in}%
\pgfsys@useobject{currentmarker}{}%
\end{pgfscope}%
\end{pgfscope}%
\begin{pgfscope}%
\pgfsetrectcap%
\pgfsetmiterjoin%
\pgfsetlinewidth{1.254687pt}%
\definecolor{currentstroke}{rgb}{0.800000,0.800000,0.800000}%
\pgfsetstrokecolor{currentstroke}%
\pgfsetdash{}{0pt}%
\pgfpathmoveto{\pgfqpoint{0.592976in}{0.451389in}}%
\pgfpathlineto{\pgfqpoint{0.592976in}{2.407638in}}%
\pgfusepath{stroke}%
\end{pgfscope}%
\begin{pgfscope}%
\pgfsetrectcap%
\pgfsetmiterjoin%
\pgfsetlinewidth{1.254687pt}%
\definecolor{currentstroke}{rgb}{0.800000,0.800000,0.800000}%
\pgfsetstrokecolor{currentstroke}%
\pgfsetdash{}{0pt}%
\pgfpathmoveto{\pgfqpoint{2.272388in}{0.451389in}}%
\pgfpathlineto{\pgfqpoint{2.272388in}{2.407638in}}%
\pgfusepath{stroke}%
\end{pgfscope}%
\begin{pgfscope}%
\pgfsetrectcap%
\pgfsetmiterjoin%
\pgfsetlinewidth{1.254687pt}%
\definecolor{currentstroke}{rgb}{0.800000,0.800000,0.800000}%
\pgfsetstrokecolor{currentstroke}%
\pgfsetdash{}{0pt}%
\pgfpathmoveto{\pgfqpoint{0.592976in}{0.451389in}}%
\pgfpathlineto{\pgfqpoint{2.272388in}{0.451389in}}%
\pgfusepath{stroke}%
\end{pgfscope}%
\begin{pgfscope}%
\pgfsetrectcap%
\pgfsetmiterjoin%
\pgfsetlinewidth{1.254687pt}%
\definecolor{currentstroke}{rgb}{0.800000,0.800000,0.800000}%
\pgfsetstrokecolor{currentstroke}%
\pgfsetdash{}{0pt}%
\pgfpathmoveto{\pgfqpoint{0.592976in}{2.407638in}}%
\pgfpathlineto{\pgfqpoint{2.272388in}{2.407638in}}%
\pgfusepath{stroke}%
\end{pgfscope}%
\begin{pgfscope}%
\pgfsetroundcap%
\pgfsetroundjoin%
\pgfsetlinewidth{1.003750pt}%
\definecolor{currentstroke}{rgb}{0.003922,0.450980,0.698039}%
\pgfsetstrokecolor{currentstroke}%
\pgfsetdash{}{0pt}%
\pgfpathmoveto{\pgfqpoint{0.669313in}{1.418659in}}%
\pgfpathlineto{\pgfqpoint{0.715578in}{1.668734in}}%
\pgfpathlineto{\pgfqpoint{0.792686in}{1.818583in}}%
\pgfpathlineto{\pgfqpoint{0.900637in}{1.922962in}}%
\pgfpathlineto{\pgfqpoint{1.039431in}{2.007828in}}%
\pgfpathlineto{\pgfqpoint{1.209069in}{2.089646in}}%
\pgfpathlineto{\pgfqpoint{1.409550in}{2.144957in}}%
\pgfpathlineto{\pgfqpoint{1.640874in}{2.213823in}}%
\pgfpathlineto{\pgfqpoint{1.903041in}{2.254033in}}%
\pgfpathlineto{\pgfqpoint{2.196051in}{2.317903in}}%
\pgfusepath{stroke}%
\end{pgfscope}%
\begin{pgfscope}%
\pgfsetbuttcap%
\pgfsetroundjoin%
\definecolor{currentfill}{rgb}{0.003922,0.450980,0.698039}%
\pgfsetfillcolor{currentfill}%
\pgfsetlinewidth{0.752812pt}%
\definecolor{currentstroke}{rgb}{1.000000,1.000000,1.000000}%
\pgfsetstrokecolor{currentstroke}%
\pgfsetdash{}{0pt}%
\pgfsys@defobject{currentmarker}{\pgfqpoint{-0.034722in}{-0.034722in}}{\pgfqpoint{0.034722in}{0.034722in}}{%
\pgfpathmoveto{\pgfqpoint{0.000000in}{-0.034722in}}%
\pgfpathcurveto{\pgfqpoint{0.009208in}{-0.034722in}}{\pgfqpoint{0.018041in}{-0.031064in}}{\pgfqpoint{0.024552in}{-0.024552in}}%
\pgfpathcurveto{\pgfqpoint{0.031064in}{-0.018041in}}{\pgfqpoint{0.034722in}{-0.009208in}}{\pgfqpoint{0.034722in}{0.000000in}}%
\pgfpathcurveto{\pgfqpoint{0.034722in}{0.009208in}}{\pgfqpoint{0.031064in}{0.018041in}}{\pgfqpoint{0.024552in}{0.024552in}}%
\pgfpathcurveto{\pgfqpoint{0.018041in}{0.031064in}}{\pgfqpoint{0.009208in}{0.034722in}}{\pgfqpoint{0.000000in}{0.034722in}}%
\pgfpathcurveto{\pgfqpoint{-0.009208in}{0.034722in}}{\pgfqpoint{-0.018041in}{0.031064in}}{\pgfqpoint{-0.024552in}{0.024552in}}%
\pgfpathcurveto{\pgfqpoint{-0.031064in}{0.018041in}}{\pgfqpoint{-0.034722in}{0.009208in}}{\pgfqpoint{-0.034722in}{0.000000in}}%
\pgfpathcurveto{\pgfqpoint{-0.034722in}{-0.009208in}}{\pgfqpoint{-0.031064in}{-0.018041in}}{\pgfqpoint{-0.024552in}{-0.024552in}}%
\pgfpathcurveto{\pgfqpoint{-0.018041in}{-0.031064in}}{\pgfqpoint{-0.009208in}{-0.034722in}}{\pgfqpoint{0.000000in}{-0.034722in}}%
\pgfpathlineto{\pgfqpoint{0.000000in}{-0.034722in}}%
\pgfpathclose%
\pgfusepath{stroke,fill}%
}%
\begin{pgfscope}%
\pgfsys@transformshift{0.669313in}{1.418659in}%
\pgfsys@useobject{currentmarker}{}%
\end{pgfscope}%
\begin{pgfscope}%
\pgfsys@transformshift{0.715578in}{1.668734in}%
\pgfsys@useobject{currentmarker}{}%
\end{pgfscope}%
\begin{pgfscope}%
\pgfsys@transformshift{0.792686in}{1.818583in}%
\pgfsys@useobject{currentmarker}{}%
\end{pgfscope}%
\begin{pgfscope}%
\pgfsys@transformshift{0.900637in}{1.922962in}%
\pgfsys@useobject{currentmarker}{}%
\end{pgfscope}%
\begin{pgfscope}%
\pgfsys@transformshift{1.039431in}{2.007828in}%
\pgfsys@useobject{currentmarker}{}%
\end{pgfscope}%
\begin{pgfscope}%
\pgfsys@transformshift{1.209069in}{2.089646in}%
\pgfsys@useobject{currentmarker}{}%
\end{pgfscope}%
\begin{pgfscope}%
\pgfsys@transformshift{1.409550in}{2.144957in}%
\pgfsys@useobject{currentmarker}{}%
\end{pgfscope}%
\begin{pgfscope}%
\pgfsys@transformshift{1.640874in}{2.213823in}%
\pgfsys@useobject{currentmarker}{}%
\end{pgfscope}%
\begin{pgfscope}%
\pgfsys@transformshift{1.903041in}{2.254033in}%
\pgfsys@useobject{currentmarker}{}%
\end{pgfscope}%
\begin{pgfscope}%
\pgfsys@transformshift{2.196051in}{2.317903in}%
\pgfsys@useobject{currentmarker}{}%
\end{pgfscope}%
\end{pgfscope}%
\begin{pgfscope}%
\pgfsetroundcap%
\pgfsetroundjoin%
\pgfsetlinewidth{1.003750pt}%
\definecolor{currentstroke}{rgb}{0.870588,0.560784,0.019608}%
\pgfsetstrokecolor{currentstroke}%
\pgfsetdash{}{0pt}%
\pgfpathmoveto{\pgfqpoint{0.669313in}{1.408620in}}%
\pgfpathlineto{\pgfqpoint{0.715578in}{1.665316in}}%
\pgfpathlineto{\pgfqpoint{0.792686in}{1.816451in}}%
\pgfpathlineto{\pgfqpoint{0.900637in}{1.921655in}}%
\pgfpathlineto{\pgfqpoint{1.039431in}{2.006777in}}%
\pgfpathlineto{\pgfqpoint{1.209069in}{2.088940in}}%
\pgfpathlineto{\pgfqpoint{1.409550in}{2.144402in}}%
\pgfpathlineto{\pgfqpoint{1.640874in}{2.213499in}}%
\pgfpathlineto{\pgfqpoint{1.903041in}{2.253773in}}%
\pgfpathlineto{\pgfqpoint{2.196051in}{2.317734in}}%
\pgfusepath{stroke}%
\end{pgfscope}%
\begin{pgfscope}%
\pgfsetbuttcap%
\pgfsetroundjoin%
\definecolor{currentfill}{rgb}{0.870588,0.560784,0.019608}%
\pgfsetfillcolor{currentfill}%
\pgfsetlinewidth{0.752812pt}%
\definecolor{currentstroke}{rgb}{1.000000,1.000000,1.000000}%
\pgfsetstrokecolor{currentstroke}%
\pgfsetdash{}{0pt}%
\pgfsys@defobject{currentmarker}{\pgfqpoint{-0.034722in}{-0.034722in}}{\pgfqpoint{0.034722in}{0.034722in}}{%
\pgfpathmoveto{\pgfqpoint{0.000000in}{-0.034722in}}%
\pgfpathcurveto{\pgfqpoint{0.009208in}{-0.034722in}}{\pgfqpoint{0.018041in}{-0.031064in}}{\pgfqpoint{0.024552in}{-0.024552in}}%
\pgfpathcurveto{\pgfqpoint{0.031064in}{-0.018041in}}{\pgfqpoint{0.034722in}{-0.009208in}}{\pgfqpoint{0.034722in}{0.000000in}}%
\pgfpathcurveto{\pgfqpoint{0.034722in}{0.009208in}}{\pgfqpoint{0.031064in}{0.018041in}}{\pgfqpoint{0.024552in}{0.024552in}}%
\pgfpathcurveto{\pgfqpoint{0.018041in}{0.031064in}}{\pgfqpoint{0.009208in}{0.034722in}}{\pgfqpoint{0.000000in}{0.034722in}}%
\pgfpathcurveto{\pgfqpoint{-0.009208in}{0.034722in}}{\pgfqpoint{-0.018041in}{0.031064in}}{\pgfqpoint{-0.024552in}{0.024552in}}%
\pgfpathcurveto{\pgfqpoint{-0.031064in}{0.018041in}}{\pgfqpoint{-0.034722in}{0.009208in}}{\pgfqpoint{-0.034722in}{0.000000in}}%
\pgfpathcurveto{\pgfqpoint{-0.034722in}{-0.009208in}}{\pgfqpoint{-0.031064in}{-0.018041in}}{\pgfqpoint{-0.024552in}{-0.024552in}}%
\pgfpathcurveto{\pgfqpoint{-0.018041in}{-0.031064in}}{\pgfqpoint{-0.009208in}{-0.034722in}}{\pgfqpoint{0.000000in}{-0.034722in}}%
\pgfpathlineto{\pgfqpoint{0.000000in}{-0.034722in}}%
\pgfpathclose%
\pgfusepath{stroke,fill}%
}%
\begin{pgfscope}%
\pgfsys@transformshift{0.669313in}{1.408620in}%
\pgfsys@useobject{currentmarker}{}%
\end{pgfscope}%
\begin{pgfscope}%
\pgfsys@transformshift{0.715578in}{1.665316in}%
\pgfsys@useobject{currentmarker}{}%
\end{pgfscope}%
\begin{pgfscope}%
\pgfsys@transformshift{0.792686in}{1.816451in}%
\pgfsys@useobject{currentmarker}{}%
\end{pgfscope}%
\begin{pgfscope}%
\pgfsys@transformshift{0.900637in}{1.921655in}%
\pgfsys@useobject{currentmarker}{}%
\end{pgfscope}%
\begin{pgfscope}%
\pgfsys@transformshift{1.039431in}{2.006777in}%
\pgfsys@useobject{currentmarker}{}%
\end{pgfscope}%
\begin{pgfscope}%
\pgfsys@transformshift{1.209069in}{2.088940in}%
\pgfsys@useobject{currentmarker}{}%
\end{pgfscope}%
\begin{pgfscope}%
\pgfsys@transformshift{1.409550in}{2.144402in}%
\pgfsys@useobject{currentmarker}{}%
\end{pgfscope}%
\begin{pgfscope}%
\pgfsys@transformshift{1.640874in}{2.213499in}%
\pgfsys@useobject{currentmarker}{}%
\end{pgfscope}%
\begin{pgfscope}%
\pgfsys@transformshift{1.903041in}{2.253773in}%
\pgfsys@useobject{currentmarker}{}%
\end{pgfscope}%
\begin{pgfscope}%
\pgfsys@transformshift{2.196051in}{2.317734in}%
\pgfsys@useobject{currentmarker}{}%
\end{pgfscope}%
\end{pgfscope}%
\begin{pgfscope}%
\pgfsetroundcap%
\pgfsetroundjoin%
\pgfsetlinewidth{1.003750pt}%
\definecolor{currentstroke}{rgb}{0.007843,0.619608,0.450980}%
\pgfsetstrokecolor{currentstroke}%
\pgfsetdash{}{0pt}%
\pgfpathmoveto{\pgfqpoint{0.669313in}{1.120279in}}%
\pgfpathlineto{\pgfqpoint{0.715578in}{1.294010in}}%
\pgfpathlineto{\pgfqpoint{0.792686in}{1.402944in}}%
\pgfpathlineto{\pgfqpoint{0.900637in}{1.464723in}}%
\pgfpathlineto{\pgfqpoint{1.039431in}{1.532098in}}%
\pgfpathlineto{\pgfqpoint{1.209069in}{1.578177in}}%
\pgfpathlineto{\pgfqpoint{1.409550in}{1.611994in}}%
\pgfpathlineto{\pgfqpoint{1.640874in}{1.626681in}}%
\pgfpathlineto{\pgfqpoint{1.903041in}{1.642691in}}%
\pgfpathlineto{\pgfqpoint{2.196051in}{1.665610in}}%
\pgfusepath{stroke}%
\end{pgfscope}%
\begin{pgfscope}%
\pgfsetbuttcap%
\pgfsetroundjoin%
\definecolor{currentfill}{rgb}{0.007843,0.619608,0.450980}%
\pgfsetfillcolor{currentfill}%
\pgfsetlinewidth{0.752812pt}%
\definecolor{currentstroke}{rgb}{1.000000,1.000000,1.000000}%
\pgfsetstrokecolor{currentstroke}%
\pgfsetdash{}{0pt}%
\pgfsys@defobject{currentmarker}{\pgfqpoint{-0.034722in}{-0.034722in}}{\pgfqpoint{0.034722in}{0.034722in}}{%
\pgfpathmoveto{\pgfqpoint{0.000000in}{-0.034722in}}%
\pgfpathcurveto{\pgfqpoint{0.009208in}{-0.034722in}}{\pgfqpoint{0.018041in}{-0.031064in}}{\pgfqpoint{0.024552in}{-0.024552in}}%
\pgfpathcurveto{\pgfqpoint{0.031064in}{-0.018041in}}{\pgfqpoint{0.034722in}{-0.009208in}}{\pgfqpoint{0.034722in}{0.000000in}}%
\pgfpathcurveto{\pgfqpoint{0.034722in}{0.009208in}}{\pgfqpoint{0.031064in}{0.018041in}}{\pgfqpoint{0.024552in}{0.024552in}}%
\pgfpathcurveto{\pgfqpoint{0.018041in}{0.031064in}}{\pgfqpoint{0.009208in}{0.034722in}}{\pgfqpoint{0.000000in}{0.034722in}}%
\pgfpathcurveto{\pgfqpoint{-0.009208in}{0.034722in}}{\pgfqpoint{-0.018041in}{0.031064in}}{\pgfqpoint{-0.024552in}{0.024552in}}%
\pgfpathcurveto{\pgfqpoint{-0.031064in}{0.018041in}}{\pgfqpoint{-0.034722in}{0.009208in}}{\pgfqpoint{-0.034722in}{0.000000in}}%
\pgfpathcurveto{\pgfqpoint{-0.034722in}{-0.009208in}}{\pgfqpoint{-0.031064in}{-0.018041in}}{\pgfqpoint{-0.024552in}{-0.024552in}}%
\pgfpathcurveto{\pgfqpoint{-0.018041in}{-0.031064in}}{\pgfqpoint{-0.009208in}{-0.034722in}}{\pgfqpoint{0.000000in}{-0.034722in}}%
\pgfpathlineto{\pgfqpoint{0.000000in}{-0.034722in}}%
\pgfpathclose%
\pgfusepath{stroke,fill}%
}%
\begin{pgfscope}%
\pgfsys@transformshift{0.669313in}{1.120279in}%
\pgfsys@useobject{currentmarker}{}%
\end{pgfscope}%
\begin{pgfscope}%
\pgfsys@transformshift{0.715578in}{1.294010in}%
\pgfsys@useobject{currentmarker}{}%
\end{pgfscope}%
\begin{pgfscope}%
\pgfsys@transformshift{0.792686in}{1.402944in}%
\pgfsys@useobject{currentmarker}{}%
\end{pgfscope}%
\begin{pgfscope}%
\pgfsys@transformshift{0.900637in}{1.464723in}%
\pgfsys@useobject{currentmarker}{}%
\end{pgfscope}%
\begin{pgfscope}%
\pgfsys@transformshift{1.039431in}{1.532098in}%
\pgfsys@useobject{currentmarker}{}%
\end{pgfscope}%
\begin{pgfscope}%
\pgfsys@transformshift{1.209069in}{1.578177in}%
\pgfsys@useobject{currentmarker}{}%
\end{pgfscope}%
\begin{pgfscope}%
\pgfsys@transformshift{1.409550in}{1.611994in}%
\pgfsys@useobject{currentmarker}{}%
\end{pgfscope}%
\begin{pgfscope}%
\pgfsys@transformshift{1.640874in}{1.626681in}%
\pgfsys@useobject{currentmarker}{}%
\end{pgfscope}%
\begin{pgfscope}%
\pgfsys@transformshift{1.903041in}{1.642691in}%
\pgfsys@useobject{currentmarker}{}%
\end{pgfscope}%
\begin{pgfscope}%
\pgfsys@transformshift{2.196051in}{1.665610in}%
\pgfsys@useobject{currentmarker}{}%
\end{pgfscope}%
\end{pgfscope}%
\begin{pgfscope}%
\pgfsetroundcap%
\pgfsetroundjoin%
\pgfsetlinewidth{1.003750pt}%
\definecolor{currentstroke}{rgb}{0.835294,0.368627,0.000000}%
\pgfsetstrokecolor{currentstroke}%
\pgfsetdash{}{0pt}%
\pgfpathmoveto{\pgfqpoint{0.669313in}{1.092859in}}%
\pgfpathlineto{\pgfqpoint{0.715578in}{1.215178in}}%
\pgfpathlineto{\pgfqpoint{0.792686in}{1.308420in}}%
\pgfpathlineto{\pgfqpoint{0.900637in}{1.356563in}}%
\pgfpathlineto{\pgfqpoint{1.039431in}{1.410462in}}%
\pgfpathlineto{\pgfqpoint{1.209069in}{1.441884in}}%
\pgfpathlineto{\pgfqpoint{1.409550in}{1.467128in}}%
\pgfpathlineto{\pgfqpoint{1.640874in}{1.489281in}}%
\pgfpathlineto{\pgfqpoint{1.903041in}{1.516557in}}%
\pgfpathlineto{\pgfqpoint{2.196051in}{1.534009in}}%
\pgfusepath{stroke}%
\end{pgfscope}%
\begin{pgfscope}%
\pgfsetbuttcap%
\pgfsetroundjoin%
\definecolor{currentfill}{rgb}{0.835294,0.368627,0.000000}%
\pgfsetfillcolor{currentfill}%
\pgfsetlinewidth{0.752812pt}%
\definecolor{currentstroke}{rgb}{1.000000,1.000000,1.000000}%
\pgfsetstrokecolor{currentstroke}%
\pgfsetdash{}{0pt}%
\pgfsys@defobject{currentmarker}{\pgfqpoint{-0.034722in}{-0.034722in}}{\pgfqpoint{0.034722in}{0.034722in}}{%
\pgfpathmoveto{\pgfqpoint{0.000000in}{-0.034722in}}%
\pgfpathcurveto{\pgfqpoint{0.009208in}{-0.034722in}}{\pgfqpoint{0.018041in}{-0.031064in}}{\pgfqpoint{0.024552in}{-0.024552in}}%
\pgfpathcurveto{\pgfqpoint{0.031064in}{-0.018041in}}{\pgfqpoint{0.034722in}{-0.009208in}}{\pgfqpoint{0.034722in}{0.000000in}}%
\pgfpathcurveto{\pgfqpoint{0.034722in}{0.009208in}}{\pgfqpoint{0.031064in}{0.018041in}}{\pgfqpoint{0.024552in}{0.024552in}}%
\pgfpathcurveto{\pgfqpoint{0.018041in}{0.031064in}}{\pgfqpoint{0.009208in}{0.034722in}}{\pgfqpoint{0.000000in}{0.034722in}}%
\pgfpathcurveto{\pgfqpoint{-0.009208in}{0.034722in}}{\pgfqpoint{-0.018041in}{0.031064in}}{\pgfqpoint{-0.024552in}{0.024552in}}%
\pgfpathcurveto{\pgfqpoint{-0.031064in}{0.018041in}}{\pgfqpoint{-0.034722in}{0.009208in}}{\pgfqpoint{-0.034722in}{0.000000in}}%
\pgfpathcurveto{\pgfqpoint{-0.034722in}{-0.009208in}}{\pgfqpoint{-0.031064in}{-0.018041in}}{\pgfqpoint{-0.024552in}{-0.024552in}}%
\pgfpathcurveto{\pgfqpoint{-0.018041in}{-0.031064in}}{\pgfqpoint{-0.009208in}{-0.034722in}}{\pgfqpoint{0.000000in}{-0.034722in}}%
\pgfpathlineto{\pgfqpoint{0.000000in}{-0.034722in}}%
\pgfpathclose%
\pgfusepath{stroke,fill}%
}%
\begin{pgfscope}%
\pgfsys@transformshift{0.669313in}{1.092859in}%
\pgfsys@useobject{currentmarker}{}%
\end{pgfscope}%
\begin{pgfscope}%
\pgfsys@transformshift{0.715578in}{1.215178in}%
\pgfsys@useobject{currentmarker}{}%
\end{pgfscope}%
\begin{pgfscope}%
\pgfsys@transformshift{0.792686in}{1.308420in}%
\pgfsys@useobject{currentmarker}{}%
\end{pgfscope}%
\begin{pgfscope}%
\pgfsys@transformshift{0.900637in}{1.356563in}%
\pgfsys@useobject{currentmarker}{}%
\end{pgfscope}%
\begin{pgfscope}%
\pgfsys@transformshift{1.039431in}{1.410462in}%
\pgfsys@useobject{currentmarker}{}%
\end{pgfscope}%
\begin{pgfscope}%
\pgfsys@transformshift{1.209069in}{1.441884in}%
\pgfsys@useobject{currentmarker}{}%
\end{pgfscope}%
\begin{pgfscope}%
\pgfsys@transformshift{1.409550in}{1.467128in}%
\pgfsys@useobject{currentmarker}{}%
\end{pgfscope}%
\begin{pgfscope}%
\pgfsys@transformshift{1.640874in}{1.489281in}%
\pgfsys@useobject{currentmarker}{}%
\end{pgfscope}%
\begin{pgfscope}%
\pgfsys@transformshift{1.903041in}{1.516557in}%
\pgfsys@useobject{currentmarker}{}%
\end{pgfscope}%
\begin{pgfscope}%
\pgfsys@transformshift{2.196051in}{1.534009in}%
\pgfsys@useobject{currentmarker}{}%
\end{pgfscope}%
\end{pgfscope}%
\begin{pgfscope}%
\pgfsetroundcap%
\pgfsetroundjoin%
\pgfsetlinewidth{1.003750pt}%
\definecolor{currentstroke}{rgb}{0.800000,0.470588,0.737255}%
\pgfsetstrokecolor{currentstroke}%
\pgfsetdash{}{0pt}%
\pgfpathmoveto{\pgfqpoint{0.669313in}{0.634575in}}%
\pgfpathlineto{\pgfqpoint{0.715578in}{0.642966in}}%
\pgfpathlineto{\pgfqpoint{0.792686in}{0.783133in}}%
\pgfpathlineto{\pgfqpoint{0.900637in}{0.787925in}}%
\pgfpathlineto{\pgfqpoint{1.039431in}{0.822671in}}%
\pgfpathlineto{\pgfqpoint{1.209069in}{1.028272in}}%
\pgfpathlineto{\pgfqpoint{1.409550in}{0.868295in}}%
\pgfpathlineto{\pgfqpoint{1.640874in}{0.925321in}}%
\pgfpathlineto{\pgfqpoint{1.903041in}{0.942700in}}%
\pgfpathlineto{\pgfqpoint{2.196051in}{0.921203in}}%
\pgfusepath{stroke}%
\end{pgfscope}%
\begin{pgfscope}%
\pgfsetbuttcap%
\pgfsetroundjoin%
\definecolor{currentfill}{rgb}{0.800000,0.470588,0.737255}%
\pgfsetfillcolor{currentfill}%
\pgfsetlinewidth{0.752812pt}%
\definecolor{currentstroke}{rgb}{1.000000,1.000000,1.000000}%
\pgfsetstrokecolor{currentstroke}%
\pgfsetdash{}{0pt}%
\pgfsys@defobject{currentmarker}{\pgfqpoint{-0.034722in}{-0.034722in}}{\pgfqpoint{0.034722in}{0.034722in}}{%
\pgfpathmoveto{\pgfqpoint{0.000000in}{-0.034722in}}%
\pgfpathcurveto{\pgfqpoint{0.009208in}{-0.034722in}}{\pgfqpoint{0.018041in}{-0.031064in}}{\pgfqpoint{0.024552in}{-0.024552in}}%
\pgfpathcurveto{\pgfqpoint{0.031064in}{-0.018041in}}{\pgfqpoint{0.034722in}{-0.009208in}}{\pgfqpoint{0.034722in}{0.000000in}}%
\pgfpathcurveto{\pgfqpoint{0.034722in}{0.009208in}}{\pgfqpoint{0.031064in}{0.018041in}}{\pgfqpoint{0.024552in}{0.024552in}}%
\pgfpathcurveto{\pgfqpoint{0.018041in}{0.031064in}}{\pgfqpoint{0.009208in}{0.034722in}}{\pgfqpoint{0.000000in}{0.034722in}}%
\pgfpathcurveto{\pgfqpoint{-0.009208in}{0.034722in}}{\pgfqpoint{-0.018041in}{0.031064in}}{\pgfqpoint{-0.024552in}{0.024552in}}%
\pgfpathcurveto{\pgfqpoint{-0.031064in}{0.018041in}}{\pgfqpoint{-0.034722in}{0.009208in}}{\pgfqpoint{-0.034722in}{0.000000in}}%
\pgfpathcurveto{\pgfqpoint{-0.034722in}{-0.009208in}}{\pgfqpoint{-0.031064in}{-0.018041in}}{\pgfqpoint{-0.024552in}{-0.024552in}}%
\pgfpathcurveto{\pgfqpoint{-0.018041in}{-0.031064in}}{\pgfqpoint{-0.009208in}{-0.034722in}}{\pgfqpoint{0.000000in}{-0.034722in}}%
\pgfpathlineto{\pgfqpoint{0.000000in}{-0.034722in}}%
\pgfpathclose%
\pgfusepath{stroke,fill}%
}%
\begin{pgfscope}%
\pgfsys@transformshift{0.669313in}{0.634575in}%
\pgfsys@useobject{currentmarker}{}%
\end{pgfscope}%
\begin{pgfscope}%
\pgfsys@transformshift{0.715578in}{0.642966in}%
\pgfsys@useobject{currentmarker}{}%
\end{pgfscope}%
\begin{pgfscope}%
\pgfsys@transformshift{0.792686in}{0.783133in}%
\pgfsys@useobject{currentmarker}{}%
\end{pgfscope}%
\begin{pgfscope}%
\pgfsys@transformshift{0.900637in}{0.787925in}%
\pgfsys@useobject{currentmarker}{}%
\end{pgfscope}%
\begin{pgfscope}%
\pgfsys@transformshift{1.039431in}{0.822671in}%
\pgfsys@useobject{currentmarker}{}%
\end{pgfscope}%
\begin{pgfscope}%
\pgfsys@transformshift{1.209069in}{1.028272in}%
\pgfsys@useobject{currentmarker}{}%
\end{pgfscope}%
\begin{pgfscope}%
\pgfsys@transformshift{1.409550in}{0.868295in}%
\pgfsys@useobject{currentmarker}{}%
\end{pgfscope}%
\begin{pgfscope}%
\pgfsys@transformshift{1.640874in}{0.925321in}%
\pgfsys@useobject{currentmarker}{}%
\end{pgfscope}%
\begin{pgfscope}%
\pgfsys@transformshift{1.903041in}{0.942700in}%
\pgfsys@useobject{currentmarker}{}%
\end{pgfscope}%
\begin{pgfscope}%
\pgfsys@transformshift{2.196051in}{0.921203in}%
\pgfsys@useobject{currentmarker}{}%
\end{pgfscope}%
\end{pgfscope}%
\begin{pgfscope}%
\pgfsetroundcap%
\pgfsetroundjoin%
\pgfsetlinewidth{1.003750pt}%
\definecolor{currentstroke}{rgb}{0.792157,0.568627,0.380392}%
\pgfsetstrokecolor{currentstroke}%
\pgfsetdash{}{0pt}%
\pgfpathmoveto{\pgfqpoint{0.669313in}{0.574159in}}%
\pgfpathlineto{\pgfqpoint{0.715578in}{0.630447in}}%
\pgfpathlineto{\pgfqpoint{0.792686in}{0.662538in}}%
\pgfpathlineto{\pgfqpoint{0.900637in}{0.773115in}}%
\pgfpathlineto{\pgfqpoint{1.039431in}{0.802521in}}%
\pgfpathlineto{\pgfqpoint{1.209069in}{0.824448in}}%
\pgfpathlineto{\pgfqpoint{1.409550in}{0.851617in}}%
\pgfpathlineto{\pgfqpoint{1.640874in}{0.862890in}}%
\pgfpathlineto{\pgfqpoint{1.903041in}{0.881900in}}%
\pgfpathlineto{\pgfqpoint{2.196051in}{0.951575in}}%
\pgfusepath{stroke}%
\end{pgfscope}%
\begin{pgfscope}%
\pgfsetbuttcap%
\pgfsetroundjoin%
\definecolor{currentfill}{rgb}{0.792157,0.568627,0.380392}%
\pgfsetfillcolor{currentfill}%
\pgfsetlinewidth{0.752812pt}%
\definecolor{currentstroke}{rgb}{1.000000,1.000000,1.000000}%
\pgfsetstrokecolor{currentstroke}%
\pgfsetdash{}{0pt}%
\pgfsys@defobject{currentmarker}{\pgfqpoint{-0.034722in}{-0.034722in}}{\pgfqpoint{0.034722in}{0.034722in}}{%
\pgfpathmoveto{\pgfqpoint{0.000000in}{-0.034722in}}%
\pgfpathcurveto{\pgfqpoint{0.009208in}{-0.034722in}}{\pgfqpoint{0.018041in}{-0.031064in}}{\pgfqpoint{0.024552in}{-0.024552in}}%
\pgfpathcurveto{\pgfqpoint{0.031064in}{-0.018041in}}{\pgfqpoint{0.034722in}{-0.009208in}}{\pgfqpoint{0.034722in}{0.000000in}}%
\pgfpathcurveto{\pgfqpoint{0.034722in}{0.009208in}}{\pgfqpoint{0.031064in}{0.018041in}}{\pgfqpoint{0.024552in}{0.024552in}}%
\pgfpathcurveto{\pgfqpoint{0.018041in}{0.031064in}}{\pgfqpoint{0.009208in}{0.034722in}}{\pgfqpoint{0.000000in}{0.034722in}}%
\pgfpathcurveto{\pgfqpoint{-0.009208in}{0.034722in}}{\pgfqpoint{-0.018041in}{0.031064in}}{\pgfqpoint{-0.024552in}{0.024552in}}%
\pgfpathcurveto{\pgfqpoint{-0.031064in}{0.018041in}}{\pgfqpoint{-0.034722in}{0.009208in}}{\pgfqpoint{-0.034722in}{0.000000in}}%
\pgfpathcurveto{\pgfqpoint{-0.034722in}{-0.009208in}}{\pgfqpoint{-0.031064in}{-0.018041in}}{\pgfqpoint{-0.024552in}{-0.024552in}}%
\pgfpathcurveto{\pgfqpoint{-0.018041in}{-0.031064in}}{\pgfqpoint{-0.009208in}{-0.034722in}}{\pgfqpoint{0.000000in}{-0.034722in}}%
\pgfpathlineto{\pgfqpoint{0.000000in}{-0.034722in}}%
\pgfpathclose%
\pgfusepath{stroke,fill}%
}%
\begin{pgfscope}%
\pgfsys@transformshift{0.669313in}{0.574159in}%
\pgfsys@useobject{currentmarker}{}%
\end{pgfscope}%
\begin{pgfscope}%
\pgfsys@transformshift{0.715578in}{0.630447in}%
\pgfsys@useobject{currentmarker}{}%
\end{pgfscope}%
\begin{pgfscope}%
\pgfsys@transformshift{0.792686in}{0.662538in}%
\pgfsys@useobject{currentmarker}{}%
\end{pgfscope}%
\begin{pgfscope}%
\pgfsys@transformshift{0.900637in}{0.773115in}%
\pgfsys@useobject{currentmarker}{}%
\end{pgfscope}%
\begin{pgfscope}%
\pgfsys@transformshift{1.039431in}{0.802521in}%
\pgfsys@useobject{currentmarker}{}%
\end{pgfscope}%
\begin{pgfscope}%
\pgfsys@transformshift{1.209069in}{0.824448in}%
\pgfsys@useobject{currentmarker}{}%
\end{pgfscope}%
\begin{pgfscope}%
\pgfsys@transformshift{1.409550in}{0.851617in}%
\pgfsys@useobject{currentmarker}{}%
\end{pgfscope}%
\begin{pgfscope}%
\pgfsys@transformshift{1.640874in}{0.862890in}%
\pgfsys@useobject{currentmarker}{}%
\end{pgfscope}%
\begin{pgfscope}%
\pgfsys@transformshift{1.903041in}{0.881900in}%
\pgfsys@useobject{currentmarker}{}%
\end{pgfscope}%
\begin{pgfscope}%
\pgfsys@transformshift{2.196051in}{0.951575in}%
\pgfsys@useobject{currentmarker}{}%
\end{pgfscope}%
\end{pgfscope}%
\begin{pgfscope}%
\pgfsetbuttcap%
\pgfsetmiterjoin%
\definecolor{currentfill}{rgb}{1.000000,1.000000,1.000000}%
\pgfsetfillcolor{currentfill}%
\pgfsetlinewidth{0.000000pt}%
\definecolor{currentstroke}{rgb}{0.000000,0.000000,0.000000}%
\pgfsetstrokecolor{currentstroke}%
\pgfsetstrokeopacity{0.000000}%
\pgfsetdash{}{0pt}%
\pgfpathmoveto{\pgfqpoint{3.025989in}{0.451389in}}%
\pgfpathlineto{\pgfqpoint{4.705401in}{0.451389in}}%
\pgfpathlineto{\pgfqpoint{4.705401in}{2.407638in}}%
\pgfpathlineto{\pgfqpoint{3.025989in}{2.407638in}}%
\pgfpathlineto{\pgfqpoint{3.025989in}{0.451389in}}%
\pgfpathclose%
\pgfusepath{fill}%
\end{pgfscope}%
\begin{pgfscope}%
\pgfpathrectangle{\pgfqpoint{3.025989in}{0.451389in}}{\pgfqpoint{1.679412in}{1.956249in}}%
\pgfusepath{clip}%
\pgfsetroundcap%
\pgfsetroundjoin%
\pgfsetlinewidth{1.003750pt}%
\definecolor{currentstroke}{rgb}{0.800000,0.800000,0.800000}%
\pgfsetstrokecolor{currentstroke}%
\pgfsetdash{}{0pt}%
\pgfpathmoveto{\pgfqpoint{3.086904in}{0.451389in}}%
\pgfpathlineto{\pgfqpoint{3.086904in}{2.407638in}}%
\pgfusepath{stroke}%
\end{pgfscope}%
\begin{pgfscope}%
\definecolor{textcolor}{rgb}{0.150000,0.150000,0.150000}%
\pgfsetstrokecolor{textcolor}%
\pgfsetfillcolor{textcolor}%
\pgftext[x=3.086904in,y=0.319444in,,top]{\color{textcolor}\sffamily\fontsize{9.000000}{10.800000}\selectfont 0k}%
\end{pgfscope}%
\begin{pgfscope}%
\pgfpathrectangle{\pgfqpoint{3.025989in}{0.451389in}}{\pgfqpoint{1.679412in}{1.956249in}}%
\pgfusepath{clip}%
\pgfsetroundcap%
\pgfsetroundjoin%
\pgfsetlinewidth{1.003750pt}%
\definecolor{currentstroke}{rgb}{0.800000,0.800000,0.800000}%
\pgfsetstrokecolor{currentstroke}%
\pgfsetdash{}{0pt}%
\pgfpathmoveto{\pgfqpoint{3.434237in}{0.451389in}}%
\pgfpathlineto{\pgfqpoint{3.434237in}{2.407638in}}%
\pgfusepath{stroke}%
\end{pgfscope}%
\begin{pgfscope}%
\definecolor{textcolor}{rgb}{0.150000,0.150000,0.150000}%
\pgfsetstrokecolor{textcolor}%
\pgfsetfillcolor{textcolor}%
\pgftext[x=3.434237in,y=0.319444in,,top]{\color{textcolor}\sffamily\fontsize{9.000000}{10.800000}\selectfont 10k}%
\end{pgfscope}%
\begin{pgfscope}%
\pgfpathrectangle{\pgfqpoint{3.025989in}{0.451389in}}{\pgfqpoint{1.679412in}{1.956249in}}%
\pgfusepath{clip}%
\pgfsetroundcap%
\pgfsetroundjoin%
\pgfsetlinewidth{1.003750pt}%
\definecolor{currentstroke}{rgb}{0.800000,0.800000,0.800000}%
\pgfsetstrokecolor{currentstroke}%
\pgfsetdash{}{0pt}%
\pgfpathmoveto{\pgfqpoint{3.781570in}{0.451389in}}%
\pgfpathlineto{\pgfqpoint{3.781570in}{2.407638in}}%
\pgfusepath{stroke}%
\end{pgfscope}%
\begin{pgfscope}%
\definecolor{textcolor}{rgb}{0.150000,0.150000,0.150000}%
\pgfsetstrokecolor{textcolor}%
\pgfsetfillcolor{textcolor}%
\pgftext[x=3.781570in,y=0.319444in,,top]{\color{textcolor}\sffamily\fontsize{9.000000}{10.800000}\selectfont 20k}%
\end{pgfscope}%
\begin{pgfscope}%
\pgfpathrectangle{\pgfqpoint{3.025989in}{0.451389in}}{\pgfqpoint{1.679412in}{1.956249in}}%
\pgfusepath{clip}%
\pgfsetroundcap%
\pgfsetroundjoin%
\pgfsetlinewidth{1.003750pt}%
\definecolor{currentstroke}{rgb}{0.800000,0.800000,0.800000}%
\pgfsetstrokecolor{currentstroke}%
\pgfsetdash{}{0pt}%
\pgfpathmoveto{\pgfqpoint{4.128904in}{0.451389in}}%
\pgfpathlineto{\pgfqpoint{4.128904in}{2.407638in}}%
\pgfusepath{stroke}%
\end{pgfscope}%
\begin{pgfscope}%
\definecolor{textcolor}{rgb}{0.150000,0.150000,0.150000}%
\pgfsetstrokecolor{textcolor}%
\pgfsetfillcolor{textcolor}%
\pgftext[x=4.128904in,y=0.319444in,,top]{\color{textcolor}\sffamily\fontsize{9.000000}{10.800000}\selectfont 30k}%
\end{pgfscope}%
\begin{pgfscope}%
\pgfpathrectangle{\pgfqpoint{3.025989in}{0.451389in}}{\pgfqpoint{1.679412in}{1.956249in}}%
\pgfusepath{clip}%
\pgfsetroundcap%
\pgfsetroundjoin%
\pgfsetlinewidth{1.003750pt}%
\definecolor{currentstroke}{rgb}{0.800000,0.800000,0.800000}%
\pgfsetstrokecolor{currentstroke}%
\pgfsetdash{}{0pt}%
\pgfpathmoveto{\pgfqpoint{4.476237in}{0.451389in}}%
\pgfpathlineto{\pgfqpoint{4.476237in}{2.407638in}}%
\pgfusepath{stroke}%
\end{pgfscope}%
\begin{pgfscope}%
\definecolor{textcolor}{rgb}{0.150000,0.150000,0.150000}%
\pgfsetstrokecolor{textcolor}%
\pgfsetfillcolor{textcolor}%
\pgftext[x=4.476237in,y=0.319444in,,top]{\color{textcolor}\sffamily\fontsize{9.000000}{10.800000}\selectfont 40k}%
\end{pgfscope}%
\begin{pgfscope}%
\definecolor{textcolor}{rgb}{0.150000,0.150000,0.150000}%
\pgfsetstrokecolor{textcolor}%
\pgfsetfillcolor{textcolor}%
\pgftext[x=3.865695in,y=0.125000in,,top]{\color{textcolor}\sffamily\fontsize{9.000000}{10.800000}\selectfont Input obstacle vertices}%
\end{pgfscope}%
\begin{pgfscope}%
\pgfpathrectangle{\pgfqpoint{3.025989in}{0.451389in}}{\pgfqpoint{1.679412in}{1.956249in}}%
\pgfusepath{clip}%
\pgfsetroundcap%
\pgfsetroundjoin%
\pgfsetlinewidth{1.003750pt}%
\definecolor{currentstroke}{rgb}{0.800000,0.800000,0.800000}%
\pgfsetstrokecolor{currentstroke}%
\pgfsetdash{}{0pt}%
\pgfpathmoveto{\pgfqpoint{3.025989in}{0.926577in}}%
\pgfpathlineto{\pgfqpoint{4.705401in}{0.926577in}}%
\pgfusepath{stroke}%
\end{pgfscope}%
\begin{pgfscope}%
\definecolor{textcolor}{rgb}{0.150000,0.150000,0.150000}%
\pgfsetstrokecolor{textcolor}%
\pgfsetfillcolor{textcolor}%
\pgftext[x=2.627457in, y=0.879091in, left, base]{\color{textcolor}\sffamily\fontsize{9.000000}{10.800000}\selectfont \(\displaystyle {10^{-5}}\)}%
\end{pgfscope}%
\begin{pgfscope}%
\pgfpathrectangle{\pgfqpoint{3.025989in}{0.451389in}}{\pgfqpoint{1.679412in}{1.956249in}}%
\pgfusepath{clip}%
\pgfsetroundcap%
\pgfsetroundjoin%
\pgfsetlinewidth{1.003750pt}%
\definecolor{currentstroke}{rgb}{0.800000,0.800000,0.800000}%
\pgfsetstrokecolor{currentstroke}%
\pgfsetdash{}{0pt}%
\pgfpathmoveto{\pgfqpoint{3.025989in}{1.483433in}}%
\pgfpathlineto{\pgfqpoint{4.705401in}{1.483433in}}%
\pgfusepath{stroke}%
\end{pgfscope}%
\begin{pgfscope}%
\definecolor{textcolor}{rgb}{0.150000,0.150000,0.150000}%
\pgfsetstrokecolor{textcolor}%
\pgfsetfillcolor{textcolor}%
\pgftext[x=2.627457in, y=1.435948in, left, base]{\color{textcolor}\sffamily\fontsize{9.000000}{10.800000}\selectfont \(\displaystyle {10^{-3}}\)}%
\end{pgfscope}%
\begin{pgfscope}%
\pgfpathrectangle{\pgfqpoint{3.025989in}{0.451389in}}{\pgfqpoint{1.679412in}{1.956249in}}%
\pgfusepath{clip}%
\pgfsetroundcap%
\pgfsetroundjoin%
\pgfsetlinewidth{1.003750pt}%
\definecolor{currentstroke}{rgb}{0.800000,0.800000,0.800000}%
\pgfsetstrokecolor{currentstroke}%
\pgfsetdash{}{0pt}%
\pgfpathmoveto{\pgfqpoint{3.025989in}{2.040289in}}%
\pgfpathlineto{\pgfqpoint{4.705401in}{2.040289in}}%
\pgfusepath{stroke}%
\end{pgfscope}%
\begin{pgfscope}%
\definecolor{textcolor}{rgb}{0.150000,0.150000,0.150000}%
\pgfsetstrokecolor{textcolor}%
\pgfsetfillcolor{textcolor}%
\pgftext[x=2.627457in, y=1.992804in, left, base]{\color{textcolor}\sffamily\fontsize{9.000000}{10.800000}\selectfont \(\displaystyle {10^{-1}}\)}%
\end{pgfscope}%
\begin{pgfscope}%
\definecolor{textcolor}{rgb}{0.150000,0.150000,0.150000}%
\pgfsetstrokecolor{textcolor}%
\pgfsetfillcolor{textcolor}%
\pgftext[x=2.558013in,y=1.429513in,,bottom,rotate=90.000000]{\color{textcolor}\sffamily\fontsize{9.000000}{10.800000}\selectfont Share of total time}%
\end{pgfscope}%
\begin{pgfscope}%
\pgfpathrectangle{\pgfqpoint{3.025989in}{0.451389in}}{\pgfqpoint{1.679412in}{1.956249in}}%
\pgfusepath{clip}%
\pgfsetbuttcap%
\pgfsetroundjoin%
\definecolor{currentfill}{rgb}{0.003922,0.450980,0.698039}%
\pgfsetfillcolor{currentfill}%
\pgfsetfillopacity{0.200000}%
\pgfsetlinewidth{1.003750pt}%
\definecolor{currentstroke}{rgb}{0.003922,0.450980,0.698039}%
\pgfsetstrokecolor{currentstroke}%
\pgfsetstrokeopacity{0.200000}%
\pgfsetdash{}{0pt}%
\pgfsys@defobject{currentmarker}{\pgfqpoint{3.102325in}{2.318717in}}{\pgfqpoint{4.629064in}{2.318717in}}{%
\pgfpathmoveto{\pgfqpoint{3.102325in}{2.318717in}}%
\pgfpathlineto{\pgfqpoint{3.102325in}{2.318717in}}%
\pgfpathlineto{\pgfqpoint{3.148590in}{2.318717in}}%
\pgfpathlineto{\pgfqpoint{3.225698in}{2.318717in}}%
\pgfpathlineto{\pgfqpoint{3.333649in}{2.318717in}}%
\pgfpathlineto{\pgfqpoint{3.472444in}{2.318717in}}%
\pgfpathlineto{\pgfqpoint{3.642081in}{2.318717in}}%
\pgfpathlineto{\pgfqpoint{3.842562in}{2.318717in}}%
\pgfpathlineto{\pgfqpoint{4.073886in}{2.318717in}}%
\pgfpathlineto{\pgfqpoint{4.336053in}{2.318717in}}%
\pgfpathlineto{\pgfqpoint{4.629064in}{2.318717in}}%
\pgfpathlineto{\pgfqpoint{4.629064in}{2.318717in}}%
\pgfpathlineto{\pgfqpoint{4.629064in}{2.318717in}}%
\pgfpathlineto{\pgfqpoint{4.336053in}{2.318717in}}%
\pgfpathlineto{\pgfqpoint{4.073886in}{2.318717in}}%
\pgfpathlineto{\pgfqpoint{3.842562in}{2.318717in}}%
\pgfpathlineto{\pgfqpoint{3.642081in}{2.318717in}}%
\pgfpathlineto{\pgfqpoint{3.472444in}{2.318717in}}%
\pgfpathlineto{\pgfqpoint{3.333649in}{2.318717in}}%
\pgfpathlineto{\pgfqpoint{3.225698in}{2.318717in}}%
\pgfpathlineto{\pgfqpoint{3.148590in}{2.318717in}}%
\pgfpathlineto{\pgfqpoint{3.102325in}{2.318717in}}%
\pgfpathlineto{\pgfqpoint{3.102325in}{2.318717in}}%
\pgfpathclose%
\pgfusepath{stroke,fill}%
}%
\begin{pgfscope}%
\pgfsys@transformshift{0.000000in}{0.000000in}%
\pgfsys@useobject{currentmarker}{}%
\end{pgfscope}%
\end{pgfscope}%
\begin{pgfscope}%
\pgfpathrectangle{\pgfqpoint{3.025989in}{0.451389in}}{\pgfqpoint{1.679412in}{1.956249in}}%
\pgfusepath{clip}%
\pgfsetbuttcap%
\pgfsetroundjoin%
\definecolor{currentfill}{rgb}{0.870588,0.560784,0.019608}%
\pgfsetfillcolor{currentfill}%
\pgfsetfillopacity{0.200000}%
\pgfsetlinewidth{1.003750pt}%
\definecolor{currentstroke}{rgb}{0.870588,0.560784,0.019608}%
\pgfsetstrokecolor{currentstroke}%
\pgfsetstrokeopacity{0.200000}%
\pgfsetdash{}{0pt}%
\pgfsys@defobject{currentmarker}{\pgfqpoint{3.102325in}{2.303686in}}{\pgfqpoint{4.629064in}{2.318511in}}{%
\pgfpathmoveto{\pgfqpoint{3.102325in}{2.307512in}}%
\pgfpathlineto{\pgfqpoint{3.102325in}{2.303686in}}%
\pgfpathlineto{\pgfqpoint{3.148590in}{2.314293in}}%
\pgfpathlineto{\pgfqpoint{3.225698in}{2.315955in}}%
\pgfpathlineto{\pgfqpoint{3.333649in}{2.317054in}}%
\pgfpathlineto{\pgfqpoint{3.472444in}{2.317328in}}%
\pgfpathlineto{\pgfqpoint{3.642081in}{2.317826in}}%
\pgfpathlineto{\pgfqpoint{3.842562in}{2.318025in}}%
\pgfpathlineto{\pgfqpoint{4.073886in}{2.318299in}}%
\pgfpathlineto{\pgfqpoint{4.336053in}{2.318393in}}%
\pgfpathlineto{\pgfqpoint{4.629064in}{2.318508in}}%
\pgfpathlineto{\pgfqpoint{4.629064in}{2.318511in}}%
\pgfpathlineto{\pgfqpoint{4.629064in}{2.318511in}}%
\pgfpathlineto{\pgfqpoint{4.336053in}{2.318401in}}%
\pgfpathlineto{\pgfqpoint{4.073886in}{2.318350in}}%
\pgfpathlineto{\pgfqpoint{3.842562in}{2.318041in}}%
\pgfpathlineto{\pgfqpoint{3.642081in}{2.317869in}}%
\pgfpathlineto{\pgfqpoint{3.472444in}{2.317550in}}%
\pgfpathlineto{\pgfqpoint{3.333649in}{2.317174in}}%
\pgfpathlineto{\pgfqpoint{3.225698in}{2.316228in}}%
\pgfpathlineto{\pgfqpoint{3.148590in}{2.314655in}}%
\pgfpathlineto{\pgfqpoint{3.102325in}{2.307512in}}%
\pgfpathlineto{\pgfqpoint{3.102325in}{2.307512in}}%
\pgfpathclose%
\pgfusepath{stroke,fill}%
}%
\begin{pgfscope}%
\pgfsys@transformshift{0.000000in}{0.000000in}%
\pgfsys@useobject{currentmarker}{}%
\end{pgfscope}%
\end{pgfscope}%
\begin{pgfscope}%
\pgfpathrectangle{\pgfqpoint{3.025989in}{0.451389in}}{\pgfqpoint{1.679412in}{1.956249in}}%
\pgfusepath{clip}%
\pgfsetbuttcap%
\pgfsetroundjoin%
\definecolor{currentfill}{rgb}{0.007843,0.619608,0.450980}%
\pgfsetfillcolor{currentfill}%
\pgfsetfillopacity{0.200000}%
\pgfsetlinewidth{1.003750pt}%
\definecolor{currentstroke}{rgb}{0.007843,0.619608,0.450980}%
\pgfsetstrokecolor{currentstroke}%
\pgfsetstrokeopacity{0.200000}%
\pgfsetdash{}{0pt}%
\pgfsys@defobject{currentmarker}{\pgfqpoint{3.102325in}{1.515512in}}{\pgfqpoint{4.629064in}{1.959118in}}{%
\pgfpathmoveto{\pgfqpoint{3.102325in}{1.959118in}}%
\pgfpathlineto{\pgfqpoint{3.102325in}{1.946386in}}%
\pgfpathlineto{\pgfqpoint{3.148590in}{1.856619in}}%
\pgfpathlineto{\pgfqpoint{3.225698in}{1.803768in}}%
\pgfpathlineto{\pgfqpoint{3.333649in}{1.753413in}}%
\pgfpathlineto{\pgfqpoint{3.472444in}{1.712444in}}%
\pgfpathlineto{\pgfqpoint{3.642081in}{1.683295in}}%
\pgfpathlineto{\pgfqpoint{3.842562in}{1.661190in}}%
\pgfpathlineto{\pgfqpoint{4.073886in}{1.585588in}}%
\pgfpathlineto{\pgfqpoint{4.336053in}{1.566052in}}%
\pgfpathlineto{\pgfqpoint{4.629064in}{1.515512in}}%
\pgfpathlineto{\pgfqpoint{4.629064in}{1.518294in}}%
\pgfpathlineto{\pgfqpoint{4.629064in}{1.518294in}}%
\pgfpathlineto{\pgfqpoint{4.336053in}{1.568571in}}%
\pgfpathlineto{\pgfqpoint{4.073886in}{1.604237in}}%
\pgfpathlineto{\pgfqpoint{3.842562in}{1.665351in}}%
\pgfpathlineto{\pgfqpoint{3.642081in}{1.695606in}}%
\pgfpathlineto{\pgfqpoint{3.472444in}{1.746840in}}%
\pgfpathlineto{\pgfqpoint{3.333649in}{1.758294in}}%
\pgfpathlineto{\pgfqpoint{3.225698in}{1.811789in}}%
\pgfpathlineto{\pgfqpoint{3.148590in}{1.861566in}}%
\pgfpathlineto{\pgfqpoint{3.102325in}{1.959118in}}%
\pgfpathlineto{\pgfqpoint{3.102325in}{1.959118in}}%
\pgfpathclose%
\pgfusepath{stroke,fill}%
}%
\begin{pgfscope}%
\pgfsys@transformshift{0.000000in}{0.000000in}%
\pgfsys@useobject{currentmarker}{}%
\end{pgfscope}%
\end{pgfscope}%
\begin{pgfscope}%
\pgfpathrectangle{\pgfqpoint{3.025989in}{0.451389in}}{\pgfqpoint{1.679412in}{1.956249in}}%
\pgfusepath{clip}%
\pgfsetbuttcap%
\pgfsetroundjoin%
\definecolor{currentfill}{rgb}{0.835294,0.368627,0.000000}%
\pgfsetfillcolor{currentfill}%
\pgfsetfillopacity{0.200000}%
\pgfsetlinewidth{1.003750pt}%
\definecolor{currentstroke}{rgb}{0.835294,0.368627,0.000000}%
\pgfsetstrokecolor{currentstroke}%
\pgfsetstrokeopacity{0.200000}%
\pgfsetdash{}{0pt}%
\pgfsys@defobject{currentmarker}{\pgfqpoint{3.102325in}{1.352512in}}{\pgfqpoint{4.629064in}{1.920687in}}{%
\pgfpathmoveto{\pgfqpoint{3.102325in}{1.920687in}}%
\pgfpathlineto{\pgfqpoint{3.102325in}{1.912860in}}%
\pgfpathlineto{\pgfqpoint{3.148590in}{1.757553in}}%
\pgfpathlineto{\pgfqpoint{3.225698in}{1.674399in}}%
\pgfpathlineto{\pgfqpoint{3.333649in}{1.612134in}}%
\pgfpathlineto{\pgfqpoint{3.472444in}{1.571175in}}%
\pgfpathlineto{\pgfqpoint{3.642081in}{1.513588in}}%
\pgfpathlineto{\pgfqpoint{3.842562in}{1.481938in}}%
\pgfpathlineto{\pgfqpoint{4.073886in}{1.424500in}}%
\pgfpathlineto{\pgfqpoint{4.336053in}{1.408545in}}%
\pgfpathlineto{\pgfqpoint{4.629064in}{1.352512in}}%
\pgfpathlineto{\pgfqpoint{4.629064in}{1.357532in}}%
\pgfpathlineto{\pgfqpoint{4.629064in}{1.357532in}}%
\pgfpathlineto{\pgfqpoint{4.336053in}{1.417790in}}%
\pgfpathlineto{\pgfqpoint{4.073886in}{1.432689in}}%
\pgfpathlineto{\pgfqpoint{3.842562in}{1.491485in}}%
\pgfpathlineto{\pgfqpoint{3.642081in}{1.536950in}}%
\pgfpathlineto{\pgfqpoint{3.472444in}{1.602529in}}%
\pgfpathlineto{\pgfqpoint{3.333649in}{1.640891in}}%
\pgfpathlineto{\pgfqpoint{3.225698in}{1.722596in}}%
\pgfpathlineto{\pgfqpoint{3.148590in}{1.763869in}}%
\pgfpathlineto{\pgfqpoint{3.102325in}{1.920687in}}%
\pgfpathlineto{\pgfqpoint{3.102325in}{1.920687in}}%
\pgfpathclose%
\pgfusepath{stroke,fill}%
}%
\begin{pgfscope}%
\pgfsys@transformshift{0.000000in}{0.000000in}%
\pgfsys@useobject{currentmarker}{}%
\end{pgfscope}%
\end{pgfscope}%
\begin{pgfscope}%
\pgfpathrectangle{\pgfqpoint{3.025989in}{0.451389in}}{\pgfqpoint{1.679412in}{1.956249in}}%
\pgfusepath{clip}%
\pgfsetbuttcap%
\pgfsetroundjoin%
\definecolor{currentfill}{rgb}{0.800000,0.470588,0.737255}%
\pgfsetfillcolor{currentfill}%
\pgfsetfillopacity{0.200000}%
\pgfsetlinewidth{1.003750pt}%
\definecolor{currentstroke}{rgb}{0.800000,0.470588,0.737255}%
\pgfsetstrokecolor{currentstroke}%
\pgfsetstrokeopacity{0.200000}%
\pgfsetdash{}{0pt}%
\pgfsys@defobject{currentmarker}{\pgfqpoint{3.102325in}{0.540309in}}{\pgfqpoint{4.629064in}{1.373093in}}{%
\pgfpathmoveto{\pgfqpoint{3.102325in}{1.373093in}}%
\pgfpathlineto{\pgfqpoint{3.102325in}{1.343553in}}%
\pgfpathlineto{\pgfqpoint{3.148590in}{1.035873in}}%
\pgfpathlineto{\pgfqpoint{3.225698in}{0.885706in}}%
\pgfpathlineto{\pgfqpoint{3.333649in}{0.788123in}}%
\pgfpathlineto{\pgfqpoint{3.472444in}{0.760832in}}%
\pgfpathlineto{\pgfqpoint{3.642081in}{0.717236in}}%
\pgfpathlineto{\pgfqpoint{3.842562in}{0.705271in}}%
\pgfpathlineto{\pgfqpoint{4.073886in}{0.596194in}}%
\pgfpathlineto{\pgfqpoint{4.336053in}{0.596725in}}%
\pgfpathlineto{\pgfqpoint{4.629064in}{0.540309in}}%
\pgfpathlineto{\pgfqpoint{4.629064in}{0.654852in}}%
\pgfpathlineto{\pgfqpoint{4.629064in}{0.654852in}}%
\pgfpathlineto{\pgfqpoint{4.336053in}{0.798322in}}%
\pgfpathlineto{\pgfqpoint{4.073886in}{0.832307in}}%
\pgfpathlineto{\pgfqpoint{3.842562in}{0.823614in}}%
\pgfpathlineto{\pgfqpoint{3.642081in}{1.167707in}}%
\pgfpathlineto{\pgfqpoint{3.472444in}{0.956909in}}%
\pgfpathlineto{\pgfqpoint{3.333649in}{1.042914in}}%
\pgfpathlineto{\pgfqpoint{3.225698in}{1.183913in}}%
\pgfpathlineto{\pgfqpoint{3.148590in}{1.078996in}}%
\pgfpathlineto{\pgfqpoint{3.102325in}{1.373093in}}%
\pgfpathlineto{\pgfqpoint{3.102325in}{1.373093in}}%
\pgfpathclose%
\pgfusepath{stroke,fill}%
}%
\begin{pgfscope}%
\pgfsys@transformshift{0.000000in}{0.000000in}%
\pgfsys@useobject{currentmarker}{}%
\end{pgfscope}%
\end{pgfscope}%
\begin{pgfscope}%
\pgfpathrectangle{\pgfqpoint{3.025989in}{0.451389in}}{\pgfqpoint{1.679412in}{1.956249in}}%
\pgfusepath{clip}%
\pgfsetbuttcap%
\pgfsetroundjoin%
\definecolor{currentfill}{rgb}{0.792157,0.568627,0.380392}%
\pgfsetfillcolor{currentfill}%
\pgfsetfillopacity{0.200000}%
\pgfsetlinewidth{1.003750pt}%
\definecolor{currentstroke}{rgb}{0.792157,0.568627,0.380392}%
\pgfsetstrokecolor{currentstroke}%
\pgfsetstrokeopacity{0.200000}%
\pgfsetdash{}{0pt}%
\pgfsys@defobject{currentmarker}{\pgfqpoint{3.102325in}{0.557244in}}{\pgfqpoint{4.629064in}{1.309277in}}{%
\pgfpathmoveto{\pgfqpoint{3.102325in}{1.309277in}}%
\pgfpathlineto{\pgfqpoint{3.102325in}{1.246481in}}%
\pgfpathlineto{\pgfqpoint{3.148590in}{1.000825in}}%
\pgfpathlineto{\pgfqpoint{3.225698in}{0.868419in}}%
\pgfpathlineto{\pgfqpoint{3.333649in}{0.838821in}}%
\pgfpathlineto{\pgfqpoint{3.472444in}{0.807718in}}%
\pgfpathlineto{\pgfqpoint{3.642081in}{0.734897in}}%
\pgfpathlineto{\pgfqpoint{3.842562in}{0.718739in}}%
\pgfpathlineto{\pgfqpoint{4.073886in}{0.629276in}}%
\pgfpathlineto{\pgfqpoint{4.336053in}{0.609150in}}%
\pgfpathlineto{\pgfqpoint{4.629064in}{0.557244in}}%
\pgfpathlineto{\pgfqpoint{4.629064in}{0.730258in}}%
\pgfpathlineto{\pgfqpoint{4.629064in}{0.730258in}}%
\pgfpathlineto{\pgfqpoint{4.336053in}{0.647289in}}%
\pgfpathlineto{\pgfqpoint{4.073886in}{0.675893in}}%
\pgfpathlineto{\pgfqpoint{3.842562in}{0.738734in}}%
\pgfpathlineto{\pgfqpoint{3.642081in}{0.797181in}}%
\pgfpathlineto{\pgfqpoint{3.472444in}{0.875247in}}%
\pgfpathlineto{\pgfqpoint{3.333649in}{0.964127in}}%
\pgfpathlineto{\pgfqpoint{3.225698in}{0.932084in}}%
\pgfpathlineto{\pgfqpoint{3.148590in}{1.086924in}}%
\pgfpathlineto{\pgfqpoint{3.102325in}{1.309277in}}%
\pgfpathlineto{\pgfqpoint{3.102325in}{1.309277in}}%
\pgfpathclose%
\pgfusepath{stroke,fill}%
}%
\begin{pgfscope}%
\pgfsys@transformshift{0.000000in}{0.000000in}%
\pgfsys@useobject{currentmarker}{}%
\end{pgfscope}%
\end{pgfscope}%
\begin{pgfscope}%
\pgfsetrectcap%
\pgfsetmiterjoin%
\pgfsetlinewidth{1.254687pt}%
\definecolor{currentstroke}{rgb}{0.800000,0.800000,0.800000}%
\pgfsetstrokecolor{currentstroke}%
\pgfsetdash{}{0pt}%
\pgfpathmoveto{\pgfqpoint{3.025989in}{0.451389in}}%
\pgfpathlineto{\pgfqpoint{3.025989in}{2.407638in}}%
\pgfusepath{stroke}%
\end{pgfscope}%
\begin{pgfscope}%
\pgfsetrectcap%
\pgfsetmiterjoin%
\pgfsetlinewidth{1.254687pt}%
\definecolor{currentstroke}{rgb}{0.800000,0.800000,0.800000}%
\pgfsetstrokecolor{currentstroke}%
\pgfsetdash{}{0pt}%
\pgfpathmoveto{\pgfqpoint{4.705401in}{0.451389in}}%
\pgfpathlineto{\pgfqpoint{4.705401in}{2.407638in}}%
\pgfusepath{stroke}%
\end{pgfscope}%
\begin{pgfscope}%
\pgfsetrectcap%
\pgfsetmiterjoin%
\pgfsetlinewidth{1.254687pt}%
\definecolor{currentstroke}{rgb}{0.800000,0.800000,0.800000}%
\pgfsetstrokecolor{currentstroke}%
\pgfsetdash{}{0pt}%
\pgfpathmoveto{\pgfqpoint{3.025989in}{0.451389in}}%
\pgfpathlineto{\pgfqpoint{4.705401in}{0.451389in}}%
\pgfusepath{stroke}%
\end{pgfscope}%
\begin{pgfscope}%
\pgfsetrectcap%
\pgfsetmiterjoin%
\pgfsetlinewidth{1.254687pt}%
\definecolor{currentstroke}{rgb}{0.800000,0.800000,0.800000}%
\pgfsetstrokecolor{currentstroke}%
\pgfsetdash{}{0pt}%
\pgfpathmoveto{\pgfqpoint{3.025989in}{2.407638in}}%
\pgfpathlineto{\pgfqpoint{4.705401in}{2.407638in}}%
\pgfusepath{stroke}%
\end{pgfscope}%
\begin{pgfscope}%
\pgfsetbuttcap%
\pgfsetmiterjoin%
\definecolor{currentfill}{rgb}{1.000000,1.000000,1.000000}%
\pgfsetfillcolor{currentfill}%
\pgfsetfillopacity{0.800000}%
\pgfsetlinewidth{1.003750pt}%
\definecolor{currentstroke}{rgb}{0.800000,0.800000,0.800000}%
\pgfsetstrokecolor{currentstroke}%
\pgfsetstrokeopacity{0.800000}%
\pgfsetdash{}{0pt}%
\pgfpathmoveto{\pgfqpoint{4.834886in}{0.538523in}}%
\pgfpathlineto{\pgfqpoint{6.029442in}{0.538523in}}%
\pgfpathquadraticcurveto{\pgfqpoint{6.054442in}{0.538523in}}{\pgfqpoint{6.054442in}{0.563523in}}%
\pgfpathlineto{\pgfqpoint{6.054442in}{2.295504in}}%
\pgfpathquadraticcurveto{\pgfqpoint{6.054442in}{2.320504in}}{\pgfqpoint{6.029442in}{2.320504in}}%
\pgfpathlineto{\pgfqpoint{4.834886in}{2.320504in}}%
\pgfpathquadraticcurveto{\pgfqpoint{4.809886in}{2.320504in}}{\pgfqpoint{4.809886in}{2.295504in}}%
\pgfpathlineto{\pgfqpoint{4.809886in}{0.563523in}}%
\pgfpathquadraticcurveto{\pgfqpoint{4.809886in}{0.538523in}}{\pgfqpoint{4.834886in}{0.538523in}}%
\pgfpathlineto{\pgfqpoint{4.834886in}{0.538523in}}%
\pgfpathclose%
\pgfusepath{stroke,fill}%
\end{pgfscope}%
\begin{pgfscope}%
\definecolor{textcolor}{rgb}{0.150000,0.150000,0.150000}%
\pgfsetstrokecolor{textcolor}%
\pgfsetfillcolor{textcolor}%
\pgftext[x=5.228764in,y=2.175533in,left,base]{\color{textcolor}\sffamily\fontsize{9.000000}{10.800000}\selectfont Legend}%
\end{pgfscope}%
\begin{pgfscope}%
\pgfsetroundcap%
\pgfsetroundjoin%
\pgfsetlinewidth{1.505625pt}%
\definecolor{currentstroke}{rgb}{0.003922,0.450980,0.698039}%
\pgfsetstrokecolor{currentstroke}%
\pgfsetdash{}{0pt}%
\pgfpathmoveto{\pgfqpoint{4.859886in}{2.031783in}}%
\pgfpathlineto{\pgfqpoint{4.984886in}{2.031783in}}%
\pgfpathlineto{\pgfqpoint{5.109886in}{2.031783in}}%
\pgfusepath{stroke}%
\end{pgfscope}%
\begin{pgfscope}%
\definecolor{textcolor}{rgb}{0.150000,0.150000,0.150000}%
\pgfsetstrokecolor{textcolor}%
\pgfsetfillcolor{textcolor}%
\pgftext[x=5.209886in,y=1.988033in,left,base]{\color{textcolor}\sffamily\fontsize{9.000000}{10.800000}\selectfont Total time}%
\end{pgfscope}%
\begin{pgfscope}%
\pgfsetroundcap%
\pgfsetroundjoin%
\pgfsetlinewidth{1.505625pt}%
\definecolor{currentstroke}{rgb}{0.870588,0.560784,0.019608}%
\pgfsetstrokecolor{currentstroke}%
\pgfsetdash{}{0pt}%
\pgfpathmoveto{\pgfqpoint{4.859886in}{1.844283in}}%
\pgfpathlineto{\pgfqpoint{4.984886in}{1.844283in}}%
\pgfpathlineto{\pgfqpoint{5.109886in}{1.844283in}}%
\pgfusepath{stroke}%
\end{pgfscope}%
\begin{pgfscope}%
\definecolor{textcolor}{rgb}{0.150000,0.150000,0.150000}%
\pgfsetstrokecolor{textcolor}%
\pgfsetfillcolor{textcolor}%
\pgftext[x=5.209886in,y=1.800533in,left,base]{\color{textcolor}\sffamily\fontsize{9.000000}{10.800000}\selectfont kNN search}%
\end{pgfscope}%
\begin{pgfscope}%
\pgfsetroundcap%
\pgfsetroundjoin%
\pgfsetlinewidth{1.505625pt}%
\definecolor{currentstroke}{rgb}{0.007843,0.619608,0.450980}%
\pgfsetstrokecolor{currentstroke}%
\pgfsetdash{}{0pt}%
\pgfpathmoveto{\pgfqpoint{4.859886in}{1.656783in}}%
\pgfpathlineto{\pgfqpoint{4.984886in}{1.656783in}}%
\pgfpathlineto{\pgfqpoint{5.109886in}{1.656783in}}%
\pgfusepath{stroke}%
\end{pgfscope}%
\begin{pgfscope}%
\definecolor{textcolor}{rgb}{0.150000,0.150000,0.150000}%
\pgfsetstrokecolor{textcolor}%
\pgfsetfillcolor{textcolor}%
\pgftext[x=5.209886in,y=1.613033in,left,base]{\color{textcolor}\sffamily\fontsize{9.000000}{10.800000}\selectfont Create graph}%
\end{pgfscope}%
\begin{pgfscope}%
\pgfsetroundcap%
\pgfsetroundjoin%
\pgfsetlinewidth{1.505625pt}%
\definecolor{currentstroke}{rgb}{0.835294,0.368627,0.000000}%
\pgfsetstrokecolor{currentstroke}%
\pgfsetdash{}{0pt}%
\pgfpathmoveto{\pgfqpoint{4.859886in}{1.382272in}}%
\pgfpathlineto{\pgfqpoint{4.984886in}{1.382272in}}%
\pgfpathlineto{\pgfqpoint{5.109886in}{1.382272in}}%
\pgfusepath{stroke}%
\end{pgfscope}%
\begin{pgfscope}%
\definecolor{textcolor}{rgb}{0.150000,0.150000,0.150000}%
\pgfsetstrokecolor{textcolor}%
\pgfsetfillcolor{textcolor}%
\pgftext[x=5.209886in, y=1.425534in, left, base]{\color{textcolor}\sffamily\fontsize{9.000000}{10.800000}\selectfont Get \& prepare}%
\end{pgfscope}%
\begin{pgfscope}%
\definecolor{textcolor}{rgb}{0.150000,0.150000,0.150000}%
\pgfsetstrokecolor{textcolor}%
\pgfsetfillcolor{textcolor}%
\pgftext[x=5.209886in, y=1.281540in, left, base]{\color{textcolor}\sffamily\fontsize{9.000000}{10.800000}\selectfont obstacles}%
\end{pgfscope}%
\begin{pgfscope}%
\pgfsetroundcap%
\pgfsetroundjoin%
\pgfsetlinewidth{1.505625pt}%
\definecolor{currentstroke}{rgb}{0.800000,0.470588,0.737255}%
\pgfsetstrokecolor{currentstroke}%
\pgfsetdash{}{0pt}%
\pgfpathmoveto{\pgfqpoint{4.859886in}{1.050778in}}%
\pgfpathlineto{\pgfqpoint{4.984886in}{1.050778in}}%
\pgfpathlineto{\pgfqpoint{5.109886in}{1.050778in}}%
\pgfusepath{stroke}%
\end{pgfscope}%
\begin{pgfscope}%
\definecolor{textcolor}{rgb}{0.150000,0.150000,0.150000}%
\pgfsetstrokecolor{textcolor}%
\pgfsetfillcolor{textcolor}%
\pgftext[x=5.209886in, y=1.094040in, left, base]{\color{textcolor}\sffamily\fontsize{9.000000}{10.800000}\selectfont Merge road}%
\end{pgfscope}%
\begin{pgfscope}%
\definecolor{textcolor}{rgb}{0.150000,0.150000,0.150000}%
\pgfsetstrokecolor{textcolor}%
\pgfsetfillcolor{textcolor}%
\pgftext[x=5.209886in, y=0.950046in, left, base]{\color{textcolor}\sffamily\fontsize{9.000000}{10.800000}\selectfont edges}%
\end{pgfscope}%
\begin{pgfscope}%
\pgfsetroundcap%
\pgfsetroundjoin%
\pgfsetlinewidth{1.505625pt}%
\definecolor{currentstroke}{rgb}{0.792157,0.568627,0.380392}%
\pgfsetstrokecolor{currentstroke}%
\pgfsetdash{}{0pt}%
\pgfpathmoveto{\pgfqpoint{4.859886in}{0.719284in}}%
\pgfpathlineto{\pgfqpoint{4.984886in}{0.719284in}}%
\pgfpathlineto{\pgfqpoint{5.109886in}{0.719284in}}%
\pgfusepath{stroke}%
\end{pgfscope}%
\begin{pgfscope}%
\definecolor{textcolor}{rgb}{0.150000,0.150000,0.150000}%
\pgfsetstrokecolor{textcolor}%
\pgfsetfillcolor{textcolor}%
\pgftext[x=5.209886in, y=0.762546in, left, base]{\color{textcolor}\sffamily\fontsize{9.000000}{10.800000}\selectfont Add POI}%
\end{pgfscope}%
\begin{pgfscope}%
\definecolor{textcolor}{rgb}{0.150000,0.150000,0.150000}%
\pgfsetstrokecolor{textcolor}%
\pgfsetfillcolor{textcolor}%
\pgftext[x=5.209886in, y=0.618552in, left, base]{\color{textcolor}\sffamily\fontsize{9.000000}{10.800000}\selectfont attributes}%
\end{pgfscope}%
\begin{pgfscope}%
\pgfsetroundcap%
\pgfsetroundjoin%
\pgfsetlinewidth{1.003750pt}%
\definecolor{currentstroke}{rgb}{0.003922,0.450980,0.698039}%
\pgfsetstrokecolor{currentstroke}%
\pgfsetdash{}{0pt}%
\pgfpathmoveto{\pgfqpoint{3.102325in}{2.318717in}}%
\pgfpathlineto{\pgfqpoint{3.148590in}{2.318717in}}%
\pgfpathlineto{\pgfqpoint{3.225698in}{2.318717in}}%
\pgfpathlineto{\pgfqpoint{3.333649in}{2.318717in}}%
\pgfpathlineto{\pgfqpoint{3.472444in}{2.318717in}}%
\pgfpathlineto{\pgfqpoint{3.642081in}{2.318717in}}%
\pgfpathlineto{\pgfqpoint{3.842562in}{2.318717in}}%
\pgfpathlineto{\pgfqpoint{4.073886in}{2.318717in}}%
\pgfpathlineto{\pgfqpoint{4.336053in}{2.318717in}}%
\pgfpathlineto{\pgfqpoint{4.629064in}{2.318717in}}%
\pgfusepath{stroke}%
\end{pgfscope}%
\begin{pgfscope}%
\pgfsetbuttcap%
\pgfsetroundjoin%
\definecolor{currentfill}{rgb}{0.003922,0.450980,0.698039}%
\pgfsetfillcolor{currentfill}%
\pgfsetlinewidth{0.752812pt}%
\definecolor{currentstroke}{rgb}{1.000000,1.000000,1.000000}%
\pgfsetstrokecolor{currentstroke}%
\pgfsetdash{}{0pt}%
\pgfsys@defobject{currentmarker}{\pgfqpoint{-0.034722in}{-0.034722in}}{\pgfqpoint{0.034722in}{0.034722in}}{%
\pgfpathmoveto{\pgfqpoint{0.000000in}{-0.034722in}}%
\pgfpathcurveto{\pgfqpoint{0.009208in}{-0.034722in}}{\pgfqpoint{0.018041in}{-0.031064in}}{\pgfqpoint{0.024552in}{-0.024552in}}%
\pgfpathcurveto{\pgfqpoint{0.031064in}{-0.018041in}}{\pgfqpoint{0.034722in}{-0.009208in}}{\pgfqpoint{0.034722in}{0.000000in}}%
\pgfpathcurveto{\pgfqpoint{0.034722in}{0.009208in}}{\pgfqpoint{0.031064in}{0.018041in}}{\pgfqpoint{0.024552in}{0.024552in}}%
\pgfpathcurveto{\pgfqpoint{0.018041in}{0.031064in}}{\pgfqpoint{0.009208in}{0.034722in}}{\pgfqpoint{0.000000in}{0.034722in}}%
\pgfpathcurveto{\pgfqpoint{-0.009208in}{0.034722in}}{\pgfqpoint{-0.018041in}{0.031064in}}{\pgfqpoint{-0.024552in}{0.024552in}}%
\pgfpathcurveto{\pgfqpoint{-0.031064in}{0.018041in}}{\pgfqpoint{-0.034722in}{0.009208in}}{\pgfqpoint{-0.034722in}{0.000000in}}%
\pgfpathcurveto{\pgfqpoint{-0.034722in}{-0.009208in}}{\pgfqpoint{-0.031064in}{-0.018041in}}{\pgfqpoint{-0.024552in}{-0.024552in}}%
\pgfpathcurveto{\pgfqpoint{-0.018041in}{-0.031064in}}{\pgfqpoint{-0.009208in}{-0.034722in}}{\pgfqpoint{0.000000in}{-0.034722in}}%
\pgfpathlineto{\pgfqpoint{0.000000in}{-0.034722in}}%
\pgfpathclose%
\pgfusepath{stroke,fill}%
}%
\begin{pgfscope}%
\pgfsys@transformshift{3.102325in}{2.318717in}%
\pgfsys@useobject{currentmarker}{}%
\end{pgfscope}%
\begin{pgfscope}%
\pgfsys@transformshift{3.148590in}{2.318717in}%
\pgfsys@useobject{currentmarker}{}%
\end{pgfscope}%
\begin{pgfscope}%
\pgfsys@transformshift{3.225698in}{2.318717in}%
\pgfsys@useobject{currentmarker}{}%
\end{pgfscope}%
\begin{pgfscope}%
\pgfsys@transformshift{3.333649in}{2.318717in}%
\pgfsys@useobject{currentmarker}{}%
\end{pgfscope}%
\begin{pgfscope}%
\pgfsys@transformshift{3.472444in}{2.318717in}%
\pgfsys@useobject{currentmarker}{}%
\end{pgfscope}%
\begin{pgfscope}%
\pgfsys@transformshift{3.642081in}{2.318717in}%
\pgfsys@useobject{currentmarker}{}%
\end{pgfscope}%
\begin{pgfscope}%
\pgfsys@transformshift{3.842562in}{2.318717in}%
\pgfsys@useobject{currentmarker}{}%
\end{pgfscope}%
\begin{pgfscope}%
\pgfsys@transformshift{4.073886in}{2.318717in}%
\pgfsys@useobject{currentmarker}{}%
\end{pgfscope}%
\begin{pgfscope}%
\pgfsys@transformshift{4.336053in}{2.318717in}%
\pgfsys@useobject{currentmarker}{}%
\end{pgfscope}%
\begin{pgfscope}%
\pgfsys@transformshift{4.629064in}{2.318717in}%
\pgfsys@useobject{currentmarker}{}%
\end{pgfscope}%
\end{pgfscope}%
\begin{pgfscope}%
\pgfsetroundcap%
\pgfsetroundjoin%
\pgfsetlinewidth{1.003750pt}%
\definecolor{currentstroke}{rgb}{0.870588,0.560784,0.019608}%
\pgfsetstrokecolor{currentstroke}%
\pgfsetdash{}{0pt}%
\pgfpathmoveto{\pgfqpoint{3.102325in}{2.306430in}}%
\pgfpathlineto{\pgfqpoint{3.148590in}{2.314517in}}%
\pgfpathlineto{\pgfqpoint{3.225698in}{2.316097in}}%
\pgfpathlineto{\pgfqpoint{3.333649in}{2.317111in}}%
\pgfpathlineto{\pgfqpoint{3.472444in}{2.317425in}}%
\pgfpathlineto{\pgfqpoint{3.642081in}{2.317848in}}%
\pgfpathlineto{\pgfqpoint{3.842562in}{2.318035in}}%
\pgfpathlineto{\pgfqpoint{4.073886in}{2.318319in}}%
\pgfpathlineto{\pgfqpoint{4.336053in}{2.318398in}}%
\pgfpathlineto{\pgfqpoint{4.629064in}{2.318509in}}%
\pgfusepath{stroke}%
\end{pgfscope}%
\begin{pgfscope}%
\pgfsetbuttcap%
\pgfsetroundjoin%
\definecolor{currentfill}{rgb}{0.870588,0.560784,0.019608}%
\pgfsetfillcolor{currentfill}%
\pgfsetlinewidth{0.752812pt}%
\definecolor{currentstroke}{rgb}{1.000000,1.000000,1.000000}%
\pgfsetstrokecolor{currentstroke}%
\pgfsetdash{}{0pt}%
\pgfsys@defobject{currentmarker}{\pgfqpoint{-0.034722in}{-0.034722in}}{\pgfqpoint{0.034722in}{0.034722in}}{%
\pgfpathmoveto{\pgfqpoint{0.000000in}{-0.034722in}}%
\pgfpathcurveto{\pgfqpoint{0.009208in}{-0.034722in}}{\pgfqpoint{0.018041in}{-0.031064in}}{\pgfqpoint{0.024552in}{-0.024552in}}%
\pgfpathcurveto{\pgfqpoint{0.031064in}{-0.018041in}}{\pgfqpoint{0.034722in}{-0.009208in}}{\pgfqpoint{0.034722in}{0.000000in}}%
\pgfpathcurveto{\pgfqpoint{0.034722in}{0.009208in}}{\pgfqpoint{0.031064in}{0.018041in}}{\pgfqpoint{0.024552in}{0.024552in}}%
\pgfpathcurveto{\pgfqpoint{0.018041in}{0.031064in}}{\pgfqpoint{0.009208in}{0.034722in}}{\pgfqpoint{0.000000in}{0.034722in}}%
\pgfpathcurveto{\pgfqpoint{-0.009208in}{0.034722in}}{\pgfqpoint{-0.018041in}{0.031064in}}{\pgfqpoint{-0.024552in}{0.024552in}}%
\pgfpathcurveto{\pgfqpoint{-0.031064in}{0.018041in}}{\pgfqpoint{-0.034722in}{0.009208in}}{\pgfqpoint{-0.034722in}{0.000000in}}%
\pgfpathcurveto{\pgfqpoint{-0.034722in}{-0.009208in}}{\pgfqpoint{-0.031064in}{-0.018041in}}{\pgfqpoint{-0.024552in}{-0.024552in}}%
\pgfpathcurveto{\pgfqpoint{-0.018041in}{-0.031064in}}{\pgfqpoint{-0.009208in}{-0.034722in}}{\pgfqpoint{0.000000in}{-0.034722in}}%
\pgfpathlineto{\pgfqpoint{0.000000in}{-0.034722in}}%
\pgfpathclose%
\pgfusepath{stroke,fill}%
}%
\begin{pgfscope}%
\pgfsys@transformshift{3.102325in}{2.306430in}%
\pgfsys@useobject{currentmarker}{}%
\end{pgfscope}%
\begin{pgfscope}%
\pgfsys@transformshift{3.148590in}{2.314517in}%
\pgfsys@useobject{currentmarker}{}%
\end{pgfscope}%
\begin{pgfscope}%
\pgfsys@transformshift{3.225698in}{2.316097in}%
\pgfsys@useobject{currentmarker}{}%
\end{pgfscope}%
\begin{pgfscope}%
\pgfsys@transformshift{3.333649in}{2.317111in}%
\pgfsys@useobject{currentmarker}{}%
\end{pgfscope}%
\begin{pgfscope}%
\pgfsys@transformshift{3.472444in}{2.317425in}%
\pgfsys@useobject{currentmarker}{}%
\end{pgfscope}%
\begin{pgfscope}%
\pgfsys@transformshift{3.642081in}{2.317848in}%
\pgfsys@useobject{currentmarker}{}%
\end{pgfscope}%
\begin{pgfscope}%
\pgfsys@transformshift{3.842562in}{2.318035in}%
\pgfsys@useobject{currentmarker}{}%
\end{pgfscope}%
\begin{pgfscope}%
\pgfsys@transformshift{4.073886in}{2.318319in}%
\pgfsys@useobject{currentmarker}{}%
\end{pgfscope}%
\begin{pgfscope}%
\pgfsys@transformshift{4.336053in}{2.318398in}%
\pgfsys@useobject{currentmarker}{}%
\end{pgfscope}%
\begin{pgfscope}%
\pgfsys@transformshift{4.629064in}{2.318509in}%
\pgfsys@useobject{currentmarker}{}%
\end{pgfscope}%
\end{pgfscope}%
\begin{pgfscope}%
\pgfsetroundcap%
\pgfsetroundjoin%
\pgfsetlinewidth{1.003750pt}%
\definecolor{currentstroke}{rgb}{0.007843,0.619608,0.450980}%
\pgfsetstrokecolor{currentstroke}%
\pgfsetdash{}{0pt}%
\pgfpathmoveto{\pgfqpoint{3.102325in}{1.952165in}}%
\pgfpathlineto{\pgfqpoint{3.148590in}{1.858107in}}%
\pgfpathlineto{\pgfqpoint{3.225698in}{1.807814in}}%
\pgfpathlineto{\pgfqpoint{3.333649in}{1.755451in}}%
\pgfpathlineto{\pgfqpoint{3.472444in}{1.734015in}}%
\pgfpathlineto{\pgfqpoint{3.642081in}{1.690040in}}%
\pgfpathlineto{\pgfqpoint{3.842562in}{1.663601in}}%
\pgfpathlineto{\pgfqpoint{4.073886in}{1.597040in}}%
\pgfpathlineto{\pgfqpoint{4.336053in}{1.567257in}}%
\pgfpathlineto{\pgfqpoint{4.629064in}{1.516918in}}%
\pgfusepath{stroke}%
\end{pgfscope}%
\begin{pgfscope}%
\pgfsetbuttcap%
\pgfsetroundjoin%
\definecolor{currentfill}{rgb}{0.007843,0.619608,0.450980}%
\pgfsetfillcolor{currentfill}%
\pgfsetlinewidth{0.752812pt}%
\definecolor{currentstroke}{rgb}{1.000000,1.000000,1.000000}%
\pgfsetstrokecolor{currentstroke}%
\pgfsetdash{}{0pt}%
\pgfsys@defobject{currentmarker}{\pgfqpoint{-0.034722in}{-0.034722in}}{\pgfqpoint{0.034722in}{0.034722in}}{%
\pgfpathmoveto{\pgfqpoint{0.000000in}{-0.034722in}}%
\pgfpathcurveto{\pgfqpoint{0.009208in}{-0.034722in}}{\pgfqpoint{0.018041in}{-0.031064in}}{\pgfqpoint{0.024552in}{-0.024552in}}%
\pgfpathcurveto{\pgfqpoint{0.031064in}{-0.018041in}}{\pgfqpoint{0.034722in}{-0.009208in}}{\pgfqpoint{0.034722in}{0.000000in}}%
\pgfpathcurveto{\pgfqpoint{0.034722in}{0.009208in}}{\pgfqpoint{0.031064in}{0.018041in}}{\pgfqpoint{0.024552in}{0.024552in}}%
\pgfpathcurveto{\pgfqpoint{0.018041in}{0.031064in}}{\pgfqpoint{0.009208in}{0.034722in}}{\pgfqpoint{0.000000in}{0.034722in}}%
\pgfpathcurveto{\pgfqpoint{-0.009208in}{0.034722in}}{\pgfqpoint{-0.018041in}{0.031064in}}{\pgfqpoint{-0.024552in}{0.024552in}}%
\pgfpathcurveto{\pgfqpoint{-0.031064in}{0.018041in}}{\pgfqpoint{-0.034722in}{0.009208in}}{\pgfqpoint{-0.034722in}{0.000000in}}%
\pgfpathcurveto{\pgfqpoint{-0.034722in}{-0.009208in}}{\pgfqpoint{-0.031064in}{-0.018041in}}{\pgfqpoint{-0.024552in}{-0.024552in}}%
\pgfpathcurveto{\pgfqpoint{-0.018041in}{-0.031064in}}{\pgfqpoint{-0.009208in}{-0.034722in}}{\pgfqpoint{0.000000in}{-0.034722in}}%
\pgfpathlineto{\pgfqpoint{0.000000in}{-0.034722in}}%
\pgfpathclose%
\pgfusepath{stroke,fill}%
}%
\begin{pgfscope}%
\pgfsys@transformshift{3.102325in}{1.952165in}%
\pgfsys@useobject{currentmarker}{}%
\end{pgfscope}%
\begin{pgfscope}%
\pgfsys@transformshift{3.148590in}{1.858107in}%
\pgfsys@useobject{currentmarker}{}%
\end{pgfscope}%
\begin{pgfscope}%
\pgfsys@transformshift{3.225698in}{1.807814in}%
\pgfsys@useobject{currentmarker}{}%
\end{pgfscope}%
\begin{pgfscope}%
\pgfsys@transformshift{3.333649in}{1.755451in}%
\pgfsys@useobject{currentmarker}{}%
\end{pgfscope}%
\begin{pgfscope}%
\pgfsys@transformshift{3.472444in}{1.734015in}%
\pgfsys@useobject{currentmarker}{}%
\end{pgfscope}%
\begin{pgfscope}%
\pgfsys@transformshift{3.642081in}{1.690040in}%
\pgfsys@useobject{currentmarker}{}%
\end{pgfscope}%
\begin{pgfscope}%
\pgfsys@transformshift{3.842562in}{1.663601in}%
\pgfsys@useobject{currentmarker}{}%
\end{pgfscope}%
\begin{pgfscope}%
\pgfsys@transformshift{4.073886in}{1.597040in}%
\pgfsys@useobject{currentmarker}{}%
\end{pgfscope}%
\begin{pgfscope}%
\pgfsys@transformshift{4.336053in}{1.567257in}%
\pgfsys@useobject{currentmarker}{}%
\end{pgfscope}%
\begin{pgfscope}%
\pgfsys@transformshift{4.629064in}{1.516918in}%
\pgfsys@useobject{currentmarker}{}%
\end{pgfscope}%
\end{pgfscope}%
\begin{pgfscope}%
\pgfsetroundcap%
\pgfsetroundjoin%
\pgfsetlinewidth{1.003750pt}%
\definecolor{currentstroke}{rgb}{0.835294,0.368627,0.000000}%
\pgfsetstrokecolor{currentstroke}%
\pgfsetdash{}{0pt}%
\pgfpathmoveto{\pgfqpoint{3.102325in}{1.918183in}}%
\pgfpathlineto{\pgfqpoint{3.148590in}{1.761212in}}%
\pgfpathlineto{\pgfqpoint{3.225698in}{1.691764in}}%
\pgfpathlineto{\pgfqpoint{3.333649in}{1.622491in}}%
\pgfpathlineto{\pgfqpoint{3.472444in}{1.584401in}}%
\pgfpathlineto{\pgfqpoint{3.642081in}{1.522434in}}%
\pgfpathlineto{\pgfqpoint{3.842562in}{1.485517in}}%
\pgfpathlineto{\pgfqpoint{4.073886in}{1.428112in}}%
\pgfpathlineto{\pgfqpoint{4.336053in}{1.412213in}}%
\pgfpathlineto{\pgfqpoint{4.629064in}{1.355152in}}%
\pgfusepath{stroke}%
\end{pgfscope}%
\begin{pgfscope}%
\pgfsetbuttcap%
\pgfsetroundjoin%
\definecolor{currentfill}{rgb}{0.835294,0.368627,0.000000}%
\pgfsetfillcolor{currentfill}%
\pgfsetlinewidth{0.752812pt}%
\definecolor{currentstroke}{rgb}{1.000000,1.000000,1.000000}%
\pgfsetstrokecolor{currentstroke}%
\pgfsetdash{}{0pt}%
\pgfsys@defobject{currentmarker}{\pgfqpoint{-0.034722in}{-0.034722in}}{\pgfqpoint{0.034722in}{0.034722in}}{%
\pgfpathmoveto{\pgfqpoint{0.000000in}{-0.034722in}}%
\pgfpathcurveto{\pgfqpoint{0.009208in}{-0.034722in}}{\pgfqpoint{0.018041in}{-0.031064in}}{\pgfqpoint{0.024552in}{-0.024552in}}%
\pgfpathcurveto{\pgfqpoint{0.031064in}{-0.018041in}}{\pgfqpoint{0.034722in}{-0.009208in}}{\pgfqpoint{0.034722in}{0.000000in}}%
\pgfpathcurveto{\pgfqpoint{0.034722in}{0.009208in}}{\pgfqpoint{0.031064in}{0.018041in}}{\pgfqpoint{0.024552in}{0.024552in}}%
\pgfpathcurveto{\pgfqpoint{0.018041in}{0.031064in}}{\pgfqpoint{0.009208in}{0.034722in}}{\pgfqpoint{0.000000in}{0.034722in}}%
\pgfpathcurveto{\pgfqpoint{-0.009208in}{0.034722in}}{\pgfqpoint{-0.018041in}{0.031064in}}{\pgfqpoint{-0.024552in}{0.024552in}}%
\pgfpathcurveto{\pgfqpoint{-0.031064in}{0.018041in}}{\pgfqpoint{-0.034722in}{0.009208in}}{\pgfqpoint{-0.034722in}{0.000000in}}%
\pgfpathcurveto{\pgfqpoint{-0.034722in}{-0.009208in}}{\pgfqpoint{-0.031064in}{-0.018041in}}{\pgfqpoint{-0.024552in}{-0.024552in}}%
\pgfpathcurveto{\pgfqpoint{-0.018041in}{-0.031064in}}{\pgfqpoint{-0.009208in}{-0.034722in}}{\pgfqpoint{0.000000in}{-0.034722in}}%
\pgfpathlineto{\pgfqpoint{0.000000in}{-0.034722in}}%
\pgfpathclose%
\pgfusepath{stroke,fill}%
}%
\begin{pgfscope}%
\pgfsys@transformshift{3.102325in}{1.918183in}%
\pgfsys@useobject{currentmarker}{}%
\end{pgfscope}%
\begin{pgfscope}%
\pgfsys@transformshift{3.148590in}{1.761212in}%
\pgfsys@useobject{currentmarker}{}%
\end{pgfscope}%
\begin{pgfscope}%
\pgfsys@transformshift{3.225698in}{1.691764in}%
\pgfsys@useobject{currentmarker}{}%
\end{pgfscope}%
\begin{pgfscope}%
\pgfsys@transformshift{3.333649in}{1.622491in}%
\pgfsys@useobject{currentmarker}{}%
\end{pgfscope}%
\begin{pgfscope}%
\pgfsys@transformshift{3.472444in}{1.584401in}%
\pgfsys@useobject{currentmarker}{}%
\end{pgfscope}%
\begin{pgfscope}%
\pgfsys@transformshift{3.642081in}{1.522434in}%
\pgfsys@useobject{currentmarker}{}%
\end{pgfscope}%
\begin{pgfscope}%
\pgfsys@transformshift{3.842562in}{1.485517in}%
\pgfsys@useobject{currentmarker}{}%
\end{pgfscope}%
\begin{pgfscope}%
\pgfsys@transformshift{4.073886in}{1.428112in}%
\pgfsys@useobject{currentmarker}{}%
\end{pgfscope}%
\begin{pgfscope}%
\pgfsys@transformshift{4.336053in}{1.412213in}%
\pgfsys@useobject{currentmarker}{}%
\end{pgfscope}%
\begin{pgfscope}%
\pgfsys@transformshift{4.629064in}{1.355152in}%
\pgfsys@useobject{currentmarker}{}%
\end{pgfscope}%
\end{pgfscope}%
\begin{pgfscope}%
\pgfsetroundcap%
\pgfsetroundjoin%
\pgfsetlinewidth{1.003750pt}%
\definecolor{currentstroke}{rgb}{0.800000,0.470588,0.737255}%
\pgfsetstrokecolor{currentstroke}%
\pgfsetdash{}{0pt}%
\pgfpathmoveto{\pgfqpoint{3.102325in}{1.354829in}}%
\pgfpathlineto{\pgfqpoint{3.148590in}{1.057761in}}%
\pgfpathlineto{\pgfqpoint{3.225698in}{1.047060in}}%
\pgfpathlineto{\pgfqpoint{3.333649in}{0.923685in}}%
\pgfpathlineto{\pgfqpoint{3.472444in}{0.861939in}}%
\pgfpathlineto{\pgfqpoint{3.642081in}{1.013924in}}%
\pgfpathlineto{\pgfqpoint{3.842562in}{0.749796in}}%
\pgfpathlineto{\pgfqpoint{4.073886in}{0.733930in}}%
\pgfpathlineto{\pgfqpoint{4.336053in}{0.706667in}}%
\pgfpathlineto{\pgfqpoint{4.629064in}{0.601875in}}%
\pgfusepath{stroke}%
\end{pgfscope}%
\begin{pgfscope}%
\pgfsetbuttcap%
\pgfsetroundjoin%
\definecolor{currentfill}{rgb}{0.800000,0.470588,0.737255}%
\pgfsetfillcolor{currentfill}%
\pgfsetlinewidth{0.752812pt}%
\definecolor{currentstroke}{rgb}{1.000000,1.000000,1.000000}%
\pgfsetstrokecolor{currentstroke}%
\pgfsetdash{}{0pt}%
\pgfsys@defobject{currentmarker}{\pgfqpoint{-0.034722in}{-0.034722in}}{\pgfqpoint{0.034722in}{0.034722in}}{%
\pgfpathmoveto{\pgfqpoint{0.000000in}{-0.034722in}}%
\pgfpathcurveto{\pgfqpoint{0.009208in}{-0.034722in}}{\pgfqpoint{0.018041in}{-0.031064in}}{\pgfqpoint{0.024552in}{-0.024552in}}%
\pgfpathcurveto{\pgfqpoint{0.031064in}{-0.018041in}}{\pgfqpoint{0.034722in}{-0.009208in}}{\pgfqpoint{0.034722in}{0.000000in}}%
\pgfpathcurveto{\pgfqpoint{0.034722in}{0.009208in}}{\pgfqpoint{0.031064in}{0.018041in}}{\pgfqpoint{0.024552in}{0.024552in}}%
\pgfpathcurveto{\pgfqpoint{0.018041in}{0.031064in}}{\pgfqpoint{0.009208in}{0.034722in}}{\pgfqpoint{0.000000in}{0.034722in}}%
\pgfpathcurveto{\pgfqpoint{-0.009208in}{0.034722in}}{\pgfqpoint{-0.018041in}{0.031064in}}{\pgfqpoint{-0.024552in}{0.024552in}}%
\pgfpathcurveto{\pgfqpoint{-0.031064in}{0.018041in}}{\pgfqpoint{-0.034722in}{0.009208in}}{\pgfqpoint{-0.034722in}{0.000000in}}%
\pgfpathcurveto{\pgfqpoint{-0.034722in}{-0.009208in}}{\pgfqpoint{-0.031064in}{-0.018041in}}{\pgfqpoint{-0.024552in}{-0.024552in}}%
\pgfpathcurveto{\pgfqpoint{-0.018041in}{-0.031064in}}{\pgfqpoint{-0.009208in}{-0.034722in}}{\pgfqpoint{0.000000in}{-0.034722in}}%
\pgfpathlineto{\pgfqpoint{0.000000in}{-0.034722in}}%
\pgfpathclose%
\pgfusepath{stroke,fill}%
}%
\begin{pgfscope}%
\pgfsys@transformshift{3.102325in}{1.354829in}%
\pgfsys@useobject{currentmarker}{}%
\end{pgfscope}%
\begin{pgfscope}%
\pgfsys@transformshift{3.148590in}{1.057761in}%
\pgfsys@useobject{currentmarker}{}%
\end{pgfscope}%
\begin{pgfscope}%
\pgfsys@transformshift{3.225698in}{1.047060in}%
\pgfsys@useobject{currentmarker}{}%
\end{pgfscope}%
\begin{pgfscope}%
\pgfsys@transformshift{3.333649in}{0.923685in}%
\pgfsys@useobject{currentmarker}{}%
\end{pgfscope}%
\begin{pgfscope}%
\pgfsys@transformshift{3.472444in}{0.861939in}%
\pgfsys@useobject{currentmarker}{}%
\end{pgfscope}%
\begin{pgfscope}%
\pgfsys@transformshift{3.642081in}{1.013924in}%
\pgfsys@useobject{currentmarker}{}%
\end{pgfscope}%
\begin{pgfscope}%
\pgfsys@transformshift{3.842562in}{0.749796in}%
\pgfsys@useobject{currentmarker}{}%
\end{pgfscope}%
\begin{pgfscope}%
\pgfsys@transformshift{4.073886in}{0.733930in}%
\pgfsys@useobject{currentmarker}{}%
\end{pgfscope}%
\begin{pgfscope}%
\pgfsys@transformshift{4.336053in}{0.706667in}%
\pgfsys@useobject{currentmarker}{}%
\end{pgfscope}%
\begin{pgfscope}%
\pgfsys@transformshift{4.629064in}{0.601875in}%
\pgfsys@useobject{currentmarker}{}%
\end{pgfscope}%
\end{pgfscope}%
\begin{pgfscope}%
\pgfsetroundcap%
\pgfsetroundjoin%
\pgfsetlinewidth{1.003750pt}%
\definecolor{currentstroke}{rgb}{0.792157,0.568627,0.380392}%
\pgfsetstrokecolor{currentstroke}%
\pgfsetdash{}{0pt}%
\pgfpathmoveto{\pgfqpoint{3.102325in}{1.279473in}}%
\pgfpathlineto{\pgfqpoint{3.148590in}{1.042278in}}%
\pgfpathlineto{\pgfqpoint{3.225698in}{0.897610in}}%
\pgfpathlineto{\pgfqpoint{3.333649in}{0.905261in}}%
\pgfpathlineto{\pgfqpoint{3.472444in}{0.836968in}}%
\pgfpathlineto{\pgfqpoint{3.642081in}{0.763547in}}%
\pgfpathlineto{\pgfqpoint{3.842562in}{0.728940in}}%
\pgfpathlineto{\pgfqpoint{4.073886in}{0.658041in}}%
\pgfpathlineto{\pgfqpoint{4.336053in}{0.632084in}}%
\pgfpathlineto{\pgfqpoint{4.629064in}{0.639387in}}%
\pgfusepath{stroke}%
\end{pgfscope}%
\begin{pgfscope}%
\pgfsetbuttcap%
\pgfsetroundjoin%
\definecolor{currentfill}{rgb}{0.792157,0.568627,0.380392}%
\pgfsetfillcolor{currentfill}%
\pgfsetlinewidth{0.752812pt}%
\definecolor{currentstroke}{rgb}{1.000000,1.000000,1.000000}%
\pgfsetstrokecolor{currentstroke}%
\pgfsetdash{}{0pt}%
\pgfsys@defobject{currentmarker}{\pgfqpoint{-0.034722in}{-0.034722in}}{\pgfqpoint{0.034722in}{0.034722in}}{%
\pgfpathmoveto{\pgfqpoint{0.000000in}{-0.034722in}}%
\pgfpathcurveto{\pgfqpoint{0.009208in}{-0.034722in}}{\pgfqpoint{0.018041in}{-0.031064in}}{\pgfqpoint{0.024552in}{-0.024552in}}%
\pgfpathcurveto{\pgfqpoint{0.031064in}{-0.018041in}}{\pgfqpoint{0.034722in}{-0.009208in}}{\pgfqpoint{0.034722in}{0.000000in}}%
\pgfpathcurveto{\pgfqpoint{0.034722in}{0.009208in}}{\pgfqpoint{0.031064in}{0.018041in}}{\pgfqpoint{0.024552in}{0.024552in}}%
\pgfpathcurveto{\pgfqpoint{0.018041in}{0.031064in}}{\pgfqpoint{0.009208in}{0.034722in}}{\pgfqpoint{0.000000in}{0.034722in}}%
\pgfpathcurveto{\pgfqpoint{-0.009208in}{0.034722in}}{\pgfqpoint{-0.018041in}{0.031064in}}{\pgfqpoint{-0.024552in}{0.024552in}}%
\pgfpathcurveto{\pgfqpoint{-0.031064in}{0.018041in}}{\pgfqpoint{-0.034722in}{0.009208in}}{\pgfqpoint{-0.034722in}{0.000000in}}%
\pgfpathcurveto{\pgfqpoint{-0.034722in}{-0.009208in}}{\pgfqpoint{-0.031064in}{-0.018041in}}{\pgfqpoint{-0.024552in}{-0.024552in}}%
\pgfpathcurveto{\pgfqpoint{-0.018041in}{-0.031064in}}{\pgfqpoint{-0.009208in}{-0.034722in}}{\pgfqpoint{0.000000in}{-0.034722in}}%
\pgfpathlineto{\pgfqpoint{0.000000in}{-0.034722in}}%
\pgfpathclose%
\pgfusepath{stroke,fill}%
}%
\begin{pgfscope}%
\pgfsys@transformshift{3.102325in}{1.279473in}%
\pgfsys@useobject{currentmarker}{}%
\end{pgfscope}%
\begin{pgfscope}%
\pgfsys@transformshift{3.148590in}{1.042278in}%
\pgfsys@useobject{currentmarker}{}%
\end{pgfscope}%
\begin{pgfscope}%
\pgfsys@transformshift{3.225698in}{0.897610in}%
\pgfsys@useobject{currentmarker}{}%
\end{pgfscope}%
\begin{pgfscope}%
\pgfsys@transformshift{3.333649in}{0.905261in}%
\pgfsys@useobject{currentmarker}{}%
\end{pgfscope}%
\begin{pgfscope}%
\pgfsys@transformshift{3.472444in}{0.836968in}%
\pgfsys@useobject{currentmarker}{}%
\end{pgfscope}%
\begin{pgfscope}%
\pgfsys@transformshift{3.642081in}{0.763547in}%
\pgfsys@useobject{currentmarker}{}%
\end{pgfscope}%
\begin{pgfscope}%
\pgfsys@transformshift{3.842562in}{0.728940in}%
\pgfsys@useobject{currentmarker}{}%
\end{pgfscope}%
\begin{pgfscope}%
\pgfsys@transformshift{4.073886in}{0.658041in}%
\pgfsys@useobject{currentmarker}{}%
\end{pgfscope}%
\begin{pgfscope}%
\pgfsys@transformshift{4.336053in}{0.632084in}%
\pgfsys@useobject{currentmarker}{}%
\end{pgfscope}%
\begin{pgfscope}%
\pgfsys@transformshift{4.629064in}{0.639387in}%
\pgfsys@useobject{currentmarker}{}%
\end{pgfscope}%
\end{pgfscope}%
\end{pgfpicture}%
\makeatother%
\endgroup%

				\end{figcenter}
				\caption[Absolute and relative graph generation time per task for the maze dataset.]{Tasks during graph generation using the maze dataset in absolute time (left) and relative share on the total time (right). The kNN search determines the total processing time covering the data points for the total time.}
				\label{fig:eval-import-pattern-maze-abs-rel}
			\end{figure}
		
		\subsubsection{Routing}
		
			Because the pattern-based datasets did not contain any roads, routing times in these datasets solely depend on the size of the visibility graph and therefore on the time required to connect the source and destination vertices, which is the most time consuming task during routing.
			Connecting vertices uses the kNN search of the graph generation and therefore its runtime also increases quadratically with the number of vertices in the graph.
			
			The A* algorithm uses this quadratically growing graph and therefore shows a quadratic runtime as well.
			However, this quadratic runtime is not clearly visible in all measurements and a linear regression on the maze- and rectangle-based measurements yield a similarly good correlation.
			The effect of a quadratic complexity might therefore only be significant in larger datasets.
			
			\begin{figure}[h!]
				\hspace{-20pt}
				%% Creator: Matplotlib, PGF backend
%%
%% To include the figure in your LaTeX document, write
%%   \input{<filename>.pgf}
%%
%% Make sure the required packages are loaded in your preamble
%%   \usepackage{pgf}
%%
%% Also ensure that all the required font packages are loaded; for instance,
%% the lmodern package is sometimes necessary when using math font.
%%   \usepackage{lmodern}
%%
%% Figures using additional raster images can only be included by \input if
%% they are in the same directory as the main LaTeX file. For loading figures
%% from other directories you can use the `import` package
%%   \usepackage{import}
%%
%% and then include the figures with
%%   \import{<path to file>}{<filename>.pgf}
%%
%% Matplotlib used the following preamble
%%   
%%   \usepackage{fontspec}
%%   \setmainfont{DejaVuSerif.ttf}[Path=\detokenize{/home/hauke/.local/lib/python3.11/site-packages/matplotlib/mpl-data/fonts/ttf/}]
%%   \setsansfont{DroidSans.ttf}[Path=\detokenize{/usr/share/fonts/droid/}]
%%   \setmonofont{DejaVuSansMono.ttf}[Path=\detokenize{/home/hauke/.local/lib/python3.11/site-packages/matplotlib/mpl-data/fonts/ttf/}]
%%   \makeatletter\@ifpackageloaded{underscore}{}{\usepackage[strings]{underscore}}\makeatother
%%
\begingroup%
\makeatletter%
\begin{pgfpicture}%
\pgfpathrectangle{\pgfpointorigin}{\pgfqpoint{6.635097in}{2.410942in}}%
\pgfusepath{use as bounding box, clip}%
\begin{pgfscope}%
\pgfsetbuttcap%
\pgfsetmiterjoin%
\definecolor{currentfill}{rgb}{1.000000,1.000000,1.000000}%
\pgfsetfillcolor{currentfill}%
\pgfsetlinewidth{0.000000pt}%
\definecolor{currentstroke}{rgb}{1.000000,1.000000,1.000000}%
\pgfsetstrokecolor{currentstroke}%
\pgfsetdash{}{0pt}%
\pgfpathmoveto{\pgfqpoint{0.000000in}{0.000000in}}%
\pgfpathlineto{\pgfqpoint{6.635097in}{0.000000in}}%
\pgfpathlineto{\pgfqpoint{6.635097in}{2.410942in}}%
\pgfpathlineto{\pgfqpoint{0.000000in}{2.410942in}}%
\pgfpathlineto{\pgfqpoint{0.000000in}{0.000000in}}%
\pgfpathclose%
\pgfusepath{fill}%
\end{pgfscope}%
\begin{pgfscope}%
\pgfsetbuttcap%
\pgfsetmiterjoin%
\definecolor{currentfill}{rgb}{1.000000,1.000000,1.000000}%
\pgfsetfillcolor{currentfill}%
\pgfsetlinewidth{0.000000pt}%
\definecolor{currentstroke}{rgb}{0.000000,0.000000,0.000000}%
\pgfsetstrokecolor{currentstroke}%
\pgfsetstrokeopacity{0.000000}%
\pgfsetdash{}{0pt}%
\pgfpathmoveto{\pgfqpoint{0.456543in}{0.451389in}}%
\pgfpathlineto{\pgfqpoint{1.852525in}{0.451389in}}%
\pgfpathlineto{\pgfqpoint{1.852525in}{2.204860in}}%
\pgfpathlineto{\pgfqpoint{0.456543in}{2.204860in}}%
\pgfpathlineto{\pgfqpoint{0.456543in}{0.451389in}}%
\pgfpathclose%
\pgfusepath{fill}%
\end{pgfscope}%
\begin{pgfscope}%
\pgfpathrectangle{\pgfqpoint{0.456543in}{0.451389in}}{\pgfqpoint{1.395982in}{1.753471in}}%
\pgfusepath{clip}%
\pgfsetroundcap%
\pgfsetroundjoin%
\pgfsetlinewidth{1.003750pt}%
\definecolor{currentstroke}{rgb}{0.800000,0.800000,0.800000}%
\pgfsetstrokecolor{currentstroke}%
\pgfsetdash{}{0pt}%
\pgfpathmoveto{\pgfqpoint{0.456543in}{0.451389in}}%
\pgfpathlineto{\pgfqpoint{0.456543in}{2.204860in}}%
\pgfusepath{stroke}%
\end{pgfscope}%
\begin{pgfscope}%
\definecolor{textcolor}{rgb}{0.150000,0.150000,0.150000}%
\pgfsetstrokecolor{textcolor}%
\pgfsetfillcolor{textcolor}%
\pgftext[x=0.456543in,y=0.319444in,,top]{\color{textcolor}\sffamily\fontsize{9.000000}{10.800000}\selectfont 0k}%
\end{pgfscope}%
\begin{pgfscope}%
\pgfpathrectangle{\pgfqpoint{0.456543in}{0.451389in}}{\pgfqpoint{1.395982in}{1.753471in}}%
\pgfusepath{clip}%
\pgfsetroundcap%
\pgfsetroundjoin%
\pgfsetlinewidth{1.003750pt}%
\definecolor{currentstroke}{rgb}{0.800000,0.800000,0.800000}%
\pgfsetstrokecolor{currentstroke}%
\pgfsetdash{}{0pt}%
\pgfpathmoveto{\pgfqpoint{0.899765in}{0.451389in}}%
\pgfpathlineto{\pgfqpoint{0.899765in}{2.204860in}}%
\pgfusepath{stroke}%
\end{pgfscope}%
\begin{pgfscope}%
\definecolor{textcolor}{rgb}{0.150000,0.150000,0.150000}%
\pgfsetstrokecolor{textcolor}%
\pgfsetfillcolor{textcolor}%
\pgftext[x=0.899765in,y=0.319444in,,top]{\color{textcolor}\sffamily\fontsize{9.000000}{10.800000}\selectfont 10k}%
\end{pgfscope}%
\begin{pgfscope}%
\pgfpathrectangle{\pgfqpoint{0.456543in}{0.451389in}}{\pgfqpoint{1.395982in}{1.753471in}}%
\pgfusepath{clip}%
\pgfsetroundcap%
\pgfsetroundjoin%
\pgfsetlinewidth{1.003750pt}%
\definecolor{currentstroke}{rgb}{0.800000,0.800000,0.800000}%
\pgfsetstrokecolor{currentstroke}%
\pgfsetdash{}{0pt}%
\pgfpathmoveto{\pgfqpoint{1.342986in}{0.451389in}}%
\pgfpathlineto{\pgfqpoint{1.342986in}{2.204860in}}%
\pgfusepath{stroke}%
\end{pgfscope}%
\begin{pgfscope}%
\definecolor{textcolor}{rgb}{0.150000,0.150000,0.150000}%
\pgfsetstrokecolor{textcolor}%
\pgfsetfillcolor{textcolor}%
\pgftext[x=1.342986in,y=0.319444in,,top]{\color{textcolor}\sffamily\fontsize{9.000000}{10.800000}\selectfont 20k}%
\end{pgfscope}%
\begin{pgfscope}%
\pgfpathrectangle{\pgfqpoint{0.456543in}{0.451389in}}{\pgfqpoint{1.395982in}{1.753471in}}%
\pgfusepath{clip}%
\pgfsetroundcap%
\pgfsetroundjoin%
\pgfsetlinewidth{1.003750pt}%
\definecolor{currentstroke}{rgb}{0.800000,0.800000,0.800000}%
\pgfsetstrokecolor{currentstroke}%
\pgfsetdash{}{0pt}%
\pgfpathmoveto{\pgfqpoint{1.786208in}{0.451389in}}%
\pgfpathlineto{\pgfqpoint{1.786208in}{2.204860in}}%
\pgfusepath{stroke}%
\end{pgfscope}%
\begin{pgfscope}%
\definecolor{textcolor}{rgb}{0.150000,0.150000,0.150000}%
\pgfsetstrokecolor{textcolor}%
\pgfsetfillcolor{textcolor}%
\pgftext[x=1.786208in,y=0.319444in,,top]{\color{textcolor}\sffamily\fontsize{9.000000}{10.800000}\selectfont 30k}%
\end{pgfscope}%
\begin{pgfscope}%
\definecolor{textcolor}{rgb}{0.150000,0.150000,0.150000}%
\pgfsetstrokecolor{textcolor}%
\pgfsetfillcolor{textcolor}%
\pgftext[x=1.154534in,y=0.125000in,,top]{\color{textcolor}\sffamily\fontsize{9.000000}{10.800000}\selectfont Input obstacle vertices}%
\end{pgfscope}%
\begin{pgfscope}%
\pgfpathrectangle{\pgfqpoint{0.456543in}{0.451389in}}{\pgfqpoint{1.395982in}{1.753471in}}%
\pgfusepath{clip}%
\pgfsetroundcap%
\pgfsetroundjoin%
\pgfsetlinewidth{1.003750pt}%
\definecolor{currentstroke}{rgb}{0.800000,0.800000,0.800000}%
\pgfsetstrokecolor{currentstroke}%
\pgfsetdash{}{0pt}%
\pgfpathmoveto{\pgfqpoint{0.456543in}{0.451389in}}%
\pgfpathlineto{\pgfqpoint{1.852525in}{0.451389in}}%
\pgfusepath{stroke}%
\end{pgfscope}%
\begin{pgfscope}%
\definecolor{textcolor}{rgb}{0.150000,0.150000,0.150000}%
\pgfsetstrokecolor{textcolor}%
\pgfsetfillcolor{textcolor}%
\pgftext[x=0.332140in, y=0.403903in, left, base]{\color{textcolor}\sffamily\fontsize{9.000000}{10.800000}\selectfont 0}%
\end{pgfscope}%
\begin{pgfscope}%
\pgfpathrectangle{\pgfqpoint{0.456543in}{0.451389in}}{\pgfqpoint{1.395982in}{1.753471in}}%
\pgfusepath{clip}%
\pgfsetroundcap%
\pgfsetroundjoin%
\pgfsetlinewidth{1.003750pt}%
\definecolor{currentstroke}{rgb}{0.800000,0.800000,0.800000}%
\pgfsetstrokecolor{currentstroke}%
\pgfsetdash{}{0pt}%
\pgfpathmoveto{\pgfqpoint{0.456543in}{0.822467in}}%
\pgfpathlineto{\pgfqpoint{1.852525in}{0.822467in}}%
\pgfusepath{stroke}%
\end{pgfscope}%
\begin{pgfscope}%
\definecolor{textcolor}{rgb}{0.150000,0.150000,0.150000}%
\pgfsetstrokecolor{textcolor}%
\pgfsetfillcolor{textcolor}%
\pgftext[x=0.263292in, y=0.774981in, left, base]{\color{textcolor}\sffamily\fontsize{9.000000}{10.800000}\selectfont 50}%
\end{pgfscope}%
\begin{pgfscope}%
\pgfpathrectangle{\pgfqpoint{0.456543in}{0.451389in}}{\pgfqpoint{1.395982in}{1.753471in}}%
\pgfusepath{clip}%
\pgfsetroundcap%
\pgfsetroundjoin%
\pgfsetlinewidth{1.003750pt}%
\definecolor{currentstroke}{rgb}{0.800000,0.800000,0.800000}%
\pgfsetstrokecolor{currentstroke}%
\pgfsetdash{}{0pt}%
\pgfpathmoveto{\pgfqpoint{0.456543in}{1.193545in}}%
\pgfpathlineto{\pgfqpoint{1.852525in}{1.193545in}}%
\pgfusepath{stroke}%
\end{pgfscope}%
\begin{pgfscope}%
\definecolor{textcolor}{rgb}{0.150000,0.150000,0.150000}%
\pgfsetstrokecolor{textcolor}%
\pgfsetfillcolor{textcolor}%
\pgftext[x=0.194444in, y=1.146059in, left, base]{\color{textcolor}\sffamily\fontsize{9.000000}{10.800000}\selectfont 100}%
\end{pgfscope}%
\begin{pgfscope}%
\pgfpathrectangle{\pgfqpoint{0.456543in}{0.451389in}}{\pgfqpoint{1.395982in}{1.753471in}}%
\pgfusepath{clip}%
\pgfsetroundcap%
\pgfsetroundjoin%
\pgfsetlinewidth{1.003750pt}%
\definecolor{currentstroke}{rgb}{0.800000,0.800000,0.800000}%
\pgfsetstrokecolor{currentstroke}%
\pgfsetdash{}{0pt}%
\pgfpathmoveto{\pgfqpoint{0.456543in}{1.564623in}}%
\pgfpathlineto{\pgfqpoint{1.852525in}{1.564623in}}%
\pgfusepath{stroke}%
\end{pgfscope}%
\begin{pgfscope}%
\definecolor{textcolor}{rgb}{0.150000,0.150000,0.150000}%
\pgfsetstrokecolor{textcolor}%
\pgfsetfillcolor{textcolor}%
\pgftext[x=0.194444in, y=1.517137in, left, base]{\color{textcolor}\sffamily\fontsize{9.000000}{10.800000}\selectfont 150}%
\end{pgfscope}%
\begin{pgfscope}%
\pgfpathrectangle{\pgfqpoint{0.456543in}{0.451389in}}{\pgfqpoint{1.395982in}{1.753471in}}%
\pgfusepath{clip}%
\pgfsetroundcap%
\pgfsetroundjoin%
\pgfsetlinewidth{1.003750pt}%
\definecolor{currentstroke}{rgb}{0.800000,0.800000,0.800000}%
\pgfsetstrokecolor{currentstroke}%
\pgfsetdash{}{0pt}%
\pgfpathmoveto{\pgfqpoint{0.456543in}{1.935701in}}%
\pgfpathlineto{\pgfqpoint{1.852525in}{1.935701in}}%
\pgfusepath{stroke}%
\end{pgfscope}%
\begin{pgfscope}%
\definecolor{textcolor}{rgb}{0.150000,0.150000,0.150000}%
\pgfsetstrokecolor{textcolor}%
\pgfsetfillcolor{textcolor}%
\pgftext[x=0.194444in, y=1.888215in, left, base]{\color{textcolor}\sffamily\fontsize{9.000000}{10.800000}\selectfont 200}%
\end{pgfscope}%
\begin{pgfscope}%
\definecolor{textcolor}{rgb}{0.150000,0.150000,0.150000}%
\pgfsetstrokecolor{textcolor}%
\pgfsetfillcolor{textcolor}%
\pgftext[x=0.125000in,y=1.328124in,,bottom,rotate=90.000000]{\color{textcolor}\sffamily\fontsize{9.000000}{10.800000}\selectfont Time in ms}%
\end{pgfscope}%
\begin{pgfscope}%
\pgfsetrectcap%
\pgfsetmiterjoin%
\pgfsetlinewidth{1.254687pt}%
\definecolor{currentstroke}{rgb}{0.800000,0.800000,0.800000}%
\pgfsetstrokecolor{currentstroke}%
\pgfsetdash{}{0pt}%
\pgfpathmoveto{\pgfqpoint{0.456543in}{0.451389in}}%
\pgfpathlineto{\pgfqpoint{0.456543in}{2.204860in}}%
\pgfusepath{stroke}%
\end{pgfscope}%
\begin{pgfscope}%
\pgfsetrectcap%
\pgfsetmiterjoin%
\pgfsetlinewidth{1.254687pt}%
\definecolor{currentstroke}{rgb}{0.800000,0.800000,0.800000}%
\pgfsetstrokecolor{currentstroke}%
\pgfsetdash{}{0pt}%
\pgfpathmoveto{\pgfqpoint{1.852525in}{0.451389in}}%
\pgfpathlineto{\pgfqpoint{1.852525in}{2.204860in}}%
\pgfusepath{stroke}%
\end{pgfscope}%
\begin{pgfscope}%
\pgfsetrectcap%
\pgfsetmiterjoin%
\pgfsetlinewidth{1.254687pt}%
\definecolor{currentstroke}{rgb}{0.800000,0.800000,0.800000}%
\pgfsetstrokecolor{currentstroke}%
\pgfsetdash{}{0pt}%
\pgfpathmoveto{\pgfqpoint{0.456543in}{0.451389in}}%
\pgfpathlineto{\pgfqpoint{1.852525in}{0.451389in}}%
\pgfusepath{stroke}%
\end{pgfscope}%
\begin{pgfscope}%
\pgfsetrectcap%
\pgfsetmiterjoin%
\pgfsetlinewidth{1.254687pt}%
\definecolor{currentstroke}{rgb}{0.800000,0.800000,0.800000}%
\pgfsetstrokecolor{currentstroke}%
\pgfsetdash{}{0pt}%
\pgfpathmoveto{\pgfqpoint{0.456543in}{2.204860in}}%
\pgfpathlineto{\pgfqpoint{1.852525in}{2.204860in}}%
\pgfusepath{stroke}%
\end{pgfscope}%
\begin{pgfscope}%
\definecolor{textcolor}{rgb}{0.150000,0.150000,0.150000}%
\pgfsetstrokecolor{textcolor}%
\pgfsetfillcolor{textcolor}%
\pgftext[x=1.154534in,y=2.315971in,,base]{\color{textcolor}\sffamily\fontsize{9.000000}{10.800000}\selectfont Rectangles}%
\end{pgfscope}%
\begin{pgfscope}%
\pgfsetroundcap%
\pgfsetroundjoin%
\pgfsetlinewidth{1.003750pt}%
\definecolor{currentstroke}{rgb}{0.003922,0.450980,0.698039}%
\pgfsetstrokecolor{currentstroke}%
\pgfsetdash{}{0pt}%
\pgfpathmoveto{\pgfqpoint{0.459867in}{0.461081in}}%
\pgfpathlineto{\pgfqpoint{0.469840in}{0.468696in}}%
\pgfpathlineto{\pgfqpoint{0.486460in}{0.492831in}}%
\pgfpathlineto{\pgfqpoint{0.509730in}{0.517923in}}%
\pgfpathlineto{\pgfqpoint{0.539647in}{0.536455in}}%
\pgfpathlineto{\pgfqpoint{0.576213in}{0.576694in}}%
\pgfpathlineto{\pgfqpoint{0.619427in}{0.607353in}}%
\pgfpathlineto{\pgfqpoint{0.669289in}{0.647437in}}%
\pgfpathlineto{\pgfqpoint{0.725800in}{0.746782in}}%
\pgfpathlineto{\pgfqpoint{0.788959in}{0.822519in}}%
\pgfpathlineto{\pgfqpoint{0.935222in}{1.012755in}}%
\pgfpathlineto{\pgfqpoint{1.108079in}{1.233985in}}%
\pgfpathlineto{\pgfqpoint{1.307529in}{1.375789in}}%
\pgfpathlineto{\pgfqpoint{1.533572in}{1.691754in}}%
\pgfpathlineto{\pgfqpoint{1.786208in}{2.121381in}}%
\pgfusepath{stroke}%
\end{pgfscope}%
\begin{pgfscope}%
\pgfsetbuttcap%
\pgfsetroundjoin%
\definecolor{currentfill}{rgb}{0.003922,0.450980,0.698039}%
\pgfsetfillcolor{currentfill}%
\pgfsetlinewidth{0.752812pt}%
\definecolor{currentstroke}{rgb}{1.000000,1.000000,1.000000}%
\pgfsetstrokecolor{currentstroke}%
\pgfsetdash{}{0pt}%
\pgfsys@defobject{currentmarker}{\pgfqpoint{-0.034722in}{-0.034722in}}{\pgfqpoint{0.034722in}{0.034722in}}{%
\pgfpathmoveto{\pgfqpoint{0.000000in}{-0.034722in}}%
\pgfpathcurveto{\pgfqpoint{0.009208in}{-0.034722in}}{\pgfqpoint{0.018041in}{-0.031064in}}{\pgfqpoint{0.024552in}{-0.024552in}}%
\pgfpathcurveto{\pgfqpoint{0.031064in}{-0.018041in}}{\pgfqpoint{0.034722in}{-0.009208in}}{\pgfqpoint{0.034722in}{0.000000in}}%
\pgfpathcurveto{\pgfqpoint{0.034722in}{0.009208in}}{\pgfqpoint{0.031064in}{0.018041in}}{\pgfqpoint{0.024552in}{0.024552in}}%
\pgfpathcurveto{\pgfqpoint{0.018041in}{0.031064in}}{\pgfqpoint{0.009208in}{0.034722in}}{\pgfqpoint{0.000000in}{0.034722in}}%
\pgfpathcurveto{\pgfqpoint{-0.009208in}{0.034722in}}{\pgfqpoint{-0.018041in}{0.031064in}}{\pgfqpoint{-0.024552in}{0.024552in}}%
\pgfpathcurveto{\pgfqpoint{-0.031064in}{0.018041in}}{\pgfqpoint{-0.034722in}{0.009208in}}{\pgfqpoint{-0.034722in}{0.000000in}}%
\pgfpathcurveto{\pgfqpoint{-0.034722in}{-0.009208in}}{\pgfqpoint{-0.031064in}{-0.018041in}}{\pgfqpoint{-0.024552in}{-0.024552in}}%
\pgfpathcurveto{\pgfqpoint{-0.018041in}{-0.031064in}}{\pgfqpoint{-0.009208in}{-0.034722in}}{\pgfqpoint{0.000000in}{-0.034722in}}%
\pgfpathlineto{\pgfqpoint{0.000000in}{-0.034722in}}%
\pgfpathclose%
\pgfusepath{stroke,fill}%
}%
\begin{pgfscope}%
\pgfsys@transformshift{0.459867in}{0.461081in}%
\pgfsys@useobject{currentmarker}{}%
\end{pgfscope}%
\begin{pgfscope}%
\pgfsys@transformshift{0.469840in}{0.468696in}%
\pgfsys@useobject{currentmarker}{}%
\end{pgfscope}%
\begin{pgfscope}%
\pgfsys@transformshift{0.486460in}{0.492831in}%
\pgfsys@useobject{currentmarker}{}%
\end{pgfscope}%
\begin{pgfscope}%
\pgfsys@transformshift{0.509730in}{0.517923in}%
\pgfsys@useobject{currentmarker}{}%
\end{pgfscope}%
\begin{pgfscope}%
\pgfsys@transformshift{0.539647in}{0.536455in}%
\pgfsys@useobject{currentmarker}{}%
\end{pgfscope}%
\begin{pgfscope}%
\pgfsys@transformshift{0.576213in}{0.576694in}%
\pgfsys@useobject{currentmarker}{}%
\end{pgfscope}%
\begin{pgfscope}%
\pgfsys@transformshift{0.619427in}{0.607353in}%
\pgfsys@useobject{currentmarker}{}%
\end{pgfscope}%
\begin{pgfscope}%
\pgfsys@transformshift{0.669289in}{0.647437in}%
\pgfsys@useobject{currentmarker}{}%
\end{pgfscope}%
\begin{pgfscope}%
\pgfsys@transformshift{0.725800in}{0.746782in}%
\pgfsys@useobject{currentmarker}{}%
\end{pgfscope}%
\begin{pgfscope}%
\pgfsys@transformshift{0.788959in}{0.822519in}%
\pgfsys@useobject{currentmarker}{}%
\end{pgfscope}%
\begin{pgfscope}%
\pgfsys@transformshift{0.935222in}{1.012755in}%
\pgfsys@useobject{currentmarker}{}%
\end{pgfscope}%
\begin{pgfscope}%
\pgfsys@transformshift{1.108079in}{1.233985in}%
\pgfsys@useobject{currentmarker}{}%
\end{pgfscope}%
\begin{pgfscope}%
\pgfsys@transformshift{1.307529in}{1.375789in}%
\pgfsys@useobject{currentmarker}{}%
\end{pgfscope}%
\begin{pgfscope}%
\pgfsys@transformshift{1.533572in}{1.691754in}%
\pgfsys@useobject{currentmarker}{}%
\end{pgfscope}%
\begin{pgfscope}%
\pgfsys@transformshift{1.786208in}{2.121381in}%
\pgfsys@useobject{currentmarker}{}%
\end{pgfscope}%
\end{pgfscope}%
\begin{pgfscope}%
\pgfsetroundcap%
\pgfsetroundjoin%
\pgfsetlinewidth{1.003750pt}%
\definecolor{currentstroke}{rgb}{0.870588,0.560784,0.019608}%
\pgfsetstrokecolor{currentstroke}%
\pgfsetdash{}{0pt}%
\pgfpathmoveto{\pgfqpoint{0.459867in}{0.451797in}}%
\pgfpathlineto{\pgfqpoint{0.469840in}{0.454046in}}%
\pgfpathlineto{\pgfqpoint{0.486460in}{0.457422in}}%
\pgfpathlineto{\pgfqpoint{0.509730in}{0.463961in}}%
\pgfpathlineto{\pgfqpoint{0.539647in}{0.472043in}}%
\pgfpathlineto{\pgfqpoint{0.576213in}{0.483724in}}%
\pgfpathlineto{\pgfqpoint{0.619427in}{0.496623in}}%
\pgfpathlineto{\pgfqpoint{0.669289in}{0.514784in}}%
\pgfpathlineto{\pgfqpoint{0.725800in}{0.530695in}}%
\pgfpathlineto{\pgfqpoint{0.788959in}{0.554007in}}%
\pgfpathlineto{\pgfqpoint{0.935222in}{0.603063in}}%
\pgfpathlineto{\pgfqpoint{1.108079in}{0.665835in}}%
\pgfpathlineto{\pgfqpoint{1.307529in}{0.736584in}}%
\pgfpathlineto{\pgfqpoint{1.533572in}{0.826645in}}%
\pgfpathlineto{\pgfqpoint{1.786208in}{0.907956in}}%
\pgfusepath{stroke}%
\end{pgfscope}%
\begin{pgfscope}%
\pgfsetbuttcap%
\pgfsetroundjoin%
\definecolor{currentfill}{rgb}{0.870588,0.560784,0.019608}%
\pgfsetfillcolor{currentfill}%
\pgfsetlinewidth{0.752812pt}%
\definecolor{currentstroke}{rgb}{1.000000,1.000000,1.000000}%
\pgfsetstrokecolor{currentstroke}%
\pgfsetdash{}{0pt}%
\pgfsys@defobject{currentmarker}{\pgfqpoint{-0.034722in}{-0.034722in}}{\pgfqpoint{0.034722in}{0.034722in}}{%
\pgfpathmoveto{\pgfqpoint{0.000000in}{-0.034722in}}%
\pgfpathcurveto{\pgfqpoint{0.009208in}{-0.034722in}}{\pgfqpoint{0.018041in}{-0.031064in}}{\pgfqpoint{0.024552in}{-0.024552in}}%
\pgfpathcurveto{\pgfqpoint{0.031064in}{-0.018041in}}{\pgfqpoint{0.034722in}{-0.009208in}}{\pgfqpoint{0.034722in}{0.000000in}}%
\pgfpathcurveto{\pgfqpoint{0.034722in}{0.009208in}}{\pgfqpoint{0.031064in}{0.018041in}}{\pgfqpoint{0.024552in}{0.024552in}}%
\pgfpathcurveto{\pgfqpoint{0.018041in}{0.031064in}}{\pgfqpoint{0.009208in}{0.034722in}}{\pgfqpoint{0.000000in}{0.034722in}}%
\pgfpathcurveto{\pgfqpoint{-0.009208in}{0.034722in}}{\pgfqpoint{-0.018041in}{0.031064in}}{\pgfqpoint{-0.024552in}{0.024552in}}%
\pgfpathcurveto{\pgfqpoint{-0.031064in}{0.018041in}}{\pgfqpoint{-0.034722in}{0.009208in}}{\pgfqpoint{-0.034722in}{0.000000in}}%
\pgfpathcurveto{\pgfqpoint{-0.034722in}{-0.009208in}}{\pgfqpoint{-0.031064in}{-0.018041in}}{\pgfqpoint{-0.024552in}{-0.024552in}}%
\pgfpathcurveto{\pgfqpoint{-0.018041in}{-0.031064in}}{\pgfqpoint{-0.009208in}{-0.034722in}}{\pgfqpoint{0.000000in}{-0.034722in}}%
\pgfpathlineto{\pgfqpoint{0.000000in}{-0.034722in}}%
\pgfpathclose%
\pgfusepath{stroke,fill}%
}%
\begin{pgfscope}%
\pgfsys@transformshift{0.459867in}{0.451797in}%
\pgfsys@useobject{currentmarker}{}%
\end{pgfscope}%
\begin{pgfscope}%
\pgfsys@transformshift{0.469840in}{0.454046in}%
\pgfsys@useobject{currentmarker}{}%
\end{pgfscope}%
\begin{pgfscope}%
\pgfsys@transformshift{0.486460in}{0.457422in}%
\pgfsys@useobject{currentmarker}{}%
\end{pgfscope}%
\begin{pgfscope}%
\pgfsys@transformshift{0.509730in}{0.463961in}%
\pgfsys@useobject{currentmarker}{}%
\end{pgfscope}%
\begin{pgfscope}%
\pgfsys@transformshift{0.539647in}{0.472043in}%
\pgfsys@useobject{currentmarker}{}%
\end{pgfscope}%
\begin{pgfscope}%
\pgfsys@transformshift{0.576213in}{0.483724in}%
\pgfsys@useobject{currentmarker}{}%
\end{pgfscope}%
\begin{pgfscope}%
\pgfsys@transformshift{0.619427in}{0.496623in}%
\pgfsys@useobject{currentmarker}{}%
\end{pgfscope}%
\begin{pgfscope}%
\pgfsys@transformshift{0.669289in}{0.514784in}%
\pgfsys@useobject{currentmarker}{}%
\end{pgfscope}%
\begin{pgfscope}%
\pgfsys@transformshift{0.725800in}{0.530695in}%
\pgfsys@useobject{currentmarker}{}%
\end{pgfscope}%
\begin{pgfscope}%
\pgfsys@transformshift{0.788959in}{0.554007in}%
\pgfsys@useobject{currentmarker}{}%
\end{pgfscope}%
\begin{pgfscope}%
\pgfsys@transformshift{0.935222in}{0.603063in}%
\pgfsys@useobject{currentmarker}{}%
\end{pgfscope}%
\begin{pgfscope}%
\pgfsys@transformshift{1.108079in}{0.665835in}%
\pgfsys@useobject{currentmarker}{}%
\end{pgfscope}%
\begin{pgfscope}%
\pgfsys@transformshift{1.307529in}{0.736584in}%
\pgfsys@useobject{currentmarker}{}%
\end{pgfscope}%
\begin{pgfscope}%
\pgfsys@transformshift{1.533572in}{0.826645in}%
\pgfsys@useobject{currentmarker}{}%
\end{pgfscope}%
\begin{pgfscope}%
\pgfsys@transformshift{1.786208in}{0.907956in}%
\pgfsys@useobject{currentmarker}{}%
\end{pgfscope}%
\end{pgfscope}%
\begin{pgfscope}%
\pgfsetroundcap%
\pgfsetroundjoin%
\pgfsetlinewidth{1.003750pt}%
\definecolor{currentstroke}{rgb}{0.007843,0.619608,0.450980}%
\pgfsetstrokecolor{currentstroke}%
\pgfsetdash{}{0pt}%
\pgfpathmoveto{\pgfqpoint{0.459867in}{0.459990in}}%
\pgfpathlineto{\pgfqpoint{0.469840in}{0.464451in}}%
\pgfpathlineto{\pgfqpoint{0.486460in}{0.483806in}}%
\pgfpathlineto{\pgfqpoint{0.509730in}{0.499347in}}%
\pgfpathlineto{\pgfqpoint{0.539647in}{0.505803in}}%
\pgfpathlineto{\pgfqpoint{0.576213in}{0.528751in}}%
\pgfpathlineto{\pgfqpoint{0.619427in}{0.540522in}}%
\pgfpathlineto{\pgfqpoint{0.669289in}{0.555402in}}%
\pgfpathlineto{\pgfqpoint{0.725800in}{0.629090in}}%
\pgfpathlineto{\pgfqpoint{0.788959in}{0.661441in}}%
\pgfpathlineto{\pgfqpoint{0.935222in}{0.778286in}}%
\pgfpathlineto{\pgfqpoint{1.108079in}{0.899725in}}%
\pgfpathlineto{\pgfqpoint{1.307529in}{0.912527in}}%
\pgfpathlineto{\pgfqpoint{1.533572in}{1.113681in}}%
\pgfpathlineto{\pgfqpoint{1.786208in}{1.388962in}}%
\pgfusepath{stroke}%
\end{pgfscope}%
\begin{pgfscope}%
\pgfsetbuttcap%
\pgfsetroundjoin%
\definecolor{currentfill}{rgb}{0.007843,0.619608,0.450980}%
\pgfsetfillcolor{currentfill}%
\pgfsetlinewidth{0.752812pt}%
\definecolor{currentstroke}{rgb}{1.000000,1.000000,1.000000}%
\pgfsetstrokecolor{currentstroke}%
\pgfsetdash{}{0pt}%
\pgfsys@defobject{currentmarker}{\pgfqpoint{-0.034722in}{-0.034722in}}{\pgfqpoint{0.034722in}{0.034722in}}{%
\pgfpathmoveto{\pgfqpoint{0.000000in}{-0.034722in}}%
\pgfpathcurveto{\pgfqpoint{0.009208in}{-0.034722in}}{\pgfqpoint{0.018041in}{-0.031064in}}{\pgfqpoint{0.024552in}{-0.024552in}}%
\pgfpathcurveto{\pgfqpoint{0.031064in}{-0.018041in}}{\pgfqpoint{0.034722in}{-0.009208in}}{\pgfqpoint{0.034722in}{0.000000in}}%
\pgfpathcurveto{\pgfqpoint{0.034722in}{0.009208in}}{\pgfqpoint{0.031064in}{0.018041in}}{\pgfqpoint{0.024552in}{0.024552in}}%
\pgfpathcurveto{\pgfqpoint{0.018041in}{0.031064in}}{\pgfqpoint{0.009208in}{0.034722in}}{\pgfqpoint{0.000000in}{0.034722in}}%
\pgfpathcurveto{\pgfqpoint{-0.009208in}{0.034722in}}{\pgfqpoint{-0.018041in}{0.031064in}}{\pgfqpoint{-0.024552in}{0.024552in}}%
\pgfpathcurveto{\pgfqpoint{-0.031064in}{0.018041in}}{\pgfqpoint{-0.034722in}{0.009208in}}{\pgfqpoint{-0.034722in}{0.000000in}}%
\pgfpathcurveto{\pgfqpoint{-0.034722in}{-0.009208in}}{\pgfqpoint{-0.031064in}{-0.018041in}}{\pgfqpoint{-0.024552in}{-0.024552in}}%
\pgfpathcurveto{\pgfqpoint{-0.018041in}{-0.031064in}}{\pgfqpoint{-0.009208in}{-0.034722in}}{\pgfqpoint{0.000000in}{-0.034722in}}%
\pgfpathlineto{\pgfqpoint{0.000000in}{-0.034722in}}%
\pgfpathclose%
\pgfusepath{stroke,fill}%
}%
\begin{pgfscope}%
\pgfsys@transformshift{0.459867in}{0.459990in}%
\pgfsys@useobject{currentmarker}{}%
\end{pgfscope}%
\begin{pgfscope}%
\pgfsys@transformshift{0.469840in}{0.464451in}%
\pgfsys@useobject{currentmarker}{}%
\end{pgfscope}%
\begin{pgfscope}%
\pgfsys@transformshift{0.486460in}{0.483806in}%
\pgfsys@useobject{currentmarker}{}%
\end{pgfscope}%
\begin{pgfscope}%
\pgfsys@transformshift{0.509730in}{0.499347in}%
\pgfsys@useobject{currentmarker}{}%
\end{pgfscope}%
\begin{pgfscope}%
\pgfsys@transformshift{0.539647in}{0.505803in}%
\pgfsys@useobject{currentmarker}{}%
\end{pgfscope}%
\begin{pgfscope}%
\pgfsys@transformshift{0.576213in}{0.528751in}%
\pgfsys@useobject{currentmarker}{}%
\end{pgfscope}%
\begin{pgfscope}%
\pgfsys@transformshift{0.619427in}{0.540522in}%
\pgfsys@useobject{currentmarker}{}%
\end{pgfscope}%
\begin{pgfscope}%
\pgfsys@transformshift{0.669289in}{0.555402in}%
\pgfsys@useobject{currentmarker}{}%
\end{pgfscope}%
\begin{pgfscope}%
\pgfsys@transformshift{0.725800in}{0.629090in}%
\pgfsys@useobject{currentmarker}{}%
\end{pgfscope}%
\begin{pgfscope}%
\pgfsys@transformshift{0.788959in}{0.661441in}%
\pgfsys@useobject{currentmarker}{}%
\end{pgfscope}%
\begin{pgfscope}%
\pgfsys@transformshift{0.935222in}{0.778286in}%
\pgfsys@useobject{currentmarker}{}%
\end{pgfscope}%
\begin{pgfscope}%
\pgfsys@transformshift{1.108079in}{0.899725in}%
\pgfsys@useobject{currentmarker}{}%
\end{pgfscope}%
\begin{pgfscope}%
\pgfsys@transformshift{1.307529in}{0.912527in}%
\pgfsys@useobject{currentmarker}{}%
\end{pgfscope}%
\begin{pgfscope}%
\pgfsys@transformshift{1.533572in}{1.113681in}%
\pgfsys@useobject{currentmarker}{}%
\end{pgfscope}%
\begin{pgfscope}%
\pgfsys@transformshift{1.786208in}{1.388962in}%
\pgfsys@useobject{currentmarker}{}%
\end{pgfscope}%
\end{pgfscope}%
\begin{pgfscope}%
\pgfsetroundcap%
\pgfsetroundjoin%
\pgfsetlinewidth{1.003750pt}%
\definecolor{currentstroke}{rgb}{0.835294,0.368627,0.000000}%
\pgfsetstrokecolor{currentstroke}%
\pgfsetdash{}{0pt}%
\pgfpathmoveto{\pgfqpoint{0.459867in}{0.451953in}}%
\pgfpathlineto{\pgfqpoint{0.469840in}{0.452858in}}%
\pgfpathlineto{\pgfqpoint{0.486460in}{0.454231in}}%
\pgfpathlineto{\pgfqpoint{0.509730in}{0.457252in}}%
\pgfpathlineto{\pgfqpoint{0.539647in}{0.461215in}}%
\pgfpathlineto{\pgfqpoint{0.576213in}{0.466803in}}%
\pgfpathlineto{\pgfqpoint{0.619427in}{0.472755in}}%
\pgfpathlineto{\pgfqpoint{0.669289in}{0.479769in}}%
\pgfpathlineto{\pgfqpoint{0.725800in}{0.489506in}}%
\pgfpathlineto{\pgfqpoint{0.788959in}{0.509551in}}%
\pgfpathlineto{\pgfqpoint{0.935222in}{0.533842in}}%
\pgfpathlineto{\pgfqpoint{1.108079in}{0.570794in}}%
\pgfpathlineto{\pgfqpoint{1.307529in}{0.629016in}}%
\pgfpathlineto{\pgfqpoint{1.533572in}{0.653760in}}%
\pgfpathlineto{\pgfqpoint{1.786208in}{0.726773in}}%
\pgfusepath{stroke}%
\end{pgfscope}%
\begin{pgfscope}%
\pgfsetbuttcap%
\pgfsetroundjoin%
\definecolor{currentfill}{rgb}{0.835294,0.368627,0.000000}%
\pgfsetfillcolor{currentfill}%
\pgfsetlinewidth{0.752812pt}%
\definecolor{currentstroke}{rgb}{1.000000,1.000000,1.000000}%
\pgfsetstrokecolor{currentstroke}%
\pgfsetdash{}{0pt}%
\pgfsys@defobject{currentmarker}{\pgfqpoint{-0.034722in}{-0.034722in}}{\pgfqpoint{0.034722in}{0.034722in}}{%
\pgfpathmoveto{\pgfqpoint{0.000000in}{-0.034722in}}%
\pgfpathcurveto{\pgfqpoint{0.009208in}{-0.034722in}}{\pgfqpoint{0.018041in}{-0.031064in}}{\pgfqpoint{0.024552in}{-0.024552in}}%
\pgfpathcurveto{\pgfqpoint{0.031064in}{-0.018041in}}{\pgfqpoint{0.034722in}{-0.009208in}}{\pgfqpoint{0.034722in}{0.000000in}}%
\pgfpathcurveto{\pgfqpoint{0.034722in}{0.009208in}}{\pgfqpoint{0.031064in}{0.018041in}}{\pgfqpoint{0.024552in}{0.024552in}}%
\pgfpathcurveto{\pgfqpoint{0.018041in}{0.031064in}}{\pgfqpoint{0.009208in}{0.034722in}}{\pgfqpoint{0.000000in}{0.034722in}}%
\pgfpathcurveto{\pgfqpoint{-0.009208in}{0.034722in}}{\pgfqpoint{-0.018041in}{0.031064in}}{\pgfqpoint{-0.024552in}{0.024552in}}%
\pgfpathcurveto{\pgfqpoint{-0.031064in}{0.018041in}}{\pgfqpoint{-0.034722in}{0.009208in}}{\pgfqpoint{-0.034722in}{0.000000in}}%
\pgfpathcurveto{\pgfqpoint{-0.034722in}{-0.009208in}}{\pgfqpoint{-0.031064in}{-0.018041in}}{\pgfqpoint{-0.024552in}{-0.024552in}}%
\pgfpathcurveto{\pgfqpoint{-0.018041in}{-0.031064in}}{\pgfqpoint{-0.009208in}{-0.034722in}}{\pgfqpoint{0.000000in}{-0.034722in}}%
\pgfpathlineto{\pgfqpoint{0.000000in}{-0.034722in}}%
\pgfpathclose%
\pgfusepath{stroke,fill}%
}%
\begin{pgfscope}%
\pgfsys@transformshift{0.459867in}{0.451953in}%
\pgfsys@useobject{currentmarker}{}%
\end{pgfscope}%
\begin{pgfscope}%
\pgfsys@transformshift{0.469840in}{0.452858in}%
\pgfsys@useobject{currentmarker}{}%
\end{pgfscope}%
\begin{pgfscope}%
\pgfsys@transformshift{0.486460in}{0.454231in}%
\pgfsys@useobject{currentmarker}{}%
\end{pgfscope}%
\begin{pgfscope}%
\pgfsys@transformshift{0.509730in}{0.457252in}%
\pgfsys@useobject{currentmarker}{}%
\end{pgfscope}%
\begin{pgfscope}%
\pgfsys@transformshift{0.539647in}{0.461215in}%
\pgfsys@useobject{currentmarker}{}%
\end{pgfscope}%
\begin{pgfscope}%
\pgfsys@transformshift{0.576213in}{0.466803in}%
\pgfsys@useobject{currentmarker}{}%
\end{pgfscope}%
\begin{pgfscope}%
\pgfsys@transformshift{0.619427in}{0.472755in}%
\pgfsys@useobject{currentmarker}{}%
\end{pgfscope}%
\begin{pgfscope}%
\pgfsys@transformshift{0.669289in}{0.479769in}%
\pgfsys@useobject{currentmarker}{}%
\end{pgfscope}%
\begin{pgfscope}%
\pgfsys@transformshift{0.725800in}{0.489506in}%
\pgfsys@useobject{currentmarker}{}%
\end{pgfscope}%
\begin{pgfscope}%
\pgfsys@transformshift{0.788959in}{0.509551in}%
\pgfsys@useobject{currentmarker}{}%
\end{pgfscope}%
\begin{pgfscope}%
\pgfsys@transformshift{0.935222in}{0.533842in}%
\pgfsys@useobject{currentmarker}{}%
\end{pgfscope}%
\begin{pgfscope}%
\pgfsys@transformshift{1.108079in}{0.570794in}%
\pgfsys@useobject{currentmarker}{}%
\end{pgfscope}%
\begin{pgfscope}%
\pgfsys@transformshift{1.307529in}{0.629016in}%
\pgfsys@useobject{currentmarker}{}%
\end{pgfscope}%
\begin{pgfscope}%
\pgfsys@transformshift{1.533572in}{0.653760in}%
\pgfsys@useobject{currentmarker}{}%
\end{pgfscope}%
\begin{pgfscope}%
\pgfsys@transformshift{1.786208in}{0.726773in}%
\pgfsys@useobject{currentmarker}{}%
\end{pgfscope}%
\end{pgfscope}%
\begin{pgfscope}%
\pgfsetbuttcap%
\pgfsetmiterjoin%
\definecolor{currentfill}{rgb}{1.000000,1.000000,1.000000}%
\pgfsetfillcolor{currentfill}%
\pgfsetlinewidth{0.000000pt}%
\definecolor{currentstroke}{rgb}{0.000000,0.000000,0.000000}%
\pgfsetstrokecolor{currentstroke}%
\pgfsetstrokeopacity{0.000000}%
\pgfsetdash{}{0pt}%
\pgfpathmoveto{\pgfqpoint{2.258303in}{0.451389in}}%
\pgfpathlineto{\pgfqpoint{3.654285in}{0.451389in}}%
\pgfpathlineto{\pgfqpoint{3.654285in}{2.204860in}}%
\pgfpathlineto{\pgfqpoint{2.258303in}{2.204860in}}%
\pgfpathlineto{\pgfqpoint{2.258303in}{0.451389in}}%
\pgfpathclose%
\pgfusepath{fill}%
\end{pgfscope}%
\begin{pgfscope}%
\pgfpathrectangle{\pgfqpoint{2.258303in}{0.451389in}}{\pgfqpoint{1.395982in}{1.753471in}}%
\pgfusepath{clip}%
\pgfsetroundcap%
\pgfsetroundjoin%
\pgfsetlinewidth{1.003750pt}%
\definecolor{currentstroke}{rgb}{0.800000,0.800000,0.800000}%
\pgfsetstrokecolor{currentstroke}%
\pgfsetdash{}{0pt}%
\pgfpathmoveto{\pgfqpoint{2.258303in}{0.451389in}}%
\pgfpathlineto{\pgfqpoint{2.258303in}{2.204860in}}%
\pgfusepath{stroke}%
\end{pgfscope}%
\begin{pgfscope}%
\definecolor{textcolor}{rgb}{0.150000,0.150000,0.150000}%
\pgfsetstrokecolor{textcolor}%
\pgfsetfillcolor{textcolor}%
\pgftext[x=2.258303in,y=0.319444in,,top]{\color{textcolor}\sffamily\fontsize{9.000000}{10.800000}\selectfont 0k}%
\end{pgfscope}%
\begin{pgfscope}%
\pgfpathrectangle{\pgfqpoint{2.258303in}{0.451389in}}{\pgfqpoint{1.395982in}{1.753471in}}%
\pgfusepath{clip}%
\pgfsetroundcap%
\pgfsetroundjoin%
\pgfsetlinewidth{1.003750pt}%
\definecolor{currentstroke}{rgb}{0.800000,0.800000,0.800000}%
\pgfsetstrokecolor{currentstroke}%
\pgfsetdash{}{0pt}%
\pgfpathmoveto{\pgfqpoint{2.685777in}{0.451389in}}%
\pgfpathlineto{\pgfqpoint{2.685777in}{2.204860in}}%
\pgfusepath{stroke}%
\end{pgfscope}%
\begin{pgfscope}%
\definecolor{textcolor}{rgb}{0.150000,0.150000,0.150000}%
\pgfsetstrokecolor{textcolor}%
\pgfsetfillcolor{textcolor}%
\pgftext[x=2.685777in,y=0.319444in,,top]{\color{textcolor}\sffamily\fontsize{9.000000}{10.800000}\selectfont 10k}%
\end{pgfscope}%
\begin{pgfscope}%
\pgfpathrectangle{\pgfqpoint{2.258303in}{0.451389in}}{\pgfqpoint{1.395982in}{1.753471in}}%
\pgfusepath{clip}%
\pgfsetroundcap%
\pgfsetroundjoin%
\pgfsetlinewidth{1.003750pt}%
\definecolor{currentstroke}{rgb}{0.800000,0.800000,0.800000}%
\pgfsetstrokecolor{currentstroke}%
\pgfsetdash{}{0pt}%
\pgfpathmoveto{\pgfqpoint{3.113252in}{0.451389in}}%
\pgfpathlineto{\pgfqpoint{3.113252in}{2.204860in}}%
\pgfusepath{stroke}%
\end{pgfscope}%
\begin{pgfscope}%
\definecolor{textcolor}{rgb}{0.150000,0.150000,0.150000}%
\pgfsetstrokecolor{textcolor}%
\pgfsetfillcolor{textcolor}%
\pgftext[x=3.113252in,y=0.319444in,,top]{\color{textcolor}\sffamily\fontsize{9.000000}{10.800000}\selectfont 20k}%
\end{pgfscope}%
\begin{pgfscope}%
\pgfpathrectangle{\pgfqpoint{2.258303in}{0.451389in}}{\pgfqpoint{1.395982in}{1.753471in}}%
\pgfusepath{clip}%
\pgfsetroundcap%
\pgfsetroundjoin%
\pgfsetlinewidth{1.003750pt}%
\definecolor{currentstroke}{rgb}{0.800000,0.800000,0.800000}%
\pgfsetstrokecolor{currentstroke}%
\pgfsetdash{}{0pt}%
\pgfpathmoveto{\pgfqpoint{3.540726in}{0.451389in}}%
\pgfpathlineto{\pgfqpoint{3.540726in}{2.204860in}}%
\pgfusepath{stroke}%
\end{pgfscope}%
\begin{pgfscope}%
\definecolor{textcolor}{rgb}{0.150000,0.150000,0.150000}%
\pgfsetstrokecolor{textcolor}%
\pgfsetfillcolor{textcolor}%
\pgftext[x=3.540726in,y=0.319444in,,top]{\color{textcolor}\sffamily\fontsize{9.000000}{10.800000}\selectfont 30k}%
\end{pgfscope}%
\begin{pgfscope}%
\definecolor{textcolor}{rgb}{0.150000,0.150000,0.150000}%
\pgfsetstrokecolor{textcolor}%
\pgfsetfillcolor{textcolor}%
\pgftext[x=2.956294in,y=0.125000in,,top]{\color{textcolor}\sffamily\fontsize{9.000000}{10.800000}\selectfont Input obstacle vertices}%
\end{pgfscope}%
\begin{pgfscope}%
\pgfpathrectangle{\pgfqpoint{2.258303in}{0.451389in}}{\pgfqpoint{1.395982in}{1.753471in}}%
\pgfusepath{clip}%
\pgfsetroundcap%
\pgfsetroundjoin%
\pgfsetlinewidth{1.003750pt}%
\definecolor{currentstroke}{rgb}{0.800000,0.800000,0.800000}%
\pgfsetstrokecolor{currentstroke}%
\pgfsetdash{}{0pt}%
\pgfpathmoveto{\pgfqpoint{2.258303in}{0.451389in}}%
\pgfpathlineto{\pgfqpoint{3.654285in}{0.451389in}}%
\pgfusepath{stroke}%
\end{pgfscope}%
\begin{pgfscope}%
\definecolor{textcolor}{rgb}{0.150000,0.150000,0.150000}%
\pgfsetstrokecolor{textcolor}%
\pgfsetfillcolor{textcolor}%
\pgftext[x=2.133899in, y=0.403903in, left, base]{\color{textcolor}\sffamily\fontsize{9.000000}{10.800000}\selectfont 0}%
\end{pgfscope}%
\begin{pgfscope}%
\pgfpathrectangle{\pgfqpoint{2.258303in}{0.451389in}}{\pgfqpoint{1.395982in}{1.753471in}}%
\pgfusepath{clip}%
\pgfsetroundcap%
\pgfsetroundjoin%
\pgfsetlinewidth{1.003750pt}%
\definecolor{currentstroke}{rgb}{0.800000,0.800000,0.800000}%
\pgfsetstrokecolor{currentstroke}%
\pgfsetdash{}{0pt}%
\pgfpathmoveto{\pgfqpoint{2.258303in}{0.837681in}}%
\pgfpathlineto{\pgfqpoint{3.654285in}{0.837681in}}%
\pgfusepath{stroke}%
\end{pgfscope}%
\begin{pgfscope}%
\definecolor{textcolor}{rgb}{0.150000,0.150000,0.150000}%
\pgfsetstrokecolor{textcolor}%
\pgfsetfillcolor{textcolor}%
\pgftext[x=2.065052in, y=0.790196in, left, base]{\color{textcolor}\sffamily\fontsize{9.000000}{10.800000}\selectfont 50}%
\end{pgfscope}%
\begin{pgfscope}%
\pgfpathrectangle{\pgfqpoint{2.258303in}{0.451389in}}{\pgfqpoint{1.395982in}{1.753471in}}%
\pgfusepath{clip}%
\pgfsetroundcap%
\pgfsetroundjoin%
\pgfsetlinewidth{1.003750pt}%
\definecolor{currentstroke}{rgb}{0.800000,0.800000,0.800000}%
\pgfsetstrokecolor{currentstroke}%
\pgfsetdash{}{0pt}%
\pgfpathmoveto{\pgfqpoint{2.258303in}{1.223973in}}%
\pgfpathlineto{\pgfqpoint{3.654285in}{1.223973in}}%
\pgfusepath{stroke}%
\end{pgfscope}%
\begin{pgfscope}%
\definecolor{textcolor}{rgb}{0.150000,0.150000,0.150000}%
\pgfsetstrokecolor{textcolor}%
\pgfsetfillcolor{textcolor}%
\pgftext[x=1.996204in, y=1.176488in, left, base]{\color{textcolor}\sffamily\fontsize{9.000000}{10.800000}\selectfont 100}%
\end{pgfscope}%
\begin{pgfscope}%
\pgfpathrectangle{\pgfqpoint{2.258303in}{0.451389in}}{\pgfqpoint{1.395982in}{1.753471in}}%
\pgfusepath{clip}%
\pgfsetroundcap%
\pgfsetroundjoin%
\pgfsetlinewidth{1.003750pt}%
\definecolor{currentstroke}{rgb}{0.800000,0.800000,0.800000}%
\pgfsetstrokecolor{currentstroke}%
\pgfsetdash{}{0pt}%
\pgfpathmoveto{\pgfqpoint{2.258303in}{1.610266in}}%
\pgfpathlineto{\pgfqpoint{3.654285in}{1.610266in}}%
\pgfusepath{stroke}%
\end{pgfscope}%
\begin{pgfscope}%
\definecolor{textcolor}{rgb}{0.150000,0.150000,0.150000}%
\pgfsetstrokecolor{textcolor}%
\pgfsetfillcolor{textcolor}%
\pgftext[x=1.996204in, y=1.562781in, left, base]{\color{textcolor}\sffamily\fontsize{9.000000}{10.800000}\selectfont 150}%
\end{pgfscope}%
\begin{pgfscope}%
\pgfpathrectangle{\pgfqpoint{2.258303in}{0.451389in}}{\pgfqpoint{1.395982in}{1.753471in}}%
\pgfusepath{clip}%
\pgfsetroundcap%
\pgfsetroundjoin%
\pgfsetlinewidth{1.003750pt}%
\definecolor{currentstroke}{rgb}{0.800000,0.800000,0.800000}%
\pgfsetstrokecolor{currentstroke}%
\pgfsetdash{}{0pt}%
\pgfpathmoveto{\pgfqpoint{2.258303in}{1.996558in}}%
\pgfpathlineto{\pgfqpoint{3.654285in}{1.996558in}}%
\pgfusepath{stroke}%
\end{pgfscope}%
\begin{pgfscope}%
\definecolor{textcolor}{rgb}{0.150000,0.150000,0.150000}%
\pgfsetstrokecolor{textcolor}%
\pgfsetfillcolor{textcolor}%
\pgftext[x=1.996204in, y=1.949073in, left, base]{\color{textcolor}\sffamily\fontsize{9.000000}{10.800000}\selectfont 200}%
\end{pgfscope}%
\begin{pgfscope}%
\pgfsetrectcap%
\pgfsetmiterjoin%
\pgfsetlinewidth{1.254687pt}%
\definecolor{currentstroke}{rgb}{0.800000,0.800000,0.800000}%
\pgfsetstrokecolor{currentstroke}%
\pgfsetdash{}{0pt}%
\pgfpathmoveto{\pgfqpoint{2.258303in}{0.451389in}}%
\pgfpathlineto{\pgfqpoint{2.258303in}{2.204860in}}%
\pgfusepath{stroke}%
\end{pgfscope}%
\begin{pgfscope}%
\pgfsetrectcap%
\pgfsetmiterjoin%
\pgfsetlinewidth{1.254687pt}%
\definecolor{currentstroke}{rgb}{0.800000,0.800000,0.800000}%
\pgfsetstrokecolor{currentstroke}%
\pgfsetdash{}{0pt}%
\pgfpathmoveto{\pgfqpoint{3.654285in}{0.451389in}}%
\pgfpathlineto{\pgfqpoint{3.654285in}{2.204860in}}%
\pgfusepath{stroke}%
\end{pgfscope}%
\begin{pgfscope}%
\pgfsetrectcap%
\pgfsetmiterjoin%
\pgfsetlinewidth{1.254687pt}%
\definecolor{currentstroke}{rgb}{0.800000,0.800000,0.800000}%
\pgfsetstrokecolor{currentstroke}%
\pgfsetdash{}{0pt}%
\pgfpathmoveto{\pgfqpoint{2.258303in}{0.451389in}}%
\pgfpathlineto{\pgfqpoint{3.654285in}{0.451389in}}%
\pgfusepath{stroke}%
\end{pgfscope}%
\begin{pgfscope}%
\pgfsetrectcap%
\pgfsetmiterjoin%
\pgfsetlinewidth{1.254687pt}%
\definecolor{currentstroke}{rgb}{0.800000,0.800000,0.800000}%
\pgfsetstrokecolor{currentstroke}%
\pgfsetdash{}{0pt}%
\pgfpathmoveto{\pgfqpoint{2.258303in}{2.204860in}}%
\pgfpathlineto{\pgfqpoint{3.654285in}{2.204860in}}%
\pgfusepath{stroke}%
\end{pgfscope}%
\begin{pgfscope}%
\definecolor{textcolor}{rgb}{0.150000,0.150000,0.150000}%
\pgfsetstrokecolor{textcolor}%
\pgfsetfillcolor{textcolor}%
\pgftext[x=2.956294in,y=2.315971in,,base]{\color{textcolor}\sffamily\fontsize{9.000000}{10.800000}\selectfont Maze}%
\end{pgfscope}%
\begin{pgfscope}%
\pgfsetroundcap%
\pgfsetroundjoin%
\pgfsetlinewidth{1.003750pt}%
\definecolor{currentstroke}{rgb}{0.003922,0.450980,0.698039}%
\pgfsetstrokecolor{currentstroke}%
\pgfsetdash{}{0pt}%
\pgfpathmoveto{\pgfqpoint{2.260611in}{0.458790in}}%
\pgfpathlineto{\pgfqpoint{2.267536in}{0.472998in}}%
\pgfpathlineto{\pgfqpoint{2.295236in}{0.486464in}}%
\pgfpathlineto{\pgfqpoint{2.341404in}{0.531761in}}%
\pgfpathlineto{\pgfqpoint{2.406038in}{0.593173in}}%
\pgfpathlineto{\pgfqpoint{2.489139in}{0.652199in}}%
\pgfpathlineto{\pgfqpoint{2.590707in}{0.790654in}}%
\pgfpathlineto{\pgfqpoint{2.710742in}{0.979736in}}%
\pgfpathlineto{\pgfqpoint{2.849243in}{1.066034in}}%
\pgfpathlineto{\pgfqpoint{3.006212in}{1.404527in}}%
\pgfpathlineto{\pgfqpoint{3.181648in}{1.380484in}}%
\pgfpathlineto{\pgfqpoint{3.375550in}{1.802253in}}%
\pgfpathlineto{\pgfqpoint{3.587919in}{2.121385in}}%
\pgfusepath{stroke}%
\end{pgfscope}%
\begin{pgfscope}%
\pgfsetbuttcap%
\pgfsetroundjoin%
\definecolor{currentfill}{rgb}{0.003922,0.450980,0.698039}%
\pgfsetfillcolor{currentfill}%
\pgfsetlinewidth{0.752812pt}%
\definecolor{currentstroke}{rgb}{1.000000,1.000000,1.000000}%
\pgfsetstrokecolor{currentstroke}%
\pgfsetdash{}{0pt}%
\pgfsys@defobject{currentmarker}{\pgfqpoint{-0.034722in}{-0.034722in}}{\pgfqpoint{0.034722in}{0.034722in}}{%
\pgfpathmoveto{\pgfqpoint{0.000000in}{-0.034722in}}%
\pgfpathcurveto{\pgfqpoint{0.009208in}{-0.034722in}}{\pgfqpoint{0.018041in}{-0.031064in}}{\pgfqpoint{0.024552in}{-0.024552in}}%
\pgfpathcurveto{\pgfqpoint{0.031064in}{-0.018041in}}{\pgfqpoint{0.034722in}{-0.009208in}}{\pgfqpoint{0.034722in}{0.000000in}}%
\pgfpathcurveto{\pgfqpoint{0.034722in}{0.009208in}}{\pgfqpoint{0.031064in}{0.018041in}}{\pgfqpoint{0.024552in}{0.024552in}}%
\pgfpathcurveto{\pgfqpoint{0.018041in}{0.031064in}}{\pgfqpoint{0.009208in}{0.034722in}}{\pgfqpoint{0.000000in}{0.034722in}}%
\pgfpathcurveto{\pgfqpoint{-0.009208in}{0.034722in}}{\pgfqpoint{-0.018041in}{0.031064in}}{\pgfqpoint{-0.024552in}{0.024552in}}%
\pgfpathcurveto{\pgfqpoint{-0.031064in}{0.018041in}}{\pgfqpoint{-0.034722in}{0.009208in}}{\pgfqpoint{-0.034722in}{0.000000in}}%
\pgfpathcurveto{\pgfqpoint{-0.034722in}{-0.009208in}}{\pgfqpoint{-0.031064in}{-0.018041in}}{\pgfqpoint{-0.024552in}{-0.024552in}}%
\pgfpathcurveto{\pgfqpoint{-0.018041in}{-0.031064in}}{\pgfqpoint{-0.009208in}{-0.034722in}}{\pgfqpoint{0.000000in}{-0.034722in}}%
\pgfpathlineto{\pgfqpoint{0.000000in}{-0.034722in}}%
\pgfpathclose%
\pgfusepath{stroke,fill}%
}%
\begin{pgfscope}%
\pgfsys@transformshift{2.260611in}{0.458790in}%
\pgfsys@useobject{currentmarker}{}%
\end{pgfscope}%
\begin{pgfscope}%
\pgfsys@transformshift{2.267536in}{0.472998in}%
\pgfsys@useobject{currentmarker}{}%
\end{pgfscope}%
\begin{pgfscope}%
\pgfsys@transformshift{2.295236in}{0.486464in}%
\pgfsys@useobject{currentmarker}{}%
\end{pgfscope}%
\begin{pgfscope}%
\pgfsys@transformshift{2.341404in}{0.531761in}%
\pgfsys@useobject{currentmarker}{}%
\end{pgfscope}%
\begin{pgfscope}%
\pgfsys@transformshift{2.406038in}{0.593173in}%
\pgfsys@useobject{currentmarker}{}%
\end{pgfscope}%
\begin{pgfscope}%
\pgfsys@transformshift{2.489139in}{0.652199in}%
\pgfsys@useobject{currentmarker}{}%
\end{pgfscope}%
\begin{pgfscope}%
\pgfsys@transformshift{2.590707in}{0.790654in}%
\pgfsys@useobject{currentmarker}{}%
\end{pgfscope}%
\begin{pgfscope}%
\pgfsys@transformshift{2.710742in}{0.979736in}%
\pgfsys@useobject{currentmarker}{}%
\end{pgfscope}%
\begin{pgfscope}%
\pgfsys@transformshift{2.849243in}{1.066034in}%
\pgfsys@useobject{currentmarker}{}%
\end{pgfscope}%
\begin{pgfscope}%
\pgfsys@transformshift{3.006212in}{1.404527in}%
\pgfsys@useobject{currentmarker}{}%
\end{pgfscope}%
\begin{pgfscope}%
\pgfsys@transformshift{3.181648in}{1.380484in}%
\pgfsys@useobject{currentmarker}{}%
\end{pgfscope}%
\begin{pgfscope}%
\pgfsys@transformshift{3.375550in}{1.802253in}%
\pgfsys@useobject{currentmarker}{}%
\end{pgfscope}%
\begin{pgfscope}%
\pgfsys@transformshift{3.587919in}{2.121385in}%
\pgfsys@useobject{currentmarker}{}%
\end{pgfscope}%
\end{pgfscope}%
\begin{pgfscope}%
\pgfsetroundcap%
\pgfsetroundjoin%
\pgfsetlinewidth{1.003750pt}%
\definecolor{currentstroke}{rgb}{0.870588,0.560784,0.019608}%
\pgfsetstrokecolor{currentstroke}%
\pgfsetdash{}{0pt}%
\pgfpathmoveto{\pgfqpoint{2.260611in}{0.451883in}}%
\pgfpathlineto{\pgfqpoint{2.267536in}{0.455669in}}%
\pgfpathlineto{\pgfqpoint{2.295236in}{0.459223in}}%
\pgfpathlineto{\pgfqpoint{2.341404in}{0.473886in}}%
\pgfpathlineto{\pgfqpoint{2.406038in}{0.488032in}}%
\pgfpathlineto{\pgfqpoint{2.489139in}{0.513103in}}%
\pgfpathlineto{\pgfqpoint{2.590707in}{0.548008in}}%
\pgfpathlineto{\pgfqpoint{2.710742in}{0.585285in}}%
\pgfpathlineto{\pgfqpoint{2.849243in}{0.614443in}}%
\pgfpathlineto{\pgfqpoint{3.006212in}{0.688240in}}%
\pgfpathlineto{\pgfqpoint{3.181648in}{0.718286in}}%
\pgfpathlineto{\pgfqpoint{3.375550in}{0.798835in}}%
\pgfpathlineto{\pgfqpoint{3.587919in}{0.871149in}}%
\pgfusepath{stroke}%
\end{pgfscope}%
\begin{pgfscope}%
\pgfsetbuttcap%
\pgfsetroundjoin%
\definecolor{currentfill}{rgb}{0.870588,0.560784,0.019608}%
\pgfsetfillcolor{currentfill}%
\pgfsetlinewidth{0.752812pt}%
\definecolor{currentstroke}{rgb}{1.000000,1.000000,1.000000}%
\pgfsetstrokecolor{currentstroke}%
\pgfsetdash{}{0pt}%
\pgfsys@defobject{currentmarker}{\pgfqpoint{-0.034722in}{-0.034722in}}{\pgfqpoint{0.034722in}{0.034722in}}{%
\pgfpathmoveto{\pgfqpoint{0.000000in}{-0.034722in}}%
\pgfpathcurveto{\pgfqpoint{0.009208in}{-0.034722in}}{\pgfqpoint{0.018041in}{-0.031064in}}{\pgfqpoint{0.024552in}{-0.024552in}}%
\pgfpathcurveto{\pgfqpoint{0.031064in}{-0.018041in}}{\pgfqpoint{0.034722in}{-0.009208in}}{\pgfqpoint{0.034722in}{0.000000in}}%
\pgfpathcurveto{\pgfqpoint{0.034722in}{0.009208in}}{\pgfqpoint{0.031064in}{0.018041in}}{\pgfqpoint{0.024552in}{0.024552in}}%
\pgfpathcurveto{\pgfqpoint{0.018041in}{0.031064in}}{\pgfqpoint{0.009208in}{0.034722in}}{\pgfqpoint{0.000000in}{0.034722in}}%
\pgfpathcurveto{\pgfqpoint{-0.009208in}{0.034722in}}{\pgfqpoint{-0.018041in}{0.031064in}}{\pgfqpoint{-0.024552in}{0.024552in}}%
\pgfpathcurveto{\pgfqpoint{-0.031064in}{0.018041in}}{\pgfqpoint{-0.034722in}{0.009208in}}{\pgfqpoint{-0.034722in}{0.000000in}}%
\pgfpathcurveto{\pgfqpoint{-0.034722in}{-0.009208in}}{\pgfqpoint{-0.031064in}{-0.018041in}}{\pgfqpoint{-0.024552in}{-0.024552in}}%
\pgfpathcurveto{\pgfqpoint{-0.018041in}{-0.031064in}}{\pgfqpoint{-0.009208in}{-0.034722in}}{\pgfqpoint{0.000000in}{-0.034722in}}%
\pgfpathlineto{\pgfqpoint{0.000000in}{-0.034722in}}%
\pgfpathclose%
\pgfusepath{stroke,fill}%
}%
\begin{pgfscope}%
\pgfsys@transformshift{2.260611in}{0.451883in}%
\pgfsys@useobject{currentmarker}{}%
\end{pgfscope}%
\begin{pgfscope}%
\pgfsys@transformshift{2.267536in}{0.455669in}%
\pgfsys@useobject{currentmarker}{}%
\end{pgfscope}%
\begin{pgfscope}%
\pgfsys@transformshift{2.295236in}{0.459223in}%
\pgfsys@useobject{currentmarker}{}%
\end{pgfscope}%
\begin{pgfscope}%
\pgfsys@transformshift{2.341404in}{0.473886in}%
\pgfsys@useobject{currentmarker}{}%
\end{pgfscope}%
\begin{pgfscope}%
\pgfsys@transformshift{2.406038in}{0.488032in}%
\pgfsys@useobject{currentmarker}{}%
\end{pgfscope}%
\begin{pgfscope}%
\pgfsys@transformshift{2.489139in}{0.513103in}%
\pgfsys@useobject{currentmarker}{}%
\end{pgfscope}%
\begin{pgfscope}%
\pgfsys@transformshift{2.590707in}{0.548008in}%
\pgfsys@useobject{currentmarker}{}%
\end{pgfscope}%
\begin{pgfscope}%
\pgfsys@transformshift{2.710742in}{0.585285in}%
\pgfsys@useobject{currentmarker}{}%
\end{pgfscope}%
\begin{pgfscope}%
\pgfsys@transformshift{2.849243in}{0.614443in}%
\pgfsys@useobject{currentmarker}{}%
\end{pgfscope}%
\begin{pgfscope}%
\pgfsys@transformshift{3.006212in}{0.688240in}%
\pgfsys@useobject{currentmarker}{}%
\end{pgfscope}%
\begin{pgfscope}%
\pgfsys@transformshift{3.181648in}{0.718286in}%
\pgfsys@useobject{currentmarker}{}%
\end{pgfscope}%
\begin{pgfscope}%
\pgfsys@transformshift{3.375550in}{0.798835in}%
\pgfsys@useobject{currentmarker}{}%
\end{pgfscope}%
\begin{pgfscope}%
\pgfsys@transformshift{3.587919in}{0.871149in}%
\pgfsys@useobject{currentmarker}{}%
\end{pgfscope}%
\end{pgfscope}%
\begin{pgfscope}%
\pgfsetroundcap%
\pgfsetroundjoin%
\pgfsetlinewidth{1.003750pt}%
\definecolor{currentstroke}{rgb}{0.007843,0.619608,0.450980}%
\pgfsetstrokecolor{currentstroke}%
\pgfsetdash{}{0pt}%
\pgfpathmoveto{\pgfqpoint{2.260611in}{0.457631in}}%
\pgfpathlineto{\pgfqpoint{2.267536in}{0.467227in}}%
\pgfpathlineto{\pgfqpoint{2.295236in}{0.474280in}}%
\pgfpathlineto{\pgfqpoint{2.341404in}{0.498910in}}%
\pgfpathlineto{\pgfqpoint{2.406038in}{0.534094in}}%
\pgfpathlineto{\pgfqpoint{2.489139in}{0.556089in}}%
\pgfpathlineto{\pgfqpoint{2.590707in}{0.636461in}}%
\pgfpathlineto{\pgfqpoint{2.710742in}{0.750881in}}%
\pgfpathlineto{\pgfqpoint{2.849243in}{0.783878in}}%
\pgfpathlineto{\pgfqpoint{3.006212in}{0.981382in}}%
\pgfpathlineto{\pgfqpoint{3.181648in}{0.932701in}}%
\pgfpathlineto{\pgfqpoint{3.375550in}{1.177974in}}%
\pgfpathlineto{\pgfqpoint{3.587919in}{1.362900in}}%
\pgfusepath{stroke}%
\end{pgfscope}%
\begin{pgfscope}%
\pgfsetbuttcap%
\pgfsetroundjoin%
\definecolor{currentfill}{rgb}{0.007843,0.619608,0.450980}%
\pgfsetfillcolor{currentfill}%
\pgfsetlinewidth{0.752812pt}%
\definecolor{currentstroke}{rgb}{1.000000,1.000000,1.000000}%
\pgfsetstrokecolor{currentstroke}%
\pgfsetdash{}{0pt}%
\pgfsys@defobject{currentmarker}{\pgfqpoint{-0.034722in}{-0.034722in}}{\pgfqpoint{0.034722in}{0.034722in}}{%
\pgfpathmoveto{\pgfqpoint{0.000000in}{-0.034722in}}%
\pgfpathcurveto{\pgfqpoint{0.009208in}{-0.034722in}}{\pgfqpoint{0.018041in}{-0.031064in}}{\pgfqpoint{0.024552in}{-0.024552in}}%
\pgfpathcurveto{\pgfqpoint{0.031064in}{-0.018041in}}{\pgfqpoint{0.034722in}{-0.009208in}}{\pgfqpoint{0.034722in}{0.000000in}}%
\pgfpathcurveto{\pgfqpoint{0.034722in}{0.009208in}}{\pgfqpoint{0.031064in}{0.018041in}}{\pgfqpoint{0.024552in}{0.024552in}}%
\pgfpathcurveto{\pgfqpoint{0.018041in}{0.031064in}}{\pgfqpoint{0.009208in}{0.034722in}}{\pgfqpoint{0.000000in}{0.034722in}}%
\pgfpathcurveto{\pgfqpoint{-0.009208in}{0.034722in}}{\pgfqpoint{-0.018041in}{0.031064in}}{\pgfqpoint{-0.024552in}{0.024552in}}%
\pgfpathcurveto{\pgfqpoint{-0.031064in}{0.018041in}}{\pgfqpoint{-0.034722in}{0.009208in}}{\pgfqpoint{-0.034722in}{0.000000in}}%
\pgfpathcurveto{\pgfqpoint{-0.034722in}{-0.009208in}}{\pgfqpoint{-0.031064in}{-0.018041in}}{\pgfqpoint{-0.024552in}{-0.024552in}}%
\pgfpathcurveto{\pgfqpoint{-0.018041in}{-0.031064in}}{\pgfqpoint{-0.009208in}{-0.034722in}}{\pgfqpoint{0.000000in}{-0.034722in}}%
\pgfpathlineto{\pgfqpoint{0.000000in}{-0.034722in}}%
\pgfpathclose%
\pgfusepath{stroke,fill}%
}%
\begin{pgfscope}%
\pgfsys@transformshift{2.260611in}{0.457631in}%
\pgfsys@useobject{currentmarker}{}%
\end{pgfscope}%
\begin{pgfscope}%
\pgfsys@transformshift{2.267536in}{0.467227in}%
\pgfsys@useobject{currentmarker}{}%
\end{pgfscope}%
\begin{pgfscope}%
\pgfsys@transformshift{2.295236in}{0.474280in}%
\pgfsys@useobject{currentmarker}{}%
\end{pgfscope}%
\begin{pgfscope}%
\pgfsys@transformshift{2.341404in}{0.498910in}%
\pgfsys@useobject{currentmarker}{}%
\end{pgfscope}%
\begin{pgfscope}%
\pgfsys@transformshift{2.406038in}{0.534094in}%
\pgfsys@useobject{currentmarker}{}%
\end{pgfscope}%
\begin{pgfscope}%
\pgfsys@transformshift{2.489139in}{0.556089in}%
\pgfsys@useobject{currentmarker}{}%
\end{pgfscope}%
\begin{pgfscope}%
\pgfsys@transformshift{2.590707in}{0.636461in}%
\pgfsys@useobject{currentmarker}{}%
\end{pgfscope}%
\begin{pgfscope}%
\pgfsys@transformshift{2.710742in}{0.750881in}%
\pgfsys@useobject{currentmarker}{}%
\end{pgfscope}%
\begin{pgfscope}%
\pgfsys@transformshift{2.849243in}{0.783878in}%
\pgfsys@useobject{currentmarker}{}%
\end{pgfscope}%
\begin{pgfscope}%
\pgfsys@transformshift{3.006212in}{0.981382in}%
\pgfsys@useobject{currentmarker}{}%
\end{pgfscope}%
\begin{pgfscope}%
\pgfsys@transformshift{3.181648in}{0.932701in}%
\pgfsys@useobject{currentmarker}{}%
\end{pgfscope}%
\begin{pgfscope}%
\pgfsys@transformshift{3.375550in}{1.177974in}%
\pgfsys@useobject{currentmarker}{}%
\end{pgfscope}%
\begin{pgfscope}%
\pgfsys@transformshift{3.587919in}{1.362900in}%
\pgfsys@useobject{currentmarker}{}%
\end{pgfscope}%
\end{pgfscope}%
\begin{pgfscope}%
\pgfsetroundcap%
\pgfsetroundjoin%
\pgfsetlinewidth{1.003750pt}%
\definecolor{currentstroke}{rgb}{0.835294,0.368627,0.000000}%
\pgfsetstrokecolor{currentstroke}%
\pgfsetdash{}{0pt}%
\pgfpathmoveto{\pgfqpoint{2.260611in}{0.451906in}}%
\pgfpathlineto{\pgfqpoint{2.267536in}{0.452663in}}%
\pgfpathlineto{\pgfqpoint{2.295236in}{0.455584in}}%
\pgfpathlineto{\pgfqpoint{2.341404in}{0.461509in}}%
\pgfpathlineto{\pgfqpoint{2.406038in}{0.473562in}}%
\pgfpathlineto{\pgfqpoint{2.489139in}{0.485460in}}%
\pgfpathlineto{\pgfqpoint{2.590707in}{0.508614in}}%
\pgfpathlineto{\pgfqpoint{2.710742in}{0.545953in}}%
\pgfpathlineto{\pgfqpoint{2.849243in}{0.570027in}}%
\pgfpathlineto{\pgfqpoint{3.006212in}{0.637188in}}%
\pgfpathlineto{\pgfqpoint{3.181648in}{0.631702in}}%
\pgfpathlineto{\pgfqpoint{3.375550in}{0.727657in}}%
\pgfpathlineto{\pgfqpoint{3.587919in}{0.789580in}}%
\pgfusepath{stroke}%
\end{pgfscope}%
\begin{pgfscope}%
\pgfsetbuttcap%
\pgfsetroundjoin%
\definecolor{currentfill}{rgb}{0.835294,0.368627,0.000000}%
\pgfsetfillcolor{currentfill}%
\pgfsetlinewidth{0.752812pt}%
\definecolor{currentstroke}{rgb}{1.000000,1.000000,1.000000}%
\pgfsetstrokecolor{currentstroke}%
\pgfsetdash{}{0pt}%
\pgfsys@defobject{currentmarker}{\pgfqpoint{-0.034722in}{-0.034722in}}{\pgfqpoint{0.034722in}{0.034722in}}{%
\pgfpathmoveto{\pgfqpoint{0.000000in}{-0.034722in}}%
\pgfpathcurveto{\pgfqpoint{0.009208in}{-0.034722in}}{\pgfqpoint{0.018041in}{-0.031064in}}{\pgfqpoint{0.024552in}{-0.024552in}}%
\pgfpathcurveto{\pgfqpoint{0.031064in}{-0.018041in}}{\pgfqpoint{0.034722in}{-0.009208in}}{\pgfqpoint{0.034722in}{0.000000in}}%
\pgfpathcurveto{\pgfqpoint{0.034722in}{0.009208in}}{\pgfqpoint{0.031064in}{0.018041in}}{\pgfqpoint{0.024552in}{0.024552in}}%
\pgfpathcurveto{\pgfqpoint{0.018041in}{0.031064in}}{\pgfqpoint{0.009208in}{0.034722in}}{\pgfqpoint{0.000000in}{0.034722in}}%
\pgfpathcurveto{\pgfqpoint{-0.009208in}{0.034722in}}{\pgfqpoint{-0.018041in}{0.031064in}}{\pgfqpoint{-0.024552in}{0.024552in}}%
\pgfpathcurveto{\pgfqpoint{-0.031064in}{0.018041in}}{\pgfqpoint{-0.034722in}{0.009208in}}{\pgfqpoint{-0.034722in}{0.000000in}}%
\pgfpathcurveto{\pgfqpoint{-0.034722in}{-0.009208in}}{\pgfqpoint{-0.031064in}{-0.018041in}}{\pgfqpoint{-0.024552in}{-0.024552in}}%
\pgfpathcurveto{\pgfqpoint{-0.018041in}{-0.031064in}}{\pgfqpoint{-0.009208in}{-0.034722in}}{\pgfqpoint{0.000000in}{-0.034722in}}%
\pgfpathlineto{\pgfqpoint{0.000000in}{-0.034722in}}%
\pgfpathclose%
\pgfusepath{stroke,fill}%
}%
\begin{pgfscope}%
\pgfsys@transformshift{2.260611in}{0.451906in}%
\pgfsys@useobject{currentmarker}{}%
\end{pgfscope}%
\begin{pgfscope}%
\pgfsys@transformshift{2.267536in}{0.452663in}%
\pgfsys@useobject{currentmarker}{}%
\end{pgfscope}%
\begin{pgfscope}%
\pgfsys@transformshift{2.295236in}{0.455584in}%
\pgfsys@useobject{currentmarker}{}%
\end{pgfscope}%
\begin{pgfscope}%
\pgfsys@transformshift{2.341404in}{0.461509in}%
\pgfsys@useobject{currentmarker}{}%
\end{pgfscope}%
\begin{pgfscope}%
\pgfsys@transformshift{2.406038in}{0.473562in}%
\pgfsys@useobject{currentmarker}{}%
\end{pgfscope}%
\begin{pgfscope}%
\pgfsys@transformshift{2.489139in}{0.485460in}%
\pgfsys@useobject{currentmarker}{}%
\end{pgfscope}%
\begin{pgfscope}%
\pgfsys@transformshift{2.590707in}{0.508614in}%
\pgfsys@useobject{currentmarker}{}%
\end{pgfscope}%
\begin{pgfscope}%
\pgfsys@transformshift{2.710742in}{0.545953in}%
\pgfsys@useobject{currentmarker}{}%
\end{pgfscope}%
\begin{pgfscope}%
\pgfsys@transformshift{2.849243in}{0.570027in}%
\pgfsys@useobject{currentmarker}{}%
\end{pgfscope}%
\begin{pgfscope}%
\pgfsys@transformshift{3.006212in}{0.637188in}%
\pgfsys@useobject{currentmarker}{}%
\end{pgfscope}%
\begin{pgfscope}%
\pgfsys@transformshift{3.181648in}{0.631702in}%
\pgfsys@useobject{currentmarker}{}%
\end{pgfscope}%
\begin{pgfscope}%
\pgfsys@transformshift{3.375550in}{0.727657in}%
\pgfsys@useobject{currentmarker}{}%
\end{pgfscope}%
\begin{pgfscope}%
\pgfsys@transformshift{3.587919in}{0.789580in}%
\pgfsys@useobject{currentmarker}{}%
\end{pgfscope}%
\end{pgfscope}%
\begin{pgfscope}%
\pgfsetbuttcap%
\pgfsetmiterjoin%
\definecolor{currentfill}{rgb}{1.000000,1.000000,1.000000}%
\pgfsetfillcolor{currentfill}%
\pgfsetlinewidth{0.000000pt}%
\definecolor{currentstroke}{rgb}{0.000000,0.000000,0.000000}%
\pgfsetstrokecolor{currentstroke}%
\pgfsetstrokeopacity{0.000000}%
\pgfsetdash{}{0pt}%
\pgfpathmoveto{\pgfqpoint{4.060062in}{0.451389in}}%
\pgfpathlineto{\pgfqpoint{5.456045in}{0.451389in}}%
\pgfpathlineto{\pgfqpoint{5.456045in}{2.204860in}}%
\pgfpathlineto{\pgfqpoint{4.060062in}{2.204860in}}%
\pgfpathlineto{\pgfqpoint{4.060062in}{0.451389in}}%
\pgfpathclose%
\pgfusepath{fill}%
\end{pgfscope}%
\begin{pgfscope}%
\pgfpathrectangle{\pgfqpoint{4.060062in}{0.451389in}}{\pgfqpoint{1.395982in}{1.753471in}}%
\pgfusepath{clip}%
\pgfsetroundcap%
\pgfsetroundjoin%
\pgfsetlinewidth{1.003750pt}%
\definecolor{currentstroke}{rgb}{0.800000,0.800000,0.800000}%
\pgfsetstrokecolor{currentstroke}%
\pgfsetdash{}{0pt}%
\pgfpathmoveto{\pgfqpoint{4.060062in}{0.451389in}}%
\pgfpathlineto{\pgfqpoint{4.060062in}{2.204860in}}%
\pgfusepath{stroke}%
\end{pgfscope}%
\begin{pgfscope}%
\definecolor{textcolor}{rgb}{0.150000,0.150000,0.150000}%
\pgfsetstrokecolor{textcolor}%
\pgfsetfillcolor{textcolor}%
\pgftext[x=4.060062in,y=0.319444in,,top]{\color{textcolor}\sffamily\fontsize{9.000000}{10.800000}\selectfont 0k}%
\end{pgfscope}%
\begin{pgfscope}%
\pgfpathrectangle{\pgfqpoint{4.060062in}{0.451389in}}{\pgfqpoint{1.395982in}{1.753471in}}%
\pgfusepath{clip}%
\pgfsetroundcap%
\pgfsetroundjoin%
\pgfsetlinewidth{1.003750pt}%
\definecolor{currentstroke}{rgb}{0.800000,0.800000,0.800000}%
\pgfsetstrokecolor{currentstroke}%
\pgfsetdash{}{0pt}%
\pgfpathmoveto{\pgfqpoint{4.359643in}{0.451389in}}%
\pgfpathlineto{\pgfqpoint{4.359643in}{2.204860in}}%
\pgfusepath{stroke}%
\end{pgfscope}%
\begin{pgfscope}%
\definecolor{textcolor}{rgb}{0.150000,0.150000,0.150000}%
\pgfsetstrokecolor{textcolor}%
\pgfsetfillcolor{textcolor}%
\pgftext[x=4.359643in,y=0.319444in,,top]{\color{textcolor}\sffamily\fontsize{9.000000}{10.800000}\selectfont 10k}%
\end{pgfscope}%
\begin{pgfscope}%
\pgfpathrectangle{\pgfqpoint{4.060062in}{0.451389in}}{\pgfqpoint{1.395982in}{1.753471in}}%
\pgfusepath{clip}%
\pgfsetroundcap%
\pgfsetroundjoin%
\pgfsetlinewidth{1.003750pt}%
\definecolor{currentstroke}{rgb}{0.800000,0.800000,0.800000}%
\pgfsetstrokecolor{currentstroke}%
\pgfsetdash{}{0pt}%
\pgfpathmoveto{\pgfqpoint{4.659225in}{0.451389in}}%
\pgfpathlineto{\pgfqpoint{4.659225in}{2.204860in}}%
\pgfusepath{stroke}%
\end{pgfscope}%
\begin{pgfscope}%
\definecolor{textcolor}{rgb}{0.150000,0.150000,0.150000}%
\pgfsetstrokecolor{textcolor}%
\pgfsetfillcolor{textcolor}%
\pgftext[x=4.659225in,y=0.319444in,,top]{\color{textcolor}\sffamily\fontsize{9.000000}{10.800000}\selectfont 20k}%
\end{pgfscope}%
\begin{pgfscope}%
\pgfpathrectangle{\pgfqpoint{4.060062in}{0.451389in}}{\pgfqpoint{1.395982in}{1.753471in}}%
\pgfusepath{clip}%
\pgfsetroundcap%
\pgfsetroundjoin%
\pgfsetlinewidth{1.003750pt}%
\definecolor{currentstroke}{rgb}{0.800000,0.800000,0.800000}%
\pgfsetstrokecolor{currentstroke}%
\pgfsetdash{}{0pt}%
\pgfpathmoveto{\pgfqpoint{4.958806in}{0.451389in}}%
\pgfpathlineto{\pgfqpoint{4.958806in}{2.204860in}}%
\pgfusepath{stroke}%
\end{pgfscope}%
\begin{pgfscope}%
\definecolor{textcolor}{rgb}{0.150000,0.150000,0.150000}%
\pgfsetstrokecolor{textcolor}%
\pgfsetfillcolor{textcolor}%
\pgftext[x=4.958806in,y=0.319444in,,top]{\color{textcolor}\sffamily\fontsize{9.000000}{10.800000}\selectfont 30k}%
\end{pgfscope}%
\begin{pgfscope}%
\pgfpathrectangle{\pgfqpoint{4.060062in}{0.451389in}}{\pgfqpoint{1.395982in}{1.753471in}}%
\pgfusepath{clip}%
\pgfsetroundcap%
\pgfsetroundjoin%
\pgfsetlinewidth{1.003750pt}%
\definecolor{currentstroke}{rgb}{0.800000,0.800000,0.800000}%
\pgfsetstrokecolor{currentstroke}%
\pgfsetdash{}{0pt}%
\pgfpathmoveto{\pgfqpoint{5.258387in}{0.451389in}}%
\pgfpathlineto{\pgfqpoint{5.258387in}{2.204860in}}%
\pgfusepath{stroke}%
\end{pgfscope}%
\begin{pgfscope}%
\definecolor{textcolor}{rgb}{0.150000,0.150000,0.150000}%
\pgfsetstrokecolor{textcolor}%
\pgfsetfillcolor{textcolor}%
\pgftext[x=5.258387in,y=0.319444in,,top]{\color{textcolor}\sffamily\fontsize{9.000000}{10.800000}\selectfont 40k}%
\end{pgfscope}%
\begin{pgfscope}%
\definecolor{textcolor}{rgb}{0.150000,0.150000,0.150000}%
\pgfsetstrokecolor{textcolor}%
\pgfsetfillcolor{textcolor}%
\pgftext[x=4.758053in,y=0.125000in,,top]{\color{textcolor}\sffamily\fontsize{9.000000}{10.800000}\selectfont Input obstacle vertices}%
\end{pgfscope}%
\begin{pgfscope}%
\pgfpathrectangle{\pgfqpoint{4.060062in}{0.451389in}}{\pgfqpoint{1.395982in}{1.753471in}}%
\pgfusepath{clip}%
\pgfsetroundcap%
\pgfsetroundjoin%
\pgfsetlinewidth{1.003750pt}%
\definecolor{currentstroke}{rgb}{0.800000,0.800000,0.800000}%
\pgfsetstrokecolor{currentstroke}%
\pgfsetdash{}{0pt}%
\pgfpathmoveto{\pgfqpoint{4.060062in}{0.451389in}}%
\pgfpathlineto{\pgfqpoint{5.456045in}{0.451389in}}%
\pgfusepath{stroke}%
\end{pgfscope}%
\begin{pgfscope}%
\definecolor{textcolor}{rgb}{0.150000,0.150000,0.150000}%
\pgfsetstrokecolor{textcolor}%
\pgfsetfillcolor{textcolor}%
\pgftext[x=3.935659in, y=0.403903in, left, base]{\color{textcolor}\sffamily\fontsize{9.000000}{10.800000}\selectfont 0}%
\end{pgfscope}%
\begin{pgfscope}%
\pgfpathrectangle{\pgfqpoint{4.060062in}{0.451389in}}{\pgfqpoint{1.395982in}{1.753471in}}%
\pgfusepath{clip}%
\pgfsetroundcap%
\pgfsetroundjoin%
\pgfsetlinewidth{1.003750pt}%
\definecolor{currentstroke}{rgb}{0.800000,0.800000,0.800000}%
\pgfsetstrokecolor{currentstroke}%
\pgfsetdash{}{0pt}%
\pgfpathmoveto{\pgfqpoint{4.060062in}{0.860421in}}%
\pgfpathlineto{\pgfqpoint{5.456045in}{0.860421in}}%
\pgfusepath{stroke}%
\end{pgfscope}%
\begin{pgfscope}%
\definecolor{textcolor}{rgb}{0.150000,0.150000,0.150000}%
\pgfsetstrokecolor{textcolor}%
\pgfsetfillcolor{textcolor}%
\pgftext[x=3.797964in, y=0.812935in, left, base]{\color{textcolor}\sffamily\fontsize{9.000000}{10.800000}\selectfont 200}%
\end{pgfscope}%
\begin{pgfscope}%
\pgfpathrectangle{\pgfqpoint{4.060062in}{0.451389in}}{\pgfqpoint{1.395982in}{1.753471in}}%
\pgfusepath{clip}%
\pgfsetroundcap%
\pgfsetroundjoin%
\pgfsetlinewidth{1.003750pt}%
\definecolor{currentstroke}{rgb}{0.800000,0.800000,0.800000}%
\pgfsetstrokecolor{currentstroke}%
\pgfsetdash{}{0pt}%
\pgfpathmoveto{\pgfqpoint{4.060062in}{1.269453in}}%
\pgfpathlineto{\pgfqpoint{5.456045in}{1.269453in}}%
\pgfusepath{stroke}%
\end{pgfscope}%
\begin{pgfscope}%
\definecolor{textcolor}{rgb}{0.150000,0.150000,0.150000}%
\pgfsetstrokecolor{textcolor}%
\pgfsetfillcolor{textcolor}%
\pgftext[x=3.797964in, y=1.221967in, left, base]{\color{textcolor}\sffamily\fontsize{9.000000}{10.800000}\selectfont 400}%
\end{pgfscope}%
\begin{pgfscope}%
\pgfpathrectangle{\pgfqpoint{4.060062in}{0.451389in}}{\pgfqpoint{1.395982in}{1.753471in}}%
\pgfusepath{clip}%
\pgfsetroundcap%
\pgfsetroundjoin%
\pgfsetlinewidth{1.003750pt}%
\definecolor{currentstroke}{rgb}{0.800000,0.800000,0.800000}%
\pgfsetstrokecolor{currentstroke}%
\pgfsetdash{}{0pt}%
\pgfpathmoveto{\pgfqpoint{4.060062in}{1.678485in}}%
\pgfpathlineto{\pgfqpoint{5.456045in}{1.678485in}}%
\pgfusepath{stroke}%
\end{pgfscope}%
\begin{pgfscope}%
\definecolor{textcolor}{rgb}{0.150000,0.150000,0.150000}%
\pgfsetstrokecolor{textcolor}%
\pgfsetfillcolor{textcolor}%
\pgftext[x=3.797964in, y=1.630999in, left, base]{\color{textcolor}\sffamily\fontsize{9.000000}{10.800000}\selectfont 600}%
\end{pgfscope}%
\begin{pgfscope}%
\pgfpathrectangle{\pgfqpoint{4.060062in}{0.451389in}}{\pgfqpoint{1.395982in}{1.753471in}}%
\pgfusepath{clip}%
\pgfsetroundcap%
\pgfsetroundjoin%
\pgfsetlinewidth{1.003750pt}%
\definecolor{currentstroke}{rgb}{0.800000,0.800000,0.800000}%
\pgfsetstrokecolor{currentstroke}%
\pgfsetdash{}{0pt}%
\pgfpathmoveto{\pgfqpoint{4.060062in}{2.087517in}}%
\pgfpathlineto{\pgfqpoint{5.456045in}{2.087517in}}%
\pgfusepath{stroke}%
\end{pgfscope}%
\begin{pgfscope}%
\definecolor{textcolor}{rgb}{0.150000,0.150000,0.150000}%
\pgfsetstrokecolor{textcolor}%
\pgfsetfillcolor{textcolor}%
\pgftext[x=3.797964in, y=2.040031in, left, base]{\color{textcolor}\sffamily\fontsize{9.000000}{10.800000}\selectfont 800}%
\end{pgfscope}%
\begin{pgfscope}%
\pgfsetrectcap%
\pgfsetmiterjoin%
\pgfsetlinewidth{1.254687pt}%
\definecolor{currentstroke}{rgb}{0.800000,0.800000,0.800000}%
\pgfsetstrokecolor{currentstroke}%
\pgfsetdash{}{0pt}%
\pgfpathmoveto{\pgfqpoint{4.060062in}{0.451389in}}%
\pgfpathlineto{\pgfqpoint{4.060062in}{2.204860in}}%
\pgfusepath{stroke}%
\end{pgfscope}%
\begin{pgfscope}%
\pgfsetrectcap%
\pgfsetmiterjoin%
\pgfsetlinewidth{1.254687pt}%
\definecolor{currentstroke}{rgb}{0.800000,0.800000,0.800000}%
\pgfsetstrokecolor{currentstroke}%
\pgfsetdash{}{0pt}%
\pgfpathmoveto{\pgfqpoint{5.456045in}{0.451389in}}%
\pgfpathlineto{\pgfqpoint{5.456045in}{2.204860in}}%
\pgfusepath{stroke}%
\end{pgfscope}%
\begin{pgfscope}%
\pgfsetrectcap%
\pgfsetmiterjoin%
\pgfsetlinewidth{1.254687pt}%
\definecolor{currentstroke}{rgb}{0.800000,0.800000,0.800000}%
\pgfsetstrokecolor{currentstroke}%
\pgfsetdash{}{0pt}%
\pgfpathmoveto{\pgfqpoint{4.060062in}{0.451389in}}%
\pgfpathlineto{\pgfqpoint{5.456045in}{0.451389in}}%
\pgfusepath{stroke}%
\end{pgfscope}%
\begin{pgfscope}%
\pgfsetrectcap%
\pgfsetmiterjoin%
\pgfsetlinewidth{1.254687pt}%
\definecolor{currentstroke}{rgb}{0.800000,0.800000,0.800000}%
\pgfsetstrokecolor{currentstroke}%
\pgfsetdash{}{0pt}%
\pgfpathmoveto{\pgfqpoint{4.060062in}{2.204860in}}%
\pgfpathlineto{\pgfqpoint{5.456045in}{2.204860in}}%
\pgfusepath{stroke}%
\end{pgfscope}%
\begin{pgfscope}%
\definecolor{textcolor}{rgb}{0.150000,0.150000,0.150000}%
\pgfsetstrokecolor{textcolor}%
\pgfsetfillcolor{textcolor}%
\pgftext[x=4.758053in,y=2.315971in,,base]{\color{textcolor}\sffamily\fontsize{9.000000}{10.800000}\selectfont Circles}%
\end{pgfscope}%
\begin{pgfscope}%
\pgfsetbuttcap%
\pgfsetmiterjoin%
\definecolor{currentfill}{rgb}{1.000000,1.000000,1.000000}%
\pgfsetfillcolor{currentfill}%
\pgfsetfillopacity{0.800000}%
\pgfsetlinewidth{1.003750pt}%
\definecolor{currentstroke}{rgb}{0.800000,0.800000,0.800000}%
\pgfsetstrokecolor{currentstroke}%
\pgfsetstrokeopacity{0.800000}%
\pgfsetdash{}{0pt}%
\pgfpathmoveto{\pgfqpoint{5.578444in}{0.552637in}}%
\pgfpathlineto{\pgfqpoint{6.610097in}{0.552637in}}%
\pgfpathquadraticcurveto{\pgfqpoint{6.635097in}{0.552637in}}{\pgfqpoint{6.635097in}{0.577637in}}%
\pgfpathlineto{\pgfqpoint{6.635097in}{2.078612in}}%
\pgfpathquadraticcurveto{\pgfqpoint{6.635097in}{2.103612in}}{\pgfqpoint{6.610097in}{2.103612in}}%
\pgfpathlineto{\pgfqpoint{5.578444in}{2.103612in}}%
\pgfpathquadraticcurveto{\pgfqpoint{5.553444in}{2.103612in}}{\pgfqpoint{5.553444in}{2.078612in}}%
\pgfpathlineto{\pgfqpoint{5.553444in}{0.577637in}}%
\pgfpathquadraticcurveto{\pgfqpoint{5.553444in}{0.552637in}}{\pgfqpoint{5.578444in}{0.552637in}}%
\pgfpathlineto{\pgfqpoint{5.578444in}{0.552637in}}%
\pgfpathclose%
\pgfusepath{stroke,fill}%
\end{pgfscope}%
\begin{pgfscope}%
\definecolor{textcolor}{rgb}{0.150000,0.150000,0.150000}%
\pgfsetstrokecolor{textcolor}%
\pgfsetfillcolor{textcolor}%
\pgftext[x=5.890871in,y=1.958641in,left,base]{\color{textcolor}\sffamily\fontsize{9.000000}{10.800000}\selectfont Legend}%
\end{pgfscope}%
\begin{pgfscope}%
\pgfsetroundcap%
\pgfsetroundjoin%
\pgfsetlinewidth{1.505625pt}%
\definecolor{currentstroke}{rgb}{0.003922,0.450980,0.698039}%
\pgfsetstrokecolor{currentstroke}%
\pgfsetdash{}{0pt}%
\pgfpathmoveto{\pgfqpoint{5.603444in}{1.814891in}}%
\pgfpathlineto{\pgfqpoint{5.728444in}{1.814891in}}%
\pgfpathlineto{\pgfqpoint{5.853444in}{1.814891in}}%
\pgfusepath{stroke}%
\end{pgfscope}%
\begin{pgfscope}%
\definecolor{textcolor}{rgb}{0.150000,0.150000,0.150000}%
\pgfsetstrokecolor{textcolor}%
\pgfsetfillcolor{textcolor}%
\pgftext[x=5.953444in,y=1.771141in,left,base]{\color{textcolor}\sffamily\fontsize{9.000000}{10.800000}\selectfont Total time}%
\end{pgfscope}%
\begin{pgfscope}%
\pgfsetroundcap%
\pgfsetroundjoin%
\pgfsetlinewidth{1.505625pt}%
\definecolor{currentstroke}{rgb}{0.870588,0.560784,0.019608}%
\pgfsetstrokecolor{currentstroke}%
\pgfsetdash{}{0pt}%
\pgfpathmoveto{\pgfqpoint{5.603444in}{1.627392in}}%
\pgfpathlineto{\pgfqpoint{5.728444in}{1.627392in}}%
\pgfpathlineto{\pgfqpoint{5.853444in}{1.627392in}}%
\pgfusepath{stroke}%
\end{pgfscope}%
\begin{pgfscope}%
\definecolor{textcolor}{rgb}{0.150000,0.150000,0.150000}%
\pgfsetstrokecolor{textcolor}%
\pgfsetfillcolor{textcolor}%
\pgftext[x=5.953444in,y=1.583642in,left,base]{\color{textcolor}\sffamily\fontsize{9.000000}{10.800000}\selectfont A* routing}%
\end{pgfscope}%
\begin{pgfscope}%
\pgfsetroundcap%
\pgfsetroundjoin%
\pgfsetlinewidth{1.505625pt}%
\definecolor{currentstroke}{rgb}{0.007843,0.619608,0.450980}%
\pgfsetstrokecolor{currentstroke}%
\pgfsetdash{}{0pt}%
\pgfpathmoveto{\pgfqpoint{5.603444in}{1.208886in}}%
\pgfpathlineto{\pgfqpoint{5.728444in}{1.208886in}}%
\pgfpathlineto{\pgfqpoint{5.853444in}{1.208886in}}%
\pgfusepath{stroke}%
\end{pgfscope}%
\begin{pgfscope}%
\definecolor{textcolor}{rgb}{0.150000,0.150000,0.150000}%
\pgfsetstrokecolor{textcolor}%
\pgfsetfillcolor{textcolor}%
\pgftext[x=5.953444in, y=1.396142in, left, base]{\color{textcolor}\sffamily\fontsize{9.000000}{10.800000}\selectfont Connect}%
\end{pgfscope}%
\begin{pgfscope}%
\definecolor{textcolor}{rgb}{0.150000,0.150000,0.150000}%
\pgfsetstrokecolor{textcolor}%
\pgfsetfillcolor{textcolor}%
\pgftext[x=5.953444in, y=1.252148in, left, base]{\color{textcolor}\sffamily\fontsize{9.000000}{10.800000}\selectfont source \&}%
\end{pgfscope}%
\begin{pgfscope}%
\definecolor{textcolor}{rgb}{0.150000,0.150000,0.150000}%
\pgfsetstrokecolor{textcolor}%
\pgfsetfillcolor{textcolor}%
\pgftext[x=5.953444in, y=1.108154in, left, base]{\color{textcolor}\sffamily\fontsize{9.000000}{10.800000}\selectfont destination}%
\end{pgfscope}%
\begin{pgfscope}%
\definecolor{textcolor}{rgb}{0.150000,0.150000,0.150000}%
\pgfsetstrokecolor{textcolor}%
\pgfsetfillcolor{textcolor}%
\pgftext[x=5.953444in, y=0.964160in, left, base]{\color{textcolor}\sffamily\fontsize{9.000000}{10.800000}\selectfont vertices}%
\end{pgfscope}%
\begin{pgfscope}%
\pgfsetroundcap%
\pgfsetroundjoin%
\pgfsetlinewidth{1.505625pt}%
\definecolor{currentstroke}{rgb}{0.835294,0.368627,0.000000}%
\pgfsetstrokecolor{currentstroke}%
\pgfsetdash{}{0pt}%
\pgfpathmoveto{\pgfqpoint{5.603444in}{0.733398in}}%
\pgfpathlineto{\pgfqpoint{5.728444in}{0.733398in}}%
\pgfpathlineto{\pgfqpoint{5.853444in}{0.733398in}}%
\pgfusepath{stroke}%
\end{pgfscope}%
\begin{pgfscope}%
\definecolor{textcolor}{rgb}{0.150000,0.150000,0.150000}%
\pgfsetstrokecolor{textcolor}%
\pgfsetfillcolor{textcolor}%
\pgftext[x=5.953444in, y=0.776660in, left, base]{\color{textcolor}\sffamily\fontsize{9.000000}{10.800000}\selectfont Restoring}%
\end{pgfscope}%
\begin{pgfscope}%
\definecolor{textcolor}{rgb}{0.150000,0.150000,0.150000}%
\pgfsetstrokecolor{textcolor}%
\pgfsetfillcolor{textcolor}%
\pgftext[x=5.953444in, y=0.632666in, left, base]{\color{textcolor}\sffamily\fontsize{9.000000}{10.800000}\selectfont graph}%
\end{pgfscope}%
\begin{pgfscope}%
\pgfsetroundcap%
\pgfsetroundjoin%
\pgfsetlinewidth{1.003750pt}%
\definecolor{currentstroke}{rgb}{0.003922,0.450980,0.698039}%
\pgfsetstrokecolor{currentstroke}%
\pgfsetdash{}{0pt}%
\pgfpathmoveto{\pgfqpoint{4.073364in}{0.475931in}}%
\pgfpathlineto{\pgfqpoint{4.113268in}{0.518685in}}%
\pgfpathlineto{\pgfqpoint{4.179775in}{0.600197in}}%
\pgfpathlineto{\pgfqpoint{4.272885in}{0.692752in}}%
\pgfpathlineto{\pgfqpoint{4.392597in}{0.778492in}}%
\pgfpathlineto{\pgfqpoint{4.538913in}{0.997606in}}%
\pgfpathlineto{\pgfqpoint{4.711831in}{1.051300in}}%
\pgfpathlineto{\pgfqpoint{4.911352in}{1.511090in}}%
\pgfpathlineto{\pgfqpoint{5.137476in}{1.667641in}}%
\pgfpathlineto{\pgfqpoint{5.390203in}{2.121374in}}%
\pgfusepath{stroke}%
\end{pgfscope}%
\begin{pgfscope}%
\pgfsetbuttcap%
\pgfsetroundjoin%
\definecolor{currentfill}{rgb}{0.003922,0.450980,0.698039}%
\pgfsetfillcolor{currentfill}%
\pgfsetlinewidth{0.752812pt}%
\definecolor{currentstroke}{rgb}{1.000000,1.000000,1.000000}%
\pgfsetstrokecolor{currentstroke}%
\pgfsetdash{}{0pt}%
\pgfsys@defobject{currentmarker}{\pgfqpoint{-0.034722in}{-0.034722in}}{\pgfqpoint{0.034722in}{0.034722in}}{%
\pgfpathmoveto{\pgfqpoint{0.000000in}{-0.034722in}}%
\pgfpathcurveto{\pgfqpoint{0.009208in}{-0.034722in}}{\pgfqpoint{0.018041in}{-0.031064in}}{\pgfqpoint{0.024552in}{-0.024552in}}%
\pgfpathcurveto{\pgfqpoint{0.031064in}{-0.018041in}}{\pgfqpoint{0.034722in}{-0.009208in}}{\pgfqpoint{0.034722in}{0.000000in}}%
\pgfpathcurveto{\pgfqpoint{0.034722in}{0.009208in}}{\pgfqpoint{0.031064in}{0.018041in}}{\pgfqpoint{0.024552in}{0.024552in}}%
\pgfpathcurveto{\pgfqpoint{0.018041in}{0.031064in}}{\pgfqpoint{0.009208in}{0.034722in}}{\pgfqpoint{0.000000in}{0.034722in}}%
\pgfpathcurveto{\pgfqpoint{-0.009208in}{0.034722in}}{\pgfqpoint{-0.018041in}{0.031064in}}{\pgfqpoint{-0.024552in}{0.024552in}}%
\pgfpathcurveto{\pgfqpoint{-0.031064in}{0.018041in}}{\pgfqpoint{-0.034722in}{0.009208in}}{\pgfqpoint{-0.034722in}{0.000000in}}%
\pgfpathcurveto{\pgfqpoint{-0.034722in}{-0.009208in}}{\pgfqpoint{-0.031064in}{-0.018041in}}{\pgfqpoint{-0.024552in}{-0.024552in}}%
\pgfpathcurveto{\pgfqpoint{-0.018041in}{-0.031064in}}{\pgfqpoint{-0.009208in}{-0.034722in}}{\pgfqpoint{0.000000in}{-0.034722in}}%
\pgfpathlineto{\pgfqpoint{0.000000in}{-0.034722in}}%
\pgfpathclose%
\pgfusepath{stroke,fill}%
}%
\begin{pgfscope}%
\pgfsys@transformshift{4.073364in}{0.475931in}%
\pgfsys@useobject{currentmarker}{}%
\end{pgfscope}%
\begin{pgfscope}%
\pgfsys@transformshift{4.113268in}{0.518685in}%
\pgfsys@useobject{currentmarker}{}%
\end{pgfscope}%
\begin{pgfscope}%
\pgfsys@transformshift{4.179775in}{0.600197in}%
\pgfsys@useobject{currentmarker}{}%
\end{pgfscope}%
\begin{pgfscope}%
\pgfsys@transformshift{4.272885in}{0.692752in}%
\pgfsys@useobject{currentmarker}{}%
\end{pgfscope}%
\begin{pgfscope}%
\pgfsys@transformshift{4.392597in}{0.778492in}%
\pgfsys@useobject{currentmarker}{}%
\end{pgfscope}%
\begin{pgfscope}%
\pgfsys@transformshift{4.538913in}{0.997606in}%
\pgfsys@useobject{currentmarker}{}%
\end{pgfscope}%
\begin{pgfscope}%
\pgfsys@transformshift{4.711831in}{1.051300in}%
\pgfsys@useobject{currentmarker}{}%
\end{pgfscope}%
\begin{pgfscope}%
\pgfsys@transformshift{4.911352in}{1.511090in}%
\pgfsys@useobject{currentmarker}{}%
\end{pgfscope}%
\begin{pgfscope}%
\pgfsys@transformshift{5.137476in}{1.667641in}%
\pgfsys@useobject{currentmarker}{}%
\end{pgfscope}%
\begin{pgfscope}%
\pgfsys@transformshift{5.390203in}{2.121374in}%
\pgfsys@useobject{currentmarker}{}%
\end{pgfscope}%
\end{pgfscope}%
\begin{pgfscope}%
\pgfsetroundcap%
\pgfsetroundjoin%
\pgfsetlinewidth{1.003750pt}%
\definecolor{currentstroke}{rgb}{0.870588,0.560784,0.019608}%
\pgfsetstrokecolor{currentstroke}%
\pgfsetdash{}{0pt}%
\pgfpathmoveto{\pgfqpoint{4.073364in}{0.451661in}}%
\pgfpathlineto{\pgfqpoint{4.113268in}{0.453299in}}%
\pgfpathlineto{\pgfqpoint{4.179775in}{0.457131in}}%
\pgfpathlineto{\pgfqpoint{4.272885in}{0.461477in}}%
\pgfpathlineto{\pgfqpoint{4.392597in}{0.467934in}}%
\pgfpathlineto{\pgfqpoint{4.538913in}{0.476319in}}%
\pgfpathlineto{\pgfqpoint{4.711831in}{0.485306in}}%
\pgfpathlineto{\pgfqpoint{4.911352in}{0.500798in}}%
\pgfpathlineto{\pgfqpoint{5.137476in}{0.513852in}}%
\pgfpathlineto{\pgfqpoint{5.390203in}{0.531557in}}%
\pgfusepath{stroke}%
\end{pgfscope}%
\begin{pgfscope}%
\pgfsetbuttcap%
\pgfsetroundjoin%
\definecolor{currentfill}{rgb}{0.870588,0.560784,0.019608}%
\pgfsetfillcolor{currentfill}%
\pgfsetlinewidth{0.752812pt}%
\definecolor{currentstroke}{rgb}{1.000000,1.000000,1.000000}%
\pgfsetstrokecolor{currentstroke}%
\pgfsetdash{}{0pt}%
\pgfsys@defobject{currentmarker}{\pgfqpoint{-0.034722in}{-0.034722in}}{\pgfqpoint{0.034722in}{0.034722in}}{%
\pgfpathmoveto{\pgfqpoint{0.000000in}{-0.034722in}}%
\pgfpathcurveto{\pgfqpoint{0.009208in}{-0.034722in}}{\pgfqpoint{0.018041in}{-0.031064in}}{\pgfqpoint{0.024552in}{-0.024552in}}%
\pgfpathcurveto{\pgfqpoint{0.031064in}{-0.018041in}}{\pgfqpoint{0.034722in}{-0.009208in}}{\pgfqpoint{0.034722in}{0.000000in}}%
\pgfpathcurveto{\pgfqpoint{0.034722in}{0.009208in}}{\pgfqpoint{0.031064in}{0.018041in}}{\pgfqpoint{0.024552in}{0.024552in}}%
\pgfpathcurveto{\pgfqpoint{0.018041in}{0.031064in}}{\pgfqpoint{0.009208in}{0.034722in}}{\pgfqpoint{0.000000in}{0.034722in}}%
\pgfpathcurveto{\pgfqpoint{-0.009208in}{0.034722in}}{\pgfqpoint{-0.018041in}{0.031064in}}{\pgfqpoint{-0.024552in}{0.024552in}}%
\pgfpathcurveto{\pgfqpoint{-0.031064in}{0.018041in}}{\pgfqpoint{-0.034722in}{0.009208in}}{\pgfqpoint{-0.034722in}{0.000000in}}%
\pgfpathcurveto{\pgfqpoint{-0.034722in}{-0.009208in}}{\pgfqpoint{-0.031064in}{-0.018041in}}{\pgfqpoint{-0.024552in}{-0.024552in}}%
\pgfpathcurveto{\pgfqpoint{-0.018041in}{-0.031064in}}{\pgfqpoint{-0.009208in}{-0.034722in}}{\pgfqpoint{0.000000in}{-0.034722in}}%
\pgfpathlineto{\pgfqpoint{0.000000in}{-0.034722in}}%
\pgfpathclose%
\pgfusepath{stroke,fill}%
}%
\begin{pgfscope}%
\pgfsys@transformshift{4.073364in}{0.451661in}%
\pgfsys@useobject{currentmarker}{}%
\end{pgfscope}%
\begin{pgfscope}%
\pgfsys@transformshift{4.113268in}{0.453299in}%
\pgfsys@useobject{currentmarker}{}%
\end{pgfscope}%
\begin{pgfscope}%
\pgfsys@transformshift{4.179775in}{0.457131in}%
\pgfsys@useobject{currentmarker}{}%
\end{pgfscope}%
\begin{pgfscope}%
\pgfsys@transformshift{4.272885in}{0.461477in}%
\pgfsys@useobject{currentmarker}{}%
\end{pgfscope}%
\begin{pgfscope}%
\pgfsys@transformshift{4.392597in}{0.467934in}%
\pgfsys@useobject{currentmarker}{}%
\end{pgfscope}%
\begin{pgfscope}%
\pgfsys@transformshift{4.538913in}{0.476319in}%
\pgfsys@useobject{currentmarker}{}%
\end{pgfscope}%
\begin{pgfscope}%
\pgfsys@transformshift{4.711831in}{0.485306in}%
\pgfsys@useobject{currentmarker}{}%
\end{pgfscope}%
\begin{pgfscope}%
\pgfsys@transformshift{4.911352in}{0.500798in}%
\pgfsys@useobject{currentmarker}{}%
\end{pgfscope}%
\begin{pgfscope}%
\pgfsys@transformshift{5.137476in}{0.513852in}%
\pgfsys@useobject{currentmarker}{}%
\end{pgfscope}%
\begin{pgfscope}%
\pgfsys@transformshift{5.390203in}{0.531557in}%
\pgfsys@useobject{currentmarker}{}%
\end{pgfscope}%
\end{pgfscope}%
\begin{pgfscope}%
\pgfsetroundcap%
\pgfsetroundjoin%
\pgfsetlinewidth{1.003750pt}%
\definecolor{currentstroke}{rgb}{0.007843,0.619608,0.450980}%
\pgfsetstrokecolor{currentstroke}%
\pgfsetdash{}{0pt}%
\pgfpathmoveto{\pgfqpoint{4.073364in}{0.474190in}}%
\pgfpathlineto{\pgfqpoint{4.113268in}{0.511946in}}%
\pgfpathlineto{\pgfqpoint{4.179775in}{0.583608in}}%
\pgfpathlineto{\pgfqpoint{4.272885in}{0.660928in}}%
\pgfpathlineto{\pgfqpoint{4.392597in}{0.732377in}}%
\pgfpathlineto{\pgfqpoint{4.538913in}{0.920342in}}%
\pgfpathlineto{\pgfqpoint{4.711831in}{0.955809in}}%
\pgfpathlineto{\pgfqpoint{4.911352in}{1.351234in}}%
\pgfpathlineto{\pgfqpoint{5.137476in}{1.489228in}}%
\pgfpathlineto{\pgfqpoint{5.390203in}{1.875879in}}%
\pgfusepath{stroke}%
\end{pgfscope}%
\begin{pgfscope}%
\pgfsetbuttcap%
\pgfsetroundjoin%
\definecolor{currentfill}{rgb}{0.007843,0.619608,0.450980}%
\pgfsetfillcolor{currentfill}%
\pgfsetlinewidth{0.752812pt}%
\definecolor{currentstroke}{rgb}{1.000000,1.000000,1.000000}%
\pgfsetstrokecolor{currentstroke}%
\pgfsetdash{}{0pt}%
\pgfsys@defobject{currentmarker}{\pgfqpoint{-0.034722in}{-0.034722in}}{\pgfqpoint{0.034722in}{0.034722in}}{%
\pgfpathmoveto{\pgfqpoint{0.000000in}{-0.034722in}}%
\pgfpathcurveto{\pgfqpoint{0.009208in}{-0.034722in}}{\pgfqpoint{0.018041in}{-0.031064in}}{\pgfqpoint{0.024552in}{-0.024552in}}%
\pgfpathcurveto{\pgfqpoint{0.031064in}{-0.018041in}}{\pgfqpoint{0.034722in}{-0.009208in}}{\pgfqpoint{0.034722in}{0.000000in}}%
\pgfpathcurveto{\pgfqpoint{0.034722in}{0.009208in}}{\pgfqpoint{0.031064in}{0.018041in}}{\pgfqpoint{0.024552in}{0.024552in}}%
\pgfpathcurveto{\pgfqpoint{0.018041in}{0.031064in}}{\pgfqpoint{0.009208in}{0.034722in}}{\pgfqpoint{0.000000in}{0.034722in}}%
\pgfpathcurveto{\pgfqpoint{-0.009208in}{0.034722in}}{\pgfqpoint{-0.018041in}{0.031064in}}{\pgfqpoint{-0.024552in}{0.024552in}}%
\pgfpathcurveto{\pgfqpoint{-0.031064in}{0.018041in}}{\pgfqpoint{-0.034722in}{0.009208in}}{\pgfqpoint{-0.034722in}{0.000000in}}%
\pgfpathcurveto{\pgfqpoint{-0.034722in}{-0.009208in}}{\pgfqpoint{-0.031064in}{-0.018041in}}{\pgfqpoint{-0.024552in}{-0.024552in}}%
\pgfpathcurveto{\pgfqpoint{-0.018041in}{-0.031064in}}{\pgfqpoint{-0.009208in}{-0.034722in}}{\pgfqpoint{0.000000in}{-0.034722in}}%
\pgfpathlineto{\pgfqpoint{0.000000in}{-0.034722in}}%
\pgfpathclose%
\pgfusepath{stroke,fill}%
}%
\begin{pgfscope}%
\pgfsys@transformshift{4.073364in}{0.474190in}%
\pgfsys@useobject{currentmarker}{}%
\end{pgfscope}%
\begin{pgfscope}%
\pgfsys@transformshift{4.113268in}{0.511946in}%
\pgfsys@useobject{currentmarker}{}%
\end{pgfscope}%
\begin{pgfscope}%
\pgfsys@transformshift{4.179775in}{0.583608in}%
\pgfsys@useobject{currentmarker}{}%
\end{pgfscope}%
\begin{pgfscope}%
\pgfsys@transformshift{4.272885in}{0.660928in}%
\pgfsys@useobject{currentmarker}{}%
\end{pgfscope}%
\begin{pgfscope}%
\pgfsys@transformshift{4.392597in}{0.732377in}%
\pgfsys@useobject{currentmarker}{}%
\end{pgfscope}%
\begin{pgfscope}%
\pgfsys@transformshift{4.538913in}{0.920342in}%
\pgfsys@useobject{currentmarker}{}%
\end{pgfscope}%
\begin{pgfscope}%
\pgfsys@transformshift{4.711831in}{0.955809in}%
\pgfsys@useobject{currentmarker}{}%
\end{pgfscope}%
\begin{pgfscope}%
\pgfsys@transformshift{4.911352in}{1.351234in}%
\pgfsys@useobject{currentmarker}{}%
\end{pgfscope}%
\begin{pgfscope}%
\pgfsys@transformshift{5.137476in}{1.489228in}%
\pgfsys@useobject{currentmarker}{}%
\end{pgfscope}%
\begin{pgfscope}%
\pgfsys@transformshift{5.390203in}{1.875879in}%
\pgfsys@useobject{currentmarker}{}%
\end{pgfscope}%
\end{pgfscope}%
\begin{pgfscope}%
\pgfsetroundcap%
\pgfsetroundjoin%
\pgfsetlinewidth{1.003750pt}%
\definecolor{currentstroke}{rgb}{0.835294,0.368627,0.000000}%
\pgfsetstrokecolor{currentstroke}%
\pgfsetdash{}{0pt}%
\pgfpathmoveto{\pgfqpoint{4.073364in}{0.452818in}}%
\pgfpathlineto{\pgfqpoint{4.113268in}{0.456174in}}%
\pgfpathlineto{\pgfqpoint{4.179775in}{0.462160in}}%
\pgfpathlineto{\pgfqpoint{4.272885in}{0.473043in}}%
\pgfpathlineto{\pgfqpoint{4.392597in}{0.480870in}}%
\pgfpathlineto{\pgfqpoint{4.538913in}{0.503614in}}%
\pgfpathlineto{\pgfqpoint{4.711831in}{0.512866in}}%
\pgfpathlineto{\pgfqpoint{4.911352in}{0.561719in}}%
\pgfpathlineto{\pgfqpoint{5.137476in}{0.567206in}}%
\pgfpathlineto{\pgfqpoint{5.390203in}{0.616597in}}%
\pgfusepath{stroke}%
\end{pgfscope}%
\begin{pgfscope}%
\pgfsetbuttcap%
\pgfsetroundjoin%
\definecolor{currentfill}{rgb}{0.835294,0.368627,0.000000}%
\pgfsetfillcolor{currentfill}%
\pgfsetlinewidth{0.752812pt}%
\definecolor{currentstroke}{rgb}{1.000000,1.000000,1.000000}%
\pgfsetstrokecolor{currentstroke}%
\pgfsetdash{}{0pt}%
\pgfsys@defobject{currentmarker}{\pgfqpoint{-0.034722in}{-0.034722in}}{\pgfqpoint{0.034722in}{0.034722in}}{%
\pgfpathmoveto{\pgfqpoint{0.000000in}{-0.034722in}}%
\pgfpathcurveto{\pgfqpoint{0.009208in}{-0.034722in}}{\pgfqpoint{0.018041in}{-0.031064in}}{\pgfqpoint{0.024552in}{-0.024552in}}%
\pgfpathcurveto{\pgfqpoint{0.031064in}{-0.018041in}}{\pgfqpoint{0.034722in}{-0.009208in}}{\pgfqpoint{0.034722in}{0.000000in}}%
\pgfpathcurveto{\pgfqpoint{0.034722in}{0.009208in}}{\pgfqpoint{0.031064in}{0.018041in}}{\pgfqpoint{0.024552in}{0.024552in}}%
\pgfpathcurveto{\pgfqpoint{0.018041in}{0.031064in}}{\pgfqpoint{0.009208in}{0.034722in}}{\pgfqpoint{0.000000in}{0.034722in}}%
\pgfpathcurveto{\pgfqpoint{-0.009208in}{0.034722in}}{\pgfqpoint{-0.018041in}{0.031064in}}{\pgfqpoint{-0.024552in}{0.024552in}}%
\pgfpathcurveto{\pgfqpoint{-0.031064in}{0.018041in}}{\pgfqpoint{-0.034722in}{0.009208in}}{\pgfqpoint{-0.034722in}{0.000000in}}%
\pgfpathcurveto{\pgfqpoint{-0.034722in}{-0.009208in}}{\pgfqpoint{-0.031064in}{-0.018041in}}{\pgfqpoint{-0.024552in}{-0.024552in}}%
\pgfpathcurveto{\pgfqpoint{-0.018041in}{-0.031064in}}{\pgfqpoint{-0.009208in}{-0.034722in}}{\pgfqpoint{0.000000in}{-0.034722in}}%
\pgfpathlineto{\pgfqpoint{0.000000in}{-0.034722in}}%
\pgfpathclose%
\pgfusepath{stroke,fill}%
}%
\begin{pgfscope}%
\pgfsys@transformshift{4.073364in}{0.452818in}%
\pgfsys@useobject{currentmarker}{}%
\end{pgfscope}%
\begin{pgfscope}%
\pgfsys@transformshift{4.113268in}{0.456174in}%
\pgfsys@useobject{currentmarker}{}%
\end{pgfscope}%
\begin{pgfscope}%
\pgfsys@transformshift{4.179775in}{0.462160in}%
\pgfsys@useobject{currentmarker}{}%
\end{pgfscope}%
\begin{pgfscope}%
\pgfsys@transformshift{4.272885in}{0.473043in}%
\pgfsys@useobject{currentmarker}{}%
\end{pgfscope}%
\begin{pgfscope}%
\pgfsys@transformshift{4.392597in}{0.480870in}%
\pgfsys@useobject{currentmarker}{}%
\end{pgfscope}%
\begin{pgfscope}%
\pgfsys@transformshift{4.538913in}{0.503614in}%
\pgfsys@useobject{currentmarker}{}%
\end{pgfscope}%
\begin{pgfscope}%
\pgfsys@transformshift{4.711831in}{0.512866in}%
\pgfsys@useobject{currentmarker}{}%
\end{pgfscope}%
\begin{pgfscope}%
\pgfsys@transformshift{4.911352in}{0.561719in}%
\pgfsys@useobject{currentmarker}{}%
\end{pgfscope}%
\begin{pgfscope}%
\pgfsys@transformshift{5.137476in}{0.567206in}%
\pgfsys@useobject{currentmarker}{}%
\end{pgfscope}%
\begin{pgfscope}%
\pgfsys@transformshift{5.390203in}{0.616597in}%
\pgfsys@useobject{currentmarker}{}%
\end{pgfscope}%
\end{pgfscope}%
\end{pgfpicture}%
\makeatother%
\endgroup%

				\caption{Routing time between the longest distant waypoints in each pattern-based dataset.}
				\label{fig:eval-pattern-routing-details}
			\end{figure}
			
	\subsection{Memory consumption}
	\label{subsec:memory-consumption}
	
		\begin{figure}[h]
			\begin{figcenter}
				\begingroup%
\makeatletter%
\begin{pgfpicture}%
\pgfpathrectangle{\pgfpointorigin}{\pgfqpoint{6.085546in}{1.978691in}}%
\pgfusepath{use as bounding box}%
\begin{pgfscope}%
\pgfsetbuttcap%
\pgfsetmiterjoin%
\definecolor{currentfill}{rgb}{1.000000,1.000000,1.000000}%
\pgfsetfillcolor{currentfill}%
\pgfsetlinewidth{0.000000pt}%
\definecolor{currentstroke}{rgb}{1.000000,1.000000,1.000000}%
\pgfsetstrokecolor{currentstroke}%
\pgfsetdash{}{0pt}%
\pgfpathmoveto{\pgfqpoint{0.000000in}{0.000000in}}%
\pgfpathlineto{\pgfqpoint{6.085546in}{0.000000in}}%
\pgfpathlineto{\pgfqpoint{6.085546in}{1.978691in}}%
\pgfpathlineto{\pgfqpoint{0.000000in}{1.978691in}}%
\pgfpathlineto{\pgfqpoint{0.000000in}{0.000000in}}%
\pgfpathclose%
\pgfusepath{fill}%
\end{pgfscope}%
\begin{pgfscope}%
\pgfsetbuttcap%
\pgfsetmiterjoin%
\definecolor{currentfill}{rgb}{1.000000,1.000000,1.000000}%
\pgfsetfillcolor{currentfill}%
\pgfsetlinewidth{0.000000pt}%
\definecolor{currentstroke}{rgb}{0.000000,0.000000,0.000000}%
\pgfsetstrokecolor{currentstroke}%
\pgfsetstrokeopacity{0.000000}%
\pgfsetdash{}{0pt}%
\pgfpathmoveto{\pgfqpoint{0.601779in}{0.451389in}}%
\pgfpathlineto{\pgfqpoint{6.085546in}{0.451389in}}%
\pgfpathlineto{\pgfqpoint{6.085546in}{1.978691in}}%
\pgfpathlineto{\pgfqpoint{0.601779in}{1.978691in}}%
\pgfpathlineto{\pgfqpoint{0.601779in}{0.451389in}}%
\pgfpathclose%
\pgfusepath{fill}%
\end{pgfscope}%
\begin{pgfscope}%
\pgfpathrectangle{\pgfqpoint{0.601779in}{0.451389in}}{\pgfqpoint{5.483766in}{1.527302in}}%
\pgfusepath{clip}%
\pgfsetroundcap%
\pgfsetroundjoin%
\pgfsetlinewidth{1.003750pt}%
\definecolor{currentstroke}{rgb}{0.800000,0.800000,0.800000}%
\pgfsetstrokecolor{currentstroke}%
\pgfsetdash{}{0pt}%
\pgfpathmoveto{\pgfqpoint{0.601779in}{0.451389in}}%
\pgfpathlineto{\pgfqpoint{0.601779in}{1.978691in}}%
\pgfusepath{stroke}%
\end{pgfscope}%
\begin{pgfscope}%
\definecolor{textcolor}{rgb}{0.150000,0.150000,0.150000}%
\pgfsetstrokecolor{textcolor}%
\pgfsetfillcolor{textcolor}%
\pgftext[x=0.601779in,y=0.319444in,,top]{\color{textcolor}\sffamily\fontsize{9.000000}{10.800000}\selectfont 0}%
\end{pgfscope}%
\begin{pgfscope}%
\pgfpathrectangle{\pgfqpoint{0.601779in}{0.451389in}}{\pgfqpoint{5.483766in}{1.527302in}}%
\pgfusepath{clip}%
\pgfsetroundcap%
\pgfsetroundjoin%
\pgfsetlinewidth{1.003750pt}%
\definecolor{currentstroke}{rgb}{0.800000,0.800000,0.800000}%
\pgfsetstrokecolor{currentstroke}%
\pgfsetdash{}{0pt}%
\pgfpathmoveto{\pgfqpoint{1.522507in}{0.451389in}}%
\pgfpathlineto{\pgfqpoint{1.522507in}{1.978691in}}%
\pgfusepath{stroke}%
\end{pgfscope}%
\begin{pgfscope}%
\definecolor{textcolor}{rgb}{0.150000,0.150000,0.150000}%
\pgfsetstrokecolor{textcolor}%
\pgfsetfillcolor{textcolor}%
\pgftext[x=1.522507in,y=0.319444in,,top]{\color{textcolor}\sffamily\fontsize{9.000000}{10.800000}\selectfont 100}%
\end{pgfscope}%
\begin{pgfscope}%
\pgfpathrectangle{\pgfqpoint{0.601779in}{0.451389in}}{\pgfqpoint{5.483766in}{1.527302in}}%
\pgfusepath{clip}%
\pgfsetroundcap%
\pgfsetroundjoin%
\pgfsetlinewidth{1.003750pt}%
\definecolor{currentstroke}{rgb}{0.800000,0.800000,0.800000}%
\pgfsetstrokecolor{currentstroke}%
\pgfsetdash{}{0pt}%
\pgfpathmoveto{\pgfqpoint{2.443234in}{0.451389in}}%
\pgfpathlineto{\pgfqpoint{2.443234in}{1.978691in}}%
\pgfusepath{stroke}%
\end{pgfscope}%
\begin{pgfscope}%
\definecolor{textcolor}{rgb}{0.150000,0.150000,0.150000}%
\pgfsetstrokecolor{textcolor}%
\pgfsetfillcolor{textcolor}%
\pgftext[x=2.443234in,y=0.319444in,,top]{\color{textcolor}\sffamily\fontsize{9.000000}{10.800000}\selectfont 200}%
\end{pgfscope}%
\begin{pgfscope}%
\pgfpathrectangle{\pgfqpoint{0.601779in}{0.451389in}}{\pgfqpoint{5.483766in}{1.527302in}}%
\pgfusepath{clip}%
\pgfsetroundcap%
\pgfsetroundjoin%
\pgfsetlinewidth{1.003750pt}%
\definecolor{currentstroke}{rgb}{0.800000,0.800000,0.800000}%
\pgfsetstrokecolor{currentstroke}%
\pgfsetdash{}{0pt}%
\pgfpathmoveto{\pgfqpoint{3.363961in}{0.451389in}}%
\pgfpathlineto{\pgfqpoint{3.363961in}{1.978691in}}%
\pgfusepath{stroke}%
\end{pgfscope}%
\begin{pgfscope}%
\definecolor{textcolor}{rgb}{0.150000,0.150000,0.150000}%
\pgfsetstrokecolor{textcolor}%
\pgfsetfillcolor{textcolor}%
\pgftext[x=3.363961in,y=0.319444in,,top]{\color{textcolor}\sffamily\fontsize{9.000000}{10.800000}\selectfont 300}%
\end{pgfscope}%
\begin{pgfscope}%
\pgfpathrectangle{\pgfqpoint{0.601779in}{0.451389in}}{\pgfqpoint{5.483766in}{1.527302in}}%
\pgfusepath{clip}%
\pgfsetroundcap%
\pgfsetroundjoin%
\pgfsetlinewidth{1.003750pt}%
\definecolor{currentstroke}{rgb}{0.800000,0.800000,0.800000}%
\pgfsetstrokecolor{currentstroke}%
\pgfsetdash{}{0pt}%
\pgfpathmoveto{\pgfqpoint{4.284689in}{0.451389in}}%
\pgfpathlineto{\pgfqpoint{4.284689in}{1.978691in}}%
\pgfusepath{stroke}%
\end{pgfscope}%
\begin{pgfscope}%
\definecolor{textcolor}{rgb}{0.150000,0.150000,0.150000}%
\pgfsetstrokecolor{textcolor}%
\pgfsetfillcolor{textcolor}%
\pgftext[x=4.284689in,y=0.319444in,,top]{\color{textcolor}\sffamily\fontsize{9.000000}{10.800000}\selectfont 400}%
\end{pgfscope}%
\begin{pgfscope}%
\pgfpathrectangle{\pgfqpoint{0.601779in}{0.451389in}}{\pgfqpoint{5.483766in}{1.527302in}}%
\pgfusepath{clip}%
\pgfsetroundcap%
\pgfsetroundjoin%
\pgfsetlinewidth{1.003750pt}%
\definecolor{currentstroke}{rgb}{0.800000,0.800000,0.800000}%
\pgfsetstrokecolor{currentstroke}%
\pgfsetdash{}{0pt}%
\pgfpathmoveto{\pgfqpoint{5.205416in}{0.451389in}}%
\pgfpathlineto{\pgfqpoint{5.205416in}{1.978691in}}%
\pgfusepath{stroke}%
\end{pgfscope}%
\begin{pgfscope}%
\definecolor{textcolor}{rgb}{0.150000,0.150000,0.150000}%
\pgfsetstrokecolor{textcolor}%
\pgfsetfillcolor{textcolor}%
\pgftext[x=5.205416in,y=0.319444in,,top]{\color{textcolor}\sffamily\fontsize{9.000000}{10.800000}\selectfont 500}%
\end{pgfscope}%
\begin{pgfscope}%
\definecolor{textcolor}{rgb}{0.150000,0.150000,0.150000}%
\pgfsetstrokecolor{textcolor}%
\pgfsetfillcolor{textcolor}%
\pgftext[x=3.343663in,y=0.125000in,,top]{\color{textcolor}\sffamily\fontsize{9.000000}{10.800000}\selectfont Seconds after start}%
\end{pgfscope}%
\begin{pgfscope}%
\pgfpathrectangle{\pgfqpoint{0.601779in}{0.451389in}}{\pgfqpoint{5.483766in}{1.527302in}}%
\pgfusepath{clip}%
\pgfsetroundcap%
\pgfsetroundjoin%
\pgfsetlinewidth{1.003750pt}%
\definecolor{currentstroke}{rgb}{0.800000,0.800000,0.800000}%
\pgfsetstrokecolor{currentstroke}%
\pgfsetdash{}{0pt}%
\pgfpathmoveto{\pgfqpoint{0.601779in}{0.451389in}}%
\pgfpathlineto{\pgfqpoint{6.085546in}{0.451389in}}%
\pgfusepath{stroke}%
\end{pgfscope}%
\begin{pgfscope}%
\definecolor{textcolor}{rgb}{0.150000,0.150000,0.150000}%
\pgfsetstrokecolor{textcolor}%
\pgfsetfillcolor{textcolor}%
\pgftext[x=0.400987in, y=0.403903in, left, base]{\color{textcolor}\sffamily\fontsize{9.000000}{10.800000}\selectfont 0}%
\end{pgfscope}%
\begin{pgfscope}%
\pgfpathrectangle{\pgfqpoint{0.601779in}{0.451389in}}{\pgfqpoint{5.483766in}{1.527302in}}%
\pgfusepath{clip}%
\pgfsetroundcap%
\pgfsetroundjoin%
\pgfsetlinewidth{1.003750pt}%
\definecolor{currentstroke}{rgb}{0.800000,0.800000,0.800000}%
\pgfsetstrokecolor{currentstroke}%
\pgfsetdash{}{0pt}%
\pgfpathmoveto{\pgfqpoint{0.601779in}{0.729956in}}%
\pgfpathlineto{\pgfqpoint{6.085546in}{0.729956in}}%
\pgfusepath{stroke}%
\end{pgfscope}%
\begin{pgfscope}%
\definecolor{textcolor}{rgb}{0.150000,0.150000,0.150000}%
\pgfsetstrokecolor{textcolor}%
\pgfsetfillcolor{textcolor}%
\pgftext[x=0.263292in, y=0.682471in, left, base]{\color{textcolor}\sffamily\fontsize{9.000000}{10.800000}\selectfont 250}%
\end{pgfscope}%
\begin{pgfscope}%
\pgfpathrectangle{\pgfqpoint{0.601779in}{0.451389in}}{\pgfqpoint{5.483766in}{1.527302in}}%
\pgfusepath{clip}%
\pgfsetroundcap%
\pgfsetroundjoin%
\pgfsetlinewidth{1.003750pt}%
\definecolor{currentstroke}{rgb}{0.800000,0.800000,0.800000}%
\pgfsetstrokecolor{currentstroke}%
\pgfsetdash{}{0pt}%
\pgfpathmoveto{\pgfqpoint{0.601779in}{1.008523in}}%
\pgfpathlineto{\pgfqpoint{6.085546in}{1.008523in}}%
\pgfusepath{stroke}%
\end{pgfscope}%
\begin{pgfscope}%
\definecolor{textcolor}{rgb}{0.150000,0.150000,0.150000}%
\pgfsetstrokecolor{textcolor}%
\pgfsetfillcolor{textcolor}%
\pgftext[x=0.263292in, y=0.961038in, left, base]{\color{textcolor}\sffamily\fontsize{9.000000}{10.800000}\selectfont 500}%
\end{pgfscope}%
\begin{pgfscope}%
\pgfpathrectangle{\pgfqpoint{0.601779in}{0.451389in}}{\pgfqpoint{5.483766in}{1.527302in}}%
\pgfusepath{clip}%
\pgfsetroundcap%
\pgfsetroundjoin%
\pgfsetlinewidth{1.003750pt}%
\definecolor{currentstroke}{rgb}{0.800000,0.800000,0.800000}%
\pgfsetstrokecolor{currentstroke}%
\pgfsetdash{}{0pt}%
\pgfpathmoveto{\pgfqpoint{0.601779in}{1.287091in}}%
\pgfpathlineto{\pgfqpoint{6.085546in}{1.287091in}}%
\pgfusepath{stroke}%
\end{pgfscope}%
\begin{pgfscope}%
\definecolor{textcolor}{rgb}{0.150000,0.150000,0.150000}%
\pgfsetstrokecolor{textcolor}%
\pgfsetfillcolor{textcolor}%
\pgftext[x=0.263292in, y=1.239606in, left, base]{\color{textcolor}\sffamily\fontsize{9.000000}{10.800000}\selectfont 750}%
\end{pgfscope}%
\begin{pgfscope}%
\pgfpathrectangle{\pgfqpoint{0.601779in}{0.451389in}}{\pgfqpoint{5.483766in}{1.527302in}}%
\pgfusepath{clip}%
\pgfsetroundcap%
\pgfsetroundjoin%
\pgfsetlinewidth{1.003750pt}%
\definecolor{currentstroke}{rgb}{0.800000,0.800000,0.800000}%
\pgfsetstrokecolor{currentstroke}%
\pgfsetdash{}{0pt}%
\pgfpathmoveto{\pgfqpoint{0.601779in}{1.565658in}}%
\pgfpathlineto{\pgfqpoint{6.085546in}{1.565658in}}%
\pgfusepath{stroke}%
\end{pgfscope}%
\begin{pgfscope}%
\definecolor{textcolor}{rgb}{0.150000,0.150000,0.150000}%
\pgfsetstrokecolor{textcolor}%
\pgfsetfillcolor{textcolor}%
\pgftext[x=0.194444in, y=1.518173in, left, base]{\color{textcolor}\sffamily\fontsize{9.000000}{10.800000}\selectfont 1000}%
\end{pgfscope}%
\begin{pgfscope}%
\pgfpathrectangle{\pgfqpoint{0.601779in}{0.451389in}}{\pgfqpoint{5.483766in}{1.527302in}}%
\pgfusepath{clip}%
\pgfsetroundcap%
\pgfsetroundjoin%
\pgfsetlinewidth{1.003750pt}%
\definecolor{currentstroke}{rgb}{0.800000,0.800000,0.800000}%
\pgfsetstrokecolor{currentstroke}%
\pgfsetdash{}{0pt}%
\pgfpathmoveto{\pgfqpoint{0.601779in}{1.844226in}}%
\pgfpathlineto{\pgfqpoint{6.085546in}{1.844226in}}%
\pgfusepath{stroke}%
\end{pgfscope}%
\begin{pgfscope}%
\definecolor{textcolor}{rgb}{0.150000,0.150000,0.150000}%
\pgfsetstrokecolor{textcolor}%
\pgfsetfillcolor{textcolor}%
\pgftext[x=0.194444in, y=1.796740in, left, base]{\color{textcolor}\sffamily\fontsize{9.000000}{10.800000}\selectfont 1250}%
\end{pgfscope}%
\begin{pgfscope}%
\definecolor{textcolor}{rgb}{0.150000,0.150000,0.150000}%
\pgfsetstrokecolor{textcolor}%
\pgfsetfillcolor{textcolor}%
\pgftext[x=0.125000in,y=1.215040in,,bottom,rotate=90.000000]{\color{textcolor}\sffamily\fontsize{9.000000}{10.800000}\selectfont RAM usage in MB}%
\end{pgfscope}%
\begin{pgfscope}%
\pgfpathrectangle{\pgfqpoint{0.601779in}{0.451389in}}{\pgfqpoint{5.483766in}{1.527302in}}%
\pgfusepath{clip}%
\pgfsetroundcap%
\pgfsetroundjoin%
\pgfsetlinewidth{0.501875pt}%
\definecolor{currentstroke}{rgb}{0.705882,0.152941,0.152941}%
\pgfsetstrokecolor{currentstroke}%
\pgfsetdash{}{0pt}%
\pgfpathmoveto{\pgfqpoint{0.613233in}{0.604119in}}%
\pgfpathlineto{\pgfqpoint{0.613233in}{1.978691in}}%
\pgfusepath{stroke}%
\end{pgfscope}%
\begin{pgfscope}%
\pgfpathrectangle{\pgfqpoint{0.601779in}{0.451389in}}{\pgfqpoint{5.483766in}{1.527302in}}%
\pgfusepath{clip}%
\pgfsetroundcap%
\pgfsetroundjoin%
\pgfsetlinewidth{0.501875pt}%
\definecolor{currentstroke}{rgb}{0.705882,0.152941,0.152941}%
\pgfsetstrokecolor{currentstroke}%
\pgfsetdash{}{0pt}%
\pgfpathmoveto{\pgfqpoint{2.072098in}{0.604119in}}%
\pgfpathlineto{\pgfqpoint{2.072098in}{1.978691in}}%
\pgfusepath{stroke}%
\end{pgfscope}%
\begin{pgfscope}%
\pgfpathrectangle{\pgfqpoint{0.601779in}{0.451389in}}{\pgfqpoint{5.483766in}{1.527302in}}%
\pgfusepath{clip}%
\pgfsetroundcap%
\pgfsetroundjoin%
\pgfsetlinewidth{0.501875pt}%
\definecolor{currentstroke}{rgb}{0.705882,0.152941,0.152941}%
\pgfsetstrokecolor{currentstroke}%
\pgfsetdash{}{0pt}%
\pgfpathmoveto{\pgfqpoint{4.020578in}{0.604119in}}%
\pgfpathlineto{\pgfqpoint{4.020578in}{1.978691in}}%
\pgfusepath{stroke}%
\end{pgfscope}%
\begin{pgfscope}%
\pgfpathrectangle{\pgfqpoint{0.601779in}{0.451389in}}{\pgfqpoint{5.483766in}{1.527302in}}%
\pgfusepath{clip}%
\pgfsetroundcap%
\pgfsetroundjoin%
\pgfsetlinewidth{0.501875pt}%
\definecolor{currentstroke}{rgb}{0.705882,0.152941,0.152941}%
\pgfsetstrokecolor{currentstroke}%
\pgfsetdash{}{0pt}%
\pgfpathmoveto{\pgfqpoint{4.029132in}{0.604119in}}%
\pgfpathlineto{\pgfqpoint{4.029132in}{1.978691in}}%
\pgfusepath{stroke}%
\end{pgfscope}%
\begin{pgfscope}%
\pgfpathrectangle{\pgfqpoint{0.601779in}{0.451389in}}{\pgfqpoint{5.483766in}{1.527302in}}%
\pgfusepath{clip}%
\pgfsetroundcap%
\pgfsetroundjoin%
\pgfsetlinewidth{0.501875pt}%
\definecolor{currentstroke}{rgb}{0.705882,0.152941,0.152941}%
\pgfsetstrokecolor{currentstroke}%
\pgfsetdash{}{0pt}%
\pgfpathmoveto{\pgfqpoint{4.880675in}{0.604119in}}%
\pgfpathlineto{\pgfqpoint{4.880675in}{1.978691in}}%
\pgfusepath{stroke}%
\end{pgfscope}%
\begin{pgfscope}%
\pgfpathrectangle{\pgfqpoint{0.601779in}{0.451389in}}{\pgfqpoint{5.483766in}{1.527302in}}%
\pgfusepath{clip}%
\pgfsetroundcap%
\pgfsetroundjoin%
\pgfsetlinewidth{0.501875pt}%
\definecolor{currentstroke}{rgb}{0.705882,0.152941,0.152941}%
\pgfsetstrokecolor{currentstroke}%
\pgfsetdash{}{0pt}%
\pgfpathmoveto{\pgfqpoint{5.279792in}{0.604119in}}%
\pgfpathlineto{\pgfqpoint{5.279792in}{1.978691in}}%
\pgfusepath{stroke}%
\end{pgfscope}%
\begin{pgfscope}%
\pgfpathrectangle{\pgfqpoint{0.601779in}{0.451389in}}{\pgfqpoint{5.483766in}{1.527302in}}%
\pgfusepath{clip}%
\pgfsetroundcap%
\pgfsetroundjoin%
\pgfsetlinewidth{0.501875pt}%
\definecolor{currentstroke}{rgb}{0.705882,0.152941,0.152941}%
\pgfsetstrokecolor{currentstroke}%
\pgfsetdash{}{0pt}%
\pgfpathmoveto{\pgfqpoint{5.792094in}{0.604119in}}%
\pgfpathlineto{\pgfqpoint{5.792094in}{1.978691in}}%
\pgfusepath{stroke}%
\end{pgfscope}%
\begin{pgfscope}%
\pgfsetrectcap%
\pgfsetmiterjoin%
\pgfsetlinewidth{1.254687pt}%
\definecolor{currentstroke}{rgb}{0.800000,0.800000,0.800000}%
\pgfsetstrokecolor{currentstroke}%
\pgfsetdash{}{0pt}%
\pgfpathmoveto{\pgfqpoint{0.601779in}{0.451389in}}%
\pgfpathlineto{\pgfqpoint{0.601779in}{1.978691in}}%
\pgfusepath{stroke}%
\end{pgfscope}%
\begin{pgfscope}%
\pgfsetrectcap%
\pgfsetmiterjoin%
\pgfsetlinewidth{1.254687pt}%
\definecolor{currentstroke}{rgb}{0.800000,0.800000,0.800000}%
\pgfsetstrokecolor{currentstroke}%
\pgfsetdash{}{0pt}%
\pgfpathmoveto{\pgfqpoint{6.085546in}{0.451389in}}%
\pgfpathlineto{\pgfqpoint{6.085546in}{1.978691in}}%
\pgfusepath{stroke}%
\end{pgfscope}%
\begin{pgfscope}%
\pgfsetrectcap%
\pgfsetmiterjoin%
\pgfsetlinewidth{1.254687pt}%
\definecolor{currentstroke}{rgb}{0.800000,0.800000,0.800000}%
\pgfsetstrokecolor{currentstroke}%
\pgfsetdash{}{0pt}%
\pgfpathmoveto{\pgfqpoint{0.601779in}{0.451389in}}%
\pgfpathlineto{\pgfqpoint{6.085546in}{0.451389in}}%
\pgfusepath{stroke}%
\end{pgfscope}%
\begin{pgfscope}%
\pgfsetrectcap%
\pgfsetmiterjoin%
\pgfsetlinewidth{1.254687pt}%
\definecolor{currentstroke}{rgb}{0.800000,0.800000,0.800000}%
\pgfsetstrokecolor{currentstroke}%
\pgfsetdash{}{0pt}%
\pgfpathmoveto{\pgfqpoint{0.601779in}{1.978691in}}%
\pgfpathlineto{\pgfqpoint{6.085546in}{1.978691in}}%
\pgfusepath{stroke}%
\end{pgfscope}%
\begin{pgfscope}%
\definecolor{textcolor}{rgb}{0.150000,0.150000,0.150000}%
\pgfsetstrokecolor{textcolor}%
\pgfsetfillcolor{textcolor}%
\pgftext[x=0.578233in,y=0.501389in,left,base]{\color{textcolor}\sffamily\fontsize{9.000000}{10.800000}\selectfont 1}%
\end{pgfscope}%
\begin{pgfscope}%
\definecolor{textcolor}{rgb}{0.150000,0.150000,0.150000}%
\pgfsetstrokecolor{textcolor}%
\pgfsetfillcolor{textcolor}%
\pgftext[x=2.037098in,y=0.501389in,left,base]{\color{textcolor}\sffamily\fontsize{9.000000}{10.800000}\selectfont 2}%
\end{pgfscope}%
\begin{pgfscope}%
\definecolor{textcolor}{rgb}{0.150000,0.150000,0.150000}%
\pgfsetstrokecolor{textcolor}%
\pgfsetfillcolor{textcolor}%
\pgftext[x=3.945578in,y=0.501389in,left,base]{\color{textcolor}\sffamily\fontsize{9.000000}{10.800000}\selectfont 3}%
\end{pgfscope}%
\begin{pgfscope}%
\definecolor{textcolor}{rgb}{0.150000,0.150000,0.150000}%
\pgfsetstrokecolor{textcolor}%
\pgfsetfillcolor{textcolor}%
\pgftext[x=4.034132in,y=0.501389in,left,base]{\color{textcolor}\sffamily\fontsize{9.000000}{10.800000}\selectfont 4}%
\end{pgfscope}%
\begin{pgfscope}%
\definecolor{textcolor}{rgb}{0.150000,0.150000,0.150000}%
\pgfsetstrokecolor{textcolor}%
\pgfsetfillcolor{textcolor}%
\pgftext[x=4.845675in,y=0.501389in,left,base]{\color{textcolor}\sffamily\fontsize{9.000000}{10.800000}\selectfont 5}%
\end{pgfscope}%
\begin{pgfscope}%
\definecolor{textcolor}{rgb}{0.150000,0.150000,0.150000}%
\pgfsetstrokecolor{textcolor}%
\pgfsetfillcolor{textcolor}%
\pgftext[x=5.244792in,y=0.501389in,left,base]{\color{textcolor}\sffamily\fontsize{9.000000}{10.800000}\selectfont 6}%
\end{pgfscope}%
\begin{pgfscope}%
\definecolor{textcolor}{rgb}{0.150000,0.150000,0.150000}%
\pgfsetstrokecolor{textcolor}%
\pgfsetfillcolor{textcolor}%
\pgftext[x=5.757094in,y=0.501389in,left,base]{\color{textcolor}\sffamily\fontsize{9.000000}{10.800000}\selectfont 7}%
\end{pgfscope}%
\begin{pgfscope}%
\pgfsetroundcap%
\pgfsetroundjoin%
\pgfsetlinewidth{0.501875pt}%
\definecolor{currentstroke}{rgb}{0.003922,0.450980,0.698039}%
\pgfsetstrokecolor{currentstroke}%
\pgfsetdash{}{0pt}%
\pgfpathmoveto{\pgfqpoint{0.602700in}{0.600558in}}%
\pgfpathlineto{\pgfqpoint{0.603639in}{0.601013in}}%
\pgfpathlineto{\pgfqpoint{0.606466in}{0.636816in}}%
\pgfpathlineto{\pgfqpoint{0.607414in}{0.641666in}}%
\pgfpathlineto{\pgfqpoint{0.608372in}{0.655211in}}%
\pgfpathlineto{\pgfqpoint{0.610259in}{0.655608in}}%
\pgfpathlineto{\pgfqpoint{0.612138in}{0.661379in}}%
\pgfpathlineto{\pgfqpoint{0.614016in}{0.674020in}}%
\pgfpathlineto{\pgfqpoint{0.687739in}{0.674728in}}%
\pgfpathlineto{\pgfqpoint{0.689617in}{0.675442in}}%
\pgfpathlineto{\pgfqpoint{2.071205in}{0.675869in}}%
\pgfpathlineto{\pgfqpoint{2.072144in}{0.677697in}}%
\pgfpathlineto{\pgfqpoint{2.074032in}{0.682582in}}%
\pgfpathlineto{\pgfqpoint{2.078755in}{0.682822in}}%
\pgfpathlineto{\pgfqpoint{2.084362in}{0.684021in}}%
\pgfpathlineto{\pgfqpoint{2.086231in}{0.683758in}}%
\pgfpathlineto{\pgfqpoint{2.091838in}{0.696537in}}%
\pgfpathlineto{\pgfqpoint{2.094656in}{0.702683in}}%
\pgfpathlineto{\pgfqpoint{2.100254in}{0.716384in}}%
\pgfpathlineto{\pgfqpoint{2.103062in}{0.722446in}}%
\pgfpathlineto{\pgfqpoint{2.103992in}{0.710131in}}%
\pgfpathlineto{\pgfqpoint{2.110538in}{0.710189in}}%
\pgfpathlineto{\pgfqpoint{2.124736in}{0.711366in}}%
\pgfpathlineto{\pgfqpoint{2.130362in}{0.724336in}}%
\pgfpathlineto{\pgfqpoint{2.132240in}{0.727861in}}%
\pgfpathlineto{\pgfqpoint{2.133179in}{0.715885in}}%
\pgfpathlineto{\pgfqpoint{2.151060in}{0.716014in}}%
\pgfpathlineto{\pgfqpoint{2.156704in}{0.716014in}}%
\pgfpathlineto{\pgfqpoint{2.158591in}{0.718800in}}%
\pgfpathlineto{\pgfqpoint{2.164244in}{0.729488in}}%
\pgfpathlineto{\pgfqpoint{2.165202in}{0.716656in}}%
\pgfpathlineto{\pgfqpoint{2.179289in}{0.716745in}}%
\pgfpathlineto{\pgfqpoint{2.187723in}{0.716808in}}%
\pgfpathlineto{\pgfqpoint{2.190531in}{0.722753in}}%
\pgfpathlineto{\pgfqpoint{2.194325in}{0.729537in}}%
\pgfpathlineto{\pgfqpoint{2.206543in}{0.729706in}}%
\pgfpathlineto{\pgfqpoint{2.208430in}{0.729599in}}%
\pgfpathlineto{\pgfqpoint{2.210308in}{0.729501in}}%
\pgfpathlineto{\pgfqpoint{2.219709in}{0.729484in}}%
\pgfpathlineto{\pgfqpoint{2.223466in}{0.729430in}}%
\pgfpathlineto{\pgfqpoint{2.224423in}{0.716669in}}%
\pgfpathlineto{\pgfqpoint{2.230040in}{0.716772in}}%
\pgfpathlineto{\pgfqpoint{2.236623in}{0.716678in}}%
\pgfpathlineto{\pgfqpoint{2.247893in}{0.716986in}}%
\pgfpathlineto{\pgfqpoint{2.252616in}{0.728534in}}%
\pgfpathlineto{\pgfqpoint{2.253555in}{0.707015in}}%
\pgfpathlineto{\pgfqpoint{2.272319in}{0.708076in}}%
\pgfpathlineto{\pgfqpoint{2.278875in}{0.722005in}}%
\pgfpathlineto{\pgfqpoint{2.281702in}{0.728565in}}%
\pgfpathlineto{\pgfqpoint{2.282641in}{0.716620in}}%
\pgfpathlineto{\pgfqpoint{2.288266in}{0.716683in}}%
\pgfpathlineto{\pgfqpoint{2.298560in}{0.716701in}}%
\pgfpathlineto{\pgfqpoint{2.302326in}{0.716959in}}%
\pgfpathlineto{\pgfqpoint{2.305153in}{0.717195in}}%
\pgfpathlineto{\pgfqpoint{2.307022in}{0.720467in}}%
\pgfpathlineto{\pgfqpoint{2.311717in}{0.729862in}}%
\pgfpathlineto{\pgfqpoint{2.312684in}{0.707127in}}%
\pgfpathlineto{\pgfqpoint{2.324875in}{0.706824in}}%
\pgfpathlineto{\pgfqpoint{2.328640in}{0.707011in}}%
\pgfpathlineto{\pgfqpoint{2.331449in}{0.708161in}}%
\pgfpathlineto{\pgfqpoint{2.340822in}{0.728819in}}%
\pgfpathlineto{\pgfqpoint{2.341770in}{0.716723in}}%
\pgfpathlineto{\pgfqpoint{2.352994in}{0.716825in}}%
\pgfpathlineto{\pgfqpoint{2.357689in}{0.716959in}}%
\pgfpathlineto{\pgfqpoint{2.361446in}{0.717151in}}%
\pgfpathlineto{\pgfqpoint{2.366132in}{0.717351in}}%
\pgfpathlineto{\pgfqpoint{2.368941in}{0.722156in}}%
\pgfpathlineto{\pgfqpoint{2.373636in}{0.730785in}}%
\pgfpathlineto{\pgfqpoint{2.378323in}{0.730691in}}%
\pgfpathlineto{\pgfqpoint{2.388635in}{0.730478in}}%
\pgfpathlineto{\pgfqpoint{2.402704in}{0.730580in}}%
\pgfpathlineto{\pgfqpoint{2.403652in}{0.718060in}}%
\pgfpathlineto{\pgfqpoint{2.428079in}{0.718978in}}%
\pgfpathlineto{\pgfqpoint{2.432765in}{0.728904in}}%
\pgfpathlineto{\pgfqpoint{2.433705in}{0.730616in}}%
\pgfpathlineto{\pgfqpoint{2.434662in}{0.721813in}}%
\pgfpathlineto{\pgfqpoint{2.445969in}{0.722129in}}%
\pgfpathlineto{\pgfqpoint{2.454402in}{0.722223in}}%
\pgfpathlineto{\pgfqpoint{2.461934in}{0.723217in}}%
\pgfpathlineto{\pgfqpoint{2.465681in}{0.730790in}}%
\pgfpathlineto{\pgfqpoint{2.466630in}{0.717846in}}%
\pgfpathlineto{\pgfqpoint{2.469438in}{0.717958in}}%
\pgfpathlineto{\pgfqpoint{2.482549in}{0.717922in}}%
\pgfpathlineto{\pgfqpoint{2.491977in}{0.719130in}}%
\pgfpathlineto{\pgfqpoint{2.497631in}{0.729684in}}%
\pgfpathlineto{\pgfqpoint{2.498570in}{0.721795in}}%
\pgfpathlineto{\pgfqpoint{2.507952in}{0.721826in}}%
\pgfpathlineto{\pgfqpoint{2.518283in}{0.722348in}}%
\pgfpathlineto{\pgfqpoint{2.523899in}{0.722406in}}%
\pgfpathlineto{\pgfqpoint{2.527637in}{0.730767in}}%
\pgfpathlineto{\pgfqpoint{2.528585in}{0.718399in}}%
\pgfpathlineto{\pgfqpoint{2.534266in}{0.718363in}}%
\pgfpathlineto{\pgfqpoint{2.552101in}{0.718229in}}%
\pgfpathlineto{\pgfqpoint{2.557717in}{0.731026in}}%
\pgfpathlineto{\pgfqpoint{2.558656in}{0.722477in}}%
\pgfpathlineto{\pgfqpoint{2.579244in}{0.722883in}}%
\pgfpathlineto{\pgfqpoint{2.582061in}{0.722749in}}%
\pgfpathlineto{\pgfqpoint{2.585799in}{0.730691in}}%
\pgfpathlineto{\pgfqpoint{2.586739in}{0.718292in}}%
\pgfpathlineto{\pgfqpoint{2.592355in}{0.718261in}}%
\pgfpathlineto{\pgfqpoint{2.606405in}{0.718292in}}%
\pgfpathlineto{\pgfqpoint{2.610162in}{0.718969in}}%
\pgfpathlineto{\pgfqpoint{2.615788in}{0.730932in}}%
\pgfpathlineto{\pgfqpoint{2.616736in}{0.710176in}}%
\pgfpathlineto{\pgfqpoint{2.627066in}{0.710033in}}%
\pgfpathlineto{\pgfqpoint{2.636439in}{0.711486in}}%
\pgfpathlineto{\pgfqpoint{2.640205in}{0.721095in}}%
\pgfpathlineto{\pgfqpoint{2.642093in}{0.728717in}}%
\pgfpathlineto{\pgfqpoint{2.644910in}{0.734498in}}%
\pgfpathlineto{\pgfqpoint{2.645849in}{0.720681in}}%
\pgfpathlineto{\pgfqpoint{2.649597in}{0.721029in}}%
\pgfpathlineto{\pgfqpoint{2.654311in}{0.720877in}}%
\pgfpathlineto{\pgfqpoint{2.666547in}{0.720676in}}%
\pgfpathlineto{\pgfqpoint{2.668426in}{0.721434in}}%
\pgfpathlineto{\pgfqpoint{2.674060in}{0.733767in}}%
\pgfpathlineto{\pgfqpoint{2.674990in}{0.734422in}}%
\pgfpathlineto{\pgfqpoint{2.675948in}{0.712480in}}%
\pgfpathlineto{\pgfqpoint{2.678756in}{0.712703in}}%
\pgfpathlineto{\pgfqpoint{2.684373in}{0.712565in}}%
\pgfpathlineto{\pgfqpoint{2.693746in}{0.712622in}}%
\pgfpathlineto{\pgfqpoint{2.702189in}{0.732657in}}%
\pgfpathlineto{\pgfqpoint{2.703137in}{0.720930in}}%
\pgfpathlineto{\pgfqpoint{2.705006in}{0.720935in}}%
\pgfpathlineto{\pgfqpoint{2.711589in}{0.720850in}}%
\pgfpathlineto{\pgfqpoint{2.726615in}{0.722201in}}%
\pgfpathlineto{\pgfqpoint{2.733180in}{0.733905in}}%
\pgfpathlineto{\pgfqpoint{2.734129in}{0.712351in}}%
\pgfpathlineto{\pgfqpoint{2.739754in}{0.712342in}}%
\pgfpathlineto{\pgfqpoint{2.748170in}{0.712582in}}%
\pgfpathlineto{\pgfqpoint{2.753805in}{0.712819in}}%
\pgfpathlineto{\pgfqpoint{2.755674in}{0.713202in}}%
\pgfpathlineto{\pgfqpoint{2.764108in}{0.729033in}}%
\pgfpathlineto{\pgfqpoint{2.766916in}{0.733847in}}%
\pgfpathlineto{\pgfqpoint{2.767864in}{0.721229in}}%
\pgfpathlineto{\pgfqpoint{2.773490in}{0.721421in}}%
\pgfpathlineto{\pgfqpoint{2.779143in}{0.721314in}}%
\pgfpathlineto{\pgfqpoint{2.785708in}{0.721113in}}%
\pgfpathlineto{\pgfqpoint{2.792291in}{0.721238in}}%
\pgfpathlineto{\pgfqpoint{2.796038in}{0.729149in}}%
\pgfpathlineto{\pgfqpoint{2.798856in}{0.733896in}}%
\pgfpathlineto{\pgfqpoint{2.800743in}{0.725142in}}%
\pgfpathlineto{\pgfqpoint{2.817593in}{0.725918in}}%
\pgfpathlineto{\pgfqpoint{2.824167in}{0.725931in}}%
\pgfpathlineto{\pgfqpoint{2.826036in}{0.726979in}}%
\pgfpathlineto{\pgfqpoint{2.829792in}{0.734828in}}%
\pgfpathlineto{\pgfqpoint{2.831670in}{0.722504in}}%
\pgfpathlineto{\pgfqpoint{2.841043in}{0.722802in}}%
\pgfpathlineto{\pgfqpoint{2.845730in}{0.722522in}}%
\pgfpathlineto{\pgfqpoint{2.850416in}{0.722490in}}%
\pgfpathlineto{\pgfqpoint{2.855112in}{0.722869in}}%
\pgfpathlineto{\pgfqpoint{2.856051in}{0.723217in}}%
\pgfpathlineto{\pgfqpoint{2.862644in}{0.736463in}}%
\pgfpathlineto{\pgfqpoint{2.877633in}{0.736615in}}%
\pgfpathlineto{\pgfqpoint{2.892641in}{0.736731in}}%
\pgfpathlineto{\pgfqpoint{2.895458in}{0.736642in}}%
\pgfpathlineto{\pgfqpoint{2.896388in}{0.724050in}}%
\pgfpathlineto{\pgfqpoint{2.923568in}{0.723957in}}%
\pgfpathlineto{\pgfqpoint{2.931091in}{0.736815in}}%
\pgfpathlineto{\pgfqpoint{2.936698in}{0.736753in}}%
\pgfpathlineto{\pgfqpoint{2.947940in}{0.737074in}}%
\pgfpathlineto{\pgfqpoint{2.954505in}{0.736958in}}%
\pgfpathlineto{\pgfqpoint{2.959182in}{0.736900in}}%
\pgfpathlineto{\pgfqpoint{2.964817in}{0.736829in}}%
\pgfpathlineto{\pgfqpoint{2.965774in}{0.707680in}}%
\pgfpathlineto{\pgfqpoint{2.978849in}{0.707782in}}%
\pgfpathlineto{\pgfqpoint{2.985395in}{0.707983in}}%
\pgfpathlineto{\pgfqpoint{2.995716in}{0.730023in}}%
\pgfpathlineto{\pgfqpoint{2.998525in}{0.736517in}}%
\pgfpathlineto{\pgfqpoint{2.999473in}{0.723698in}}%
\pgfpathlineto{\pgfqpoint{3.013579in}{0.723912in}}%
\pgfpathlineto{\pgfqpoint{3.017344in}{0.723712in}}%
\pgfpathlineto{\pgfqpoint{3.023937in}{0.724407in}}%
\pgfpathlineto{\pgfqpoint{3.029553in}{0.735202in}}%
\pgfpathlineto{\pgfqpoint{3.031441in}{0.736347in}}%
\pgfpathlineto{\pgfqpoint{3.032389in}{0.707800in}}%
\pgfpathlineto{\pgfqpoint{3.047397in}{0.708197in}}%
\pgfpathlineto{\pgfqpoint{3.056788in}{0.728909in}}%
\pgfpathlineto{\pgfqpoint{3.059606in}{0.735590in}}%
\pgfpathlineto{\pgfqpoint{3.060545in}{0.723859in}}%
\pgfpathlineto{\pgfqpoint{3.063362in}{0.724015in}}%
\pgfpathlineto{\pgfqpoint{3.075580in}{0.723801in}}%
\pgfpathlineto{\pgfqpoint{3.078398in}{0.723952in}}%
\pgfpathlineto{\pgfqpoint{3.085957in}{0.724282in}}%
\pgfpathlineto{\pgfqpoint{3.087863in}{0.727326in}}%
\pgfpathlineto{\pgfqpoint{3.092595in}{0.736579in}}%
\pgfpathlineto{\pgfqpoint{3.093544in}{0.720922in}}%
\pgfpathlineto{\pgfqpoint{3.097300in}{0.721176in}}%
\pgfpathlineto{\pgfqpoint{3.099179in}{0.725566in}}%
\pgfpathlineto{\pgfqpoint{3.101057in}{0.725490in}}%
\pgfpathlineto{\pgfqpoint{3.102935in}{0.723279in}}%
\pgfpathlineto{\pgfqpoint{3.107631in}{0.723364in}}%
\pgfpathlineto{\pgfqpoint{3.116074in}{0.723502in}}%
\pgfpathlineto{\pgfqpoint{3.120770in}{0.733107in}}%
\pgfpathlineto{\pgfqpoint{3.123587in}{0.738683in}}%
\pgfpathlineto{\pgfqpoint{3.124526in}{0.726506in}}%
\pgfpathlineto{\pgfqpoint{3.127353in}{0.726569in}}%
\pgfpathlineto{\pgfqpoint{3.148014in}{0.728213in}}%
\pgfpathlineto{\pgfqpoint{3.154579in}{0.739040in}}%
\pgfpathlineto{\pgfqpoint{3.155527in}{0.710906in}}%
\pgfpathlineto{\pgfqpoint{3.173334in}{0.711771in}}%
\pgfpathlineto{\pgfqpoint{3.186528in}{0.738808in}}%
\pgfpathlineto{\pgfqpoint{3.188415in}{0.727411in}}%
\pgfpathlineto{\pgfqpoint{3.207235in}{0.727674in}}%
\pgfpathlineto{\pgfqpoint{3.210053in}{0.727982in}}%
\pgfpathlineto{\pgfqpoint{3.213837in}{0.728681in}}%
\pgfpathlineto{\pgfqpoint{3.219462in}{0.740319in}}%
\pgfpathlineto{\pgfqpoint{3.220402in}{0.724407in}}%
\pgfpathlineto{\pgfqpoint{3.229848in}{0.724452in}}%
\pgfpathlineto{\pgfqpoint{3.233623in}{0.724692in}}%
\pgfpathlineto{\pgfqpoint{3.244957in}{0.724830in}}%
\pgfpathlineto{\pgfqpoint{3.255362in}{0.741656in}}%
\pgfpathlineto{\pgfqpoint{3.256319in}{0.712658in}}%
\pgfpathlineto{\pgfqpoint{3.260076in}{0.712908in}}%
\pgfpathlineto{\pgfqpoint{3.263841in}{0.712649in}}%
\pgfpathlineto{\pgfqpoint{3.273279in}{0.712957in}}%
\pgfpathlineto{\pgfqpoint{3.275166in}{0.712926in}}%
\pgfpathlineto{\pgfqpoint{3.282707in}{0.727794in}}%
\pgfpathlineto{\pgfqpoint{3.287421in}{0.739378in}}%
\pgfpathlineto{\pgfqpoint{3.288360in}{0.728124in}}%
\pgfpathlineto{\pgfqpoint{3.314684in}{0.728525in}}%
\pgfpathlineto{\pgfqpoint{3.316572in}{0.729252in}}%
\pgfpathlineto{\pgfqpoint{3.320356in}{0.736731in}}%
\pgfpathlineto{\pgfqpoint{3.323155in}{0.741714in}}%
\pgfpathlineto{\pgfqpoint{3.324094in}{0.741794in}}%
\pgfpathlineto{\pgfqpoint{3.325033in}{0.712979in}}%
\pgfpathlineto{\pgfqpoint{3.346652in}{0.713572in}}%
\pgfpathlineto{\pgfqpoint{3.352287in}{0.723631in}}%
\pgfpathlineto{\pgfqpoint{3.356982in}{0.730990in}}%
\pgfpathlineto{\pgfqpoint{3.362626in}{0.740889in}}%
\pgfpathlineto{\pgfqpoint{3.363584in}{0.728539in}}%
\pgfpathlineto{\pgfqpoint{3.373979in}{0.728383in}}%
\pgfpathlineto{\pgfqpoint{3.384319in}{0.728463in}}%
\pgfpathlineto{\pgfqpoint{3.389024in}{0.728824in}}%
\pgfpathlineto{\pgfqpoint{3.392789in}{0.735514in}}%
\pgfpathlineto{\pgfqpoint{3.396574in}{0.741937in}}%
\pgfpathlineto{\pgfqpoint{3.397522in}{0.741803in}}%
\pgfpathlineto{\pgfqpoint{3.400339in}{0.713215in}}%
\pgfpathlineto{\pgfqpoint{3.411618in}{0.713220in}}%
\pgfpathlineto{\pgfqpoint{3.419159in}{0.714441in}}%
\pgfpathlineto{\pgfqpoint{3.432316in}{0.738995in}}%
\pgfpathlineto{\pgfqpoint{3.433255in}{0.740649in}}%
\pgfpathlineto{\pgfqpoint{3.435134in}{0.728757in}}%
\pgfpathlineto{\pgfqpoint{3.449239in}{0.728824in}}%
\pgfpathlineto{\pgfqpoint{3.459579in}{0.729555in}}%
\pgfpathlineto{\pgfqpoint{3.463335in}{0.737484in}}%
\pgfpathlineto{\pgfqpoint{3.467120in}{0.742226in}}%
\pgfpathlineto{\pgfqpoint{3.468077in}{0.742097in}}%
\pgfpathlineto{\pgfqpoint{3.469016in}{0.713064in}}%
\pgfpathlineto{\pgfqpoint{3.471834in}{0.713180in}}%
\pgfpathlineto{\pgfqpoint{3.485000in}{0.713238in}}%
\pgfpathlineto{\pgfqpoint{3.488738in}{0.714026in}}%
\pgfpathlineto{\pgfqpoint{3.491556in}{0.720084in}}%
\pgfpathlineto{\pgfqpoint{3.493434in}{0.726203in}}%
\pgfpathlineto{\pgfqpoint{3.494373in}{0.727638in}}%
\pgfpathlineto{\pgfqpoint{3.495312in}{0.727121in}}%
\pgfpathlineto{\pgfqpoint{3.502816in}{0.741469in}}%
\pgfpathlineto{\pgfqpoint{3.504713in}{0.728811in}}%
\pgfpathlineto{\pgfqpoint{3.515973in}{0.728819in}}%
\pgfpathlineto{\pgfqpoint{3.519739in}{0.728846in}}%
\pgfpathlineto{\pgfqpoint{3.525392in}{0.728944in}}%
\pgfpathlineto{\pgfqpoint{3.529149in}{0.729715in}}%
\pgfpathlineto{\pgfqpoint{3.535714in}{0.741892in}}%
\pgfpathlineto{\pgfqpoint{3.537592in}{0.741660in}}%
\pgfpathlineto{\pgfqpoint{3.538531in}{0.713242in}}%
\pgfpathlineto{\pgfqpoint{3.555445in}{0.713398in}}%
\pgfpathlineto{\pgfqpoint{3.557314in}{0.713367in}}%
\pgfpathlineto{\pgfqpoint{3.565766in}{0.729149in}}%
\pgfpathlineto{\pgfqpoint{3.571392in}{0.740666in}}%
\pgfpathlineto{\pgfqpoint{3.572331in}{0.741861in}}%
\pgfpathlineto{\pgfqpoint{3.573298in}{0.729131in}}%
\pgfpathlineto{\pgfqpoint{3.581741in}{0.729025in}}%
\pgfpathlineto{\pgfqpoint{3.585498in}{0.729377in}}%
\pgfpathlineto{\pgfqpoint{3.588315in}{0.729074in}}%
\pgfpathlineto{\pgfqpoint{3.592081in}{0.729323in}}%
\pgfpathlineto{\pgfqpoint{3.595856in}{0.729800in}}%
\pgfpathlineto{\pgfqpoint{3.597734in}{0.729595in}}%
\pgfpathlineto{\pgfqpoint{3.604290in}{0.741192in}}%
\pgfpathlineto{\pgfqpoint{3.605238in}{0.726007in}}%
\pgfpathlineto{\pgfqpoint{3.610873in}{0.726078in}}%
\pgfpathlineto{\pgfqpoint{3.616498in}{0.726310in}}%
\pgfpathlineto{\pgfqpoint{3.621213in}{0.726849in}}%
\pgfpathlineto{\pgfqpoint{3.629683in}{0.726564in}}%
\pgfpathlineto{\pgfqpoint{3.631571in}{0.727068in}}%
\pgfpathlineto{\pgfqpoint{3.636285in}{0.736392in}}%
\pgfpathlineto{\pgfqpoint{3.638163in}{0.742529in}}%
\pgfpathlineto{\pgfqpoint{3.639111in}{0.744384in}}%
\pgfpathlineto{\pgfqpoint{3.640051in}{0.728048in}}%
\pgfpathlineto{\pgfqpoint{3.649460in}{0.727910in}}%
\pgfpathlineto{\pgfqpoint{3.655114in}{0.727919in}}%
\pgfpathlineto{\pgfqpoint{3.663612in}{0.728253in}}%
\pgfpathlineto{\pgfqpoint{3.668317in}{0.728565in}}%
\pgfpathlineto{\pgfqpoint{3.674909in}{0.741237in}}%
\pgfpathlineto{\pgfqpoint{3.675849in}{0.724933in}}%
\pgfpathlineto{\pgfqpoint{3.694659in}{0.725325in}}%
\pgfpathlineto{\pgfqpoint{3.696565in}{0.725597in}}%
\pgfpathlineto{\pgfqpoint{3.704124in}{0.740724in}}%
\pgfpathlineto{\pgfqpoint{3.705072in}{0.740760in}}%
\pgfpathlineto{\pgfqpoint{3.706002in}{0.712382in}}%
\pgfpathlineto{\pgfqpoint{3.709759in}{0.712658in}}%
\pgfpathlineto{\pgfqpoint{3.715394in}{0.712435in}}%
\pgfpathlineto{\pgfqpoint{3.722916in}{0.712480in}}%
\pgfpathlineto{\pgfqpoint{3.734195in}{0.737043in}}%
\pgfpathlineto{\pgfqpoint{3.736073in}{0.740858in}}%
\pgfpathlineto{\pgfqpoint{3.737022in}{0.728164in}}%
\pgfpathlineto{\pgfqpoint{3.740760in}{0.728356in}}%
\pgfpathlineto{\pgfqpoint{3.749184in}{0.728351in}}%
\pgfpathlineto{\pgfqpoint{3.759506in}{0.728614in}}%
\pgfpathlineto{\pgfqpoint{3.765131in}{0.741536in}}%
\pgfpathlineto{\pgfqpoint{3.767940in}{0.742044in}}%
\pgfpathlineto{\pgfqpoint{3.768897in}{0.709248in}}%
\pgfpathlineto{\pgfqpoint{3.785829in}{0.710942in}}%
\pgfpathlineto{\pgfqpoint{3.787708in}{0.714922in}}%
\pgfpathlineto{\pgfqpoint{3.800856in}{0.741130in}}%
\pgfpathlineto{\pgfqpoint{3.801813in}{0.728775in}}%
\pgfpathlineto{\pgfqpoint{3.808396in}{0.728877in}}%
\pgfpathlineto{\pgfqpoint{3.815992in}{0.728753in}}%
\pgfpathlineto{\pgfqpoint{3.818810in}{0.728686in}}%
\pgfpathlineto{\pgfqpoint{3.827317in}{0.729755in}}%
\pgfpathlineto{\pgfqpoint{3.833919in}{0.742235in}}%
\pgfpathlineto{\pgfqpoint{3.835788in}{0.742204in}}%
\pgfpathlineto{\pgfqpoint{3.836746in}{0.710028in}}%
\pgfpathlineto{\pgfqpoint{3.851836in}{0.710416in}}%
\pgfpathlineto{\pgfqpoint{3.852775in}{0.710104in}}%
\pgfpathlineto{\pgfqpoint{3.854672in}{0.712796in}}%
\pgfpathlineto{\pgfqpoint{3.861246in}{0.726007in}}%
\pgfpathlineto{\pgfqpoint{3.867839in}{0.740880in}}%
\pgfpathlineto{\pgfqpoint{3.868778in}{0.729925in}}%
\pgfpathlineto{\pgfqpoint{3.873501in}{0.729903in}}%
\pgfpathlineto{\pgfqpoint{3.884798in}{0.730018in}}%
\pgfpathlineto{\pgfqpoint{3.893278in}{0.730357in}}%
\pgfpathlineto{\pgfqpoint{3.897035in}{0.738113in}}%
\pgfpathlineto{\pgfqpoint{3.899843in}{0.743122in}}%
\pgfpathlineto{\pgfqpoint{3.901721in}{0.743104in}}%
\pgfpathlineto{\pgfqpoint{3.902660in}{0.710265in}}%
\pgfpathlineto{\pgfqpoint{3.910174in}{0.710251in}}%
\pgfpathlineto{\pgfqpoint{3.916766in}{0.710100in}}%
\pgfpathlineto{\pgfqpoint{3.920513in}{0.719362in}}%
\pgfpathlineto{\pgfqpoint{3.928956in}{0.741504in}}%
\pgfpathlineto{\pgfqpoint{3.930872in}{0.730393in}}%
\pgfpathlineto{\pgfqpoint{3.940254in}{0.730433in}}%
\pgfpathlineto{\pgfqpoint{3.947758in}{0.730241in}}%
\pgfpathlineto{\pgfqpoint{3.948697in}{0.731298in}}%
\pgfpathlineto{\pgfqpoint{3.953374in}{0.742645in}}%
\pgfpathlineto{\pgfqpoint{3.956201in}{0.742819in}}%
\pgfpathlineto{\pgfqpoint{3.957140in}{0.710523in}}%
\pgfpathlineto{\pgfqpoint{3.968400in}{0.710532in}}%
\pgfpathlineto{\pgfqpoint{3.971236in}{0.710465in}}%
\pgfpathlineto{\pgfqpoint{3.972175in}{0.710425in}}%
\pgfpathlineto{\pgfqpoint{3.977829in}{0.723533in}}%
\pgfpathlineto{\pgfqpoint{3.985351in}{0.742316in}}%
\pgfpathlineto{\pgfqpoint{3.986290in}{0.730330in}}%
\pgfpathlineto{\pgfqpoint{4.006942in}{0.730504in}}%
\pgfpathlineto{\pgfqpoint{4.011638in}{0.743069in}}%
\pgfpathlineto{\pgfqpoint{4.014455in}{0.743867in}}%
\pgfpathlineto{\pgfqpoint{4.015413in}{0.711334in}}%
\pgfpathlineto{\pgfqpoint{4.020099in}{0.711196in}}%
\pgfpathlineto{\pgfqpoint{4.023902in}{0.749710in}}%
\pgfpathlineto{\pgfqpoint{4.025789in}{0.777192in}}%
\pgfpathlineto{\pgfqpoint{4.026747in}{0.751724in}}%
\pgfpathlineto{\pgfqpoint{4.028644in}{0.762163in}}%
\pgfpathlineto{\pgfqpoint{4.030513in}{0.787537in}}%
\pgfpathlineto{\pgfqpoint{4.031452in}{0.792435in}}%
\pgfpathlineto{\pgfqpoint{4.033349in}{0.806266in}}%
\pgfpathlineto{\pgfqpoint{4.040880in}{0.859488in}}%
\pgfpathlineto{\pgfqpoint{4.041847in}{0.851960in}}%
\pgfpathlineto{\pgfqpoint{4.044664in}{0.861881in}}%
\pgfpathlineto{\pgfqpoint{4.046543in}{0.862728in}}%
\pgfpathlineto{\pgfqpoint{4.047482in}{0.862871in}}%
\pgfpathlineto{\pgfqpoint{4.048430in}{0.866699in}}%
\pgfpathlineto{\pgfqpoint{4.049369in}{0.861262in}}%
\pgfpathlineto{\pgfqpoint{4.050299in}{0.863111in}}%
\pgfpathlineto{\pgfqpoint{4.052177in}{0.875827in}}%
\pgfpathlineto{\pgfqpoint{4.053117in}{0.880249in}}%
\pgfpathlineto{\pgfqpoint{4.054995in}{0.894765in}}%
\pgfpathlineto{\pgfqpoint{4.055934in}{0.896606in}}%
\pgfpathlineto{\pgfqpoint{4.056873in}{0.903167in}}%
\pgfpathlineto{\pgfqpoint{4.063475in}{0.996302in}}%
\pgfpathlineto{\pgfqpoint{4.064414in}{0.998584in}}%
\pgfpathlineto{\pgfqpoint{4.065362in}{0.897814in}}%
\pgfpathlineto{\pgfqpoint{4.066301in}{0.897814in}}%
\pgfpathlineto{\pgfqpoint{4.067250in}{0.888579in}}%
\pgfpathlineto{\pgfqpoint{4.068189in}{0.889292in}}%
\pgfpathlineto{\pgfqpoint{4.070067in}{0.876670in}}%
\pgfpathlineto{\pgfqpoint{4.072875in}{0.877325in}}%
\pgfpathlineto{\pgfqpoint{4.073814in}{0.878034in}}%
\pgfpathlineto{\pgfqpoint{4.075693in}{0.881528in}}%
\pgfpathlineto{\pgfqpoint{4.078547in}{0.909995in}}%
\pgfpathlineto{\pgfqpoint{4.079495in}{0.881622in}}%
\pgfpathlineto{\pgfqpoint{4.081392in}{0.881702in}}%
\pgfpathlineto{\pgfqpoint{4.083280in}{0.884554in}}%
\pgfpathlineto{\pgfqpoint{4.085204in}{0.885918in}}%
\pgfpathlineto{\pgfqpoint{4.088979in}{0.893500in}}%
\pgfpathlineto{\pgfqpoint{4.091815in}{0.919467in}}%
\pgfpathlineto{\pgfqpoint{4.094623in}{0.945982in}}%
\pgfpathlineto{\pgfqpoint{4.096510in}{0.959848in}}%
\pgfpathlineto{\pgfqpoint{4.098398in}{0.973522in}}%
\pgfpathlineto{\pgfqpoint{4.099355in}{0.905828in}}%
\pgfpathlineto{\pgfqpoint{4.101252in}{0.908101in}}%
\pgfpathlineto{\pgfqpoint{4.105036in}{0.906911in}}%
\pgfpathlineto{\pgfqpoint{4.108811in}{0.914871in}}%
\pgfpathlineto{\pgfqpoint{4.114483in}{0.963578in}}%
\pgfpathlineto{\pgfqpoint{4.115431in}{0.921584in}}%
\pgfpathlineto{\pgfqpoint{4.117319in}{0.921789in}}%
\pgfpathlineto{\pgfqpoint{4.120155in}{0.927200in}}%
\pgfpathlineto{\pgfqpoint{4.122033in}{0.927596in}}%
\pgfpathlineto{\pgfqpoint{4.122991in}{0.891556in}}%
\pgfpathlineto{\pgfqpoint{4.124887in}{0.893945in}}%
\pgfpathlineto{\pgfqpoint{4.129611in}{0.893825in}}%
\pgfpathlineto{\pgfqpoint{4.131507in}{0.893767in}}%
\pgfpathlineto{\pgfqpoint{4.132446in}{0.892007in}}%
\pgfpathlineto{\pgfqpoint{4.133386in}{0.895184in}}%
\pgfpathlineto{\pgfqpoint{4.137170in}{0.935352in}}%
\pgfpathlineto{\pgfqpoint{4.138118in}{0.940557in}}%
\pgfpathlineto{\pgfqpoint{4.139057in}{0.957454in}}%
\pgfpathlineto{\pgfqpoint{4.140024in}{0.957107in}}%
\pgfpathlineto{\pgfqpoint{4.140972in}{0.926108in}}%
\pgfpathlineto{\pgfqpoint{4.143799in}{0.932780in}}%
\pgfpathlineto{\pgfqpoint{4.144729in}{0.941908in}}%
\pgfpathlineto{\pgfqpoint{4.145686in}{0.941480in}}%
\pgfpathlineto{\pgfqpoint{4.146644in}{0.943620in}}%
\pgfpathlineto{\pgfqpoint{4.147602in}{0.948897in}}%
\pgfpathlineto{\pgfqpoint{4.148559in}{0.931385in}}%
\pgfpathlineto{\pgfqpoint{4.149498in}{0.934523in}}%
\pgfpathlineto{\pgfqpoint{4.150428in}{0.934309in}}%
\pgfpathlineto{\pgfqpoint{4.153246in}{0.942198in}}%
\pgfpathlineto{\pgfqpoint{4.155133in}{0.939911in}}%
\pgfpathlineto{\pgfqpoint{4.156072in}{0.940005in}}%
\pgfpathlineto{\pgfqpoint{4.158890in}{0.947060in}}%
\pgfpathlineto{\pgfqpoint{4.160768in}{0.948157in}}%
\pgfpathlineto{\pgfqpoint{4.162646in}{0.952052in}}%
\pgfpathlineto{\pgfqpoint{4.166449in}{0.993521in}}%
\pgfpathlineto{\pgfqpoint{4.168364in}{0.933230in}}%
\pgfpathlineto{\pgfqpoint{4.169303in}{0.935548in}}%
\pgfpathlineto{\pgfqpoint{4.171209in}{0.939933in}}%
\pgfpathlineto{\pgfqpoint{4.173087in}{0.946994in}}%
\pgfpathlineto{\pgfqpoint{4.174036in}{0.949703in}}%
\pgfpathlineto{\pgfqpoint{4.175002in}{0.917653in}}%
\pgfpathlineto{\pgfqpoint{4.176918in}{0.917456in}}%
\pgfpathlineto{\pgfqpoint{4.179790in}{0.922524in}}%
\pgfpathlineto{\pgfqpoint{4.180739in}{0.922667in}}%
\pgfpathlineto{\pgfqpoint{4.181687in}{0.920313in}}%
\pgfpathlineto{\pgfqpoint{4.183565in}{0.925016in}}%
\pgfpathlineto{\pgfqpoint{4.184514in}{0.927155in}}%
\pgfpathlineto{\pgfqpoint{4.186392in}{0.934607in}}%
\pgfpathlineto{\pgfqpoint{4.187331in}{0.935566in}}%
\pgfpathlineto{\pgfqpoint{4.192054in}{0.958025in}}%
\pgfpathlineto{\pgfqpoint{4.193951in}{0.979882in}}%
\pgfpathlineto{\pgfqpoint{4.194909in}{0.935494in}}%
\pgfpathlineto{\pgfqpoint{4.195848in}{0.935494in}}%
\pgfpathlineto{\pgfqpoint{4.197726in}{0.938066in}}%
\pgfpathlineto{\pgfqpoint{4.199613in}{0.925096in}}%
\pgfpathlineto{\pgfqpoint{4.200553in}{0.928777in}}%
\pgfpathlineto{\pgfqpoint{4.201501in}{0.925680in}}%
\pgfpathlineto{\pgfqpoint{4.202449in}{0.927939in}}%
\pgfpathlineto{\pgfqpoint{4.204318in}{0.941444in}}%
\pgfpathlineto{\pgfqpoint{4.208103in}{0.956438in}}%
\pgfpathlineto{\pgfqpoint{4.209999in}{0.963119in}}%
\pgfpathlineto{\pgfqpoint{4.210938in}{0.963471in}}%
\pgfpathlineto{\pgfqpoint{4.211887in}{0.961697in}}%
\pgfpathlineto{\pgfqpoint{4.214723in}{0.972149in}}%
\pgfpathlineto{\pgfqpoint{4.218525in}{0.991649in}}%
\pgfpathlineto{\pgfqpoint{4.219464in}{0.997185in}}%
\pgfpathlineto{\pgfqpoint{4.220403in}{1.011131in}}%
\pgfpathlineto{\pgfqpoint{4.221361in}{0.941667in}}%
\pgfpathlineto{\pgfqpoint{4.222309in}{0.941953in}}%
\pgfpathlineto{\pgfqpoint{4.224206in}{0.953211in}}%
\pgfpathlineto{\pgfqpoint{4.227042in}{0.954308in}}%
\pgfpathlineto{\pgfqpoint{4.228939in}{0.957258in}}%
\pgfpathlineto{\pgfqpoint{4.229887in}{0.948562in}}%
\pgfpathlineto{\pgfqpoint{4.230845in}{0.950987in}}%
\pgfpathlineto{\pgfqpoint{4.231793in}{0.959518in}}%
\pgfpathlineto{\pgfqpoint{4.233680in}{0.959286in}}%
\pgfpathlineto{\pgfqpoint{4.236535in}{0.966787in}}%
\pgfpathlineto{\pgfqpoint{4.237474in}{0.964439in}}%
\pgfpathlineto{\pgfqpoint{4.238422in}{0.964621in}}%
\pgfpathlineto{\pgfqpoint{4.240319in}{0.969110in}}%
\pgfpathlineto{\pgfqpoint{4.243145in}{0.985997in}}%
\pgfpathlineto{\pgfqpoint{4.244094in}{0.956853in}}%
\pgfpathlineto{\pgfqpoint{4.245033in}{0.956808in}}%
\pgfpathlineto{\pgfqpoint{4.247878in}{0.972648in}}%
\pgfpathlineto{\pgfqpoint{4.249765in}{0.982565in}}%
\pgfpathlineto{\pgfqpoint{4.250732in}{0.960151in}}%
\pgfpathlineto{\pgfqpoint{4.251681in}{0.964817in}}%
\pgfpathlineto{\pgfqpoint{4.253568in}{0.962526in}}%
\pgfpathlineto{\pgfqpoint{4.255456in}{0.970032in}}%
\pgfpathlineto{\pgfqpoint{4.256395in}{0.967781in}}%
\pgfpathlineto{\pgfqpoint{4.257352in}{0.949641in}}%
\pgfpathlineto{\pgfqpoint{4.259240in}{0.951910in}}%
\pgfpathlineto{\pgfqpoint{4.260188in}{0.952043in}}%
\pgfpathlineto{\pgfqpoint{4.261127in}{0.954134in}}%
\pgfpathlineto{\pgfqpoint{4.263024in}{0.959607in}}%
\pgfpathlineto{\pgfqpoint{4.263963in}{0.960160in}}%
\pgfpathlineto{\pgfqpoint{4.264902in}{0.967073in}}%
\pgfpathlineto{\pgfqpoint{4.265878in}{0.958346in}}%
\pgfpathlineto{\pgfqpoint{4.266817in}{0.960485in}}%
\pgfpathlineto{\pgfqpoint{4.267756in}{0.965334in}}%
\pgfpathlineto{\pgfqpoint{4.268705in}{0.963124in}}%
\pgfpathlineto{\pgfqpoint{4.269653in}{0.964033in}}%
\pgfpathlineto{\pgfqpoint{4.271531in}{0.967349in}}%
\pgfpathlineto{\pgfqpoint{4.272498in}{0.967349in}}%
\pgfpathlineto{\pgfqpoint{4.273456in}{0.968347in}}%
\pgfpathlineto{\pgfqpoint{4.274413in}{0.966235in}}%
\pgfpathlineto{\pgfqpoint{4.275362in}{0.966618in}}%
\pgfpathlineto{\pgfqpoint{4.276319in}{0.968249in}}%
\pgfpathlineto{\pgfqpoint{4.278207in}{0.976522in}}%
\pgfpathlineto{\pgfqpoint{4.279146in}{0.976811in}}%
\pgfpathlineto{\pgfqpoint{4.280085in}{0.974690in}}%
\pgfpathlineto{\pgfqpoint{4.282921in}{0.985463in}}%
\pgfpathlineto{\pgfqpoint{4.283869in}{0.959126in}}%
\pgfpathlineto{\pgfqpoint{4.284818in}{0.958930in}}%
\pgfpathlineto{\pgfqpoint{4.286696in}{0.967928in}}%
\pgfpathlineto{\pgfqpoint{4.287653in}{0.966261in}}%
\pgfpathlineto{\pgfqpoint{4.288593in}{0.973941in}}%
\pgfpathlineto{\pgfqpoint{4.289541in}{0.967394in}}%
\pgfpathlineto{\pgfqpoint{4.291428in}{0.967616in}}%
\pgfpathlineto{\pgfqpoint{4.292386in}{0.964135in}}%
\pgfpathlineto{\pgfqpoint{4.293325in}{0.969047in}}%
\pgfpathlineto{\pgfqpoint{4.295213in}{0.966756in}}%
\pgfpathlineto{\pgfqpoint{4.298067in}{0.971427in}}%
\pgfpathlineto{\pgfqpoint{4.299006in}{0.971793in}}%
\pgfpathlineto{\pgfqpoint{4.299954in}{0.959161in}}%
\pgfpathlineto{\pgfqpoint{4.301860in}{0.960026in}}%
\pgfpathlineto{\pgfqpoint{4.305617in}{0.975800in}}%
\pgfpathlineto{\pgfqpoint{4.308462in}{0.988502in}}%
\pgfpathlineto{\pgfqpoint{4.309438in}{0.966079in}}%
\pgfpathlineto{\pgfqpoint{4.313185in}{0.982664in}}%
\pgfpathlineto{\pgfqpoint{4.314134in}{0.981759in}}%
\pgfpathlineto{\pgfqpoint{4.316049in}{0.984972in}}%
\pgfpathlineto{\pgfqpoint{4.317955in}{0.988912in}}%
\pgfpathlineto{\pgfqpoint{4.318903in}{0.980769in}}%
\pgfpathlineto{\pgfqpoint{4.319860in}{0.980988in}}%
\pgfpathlineto{\pgfqpoint{4.322705in}{0.992295in}}%
\pgfpathlineto{\pgfqpoint{4.323654in}{1.003086in}}%
\pgfpathlineto{\pgfqpoint{4.325569in}{1.001517in}}%
\pgfpathlineto{\pgfqpoint{4.326536in}{0.992246in}}%
\pgfpathlineto{\pgfqpoint{4.328423in}{0.992340in}}%
\pgfpathlineto{\pgfqpoint{4.330348in}{0.999106in}}%
\pgfpathlineto{\pgfqpoint{4.331287in}{0.998473in}}%
\pgfpathlineto{\pgfqpoint{4.333183in}{0.996730in}}%
\pgfpathlineto{\pgfqpoint{4.334122in}{0.995785in}}%
\pgfpathlineto{\pgfqpoint{4.336968in}{1.000327in}}%
\pgfpathlineto{\pgfqpoint{4.337916in}{1.000256in}}%
\pgfpathlineto{\pgfqpoint{4.339794in}{1.004949in}}%
\pgfpathlineto{\pgfqpoint{4.341691in}{0.975144in}}%
\pgfpathlineto{\pgfqpoint{4.346414in}{0.983470in}}%
\pgfpathlineto{\pgfqpoint{4.348311in}{0.989313in}}%
\pgfpathlineto{\pgfqpoint{4.349259in}{0.981901in}}%
\pgfpathlineto{\pgfqpoint{4.350217in}{0.983894in}}%
\pgfpathlineto{\pgfqpoint{4.352104in}{0.993579in}}%
\pgfpathlineto{\pgfqpoint{4.353973in}{0.997104in}}%
\pgfpathlineto{\pgfqpoint{4.355861in}{1.002756in}}%
\pgfpathlineto{\pgfqpoint{4.356809in}{1.002738in}}%
\pgfpathlineto{\pgfqpoint{4.357776in}{1.008296in}}%
\pgfpathlineto{\pgfqpoint{4.359682in}{1.004873in}}%
\pgfpathlineto{\pgfqpoint{4.361560in}{1.014487in}}%
\pgfpathlineto{\pgfqpoint{4.362499in}{1.013814in}}%
\pgfpathlineto{\pgfqpoint{4.364368in}{1.017464in}}%
\pgfpathlineto{\pgfqpoint{4.366256in}{1.024832in}}%
\pgfpathlineto{\pgfqpoint{4.367204in}{1.028099in}}%
\pgfpathlineto{\pgfqpoint{4.368162in}{0.991431in}}%
\pgfpathlineto{\pgfqpoint{4.371016in}{1.000617in}}%
\pgfpathlineto{\pgfqpoint{4.373861in}{1.014135in}}%
\pgfpathlineto{\pgfqpoint{4.374809in}{1.014705in}}%
\pgfpathlineto{\pgfqpoint{4.376706in}{1.022791in}}%
\pgfpathlineto{\pgfqpoint{4.377654in}{1.017469in}}%
\pgfpathlineto{\pgfqpoint{4.378594in}{1.000394in}}%
\pgfpathlineto{\pgfqpoint{4.379542in}{1.000238in}}%
\pgfpathlineto{\pgfqpoint{4.381439in}{1.009771in}}%
\pgfpathlineto{\pgfqpoint{4.386217in}{1.015579in}}%
\pgfpathlineto{\pgfqpoint{4.387193in}{1.012049in}}%
\pgfpathlineto{\pgfqpoint{4.389090in}{1.012147in}}%
\pgfpathlineto{\pgfqpoint{4.390038in}{1.010017in}}%
\pgfpathlineto{\pgfqpoint{4.390977in}{1.010418in}}%
\pgfpathlineto{\pgfqpoint{4.392874in}{1.020388in}}%
\pgfpathlineto{\pgfqpoint{4.394771in}{1.019702in}}%
\pgfpathlineto{\pgfqpoint{4.395747in}{0.987187in}}%
\pgfpathlineto{\pgfqpoint{4.396677in}{0.987094in}}%
\pgfpathlineto{\pgfqpoint{4.401437in}{1.004209in}}%
\pgfpathlineto{\pgfqpoint{4.404310in}{1.009629in}}%
\pgfpathlineto{\pgfqpoint{4.405267in}{1.008256in}}%
\pgfpathlineto{\pgfqpoint{4.406206in}{1.008969in}}%
\pgfpathlineto{\pgfqpoint{4.409963in}{1.029343in}}%
\pgfpathlineto{\pgfqpoint{4.411878in}{0.996373in}}%
\pgfpathlineto{\pgfqpoint{4.412826in}{0.997644in}}%
\pgfpathlineto{\pgfqpoint{4.413765in}{0.996944in}}%
\pgfpathlineto{\pgfqpoint{4.416610in}{1.002604in}}%
\pgfpathlineto{\pgfqpoint{4.417568in}{1.002506in}}%
\pgfpathlineto{\pgfqpoint{4.418526in}{1.006834in}}%
\pgfpathlineto{\pgfqpoint{4.420441in}{1.004566in}}%
\pgfpathlineto{\pgfqpoint{4.424206in}{1.020486in}}%
\pgfpathlineto{\pgfqpoint{4.426085in}{1.001660in}}%
\pgfpathlineto{\pgfqpoint{4.427033in}{1.010645in}}%
\pgfpathlineto{\pgfqpoint{4.427991in}{1.007779in}}%
\pgfpathlineto{\pgfqpoint{4.431766in}{1.021779in}}%
\pgfpathlineto{\pgfqpoint{4.433653in}{1.016885in}}%
\pgfpathlineto{\pgfqpoint{4.434583in}{1.019310in}}%
\pgfpathlineto{\pgfqpoint{4.435522in}{1.024961in}}%
\pgfpathlineto{\pgfqpoint{4.437400in}{1.024854in}}%
\pgfpathlineto{\pgfqpoint{4.439306in}{1.033693in}}%
\pgfpathlineto{\pgfqpoint{4.440273in}{0.998214in}}%
\pgfpathlineto{\pgfqpoint{4.441212in}{0.999342in}}%
\pgfpathlineto{\pgfqpoint{4.442142in}{1.010324in}}%
\pgfpathlineto{\pgfqpoint{4.443100in}{1.009134in}}%
\pgfpathlineto{\pgfqpoint{4.444048in}{1.016239in}}%
\pgfpathlineto{\pgfqpoint{4.444987in}{1.015958in}}%
\pgfpathlineto{\pgfqpoint{4.445936in}{1.018088in}}%
\pgfpathlineto{\pgfqpoint{4.446893in}{1.007623in}}%
\pgfpathlineto{\pgfqpoint{4.447860in}{1.008060in}}%
\pgfpathlineto{\pgfqpoint{4.451635in}{1.019114in}}%
\pgfpathlineto{\pgfqpoint{4.453513in}{1.019042in}}%
\pgfpathlineto{\pgfqpoint{4.454462in}{1.023455in}}%
\pgfpathlineto{\pgfqpoint{4.456367in}{1.020928in}}%
\pgfpathlineto{\pgfqpoint{4.459194in}{1.020856in}}%
\pgfpathlineto{\pgfqpoint{4.461100in}{1.025309in}}%
\pgfpathlineto{\pgfqpoint{4.463973in}{1.021467in}}%
\pgfpathlineto{\pgfqpoint{4.464921in}{1.021690in}}%
\pgfpathlineto{\pgfqpoint{4.465851in}{1.026169in}}%
\pgfpathlineto{\pgfqpoint{4.467738in}{1.026107in}}%
\pgfpathlineto{\pgfqpoint{4.469635in}{1.037450in}}%
\pgfpathlineto{\pgfqpoint{4.470583in}{1.028197in}}%
\pgfpathlineto{\pgfqpoint{4.472453in}{1.034999in}}%
\pgfpathlineto{\pgfqpoint{4.473401in}{1.032810in}}%
\pgfpathlineto{\pgfqpoint{4.475279in}{1.033180in}}%
\pgfpathlineto{\pgfqpoint{4.476218in}{1.034276in}}%
\pgfpathlineto{\pgfqpoint{4.478097in}{1.040882in}}%
\pgfpathlineto{\pgfqpoint{4.479984in}{1.005163in}}%
\pgfpathlineto{\pgfqpoint{4.481853in}{1.015089in}}%
\pgfpathlineto{\pgfqpoint{4.486567in}{1.042825in}}%
\pgfpathlineto{\pgfqpoint{4.487516in}{1.020205in}}%
\pgfpathlineto{\pgfqpoint{4.492221in}{1.038956in}}%
\pgfpathlineto{\pgfqpoint{4.493160in}{1.037597in}}%
\pgfpathlineto{\pgfqpoint{4.494136in}{1.039166in}}%
\pgfpathlineto{\pgfqpoint{4.495093in}{1.031237in}}%
\pgfpathlineto{\pgfqpoint{4.496042in}{1.035801in}}%
\pgfpathlineto{\pgfqpoint{4.496981in}{1.033452in}}%
\pgfpathlineto{\pgfqpoint{4.498868in}{1.033479in}}%
\pgfpathlineto{\pgfqpoint{4.499817in}{1.035903in}}%
\pgfpathlineto{\pgfqpoint{4.500774in}{1.035725in}}%
\pgfpathlineto{\pgfqpoint{4.501732in}{1.018030in}}%
\pgfpathlineto{\pgfqpoint{4.502671in}{1.024689in}}%
\pgfpathlineto{\pgfqpoint{4.504549in}{1.020170in}}%
\pgfpathlineto{\pgfqpoint{4.505488in}{1.020312in}}%
\pgfpathlineto{\pgfqpoint{4.506427in}{1.023735in}}%
\pgfpathlineto{\pgfqpoint{4.507376in}{1.022773in}}%
\pgfpathlineto{\pgfqpoint{4.509291in}{1.026766in}}%
\pgfpathlineto{\pgfqpoint{4.510221in}{1.026174in}}%
\pgfpathlineto{\pgfqpoint{4.513029in}{1.033858in}}%
\pgfpathlineto{\pgfqpoint{4.513977in}{1.031518in}}%
\pgfpathlineto{\pgfqpoint{4.514917in}{1.033800in}}%
\pgfpathlineto{\pgfqpoint{4.515865in}{1.006175in}}%
\pgfpathlineto{\pgfqpoint{4.516804in}{1.005956in}}%
\pgfpathlineto{\pgfqpoint{4.517734in}{1.007810in}}%
\pgfpathlineto{\pgfqpoint{4.522420in}{1.028375in}}%
\pgfpathlineto{\pgfqpoint{4.523406in}{1.028803in}}%
\pgfpathlineto{\pgfqpoint{4.524345in}{1.009959in}}%
\pgfpathlineto{\pgfqpoint{4.528138in}{1.030608in}}%
\pgfpathlineto{\pgfqpoint{4.529086in}{1.032177in}}%
\pgfpathlineto{\pgfqpoint{4.530992in}{1.044528in}}%
\pgfpathlineto{\pgfqpoint{4.531959in}{1.020027in}}%
\pgfpathlineto{\pgfqpoint{4.536682in}{1.036469in}}%
\pgfpathlineto{\pgfqpoint{4.537649in}{1.016644in}}%
\pgfpathlineto{\pgfqpoint{4.539555in}{1.019483in}}%
\pgfpathlineto{\pgfqpoint{4.541452in}{1.027011in}}%
\pgfpathlineto{\pgfqpoint{4.542409in}{1.026490in}}%
\pgfpathlineto{\pgfqpoint{4.545245in}{1.036420in}}%
\pgfpathlineto{\pgfqpoint{4.546184in}{1.038916in}}%
\pgfpathlineto{\pgfqpoint{4.548090in}{1.039380in}}%
\pgfpathlineto{\pgfqpoint{4.551865in}{1.047242in}}%
\pgfpathlineto{\pgfqpoint{4.554692in}{1.055826in}}%
\pgfpathlineto{\pgfqpoint{4.556570in}{1.047059in}}%
\pgfpathlineto{\pgfqpoint{4.558458in}{1.045392in}}%
\pgfpathlineto{\pgfqpoint{4.563190in}{1.046021in}}%
\pgfpathlineto{\pgfqpoint{4.564129in}{1.047572in}}%
\pgfpathlineto{\pgfqpoint{4.565069in}{1.044109in}}%
\pgfpathlineto{\pgfqpoint{4.566008in}{1.050955in}}%
\pgfpathlineto{\pgfqpoint{4.566938in}{1.048619in}}%
\pgfpathlineto{\pgfqpoint{4.568834in}{1.051686in}}%
\pgfpathlineto{\pgfqpoint{4.569792in}{1.052969in}}%
\pgfpathlineto{\pgfqpoint{4.570740in}{1.023156in}}%
\pgfpathlineto{\pgfqpoint{4.572618in}{1.023218in}}%
\pgfpathlineto{\pgfqpoint{4.573558in}{1.021382in}}%
\pgfpathlineto{\pgfqpoint{4.574497in}{1.024012in}}%
\pgfpathlineto{\pgfqpoint{4.576375in}{1.034990in}}%
\pgfpathlineto{\pgfqpoint{4.579211in}{1.034923in}}%
\pgfpathlineto{\pgfqpoint{4.582047in}{1.045927in}}%
\pgfpathlineto{\pgfqpoint{4.582986in}{1.043935in}}%
\pgfpathlineto{\pgfqpoint{4.583925in}{1.046034in}}%
\pgfpathlineto{\pgfqpoint{4.585803in}{1.022666in}}%
\pgfpathlineto{\pgfqpoint{4.588621in}{1.041903in}}%
\pgfpathlineto{\pgfqpoint{4.589569in}{1.043498in}}%
\pgfpathlineto{\pgfqpoint{4.592377in}{1.058697in}}%
\pgfpathlineto{\pgfqpoint{4.594256in}{1.063978in}}%
\pgfpathlineto{\pgfqpoint{4.595195in}{1.023125in}}%
\pgfpathlineto{\pgfqpoint{4.597073in}{1.027404in}}%
\pgfpathlineto{\pgfqpoint{4.598951in}{1.034098in}}%
\pgfpathlineto{\pgfqpoint{4.603665in}{1.048905in}}%
\pgfpathlineto{\pgfqpoint{4.604623in}{1.028906in}}%
\pgfpathlineto{\pgfqpoint{4.606492in}{1.032253in}}%
\pgfpathlineto{\pgfqpoint{4.607431in}{1.032498in}}%
\pgfpathlineto{\pgfqpoint{4.610258in}{1.041314in}}%
\pgfpathlineto{\pgfqpoint{4.611206in}{1.041519in}}%
\pgfpathlineto{\pgfqpoint{4.612145in}{1.044229in}}%
\pgfpathlineto{\pgfqpoint{4.614033in}{1.037753in}}%
\pgfpathlineto{\pgfqpoint{4.616832in}{1.041684in}}%
\pgfpathlineto{\pgfqpoint{4.620634in}{1.048713in}}%
\pgfpathlineto{\pgfqpoint{4.623461in}{1.048120in}}%
\pgfpathlineto{\pgfqpoint{4.624400in}{1.049226in}}%
\pgfpathlineto{\pgfqpoint{4.625339in}{1.051793in}}%
\pgfpathlineto{\pgfqpoint{4.628166in}{1.070009in}}%
\pgfpathlineto{\pgfqpoint{4.629124in}{1.042451in}}%
\pgfpathlineto{\pgfqpoint{4.631941in}{1.042464in}}%
\pgfpathlineto{\pgfqpoint{4.632889in}{1.049680in}}%
\pgfpathlineto{\pgfqpoint{4.633838in}{1.046016in}}%
\pgfpathlineto{\pgfqpoint{4.634804in}{1.048272in}}%
\pgfpathlineto{\pgfqpoint{4.635771in}{1.030225in}}%
\pgfpathlineto{\pgfqpoint{4.638579in}{1.035516in}}%
\pgfpathlineto{\pgfqpoint{4.641388in}{1.036616in}}%
\pgfpathlineto{\pgfqpoint{4.642327in}{1.032360in}}%
\pgfpathlineto{\pgfqpoint{4.643257in}{1.034927in}}%
\pgfpathlineto{\pgfqpoint{4.644196in}{1.034317in}}%
\pgfpathlineto{\pgfqpoint{4.646074in}{1.039812in}}%
\pgfpathlineto{\pgfqpoint{4.647004in}{1.045802in}}%
\pgfpathlineto{\pgfqpoint{4.647943in}{1.045165in}}%
\pgfpathlineto{\pgfqpoint{4.649831in}{1.042651in}}%
\pgfpathlineto{\pgfqpoint{4.651718in}{1.043703in}}%
\pgfpathlineto{\pgfqpoint{4.653606in}{1.048049in}}%
\pgfpathlineto{\pgfqpoint{4.656432in}{1.057944in}}%
\pgfpathlineto{\pgfqpoint{4.657381in}{1.060939in}}%
\pgfpathlineto{\pgfqpoint{4.659277in}{1.035872in}}%
\pgfpathlineto{\pgfqpoint{4.662104in}{1.044448in}}%
\pgfpathlineto{\pgfqpoint{4.665870in}{1.057159in}}%
\pgfpathlineto{\pgfqpoint{4.666818in}{1.048530in}}%
\pgfpathlineto{\pgfqpoint{4.668687in}{1.053745in}}%
\pgfpathlineto{\pgfqpoint{4.669626in}{1.055341in}}%
\pgfpathlineto{\pgfqpoint{4.670575in}{1.058906in}}%
\pgfpathlineto{\pgfqpoint{4.672462in}{1.058897in}}%
\pgfpathlineto{\pgfqpoint{4.674377in}{1.060720in}}%
\pgfpathlineto{\pgfqpoint{4.675316in}{1.058438in}}%
\pgfpathlineto{\pgfqpoint{4.676265in}{1.060863in}}%
\pgfpathlineto{\pgfqpoint{4.678143in}{1.058496in}}%
\pgfpathlineto{\pgfqpoint{4.680021in}{1.060297in}}%
\pgfpathlineto{\pgfqpoint{4.680979in}{1.062722in}}%
\pgfpathlineto{\pgfqpoint{4.681927in}{1.036625in}}%
\pgfpathlineto{\pgfqpoint{4.682866in}{1.041189in}}%
\pgfpathlineto{\pgfqpoint{4.683806in}{1.038925in}}%
\pgfpathlineto{\pgfqpoint{4.684745in}{1.042206in}}%
\pgfpathlineto{\pgfqpoint{4.685684in}{1.049480in}}%
\pgfpathlineto{\pgfqpoint{4.686641in}{1.047416in}}%
\pgfpathlineto{\pgfqpoint{4.687571in}{1.048985in}}%
\pgfpathlineto{\pgfqpoint{4.689459in}{1.062766in}}%
\pgfpathlineto{\pgfqpoint{4.691383in}{1.050179in}}%
\pgfpathlineto{\pgfqpoint{4.692322in}{1.052319in}}%
\pgfpathlineto{\pgfqpoint{4.693271in}{1.050322in}}%
\pgfpathlineto{\pgfqpoint{4.694210in}{1.053848in}}%
\pgfpathlineto{\pgfqpoint{4.695158in}{1.053696in}}%
\pgfpathlineto{\pgfqpoint{4.697994in}{1.061023in}}%
\pgfpathlineto{\pgfqpoint{4.698933in}{1.061794in}}%
\pgfpathlineto{\pgfqpoint{4.699891in}{1.065494in}}%
\pgfpathlineto{\pgfqpoint{4.700857in}{1.036924in}}%
\pgfpathlineto{\pgfqpoint{4.703666in}{1.056415in}}%
\pgfpathlineto{\pgfqpoint{4.704605in}{1.056076in}}%
\pgfpathlineto{\pgfqpoint{4.706492in}{1.059383in}}%
\pgfpathlineto{\pgfqpoint{4.707450in}{1.061095in}}%
\pgfpathlineto{\pgfqpoint{4.708398in}{1.052029in}}%
\pgfpathlineto{\pgfqpoint{4.709337in}{1.056165in}}%
\pgfpathlineto{\pgfqpoint{4.710276in}{1.054685in}}%
\pgfpathlineto{\pgfqpoint{4.714061in}{1.064050in}}%
\pgfpathlineto{\pgfqpoint{4.715994in}{1.058572in}}%
\pgfpathlineto{\pgfqpoint{4.716942in}{1.060426in}}%
\pgfpathlineto{\pgfqpoint{4.719788in}{1.070027in}}%
\pgfpathlineto{\pgfqpoint{4.720727in}{1.070348in}}%
\pgfpathlineto{\pgfqpoint{4.721666in}{1.073307in}}%
\pgfpathlineto{\pgfqpoint{4.723563in}{1.072077in}}%
\pgfpathlineto{\pgfqpoint{4.724511in}{1.047100in}}%
\pgfpathlineto{\pgfqpoint{4.727328in}{1.047340in}}%
\pgfpathlineto{\pgfqpoint{4.728267in}{1.046493in}}%
\pgfpathlineto{\pgfqpoint{4.730146in}{1.055412in}}%
\pgfpathlineto{\pgfqpoint{4.734841in}{1.082480in}}%
\pgfpathlineto{\pgfqpoint{4.735790in}{1.055055in}}%
\pgfpathlineto{\pgfqpoint{4.737677in}{1.055274in}}%
\pgfpathlineto{\pgfqpoint{4.739565in}{1.061006in}}%
\pgfpathlineto{\pgfqpoint{4.741443in}{1.063546in}}%
\pgfpathlineto{\pgfqpoint{4.742382in}{1.064687in}}%
\pgfpathlineto{\pgfqpoint{4.743312in}{1.068061in}}%
\pgfpathlineto{\pgfqpoint{4.745218in}{1.049127in}}%
\pgfpathlineto{\pgfqpoint{4.746166in}{1.049796in}}%
\pgfpathlineto{\pgfqpoint{4.752750in}{1.067281in}}%
\pgfpathlineto{\pgfqpoint{4.753707in}{1.049386in}}%
\pgfpathlineto{\pgfqpoint{4.754665in}{1.049952in}}%
\pgfpathlineto{\pgfqpoint{4.756543in}{1.053344in}}%
\pgfpathlineto{\pgfqpoint{4.757482in}{1.055911in}}%
\pgfpathlineto{\pgfqpoint{4.759360in}{1.062231in}}%
\pgfpathlineto{\pgfqpoint{4.761248in}{1.068672in}}%
\pgfpathlineto{\pgfqpoint{4.762187in}{1.057832in}}%
\pgfpathlineto{\pgfqpoint{4.764065in}{1.065668in}}%
\pgfpathlineto{\pgfqpoint{4.765934in}{1.072068in}}%
\pgfpathlineto{\pgfqpoint{4.766883in}{1.070651in}}%
\pgfpathlineto{\pgfqpoint{4.769700in}{1.074141in}}%
\pgfpathlineto{\pgfqpoint{4.770658in}{1.047835in}}%
\pgfpathlineto{\pgfqpoint{4.772536in}{1.051931in}}%
\pgfpathlineto{\pgfqpoint{4.775344in}{1.070709in}}%
\pgfpathlineto{\pgfqpoint{4.777232in}{1.082239in}}%
\pgfpathlineto{\pgfqpoint{4.778180in}{1.055510in}}%
\pgfpathlineto{\pgfqpoint{4.779119in}{1.056116in}}%
\pgfpathlineto{\pgfqpoint{4.783833in}{1.080603in}}%
\pgfpathlineto{\pgfqpoint{4.784782in}{1.083884in}}%
\pgfpathlineto{\pgfqpoint{4.785739in}{1.080407in}}%
\pgfpathlineto{\pgfqpoint{4.787608in}{1.085381in}}%
\pgfpathlineto{\pgfqpoint{4.790417in}{1.094630in}}%
\pgfpathlineto{\pgfqpoint{4.791365in}{1.093921in}}%
\pgfpathlineto{\pgfqpoint{4.792304in}{1.094634in}}%
\pgfpathlineto{\pgfqpoint{4.793262in}{1.055907in}}%
\pgfpathlineto{\pgfqpoint{4.795158in}{1.057115in}}%
\pgfpathlineto{\pgfqpoint{4.796107in}{1.055572in}}%
\pgfpathlineto{\pgfqpoint{4.799872in}{1.075888in}}%
\pgfpathlineto{\pgfqpoint{4.804587in}{1.090146in}}%
\pgfpathlineto{\pgfqpoint{4.805526in}{1.090396in}}%
\pgfpathlineto{\pgfqpoint{4.806483in}{1.092963in}}%
\pgfpathlineto{\pgfqpoint{4.807450in}{1.062811in}}%
\pgfpathlineto{\pgfqpoint{4.808398in}{1.062869in}}%
\pgfpathlineto{\pgfqpoint{4.811234in}{1.069563in}}%
\pgfpathlineto{\pgfqpoint{4.813131in}{1.052444in}}%
\pgfpathlineto{\pgfqpoint{4.817827in}{1.079797in}}%
\pgfpathlineto{\pgfqpoint{4.819732in}{1.076784in}}%
\pgfpathlineto{\pgfqpoint{4.821611in}{1.086259in}}%
\pgfpathlineto{\pgfqpoint{4.822550in}{1.087819in}}%
\pgfpathlineto{\pgfqpoint{4.824437in}{1.096479in}}%
\pgfpathlineto{\pgfqpoint{4.826343in}{1.099042in}}%
\pgfpathlineto{\pgfqpoint{4.828240in}{1.088305in}}%
\pgfpathlineto{\pgfqpoint{4.829198in}{1.088448in}}%
\pgfpathlineto{\pgfqpoint{4.831076in}{1.092825in}}%
\pgfpathlineto{\pgfqpoint{4.832015in}{1.092954in}}%
\pgfpathlineto{\pgfqpoint{4.833902in}{1.103860in}}%
\pgfpathlineto{\pgfqpoint{4.835799in}{1.107319in}}%
\pgfpathlineto{\pgfqpoint{4.836766in}{1.064117in}}%
\pgfpathlineto{\pgfqpoint{4.840541in}{1.064687in}}%
\pgfpathlineto{\pgfqpoint{4.842419in}{1.067914in}}%
\pgfpathlineto{\pgfqpoint{4.843358in}{1.063386in}}%
\pgfpathlineto{\pgfqpoint{4.848063in}{1.080871in}}%
\pgfpathlineto{\pgfqpoint{4.849002in}{1.106677in}}%
\pgfpathlineto{\pgfqpoint{4.849942in}{1.102185in}}%
\pgfpathlineto{\pgfqpoint{4.851820in}{1.104542in}}%
\pgfpathlineto{\pgfqpoint{4.854646in}{1.110296in}}%
\pgfpathlineto{\pgfqpoint{4.855586in}{1.109659in}}%
\pgfpathlineto{\pgfqpoint{4.856534in}{1.113456in}}%
\pgfpathlineto{\pgfqpoint{4.858431in}{1.084280in}}%
\pgfpathlineto{\pgfqpoint{4.859370in}{1.086317in}}%
\pgfpathlineto{\pgfqpoint{4.863126in}{1.101534in}}%
\pgfpathlineto{\pgfqpoint{4.866901in}{1.115770in}}%
\pgfpathlineto{\pgfqpoint{4.868798in}{1.102341in}}%
\pgfpathlineto{\pgfqpoint{4.869737in}{1.102554in}}%
\pgfpathlineto{\pgfqpoint{4.873494in}{1.119884in}}%
\pgfpathlineto{\pgfqpoint{4.875427in}{1.120508in}}%
\pgfpathlineto{\pgfqpoint{4.876394in}{1.112534in}}%
\pgfpathlineto{\pgfqpoint{4.877342in}{1.112440in}}%
\pgfpathlineto{\pgfqpoint{4.879230in}{1.116113in}}%
\pgfpathlineto{\pgfqpoint{4.883014in}{1.119514in}}%
\pgfpathlineto{\pgfqpoint{4.884920in}{1.101405in}}%
\pgfpathlineto{\pgfqpoint{4.888667in}{1.101636in}}%
\pgfpathlineto{\pgfqpoint{4.890546in}{1.104199in}}%
\pgfpathlineto{\pgfqpoint{4.897120in}{1.114976in}}%
\pgfpathlineto{\pgfqpoint{4.906539in}{1.130220in}}%
\pgfpathlineto{\pgfqpoint{4.913131in}{1.143248in}}%
\pgfpathlineto{\pgfqpoint{4.914079in}{1.128606in}}%
\pgfpathlineto{\pgfqpoint{4.915976in}{1.130019in}}%
\pgfpathlineto{\pgfqpoint{4.923489in}{1.131187in}}%
\pgfpathlineto{\pgfqpoint{4.925368in}{1.131632in}}%
\pgfpathlineto{\pgfqpoint{4.928176in}{1.136335in}}%
\pgfpathlineto{\pgfqpoint{4.937567in}{1.153080in}}%
\pgfpathlineto{\pgfqpoint{4.938525in}{1.138104in}}%
\pgfpathlineto{\pgfqpoint{4.946949in}{1.139771in}}%
\pgfpathlineto{\pgfqpoint{4.950706in}{1.140110in}}%
\pgfpathlineto{\pgfqpoint{4.953542in}{1.143283in}}%
\pgfpathlineto{\pgfqpoint{4.966671in}{1.168123in}}%
\pgfpathlineto{\pgfqpoint{4.968540in}{1.174643in}}%
\pgfpathlineto{\pgfqpoint{4.970437in}{1.177398in}}%
\pgfpathlineto{\pgfqpoint{4.978880in}{1.182269in}}%
\pgfpathlineto{\pgfqpoint{4.981688in}{1.184288in}}%
\pgfpathlineto{\pgfqpoint{4.982637in}{1.185287in}}%
\pgfpathlineto{\pgfqpoint{4.983576in}{1.215096in}}%
\pgfpathlineto{\pgfqpoint{4.984515in}{1.215096in}}%
\pgfpathlineto{\pgfqpoint{4.985454in}{1.163862in}}%
\pgfpathlineto{\pgfqpoint{4.994855in}{1.166157in}}%
\pgfpathlineto{\pgfqpoint{4.998602in}{1.170084in}}%
\pgfpathlineto{\pgfqpoint{5.003289in}{1.178722in}}%
\pgfpathlineto{\pgfqpoint{5.019245in}{1.209248in}}%
\pgfpathlineto{\pgfqpoint{5.026749in}{1.225146in}}%
\pgfpathlineto{\pgfqpoint{5.038065in}{1.245350in}}%
\pgfpathlineto{\pgfqpoint{5.039022in}{1.246019in}}%
\pgfpathlineto{\pgfqpoint{5.039980in}{1.159525in}}%
\pgfpathlineto{\pgfqpoint{5.042806in}{1.160296in}}%
\pgfpathlineto{\pgfqpoint{5.045615in}{1.160430in}}%
\pgfpathlineto{\pgfqpoint{5.050301in}{1.161138in}}%
\pgfpathlineto{\pgfqpoint{5.052198in}{1.161468in}}%
\pgfpathlineto{\pgfqpoint{5.058772in}{1.168809in}}%
\pgfpathlineto{\pgfqpoint{5.063467in}{1.173591in}}%
\pgfpathlineto{\pgfqpoint{5.065346in}{1.175446in}}%
\pgfpathlineto{\pgfqpoint{5.067224in}{1.178543in}}%
\pgfpathlineto{\pgfqpoint{5.074719in}{1.192396in}}%
\pgfpathlineto{\pgfqpoint{5.079396in}{1.201814in}}%
\pgfpathlineto{\pgfqpoint{5.089708in}{1.221928in}}%
\pgfpathlineto{\pgfqpoint{5.098142in}{1.236806in}}%
\pgfpathlineto{\pgfqpoint{5.101899in}{1.248305in}}%
\pgfpathlineto{\pgfqpoint{5.106576in}{1.259310in}}%
\pgfpathlineto{\pgfqpoint{5.111253in}{1.271647in}}%
\pgfpathlineto{\pgfqpoint{5.116906in}{1.281185in}}%
\pgfpathlineto{\pgfqpoint{5.117855in}{1.206404in}}%
\pgfpathlineto{\pgfqpoint{5.119724in}{1.206783in}}%
\pgfpathlineto{\pgfqpoint{5.124401in}{1.210991in}}%
\pgfpathlineto{\pgfqpoint{5.125340in}{1.211267in}}%
\pgfpathlineto{\pgfqpoint{5.128149in}{1.215822in}}%
\pgfpathlineto{\pgfqpoint{5.132817in}{1.219704in}}%
\pgfpathlineto{\pgfqpoint{5.138424in}{1.228048in}}%
\pgfpathlineto{\pgfqpoint{5.144031in}{1.242796in}}%
\pgfpathlineto{\pgfqpoint{5.178696in}{1.324878in}}%
\pgfpathlineto{\pgfqpoint{5.181505in}{1.331956in}}%
\pgfpathlineto{\pgfqpoint{5.192747in}{1.356871in}}%
\pgfpathlineto{\pgfqpoint{5.194616in}{1.360191in}}%
\pgfpathlineto{\pgfqpoint{5.198354in}{1.369266in}}%
\pgfpathlineto{\pgfqpoint{5.202083in}{1.373335in}}%
\pgfpathlineto{\pgfqpoint{5.211438in}{1.379878in}}%
\pgfpathlineto{\pgfqpoint{5.215185in}{1.383925in}}%
\pgfpathlineto{\pgfqpoint{5.219862in}{1.394854in}}%
\pgfpathlineto{\pgfqpoint{5.225479in}{1.402935in}}%
\pgfpathlineto{\pgfqpoint{5.226409in}{1.402935in}}%
\pgfpathlineto{\pgfqpoint{5.227357in}{1.326032in}}%
\pgfpathlineto{\pgfqpoint{5.232016in}{1.328118in}}%
\pgfpathlineto{\pgfqpoint{5.236675in}{1.331341in}}%
\pgfpathlineto{\pgfqpoint{5.239474in}{1.333226in}}%
\pgfpathlineto{\pgfqpoint{5.248837in}{1.347565in}}%
\pgfpathlineto{\pgfqpoint{5.260043in}{1.373104in}}%
\pgfpathlineto{\pgfqpoint{5.263781in}{1.380850in}}%
\pgfpathlineto{\pgfqpoint{5.270300in}{1.396575in}}%
\pgfpathlineto{\pgfqpoint{5.272169in}{1.413650in}}%
\pgfpathlineto{\pgfqpoint{5.277767in}{1.427364in}}%
\pgfpathlineto{\pgfqpoint{5.279636in}{1.431260in}}%
\pgfpathlineto{\pgfqpoint{5.281505in}{1.445741in}}%
\pgfpathlineto{\pgfqpoint{5.284322in}{1.502595in}}%
\pgfpathlineto{\pgfqpoint{5.285261in}{1.502100in}}%
\pgfpathlineto{\pgfqpoint{5.289018in}{1.511853in}}%
\pgfpathlineto{\pgfqpoint{5.289948in}{1.510974in}}%
\pgfpathlineto{\pgfqpoint{5.293732in}{1.541439in}}%
\pgfpathlineto{\pgfqpoint{5.294680in}{1.551859in}}%
\pgfpathlineto{\pgfqpoint{5.295629in}{1.551743in}}%
\pgfpathlineto{\pgfqpoint{5.300352in}{1.606926in}}%
\pgfpathlineto{\pgfqpoint{5.302230in}{1.610122in}}%
\pgfpathlineto{\pgfqpoint{5.303160in}{1.610376in}}%
\pgfpathlineto{\pgfqpoint{5.305048in}{1.608010in}}%
\pgfpathlineto{\pgfqpoint{5.305987in}{1.608295in}}%
\pgfpathlineto{\pgfqpoint{5.306935in}{1.613692in}}%
\pgfpathlineto{\pgfqpoint{5.308814in}{1.635929in}}%
\pgfpathlineto{\pgfqpoint{5.313528in}{1.699509in}}%
\pgfpathlineto{\pgfqpoint{5.314467in}{1.695061in}}%
\pgfpathlineto{\pgfqpoint{5.316336in}{1.695364in}}%
\pgfpathlineto{\pgfqpoint{5.318205in}{1.690876in}}%
\pgfpathlineto{\pgfqpoint{5.320074in}{1.715291in}}%
\pgfpathlineto{\pgfqpoint{5.322910in}{1.715220in}}%
\pgfpathlineto{\pgfqpoint{5.323849in}{1.473432in}}%
\pgfpathlineto{\pgfqpoint{5.325755in}{1.486946in}}%
\pgfpathlineto{\pgfqpoint{5.326694in}{1.489175in}}%
\pgfpathlineto{\pgfqpoint{5.328572in}{1.496596in}}%
\pgfpathlineto{\pgfqpoint{5.329512in}{1.496596in}}%
\pgfpathlineto{\pgfqpoint{5.333241in}{1.504333in}}%
\pgfpathlineto{\pgfqpoint{5.334180in}{1.500393in}}%
\pgfpathlineto{\pgfqpoint{5.335119in}{1.501008in}}%
\pgfpathlineto{\pgfqpoint{5.336049in}{1.506188in}}%
\pgfpathlineto{\pgfqpoint{5.339833in}{1.559913in}}%
\pgfpathlineto{\pgfqpoint{5.344510in}{1.603298in}}%
\pgfpathlineto{\pgfqpoint{5.350117in}{1.625735in}}%
\pgfpathlineto{\pgfqpoint{5.351047in}{1.624764in}}%
\pgfpathlineto{\pgfqpoint{5.351987in}{1.621728in}}%
\pgfpathlineto{\pgfqpoint{5.353856in}{1.626939in}}%
\pgfpathlineto{\pgfqpoint{5.360457in}{1.725204in}}%
\pgfpathlineto{\pgfqpoint{5.366074in}{1.726559in}}%
\pgfpathlineto{\pgfqpoint{5.367013in}{1.715037in}}%
\pgfpathlineto{\pgfqpoint{5.368900in}{1.715002in}}%
\pgfpathlineto{\pgfqpoint{5.369839in}{1.716629in}}%
\pgfpathlineto{\pgfqpoint{5.371718in}{1.751817in}}%
\pgfpathlineto{\pgfqpoint{5.372648in}{1.758427in}}%
\pgfpathlineto{\pgfqpoint{5.374535in}{1.789346in}}%
\pgfpathlineto{\pgfqpoint{5.376413in}{1.799254in}}%
\pgfpathlineto{\pgfqpoint{5.379222in}{1.799624in}}%
\pgfpathlineto{\pgfqpoint{5.381091in}{1.796963in}}%
\pgfpathlineto{\pgfqpoint{5.382030in}{1.805012in}}%
\pgfpathlineto{\pgfqpoint{5.383927in}{1.826990in}}%
\pgfpathlineto{\pgfqpoint{5.386772in}{1.877346in}}%
\pgfpathlineto{\pgfqpoint{5.388641in}{1.877774in}}%
\pgfpathlineto{\pgfqpoint{5.391458in}{1.471070in}}%
\pgfpathlineto{\pgfqpoint{5.392406in}{1.470879in}}%
\pgfpathlineto{\pgfqpoint{5.394276in}{1.466399in}}%
\pgfpathlineto{\pgfqpoint{5.398078in}{1.466417in}}%
\pgfpathlineto{\pgfqpoint{5.399975in}{1.477622in}}%
\pgfpathlineto{\pgfqpoint{5.400932in}{1.473691in}}%
\pgfpathlineto{\pgfqpoint{5.402838in}{1.492121in}}%
\pgfpathlineto{\pgfqpoint{5.404717in}{1.520455in}}%
\pgfpathlineto{\pgfqpoint{5.405656in}{1.516234in}}%
\pgfpathlineto{\pgfqpoint{5.410351in}{1.531611in}}%
\pgfpathlineto{\pgfqpoint{5.411291in}{1.530635in}}%
\pgfpathlineto{\pgfqpoint{5.414108in}{1.552880in}}%
\pgfpathlineto{\pgfqpoint{5.417929in}{1.604194in}}%
\pgfpathlineto{\pgfqpoint{5.419807in}{1.624483in}}%
\pgfpathlineto{\pgfqpoint{5.425433in}{1.643541in}}%
\pgfpathlineto{\pgfqpoint{5.431132in}{1.721625in}}%
\pgfpathlineto{\pgfqpoint{5.432071in}{1.725476in}}%
\pgfpathlineto{\pgfqpoint{5.433977in}{1.737742in}}%
\pgfpathlineto{\pgfqpoint{5.434926in}{1.533077in}}%
\pgfpathlineto{\pgfqpoint{5.435874in}{1.533345in}}%
\pgfpathlineto{\pgfqpoint{5.436804in}{1.526703in}}%
\pgfpathlineto{\pgfqpoint{5.437752in}{1.509071in}}%
\pgfpathlineto{\pgfqpoint{5.438682in}{1.519817in}}%
\pgfpathlineto{\pgfqpoint{5.442494in}{1.520312in}}%
\pgfpathlineto{\pgfqpoint{5.443442in}{1.526262in}}%
\pgfpathlineto{\pgfqpoint{5.445330in}{1.526169in}}%
\pgfpathlineto{\pgfqpoint{5.447199in}{1.514683in}}%
\pgfpathlineto{\pgfqpoint{5.453856in}{1.606998in}}%
\pgfpathlineto{\pgfqpoint{5.455725in}{1.608045in}}%
\pgfpathlineto{\pgfqpoint{5.456655in}{1.597027in}}%
\pgfpathlineto{\pgfqpoint{5.459481in}{1.645168in}}%
\pgfpathlineto{\pgfqpoint{5.464196in}{1.698426in}}%
\pgfpathlineto{\pgfqpoint{5.465135in}{1.703195in}}%
\pgfpathlineto{\pgfqpoint{5.466074in}{1.692948in}}%
\pgfpathlineto{\pgfqpoint{5.467013in}{1.699001in}}%
\pgfpathlineto{\pgfqpoint{5.473605in}{1.794913in}}%
\pgfpathlineto{\pgfqpoint{5.481100in}{1.825970in}}%
\pgfpathlineto{\pgfqpoint{5.483936in}{1.867826in}}%
\pgfpathlineto{\pgfqpoint{5.485833in}{1.888436in}}%
\pgfpathlineto{\pgfqpoint{5.487739in}{1.900407in}}%
\pgfpathlineto{\pgfqpoint{5.488687in}{1.602603in}}%
\pgfpathlineto{\pgfqpoint{5.489635in}{1.602603in}}%
\pgfpathlineto{\pgfqpoint{5.492453in}{1.530033in}}%
\pgfpathlineto{\pgfqpoint{5.493383in}{1.529930in}}%
\pgfpathlineto{\pgfqpoint{5.495270in}{1.525277in}}%
\pgfpathlineto{\pgfqpoint{5.496209in}{1.525340in}}%
\pgfpathlineto{\pgfqpoint{5.497148in}{1.532877in}}%
\pgfpathlineto{\pgfqpoint{5.498088in}{1.532315in}}%
\pgfpathlineto{\pgfqpoint{5.499027in}{1.541345in}}%
\pgfpathlineto{\pgfqpoint{5.501853in}{1.534521in}}%
\pgfpathlineto{\pgfqpoint{5.504652in}{1.536625in}}%
\pgfpathlineto{\pgfqpoint{5.508418in}{1.550032in}}%
\pgfpathlineto{\pgfqpoint{5.509357in}{1.550179in}}%
\pgfpathlineto{\pgfqpoint{5.510287in}{1.551293in}}%
\pgfpathlineto{\pgfqpoint{5.514044in}{1.602960in}}%
\pgfpathlineto{\pgfqpoint{5.515931in}{1.632599in}}%
\pgfpathlineto{\pgfqpoint{5.520645in}{1.648752in}}%
\pgfpathlineto{\pgfqpoint{5.522542in}{1.654564in}}%
\pgfpathlineto{\pgfqpoint{5.527219in}{1.671224in}}%
\pgfpathlineto{\pgfqpoint{5.528158in}{1.668902in}}%
\pgfpathlineto{\pgfqpoint{5.529098in}{1.669406in}}%
\pgfpathlineto{\pgfqpoint{5.531933in}{1.711757in}}%
\pgfpathlineto{\pgfqpoint{5.534751in}{1.747431in}}%
\pgfpathlineto{\pgfqpoint{5.535690in}{1.754991in}}%
\pgfpathlineto{\pgfqpoint{5.537587in}{1.755133in}}%
\pgfpathlineto{\pgfqpoint{5.540423in}{1.550937in}}%
\pgfpathlineto{\pgfqpoint{5.542301in}{1.550865in}}%
\pgfpathlineto{\pgfqpoint{5.543240in}{1.546292in}}%
\pgfpathlineto{\pgfqpoint{5.544170in}{1.526926in}}%
\pgfpathlineto{\pgfqpoint{5.545109in}{1.533759in}}%
\pgfpathlineto{\pgfqpoint{5.551692in}{1.533790in}}%
\pgfpathlineto{\pgfqpoint{5.558239in}{1.534062in}}%
\pgfpathlineto{\pgfqpoint{5.559169in}{1.526619in}}%
\pgfpathlineto{\pgfqpoint{5.561047in}{1.548035in}}%
\pgfpathlineto{\pgfqpoint{5.563864in}{1.592526in}}%
\pgfpathlineto{\pgfqpoint{5.565743in}{1.606503in}}%
\pgfpathlineto{\pgfqpoint{5.568551in}{1.616344in}}%
\pgfpathlineto{\pgfqpoint{5.572307in}{1.630179in}}%
\pgfpathlineto{\pgfqpoint{5.573247in}{1.630415in}}%
\pgfpathlineto{\pgfqpoint{5.574186in}{1.624902in}}%
\pgfpathlineto{\pgfqpoint{5.576082in}{1.635577in}}%
\pgfpathlineto{\pgfqpoint{5.582693in}{1.732157in}}%
\pgfpathlineto{\pgfqpoint{5.586441in}{1.746286in}}%
\pgfpathlineto{\pgfqpoint{5.587380in}{1.743010in}}%
\pgfpathlineto{\pgfqpoint{5.593042in}{1.819493in}}%
\pgfpathlineto{\pgfqpoint{5.597719in}{1.830890in}}%
\pgfpathlineto{\pgfqpoint{5.601458in}{1.831015in}}%
\pgfpathlineto{\pgfqpoint{5.606135in}{1.836105in}}%
\pgfpathlineto{\pgfqpoint{5.608980in}{1.875889in}}%
\pgfpathlineto{\pgfqpoint{5.611889in}{1.913065in}}%
\pgfpathlineto{\pgfqpoint{5.614734in}{1.581227in}}%
\pgfpathlineto{\pgfqpoint{5.622220in}{1.581485in}}%
\pgfpathlineto{\pgfqpoint{5.623159in}{1.585867in}}%
\pgfpathlineto{\pgfqpoint{5.624107in}{1.579163in}}%
\pgfpathlineto{\pgfqpoint{5.626934in}{1.585764in}}%
\pgfpathlineto{\pgfqpoint{5.628803in}{1.578802in}}%
\pgfpathlineto{\pgfqpoint{5.632615in}{1.588064in}}%
\pgfpathlineto{\pgfqpoint{5.637311in}{1.612377in}}%
\pgfpathlineto{\pgfqpoint{5.638250in}{1.607751in}}%
\pgfpathlineto{\pgfqpoint{5.639198in}{1.612458in}}%
\pgfpathlineto{\pgfqpoint{5.642052in}{1.648480in}}%
\pgfpathlineto{\pgfqpoint{5.643940in}{1.653414in}}%
\pgfpathlineto{\pgfqpoint{5.651453in}{1.685206in}}%
\pgfpathlineto{\pgfqpoint{5.653322in}{1.694178in}}%
\pgfpathlineto{\pgfqpoint{5.654261in}{1.690278in}}%
\pgfpathlineto{\pgfqpoint{5.658975in}{1.773211in}}%
\pgfpathlineto{\pgfqpoint{5.659915in}{1.768946in}}%
\pgfpathlineto{\pgfqpoint{5.661793in}{1.781925in}}%
\pgfpathlineto{\pgfqpoint{5.663699in}{1.787059in}}%
\pgfpathlineto{\pgfqpoint{5.666516in}{1.787483in}}%
\pgfpathlineto{\pgfqpoint{5.667446in}{1.789052in}}%
\pgfpathlineto{\pgfqpoint{5.672151in}{1.857004in}}%
\pgfpathlineto{\pgfqpoint{5.673099in}{1.869943in}}%
\pgfpathlineto{\pgfqpoint{5.674048in}{1.855961in}}%
\pgfpathlineto{\pgfqpoint{5.675935in}{1.611388in}}%
\pgfpathlineto{\pgfqpoint{5.685391in}{1.611419in}}%
\pgfpathlineto{\pgfqpoint{5.688190in}{1.583701in}}%
\pgfpathlineto{\pgfqpoint{5.693862in}{1.583870in}}%
\pgfpathlineto{\pgfqpoint{5.694801in}{1.583714in}}%
\pgfpathlineto{\pgfqpoint{5.695749in}{1.582141in}}%
\pgfpathlineto{\pgfqpoint{5.702360in}{1.606383in}}%
\pgfpathlineto{\pgfqpoint{5.704248in}{1.613037in}}%
\pgfpathlineto{\pgfqpoint{5.705196in}{1.608549in}}%
\pgfpathlineto{\pgfqpoint{5.707083in}{1.632483in}}%
\pgfpathlineto{\pgfqpoint{5.708962in}{1.662725in}}%
\pgfpathlineto{\pgfqpoint{5.709901in}{1.662061in}}%
\pgfpathlineto{\pgfqpoint{5.716466in}{1.687011in}}%
\pgfpathlineto{\pgfqpoint{5.723979in}{1.779464in}}%
\pgfpathlineto{\pgfqpoint{5.725857in}{1.785713in}}%
\pgfpathlineto{\pgfqpoint{5.730544in}{1.786132in}}%
\pgfpathlineto{\pgfqpoint{5.731483in}{1.781662in}}%
\pgfpathlineto{\pgfqpoint{5.732431in}{1.788642in}}%
\pgfpathlineto{\pgfqpoint{5.735258in}{1.848549in}}%
\pgfpathlineto{\pgfqpoint{5.737136in}{1.873834in}}%
\pgfpathlineto{\pgfqpoint{5.738075in}{1.867438in}}%
\pgfpathlineto{\pgfqpoint{5.741887in}{1.870433in}}%
\pgfpathlineto{\pgfqpoint{5.744741in}{1.570236in}}%
\pgfpathlineto{\pgfqpoint{5.747549in}{1.570378in}}%
\pgfpathlineto{\pgfqpoint{5.748489in}{1.571377in}}%
\pgfpathlineto{\pgfqpoint{5.751315in}{1.582020in}}%
\pgfpathlineto{\pgfqpoint{5.752264in}{1.577465in}}%
\pgfpathlineto{\pgfqpoint{5.754160in}{1.581927in}}%
\pgfpathlineto{\pgfqpoint{5.755099in}{1.575067in}}%
\pgfpathlineto{\pgfqpoint{5.762612in}{1.575058in}}%
\pgfpathlineto{\pgfqpoint{5.765448in}{1.581695in}}%
\pgfpathlineto{\pgfqpoint{5.766387in}{1.579119in}}%
\pgfpathlineto{\pgfqpoint{5.768284in}{1.600326in}}%
\pgfpathlineto{\pgfqpoint{5.769233in}{1.595708in}}%
\pgfpathlineto{\pgfqpoint{5.773928in}{1.609226in}}%
\pgfpathlineto{\pgfqpoint{5.778642in}{1.696001in}}%
\pgfpathlineto{\pgfqpoint{5.779581in}{1.693340in}}%
\pgfpathlineto{\pgfqpoint{5.790851in}{1.733507in}}%
\pgfpathlineto{\pgfqpoint{5.792739in}{1.770492in}}%
\pgfpathlineto{\pgfqpoint{5.794691in}{1.763994in}}%
\pgfpathlineto{\pgfqpoint{5.795667in}{1.763802in}}%
\pgfpathlineto{\pgfqpoint{5.796615in}{1.764903in}}%
\pgfpathlineto{\pgfqpoint{5.800491in}{1.763348in}}%
\pgfpathlineto{\pgfqpoint{5.802434in}{1.764395in}}%
\pgfpathlineto{\pgfqpoint{5.805288in}{1.764199in}}%
\pgfpathlineto{\pgfqpoint{5.807222in}{1.764422in}}%
\pgfpathlineto{\pgfqpoint{5.810150in}{1.763753in}}%
\pgfpathlineto{\pgfqpoint{5.812092in}{1.763981in}}%
\pgfpathlineto{\pgfqpoint{5.813032in}{1.764172in}}%
\pgfpathlineto{\pgfqpoint{5.814928in}{1.763521in}}%
\pgfpathlineto{\pgfqpoint{5.815913in}{1.764689in}}%
\pgfpathlineto{\pgfqpoint{5.817819in}{1.764123in}}%
\pgfpathlineto{\pgfqpoint{5.820683in}{1.763936in}}%
\pgfpathlineto{\pgfqpoint{5.821650in}{1.766004in}}%
\pgfpathlineto{\pgfqpoint{5.822589in}{1.774615in}}%
\pgfpathlineto{\pgfqpoint{5.824458in}{1.806537in}}%
\pgfpathlineto{\pgfqpoint{5.824458in}{1.806537in}}%
\pgfusepath{stroke}%
\end{pgfscope}%
\end{pgfpicture}%
\makeatother%
\endgroup%

			\end{figcenter}
			\caption[Memory usage of the 4km\textsuperscript{2} \enquote{OSM city} dataset]{Memory usage of the 4km\textsuperscript{2} \enquote{OSM city} dataset. The marks represent the following operations: \textbf{(1)} Start of unwrapping and triangulating obstacles and also determining obstacle neighbors, \textbf{(2)} start of kNN search, \textbf{(3)} create visibility graph, \textbf{(4)} prepare road merge operation, \textbf{(5)} start of merging road edges into graph, \textbf{(6)} end of graph generation and start of agent routing, \textbf{(7)} end agent routing and end of simulation.}
			\label{fig:eval-memory-usage}
		\end{figure}
	
		In this section, the memory usage of the hybrid routing algorithm is analyzed using the 4km\textsuperscript{2} \enquote{OSM city} dataset as presented in \Cref{fig:eval-memory-usage}.
		
		Until the creation of the visibility graph (3) the memory requirement only shows a very slight linear increase due to simple mapping without new and larger data structures.
		With the start of the merge operation (4), the memory usage rises due to additional data structures and newly added vertices and edges to the graph.
		The period between (4 to 5) determined dead-end road vertices, which are connected with new nodes separately to the graph.
		Merging road edges (5 to 6) then introduces many new nodes and edges due to the large number of intersections.
		Routing requests (6 to 7) do also require a certain amount of memory and the garbage collection regularly frees up the memory of prior requests resulting in recurring spikes in the memory usage.
		
		Possible memory optimizations and other performance improvements are discussed in \Cref{sec:future-work}.
		
	\subsection{Optimizations}
	
		Impacts of the implemented optimizations were measured using the 0.5km\textsuperscript{2} \enquote{OSM city} dataset as listed in \Cref{table:optimization-impact}.
		Noteworthy is the discrepancy between the product of all speedup/slowdown factors and the overall speedup of about 71 when all optimizations are active.
		Activating all optimizations should, according to the separate speedup factors on the right side of \Cref{table:optimization-impact}, result in a speedup of over 255 but only 71 can be observed.
		This is because the filtering methods interfere with each other.
		When a vertex if filtered out by one filtering method, it reduces the efficiency of the other filtering methods.
		Therefore, the selective (de)activation of single optimizations yields a differently strong impact than the (de)activation of all optimizations together.
		
		\begin{table}[h]
			\begin{figcenter}
				\begin{tabularx}{\textwidth}{p{4.25cm}RRRR}
\toprule
							& \multicolumn{2}{c}{\textbf{All activated except}}	& \multicolumn{2}{c}{\textbf{All deactivated except}}	\\
							  \cmidrule(lr){2-3}									  \cmidrule(lr){4-5}
\textbf{Optimization}		& \textbf{Time}	& \textbf{Slowdown}						& \textbf{Time}	& \textbf{Speedup}						\\
\midrule
Shadow areas				& 121.6 s 		& 10.97									&  22.2 s		& 35.56									\\
kNN filtering				&  11.3 s		&  1.02									& 751.7 s		&  1.05									\\
Vertices on convex hull		&  18.6 s		&  1.68									& 287.9 s		&  2.74									\\
Valid angle areas			&  12.8 s		&  1.16									& 322.5 s		&  2.44									\\
Custom collision detection	&  11.6 s		&  1.05									& 769.3 s		&  1.02									\\
\midrule
No active optimization		& 788.5 s 		& 71.13									& 788.5 s		&  1.00									\\
All optimizations active	&  11.1 s		&  1.00									&  11.1 s		& 71.13									\\
\bottomrule
				\end{tabularx}
			\end{figcenter}
			\caption[Comparison of optimizations regarding performance.]{Optimization impact on the 0.5km\textsuperscript{2} \enquote{OSM city} dataset import. The optimization of a row was selectively deactivated (left) or activated (right).}
			\label{table:optimization-impact}
		\end{table}
		
		Some filtering optimizations presented in \Cref{chap:implementation}, namely the convex hull and valid angle area filtering, remove vertices that might be useful when determining non-shortest but otherwise optimal paths based on a custom weighting.
		Deactivating these two optimizations results in a measured slowdown factor of about 1.9.
		
\section{Route correctness and quality}

	\subsection{Weight function}
	
		Before the correctness and quality of the routes are discussed, a brief introduction into the functioning of the weight function is given.
		
		When determining true shortest paths, the length of an edge is its weight, which should be minimized in order to obtain the shortest path.
		Whenever certain edges should be preferred or avoided, a factor is determined based on the edges attributes.
		Due to the minimization algorithm a factor lower one means an edge is preferred, a factor greater one rather avoids those edges.
		A factor of positive infinity on an edge $e$ completely avoids this edge since any other path has a lower total weight than a path containing $e$.
		
		The implementation of the hybrid visibility graph provides three routing methods.
		One determined the true shortest path (all weight factors are one), one determines a weighted path (factor of 0.8 for road edges) and one accepts a custom weight function.
		
		All following analyses use the predefined weight function with a factor of 0.8 on road edges.
		The factor 0.8 is chosen rather arbitrarily but yields routes where the agent does indeed switch between road and visibility edges.
		Impacts of different weight factors on the resulting paths are discussed in the following \hyperref[subsubsec:manual-route-analysis]{manual route analysis}.

	\subsection{Correctness}
	\label{subsec:correctness}
	
		In this section, theoretic considerations on the correctness of the hybrid routing algorithm are presented, which therefore does not include mistakes in the implementation.
		The hybrid routing algorithm is considered \emph{correct} if the resulting path between any two given locations is the shortest possible path with respect to the obstacles of the dataset and the given weight function.
		In the following the weight function represents the distance of edges to find truly shortest paths.
		
		Edges in the generated visibility graph represent shortest segments between vertices of obstacles.
		In other words, for any two vertices $v$ and $v'$, a visibility edge $e=(v, v')$ is the shortest possible connection between these two vertices.
		A shortest path through a visibility graph is therefore equal to a shortest geometric path around obstacles in the plain.
		
		For shortest path algorithms, such as A* or Dijkstra, the term \emph{correct} refers to the fact that their result is the shortest possible path between two vertices in a graph.
		Using a correct graph-based shortest path algorithm in a visibility graph yields the shortest path in the plain.
		Regarding the shortest path between existing vertices, the hybrid visibility routing is therefore considered correct.
		
		In addition to routing on a visibility graph, the presented hybrid routing algorithm actively connects locations to the graph as part of answering the routing query yielding a temporarily augmented graph.
		This means the resulting path $p$ from a newly added vertex $s$ to a newly added vertex $t$ contains two additional new edges $e_s=(s, v_s)$ and $e_t=(v_t, t)$, which are not part of the normal non-augmented hybrid visibility graph, resulting in $p=\left\langle s, v_s, ..., v_t, t \right\rangle$.
		The correctness of the path from $v_s$ to $v_t$ follows from the argumentation above.
		The edges $e_s$ and $e_t$ are visibility edges and the correctness argument applies to them as well, meaning that the overall path $p$ is shortest on the augmented visibility graph and therefore shortest in the plain with the presence of obstacles.
		Therefore, the overall hybrid routing algorithm is considered to be correct.
		
		One noteworthy aspect is the filtering of vertices, as described in \Cref{subsec:step-2-knn-search}, which might result in missing edges.
		However, these filters are solely performance optimizations and can be deactivated without negative effects on the algorithms functioning, except the performance.
		
	\subsection{Quality of routes}
	
		\subsubsection{Definition and measurement of route quality}
	
			An ideal route, meaning a route of maximum quality, would exactly match a path chosen by a real pedestrian.
			To approximate a real pedestrian, the expected paths used in the following sections were obtained using aerial imagery, data contained in OSM and local knowledge.
			Therefore, an expected path is realistically walkable in the real world and would likely be chosen by actual pedestrians.
			This also means that an expected path contains a certain subjective component, for example the exact location of a road crossing, the preference of shortcuts via unpaved areas or the avoidance of steps.
			Therefore, the quality analysis of routes uses mathematical metrics as well as a manual analysis.
		
		% How realistic are the routes (= can I go there in real life)? If not: Why not?
		\subsubsection{Manual route analysis}
		\label{subsubsec:manual-route-analysis}
		
			\begin{wrapfigure}[13]{r}{0.5\textwidth}
				\vspace{-2.5\baselineskip}
				\begin{figcenter}
					\includegraphics[width=\linewidth]{images/qgis-routing-osterstrasse}
				\end{figcenter}
				\caption[Comparison of normal routing with hybrid visibility routing.]{Graph-based route (red), hybrid routing algorithm result (blue) and the expected route (green). The dashed part of the expected result is an equally good alternative route.}
				\label{fig:eval-osterstrasse}
			\end{wrapfigure}
		
			In the following examples, routes determined by graph-based routing and the hybrid routing algorithm are compared against an expected result.
			This expected route was created based on OpenStreetMap data, aerial imagery and also local knowledge.
			It was \emph{not} created to intentionally match the results from the hybrid routing algorithm.
		
			% From motivation
			The first situation, shown in \Cref{fig:eval-osterstrasse}, illustrates the problem of missing edges for graph-based algorithms.
			The square \enquote{Fanny-Mendelssohn Platz} in Hamburg, Germany, is a mostly unconnected area and therefore not traversable by graph-based algorithms.
			A routing request using Graphhopper\footnote{\url{https://www.osm.org/directions?engine=graphhopper\_foot\&route=53.57657,9.95210;53.57601,9.95268}} yields the red route, which does not cross the square due to missing edges and therefore takes a detour.
			The hybrid routing algorithm (blue path) does not completely match the expected route but is a better approximation than the graph-based result.
			Next to this example, analyzing routes determined within the 1km\textsuperscript{2} \enquote{OSM city} dataset yields some noteworthy findings.
			
			\begin{itemize}
				\item The route quality significantly decreases with missing or wrong data as shown in \Cref{fig:eval-osterstrasse} as well as \Cref{fig:eval-city-usefulness-b} to \ref{fig:eval-city-usefulness-c}.
				Red passages are not usable in the real world, often traversing private areas, which usually contain obstacles such as fences or vegetation.
				As described in \Cref{subsubsec:data-not-in-osm}, data of private areas in OSM is often sparse resulting in these unrealistic routes.
				\item Within densely built-up areas, larger open spaces are rare and the combination of roads and buildings form corridors in which routes tend to lie.
				When and how often a route determined by the hybrid routing algorithm follows a road depends on the weighting function.
				\item Larger roads result in wide obstacle-free corridor areas, in which unrealistic road crossings might appear.
				Such situation can be seen in \Cref{fig:eval-city-road-crossing} with crossings over a six-lane road, which would be dangerous and unrealistic for real-world pedestrians.
				\item The weight function has a huge impact on the route quality.
				The weighting factors used in this work specify how strongly roads should be preferred, which is illustrated in \Cref{fig:eval-city-weights} with the same route using different weights for road edges.
				Adjusting and defining fine-grained weight functions may help to improve the route quality.
				\item In the \enquote{OSM city} datasets with only smaller open areas, graph-based routes and routes determined by the hybrid routing algorithm were often similar and some parts are even identical.
			\end{itemize}
			
			\begin{figure}[h!]
				\begin{minipage}[t]{.48\textwidth}
					\begin{subfigure}[t]{\linewidth}
						\begin{figcenter}
							\includegraphics[width=\textwidth]{images/qgis-routing-city-routing-1}
						\end{figcenter}
						\caption{Simple route with only one small passage of missing data.}
						\label{fig:eval-city-usefulness-1}
					\end{subfigure}
				\end{minipage}
				\hfill
				\begin{minipage}[t]{.48\textwidth}
					\begin{subfigure}[t]{\linewidth}
						\begin{figcenter}
							\includegraphics[width=\textwidth]{images/qgis-routing-city-routing-3}
						\end{figcenter}
						\caption{Missing obstacles (mainly walls and fences between buildings) lead to non-realistic routes.}
						\label{fig:eval-city-usefulness-b}
					\end{subfigure}
				\end{minipage}
				\\[3ex]
				\begin{minipage}[t]{.48\textwidth}
					\begin{subfigure}[t]{\linewidth}
						\begin{figcenter}
							\includegraphics[width=\textwidth]{images/qgis-routing-city-routing-6}
						\end{figcenter}
						\caption{Missing walls, fences and hedges on private property are a common type of missing obstacles.}
						\label{fig:eval-city-usefulness-c}
					\end{subfigure}
				\end{minipage}
				\hfill
				\begin{minipage}[t]{.48\textwidth}
					\begin{subfigure}[t]{\linewidth}
						\begin{figcenter}
							\includegraphics[width=\textwidth]{images/qgis-routing-city-routing-18}
						\end{figcenter}
						\caption{Cities with closed rows of buildings do not provide many degrees of freedom, which results in similar or even equal routes, regardles of the algorithm.}
						\label{fig:eval-city-usefulness-d}
					\end{subfigure}
				\end{minipage}
				\caption[Comparison of graph-based, actual and expected routes.]{Expected (green), graph-based (black) and hybrid routing algorithm results (blue) using the 1km\textsuperscript{2} city dataset. Unexpected parts due to missing or faulty data are marked in red.}
				\label{fig:eval-city-usefulness}
			\end{figure}
			
			\begin{figure}[h!]
				\begin{minipage}[t]{.48\textwidth}
					\begin{figcenter}
						\includegraphics[width=\textwidth]{images/qgis-routing-city-roads}
					\end{figcenter}
					\caption{Two unrealistic road crossings (yellow) across a six-lane road.}
					\label{fig:eval-city-road-crossing}
				\end{minipage}
				\hfill
				\begin{minipage}[t]{.48\textwidth}
					\begin{figcenter}
						\includegraphics[width=\textwidth]{images/qgis-routing-city-weights}
					\end{figcenter}
					\caption{Influence of weight factors. Lower value result in a stronger preference for roads.}
					\label{fig:eval-city-weights}
				\end{minipage}
			\end{figure}
			
			% Interesting passages from the rural dataset
			\noindent
			Figures \Cref{fig:eval-rural-routing-6} and \ref{fig:eval-rural-graph-based-comparison} give some examples on routes in a rural area with large open spaces.
			Two findings can be inferred from the analysis of the rural routing results.
			
			\begin{itemize}
				\item Datasets with larger open areas and irregular distributed obstacles, as in the \enquote{OSM rural} datasets, can greatly benefit from the hybrid routing algorithm because graph-based routes might take long detours due to a sparse road network.
				\item The difference between graph-based routes and routes from the hybrid routing algorithm is significantly larger.
				They might not have any location in common and graph-based routes tend to be significantly longer.
				This can be seen in \Cref{fig:eval-rural-graph-based-comparison-6} with a 1.73 km long graph-based and 0.67km long expected route.
				\Cref{fig:eval-rural-graph-based-comparison-17} shows a similar behavior even though the difference in distance is smaller because the graph-based, expected and actual route share common parts.
				\item The accuracy and therefore the usefulness for real-world applications (such as navigation apps) heavily relies on the amount of details in the dataset.
				\Cref{fig:eval-rural-routing-6-osm} illustrates the problem of missing data, which are in this case missing ditches within the farmland.
				The determined route has a length of 365m while the shortest possible route (determined using aerial imagery as seen in \Cref{fig:eval-rural-routing-6-aerial}) is 677m long.
			\end{itemize}
			
			\begin{figure}[h!]
				\begin{minipage}[t]{.48\textwidth}
					\begin{subfigure}[t]{\linewidth}
						\includegraphics[width=\textwidth]{images/qgis-routing-rural-routing-6-osm}
						\caption{The expected route differs significantly from the actual route taken by the agent. The dashed line is an alternative route under the assumption, that the farmland is reachable from the upper road.}
						\label{fig:eval-rural-routing-6-osm}
					\end{subfigure}
				\end{minipage}
				\hfill
				\begin{minipage}[t]{.48\textwidth}
					\begin{subfigure}[t]{\linewidth}
						\includegraphics[width=\textwidth]{images/qgis-routing-rural-routing-6-aerial}
						\caption{Aerial imagery shows the amount of missing data, which in this case are numerous missing ditches in the arable land.}
						\label{fig:eval-rural-routing-6-aerial}
					\end{subfigure}
				\end{minipage}
				\caption[Illustration of routing problems and different weight-function values.]{Routing on farmland illustrating the importance of correct data in the routing result showing the expected route (green) and determined route by the hybrid routing algorithm (green).}
				\label{fig:eval-rural-routing-6}
			\end{figure}
			
			\begin{figure}[h!]
				\begin{minipage}[t]{.48\textwidth}
					\begin{subfigure}[t]{\linewidth}
						\includegraphics[width=\textwidth]{images/qgis-routing-rural-routing-6-graph-based}
						\caption{Same waypoints from \Cref{fig:eval-rural-routing-6} with the additional result of a graph-based routing request creating a 2.6 times longer path.}
						\label{fig:eval-rural-graph-based-comparison-6}
					\end{subfigure}
				\end{minipage}
				\hfill
				\begin{minipage}[t]{.48\textwidth}
					\begin{subfigure}[t]{\linewidth}
						\includegraphics[width=\textwidth]{images/qgis-routing-rural-routing-17-graph-based}
						\caption{Detours created by a graph-based routing algorithm, even though the first part of the result is equal to the one of the hybrid routing algorithm. The shortcut of the actual result (horizontal part of the blue route) is a result of missing obstacles similar to the situation visible in \Cref{fig:eval-rural-routing-6-aerial}.}
						\label{fig:eval-rural-graph-based-comparison-17}
					\end{subfigure}
				\end{minipage}
				\caption[Visualization of detour of graph-based routes.]{Comparison of graph-based routing results (red) with the expected route (green) and actual results (blue) of the hybrid routing algorithm.}
				\label{fig:eval-rural-graph-based-comparison}
			\end{figure}
			
			In general the manual route analysis showed that the route quality is determined by two main factors:
			The quality (mainly completeness) of the data and the weighting factor.
			For densely built-up city datasets the routes might still be good, even though a graph-based routing algorithm might yield similar routes.
		
		\subsubsection{Mathematical route quality analysis}
		
			Two mathematical approaches were used to obtain quantifiable results on the route quality.
			This was done by comparing the expected route to the graph-based and hybrid routing algorithm results.
			The first metric uses the beeline distance between two waypoints and compares it to the distances of the three corresponding routes.
			The second metric uses the Hausdorff distance (described below) to obtain a numeric similarity between the expected route and the two corresponding routing results.
			Both comparisons were performed on the first ten routing requests of the city and rural OSM datasets.
			
			\Cref{fig:eval-route-distances} shows the relation between route and the beeline distances.
			The hybrid routing algorithm creates the shortest routes on average, which tend to be even shorter than the expected routes.
			Even though determining shortest routes is the overall goal, it indicates that missing obstacles lead to unrealistically short routes.
			This is in line with the results of the previous manual route analysis, according to which missing data leads to short but unrealistic routes.
			\begin{wrapfigure}{r}{0.35\textwidth}
%				\vspace{-0.55\baselineskip}
				\begin{figcenter}
					\begin{tikzpicture}
						[
						every node/.append style={outer sep=0.5mm, inner sep=0}
						]
						\def\d{0.75}
						
						\node (c00) at (0.6*\d ,0.1) {};
						\node (c01) at (1*\d   ,1*\d) {};
						\node (c02) at (0.15*\d ,2*\d) {};
						\node (c03) at (0.55*\d,3*\d) {};
						\node (c04) at (0      ,3.7*\d) {};
						
						\node (c10) at (1.8*\d  ,0.25) {};
						\node (c11) at (2.0*\d,0.9*\d) {};
						\node (c12) at (1.2*\d,1.8*\d) {};
						\node (c13) at (3*\d  ,2.8*\d) {};
						\node (c14) at (1.2*\d,3.8*\d) {};
						
						\draw (c00.center) -- (c01.center) -- (c02.center) -- (c03.center) -- (c04.center);
						\draw (c10.center) -- (c11.center) -- (c12.center) -- (c13.center) -- (c14.center);
						%						\draw (c01.center) -- (c02.center) -- (c03.center) -- (c04.center);
						%						\draw (c11.center) -- (c12.center) -- (c13.center) -- (c14.center);
						
						\draw[>=stealth',<->,densely dotted] (c00) -- (c10);
						\draw[>=stealth',<->,densely dotted] (c01) -- (c11);
						\draw[>=stealth',<->,densely dotted] (c02) -- (c12);
						\draw[>=stealth',<->,thick,red] (c03) -- (c13);
						\draw[>=stealth',<->,densely dotted] (c04) -- (c14);
						
						% Enlarge BBOX to avoid cropping off some of the arrow heads:
						\node (bbox_a) at (-0.05,0.05) {};
						\node (bbox_b) at (3*\d+0.05,3.8*\d+0.05) {};
					\end{tikzpicture}
				\end{figcenter}
				\caption[Illustration of the Hausdorff distance.]{Hausdorff distance (red) between two linestrings.}
				\label{fig:hausdorff-distance}
			\end{wrapfigure}
			The route length comparison also supports the previous results, that graph-based routing leads to longer routes with sometimes significant detours, which can be seen in the routing requests 6 and 7 in \Cref{fig:eval-route-distances-rural}.
			But comparing route lengths only yields a vague indication of the route's quality.
			The second metric uses the \term{Hausdorff distance} to quantify the similarity of two routes by determining the maximum distance between them as illustrated in \Cref{fig:hausdorff-distance}.
			\Cref{fig:eval-hausdorff} shows the results for both OSM datasets and yields some insights into the route similarities:
			\begin{itemize}
				\item In both datasets, the average Hausdorff distances of the routes determined using the hybrid routing algorithm are shorter than those of the graph-based algorithm.
				\item The difference is smaller in the city dataset, which supports the aforementioned hypothesis, that the buildings in a city create corridors leading to similar routes for graph-based and geometric routing algorithms.
				\item Except for the routing requests 6 and 7 in the \enquote{OSM rural} datasets, the Hausdorff distances of the routes by the hybrid routing algorithm tend to be larger than the distances of the graph-based routes.
				This is likely due to missing data as discussed in the previous section and illustrated in \Cref{fig:eval-rural-routing-6} leading to shorter but less realistic routes.
			\end{itemize}
			
			\begin{figure}[h!]
				\begin{subfigure}[t]{\linewidth}
					\begin{figcenter}
						%% Creator: Matplotlib, PGF backend
%%
%% To include the figure in your LaTeX document, write
%%   \input{<filename>.pgf}
%%
%% Make sure the required packages are loaded in your preamble
%%   \usepackage{pgf}
%%
%% Also ensure that all the required font packages are loaded; for instance,
%% the lmodern package is sometimes necessary when using math font.
%%   \usepackage{lmodern}
%%
%% Figures using additional raster images can only be included by \input if
%% they are in the same directory as the main LaTeX file. For loading figures
%% from other directories you can use the `import` package
%%   \usepackage{import}
%%
%% and then include the figures with
%%   \import{<path to file>}{<filename>.pgf}
%%
%% Matplotlib used the following preamble
%%   
%%   \usepackage{fontspec}
%%   \setmainfont{DejaVuSerif.ttf}[Path=\detokenize{/home/hauke/.local/lib/python3.11/site-packages/matplotlib/mpl-data/fonts/ttf/}]
%%   \setsansfont{DroidSans.ttf}[Path=\detokenize{/usr/share/fonts/droid/}]
%%   \setmonofont{DejaVuSansMono.ttf}[Path=\detokenize{/home/hauke/.local/lib/python3.11/site-packages/matplotlib/mpl-data/fonts/ttf/}]
%%   \makeatletter\@ifpackageloaded{underscore}{}{\usepackage[strings]{underscore}}\makeatother
%%
\begingroup%
\makeatletter%
\begin{pgfpicture}%
\pgfpathrectangle{\pgfpointorigin}{\pgfqpoint{6.086410in}{1.715788in}}%
\pgfusepath{use as bounding box, clip}%
\begin{pgfscope}%
\pgfsetbuttcap%
\pgfsetmiterjoin%
\definecolor{currentfill}{rgb}{1.000000,1.000000,1.000000}%
\pgfsetfillcolor{currentfill}%
\pgfsetlinewidth{0.000000pt}%
\definecolor{currentstroke}{rgb}{1.000000,1.000000,1.000000}%
\pgfsetstrokecolor{currentstroke}%
\pgfsetdash{}{0pt}%
\pgfpathmoveto{\pgfqpoint{0.000000in}{0.000000in}}%
\pgfpathlineto{\pgfqpoint{6.086410in}{0.000000in}}%
\pgfpathlineto{\pgfqpoint{6.086410in}{1.715788in}}%
\pgfpathlineto{\pgfqpoint{0.000000in}{1.715788in}}%
\pgfpathlineto{\pgfqpoint{0.000000in}{0.000000in}}%
\pgfpathclose%
\pgfusepath{fill}%
\end{pgfscope}%
\begin{pgfscope}%
\pgfsetbuttcap%
\pgfsetmiterjoin%
\definecolor{currentfill}{rgb}{1.000000,1.000000,1.000000}%
\pgfsetfillcolor{currentfill}%
\pgfsetlinewidth{0.000000pt}%
\definecolor{currentstroke}{rgb}{0.000000,0.000000,0.000000}%
\pgfsetstrokecolor{currentstroke}%
\pgfsetstrokeopacity{0.000000}%
\pgfsetdash{}{0pt}%
\pgfpathmoveto{\pgfqpoint{0.641586in}{0.451389in}}%
\pgfpathlineto{\pgfqpoint{4.921145in}{0.451389in}}%
\pgfpathlineto{\pgfqpoint{4.921145in}{1.715788in}}%
\pgfpathlineto{\pgfqpoint{0.641586in}{1.715788in}}%
\pgfpathlineto{\pgfqpoint{0.641586in}{0.451389in}}%
\pgfpathclose%
\pgfusepath{fill}%
\end{pgfscope}%
\begin{pgfscope}%
\definecolor{textcolor}{rgb}{0.150000,0.150000,0.150000}%
\pgfsetstrokecolor{textcolor}%
\pgfsetfillcolor{textcolor}%
\pgftext[x=0.836112in,y=0.319444in,,top]{\color{textcolor}\sffamily\fontsize{9.000000}{10.800000}\selectfont 1}%
\end{pgfscope}%
\begin{pgfscope}%
\definecolor{textcolor}{rgb}{0.150000,0.150000,0.150000}%
\pgfsetstrokecolor{textcolor}%
\pgfsetfillcolor{textcolor}%
\pgftext[x=1.225163in,y=0.319444in,,top]{\color{textcolor}\sffamily\fontsize{9.000000}{10.800000}\selectfont 2}%
\end{pgfscope}%
\begin{pgfscope}%
\definecolor{textcolor}{rgb}{0.150000,0.150000,0.150000}%
\pgfsetstrokecolor{textcolor}%
\pgfsetfillcolor{textcolor}%
\pgftext[x=1.614213in,y=0.319444in,,top]{\color{textcolor}\sffamily\fontsize{9.000000}{10.800000}\selectfont 3}%
\end{pgfscope}%
\begin{pgfscope}%
\definecolor{textcolor}{rgb}{0.150000,0.150000,0.150000}%
\pgfsetstrokecolor{textcolor}%
\pgfsetfillcolor{textcolor}%
\pgftext[x=2.003264in,y=0.319444in,,top]{\color{textcolor}\sffamily\fontsize{9.000000}{10.800000}\selectfont 4}%
\end{pgfscope}%
\begin{pgfscope}%
\definecolor{textcolor}{rgb}{0.150000,0.150000,0.150000}%
\pgfsetstrokecolor{textcolor}%
\pgfsetfillcolor{textcolor}%
\pgftext[x=2.392315in,y=0.319444in,,top]{\color{textcolor}\sffamily\fontsize{9.000000}{10.800000}\selectfont 5}%
\end{pgfscope}%
\begin{pgfscope}%
\definecolor{textcolor}{rgb}{0.150000,0.150000,0.150000}%
\pgfsetstrokecolor{textcolor}%
\pgfsetfillcolor{textcolor}%
\pgftext[x=2.781366in,y=0.319444in,,top]{\color{textcolor}\sffamily\fontsize{9.000000}{10.800000}\selectfont 6}%
\end{pgfscope}%
\begin{pgfscope}%
\definecolor{textcolor}{rgb}{0.150000,0.150000,0.150000}%
\pgfsetstrokecolor{textcolor}%
\pgfsetfillcolor{textcolor}%
\pgftext[x=3.170416in,y=0.319444in,,top]{\color{textcolor}\sffamily\fontsize{9.000000}{10.800000}\selectfont 7}%
\end{pgfscope}%
\begin{pgfscope}%
\definecolor{textcolor}{rgb}{0.150000,0.150000,0.150000}%
\pgfsetstrokecolor{textcolor}%
\pgfsetfillcolor{textcolor}%
\pgftext[x=3.559467in,y=0.319444in,,top]{\color{textcolor}\sffamily\fontsize{9.000000}{10.800000}\selectfont 8}%
\end{pgfscope}%
\begin{pgfscope}%
\definecolor{textcolor}{rgb}{0.150000,0.150000,0.150000}%
\pgfsetstrokecolor{textcolor}%
\pgfsetfillcolor{textcolor}%
\pgftext[x=3.948518in,y=0.319444in,,top]{\color{textcolor}\sffamily\fontsize{9.000000}{10.800000}\selectfont 9}%
\end{pgfscope}%
\begin{pgfscope}%
\definecolor{textcolor}{rgb}{0.150000,0.150000,0.150000}%
\pgfsetstrokecolor{textcolor}%
\pgfsetfillcolor{textcolor}%
\pgftext[x=4.337569in,y=0.319444in,,top]{\color{textcolor}\sffamily\fontsize{9.000000}{10.800000}\selectfont 10}%
\end{pgfscope}%
\begin{pgfscope}%
\definecolor{textcolor}{rgb}{0.150000,0.150000,0.150000}%
\pgfsetstrokecolor{textcolor}%
\pgfsetfillcolor{textcolor}%
\pgftext[x=4.726620in,y=0.319444in,,top]{\color{textcolor}\sffamily\fontsize{9.000000}{10.800000}\selectfont mean}%
\end{pgfscope}%
\begin{pgfscope}%
\definecolor{textcolor}{rgb}{0.150000,0.150000,0.150000}%
\pgfsetstrokecolor{textcolor}%
\pgfsetfillcolor{textcolor}%
\pgftext[x=2.781366in,y=0.125000in,,top]{\color{textcolor}\sffamily\fontsize{9.000000}{10.800000}\selectfont Routing request}%
\end{pgfscope}%
\begin{pgfscope}%
\pgfpathrectangle{\pgfqpoint{0.641586in}{0.451389in}}{\pgfqpoint{4.279559in}{1.264399in}}%
\pgfusepath{clip}%
\pgfsetroundcap%
\pgfsetroundjoin%
\pgfsetlinewidth{1.003750pt}%
\definecolor{currentstroke}{rgb}{0.800000,0.800000,0.800000}%
\pgfsetstrokecolor{currentstroke}%
\pgfsetdash{}{0pt}%
\pgfpathmoveto{\pgfqpoint{0.641586in}{0.537753in}}%
\pgfpathlineto{\pgfqpoint{4.921145in}{0.537753in}}%
\pgfusepath{stroke}%
\end{pgfscope}%
\begin{pgfscope}%
\definecolor{textcolor}{rgb}{0.150000,0.150000,0.150000}%
\pgfsetstrokecolor{textcolor}%
\pgfsetfillcolor{textcolor}%
\pgftext[x=0.338438in, y=0.490267in, left, base]{\color{textcolor}\sffamily\fontsize{9.000000}{10.800000}\selectfont 1.0}%
\end{pgfscope}%
\begin{pgfscope}%
\pgfpathrectangle{\pgfqpoint{0.641586in}{0.451389in}}{\pgfqpoint{4.279559in}{1.264399in}}%
\pgfusepath{clip}%
\pgfsetroundcap%
\pgfsetroundjoin%
\pgfsetlinewidth{1.003750pt}%
\definecolor{currentstroke}{rgb}{0.800000,0.800000,0.800000}%
\pgfsetstrokecolor{currentstroke}%
\pgfsetdash{}{0pt}%
\pgfpathmoveto{\pgfqpoint{0.641586in}{0.969573in}}%
\pgfpathlineto{\pgfqpoint{4.921145in}{0.969573in}}%
\pgfusepath{stroke}%
\end{pgfscope}%
\begin{pgfscope}%
\definecolor{textcolor}{rgb}{0.150000,0.150000,0.150000}%
\pgfsetstrokecolor{textcolor}%
\pgfsetfillcolor{textcolor}%
\pgftext[x=0.338438in, y=0.922088in, left, base]{\color{textcolor}\sffamily\fontsize{9.000000}{10.800000}\selectfont 1.5}%
\end{pgfscope}%
\begin{pgfscope}%
\pgfpathrectangle{\pgfqpoint{0.641586in}{0.451389in}}{\pgfqpoint{4.279559in}{1.264399in}}%
\pgfusepath{clip}%
\pgfsetroundcap%
\pgfsetroundjoin%
\pgfsetlinewidth{1.003750pt}%
\definecolor{currentstroke}{rgb}{0.800000,0.800000,0.800000}%
\pgfsetstrokecolor{currentstroke}%
\pgfsetdash{}{0pt}%
\pgfpathmoveto{\pgfqpoint{0.641586in}{1.401393in}}%
\pgfpathlineto{\pgfqpoint{4.921145in}{1.401393in}}%
\pgfusepath{stroke}%
\end{pgfscope}%
\begin{pgfscope}%
\definecolor{textcolor}{rgb}{0.150000,0.150000,0.150000}%
\pgfsetstrokecolor{textcolor}%
\pgfsetfillcolor{textcolor}%
\pgftext[x=0.338438in, y=1.353908in, left, base]{\color{textcolor}\sffamily\fontsize{9.000000}{10.800000}\selectfont 2.0}%
\end{pgfscope}%
\begin{pgfscope}%
\definecolor{textcolor}{rgb}{0.150000,0.150000,0.150000}%
\pgfsetstrokecolor{textcolor}%
\pgfsetfillcolor{textcolor}%
\pgftext[x=0.094971in, y=0.628907in, left, base,rotate=90.000000]{\color{textcolor}\sffamily\fontsize{9.000000}{10.800000}\selectfont Route distance /}%
\end{pgfscope}%
\begin{pgfscope}%
\definecolor{textcolor}{rgb}{0.150000,0.150000,0.150000}%
\pgfsetstrokecolor{textcolor}%
\pgfsetfillcolor{textcolor}%
\pgftext[x=0.238965in, y=0.626709in, left, base,rotate=90.000000]{\color{textcolor}\sffamily\fontsize{9.000000}{10.800000}\selectfont beeline distance}%
\end{pgfscope}%
\begin{pgfscope}%
\pgfpathrectangle{\pgfqpoint{0.641586in}{0.451389in}}{\pgfqpoint{4.279559in}{1.264399in}}%
\pgfusepath{clip}%
\pgfsetbuttcap%
\pgfsetmiterjoin%
\definecolor{currentfill}{rgb}{0.460784,0.749020,0.443137}%
\pgfsetfillcolor{currentfill}%
\pgfsetlinewidth{1.003750pt}%
\definecolor{currentstroke}{rgb}{1.000000,1.000000,1.000000}%
\pgfsetstrokecolor{currentstroke}%
\pgfsetdash{}{0pt}%
\pgfpathmoveto{\pgfqpoint{0.680491in}{-0.325888in}}%
\pgfpathlineto{\pgfqpoint{0.784238in}{-0.325888in}}%
\pgfpathlineto{\pgfqpoint{0.784238in}{1.307054in}}%
\pgfpathlineto{\pgfqpoint{0.680491in}{1.307054in}}%
\pgfpathlineto{\pgfqpoint{0.680491in}{-0.325888in}}%
\pgfpathclose%
\pgfusepath{stroke,fill}%
\end{pgfscope}%
\begin{pgfscope}%
\pgfpathrectangle{\pgfqpoint{0.641586in}{0.451389in}}{\pgfqpoint{4.279559in}{1.264399in}}%
\pgfusepath{clip}%
\pgfsetbuttcap%
\pgfsetmiterjoin%
\definecolor{currentfill}{rgb}{0.460784,0.749020,0.443137}%
\pgfsetfillcolor{currentfill}%
\pgfsetlinewidth{1.003750pt}%
\definecolor{currentstroke}{rgb}{1.000000,1.000000,1.000000}%
\pgfsetstrokecolor{currentstroke}%
\pgfsetdash{}{0pt}%
\pgfpathmoveto{\pgfqpoint{1.069542in}{-0.325888in}}%
\pgfpathlineto{\pgfqpoint{1.173289in}{-0.325888in}}%
\pgfpathlineto{\pgfqpoint{1.173289in}{0.742690in}}%
\pgfpathlineto{\pgfqpoint{1.069542in}{0.742690in}}%
\pgfpathlineto{\pgfqpoint{1.069542in}{-0.325888in}}%
\pgfpathclose%
\pgfusepath{stroke,fill}%
\end{pgfscope}%
\begin{pgfscope}%
\pgfpathrectangle{\pgfqpoint{0.641586in}{0.451389in}}{\pgfqpoint{4.279559in}{1.264399in}}%
\pgfusepath{clip}%
\pgfsetbuttcap%
\pgfsetmiterjoin%
\definecolor{currentfill}{rgb}{0.460784,0.749020,0.443137}%
\pgfsetfillcolor{currentfill}%
\pgfsetlinewidth{1.003750pt}%
\definecolor{currentstroke}{rgb}{1.000000,1.000000,1.000000}%
\pgfsetstrokecolor{currentstroke}%
\pgfsetdash{}{0pt}%
\pgfpathmoveto{\pgfqpoint{1.458593in}{-0.325888in}}%
\pgfpathlineto{\pgfqpoint{1.562340in}{-0.325888in}}%
\pgfpathlineto{\pgfqpoint{1.562340in}{0.919651in}}%
\pgfpathlineto{\pgfqpoint{1.458593in}{0.919651in}}%
\pgfpathlineto{\pgfqpoint{1.458593in}{-0.325888in}}%
\pgfpathclose%
\pgfusepath{stroke,fill}%
\end{pgfscope}%
\begin{pgfscope}%
\pgfpathrectangle{\pgfqpoint{0.641586in}{0.451389in}}{\pgfqpoint{4.279559in}{1.264399in}}%
\pgfusepath{clip}%
\pgfsetbuttcap%
\pgfsetmiterjoin%
\definecolor{currentfill}{rgb}{0.460784,0.749020,0.443137}%
\pgfsetfillcolor{currentfill}%
\pgfsetlinewidth{1.003750pt}%
\definecolor{currentstroke}{rgb}{1.000000,1.000000,1.000000}%
\pgfsetstrokecolor{currentstroke}%
\pgfsetdash{}{0pt}%
\pgfpathmoveto{\pgfqpoint{1.847644in}{-0.325888in}}%
\pgfpathlineto{\pgfqpoint{1.951391in}{-0.325888in}}%
\pgfpathlineto{\pgfqpoint{1.951391in}{0.553689in}}%
\pgfpathlineto{\pgfqpoint{1.847644in}{0.553689in}}%
\pgfpathlineto{\pgfqpoint{1.847644in}{-0.325888in}}%
\pgfpathclose%
\pgfusepath{stroke,fill}%
\end{pgfscope}%
\begin{pgfscope}%
\pgfpathrectangle{\pgfqpoint{0.641586in}{0.451389in}}{\pgfqpoint{4.279559in}{1.264399in}}%
\pgfusepath{clip}%
\pgfsetbuttcap%
\pgfsetmiterjoin%
\definecolor{currentfill}{rgb}{0.460784,0.749020,0.443137}%
\pgfsetfillcolor{currentfill}%
\pgfsetlinewidth{1.003750pt}%
\definecolor{currentstroke}{rgb}{1.000000,1.000000,1.000000}%
\pgfsetstrokecolor{currentstroke}%
\pgfsetdash{}{0pt}%
\pgfpathmoveto{\pgfqpoint{2.236695in}{-0.325888in}}%
\pgfpathlineto{\pgfqpoint{2.340441in}{-0.325888in}}%
\pgfpathlineto{\pgfqpoint{2.340441in}{1.254489in}}%
\pgfpathlineto{\pgfqpoint{2.236695in}{1.254489in}}%
\pgfpathlineto{\pgfqpoint{2.236695in}{-0.325888in}}%
\pgfpathclose%
\pgfusepath{stroke,fill}%
\end{pgfscope}%
\begin{pgfscope}%
\pgfpathrectangle{\pgfqpoint{0.641586in}{0.451389in}}{\pgfqpoint{4.279559in}{1.264399in}}%
\pgfusepath{clip}%
\pgfsetbuttcap%
\pgfsetmiterjoin%
\definecolor{currentfill}{rgb}{0.460784,0.749020,0.443137}%
\pgfsetfillcolor{currentfill}%
\pgfsetlinewidth{1.003750pt}%
\definecolor{currentstroke}{rgb}{1.000000,1.000000,1.000000}%
\pgfsetstrokecolor{currentstroke}%
\pgfsetdash{}{0pt}%
\pgfpathmoveto{\pgfqpoint{2.625745in}{-0.325888in}}%
\pgfpathlineto{\pgfqpoint{2.729492in}{-0.325888in}}%
\pgfpathlineto{\pgfqpoint{2.729492in}{0.860332in}}%
\pgfpathlineto{\pgfqpoint{2.625745in}{0.860332in}}%
\pgfpathlineto{\pgfqpoint{2.625745in}{-0.325888in}}%
\pgfpathclose%
\pgfusepath{stroke,fill}%
\end{pgfscope}%
\begin{pgfscope}%
\pgfpathrectangle{\pgfqpoint{0.641586in}{0.451389in}}{\pgfqpoint{4.279559in}{1.264399in}}%
\pgfusepath{clip}%
\pgfsetbuttcap%
\pgfsetmiterjoin%
\definecolor{currentfill}{rgb}{0.460784,0.749020,0.443137}%
\pgfsetfillcolor{currentfill}%
\pgfsetlinewidth{1.003750pt}%
\definecolor{currentstroke}{rgb}{1.000000,1.000000,1.000000}%
\pgfsetstrokecolor{currentstroke}%
\pgfsetdash{}{0pt}%
\pgfpathmoveto{\pgfqpoint{3.014796in}{-0.325888in}}%
\pgfpathlineto{\pgfqpoint{3.118543in}{-0.325888in}}%
\pgfpathlineto{\pgfqpoint{3.118543in}{0.930075in}}%
\pgfpathlineto{\pgfqpoint{3.014796in}{0.930075in}}%
\pgfpathlineto{\pgfqpoint{3.014796in}{-0.325888in}}%
\pgfpathclose%
\pgfusepath{stroke,fill}%
\end{pgfscope}%
\begin{pgfscope}%
\pgfpathrectangle{\pgfqpoint{0.641586in}{0.451389in}}{\pgfqpoint{4.279559in}{1.264399in}}%
\pgfusepath{clip}%
\pgfsetbuttcap%
\pgfsetmiterjoin%
\definecolor{currentfill}{rgb}{0.460784,0.749020,0.443137}%
\pgfsetfillcolor{currentfill}%
\pgfsetlinewidth{1.003750pt}%
\definecolor{currentstroke}{rgb}{1.000000,1.000000,1.000000}%
\pgfsetstrokecolor{currentstroke}%
\pgfsetdash{}{0pt}%
\pgfpathmoveto{\pgfqpoint{3.403847in}{-0.325888in}}%
\pgfpathlineto{\pgfqpoint{3.507594in}{-0.325888in}}%
\pgfpathlineto{\pgfqpoint{3.507594in}{1.025255in}}%
\pgfpathlineto{\pgfqpoint{3.403847in}{1.025255in}}%
\pgfpathlineto{\pgfqpoint{3.403847in}{-0.325888in}}%
\pgfpathclose%
\pgfusepath{stroke,fill}%
\end{pgfscope}%
\begin{pgfscope}%
\pgfpathrectangle{\pgfqpoint{0.641586in}{0.451389in}}{\pgfqpoint{4.279559in}{1.264399in}}%
\pgfusepath{clip}%
\pgfsetbuttcap%
\pgfsetmiterjoin%
\definecolor{currentfill}{rgb}{0.460784,0.749020,0.443137}%
\pgfsetfillcolor{currentfill}%
\pgfsetlinewidth{1.003750pt}%
\definecolor{currentstroke}{rgb}{1.000000,1.000000,1.000000}%
\pgfsetstrokecolor{currentstroke}%
\pgfsetdash{}{0pt}%
\pgfpathmoveto{\pgfqpoint{3.792898in}{-0.325888in}}%
\pgfpathlineto{\pgfqpoint{3.896645in}{-0.325888in}}%
\pgfpathlineto{\pgfqpoint{3.896645in}{0.923694in}}%
\pgfpathlineto{\pgfqpoint{3.792898in}{0.923694in}}%
\pgfpathlineto{\pgfqpoint{3.792898in}{-0.325888in}}%
\pgfpathclose%
\pgfusepath{stroke,fill}%
\end{pgfscope}%
\begin{pgfscope}%
\pgfpathrectangle{\pgfqpoint{0.641586in}{0.451389in}}{\pgfqpoint{4.279559in}{1.264399in}}%
\pgfusepath{clip}%
\pgfsetbuttcap%
\pgfsetmiterjoin%
\definecolor{currentfill}{rgb}{0.460784,0.749020,0.443137}%
\pgfsetfillcolor{currentfill}%
\pgfsetlinewidth{1.003750pt}%
\definecolor{currentstroke}{rgb}{1.000000,1.000000,1.000000}%
\pgfsetstrokecolor{currentstroke}%
\pgfsetdash{}{0pt}%
\pgfpathmoveto{\pgfqpoint{4.181948in}{-0.325888in}}%
\pgfpathlineto{\pgfqpoint{4.285695in}{-0.325888in}}%
\pgfpathlineto{\pgfqpoint{4.285695in}{1.087734in}}%
\pgfpathlineto{\pgfqpoint{4.181948in}{1.087734in}}%
\pgfpathlineto{\pgfqpoint{4.181948in}{-0.325888in}}%
\pgfpathclose%
\pgfusepath{stroke,fill}%
\end{pgfscope}%
\begin{pgfscope}%
\pgfpathrectangle{\pgfqpoint{0.641586in}{0.451389in}}{\pgfqpoint{4.279559in}{1.264399in}}%
\pgfusepath{clip}%
\pgfsetbuttcap%
\pgfsetmiterjoin%
\definecolor{currentfill}{rgb}{0.460784,0.749020,0.443137}%
\pgfsetfillcolor{currentfill}%
\pgfsetlinewidth{1.003750pt}%
\definecolor{currentstroke}{rgb}{1.000000,1.000000,1.000000}%
\pgfsetstrokecolor{currentstroke}%
\pgfsetdash{}{0pt}%
\pgfpathmoveto{\pgfqpoint{4.570999in}{-0.325888in}}%
\pgfpathlineto{\pgfqpoint{4.674746in}{-0.325888in}}%
\pgfpathlineto{\pgfqpoint{4.674746in}{0.960466in}}%
\pgfpathlineto{\pgfqpoint{4.570999in}{0.960466in}}%
\pgfpathlineto{\pgfqpoint{4.570999in}{-0.325888in}}%
\pgfpathclose%
\pgfusepath{stroke,fill}%
\end{pgfscope}%
\begin{pgfscope}%
\pgfpathrectangle{\pgfqpoint{0.641586in}{0.451389in}}{\pgfqpoint{4.279559in}{1.264399in}}%
\pgfusepath{clip}%
\pgfsetbuttcap%
\pgfsetmiterjoin%
\definecolor{currentfill}{rgb}{0.349020,0.490196,0.749020}%
\pgfsetfillcolor{currentfill}%
\pgfsetlinewidth{1.003750pt}%
\definecolor{currentstroke}{rgb}{1.000000,1.000000,1.000000}%
\pgfsetstrokecolor{currentstroke}%
\pgfsetdash{}{0pt}%
\pgfpathmoveto{\pgfqpoint{0.784238in}{-0.325888in}}%
\pgfpathlineto{\pgfqpoint{0.887985in}{-0.325888in}}%
\pgfpathlineto{\pgfqpoint{0.887985in}{1.271194in}}%
\pgfpathlineto{\pgfqpoint{0.784238in}{1.271194in}}%
\pgfpathlineto{\pgfqpoint{0.784238in}{-0.325888in}}%
\pgfpathclose%
\pgfusepath{stroke,fill}%
\end{pgfscope}%
\begin{pgfscope}%
\pgfpathrectangle{\pgfqpoint{0.641586in}{0.451389in}}{\pgfqpoint{4.279559in}{1.264399in}}%
\pgfusepath{clip}%
\pgfsetbuttcap%
\pgfsetmiterjoin%
\definecolor{currentfill}{rgb}{0.349020,0.490196,0.749020}%
\pgfsetfillcolor{currentfill}%
\pgfsetlinewidth{1.003750pt}%
\definecolor{currentstroke}{rgb}{1.000000,1.000000,1.000000}%
\pgfsetstrokecolor{currentstroke}%
\pgfsetdash{}{0pt}%
\pgfpathmoveto{\pgfqpoint{1.173289in}{-0.325888in}}%
\pgfpathlineto{\pgfqpoint{1.277036in}{-0.325888in}}%
\pgfpathlineto{\pgfqpoint{1.277036in}{0.682497in}}%
\pgfpathlineto{\pgfqpoint{1.173289in}{0.682497in}}%
\pgfpathlineto{\pgfqpoint{1.173289in}{-0.325888in}}%
\pgfpathclose%
\pgfusepath{stroke,fill}%
\end{pgfscope}%
\begin{pgfscope}%
\pgfpathrectangle{\pgfqpoint{0.641586in}{0.451389in}}{\pgfqpoint{4.279559in}{1.264399in}}%
\pgfusepath{clip}%
\pgfsetbuttcap%
\pgfsetmiterjoin%
\definecolor{currentfill}{rgb}{0.349020,0.490196,0.749020}%
\pgfsetfillcolor{currentfill}%
\pgfsetlinewidth{1.003750pt}%
\definecolor{currentstroke}{rgb}{1.000000,1.000000,1.000000}%
\pgfsetstrokecolor{currentstroke}%
\pgfsetdash{}{0pt}%
\pgfpathmoveto{\pgfqpoint{1.562340in}{-0.325888in}}%
\pgfpathlineto{\pgfqpoint{1.666087in}{-0.325888in}}%
\pgfpathlineto{\pgfqpoint{1.666087in}{0.722822in}}%
\pgfpathlineto{\pgfqpoint{1.562340in}{0.722822in}}%
\pgfpathlineto{\pgfqpoint{1.562340in}{-0.325888in}}%
\pgfpathclose%
\pgfusepath{stroke,fill}%
\end{pgfscope}%
\begin{pgfscope}%
\pgfpathrectangle{\pgfqpoint{0.641586in}{0.451389in}}{\pgfqpoint{4.279559in}{1.264399in}}%
\pgfusepath{clip}%
\pgfsetbuttcap%
\pgfsetmiterjoin%
\definecolor{currentfill}{rgb}{0.349020,0.490196,0.749020}%
\pgfsetfillcolor{currentfill}%
\pgfsetlinewidth{1.003750pt}%
\definecolor{currentstroke}{rgb}{1.000000,1.000000,1.000000}%
\pgfsetstrokecolor{currentstroke}%
\pgfsetdash{}{0pt}%
\pgfpathmoveto{\pgfqpoint{1.951391in}{-0.325888in}}%
\pgfpathlineto{\pgfqpoint{2.055138in}{-0.325888in}}%
\pgfpathlineto{\pgfqpoint{2.055138in}{0.560543in}}%
\pgfpathlineto{\pgfqpoint{1.951391in}{0.560543in}}%
\pgfpathlineto{\pgfqpoint{1.951391in}{-0.325888in}}%
\pgfpathclose%
\pgfusepath{stroke,fill}%
\end{pgfscope}%
\begin{pgfscope}%
\pgfpathrectangle{\pgfqpoint{0.641586in}{0.451389in}}{\pgfqpoint{4.279559in}{1.264399in}}%
\pgfusepath{clip}%
\pgfsetbuttcap%
\pgfsetmiterjoin%
\definecolor{currentfill}{rgb}{0.349020,0.490196,0.749020}%
\pgfsetfillcolor{currentfill}%
\pgfsetlinewidth{1.003750pt}%
\definecolor{currentstroke}{rgb}{1.000000,1.000000,1.000000}%
\pgfsetstrokecolor{currentstroke}%
\pgfsetdash{}{0pt}%
\pgfpathmoveto{\pgfqpoint{2.340441in}{-0.325888in}}%
\pgfpathlineto{\pgfqpoint{2.444188in}{-0.325888in}}%
\pgfpathlineto{\pgfqpoint{2.444188in}{0.634284in}}%
\pgfpathlineto{\pgfqpoint{2.340441in}{0.634284in}}%
\pgfpathlineto{\pgfqpoint{2.340441in}{-0.325888in}}%
\pgfpathclose%
\pgfusepath{stroke,fill}%
\end{pgfscope}%
\begin{pgfscope}%
\pgfpathrectangle{\pgfqpoint{0.641586in}{0.451389in}}{\pgfqpoint{4.279559in}{1.264399in}}%
\pgfusepath{clip}%
\pgfsetbuttcap%
\pgfsetmiterjoin%
\definecolor{currentfill}{rgb}{0.349020,0.490196,0.749020}%
\pgfsetfillcolor{currentfill}%
\pgfsetlinewidth{1.003750pt}%
\definecolor{currentstroke}{rgb}{1.000000,1.000000,1.000000}%
\pgfsetstrokecolor{currentstroke}%
\pgfsetdash{}{0pt}%
\pgfpathmoveto{\pgfqpoint{2.729492in}{-0.325888in}}%
\pgfpathlineto{\pgfqpoint{2.833239in}{-0.325888in}}%
\pgfpathlineto{\pgfqpoint{2.833239in}{0.678706in}}%
\pgfpathlineto{\pgfqpoint{2.729492in}{0.678706in}}%
\pgfpathlineto{\pgfqpoint{2.729492in}{-0.325888in}}%
\pgfpathclose%
\pgfusepath{stroke,fill}%
\end{pgfscope}%
\begin{pgfscope}%
\pgfpathrectangle{\pgfqpoint{0.641586in}{0.451389in}}{\pgfqpoint{4.279559in}{1.264399in}}%
\pgfusepath{clip}%
\pgfsetbuttcap%
\pgfsetmiterjoin%
\definecolor{currentfill}{rgb}{0.349020,0.490196,0.749020}%
\pgfsetfillcolor{currentfill}%
\pgfsetlinewidth{1.003750pt}%
\definecolor{currentstroke}{rgb}{1.000000,1.000000,1.000000}%
\pgfsetstrokecolor{currentstroke}%
\pgfsetdash{}{0pt}%
\pgfpathmoveto{\pgfqpoint{3.118543in}{-0.325888in}}%
\pgfpathlineto{\pgfqpoint{3.222290in}{-0.325888in}}%
\pgfpathlineto{\pgfqpoint{3.222290in}{0.859033in}}%
\pgfpathlineto{\pgfqpoint{3.118543in}{0.859033in}}%
\pgfpathlineto{\pgfqpoint{3.118543in}{-0.325888in}}%
\pgfpathclose%
\pgfusepath{stroke,fill}%
\end{pgfscope}%
\begin{pgfscope}%
\pgfpathrectangle{\pgfqpoint{0.641586in}{0.451389in}}{\pgfqpoint{4.279559in}{1.264399in}}%
\pgfusepath{clip}%
\pgfsetbuttcap%
\pgfsetmiterjoin%
\definecolor{currentfill}{rgb}{0.349020,0.490196,0.749020}%
\pgfsetfillcolor{currentfill}%
\pgfsetlinewidth{1.003750pt}%
\definecolor{currentstroke}{rgb}{1.000000,1.000000,1.000000}%
\pgfsetstrokecolor{currentstroke}%
\pgfsetdash{}{0pt}%
\pgfpathmoveto{\pgfqpoint{3.507594in}{-0.325888in}}%
\pgfpathlineto{\pgfqpoint{3.611341in}{-0.325888in}}%
\pgfpathlineto{\pgfqpoint{3.611341in}{0.845438in}}%
\pgfpathlineto{\pgfqpoint{3.507594in}{0.845438in}}%
\pgfpathlineto{\pgfqpoint{3.507594in}{-0.325888in}}%
\pgfpathclose%
\pgfusepath{stroke,fill}%
\end{pgfscope}%
\begin{pgfscope}%
\pgfpathrectangle{\pgfqpoint{0.641586in}{0.451389in}}{\pgfqpoint{4.279559in}{1.264399in}}%
\pgfusepath{clip}%
\pgfsetbuttcap%
\pgfsetmiterjoin%
\definecolor{currentfill}{rgb}{0.349020,0.490196,0.749020}%
\pgfsetfillcolor{currentfill}%
\pgfsetlinewidth{1.003750pt}%
\definecolor{currentstroke}{rgb}{1.000000,1.000000,1.000000}%
\pgfsetstrokecolor{currentstroke}%
\pgfsetdash{}{0pt}%
\pgfpathmoveto{\pgfqpoint{3.896645in}{-0.325888in}}%
\pgfpathlineto{\pgfqpoint{4.000391in}{-0.325888in}}%
\pgfpathlineto{\pgfqpoint{4.000391in}{0.770601in}}%
\pgfpathlineto{\pgfqpoint{3.896645in}{0.770601in}}%
\pgfpathlineto{\pgfqpoint{3.896645in}{-0.325888in}}%
\pgfpathclose%
\pgfusepath{stroke,fill}%
\end{pgfscope}%
\begin{pgfscope}%
\pgfpathrectangle{\pgfqpoint{0.641586in}{0.451389in}}{\pgfqpoint{4.279559in}{1.264399in}}%
\pgfusepath{clip}%
\pgfsetbuttcap%
\pgfsetmiterjoin%
\definecolor{currentfill}{rgb}{0.349020,0.490196,0.749020}%
\pgfsetfillcolor{currentfill}%
\pgfsetlinewidth{1.003750pt}%
\definecolor{currentstroke}{rgb}{1.000000,1.000000,1.000000}%
\pgfsetstrokecolor{currentstroke}%
\pgfsetdash{}{0pt}%
\pgfpathmoveto{\pgfqpoint{4.285695in}{-0.325888in}}%
\pgfpathlineto{\pgfqpoint{4.389442in}{-0.325888in}}%
\pgfpathlineto{\pgfqpoint{4.389442in}{0.892496in}}%
\pgfpathlineto{\pgfqpoint{4.285695in}{0.892496in}}%
\pgfpathlineto{\pgfqpoint{4.285695in}{-0.325888in}}%
\pgfpathclose%
\pgfusepath{stroke,fill}%
\end{pgfscope}%
\begin{pgfscope}%
\pgfpathrectangle{\pgfqpoint{0.641586in}{0.451389in}}{\pgfqpoint{4.279559in}{1.264399in}}%
\pgfusepath{clip}%
\pgfsetbuttcap%
\pgfsetmiterjoin%
\definecolor{currentfill}{rgb}{0.349020,0.490196,0.749020}%
\pgfsetfillcolor{currentfill}%
\pgfsetlinewidth{1.003750pt}%
\definecolor{currentstroke}{rgb}{1.000000,1.000000,1.000000}%
\pgfsetstrokecolor{currentstroke}%
\pgfsetdash{}{0pt}%
\pgfpathmoveto{\pgfqpoint{4.674746in}{-0.325888in}}%
\pgfpathlineto{\pgfqpoint{4.778493in}{-0.325888in}}%
\pgfpathlineto{\pgfqpoint{4.778493in}{0.791761in}}%
\pgfpathlineto{\pgfqpoint{4.674746in}{0.791761in}}%
\pgfpathlineto{\pgfqpoint{4.674746in}{-0.325888in}}%
\pgfpathclose%
\pgfusepath{stroke,fill}%
\end{pgfscope}%
\begin{pgfscope}%
\pgfpathrectangle{\pgfqpoint{0.641586in}{0.451389in}}{\pgfqpoint{4.279559in}{1.264399in}}%
\pgfusepath{clip}%
\pgfsetbuttcap%
\pgfsetmiterjoin%
\definecolor{currentfill}{rgb}{0.852941,0.544118,0.370588}%
\pgfsetfillcolor{currentfill}%
\pgfsetlinewidth{1.003750pt}%
\definecolor{currentstroke}{rgb}{1.000000,1.000000,1.000000}%
\pgfsetstrokecolor{currentstroke}%
\pgfsetdash{}{0pt}%
\pgfpathmoveto{\pgfqpoint{0.887985in}{-0.325888in}}%
\pgfpathlineto{\pgfqpoint{0.991732in}{-0.325888in}}%
\pgfpathlineto{\pgfqpoint{0.991732in}{1.618565in}}%
\pgfpathlineto{\pgfqpoint{0.887985in}{1.618565in}}%
\pgfpathlineto{\pgfqpoint{0.887985in}{-0.325888in}}%
\pgfpathclose%
\pgfusepath{stroke,fill}%
\end{pgfscope}%
\begin{pgfscope}%
\pgfpathrectangle{\pgfqpoint{0.641586in}{0.451389in}}{\pgfqpoint{4.279559in}{1.264399in}}%
\pgfusepath{clip}%
\pgfsetbuttcap%
\pgfsetmiterjoin%
\definecolor{currentfill}{rgb}{0.852941,0.544118,0.370588}%
\pgfsetfillcolor{currentfill}%
\pgfsetlinewidth{1.003750pt}%
\definecolor{currentstroke}{rgb}{1.000000,1.000000,1.000000}%
\pgfsetstrokecolor{currentstroke}%
\pgfsetdash{}{0pt}%
\pgfpathmoveto{\pgfqpoint{1.277036in}{-0.325888in}}%
\pgfpathlineto{\pgfqpoint{1.380783in}{-0.325888in}}%
\pgfpathlineto{\pgfqpoint{1.380783in}{1.068830in}}%
\pgfpathlineto{\pgfqpoint{1.277036in}{1.068830in}}%
\pgfpathlineto{\pgfqpoint{1.277036in}{-0.325888in}}%
\pgfpathclose%
\pgfusepath{stroke,fill}%
\end{pgfscope}%
\begin{pgfscope}%
\pgfpathrectangle{\pgfqpoint{0.641586in}{0.451389in}}{\pgfqpoint{4.279559in}{1.264399in}}%
\pgfusepath{clip}%
\pgfsetbuttcap%
\pgfsetmiterjoin%
\definecolor{currentfill}{rgb}{0.852941,0.544118,0.370588}%
\pgfsetfillcolor{currentfill}%
\pgfsetlinewidth{1.003750pt}%
\definecolor{currentstroke}{rgb}{1.000000,1.000000,1.000000}%
\pgfsetstrokecolor{currentstroke}%
\pgfsetdash{}{0pt}%
\pgfpathmoveto{\pgfqpoint{1.666087in}{-0.325888in}}%
\pgfpathlineto{\pgfqpoint{1.769834in}{-0.325888in}}%
\pgfpathlineto{\pgfqpoint{1.769834in}{1.161862in}}%
\pgfpathlineto{\pgfqpoint{1.666087in}{1.161862in}}%
\pgfpathlineto{\pgfqpoint{1.666087in}{-0.325888in}}%
\pgfpathclose%
\pgfusepath{stroke,fill}%
\end{pgfscope}%
\begin{pgfscope}%
\pgfpathrectangle{\pgfqpoint{0.641586in}{0.451389in}}{\pgfqpoint{4.279559in}{1.264399in}}%
\pgfusepath{clip}%
\pgfsetbuttcap%
\pgfsetmiterjoin%
\definecolor{currentfill}{rgb}{0.852941,0.544118,0.370588}%
\pgfsetfillcolor{currentfill}%
\pgfsetlinewidth{1.003750pt}%
\definecolor{currentstroke}{rgb}{1.000000,1.000000,1.000000}%
\pgfsetstrokecolor{currentstroke}%
\pgfsetdash{}{0pt}%
\pgfpathmoveto{\pgfqpoint{2.055138in}{-0.325888in}}%
\pgfpathlineto{\pgfqpoint{2.158884in}{-0.325888in}}%
\pgfpathlineto{\pgfqpoint{2.158884in}{0.598130in}}%
\pgfpathlineto{\pgfqpoint{2.055138in}{0.598130in}}%
\pgfpathlineto{\pgfqpoint{2.055138in}{-0.325888in}}%
\pgfpathclose%
\pgfusepath{stroke,fill}%
\end{pgfscope}%
\begin{pgfscope}%
\pgfpathrectangle{\pgfqpoint{0.641586in}{0.451389in}}{\pgfqpoint{4.279559in}{1.264399in}}%
\pgfusepath{clip}%
\pgfsetbuttcap%
\pgfsetmiterjoin%
\definecolor{currentfill}{rgb}{0.852941,0.544118,0.370588}%
\pgfsetfillcolor{currentfill}%
\pgfsetlinewidth{1.003750pt}%
\definecolor{currentstroke}{rgb}{1.000000,1.000000,1.000000}%
\pgfsetstrokecolor{currentstroke}%
\pgfsetdash{}{0pt}%
\pgfpathmoveto{\pgfqpoint{2.444188in}{-0.325888in}}%
\pgfpathlineto{\pgfqpoint{2.547935in}{-0.325888in}}%
\pgfpathlineto{\pgfqpoint{2.547935in}{1.199276in}}%
\pgfpathlineto{\pgfqpoint{2.444188in}{1.199276in}}%
\pgfpathlineto{\pgfqpoint{2.444188in}{-0.325888in}}%
\pgfpathclose%
\pgfusepath{stroke,fill}%
\end{pgfscope}%
\begin{pgfscope}%
\pgfpathrectangle{\pgfqpoint{0.641586in}{0.451389in}}{\pgfqpoint{4.279559in}{1.264399in}}%
\pgfusepath{clip}%
\pgfsetbuttcap%
\pgfsetmiterjoin%
\definecolor{currentfill}{rgb}{0.852941,0.544118,0.370588}%
\pgfsetfillcolor{currentfill}%
\pgfsetlinewidth{1.003750pt}%
\definecolor{currentstroke}{rgb}{1.000000,1.000000,1.000000}%
\pgfsetstrokecolor{currentstroke}%
\pgfsetdash{}{0pt}%
\pgfpathmoveto{\pgfqpoint{2.833239in}{-0.325888in}}%
\pgfpathlineto{\pgfqpoint{2.936986in}{-0.325888in}}%
\pgfpathlineto{\pgfqpoint{2.936986in}{1.050470in}}%
\pgfpathlineto{\pgfqpoint{2.833239in}{1.050470in}}%
\pgfpathlineto{\pgfqpoint{2.833239in}{-0.325888in}}%
\pgfpathclose%
\pgfusepath{stroke,fill}%
\end{pgfscope}%
\begin{pgfscope}%
\pgfpathrectangle{\pgfqpoint{0.641586in}{0.451389in}}{\pgfqpoint{4.279559in}{1.264399in}}%
\pgfusepath{clip}%
\pgfsetbuttcap%
\pgfsetmiterjoin%
\definecolor{currentfill}{rgb}{0.852941,0.544118,0.370588}%
\pgfsetfillcolor{currentfill}%
\pgfsetlinewidth{1.003750pt}%
\definecolor{currentstroke}{rgb}{1.000000,1.000000,1.000000}%
\pgfsetstrokecolor{currentstroke}%
\pgfsetdash{}{0pt}%
\pgfpathmoveto{\pgfqpoint{3.222290in}{-0.325888in}}%
\pgfpathlineto{\pgfqpoint{3.326037in}{-0.325888in}}%
\pgfpathlineto{\pgfqpoint{3.326037in}{1.088948in}}%
\pgfpathlineto{\pgfqpoint{3.222290in}{1.088948in}}%
\pgfpathlineto{\pgfqpoint{3.222290in}{-0.325888in}}%
\pgfpathclose%
\pgfusepath{stroke,fill}%
\end{pgfscope}%
\begin{pgfscope}%
\pgfpathrectangle{\pgfqpoint{0.641586in}{0.451389in}}{\pgfqpoint{4.279559in}{1.264399in}}%
\pgfusepath{clip}%
\pgfsetbuttcap%
\pgfsetmiterjoin%
\definecolor{currentfill}{rgb}{0.852941,0.544118,0.370588}%
\pgfsetfillcolor{currentfill}%
\pgfsetlinewidth{1.003750pt}%
\definecolor{currentstroke}{rgb}{1.000000,1.000000,1.000000}%
\pgfsetstrokecolor{currentstroke}%
\pgfsetdash{}{0pt}%
\pgfpathmoveto{\pgfqpoint{3.611341in}{-0.325888in}}%
\pgfpathlineto{\pgfqpoint{3.715087in}{-0.325888in}}%
\pgfpathlineto{\pgfqpoint{3.715087in}{1.142096in}}%
\pgfpathlineto{\pgfqpoint{3.611341in}{1.142096in}}%
\pgfpathlineto{\pgfqpoint{3.611341in}{-0.325888in}}%
\pgfpathclose%
\pgfusepath{stroke,fill}%
\end{pgfscope}%
\begin{pgfscope}%
\pgfpathrectangle{\pgfqpoint{0.641586in}{0.451389in}}{\pgfqpoint{4.279559in}{1.264399in}}%
\pgfusepath{clip}%
\pgfsetbuttcap%
\pgfsetmiterjoin%
\definecolor{currentfill}{rgb}{0.852941,0.544118,0.370588}%
\pgfsetfillcolor{currentfill}%
\pgfsetlinewidth{1.003750pt}%
\definecolor{currentstroke}{rgb}{1.000000,1.000000,1.000000}%
\pgfsetstrokecolor{currentstroke}%
\pgfsetdash{}{0pt}%
\pgfpathmoveto{\pgfqpoint{4.000391in}{-0.325888in}}%
\pgfpathlineto{\pgfqpoint{4.104138in}{-0.325888in}}%
\pgfpathlineto{\pgfqpoint{4.104138in}{1.009572in}}%
\pgfpathlineto{\pgfqpoint{4.000391in}{1.009572in}}%
\pgfpathlineto{\pgfqpoint{4.000391in}{-0.325888in}}%
\pgfpathclose%
\pgfusepath{stroke,fill}%
\end{pgfscope}%
\begin{pgfscope}%
\pgfpathrectangle{\pgfqpoint{0.641586in}{0.451389in}}{\pgfqpoint{4.279559in}{1.264399in}}%
\pgfusepath{clip}%
\pgfsetbuttcap%
\pgfsetmiterjoin%
\definecolor{currentfill}{rgb}{0.852941,0.544118,0.370588}%
\pgfsetfillcolor{currentfill}%
\pgfsetlinewidth{1.003750pt}%
\definecolor{currentstroke}{rgb}{1.000000,1.000000,1.000000}%
\pgfsetstrokecolor{currentstroke}%
\pgfsetdash{}{0pt}%
\pgfpathmoveto{\pgfqpoint{4.389442in}{-0.325888in}}%
\pgfpathlineto{\pgfqpoint{4.493189in}{-0.325888in}}%
\pgfpathlineto{\pgfqpoint{4.493189in}{1.128945in}}%
\pgfpathlineto{\pgfqpoint{4.389442in}{1.128945in}}%
\pgfpathlineto{\pgfqpoint{4.389442in}{-0.325888in}}%
\pgfpathclose%
\pgfusepath{stroke,fill}%
\end{pgfscope}%
\begin{pgfscope}%
\pgfpathrectangle{\pgfqpoint{0.641586in}{0.451389in}}{\pgfqpoint{4.279559in}{1.264399in}}%
\pgfusepath{clip}%
\pgfsetbuttcap%
\pgfsetmiterjoin%
\definecolor{currentfill}{rgb}{0.852941,0.544118,0.370588}%
\pgfsetfillcolor{currentfill}%
\pgfsetlinewidth{1.003750pt}%
\definecolor{currentstroke}{rgb}{1.000000,1.000000,1.000000}%
\pgfsetstrokecolor{currentstroke}%
\pgfsetdash{}{0pt}%
\pgfpathmoveto{\pgfqpoint{4.778493in}{-0.325888in}}%
\pgfpathlineto{\pgfqpoint{4.882240in}{-0.325888in}}%
\pgfpathlineto{\pgfqpoint{4.882240in}{1.106669in}}%
\pgfpathlineto{\pgfqpoint{4.778493in}{1.106669in}}%
\pgfpathlineto{\pgfqpoint{4.778493in}{-0.325888in}}%
\pgfpathclose%
\pgfusepath{stroke,fill}%
\end{pgfscope}%
\begin{pgfscope}%
\pgfsetrectcap%
\pgfsetmiterjoin%
\pgfsetlinewidth{1.254687pt}%
\definecolor{currentstroke}{rgb}{0.800000,0.800000,0.800000}%
\pgfsetstrokecolor{currentstroke}%
\pgfsetdash{}{0pt}%
\pgfpathmoveto{\pgfqpoint{0.641586in}{0.451389in}}%
\pgfpathlineto{\pgfqpoint{0.641586in}{1.715788in}}%
\pgfusepath{stroke}%
\end{pgfscope}%
\begin{pgfscope}%
\pgfsetrectcap%
\pgfsetmiterjoin%
\pgfsetlinewidth{1.254687pt}%
\definecolor{currentstroke}{rgb}{0.800000,0.800000,0.800000}%
\pgfsetstrokecolor{currentstroke}%
\pgfsetdash{}{0pt}%
\pgfpathmoveto{\pgfqpoint{4.921145in}{0.451389in}}%
\pgfpathlineto{\pgfqpoint{4.921145in}{1.715788in}}%
\pgfusepath{stroke}%
\end{pgfscope}%
\begin{pgfscope}%
\pgfsetrectcap%
\pgfsetmiterjoin%
\pgfsetlinewidth{1.254687pt}%
\definecolor{currentstroke}{rgb}{0.800000,0.800000,0.800000}%
\pgfsetstrokecolor{currentstroke}%
\pgfsetdash{}{0pt}%
\pgfpathmoveto{\pgfqpoint{0.641586in}{0.451389in}}%
\pgfpathlineto{\pgfqpoint{4.921145in}{0.451389in}}%
\pgfusepath{stroke}%
\end{pgfscope}%
\begin{pgfscope}%
\pgfsetrectcap%
\pgfsetmiterjoin%
\pgfsetlinewidth{1.254687pt}%
\definecolor{currentstroke}{rgb}{0.800000,0.800000,0.800000}%
\pgfsetstrokecolor{currentstroke}%
\pgfsetdash{}{0pt}%
\pgfpathmoveto{\pgfqpoint{0.641586in}{1.715788in}}%
\pgfpathlineto{\pgfqpoint{4.921145in}{1.715788in}}%
\pgfusepath{stroke}%
\end{pgfscope}%
\begin{pgfscope}%
\pgfsetbuttcap%
\pgfsetmiterjoin%
\definecolor{currentfill}{rgb}{1.000000,1.000000,1.000000}%
\pgfsetfillcolor{currentfill}%
\pgfsetfillopacity{0.800000}%
\pgfsetlinewidth{1.003750pt}%
\definecolor{currentstroke}{rgb}{0.800000,0.800000,0.800000}%
\pgfsetstrokecolor{currentstroke}%
\pgfsetstrokeopacity{0.800000}%
\pgfsetdash{}{0pt}%
\pgfpathmoveto{\pgfqpoint{5.115634in}{0.385671in}}%
\pgfpathlineto{\pgfqpoint{6.061410in}{0.385671in}}%
\pgfpathquadraticcurveto{\pgfqpoint{6.086410in}{0.385671in}}{\pgfqpoint{6.086410in}{0.410671in}}%
\pgfpathlineto{\pgfqpoint{6.086410in}{1.680641in}}%
\pgfpathquadraticcurveto{\pgfqpoint{6.086410in}{1.705641in}}{\pgfqpoint{6.061410in}{1.705641in}}%
\pgfpathlineto{\pgfqpoint{5.115634in}{1.705641in}}%
\pgfpathquadraticcurveto{\pgfqpoint{5.090634in}{1.705641in}}{\pgfqpoint{5.090634in}{1.680641in}}%
\pgfpathlineto{\pgfqpoint{5.090634in}{0.410671in}}%
\pgfpathquadraticcurveto{\pgfqpoint{5.090634in}{0.385671in}}{\pgfqpoint{5.115634in}{0.385671in}}%
\pgfpathlineto{\pgfqpoint{5.115634in}{0.385671in}}%
\pgfpathclose%
\pgfusepath{stroke,fill}%
\end{pgfscope}%
\begin{pgfscope}%
\pgfsetbuttcap%
\pgfsetmiterjoin%
\definecolor{currentfill}{rgb}{0.460784,0.749020,0.443137}%
\pgfsetfillcolor{currentfill}%
\pgfsetlinewidth{1.003750pt}%
\definecolor{currentstroke}{rgb}{1.000000,1.000000,1.000000}%
\pgfsetstrokecolor{currentstroke}%
\pgfsetdash{}{0pt}%
\pgfpathmoveto{\pgfqpoint{5.140634in}{1.473659in}}%
\pgfpathlineto{\pgfqpoint{5.390634in}{1.473659in}}%
\pgfpathlineto{\pgfqpoint{5.390634in}{1.561159in}}%
\pgfpathlineto{\pgfqpoint{5.140634in}{1.561159in}}%
\pgfpathlineto{\pgfqpoint{5.140634in}{1.473659in}}%
\pgfpathclose%
\pgfusepath{stroke,fill}%
\end{pgfscope}%
\begin{pgfscope}%
\definecolor{textcolor}{rgb}{0.150000,0.150000,0.150000}%
\pgfsetstrokecolor{textcolor}%
\pgfsetfillcolor{textcolor}%
\pgftext[x=5.490634in, y=1.560670in, left, base]{\color{textcolor}\sffamily\fontsize{9.000000}{10.800000}\selectfont Expected}%
\end{pgfscope}%
\begin{pgfscope}%
\definecolor{textcolor}{rgb}{0.150000,0.150000,0.150000}%
\pgfsetstrokecolor{textcolor}%
\pgfsetfillcolor{textcolor}%
\pgftext[x=5.490634in, y=1.416676in, left, base]{\color{textcolor}\sffamily\fontsize{9.000000}{10.800000}\selectfont route}%
\end{pgfscope}%
\begin{pgfscope}%
\pgfsetbuttcap%
\pgfsetmiterjoin%
\definecolor{currentfill}{rgb}{0.349020,0.490196,0.749020}%
\pgfsetfillcolor{currentfill}%
\pgfsetlinewidth{1.003750pt}%
\definecolor{currentstroke}{rgb}{1.000000,1.000000,1.000000}%
\pgfsetstrokecolor{currentstroke}%
\pgfsetdash{}{0pt}%
\pgfpathmoveto{\pgfqpoint{5.140634in}{1.070168in}}%
\pgfpathlineto{\pgfqpoint{5.390634in}{1.070168in}}%
\pgfpathlineto{\pgfqpoint{5.390634in}{1.157668in}}%
\pgfpathlineto{\pgfqpoint{5.140634in}{1.157668in}}%
\pgfpathlineto{\pgfqpoint{5.140634in}{1.070168in}}%
\pgfpathclose%
\pgfusepath{stroke,fill}%
\end{pgfscope}%
\begin{pgfscope}%
\definecolor{textcolor}{rgb}{0.150000,0.150000,0.150000}%
\pgfsetstrokecolor{textcolor}%
\pgfsetfillcolor{textcolor}%
\pgftext[x=5.490634in, y=1.229176in, left, base]{\color{textcolor}\sffamily\fontsize{9.000000}{10.800000}\selectfont Hybrid}%
\end{pgfscope}%
\begin{pgfscope}%
\definecolor{textcolor}{rgb}{0.150000,0.150000,0.150000}%
\pgfsetstrokecolor{textcolor}%
\pgfsetfillcolor{textcolor}%
\pgftext[x=5.490634in, y=1.085182in, left, base]{\color{textcolor}\sffamily\fontsize{9.000000}{10.800000}\selectfont routing}%
\end{pgfscope}%
\begin{pgfscope}%
\definecolor{textcolor}{rgb}{0.150000,0.150000,0.150000}%
\pgfsetstrokecolor{textcolor}%
\pgfsetfillcolor{textcolor}%
\pgftext[x=5.490634in, y=0.941188in, left, base]{\color{textcolor}\sffamily\fontsize{9.000000}{10.800000}\selectfont algorithm}%
\end{pgfscope}%
\begin{pgfscope}%
\pgfsetbuttcap%
\pgfsetmiterjoin%
\definecolor{currentfill}{rgb}{0.852941,0.544118,0.370588}%
\pgfsetfillcolor{currentfill}%
\pgfsetlinewidth{1.003750pt}%
\definecolor{currentstroke}{rgb}{1.000000,1.000000,1.000000}%
\pgfsetstrokecolor{currentstroke}%
\pgfsetdash{}{0pt}%
\pgfpathmoveto{\pgfqpoint{5.140634in}{0.594680in}}%
\pgfpathlineto{\pgfqpoint{5.390634in}{0.594680in}}%
\pgfpathlineto{\pgfqpoint{5.390634in}{0.682180in}}%
\pgfpathlineto{\pgfqpoint{5.140634in}{0.682180in}}%
\pgfpathlineto{\pgfqpoint{5.140634in}{0.594680in}}%
\pgfpathclose%
\pgfusepath{stroke,fill}%
\end{pgfscope}%
\begin{pgfscope}%
\definecolor{textcolor}{rgb}{0.150000,0.150000,0.150000}%
\pgfsetstrokecolor{textcolor}%
\pgfsetfillcolor{textcolor}%
\pgftext[x=5.490634in, y=0.753689in, left, base]{\color{textcolor}\sffamily\fontsize{9.000000}{10.800000}\selectfont Graph-}%
\end{pgfscope}%
\begin{pgfscope}%
\definecolor{textcolor}{rgb}{0.150000,0.150000,0.150000}%
\pgfsetstrokecolor{textcolor}%
\pgfsetfillcolor{textcolor}%
\pgftext[x=5.490634in, y=0.609695in, left, base]{\color{textcolor}\sffamily\fontsize{9.000000}{10.800000}\selectfont based}%
\end{pgfscope}%
\begin{pgfscope}%
\definecolor{textcolor}{rgb}{0.150000,0.150000,0.150000}%
\pgfsetstrokecolor{textcolor}%
\pgfsetfillcolor{textcolor}%
\pgftext[x=5.490634in, y=0.465701in, left, base]{\color{textcolor}\sffamily\fontsize{9.000000}{10.800000}\selectfont routing}%
\end{pgfscope}%
\end{pgfpicture}%
\makeatother%
\endgroup%

					\end{figcenter}
					\caption{\enquote{OSM city} dataset.}
					\label{fig:eval-route-distances-city}
				\end{subfigure}
				\\[3ex]
				\begin{subfigure}[t]{\linewidth}
					\begin{figcenter}
						%% Creator: Matplotlib, PGF backend
%%
%% To include the figure in your LaTeX document, write
%%   \input{<filename>.pgf}
%%
%% Make sure the required packages are loaded in your preamble
%%   \usepackage{pgf}
%%
%% Also ensure that all the required font packages are loaded; for instance,
%% the lmodern package is sometimes necessary when using math font.
%%   \usepackage{lmodern}
%%
%% Figures using additional raster images can only be included by \input if
%% they are in the same directory as the main LaTeX file. For loading figures
%% from other directories you can use the `import` package
%%   \usepackage{import}
%%
%% and then include the figures with
%%   \import{<path to file>}{<filename>.pgf}
%%
%% Matplotlib used the following preamble
%%   
%%   \usepackage{fontspec}
%%   \setmainfont{DejaVuSerif.ttf}[Path=\detokenize{/home/hauke/.local/lib/python3.11/site-packages/matplotlib/mpl-data/fonts/ttf/}]
%%   \setsansfont{DroidSans.ttf}[Path=\detokenize{/usr/share/fonts/droid/}]
%%   \setmonofont{DejaVuSansMono.ttf}[Path=\detokenize{/home/hauke/.local/lib/python3.11/site-packages/matplotlib/mpl-data/fonts/ttf/}]
%%   \makeatletter\@ifpackageloaded{underscore}{}{\usepackage[strings]{underscore}}\makeatother
%%
\begingroup%
\makeatletter%
\begin{pgfpicture}%
\pgfpathrectangle{\pgfpointorigin}{\pgfqpoint{6.086074in}{1.715788in}}%
\pgfusepath{use as bounding box, clip}%
\begin{pgfscope}%
\pgfsetbuttcap%
\pgfsetmiterjoin%
\definecolor{currentfill}{rgb}{1.000000,1.000000,1.000000}%
\pgfsetfillcolor{currentfill}%
\pgfsetlinewidth{0.000000pt}%
\definecolor{currentstroke}{rgb}{1.000000,1.000000,1.000000}%
\pgfsetstrokecolor{currentstroke}%
\pgfsetdash{}{0pt}%
\pgfpathmoveto{\pgfqpoint{0.000000in}{0.000000in}}%
\pgfpathlineto{\pgfqpoint{6.086074in}{0.000000in}}%
\pgfpathlineto{\pgfqpoint{6.086074in}{1.715788in}}%
\pgfpathlineto{\pgfqpoint{0.000000in}{1.715788in}}%
\pgfpathlineto{\pgfqpoint{0.000000in}{0.000000in}}%
\pgfpathclose%
\pgfusepath{fill}%
\end{pgfscope}%
\begin{pgfscope}%
\pgfsetbuttcap%
\pgfsetmiterjoin%
\definecolor{currentfill}{rgb}{1.000000,1.000000,1.000000}%
\pgfsetfillcolor{currentfill}%
\pgfsetlinewidth{0.000000pt}%
\definecolor{currentstroke}{rgb}{0.000000,0.000000,0.000000}%
\pgfsetstrokecolor{currentstroke}%
\pgfsetstrokeopacity{0.000000}%
\pgfsetdash{}{0pt}%
\pgfpathmoveto{\pgfqpoint{0.539230in}{0.451389in}}%
\pgfpathlineto{\pgfqpoint{4.918320in}{0.451389in}}%
\pgfpathlineto{\pgfqpoint{4.918320in}{1.715788in}}%
\pgfpathlineto{\pgfqpoint{0.539230in}{1.715788in}}%
\pgfpathlineto{\pgfqpoint{0.539230in}{0.451389in}}%
\pgfpathclose%
\pgfusepath{fill}%
\end{pgfscope}%
\begin{pgfscope}%
\definecolor{textcolor}{rgb}{0.150000,0.150000,0.150000}%
\pgfsetstrokecolor{textcolor}%
\pgfsetfillcolor{textcolor}%
\pgftext[x=0.738280in,y=0.319444in,,top]{\color{textcolor}\sffamily\fontsize{9.000000}{10.800000}\selectfont 1}%
\end{pgfscope}%
\begin{pgfscope}%
\definecolor{textcolor}{rgb}{0.150000,0.150000,0.150000}%
\pgfsetstrokecolor{textcolor}%
\pgfsetfillcolor{textcolor}%
\pgftext[x=1.136379in,y=0.319444in,,top]{\color{textcolor}\sffamily\fontsize{9.000000}{10.800000}\selectfont 2}%
\end{pgfscope}%
\begin{pgfscope}%
\definecolor{textcolor}{rgb}{0.150000,0.150000,0.150000}%
\pgfsetstrokecolor{textcolor}%
\pgfsetfillcolor{textcolor}%
\pgftext[x=1.534478in,y=0.319444in,,top]{\color{textcolor}\sffamily\fontsize{9.000000}{10.800000}\selectfont 3}%
\end{pgfscope}%
\begin{pgfscope}%
\definecolor{textcolor}{rgb}{0.150000,0.150000,0.150000}%
\pgfsetstrokecolor{textcolor}%
\pgfsetfillcolor{textcolor}%
\pgftext[x=1.932577in,y=0.319444in,,top]{\color{textcolor}\sffamily\fontsize{9.000000}{10.800000}\selectfont 4}%
\end{pgfscope}%
\begin{pgfscope}%
\definecolor{textcolor}{rgb}{0.150000,0.150000,0.150000}%
\pgfsetstrokecolor{textcolor}%
\pgfsetfillcolor{textcolor}%
\pgftext[x=2.330676in,y=0.319444in,,top]{\color{textcolor}\sffamily\fontsize{9.000000}{10.800000}\selectfont 5}%
\end{pgfscope}%
\begin{pgfscope}%
\definecolor{textcolor}{rgb}{0.150000,0.150000,0.150000}%
\pgfsetstrokecolor{textcolor}%
\pgfsetfillcolor{textcolor}%
\pgftext[x=2.728775in,y=0.319444in,,top]{\color{textcolor}\sffamily\fontsize{9.000000}{10.800000}\selectfont 6}%
\end{pgfscope}%
\begin{pgfscope}%
\definecolor{textcolor}{rgb}{0.150000,0.150000,0.150000}%
\pgfsetstrokecolor{textcolor}%
\pgfsetfillcolor{textcolor}%
\pgftext[x=3.126874in,y=0.319444in,,top]{\color{textcolor}\sffamily\fontsize{9.000000}{10.800000}\selectfont 7}%
\end{pgfscope}%
\begin{pgfscope}%
\definecolor{textcolor}{rgb}{0.150000,0.150000,0.150000}%
\pgfsetstrokecolor{textcolor}%
\pgfsetfillcolor{textcolor}%
\pgftext[x=3.524973in,y=0.319444in,,top]{\color{textcolor}\sffamily\fontsize{9.000000}{10.800000}\selectfont 8}%
\end{pgfscope}%
\begin{pgfscope}%
\definecolor{textcolor}{rgb}{0.150000,0.150000,0.150000}%
\pgfsetstrokecolor{textcolor}%
\pgfsetfillcolor{textcolor}%
\pgftext[x=3.923072in,y=0.319444in,,top]{\color{textcolor}\sffamily\fontsize{9.000000}{10.800000}\selectfont 9}%
\end{pgfscope}%
\begin{pgfscope}%
\definecolor{textcolor}{rgb}{0.150000,0.150000,0.150000}%
\pgfsetstrokecolor{textcolor}%
\pgfsetfillcolor{textcolor}%
\pgftext[x=4.321172in,y=0.319444in,,top]{\color{textcolor}\sffamily\fontsize{9.000000}{10.800000}\selectfont 10}%
\end{pgfscope}%
\begin{pgfscope}%
\definecolor{textcolor}{rgb}{0.150000,0.150000,0.150000}%
\pgfsetstrokecolor{textcolor}%
\pgfsetfillcolor{textcolor}%
\pgftext[x=4.719271in,y=0.319444in,,top]{\color{textcolor}\sffamily\fontsize{9.000000}{10.800000}\selectfont mean}%
\end{pgfscope}%
\begin{pgfscope}%
\definecolor{textcolor}{rgb}{0.150000,0.150000,0.150000}%
\pgfsetstrokecolor{textcolor}%
\pgfsetfillcolor{textcolor}%
\pgftext[x=2.728775in,y=0.125000in,,top]{\color{textcolor}\sffamily\fontsize{9.000000}{10.800000}\selectfont Routing request}%
\end{pgfscope}%
\begin{pgfscope}%
\pgfpathrectangle{\pgfqpoint{0.539230in}{0.451389in}}{\pgfqpoint{4.379090in}{1.264399in}}%
\pgfusepath{clip}%
\pgfsetroundcap%
\pgfsetroundjoin%
\pgfsetlinewidth{1.003750pt}%
\definecolor{currentstroke}{rgb}{0.800000,0.800000,0.800000}%
\pgfsetstrokecolor{currentstroke}%
\pgfsetdash{}{0pt}%
\pgfpathmoveto{\pgfqpoint{0.539230in}{0.479840in}}%
\pgfpathlineto{\pgfqpoint{4.918320in}{0.479840in}}%
\pgfusepath{stroke}%
\end{pgfscope}%
\begin{pgfscope}%
\definecolor{textcolor}{rgb}{0.150000,0.150000,0.150000}%
\pgfsetstrokecolor{textcolor}%
\pgfsetfillcolor{textcolor}%
\pgftext[x=0.338438in, y=0.432354in, left, base]{\color{textcolor}\sffamily\fontsize{9.000000}{10.800000}\selectfont 1}%
\end{pgfscope}%
\begin{pgfscope}%
\pgfpathrectangle{\pgfqpoint{0.539230in}{0.451389in}}{\pgfqpoint{4.379090in}{1.264399in}}%
\pgfusepath{clip}%
\pgfsetroundcap%
\pgfsetroundjoin%
\pgfsetlinewidth{1.003750pt}%
\definecolor{currentstroke}{rgb}{0.800000,0.800000,0.800000}%
\pgfsetstrokecolor{currentstroke}%
\pgfsetdash{}{0pt}%
\pgfpathmoveto{\pgfqpoint{0.539230in}{0.764351in}}%
\pgfpathlineto{\pgfqpoint{4.918320in}{0.764351in}}%
\pgfusepath{stroke}%
\end{pgfscope}%
\begin{pgfscope}%
\definecolor{textcolor}{rgb}{0.150000,0.150000,0.150000}%
\pgfsetstrokecolor{textcolor}%
\pgfsetfillcolor{textcolor}%
\pgftext[x=0.338438in, y=0.716866in, left, base]{\color{textcolor}\sffamily\fontsize{9.000000}{10.800000}\selectfont 2}%
\end{pgfscope}%
\begin{pgfscope}%
\pgfpathrectangle{\pgfqpoint{0.539230in}{0.451389in}}{\pgfqpoint{4.379090in}{1.264399in}}%
\pgfusepath{clip}%
\pgfsetroundcap%
\pgfsetroundjoin%
\pgfsetlinewidth{1.003750pt}%
\definecolor{currentstroke}{rgb}{0.800000,0.800000,0.800000}%
\pgfsetstrokecolor{currentstroke}%
\pgfsetdash{}{0pt}%
\pgfpathmoveto{\pgfqpoint{0.539230in}{1.048863in}}%
\pgfpathlineto{\pgfqpoint{4.918320in}{1.048863in}}%
\pgfusepath{stroke}%
\end{pgfscope}%
\begin{pgfscope}%
\definecolor{textcolor}{rgb}{0.150000,0.150000,0.150000}%
\pgfsetstrokecolor{textcolor}%
\pgfsetfillcolor{textcolor}%
\pgftext[x=0.338438in, y=1.001377in, left, base]{\color{textcolor}\sffamily\fontsize{9.000000}{10.800000}\selectfont 3}%
\end{pgfscope}%
\begin{pgfscope}%
\pgfpathrectangle{\pgfqpoint{0.539230in}{0.451389in}}{\pgfqpoint{4.379090in}{1.264399in}}%
\pgfusepath{clip}%
\pgfsetroundcap%
\pgfsetroundjoin%
\pgfsetlinewidth{1.003750pt}%
\definecolor{currentstroke}{rgb}{0.800000,0.800000,0.800000}%
\pgfsetstrokecolor{currentstroke}%
\pgfsetdash{}{0pt}%
\pgfpathmoveto{\pgfqpoint{0.539230in}{1.333374in}}%
\pgfpathlineto{\pgfqpoint{4.918320in}{1.333374in}}%
\pgfusepath{stroke}%
\end{pgfscope}%
\begin{pgfscope}%
\definecolor{textcolor}{rgb}{0.150000,0.150000,0.150000}%
\pgfsetstrokecolor{textcolor}%
\pgfsetfillcolor{textcolor}%
\pgftext[x=0.338438in, y=1.285889in, left, base]{\color{textcolor}\sffamily\fontsize{9.000000}{10.800000}\selectfont 4}%
\end{pgfscope}%
\begin{pgfscope}%
\pgfpathrectangle{\pgfqpoint{0.539230in}{0.451389in}}{\pgfqpoint{4.379090in}{1.264399in}}%
\pgfusepath{clip}%
\pgfsetroundcap%
\pgfsetroundjoin%
\pgfsetlinewidth{1.003750pt}%
\definecolor{currentstroke}{rgb}{0.800000,0.800000,0.800000}%
\pgfsetstrokecolor{currentstroke}%
\pgfsetdash{}{0pt}%
\pgfpathmoveto{\pgfqpoint{0.539230in}{1.617886in}}%
\pgfpathlineto{\pgfqpoint{4.918320in}{1.617886in}}%
\pgfusepath{stroke}%
\end{pgfscope}%
\begin{pgfscope}%
\definecolor{textcolor}{rgb}{0.150000,0.150000,0.150000}%
\pgfsetstrokecolor{textcolor}%
\pgfsetfillcolor{textcolor}%
\pgftext[x=0.338438in, y=1.570400in, left, base]{\color{textcolor}\sffamily\fontsize{9.000000}{10.800000}\selectfont 5}%
\end{pgfscope}%
\begin{pgfscope}%
\definecolor{textcolor}{rgb}{0.150000,0.150000,0.150000}%
\pgfsetstrokecolor{textcolor}%
\pgfsetfillcolor{textcolor}%
\pgftext[x=0.094971in, y=0.628907in, left, base,rotate=90.000000]{\color{textcolor}\sffamily\fontsize{9.000000}{10.800000}\selectfont Route distance /}%
\end{pgfscope}%
\begin{pgfscope}%
\definecolor{textcolor}{rgb}{0.150000,0.150000,0.150000}%
\pgfsetstrokecolor{textcolor}%
\pgfsetfillcolor{textcolor}%
\pgftext[x=0.238965in, y=0.626709in, left, base,rotate=90.000000]{\color{textcolor}\sffamily\fontsize{9.000000}{10.800000}\selectfont beeline distance}%
\end{pgfscope}%
\begin{pgfscope}%
\pgfpathrectangle{\pgfqpoint{0.539230in}{0.451389in}}{\pgfqpoint{4.379090in}{1.264399in}}%
\pgfusepath{clip}%
\pgfsetbuttcap%
\pgfsetmiterjoin%
\definecolor{currentfill}{rgb}{0.460784,0.749020,0.443137}%
\pgfsetfillcolor{currentfill}%
\pgfsetlinewidth{1.003750pt}%
\definecolor{currentstroke}{rgb}{1.000000,1.000000,1.000000}%
\pgfsetstrokecolor{currentstroke}%
\pgfsetdash{}{0pt}%
\pgfpathmoveto{\pgfqpoint{0.579040in}{0.195328in}}%
\pgfpathlineto{\pgfqpoint{0.685200in}{0.195328in}}%
\pgfpathlineto{\pgfqpoint{0.685200in}{1.020412in}}%
\pgfpathlineto{\pgfqpoint{0.579040in}{1.020412in}}%
\pgfpathlineto{\pgfqpoint{0.579040in}{0.195328in}}%
\pgfpathclose%
\pgfusepath{stroke,fill}%
\end{pgfscope}%
\begin{pgfscope}%
\pgfpathrectangle{\pgfqpoint{0.539230in}{0.451389in}}{\pgfqpoint{4.379090in}{1.264399in}}%
\pgfusepath{clip}%
\pgfsetbuttcap%
\pgfsetmiterjoin%
\definecolor{currentfill}{rgb}{0.460784,0.749020,0.443137}%
\pgfsetfillcolor{currentfill}%
\pgfsetlinewidth{1.003750pt}%
\definecolor{currentstroke}{rgb}{1.000000,1.000000,1.000000}%
\pgfsetstrokecolor{currentstroke}%
\pgfsetdash{}{0pt}%
\pgfpathmoveto{\pgfqpoint{0.977139in}{0.195328in}}%
\pgfpathlineto{\pgfqpoint{1.083299in}{0.195328in}}%
\pgfpathlineto{\pgfqpoint{1.083299in}{0.602445in}}%
\pgfpathlineto{\pgfqpoint{0.977139in}{0.602445in}}%
\pgfpathlineto{\pgfqpoint{0.977139in}{0.195328in}}%
\pgfpathclose%
\pgfusepath{stroke,fill}%
\end{pgfscope}%
\begin{pgfscope}%
\pgfpathrectangle{\pgfqpoint{0.539230in}{0.451389in}}{\pgfqpoint{4.379090in}{1.264399in}}%
\pgfusepath{clip}%
\pgfsetbuttcap%
\pgfsetmiterjoin%
\definecolor{currentfill}{rgb}{0.460784,0.749020,0.443137}%
\pgfsetfillcolor{currentfill}%
\pgfsetlinewidth{1.003750pt}%
\definecolor{currentstroke}{rgb}{1.000000,1.000000,1.000000}%
\pgfsetstrokecolor{currentstroke}%
\pgfsetdash{}{0pt}%
\pgfpathmoveto{\pgfqpoint{1.375238in}{0.195328in}}%
\pgfpathlineto{\pgfqpoint{1.481398in}{0.195328in}}%
\pgfpathlineto{\pgfqpoint{1.481398in}{0.544973in}}%
\pgfpathlineto{\pgfqpoint{1.375238in}{0.544973in}}%
\pgfpathlineto{\pgfqpoint{1.375238in}{0.195328in}}%
\pgfpathclose%
\pgfusepath{stroke,fill}%
\end{pgfscope}%
\begin{pgfscope}%
\pgfpathrectangle{\pgfqpoint{0.539230in}{0.451389in}}{\pgfqpoint{4.379090in}{1.264399in}}%
\pgfusepath{clip}%
\pgfsetbuttcap%
\pgfsetmiterjoin%
\definecolor{currentfill}{rgb}{0.460784,0.749020,0.443137}%
\pgfsetfillcolor{currentfill}%
\pgfsetlinewidth{1.003750pt}%
\definecolor{currentstroke}{rgb}{1.000000,1.000000,1.000000}%
\pgfsetstrokecolor{currentstroke}%
\pgfsetdash{}{0pt}%
\pgfpathmoveto{\pgfqpoint{1.773338in}{0.195328in}}%
\pgfpathlineto{\pgfqpoint{1.879497in}{0.195328in}}%
\pgfpathlineto{\pgfqpoint{1.879497in}{0.922312in}}%
\pgfpathlineto{\pgfqpoint{1.773338in}{0.922312in}}%
\pgfpathlineto{\pgfqpoint{1.773338in}{0.195328in}}%
\pgfpathclose%
\pgfusepath{stroke,fill}%
\end{pgfscope}%
\begin{pgfscope}%
\pgfpathrectangle{\pgfqpoint{0.539230in}{0.451389in}}{\pgfqpoint{4.379090in}{1.264399in}}%
\pgfusepath{clip}%
\pgfsetbuttcap%
\pgfsetmiterjoin%
\definecolor{currentfill}{rgb}{0.460784,0.749020,0.443137}%
\pgfsetfillcolor{currentfill}%
\pgfsetlinewidth{1.003750pt}%
\definecolor{currentstroke}{rgb}{1.000000,1.000000,1.000000}%
\pgfsetstrokecolor{currentstroke}%
\pgfsetdash{}{0pt}%
\pgfpathmoveto{\pgfqpoint{2.171437in}{0.195328in}}%
\pgfpathlineto{\pgfqpoint{2.277596in}{0.195328in}}%
\pgfpathlineto{\pgfqpoint{2.277596in}{0.605322in}}%
\pgfpathlineto{\pgfqpoint{2.171437in}{0.605322in}}%
\pgfpathlineto{\pgfqpoint{2.171437in}{0.195328in}}%
\pgfpathclose%
\pgfusepath{stroke,fill}%
\end{pgfscope}%
\begin{pgfscope}%
\pgfpathrectangle{\pgfqpoint{0.539230in}{0.451389in}}{\pgfqpoint{4.379090in}{1.264399in}}%
\pgfusepath{clip}%
\pgfsetbuttcap%
\pgfsetmiterjoin%
\definecolor{currentfill}{rgb}{0.460784,0.749020,0.443137}%
\pgfsetfillcolor{currentfill}%
\pgfsetlinewidth{1.003750pt}%
\definecolor{currentstroke}{rgb}{1.000000,1.000000,1.000000}%
\pgfsetstrokecolor{currentstroke}%
\pgfsetdash{}{0pt}%
\pgfpathmoveto{\pgfqpoint{2.569536in}{0.195328in}}%
\pgfpathlineto{\pgfqpoint{2.675695in}{0.195328in}}%
\pgfpathlineto{\pgfqpoint{2.675695in}{0.744342in}}%
\pgfpathlineto{\pgfqpoint{2.569536in}{0.744342in}}%
\pgfpathlineto{\pgfqpoint{2.569536in}{0.195328in}}%
\pgfpathclose%
\pgfusepath{stroke,fill}%
\end{pgfscope}%
\begin{pgfscope}%
\pgfpathrectangle{\pgfqpoint{0.539230in}{0.451389in}}{\pgfqpoint{4.379090in}{1.264399in}}%
\pgfusepath{clip}%
\pgfsetbuttcap%
\pgfsetmiterjoin%
\definecolor{currentfill}{rgb}{0.460784,0.749020,0.443137}%
\pgfsetfillcolor{currentfill}%
\pgfsetlinewidth{1.003750pt}%
\definecolor{currentstroke}{rgb}{1.000000,1.000000,1.000000}%
\pgfsetstrokecolor{currentstroke}%
\pgfsetdash{}{0pt}%
\pgfpathmoveto{\pgfqpoint{2.967635in}{0.195328in}}%
\pgfpathlineto{\pgfqpoint{3.073794in}{0.195328in}}%
\pgfpathlineto{\pgfqpoint{3.073794in}{0.516989in}}%
\pgfpathlineto{\pgfqpoint{2.967635in}{0.516989in}}%
\pgfpathlineto{\pgfqpoint{2.967635in}{0.195328in}}%
\pgfpathclose%
\pgfusepath{stroke,fill}%
\end{pgfscope}%
\begin{pgfscope}%
\pgfpathrectangle{\pgfqpoint{0.539230in}{0.451389in}}{\pgfqpoint{4.379090in}{1.264399in}}%
\pgfusepath{clip}%
\pgfsetbuttcap%
\pgfsetmiterjoin%
\definecolor{currentfill}{rgb}{0.460784,0.749020,0.443137}%
\pgfsetfillcolor{currentfill}%
\pgfsetlinewidth{1.003750pt}%
\definecolor{currentstroke}{rgb}{1.000000,1.000000,1.000000}%
\pgfsetstrokecolor{currentstroke}%
\pgfsetdash{}{0pt}%
\pgfpathmoveto{\pgfqpoint{3.365734in}{0.195328in}}%
\pgfpathlineto{\pgfqpoint{3.471894in}{0.195328in}}%
\pgfpathlineto{\pgfqpoint{3.471894in}{0.549805in}}%
\pgfpathlineto{\pgfqpoint{3.365734in}{0.549805in}}%
\pgfpathlineto{\pgfqpoint{3.365734in}{0.195328in}}%
\pgfpathclose%
\pgfusepath{stroke,fill}%
\end{pgfscope}%
\begin{pgfscope}%
\pgfpathrectangle{\pgfqpoint{0.539230in}{0.451389in}}{\pgfqpoint{4.379090in}{1.264399in}}%
\pgfusepath{clip}%
\pgfsetbuttcap%
\pgfsetmiterjoin%
\definecolor{currentfill}{rgb}{0.460784,0.749020,0.443137}%
\pgfsetfillcolor{currentfill}%
\pgfsetlinewidth{1.003750pt}%
\definecolor{currentstroke}{rgb}{1.000000,1.000000,1.000000}%
\pgfsetstrokecolor{currentstroke}%
\pgfsetdash{}{0pt}%
\pgfpathmoveto{\pgfqpoint{3.763833in}{0.195328in}}%
\pgfpathlineto{\pgfqpoint{3.869993in}{0.195328in}}%
\pgfpathlineto{\pgfqpoint{3.869993in}{0.589539in}}%
\pgfpathlineto{\pgfqpoint{3.763833in}{0.589539in}}%
\pgfpathlineto{\pgfqpoint{3.763833in}{0.195328in}}%
\pgfpathclose%
\pgfusepath{stroke,fill}%
\end{pgfscope}%
\begin{pgfscope}%
\pgfpathrectangle{\pgfqpoint{0.539230in}{0.451389in}}{\pgfqpoint{4.379090in}{1.264399in}}%
\pgfusepath{clip}%
\pgfsetbuttcap%
\pgfsetmiterjoin%
\definecolor{currentfill}{rgb}{0.460784,0.749020,0.443137}%
\pgfsetfillcolor{currentfill}%
\pgfsetlinewidth{1.003750pt}%
\definecolor{currentstroke}{rgb}{1.000000,1.000000,1.000000}%
\pgfsetstrokecolor{currentstroke}%
\pgfsetdash{}{0pt}%
\pgfpathmoveto{\pgfqpoint{4.161932in}{0.195328in}}%
\pgfpathlineto{\pgfqpoint{4.268092in}{0.195328in}}%
\pgfpathlineto{\pgfqpoint{4.268092in}{0.578804in}}%
\pgfpathlineto{\pgfqpoint{4.161932in}{0.578804in}}%
\pgfpathlineto{\pgfqpoint{4.161932in}{0.195328in}}%
\pgfpathclose%
\pgfusepath{stroke,fill}%
\end{pgfscope}%
\begin{pgfscope}%
\pgfpathrectangle{\pgfqpoint{0.539230in}{0.451389in}}{\pgfqpoint{4.379090in}{1.264399in}}%
\pgfusepath{clip}%
\pgfsetbuttcap%
\pgfsetmiterjoin%
\definecolor{currentfill}{rgb}{0.460784,0.749020,0.443137}%
\pgfsetfillcolor{currentfill}%
\pgfsetlinewidth{1.003750pt}%
\definecolor{currentstroke}{rgb}{1.000000,1.000000,1.000000}%
\pgfsetstrokecolor{currentstroke}%
\pgfsetdash{}{0pt}%
\pgfpathmoveto{\pgfqpoint{4.560031in}{0.195328in}}%
\pgfpathlineto{\pgfqpoint{4.666191in}{0.195328in}}%
\pgfpathlineto{\pgfqpoint{4.666191in}{0.667494in}}%
\pgfpathlineto{\pgfqpoint{4.560031in}{0.667494in}}%
\pgfpathlineto{\pgfqpoint{4.560031in}{0.195328in}}%
\pgfpathclose%
\pgfusepath{stroke,fill}%
\end{pgfscope}%
\begin{pgfscope}%
\pgfpathrectangle{\pgfqpoint{0.539230in}{0.451389in}}{\pgfqpoint{4.379090in}{1.264399in}}%
\pgfusepath{clip}%
\pgfsetbuttcap%
\pgfsetmiterjoin%
\definecolor{currentfill}{rgb}{0.349020,0.490196,0.749020}%
\pgfsetfillcolor{currentfill}%
\pgfsetlinewidth{1.003750pt}%
\definecolor{currentstroke}{rgb}{1.000000,1.000000,1.000000}%
\pgfsetstrokecolor{currentstroke}%
\pgfsetdash{}{0pt}%
\pgfpathmoveto{\pgfqpoint{0.685200in}{0.195328in}}%
\pgfpathlineto{\pgfqpoint{0.791360in}{0.195328in}}%
\pgfpathlineto{\pgfqpoint{0.791360in}{0.479840in}}%
\pgfpathlineto{\pgfqpoint{0.685200in}{0.479840in}}%
\pgfpathlineto{\pgfqpoint{0.685200in}{0.195328in}}%
\pgfpathclose%
\pgfusepath{stroke,fill}%
\end{pgfscope}%
\begin{pgfscope}%
\pgfpathrectangle{\pgfqpoint{0.539230in}{0.451389in}}{\pgfqpoint{4.379090in}{1.264399in}}%
\pgfusepath{clip}%
\pgfsetbuttcap%
\pgfsetmiterjoin%
\definecolor{currentfill}{rgb}{0.349020,0.490196,0.749020}%
\pgfsetfillcolor{currentfill}%
\pgfsetlinewidth{1.003750pt}%
\definecolor{currentstroke}{rgb}{1.000000,1.000000,1.000000}%
\pgfsetstrokecolor{currentstroke}%
\pgfsetdash{}{0pt}%
\pgfpathmoveto{\pgfqpoint{1.083299in}{0.195328in}}%
\pgfpathlineto{\pgfqpoint{1.189459in}{0.195328in}}%
\pgfpathlineto{\pgfqpoint{1.189459in}{0.487349in}}%
\pgfpathlineto{\pgfqpoint{1.083299in}{0.487349in}}%
\pgfpathlineto{\pgfqpoint{1.083299in}{0.195328in}}%
\pgfpathclose%
\pgfusepath{stroke,fill}%
\end{pgfscope}%
\begin{pgfscope}%
\pgfpathrectangle{\pgfqpoint{0.539230in}{0.451389in}}{\pgfqpoint{4.379090in}{1.264399in}}%
\pgfusepath{clip}%
\pgfsetbuttcap%
\pgfsetmiterjoin%
\definecolor{currentfill}{rgb}{0.349020,0.490196,0.749020}%
\pgfsetfillcolor{currentfill}%
\pgfsetlinewidth{1.003750pt}%
\definecolor{currentstroke}{rgb}{1.000000,1.000000,1.000000}%
\pgfsetstrokecolor{currentstroke}%
\pgfsetdash{}{0pt}%
\pgfpathmoveto{\pgfqpoint{1.481398in}{0.195328in}}%
\pgfpathlineto{\pgfqpoint{1.587558in}{0.195328in}}%
\pgfpathlineto{\pgfqpoint{1.587558in}{0.495505in}}%
\pgfpathlineto{\pgfqpoint{1.481398in}{0.495505in}}%
\pgfpathlineto{\pgfqpoint{1.481398in}{0.195328in}}%
\pgfpathclose%
\pgfusepath{stroke,fill}%
\end{pgfscope}%
\begin{pgfscope}%
\pgfpathrectangle{\pgfqpoint{0.539230in}{0.451389in}}{\pgfqpoint{4.379090in}{1.264399in}}%
\pgfusepath{clip}%
\pgfsetbuttcap%
\pgfsetmiterjoin%
\definecolor{currentfill}{rgb}{0.349020,0.490196,0.749020}%
\pgfsetfillcolor{currentfill}%
\pgfsetlinewidth{1.003750pt}%
\definecolor{currentstroke}{rgb}{1.000000,1.000000,1.000000}%
\pgfsetstrokecolor{currentstroke}%
\pgfsetdash{}{0pt}%
\pgfpathmoveto{\pgfqpoint{1.879497in}{0.195328in}}%
\pgfpathlineto{\pgfqpoint{1.985657in}{0.195328in}}%
\pgfpathlineto{\pgfqpoint{1.985657in}{0.481266in}}%
\pgfpathlineto{\pgfqpoint{1.879497in}{0.481266in}}%
\pgfpathlineto{\pgfqpoint{1.879497in}{0.195328in}}%
\pgfpathclose%
\pgfusepath{stroke,fill}%
\end{pgfscope}%
\begin{pgfscope}%
\pgfpathrectangle{\pgfqpoint{0.539230in}{0.451389in}}{\pgfqpoint{4.379090in}{1.264399in}}%
\pgfusepath{clip}%
\pgfsetbuttcap%
\pgfsetmiterjoin%
\definecolor{currentfill}{rgb}{0.349020,0.490196,0.749020}%
\pgfsetfillcolor{currentfill}%
\pgfsetlinewidth{1.003750pt}%
\definecolor{currentstroke}{rgb}{1.000000,1.000000,1.000000}%
\pgfsetstrokecolor{currentstroke}%
\pgfsetdash{}{0pt}%
\pgfpathmoveto{\pgfqpoint{2.277596in}{0.195328in}}%
\pgfpathlineto{\pgfqpoint{2.383756in}{0.195328in}}%
\pgfpathlineto{\pgfqpoint{2.383756in}{0.494642in}}%
\pgfpathlineto{\pgfqpoint{2.277596in}{0.494642in}}%
\pgfpathlineto{\pgfqpoint{2.277596in}{0.195328in}}%
\pgfpathclose%
\pgfusepath{stroke,fill}%
\end{pgfscope}%
\begin{pgfscope}%
\pgfpathrectangle{\pgfqpoint{0.539230in}{0.451389in}}{\pgfqpoint{4.379090in}{1.264399in}}%
\pgfusepath{clip}%
\pgfsetbuttcap%
\pgfsetmiterjoin%
\definecolor{currentfill}{rgb}{0.349020,0.490196,0.749020}%
\pgfsetfillcolor{currentfill}%
\pgfsetlinewidth{1.003750pt}%
\definecolor{currentstroke}{rgb}{1.000000,1.000000,1.000000}%
\pgfsetstrokecolor{currentstroke}%
\pgfsetdash{}{0pt}%
\pgfpathmoveto{\pgfqpoint{2.675695in}{0.195328in}}%
\pgfpathlineto{\pgfqpoint{2.781855in}{0.195328in}}%
\pgfpathlineto{\pgfqpoint{2.781855in}{0.492109in}}%
\pgfpathlineto{\pgfqpoint{2.675695in}{0.492109in}}%
\pgfpathlineto{\pgfqpoint{2.675695in}{0.195328in}}%
\pgfpathclose%
\pgfusepath{stroke,fill}%
\end{pgfscope}%
\begin{pgfscope}%
\pgfpathrectangle{\pgfqpoint{0.539230in}{0.451389in}}{\pgfqpoint{4.379090in}{1.264399in}}%
\pgfusepath{clip}%
\pgfsetbuttcap%
\pgfsetmiterjoin%
\definecolor{currentfill}{rgb}{0.349020,0.490196,0.749020}%
\pgfsetfillcolor{currentfill}%
\pgfsetlinewidth{1.003750pt}%
\definecolor{currentstroke}{rgb}{1.000000,1.000000,1.000000}%
\pgfsetstrokecolor{currentstroke}%
\pgfsetdash{}{0pt}%
\pgfpathmoveto{\pgfqpoint{3.073794in}{0.195328in}}%
\pgfpathlineto{\pgfqpoint{3.179954in}{0.195328in}}%
\pgfpathlineto{\pgfqpoint{3.179954in}{0.479840in}}%
\pgfpathlineto{\pgfqpoint{3.073794in}{0.479840in}}%
\pgfpathlineto{\pgfqpoint{3.073794in}{0.195328in}}%
\pgfpathclose%
\pgfusepath{stroke,fill}%
\end{pgfscope}%
\begin{pgfscope}%
\pgfpathrectangle{\pgfqpoint{0.539230in}{0.451389in}}{\pgfqpoint{4.379090in}{1.264399in}}%
\pgfusepath{clip}%
\pgfsetbuttcap%
\pgfsetmiterjoin%
\definecolor{currentfill}{rgb}{0.349020,0.490196,0.749020}%
\pgfsetfillcolor{currentfill}%
\pgfsetlinewidth{1.003750pt}%
\definecolor{currentstroke}{rgb}{1.000000,1.000000,1.000000}%
\pgfsetstrokecolor{currentstroke}%
\pgfsetdash{}{0pt}%
\pgfpathmoveto{\pgfqpoint{3.471894in}{0.195328in}}%
\pgfpathlineto{\pgfqpoint{3.578053in}{0.195328in}}%
\pgfpathlineto{\pgfqpoint{3.578053in}{0.504163in}}%
\pgfpathlineto{\pgfqpoint{3.471894in}{0.504163in}}%
\pgfpathlineto{\pgfqpoint{3.471894in}{0.195328in}}%
\pgfpathclose%
\pgfusepath{stroke,fill}%
\end{pgfscope}%
\begin{pgfscope}%
\pgfpathrectangle{\pgfqpoint{0.539230in}{0.451389in}}{\pgfqpoint{4.379090in}{1.264399in}}%
\pgfusepath{clip}%
\pgfsetbuttcap%
\pgfsetmiterjoin%
\definecolor{currentfill}{rgb}{0.349020,0.490196,0.749020}%
\pgfsetfillcolor{currentfill}%
\pgfsetlinewidth{1.003750pt}%
\definecolor{currentstroke}{rgb}{1.000000,1.000000,1.000000}%
\pgfsetstrokecolor{currentstroke}%
\pgfsetdash{}{0pt}%
\pgfpathmoveto{\pgfqpoint{3.869993in}{0.195328in}}%
\pgfpathlineto{\pgfqpoint{3.976152in}{0.195328in}}%
\pgfpathlineto{\pgfqpoint{3.976152in}{0.561244in}}%
\pgfpathlineto{\pgfqpoint{3.869993in}{0.561244in}}%
\pgfpathlineto{\pgfqpoint{3.869993in}{0.195328in}}%
\pgfpathclose%
\pgfusepath{stroke,fill}%
\end{pgfscope}%
\begin{pgfscope}%
\pgfpathrectangle{\pgfqpoint{0.539230in}{0.451389in}}{\pgfqpoint{4.379090in}{1.264399in}}%
\pgfusepath{clip}%
\pgfsetbuttcap%
\pgfsetmiterjoin%
\definecolor{currentfill}{rgb}{0.349020,0.490196,0.749020}%
\pgfsetfillcolor{currentfill}%
\pgfsetlinewidth{1.003750pt}%
\definecolor{currentstroke}{rgb}{1.000000,1.000000,1.000000}%
\pgfsetstrokecolor{currentstroke}%
\pgfsetdash{}{0pt}%
\pgfpathmoveto{\pgfqpoint{4.268092in}{0.195328in}}%
\pgfpathlineto{\pgfqpoint{4.374251in}{0.195328in}}%
\pgfpathlineto{\pgfqpoint{4.374251in}{0.492514in}}%
\pgfpathlineto{\pgfqpoint{4.268092in}{0.492514in}}%
\pgfpathlineto{\pgfqpoint{4.268092in}{0.195328in}}%
\pgfpathclose%
\pgfusepath{stroke,fill}%
\end{pgfscope}%
\begin{pgfscope}%
\pgfpathrectangle{\pgfqpoint{0.539230in}{0.451389in}}{\pgfqpoint{4.379090in}{1.264399in}}%
\pgfusepath{clip}%
\pgfsetbuttcap%
\pgfsetmiterjoin%
\definecolor{currentfill}{rgb}{0.349020,0.490196,0.749020}%
\pgfsetfillcolor{currentfill}%
\pgfsetlinewidth{1.003750pt}%
\definecolor{currentstroke}{rgb}{1.000000,1.000000,1.000000}%
\pgfsetstrokecolor{currentstroke}%
\pgfsetdash{}{0pt}%
\pgfpathmoveto{\pgfqpoint{4.666191in}{0.195328in}}%
\pgfpathlineto{\pgfqpoint{4.772351in}{0.195328in}}%
\pgfpathlineto{\pgfqpoint{4.772351in}{0.496847in}}%
\pgfpathlineto{\pgfqpoint{4.666191in}{0.496847in}}%
\pgfpathlineto{\pgfqpoint{4.666191in}{0.195328in}}%
\pgfpathclose%
\pgfusepath{stroke,fill}%
\end{pgfscope}%
\begin{pgfscope}%
\pgfpathrectangle{\pgfqpoint{0.539230in}{0.451389in}}{\pgfqpoint{4.379090in}{1.264399in}}%
\pgfusepath{clip}%
\pgfsetbuttcap%
\pgfsetmiterjoin%
\definecolor{currentfill}{rgb}{0.852941,0.544118,0.370588}%
\pgfsetfillcolor{currentfill}%
\pgfsetlinewidth{1.003750pt}%
\definecolor{currentstroke}{rgb}{1.000000,1.000000,1.000000}%
\pgfsetstrokecolor{currentstroke}%
\pgfsetdash{}{0pt}%
\pgfpathmoveto{\pgfqpoint{0.791360in}{0.195328in}}%
\pgfpathlineto{\pgfqpoint{0.897520in}{0.195328in}}%
\pgfpathlineto{\pgfqpoint{0.897520in}{0.832529in}}%
\pgfpathlineto{\pgfqpoint{0.791360in}{0.832529in}}%
\pgfpathlineto{\pgfqpoint{0.791360in}{0.195328in}}%
\pgfpathclose%
\pgfusepath{stroke,fill}%
\end{pgfscope}%
\begin{pgfscope}%
\pgfpathrectangle{\pgfqpoint{0.539230in}{0.451389in}}{\pgfqpoint{4.379090in}{1.264399in}}%
\pgfusepath{clip}%
\pgfsetbuttcap%
\pgfsetmiterjoin%
\definecolor{currentfill}{rgb}{0.852941,0.544118,0.370588}%
\pgfsetfillcolor{currentfill}%
\pgfsetlinewidth{1.003750pt}%
\definecolor{currentstroke}{rgb}{1.000000,1.000000,1.000000}%
\pgfsetstrokecolor{currentstroke}%
\pgfsetdash{}{0pt}%
\pgfpathmoveto{\pgfqpoint{1.189459in}{0.195328in}}%
\pgfpathlineto{\pgfqpoint{1.295619in}{0.195328in}}%
\pgfpathlineto{\pgfqpoint{1.295619in}{0.671013in}}%
\pgfpathlineto{\pgfqpoint{1.189459in}{0.671013in}}%
\pgfpathlineto{\pgfqpoint{1.189459in}{0.195328in}}%
\pgfpathclose%
\pgfusepath{stroke,fill}%
\end{pgfscope}%
\begin{pgfscope}%
\pgfpathrectangle{\pgfqpoint{0.539230in}{0.451389in}}{\pgfqpoint{4.379090in}{1.264399in}}%
\pgfusepath{clip}%
\pgfsetbuttcap%
\pgfsetmiterjoin%
\definecolor{currentfill}{rgb}{0.852941,0.544118,0.370588}%
\pgfsetfillcolor{currentfill}%
\pgfsetlinewidth{1.003750pt}%
\definecolor{currentstroke}{rgb}{1.000000,1.000000,1.000000}%
\pgfsetstrokecolor{currentstroke}%
\pgfsetdash{}{0pt}%
\pgfpathmoveto{\pgfqpoint{1.587558in}{0.195328in}}%
\pgfpathlineto{\pgfqpoint{1.693718in}{0.195328in}}%
\pgfpathlineto{\pgfqpoint{1.693718in}{0.626446in}}%
\pgfpathlineto{\pgfqpoint{1.587558in}{0.626446in}}%
\pgfpathlineto{\pgfqpoint{1.587558in}{0.195328in}}%
\pgfpathclose%
\pgfusepath{stroke,fill}%
\end{pgfscope}%
\begin{pgfscope}%
\pgfpathrectangle{\pgfqpoint{0.539230in}{0.451389in}}{\pgfqpoint{4.379090in}{1.264399in}}%
\pgfusepath{clip}%
\pgfsetbuttcap%
\pgfsetmiterjoin%
\definecolor{currentfill}{rgb}{0.852941,0.544118,0.370588}%
\pgfsetfillcolor{currentfill}%
\pgfsetlinewidth{1.003750pt}%
\definecolor{currentstroke}{rgb}{1.000000,1.000000,1.000000}%
\pgfsetstrokecolor{currentstroke}%
\pgfsetdash{}{0pt}%
\pgfpathmoveto{\pgfqpoint{1.985657in}{0.195328in}}%
\pgfpathlineto{\pgfqpoint{2.091817in}{0.195328in}}%
\pgfpathlineto{\pgfqpoint{2.091817in}{1.083344in}}%
\pgfpathlineto{\pgfqpoint{1.985657in}{1.083344in}}%
\pgfpathlineto{\pgfqpoint{1.985657in}{0.195328in}}%
\pgfpathclose%
\pgfusepath{stroke,fill}%
\end{pgfscope}%
\begin{pgfscope}%
\pgfpathrectangle{\pgfqpoint{0.539230in}{0.451389in}}{\pgfqpoint{4.379090in}{1.264399in}}%
\pgfusepath{clip}%
\pgfsetbuttcap%
\pgfsetmiterjoin%
\definecolor{currentfill}{rgb}{0.852941,0.544118,0.370588}%
\pgfsetfillcolor{currentfill}%
\pgfsetlinewidth{1.003750pt}%
\definecolor{currentstroke}{rgb}{1.000000,1.000000,1.000000}%
\pgfsetstrokecolor{currentstroke}%
\pgfsetdash{}{0pt}%
\pgfpathmoveto{\pgfqpoint{2.383756in}{0.195328in}}%
\pgfpathlineto{\pgfqpoint{2.489916in}{0.195328in}}%
\pgfpathlineto{\pgfqpoint{2.489916in}{0.620468in}}%
\pgfpathlineto{\pgfqpoint{2.383756in}{0.620468in}}%
\pgfpathlineto{\pgfqpoint{2.383756in}{0.195328in}}%
\pgfpathclose%
\pgfusepath{stroke,fill}%
\end{pgfscope}%
\begin{pgfscope}%
\pgfpathrectangle{\pgfqpoint{0.539230in}{0.451389in}}{\pgfqpoint{4.379090in}{1.264399in}}%
\pgfusepath{clip}%
\pgfsetbuttcap%
\pgfsetmiterjoin%
\definecolor{currentfill}{rgb}{0.852941,0.544118,0.370588}%
\pgfsetfillcolor{currentfill}%
\pgfsetlinewidth{1.003750pt}%
\definecolor{currentstroke}{rgb}{1.000000,1.000000,1.000000}%
\pgfsetstrokecolor{currentstroke}%
\pgfsetdash{}{0pt}%
\pgfpathmoveto{\pgfqpoint{2.781855in}{0.195328in}}%
\pgfpathlineto{\pgfqpoint{2.888015in}{0.195328in}}%
\pgfpathlineto{\pgfqpoint{2.888015in}{1.643385in}}%
\pgfpathlineto{\pgfqpoint{2.781855in}{1.643385in}}%
\pgfpathlineto{\pgfqpoint{2.781855in}{0.195328in}}%
\pgfpathclose%
\pgfusepath{stroke,fill}%
\end{pgfscope}%
\begin{pgfscope}%
\pgfpathrectangle{\pgfqpoint{0.539230in}{0.451389in}}{\pgfqpoint{4.379090in}{1.264399in}}%
\pgfusepath{clip}%
\pgfsetbuttcap%
\pgfsetmiterjoin%
\definecolor{currentfill}{rgb}{0.852941,0.544118,0.370588}%
\pgfsetfillcolor{currentfill}%
\pgfsetlinewidth{1.003750pt}%
\definecolor{currentstroke}{rgb}{1.000000,1.000000,1.000000}%
\pgfsetstrokecolor{currentstroke}%
\pgfsetdash{}{0pt}%
\pgfpathmoveto{\pgfqpoint{3.179954in}{0.195328in}}%
\pgfpathlineto{\pgfqpoint{3.286114in}{0.195328in}}%
\pgfpathlineto{\pgfqpoint{3.286114in}{1.414755in}}%
\pgfpathlineto{\pgfqpoint{3.179954in}{1.414755in}}%
\pgfpathlineto{\pgfqpoint{3.179954in}{0.195328in}}%
\pgfpathclose%
\pgfusepath{stroke,fill}%
\end{pgfscope}%
\begin{pgfscope}%
\pgfpathrectangle{\pgfqpoint{0.539230in}{0.451389in}}{\pgfqpoint{4.379090in}{1.264399in}}%
\pgfusepath{clip}%
\pgfsetbuttcap%
\pgfsetmiterjoin%
\definecolor{currentfill}{rgb}{0.852941,0.544118,0.370588}%
\pgfsetfillcolor{currentfill}%
\pgfsetlinewidth{1.003750pt}%
\definecolor{currentstroke}{rgb}{1.000000,1.000000,1.000000}%
\pgfsetstrokecolor{currentstroke}%
\pgfsetdash{}{0pt}%
\pgfpathmoveto{\pgfqpoint{3.578053in}{0.195328in}}%
\pgfpathlineto{\pgfqpoint{3.684213in}{0.195328in}}%
\pgfpathlineto{\pgfqpoint{3.684213in}{0.558581in}}%
\pgfpathlineto{\pgfqpoint{3.578053in}{0.558581in}}%
\pgfpathlineto{\pgfqpoint{3.578053in}{0.195328in}}%
\pgfpathclose%
\pgfusepath{stroke,fill}%
\end{pgfscope}%
\begin{pgfscope}%
\pgfpathrectangle{\pgfqpoint{0.539230in}{0.451389in}}{\pgfqpoint{4.379090in}{1.264399in}}%
\pgfusepath{clip}%
\pgfsetbuttcap%
\pgfsetmiterjoin%
\definecolor{currentfill}{rgb}{0.852941,0.544118,0.370588}%
\pgfsetfillcolor{currentfill}%
\pgfsetlinewidth{1.003750pt}%
\definecolor{currentstroke}{rgb}{1.000000,1.000000,1.000000}%
\pgfsetstrokecolor{currentstroke}%
\pgfsetdash{}{0pt}%
\pgfpathmoveto{\pgfqpoint{3.976152in}{0.195328in}}%
\pgfpathlineto{\pgfqpoint{4.082312in}{0.195328in}}%
\pgfpathlineto{\pgfqpoint{4.082312in}{0.594022in}}%
\pgfpathlineto{\pgfqpoint{3.976152in}{0.594022in}}%
\pgfpathlineto{\pgfqpoint{3.976152in}{0.195328in}}%
\pgfpathclose%
\pgfusepath{stroke,fill}%
\end{pgfscope}%
\begin{pgfscope}%
\pgfpathrectangle{\pgfqpoint{0.539230in}{0.451389in}}{\pgfqpoint{4.379090in}{1.264399in}}%
\pgfusepath{clip}%
\pgfsetbuttcap%
\pgfsetmiterjoin%
\definecolor{currentfill}{rgb}{0.852941,0.544118,0.370588}%
\pgfsetfillcolor{currentfill}%
\pgfsetlinewidth{1.003750pt}%
\definecolor{currentstroke}{rgb}{1.000000,1.000000,1.000000}%
\pgfsetstrokecolor{currentstroke}%
\pgfsetdash{}{0pt}%
\pgfpathmoveto{\pgfqpoint{4.374251in}{0.195328in}}%
\pgfpathlineto{\pgfqpoint{4.480411in}{0.195328in}}%
\pgfpathlineto{\pgfqpoint{4.480411in}{0.517949in}}%
\pgfpathlineto{\pgfqpoint{4.374251in}{0.517949in}}%
\pgfpathlineto{\pgfqpoint{4.374251in}{0.195328in}}%
\pgfpathclose%
\pgfusepath{stroke,fill}%
\end{pgfscope}%
\begin{pgfscope}%
\pgfpathrectangle{\pgfqpoint{0.539230in}{0.451389in}}{\pgfqpoint{4.379090in}{1.264399in}}%
\pgfusepath{clip}%
\pgfsetbuttcap%
\pgfsetmiterjoin%
\definecolor{currentfill}{rgb}{0.852941,0.544118,0.370588}%
\pgfsetfillcolor{currentfill}%
\pgfsetlinewidth{1.003750pt}%
\definecolor{currentstroke}{rgb}{1.000000,1.000000,1.000000}%
\pgfsetstrokecolor{currentstroke}%
\pgfsetdash{}{0pt}%
\pgfpathmoveto{\pgfqpoint{4.772351in}{0.195328in}}%
\pgfpathlineto{\pgfqpoint{4.878510in}{0.195328in}}%
\pgfpathlineto{\pgfqpoint{4.878510in}{0.856249in}}%
\pgfpathlineto{\pgfqpoint{4.772351in}{0.856249in}}%
\pgfpathlineto{\pgfqpoint{4.772351in}{0.195328in}}%
\pgfpathclose%
\pgfusepath{stroke,fill}%
\end{pgfscope}%
\begin{pgfscope}%
\pgfsetrectcap%
\pgfsetmiterjoin%
\pgfsetlinewidth{1.254687pt}%
\definecolor{currentstroke}{rgb}{0.800000,0.800000,0.800000}%
\pgfsetstrokecolor{currentstroke}%
\pgfsetdash{}{0pt}%
\pgfpathmoveto{\pgfqpoint{0.539230in}{0.451389in}}%
\pgfpathlineto{\pgfqpoint{0.539230in}{1.715788in}}%
\pgfusepath{stroke}%
\end{pgfscope}%
\begin{pgfscope}%
\pgfsetrectcap%
\pgfsetmiterjoin%
\pgfsetlinewidth{1.254687pt}%
\definecolor{currentstroke}{rgb}{0.800000,0.800000,0.800000}%
\pgfsetstrokecolor{currentstroke}%
\pgfsetdash{}{0pt}%
\pgfpathmoveto{\pgfqpoint{4.918320in}{0.451389in}}%
\pgfpathlineto{\pgfqpoint{4.918320in}{1.715788in}}%
\pgfusepath{stroke}%
\end{pgfscope}%
\begin{pgfscope}%
\pgfsetrectcap%
\pgfsetmiterjoin%
\pgfsetlinewidth{1.254687pt}%
\definecolor{currentstroke}{rgb}{0.800000,0.800000,0.800000}%
\pgfsetstrokecolor{currentstroke}%
\pgfsetdash{}{0pt}%
\pgfpathmoveto{\pgfqpoint{0.539230in}{0.451389in}}%
\pgfpathlineto{\pgfqpoint{4.918320in}{0.451389in}}%
\pgfusepath{stroke}%
\end{pgfscope}%
\begin{pgfscope}%
\pgfsetrectcap%
\pgfsetmiterjoin%
\pgfsetlinewidth{1.254687pt}%
\definecolor{currentstroke}{rgb}{0.800000,0.800000,0.800000}%
\pgfsetstrokecolor{currentstroke}%
\pgfsetdash{}{0pt}%
\pgfpathmoveto{\pgfqpoint{0.539230in}{1.715788in}}%
\pgfpathlineto{\pgfqpoint{4.918320in}{1.715788in}}%
\pgfusepath{stroke}%
\end{pgfscope}%
\begin{pgfscope}%
\pgfsetbuttcap%
\pgfsetmiterjoin%
\definecolor{currentfill}{rgb}{1.000000,1.000000,1.000000}%
\pgfsetfillcolor{currentfill}%
\pgfsetfillopacity{0.800000}%
\pgfsetlinewidth{1.003750pt}%
\definecolor{currentstroke}{rgb}{0.800000,0.800000,0.800000}%
\pgfsetstrokecolor{currentstroke}%
\pgfsetstrokeopacity{0.800000}%
\pgfsetdash{}{0pt}%
\pgfpathmoveto{\pgfqpoint{5.115297in}{0.385671in}}%
\pgfpathlineto{\pgfqpoint{6.061074in}{0.385671in}}%
\pgfpathquadraticcurveto{\pgfqpoint{6.086074in}{0.385671in}}{\pgfqpoint{6.086074in}{0.410671in}}%
\pgfpathlineto{\pgfqpoint{6.086074in}{1.680641in}}%
\pgfpathquadraticcurveto{\pgfqpoint{6.086074in}{1.705641in}}{\pgfqpoint{6.061074in}{1.705641in}}%
\pgfpathlineto{\pgfqpoint{5.115297in}{1.705641in}}%
\pgfpathquadraticcurveto{\pgfqpoint{5.090297in}{1.705641in}}{\pgfqpoint{5.090297in}{1.680641in}}%
\pgfpathlineto{\pgfqpoint{5.090297in}{0.410671in}}%
\pgfpathquadraticcurveto{\pgfqpoint{5.090297in}{0.385671in}}{\pgfqpoint{5.115297in}{0.385671in}}%
\pgfpathlineto{\pgfqpoint{5.115297in}{0.385671in}}%
\pgfpathclose%
\pgfusepath{stroke,fill}%
\end{pgfscope}%
\begin{pgfscope}%
\pgfsetbuttcap%
\pgfsetmiterjoin%
\definecolor{currentfill}{rgb}{0.460784,0.749020,0.443137}%
\pgfsetfillcolor{currentfill}%
\pgfsetlinewidth{1.003750pt}%
\definecolor{currentstroke}{rgb}{1.000000,1.000000,1.000000}%
\pgfsetstrokecolor{currentstroke}%
\pgfsetdash{}{0pt}%
\pgfpathmoveto{\pgfqpoint{5.140297in}{1.473659in}}%
\pgfpathlineto{\pgfqpoint{5.390297in}{1.473659in}}%
\pgfpathlineto{\pgfqpoint{5.390297in}{1.561159in}}%
\pgfpathlineto{\pgfqpoint{5.140297in}{1.561159in}}%
\pgfpathlineto{\pgfqpoint{5.140297in}{1.473659in}}%
\pgfpathclose%
\pgfusepath{stroke,fill}%
\end{pgfscope}%
\begin{pgfscope}%
\definecolor{textcolor}{rgb}{0.150000,0.150000,0.150000}%
\pgfsetstrokecolor{textcolor}%
\pgfsetfillcolor{textcolor}%
\pgftext[x=5.490297in, y=1.560670in, left, base]{\color{textcolor}\sffamily\fontsize{9.000000}{10.800000}\selectfont Expected}%
\end{pgfscope}%
\begin{pgfscope}%
\definecolor{textcolor}{rgb}{0.150000,0.150000,0.150000}%
\pgfsetstrokecolor{textcolor}%
\pgfsetfillcolor{textcolor}%
\pgftext[x=5.490297in, y=1.416676in, left, base]{\color{textcolor}\sffamily\fontsize{9.000000}{10.800000}\selectfont route}%
\end{pgfscope}%
\begin{pgfscope}%
\pgfsetbuttcap%
\pgfsetmiterjoin%
\definecolor{currentfill}{rgb}{0.349020,0.490196,0.749020}%
\pgfsetfillcolor{currentfill}%
\pgfsetlinewidth{1.003750pt}%
\definecolor{currentstroke}{rgb}{1.000000,1.000000,1.000000}%
\pgfsetstrokecolor{currentstroke}%
\pgfsetdash{}{0pt}%
\pgfpathmoveto{\pgfqpoint{5.140297in}{1.070168in}}%
\pgfpathlineto{\pgfqpoint{5.390297in}{1.070168in}}%
\pgfpathlineto{\pgfqpoint{5.390297in}{1.157668in}}%
\pgfpathlineto{\pgfqpoint{5.140297in}{1.157668in}}%
\pgfpathlineto{\pgfqpoint{5.140297in}{1.070168in}}%
\pgfpathclose%
\pgfusepath{stroke,fill}%
\end{pgfscope}%
\begin{pgfscope}%
\definecolor{textcolor}{rgb}{0.150000,0.150000,0.150000}%
\pgfsetstrokecolor{textcolor}%
\pgfsetfillcolor{textcolor}%
\pgftext[x=5.490297in, y=1.229176in, left, base]{\color{textcolor}\sffamily\fontsize{9.000000}{10.800000}\selectfont Hybrid}%
\end{pgfscope}%
\begin{pgfscope}%
\definecolor{textcolor}{rgb}{0.150000,0.150000,0.150000}%
\pgfsetstrokecolor{textcolor}%
\pgfsetfillcolor{textcolor}%
\pgftext[x=5.490297in, y=1.085182in, left, base]{\color{textcolor}\sffamily\fontsize{9.000000}{10.800000}\selectfont routing}%
\end{pgfscope}%
\begin{pgfscope}%
\definecolor{textcolor}{rgb}{0.150000,0.150000,0.150000}%
\pgfsetstrokecolor{textcolor}%
\pgfsetfillcolor{textcolor}%
\pgftext[x=5.490297in, y=0.941188in, left, base]{\color{textcolor}\sffamily\fontsize{9.000000}{10.800000}\selectfont algorithm}%
\end{pgfscope}%
\begin{pgfscope}%
\pgfsetbuttcap%
\pgfsetmiterjoin%
\definecolor{currentfill}{rgb}{0.852941,0.544118,0.370588}%
\pgfsetfillcolor{currentfill}%
\pgfsetlinewidth{1.003750pt}%
\definecolor{currentstroke}{rgb}{1.000000,1.000000,1.000000}%
\pgfsetstrokecolor{currentstroke}%
\pgfsetdash{}{0pt}%
\pgfpathmoveto{\pgfqpoint{5.140297in}{0.594680in}}%
\pgfpathlineto{\pgfqpoint{5.390297in}{0.594680in}}%
\pgfpathlineto{\pgfqpoint{5.390297in}{0.682180in}}%
\pgfpathlineto{\pgfqpoint{5.140297in}{0.682180in}}%
\pgfpathlineto{\pgfqpoint{5.140297in}{0.594680in}}%
\pgfpathclose%
\pgfusepath{stroke,fill}%
\end{pgfscope}%
\begin{pgfscope}%
\definecolor{textcolor}{rgb}{0.150000,0.150000,0.150000}%
\pgfsetstrokecolor{textcolor}%
\pgfsetfillcolor{textcolor}%
\pgftext[x=5.490297in, y=0.753689in, left, base]{\color{textcolor}\sffamily\fontsize{9.000000}{10.800000}\selectfont Graph-}%
\end{pgfscope}%
\begin{pgfscope}%
\definecolor{textcolor}{rgb}{0.150000,0.150000,0.150000}%
\pgfsetstrokecolor{textcolor}%
\pgfsetfillcolor{textcolor}%
\pgftext[x=5.490297in, y=0.609695in, left, base]{\color{textcolor}\sffamily\fontsize{9.000000}{10.800000}\selectfont based}%
\end{pgfscope}%
\begin{pgfscope}%
\definecolor{textcolor}{rgb}{0.150000,0.150000,0.150000}%
\pgfsetstrokecolor{textcolor}%
\pgfsetfillcolor{textcolor}%
\pgftext[x=5.490297in, y=0.465701in, left, base]{\color{textcolor}\sffamily\fontsize{9.000000}{10.800000}\selectfont routing}%
\end{pgfscope}%
\end{pgfpicture}%
\makeatother%
\endgroup%

					\end{figcenter}
					\caption{\enquote{OSM rural} dataset.}
					\label{fig:eval-route-distances-rural}
				\end{subfigure}
				\caption[Relative route distance comparison.]{Relative route distances compared to the beeline distance between the waypoints of each routing request using the 0.5km\textsuperscript{2} OSM datasets.}
				\label{fig:eval-route-distances}
				\vspace{3ex}
	%			\end{figure}
				
	%			\begin{figure}[h]
				\begin{subfigure}[t]{\linewidth}
					\begin{figcenter}
						%% Creator: Matplotlib, PGF backend
%%
%% To include the figure in your LaTeX document, write
%%   \input{<filename>.pgf}
%%
%% Make sure the required packages are loaded in your preamble
%%   \usepackage{pgf}
%%
%% Also ensure that all the required font packages are loaded; for instance,
%% the lmodern package is sometimes necessary when using math font.
%%   \usepackage{lmodern}
%%
%% Figures using additional raster images can only be included by \input if
%% they are in the same directory as the main LaTeX file. For loading figures
%% from other directories you can use the `import` package
%%   \usepackage{import}
%%
%% and then include the figures with
%%   \import{<path to file>}{<filename>.pgf}
%%
%% Matplotlib used the following preamble
%%   
%%   \usepackage{fontspec}
%%   \setmainfont{DejaVuSerif.ttf}[Path=\detokenize{/home/hauke/.local/lib/python3.11/site-packages/matplotlib/mpl-data/fonts/ttf/}]
%%   \setsansfont{DroidSans.ttf}[Path=\detokenize{/usr/share/fonts/droid/}]
%%   \setmonofont{DejaVuSansMono.ttf}[Path=\detokenize{/home/hauke/.local/lib/python3.11/site-packages/matplotlib/mpl-data/fonts/ttf/}]
%%   \makeatletter\@ifpackageloaded{underscore}{}{\usepackage[strings]{underscore}}\makeatother
%%
\begingroup%
\makeatletter%
\begin{pgfpicture}%
\pgfpathrectangle{\pgfpointorigin}{\pgfqpoint{6.084564in}{1.715788in}}%
\pgfusepath{use as bounding box, clip}%
\begin{pgfscope}%
\pgfsetbuttcap%
\pgfsetmiterjoin%
\definecolor{currentfill}{rgb}{1.000000,1.000000,1.000000}%
\pgfsetfillcolor{currentfill}%
\pgfsetlinewidth{0.000000pt}%
\definecolor{currentstroke}{rgb}{1.000000,1.000000,1.000000}%
\pgfsetstrokecolor{currentstroke}%
\pgfsetdash{}{0pt}%
\pgfpathmoveto{\pgfqpoint{0.000000in}{0.000000in}}%
\pgfpathlineto{\pgfqpoint{6.084564in}{0.000000in}}%
\pgfpathlineto{\pgfqpoint{6.084564in}{1.715788in}}%
\pgfpathlineto{\pgfqpoint{0.000000in}{1.715788in}}%
\pgfpathlineto{\pgfqpoint{0.000000in}{0.000000in}}%
\pgfpathclose%
\pgfusepath{fill}%
\end{pgfscope}%
\begin{pgfscope}%
\pgfsetbuttcap%
\pgfsetmiterjoin%
\definecolor{currentfill}{rgb}{1.000000,1.000000,1.000000}%
\pgfsetfillcolor{currentfill}%
\pgfsetlinewidth{0.000000pt}%
\definecolor{currentstroke}{rgb}{0.000000,0.000000,0.000000}%
\pgfsetstrokecolor{currentstroke}%
\pgfsetstrokeopacity{0.000000}%
\pgfsetdash{}{0pt}%
\pgfpathmoveto{\pgfqpoint{0.532932in}{0.451389in}}%
\pgfpathlineto{\pgfqpoint{4.916693in}{0.451389in}}%
\pgfpathlineto{\pgfqpoint{4.916693in}{1.715788in}}%
\pgfpathlineto{\pgfqpoint{0.532932in}{1.715788in}}%
\pgfpathlineto{\pgfqpoint{0.532932in}{0.451389in}}%
\pgfpathclose%
\pgfusepath{fill}%
\end{pgfscope}%
\begin{pgfscope}%
\definecolor{textcolor}{rgb}{0.150000,0.150000,0.150000}%
\pgfsetstrokecolor{textcolor}%
\pgfsetfillcolor{textcolor}%
\pgftext[x=0.732194in,y=0.319444in,,top]{\color{textcolor}\sffamily\fontsize{9.000000}{10.800000}\selectfont 1}%
\end{pgfscope}%
\begin{pgfscope}%
\definecolor{textcolor}{rgb}{0.150000,0.150000,0.150000}%
\pgfsetstrokecolor{textcolor}%
\pgfsetfillcolor{textcolor}%
\pgftext[x=1.130717in,y=0.319444in,,top]{\color{textcolor}\sffamily\fontsize{9.000000}{10.800000}\selectfont 2}%
\end{pgfscope}%
\begin{pgfscope}%
\definecolor{textcolor}{rgb}{0.150000,0.150000,0.150000}%
\pgfsetstrokecolor{textcolor}%
\pgfsetfillcolor{textcolor}%
\pgftext[x=1.529241in,y=0.319444in,,top]{\color{textcolor}\sffamily\fontsize{9.000000}{10.800000}\selectfont 3}%
\end{pgfscope}%
\begin{pgfscope}%
\definecolor{textcolor}{rgb}{0.150000,0.150000,0.150000}%
\pgfsetstrokecolor{textcolor}%
\pgfsetfillcolor{textcolor}%
\pgftext[x=1.927765in,y=0.319444in,,top]{\color{textcolor}\sffamily\fontsize{9.000000}{10.800000}\selectfont 4}%
\end{pgfscope}%
\begin{pgfscope}%
\definecolor{textcolor}{rgb}{0.150000,0.150000,0.150000}%
\pgfsetstrokecolor{textcolor}%
\pgfsetfillcolor{textcolor}%
\pgftext[x=2.326289in,y=0.319444in,,top]{\color{textcolor}\sffamily\fontsize{9.000000}{10.800000}\selectfont 5}%
\end{pgfscope}%
\begin{pgfscope}%
\definecolor{textcolor}{rgb}{0.150000,0.150000,0.150000}%
\pgfsetstrokecolor{textcolor}%
\pgfsetfillcolor{textcolor}%
\pgftext[x=2.724813in,y=0.319444in,,top]{\color{textcolor}\sffamily\fontsize{9.000000}{10.800000}\selectfont 6}%
\end{pgfscope}%
\begin{pgfscope}%
\definecolor{textcolor}{rgb}{0.150000,0.150000,0.150000}%
\pgfsetstrokecolor{textcolor}%
\pgfsetfillcolor{textcolor}%
\pgftext[x=3.123336in,y=0.319444in,,top]{\color{textcolor}\sffamily\fontsize{9.000000}{10.800000}\selectfont 7}%
\end{pgfscope}%
\begin{pgfscope}%
\definecolor{textcolor}{rgb}{0.150000,0.150000,0.150000}%
\pgfsetstrokecolor{textcolor}%
\pgfsetfillcolor{textcolor}%
\pgftext[x=3.521860in,y=0.319444in,,top]{\color{textcolor}\sffamily\fontsize{9.000000}{10.800000}\selectfont 8}%
\end{pgfscope}%
\begin{pgfscope}%
\definecolor{textcolor}{rgb}{0.150000,0.150000,0.150000}%
\pgfsetstrokecolor{textcolor}%
\pgfsetfillcolor{textcolor}%
\pgftext[x=3.920384in,y=0.319444in,,top]{\color{textcolor}\sffamily\fontsize{9.000000}{10.800000}\selectfont 9}%
\end{pgfscope}%
\begin{pgfscope}%
\definecolor{textcolor}{rgb}{0.150000,0.150000,0.150000}%
\pgfsetstrokecolor{textcolor}%
\pgfsetfillcolor{textcolor}%
\pgftext[x=4.318908in,y=0.319444in,,top]{\color{textcolor}\sffamily\fontsize{9.000000}{10.800000}\selectfont 10}%
\end{pgfscope}%
\begin{pgfscope}%
\definecolor{textcolor}{rgb}{0.150000,0.150000,0.150000}%
\pgfsetstrokecolor{textcolor}%
\pgfsetfillcolor{textcolor}%
\pgftext[x=4.717432in,y=0.319444in,,top]{\color{textcolor}\sffamily\fontsize{9.000000}{10.800000}\selectfont mean}%
\end{pgfscope}%
\begin{pgfscope}%
\definecolor{textcolor}{rgb}{0.150000,0.150000,0.150000}%
\pgfsetstrokecolor{textcolor}%
\pgfsetfillcolor{textcolor}%
\pgftext[x=2.724813in,y=0.125000in,,top]{\color{textcolor}\sffamily\fontsize{9.000000}{10.800000}\selectfont Routing request}%
\end{pgfscope}%
\begin{pgfscope}%
\pgfpathrectangle{\pgfqpoint{0.532932in}{0.451389in}}{\pgfqpoint{4.383762in}{1.264399in}}%
\pgfusepath{clip}%
\pgfsetroundcap%
\pgfsetroundjoin%
\pgfsetlinewidth{1.003750pt}%
\definecolor{currentstroke}{rgb}{0.800000,0.800000,0.800000}%
\pgfsetstrokecolor{currentstroke}%
\pgfsetdash{}{0pt}%
\pgfpathmoveto{\pgfqpoint{0.532932in}{0.451389in}}%
\pgfpathlineto{\pgfqpoint{4.916693in}{0.451389in}}%
\pgfusepath{stroke}%
\end{pgfscope}%
\begin{pgfscope}%
\definecolor{textcolor}{rgb}{0.150000,0.150000,0.150000}%
\pgfsetstrokecolor{textcolor}%
\pgfsetfillcolor{textcolor}%
\pgftext[x=0.332140in, y=0.403903in, left, base]{\color{textcolor}\sffamily\fontsize{9.000000}{10.800000}\selectfont 0}%
\end{pgfscope}%
\begin{pgfscope}%
\pgfpathrectangle{\pgfqpoint{0.532932in}{0.451389in}}{\pgfqpoint{4.383762in}{1.264399in}}%
\pgfusepath{clip}%
\pgfsetroundcap%
\pgfsetroundjoin%
\pgfsetlinewidth{1.003750pt}%
\definecolor{currentstroke}{rgb}{0.800000,0.800000,0.800000}%
\pgfsetstrokecolor{currentstroke}%
\pgfsetdash{}{0pt}%
\pgfpathmoveto{\pgfqpoint{0.532932in}{0.884223in}}%
\pgfpathlineto{\pgfqpoint{4.916693in}{0.884223in}}%
\pgfusepath{stroke}%
\end{pgfscope}%
\begin{pgfscope}%
\definecolor{textcolor}{rgb}{0.150000,0.150000,0.150000}%
\pgfsetstrokecolor{textcolor}%
\pgfsetfillcolor{textcolor}%
\pgftext[x=0.194444in, y=0.836738in, left, base]{\color{textcolor}\sffamily\fontsize{9.000000}{10.800000}\selectfont 100}%
\end{pgfscope}%
\begin{pgfscope}%
\pgfpathrectangle{\pgfqpoint{0.532932in}{0.451389in}}{\pgfqpoint{4.383762in}{1.264399in}}%
\pgfusepath{clip}%
\pgfsetroundcap%
\pgfsetroundjoin%
\pgfsetlinewidth{1.003750pt}%
\definecolor{currentstroke}{rgb}{0.800000,0.800000,0.800000}%
\pgfsetstrokecolor{currentstroke}%
\pgfsetdash{}{0pt}%
\pgfpathmoveto{\pgfqpoint{0.532932in}{1.317058in}}%
\pgfpathlineto{\pgfqpoint{4.916693in}{1.317058in}}%
\pgfusepath{stroke}%
\end{pgfscope}%
\begin{pgfscope}%
\definecolor{textcolor}{rgb}{0.150000,0.150000,0.150000}%
\pgfsetstrokecolor{textcolor}%
\pgfsetfillcolor{textcolor}%
\pgftext[x=0.194444in, y=1.269573in, left, base]{\color{textcolor}\sffamily\fontsize{9.000000}{10.800000}\selectfont 200}%
\end{pgfscope}%
\begin{pgfscope}%
\definecolor{textcolor}{rgb}{0.150000,0.150000,0.150000}%
\pgfsetstrokecolor{textcolor}%
\pgfsetfillcolor{textcolor}%
\pgftext[x=0.125000in,y=1.083588in,,bottom,rotate=90.000000]{\color{textcolor}\sffamily\fontsize{9.000000}{10.800000}\selectfont Hausdorff distance}%
\end{pgfscope}%
\begin{pgfscope}%
\pgfpathrectangle{\pgfqpoint{0.532932in}{0.451389in}}{\pgfqpoint{4.383762in}{1.264399in}}%
\pgfusepath{clip}%
\pgfsetbuttcap%
\pgfsetmiterjoin%
\definecolor{currentfill}{rgb}{0.349020,0.490196,0.749020}%
\pgfsetfillcolor{currentfill}%
\pgfsetlinewidth{1.003750pt}%
\definecolor{currentstroke}{rgb}{1.000000,1.000000,1.000000}%
\pgfsetstrokecolor{currentstroke}%
\pgfsetdash{}{0pt}%
\pgfpathmoveto{\pgfqpoint{0.572784in}{0.451389in}}%
\pgfpathlineto{\pgfqpoint{0.732194in}{0.451389in}}%
\pgfpathlineto{\pgfqpoint{0.732194in}{0.511243in}}%
\pgfpathlineto{\pgfqpoint{0.572784in}{0.511243in}}%
\pgfpathlineto{\pgfqpoint{0.572784in}{0.451389in}}%
\pgfpathclose%
\pgfusepath{stroke,fill}%
\end{pgfscope}%
\begin{pgfscope}%
\pgfpathrectangle{\pgfqpoint{0.532932in}{0.451389in}}{\pgfqpoint{4.383762in}{1.264399in}}%
\pgfusepath{clip}%
\pgfsetbuttcap%
\pgfsetmiterjoin%
\definecolor{currentfill}{rgb}{0.349020,0.490196,0.749020}%
\pgfsetfillcolor{currentfill}%
\pgfsetlinewidth{1.003750pt}%
\definecolor{currentstroke}{rgb}{1.000000,1.000000,1.000000}%
\pgfsetstrokecolor{currentstroke}%
\pgfsetdash{}{0pt}%
\pgfpathmoveto{\pgfqpoint{0.971308in}{0.451389in}}%
\pgfpathlineto{\pgfqpoint{1.130717in}{0.451389in}}%
\pgfpathlineto{\pgfqpoint{1.130717in}{0.520923in}}%
\pgfpathlineto{\pgfqpoint{0.971308in}{0.520923in}}%
\pgfpathlineto{\pgfqpoint{0.971308in}{0.451389in}}%
\pgfpathclose%
\pgfusepath{stroke,fill}%
\end{pgfscope}%
\begin{pgfscope}%
\pgfpathrectangle{\pgfqpoint{0.532932in}{0.451389in}}{\pgfqpoint{4.383762in}{1.264399in}}%
\pgfusepath{clip}%
\pgfsetbuttcap%
\pgfsetmiterjoin%
\definecolor{currentfill}{rgb}{0.349020,0.490196,0.749020}%
\pgfsetfillcolor{currentfill}%
\pgfsetlinewidth{1.003750pt}%
\definecolor{currentstroke}{rgb}{1.000000,1.000000,1.000000}%
\pgfsetstrokecolor{currentstroke}%
\pgfsetdash{}{0pt}%
\pgfpathmoveto{\pgfqpoint{1.369832in}{0.451389in}}%
\pgfpathlineto{\pgfqpoint{1.529241in}{0.451389in}}%
\pgfpathlineto{\pgfqpoint{1.529241in}{0.720067in}}%
\pgfpathlineto{\pgfqpoint{1.369832in}{0.720067in}}%
\pgfpathlineto{\pgfqpoint{1.369832in}{0.451389in}}%
\pgfpathclose%
\pgfusepath{stroke,fill}%
\end{pgfscope}%
\begin{pgfscope}%
\pgfpathrectangle{\pgfqpoint{0.532932in}{0.451389in}}{\pgfqpoint{4.383762in}{1.264399in}}%
\pgfusepath{clip}%
\pgfsetbuttcap%
\pgfsetmiterjoin%
\definecolor{currentfill}{rgb}{0.349020,0.490196,0.749020}%
\pgfsetfillcolor{currentfill}%
\pgfsetlinewidth{1.003750pt}%
\definecolor{currentstroke}{rgb}{1.000000,1.000000,1.000000}%
\pgfsetstrokecolor{currentstroke}%
\pgfsetdash{}{0pt}%
\pgfpathmoveto{\pgfqpoint{1.768356in}{0.451389in}}%
\pgfpathlineto{\pgfqpoint{1.927765in}{0.451389in}}%
\pgfpathlineto{\pgfqpoint{1.927765in}{0.495508in}}%
\pgfpathlineto{\pgfqpoint{1.768356in}{0.495508in}}%
\pgfpathlineto{\pgfqpoint{1.768356in}{0.451389in}}%
\pgfpathclose%
\pgfusepath{stroke,fill}%
\end{pgfscope}%
\begin{pgfscope}%
\pgfpathrectangle{\pgfqpoint{0.532932in}{0.451389in}}{\pgfqpoint{4.383762in}{1.264399in}}%
\pgfusepath{clip}%
\pgfsetbuttcap%
\pgfsetmiterjoin%
\definecolor{currentfill}{rgb}{0.349020,0.490196,0.749020}%
\pgfsetfillcolor{currentfill}%
\pgfsetlinewidth{1.003750pt}%
\definecolor{currentstroke}{rgb}{1.000000,1.000000,1.000000}%
\pgfsetstrokecolor{currentstroke}%
\pgfsetdash{}{0pt}%
\pgfpathmoveto{\pgfqpoint{2.166879in}{0.451389in}}%
\pgfpathlineto{\pgfqpoint{2.326289in}{0.451389in}}%
\pgfpathlineto{\pgfqpoint{2.326289in}{0.995589in}}%
\pgfpathlineto{\pgfqpoint{2.166879in}{0.995589in}}%
\pgfpathlineto{\pgfqpoint{2.166879in}{0.451389in}}%
\pgfpathclose%
\pgfusepath{stroke,fill}%
\end{pgfscope}%
\begin{pgfscope}%
\pgfpathrectangle{\pgfqpoint{0.532932in}{0.451389in}}{\pgfqpoint{4.383762in}{1.264399in}}%
\pgfusepath{clip}%
\pgfsetbuttcap%
\pgfsetmiterjoin%
\definecolor{currentfill}{rgb}{0.349020,0.490196,0.749020}%
\pgfsetfillcolor{currentfill}%
\pgfsetlinewidth{1.003750pt}%
\definecolor{currentstroke}{rgb}{1.000000,1.000000,1.000000}%
\pgfsetstrokecolor{currentstroke}%
\pgfsetdash{}{0pt}%
\pgfpathmoveto{\pgfqpoint{2.565403in}{0.451389in}}%
\pgfpathlineto{\pgfqpoint{2.724813in}{0.451389in}}%
\pgfpathlineto{\pgfqpoint{2.724813in}{0.908684in}}%
\pgfpathlineto{\pgfqpoint{2.565403in}{0.908684in}}%
\pgfpathlineto{\pgfqpoint{2.565403in}{0.451389in}}%
\pgfpathclose%
\pgfusepath{stroke,fill}%
\end{pgfscope}%
\begin{pgfscope}%
\pgfpathrectangle{\pgfqpoint{0.532932in}{0.451389in}}{\pgfqpoint{4.383762in}{1.264399in}}%
\pgfusepath{clip}%
\pgfsetbuttcap%
\pgfsetmiterjoin%
\definecolor{currentfill}{rgb}{0.349020,0.490196,0.749020}%
\pgfsetfillcolor{currentfill}%
\pgfsetlinewidth{1.003750pt}%
\definecolor{currentstroke}{rgb}{1.000000,1.000000,1.000000}%
\pgfsetstrokecolor{currentstroke}%
\pgfsetdash{}{0pt}%
\pgfpathmoveto{\pgfqpoint{2.963927in}{0.451389in}}%
\pgfpathlineto{\pgfqpoint{3.123336in}{0.451389in}}%
\pgfpathlineto{\pgfqpoint{3.123336in}{0.578894in}}%
\pgfpathlineto{\pgfqpoint{2.963927in}{0.578894in}}%
\pgfpathlineto{\pgfqpoint{2.963927in}{0.451389in}}%
\pgfpathclose%
\pgfusepath{stroke,fill}%
\end{pgfscope}%
\begin{pgfscope}%
\pgfpathrectangle{\pgfqpoint{0.532932in}{0.451389in}}{\pgfqpoint{4.383762in}{1.264399in}}%
\pgfusepath{clip}%
\pgfsetbuttcap%
\pgfsetmiterjoin%
\definecolor{currentfill}{rgb}{0.349020,0.490196,0.749020}%
\pgfsetfillcolor{currentfill}%
\pgfsetlinewidth{1.003750pt}%
\definecolor{currentstroke}{rgb}{1.000000,1.000000,1.000000}%
\pgfsetstrokecolor{currentstroke}%
\pgfsetdash{}{0pt}%
\pgfpathmoveto{\pgfqpoint{3.362451in}{0.451389in}}%
\pgfpathlineto{\pgfqpoint{3.521860in}{0.451389in}}%
\pgfpathlineto{\pgfqpoint{3.521860in}{0.916688in}}%
\pgfpathlineto{\pgfqpoint{3.362451in}{0.916688in}}%
\pgfpathlineto{\pgfqpoint{3.362451in}{0.451389in}}%
\pgfpathclose%
\pgfusepath{stroke,fill}%
\end{pgfscope}%
\begin{pgfscope}%
\pgfpathrectangle{\pgfqpoint{0.532932in}{0.451389in}}{\pgfqpoint{4.383762in}{1.264399in}}%
\pgfusepath{clip}%
\pgfsetbuttcap%
\pgfsetmiterjoin%
\definecolor{currentfill}{rgb}{0.349020,0.490196,0.749020}%
\pgfsetfillcolor{currentfill}%
\pgfsetlinewidth{1.003750pt}%
\definecolor{currentstroke}{rgb}{1.000000,1.000000,1.000000}%
\pgfsetstrokecolor{currentstroke}%
\pgfsetdash{}{0pt}%
\pgfpathmoveto{\pgfqpoint{3.760974in}{0.451389in}}%
\pgfpathlineto{\pgfqpoint{3.920384in}{0.451389in}}%
\pgfpathlineto{\pgfqpoint{3.920384in}{0.646408in}}%
\pgfpathlineto{\pgfqpoint{3.760974in}{0.646408in}}%
\pgfpathlineto{\pgfqpoint{3.760974in}{0.451389in}}%
\pgfpathclose%
\pgfusepath{stroke,fill}%
\end{pgfscope}%
\begin{pgfscope}%
\pgfpathrectangle{\pgfqpoint{0.532932in}{0.451389in}}{\pgfqpoint{4.383762in}{1.264399in}}%
\pgfusepath{clip}%
\pgfsetbuttcap%
\pgfsetmiterjoin%
\definecolor{currentfill}{rgb}{0.349020,0.490196,0.749020}%
\pgfsetfillcolor{currentfill}%
\pgfsetlinewidth{1.003750pt}%
\definecolor{currentstroke}{rgb}{1.000000,1.000000,1.000000}%
\pgfsetstrokecolor{currentstroke}%
\pgfsetdash{}{0pt}%
\pgfpathmoveto{\pgfqpoint{4.159498in}{0.451389in}}%
\pgfpathlineto{\pgfqpoint{4.318908in}{0.451389in}}%
\pgfpathlineto{\pgfqpoint{4.318908in}{1.148726in}}%
\pgfpathlineto{\pgfqpoint{4.159498in}{1.148726in}}%
\pgfpathlineto{\pgfqpoint{4.159498in}{0.451389in}}%
\pgfpathclose%
\pgfusepath{stroke,fill}%
\end{pgfscope}%
\begin{pgfscope}%
\pgfpathrectangle{\pgfqpoint{0.532932in}{0.451389in}}{\pgfqpoint{4.383762in}{1.264399in}}%
\pgfusepath{clip}%
\pgfsetbuttcap%
\pgfsetmiterjoin%
\definecolor{currentfill}{rgb}{0.349020,0.490196,0.749020}%
\pgfsetfillcolor{currentfill}%
\pgfsetlinewidth{1.003750pt}%
\definecolor{currentstroke}{rgb}{1.000000,1.000000,1.000000}%
\pgfsetstrokecolor{currentstroke}%
\pgfsetdash{}{0pt}%
\pgfpathmoveto{\pgfqpoint{4.558022in}{0.451389in}}%
\pgfpathlineto{\pgfqpoint{4.717432in}{0.451389in}}%
\pgfpathlineto{\pgfqpoint{4.717432in}{0.744273in}}%
\pgfpathlineto{\pgfqpoint{4.558022in}{0.744273in}}%
\pgfpathlineto{\pgfqpoint{4.558022in}{0.451389in}}%
\pgfpathclose%
\pgfusepath{stroke,fill}%
\end{pgfscope}%
\begin{pgfscope}%
\pgfpathrectangle{\pgfqpoint{0.532932in}{0.451389in}}{\pgfqpoint{4.383762in}{1.264399in}}%
\pgfusepath{clip}%
\pgfsetbuttcap%
\pgfsetmiterjoin%
\definecolor{currentfill}{rgb}{0.852941,0.544118,0.370588}%
\pgfsetfillcolor{currentfill}%
\pgfsetlinewidth{1.003750pt}%
\definecolor{currentstroke}{rgb}{1.000000,1.000000,1.000000}%
\pgfsetstrokecolor{currentstroke}%
\pgfsetdash{}{0pt}%
\pgfpathmoveto{\pgfqpoint{0.732194in}{0.451389in}}%
\pgfpathlineto{\pgfqpoint{0.891603in}{0.451389in}}%
\pgfpathlineto{\pgfqpoint{0.891603in}{0.517848in}}%
\pgfpathlineto{\pgfqpoint{0.732194in}{0.517848in}}%
\pgfpathlineto{\pgfqpoint{0.732194in}{0.451389in}}%
\pgfpathclose%
\pgfusepath{stroke,fill}%
\end{pgfscope}%
\begin{pgfscope}%
\pgfpathrectangle{\pgfqpoint{0.532932in}{0.451389in}}{\pgfqpoint{4.383762in}{1.264399in}}%
\pgfusepath{clip}%
\pgfsetbuttcap%
\pgfsetmiterjoin%
\definecolor{currentfill}{rgb}{0.852941,0.544118,0.370588}%
\pgfsetfillcolor{currentfill}%
\pgfsetlinewidth{1.003750pt}%
\definecolor{currentstroke}{rgb}{1.000000,1.000000,1.000000}%
\pgfsetstrokecolor{currentstroke}%
\pgfsetdash{}{0pt}%
\pgfpathmoveto{\pgfqpoint{1.130717in}{0.451389in}}%
\pgfpathlineto{\pgfqpoint{1.290127in}{0.451389in}}%
\pgfpathlineto{\pgfqpoint{1.290127in}{0.570385in}}%
\pgfpathlineto{\pgfqpoint{1.130717in}{0.570385in}}%
\pgfpathlineto{\pgfqpoint{1.130717in}{0.451389in}}%
\pgfpathclose%
\pgfusepath{stroke,fill}%
\end{pgfscope}%
\begin{pgfscope}%
\pgfpathrectangle{\pgfqpoint{0.532932in}{0.451389in}}{\pgfqpoint{4.383762in}{1.264399in}}%
\pgfusepath{clip}%
\pgfsetbuttcap%
\pgfsetmiterjoin%
\definecolor{currentfill}{rgb}{0.852941,0.544118,0.370588}%
\pgfsetfillcolor{currentfill}%
\pgfsetlinewidth{1.003750pt}%
\definecolor{currentstroke}{rgb}{1.000000,1.000000,1.000000}%
\pgfsetstrokecolor{currentstroke}%
\pgfsetdash{}{0pt}%
\pgfpathmoveto{\pgfqpoint{1.529241in}{0.451389in}}%
\pgfpathlineto{\pgfqpoint{1.688651in}{0.451389in}}%
\pgfpathlineto{\pgfqpoint{1.688651in}{0.544182in}}%
\pgfpathlineto{\pgfqpoint{1.529241in}{0.544182in}}%
\pgfpathlineto{\pgfqpoint{1.529241in}{0.451389in}}%
\pgfpathclose%
\pgfusepath{stroke,fill}%
\end{pgfscope}%
\begin{pgfscope}%
\pgfpathrectangle{\pgfqpoint{0.532932in}{0.451389in}}{\pgfqpoint{4.383762in}{1.264399in}}%
\pgfusepath{clip}%
\pgfsetbuttcap%
\pgfsetmiterjoin%
\definecolor{currentfill}{rgb}{0.852941,0.544118,0.370588}%
\pgfsetfillcolor{currentfill}%
\pgfsetlinewidth{1.003750pt}%
\definecolor{currentstroke}{rgb}{1.000000,1.000000,1.000000}%
\pgfsetstrokecolor{currentstroke}%
\pgfsetdash{}{0pt}%
\pgfpathmoveto{\pgfqpoint{1.927765in}{0.451389in}}%
\pgfpathlineto{\pgfqpoint{2.087175in}{0.451389in}}%
\pgfpathlineto{\pgfqpoint{2.087175in}{0.498410in}}%
\pgfpathlineto{\pgfqpoint{1.927765in}{0.498410in}}%
\pgfpathlineto{\pgfqpoint{1.927765in}{0.451389in}}%
\pgfpathclose%
\pgfusepath{stroke,fill}%
\end{pgfscope}%
\begin{pgfscope}%
\pgfpathrectangle{\pgfqpoint{0.532932in}{0.451389in}}{\pgfqpoint{4.383762in}{1.264399in}}%
\pgfusepath{clip}%
\pgfsetbuttcap%
\pgfsetmiterjoin%
\definecolor{currentfill}{rgb}{0.852941,0.544118,0.370588}%
\pgfsetfillcolor{currentfill}%
\pgfsetlinewidth{1.003750pt}%
\definecolor{currentstroke}{rgb}{1.000000,1.000000,1.000000}%
\pgfsetstrokecolor{currentstroke}%
\pgfsetdash{}{0pt}%
\pgfpathmoveto{\pgfqpoint{2.326289in}{0.451389in}}%
\pgfpathlineto{\pgfqpoint{2.485698in}{0.451389in}}%
\pgfpathlineto{\pgfqpoint{2.485698in}{1.369560in}}%
\pgfpathlineto{\pgfqpoint{2.326289in}{1.369560in}}%
\pgfpathlineto{\pgfqpoint{2.326289in}{0.451389in}}%
\pgfpathclose%
\pgfusepath{stroke,fill}%
\end{pgfscope}%
\begin{pgfscope}%
\pgfpathrectangle{\pgfqpoint{0.532932in}{0.451389in}}{\pgfqpoint{4.383762in}{1.264399in}}%
\pgfusepath{clip}%
\pgfsetbuttcap%
\pgfsetmiterjoin%
\definecolor{currentfill}{rgb}{0.852941,0.544118,0.370588}%
\pgfsetfillcolor{currentfill}%
\pgfsetlinewidth{1.003750pt}%
\definecolor{currentstroke}{rgb}{1.000000,1.000000,1.000000}%
\pgfsetstrokecolor{currentstroke}%
\pgfsetdash{}{0pt}%
\pgfpathmoveto{\pgfqpoint{2.724813in}{0.451389in}}%
\pgfpathlineto{\pgfqpoint{2.884222in}{0.451389in}}%
\pgfpathlineto{\pgfqpoint{2.884222in}{0.619132in}}%
\pgfpathlineto{\pgfqpoint{2.724813in}{0.619132in}}%
\pgfpathlineto{\pgfqpoint{2.724813in}{0.451389in}}%
\pgfpathclose%
\pgfusepath{stroke,fill}%
\end{pgfscope}%
\begin{pgfscope}%
\pgfpathrectangle{\pgfqpoint{0.532932in}{0.451389in}}{\pgfqpoint{4.383762in}{1.264399in}}%
\pgfusepath{clip}%
\pgfsetbuttcap%
\pgfsetmiterjoin%
\definecolor{currentfill}{rgb}{0.852941,0.544118,0.370588}%
\pgfsetfillcolor{currentfill}%
\pgfsetlinewidth{1.003750pt}%
\definecolor{currentstroke}{rgb}{1.000000,1.000000,1.000000}%
\pgfsetstrokecolor{currentstroke}%
\pgfsetdash{}{0pt}%
\pgfpathmoveto{\pgfqpoint{3.123336in}{0.451389in}}%
\pgfpathlineto{\pgfqpoint{3.282746in}{0.451389in}}%
\pgfpathlineto{\pgfqpoint{3.282746in}{0.619132in}}%
\pgfpathlineto{\pgfqpoint{3.123336in}{0.619132in}}%
\pgfpathlineto{\pgfqpoint{3.123336in}{0.451389in}}%
\pgfpathclose%
\pgfusepath{stroke,fill}%
\end{pgfscope}%
\begin{pgfscope}%
\pgfpathrectangle{\pgfqpoint{0.532932in}{0.451389in}}{\pgfqpoint{4.383762in}{1.264399in}}%
\pgfusepath{clip}%
\pgfsetbuttcap%
\pgfsetmiterjoin%
\definecolor{currentfill}{rgb}{0.852941,0.544118,0.370588}%
\pgfsetfillcolor{currentfill}%
\pgfsetlinewidth{1.003750pt}%
\definecolor{currentstroke}{rgb}{1.000000,1.000000,1.000000}%
\pgfsetstrokecolor{currentstroke}%
\pgfsetdash{}{0pt}%
\pgfpathmoveto{\pgfqpoint{3.521860in}{0.451389in}}%
\pgfpathlineto{\pgfqpoint{3.681270in}{0.451389in}}%
\pgfpathlineto{\pgfqpoint{3.681270in}{1.140506in}}%
\pgfpathlineto{\pgfqpoint{3.521860in}{1.140506in}}%
\pgfpathlineto{\pgfqpoint{3.521860in}{0.451389in}}%
\pgfpathclose%
\pgfusepath{stroke,fill}%
\end{pgfscope}%
\begin{pgfscope}%
\pgfpathrectangle{\pgfqpoint{0.532932in}{0.451389in}}{\pgfqpoint{4.383762in}{1.264399in}}%
\pgfusepath{clip}%
\pgfsetbuttcap%
\pgfsetmiterjoin%
\definecolor{currentfill}{rgb}{0.852941,0.544118,0.370588}%
\pgfsetfillcolor{currentfill}%
\pgfsetlinewidth{1.003750pt}%
\definecolor{currentstroke}{rgb}{1.000000,1.000000,1.000000}%
\pgfsetstrokecolor{currentstroke}%
\pgfsetdash{}{0pt}%
\pgfpathmoveto{\pgfqpoint{3.920384in}{0.451389in}}%
\pgfpathlineto{\pgfqpoint{4.079794in}{0.451389in}}%
\pgfpathlineto{\pgfqpoint{4.079794in}{0.564899in}}%
\pgfpathlineto{\pgfqpoint{3.920384in}{0.564899in}}%
\pgfpathlineto{\pgfqpoint{3.920384in}{0.451389in}}%
\pgfpathclose%
\pgfusepath{stroke,fill}%
\end{pgfscope}%
\begin{pgfscope}%
\pgfpathrectangle{\pgfqpoint{0.532932in}{0.451389in}}{\pgfqpoint{4.383762in}{1.264399in}}%
\pgfusepath{clip}%
\pgfsetbuttcap%
\pgfsetmiterjoin%
\definecolor{currentfill}{rgb}{0.852941,0.544118,0.370588}%
\pgfsetfillcolor{currentfill}%
\pgfsetlinewidth{1.003750pt}%
\definecolor{currentstroke}{rgb}{1.000000,1.000000,1.000000}%
\pgfsetstrokecolor{currentstroke}%
\pgfsetdash{}{0pt}%
\pgfpathmoveto{\pgfqpoint{4.318908in}{0.451389in}}%
\pgfpathlineto{\pgfqpoint{4.478317in}{0.451389in}}%
\pgfpathlineto{\pgfqpoint{4.478317in}{1.655578in}}%
\pgfpathlineto{\pgfqpoint{4.318908in}{1.655578in}}%
\pgfpathlineto{\pgfqpoint{4.318908in}{0.451389in}}%
\pgfpathclose%
\pgfusepath{stroke,fill}%
\end{pgfscope}%
\begin{pgfscope}%
\pgfpathrectangle{\pgfqpoint{0.532932in}{0.451389in}}{\pgfqpoint{4.383762in}{1.264399in}}%
\pgfusepath{clip}%
\pgfsetbuttcap%
\pgfsetmiterjoin%
\definecolor{currentfill}{rgb}{0.852941,0.544118,0.370588}%
\pgfsetfillcolor{currentfill}%
\pgfsetlinewidth{1.003750pt}%
\definecolor{currentstroke}{rgb}{1.000000,1.000000,1.000000}%
\pgfsetstrokecolor{currentstroke}%
\pgfsetdash{}{0pt}%
\pgfpathmoveto{\pgfqpoint{4.717432in}{0.451389in}}%
\pgfpathlineto{\pgfqpoint{4.876841in}{0.451389in}}%
\pgfpathlineto{\pgfqpoint{4.876841in}{0.809963in}}%
\pgfpathlineto{\pgfqpoint{4.717432in}{0.809963in}}%
\pgfpathlineto{\pgfqpoint{4.717432in}{0.451389in}}%
\pgfpathclose%
\pgfusepath{stroke,fill}%
\end{pgfscope}%
\begin{pgfscope}%
\pgfsetrectcap%
\pgfsetmiterjoin%
\pgfsetlinewidth{1.254687pt}%
\definecolor{currentstroke}{rgb}{0.800000,0.800000,0.800000}%
\pgfsetstrokecolor{currentstroke}%
\pgfsetdash{}{0pt}%
\pgfpathmoveto{\pgfqpoint{0.532932in}{0.451389in}}%
\pgfpathlineto{\pgfqpoint{0.532932in}{1.715788in}}%
\pgfusepath{stroke}%
\end{pgfscope}%
\begin{pgfscope}%
\pgfsetrectcap%
\pgfsetmiterjoin%
\pgfsetlinewidth{1.254687pt}%
\definecolor{currentstroke}{rgb}{0.800000,0.800000,0.800000}%
\pgfsetstrokecolor{currentstroke}%
\pgfsetdash{}{0pt}%
\pgfpathmoveto{\pgfqpoint{4.916693in}{0.451389in}}%
\pgfpathlineto{\pgfqpoint{4.916693in}{1.715788in}}%
\pgfusepath{stroke}%
\end{pgfscope}%
\begin{pgfscope}%
\pgfsetrectcap%
\pgfsetmiterjoin%
\pgfsetlinewidth{1.254687pt}%
\definecolor{currentstroke}{rgb}{0.800000,0.800000,0.800000}%
\pgfsetstrokecolor{currentstroke}%
\pgfsetdash{}{0pt}%
\pgfpathmoveto{\pgfqpoint{0.532932in}{0.451389in}}%
\pgfpathlineto{\pgfqpoint{4.916693in}{0.451389in}}%
\pgfusepath{stroke}%
\end{pgfscope}%
\begin{pgfscope}%
\pgfsetrectcap%
\pgfsetmiterjoin%
\pgfsetlinewidth{1.254687pt}%
\definecolor{currentstroke}{rgb}{0.800000,0.800000,0.800000}%
\pgfsetstrokecolor{currentstroke}%
\pgfsetdash{}{0pt}%
\pgfpathmoveto{\pgfqpoint{0.532932in}{1.715788in}}%
\pgfpathlineto{\pgfqpoint{4.916693in}{1.715788in}}%
\pgfusepath{stroke}%
\end{pgfscope}%
\begin{pgfscope}%
\pgfsetbuttcap%
\pgfsetmiterjoin%
\definecolor{currentfill}{rgb}{1.000000,1.000000,1.000000}%
\pgfsetfillcolor{currentfill}%
\pgfsetfillopacity{0.800000}%
\pgfsetlinewidth{1.003750pt}%
\definecolor{currentstroke}{rgb}{0.800000,0.800000,0.800000}%
\pgfsetstrokecolor{currentstroke}%
\pgfsetstrokeopacity{0.800000}%
\pgfsetdash{}{0pt}%
\pgfpathmoveto{\pgfqpoint{5.113788in}{0.589350in}}%
\pgfpathlineto{\pgfqpoint{6.059564in}{0.589350in}}%
\pgfpathquadraticcurveto{\pgfqpoint{6.084564in}{0.589350in}}{\pgfqpoint{6.084564in}{0.614350in}}%
\pgfpathlineto{\pgfqpoint{6.084564in}{1.552826in}}%
\pgfpathquadraticcurveto{\pgfqpoint{6.084564in}{1.577826in}}{\pgfqpoint{6.059564in}{1.577826in}}%
\pgfpathlineto{\pgfqpoint{5.113788in}{1.577826in}}%
\pgfpathquadraticcurveto{\pgfqpoint{5.088788in}{1.577826in}}{\pgfqpoint{5.088788in}{1.552826in}}%
\pgfpathlineto{\pgfqpoint{5.088788in}{0.614350in}}%
\pgfpathquadraticcurveto{\pgfqpoint{5.088788in}{0.589350in}}{\pgfqpoint{5.113788in}{0.589350in}}%
\pgfpathlineto{\pgfqpoint{5.113788in}{0.589350in}}%
\pgfpathclose%
\pgfusepath{stroke,fill}%
\end{pgfscope}%
\begin{pgfscope}%
\pgfsetbuttcap%
\pgfsetmiterjoin%
\definecolor{currentfill}{rgb}{0.349020,0.490196,0.749020}%
\pgfsetfillcolor{currentfill}%
\pgfsetlinewidth{1.003750pt}%
\definecolor{currentstroke}{rgb}{1.000000,1.000000,1.000000}%
\pgfsetstrokecolor{currentstroke}%
\pgfsetdash{}{0pt}%
\pgfpathmoveto{\pgfqpoint{5.138788in}{1.273847in}}%
\pgfpathlineto{\pgfqpoint{5.388788in}{1.273847in}}%
\pgfpathlineto{\pgfqpoint{5.388788in}{1.361347in}}%
\pgfpathlineto{\pgfqpoint{5.138788in}{1.361347in}}%
\pgfpathlineto{\pgfqpoint{5.138788in}{1.273847in}}%
\pgfpathclose%
\pgfusepath{stroke,fill}%
\end{pgfscope}%
\begin{pgfscope}%
\definecolor{textcolor}{rgb}{0.150000,0.150000,0.150000}%
\pgfsetstrokecolor{textcolor}%
\pgfsetfillcolor{textcolor}%
\pgftext[x=5.488788in, y=1.432855in, left, base]{\color{textcolor}\sffamily\fontsize{9.000000}{10.800000}\selectfont Hybrid}%
\end{pgfscope}%
\begin{pgfscope}%
\definecolor{textcolor}{rgb}{0.150000,0.150000,0.150000}%
\pgfsetstrokecolor{textcolor}%
\pgfsetfillcolor{textcolor}%
\pgftext[x=5.488788in, y=1.288861in, left, base]{\color{textcolor}\sffamily\fontsize{9.000000}{10.800000}\selectfont routing}%
\end{pgfscope}%
\begin{pgfscope}%
\definecolor{textcolor}{rgb}{0.150000,0.150000,0.150000}%
\pgfsetstrokecolor{textcolor}%
\pgfsetfillcolor{textcolor}%
\pgftext[x=5.488788in, y=1.144867in, left, base]{\color{textcolor}\sffamily\fontsize{9.000000}{10.800000}\selectfont algorithm}%
\end{pgfscope}%
\begin{pgfscope}%
\pgfsetbuttcap%
\pgfsetmiterjoin%
\definecolor{currentfill}{rgb}{0.852941,0.544118,0.370588}%
\pgfsetfillcolor{currentfill}%
\pgfsetlinewidth{1.003750pt}%
\definecolor{currentstroke}{rgb}{1.000000,1.000000,1.000000}%
\pgfsetstrokecolor{currentstroke}%
\pgfsetdash{}{0pt}%
\pgfpathmoveto{\pgfqpoint{5.138788in}{0.798359in}}%
\pgfpathlineto{\pgfqpoint{5.388788in}{0.798359in}}%
\pgfpathlineto{\pgfqpoint{5.388788in}{0.885859in}}%
\pgfpathlineto{\pgfqpoint{5.138788in}{0.885859in}}%
\pgfpathlineto{\pgfqpoint{5.138788in}{0.798359in}}%
\pgfpathclose%
\pgfusepath{stroke,fill}%
\end{pgfscope}%
\begin{pgfscope}%
\definecolor{textcolor}{rgb}{0.150000,0.150000,0.150000}%
\pgfsetstrokecolor{textcolor}%
\pgfsetfillcolor{textcolor}%
\pgftext[x=5.488788in, y=0.957367in, left, base]{\color{textcolor}\sffamily\fontsize{9.000000}{10.800000}\selectfont Graph-}%
\end{pgfscope}%
\begin{pgfscope}%
\definecolor{textcolor}{rgb}{0.150000,0.150000,0.150000}%
\pgfsetstrokecolor{textcolor}%
\pgfsetfillcolor{textcolor}%
\pgftext[x=5.488788in, y=0.813373in, left, base]{\color{textcolor}\sffamily\fontsize{9.000000}{10.800000}\selectfont based}%
\end{pgfscope}%
\begin{pgfscope}%
\definecolor{textcolor}{rgb}{0.150000,0.150000,0.150000}%
\pgfsetstrokecolor{textcolor}%
\pgfsetfillcolor{textcolor}%
\pgftext[x=5.488788in, y=0.669379in, left, base]{\color{textcolor}\sffamily\fontsize{9.000000}{10.800000}\selectfont routing}%
\end{pgfscope}%
\end{pgfpicture}%
\makeatother%
\endgroup%

					\end{figcenter}
					\caption{\enquote{OSM city} dataset.}
					\label{fig:eval-hausdorff-city}
				\end{subfigure}
				\\[3ex]
				\begin{subfigure}[t]{\linewidth}
					\begin{figcenter}
						%% Creator: Matplotlib, PGF backend
%%
%% To include the figure in your LaTeX document, write
%%   \input{<filename>.pgf}
%%
%% Make sure the required packages are loaded in your preamble
%%   \usepackage{pgf}
%%
%% Also ensure that all the required font packages are loaded; for instance,
%% the lmodern package is sometimes necessary when using math font.
%%   \usepackage{lmodern}
%%
%% Figures using additional raster images can only be included by \input if
%% they are in the same directory as the main LaTeX file. For loading figures
%% from other directories you can use the `import` package
%%   \usepackage{import}
%%
%% and then include the figures with
%%   \import{<path to file>}{<filename>.pgf}
%%
%% Matplotlib used the following preamble
%%   
%%   \usepackage{fontspec}
%%   \setmainfont{DejaVuSerif.ttf}[Path=\detokenize{/home/hauke/.local/lib/python3.11/site-packages/matplotlib/mpl-data/fonts/ttf/}]
%%   \setsansfont{DroidSans.ttf}[Path=\detokenize{/usr/share/fonts/droid/}]
%%   \setmonofont{DejaVuSansMono.ttf}[Path=\detokenize{/home/hauke/.local/lib/python3.11/site-packages/matplotlib/mpl-data/fonts/ttf/}]
%%   \makeatletter\@ifpackageloaded{underscore}{}{\usepackage[strings]{underscore}}\makeatother
%%
\begingroup%
\makeatletter%
\begin{pgfpicture}%
\pgfpathrectangle{\pgfpointorigin}{\pgfqpoint{6.083523in}{1.715788in}}%
\pgfusepath{use as bounding box, clip}%
\begin{pgfscope}%
\pgfsetbuttcap%
\pgfsetmiterjoin%
\definecolor{currentfill}{rgb}{1.000000,1.000000,1.000000}%
\pgfsetfillcolor{currentfill}%
\pgfsetlinewidth{0.000000pt}%
\definecolor{currentstroke}{rgb}{1.000000,1.000000,1.000000}%
\pgfsetstrokecolor{currentstroke}%
\pgfsetdash{}{0pt}%
\pgfpathmoveto{\pgfqpoint{0.000000in}{0.000000in}}%
\pgfpathlineto{\pgfqpoint{6.083523in}{0.000000in}}%
\pgfpathlineto{\pgfqpoint{6.083523in}{1.715788in}}%
\pgfpathlineto{\pgfqpoint{0.000000in}{1.715788in}}%
\pgfpathlineto{\pgfqpoint{0.000000in}{0.000000in}}%
\pgfpathclose%
\pgfusepath{fill}%
\end{pgfscope}%
\begin{pgfscope}%
\pgfsetbuttcap%
\pgfsetmiterjoin%
\definecolor{currentfill}{rgb}{1.000000,1.000000,1.000000}%
\pgfsetfillcolor{currentfill}%
\pgfsetlinewidth{0.000000pt}%
\definecolor{currentstroke}{rgb}{0.000000,0.000000,0.000000}%
\pgfsetstrokecolor{currentstroke}%
\pgfsetstrokeopacity{0.000000}%
\pgfsetdash{}{0pt}%
\pgfpathmoveto{\pgfqpoint{0.532932in}{0.451389in}}%
\pgfpathlineto{\pgfqpoint{4.915678in}{0.451389in}}%
\pgfpathlineto{\pgfqpoint{4.915678in}{1.715788in}}%
\pgfpathlineto{\pgfqpoint{0.532932in}{1.715788in}}%
\pgfpathlineto{\pgfqpoint{0.532932in}{0.451389in}}%
\pgfpathclose%
\pgfusepath{fill}%
\end{pgfscope}%
\begin{pgfscope}%
\definecolor{textcolor}{rgb}{0.150000,0.150000,0.150000}%
\pgfsetstrokecolor{textcolor}%
\pgfsetfillcolor{textcolor}%
\pgftext[x=0.732148in,y=0.319444in,,top]{\color{textcolor}\sffamily\fontsize{9.000000}{10.800000}\selectfont 1}%
\end{pgfscope}%
\begin{pgfscope}%
\definecolor{textcolor}{rgb}{0.150000,0.150000,0.150000}%
\pgfsetstrokecolor{textcolor}%
\pgfsetfillcolor{textcolor}%
\pgftext[x=1.130579in,y=0.319444in,,top]{\color{textcolor}\sffamily\fontsize{9.000000}{10.800000}\selectfont 2}%
\end{pgfscope}%
\begin{pgfscope}%
\definecolor{textcolor}{rgb}{0.150000,0.150000,0.150000}%
\pgfsetstrokecolor{textcolor}%
\pgfsetfillcolor{textcolor}%
\pgftext[x=1.529010in,y=0.319444in,,top]{\color{textcolor}\sffamily\fontsize{9.000000}{10.800000}\selectfont 3}%
\end{pgfscope}%
\begin{pgfscope}%
\definecolor{textcolor}{rgb}{0.150000,0.150000,0.150000}%
\pgfsetstrokecolor{textcolor}%
\pgfsetfillcolor{textcolor}%
\pgftext[x=1.927442in,y=0.319444in,,top]{\color{textcolor}\sffamily\fontsize{9.000000}{10.800000}\selectfont 4}%
\end{pgfscope}%
\begin{pgfscope}%
\definecolor{textcolor}{rgb}{0.150000,0.150000,0.150000}%
\pgfsetstrokecolor{textcolor}%
\pgfsetfillcolor{textcolor}%
\pgftext[x=2.325873in,y=0.319444in,,top]{\color{textcolor}\sffamily\fontsize{9.000000}{10.800000}\selectfont 5}%
\end{pgfscope}%
\begin{pgfscope}%
\definecolor{textcolor}{rgb}{0.150000,0.150000,0.150000}%
\pgfsetstrokecolor{textcolor}%
\pgfsetfillcolor{textcolor}%
\pgftext[x=2.724305in,y=0.319444in,,top]{\color{textcolor}\sffamily\fontsize{9.000000}{10.800000}\selectfont 6}%
\end{pgfscope}%
\begin{pgfscope}%
\definecolor{textcolor}{rgb}{0.150000,0.150000,0.150000}%
\pgfsetstrokecolor{textcolor}%
\pgfsetfillcolor{textcolor}%
\pgftext[x=3.122736in,y=0.319444in,,top]{\color{textcolor}\sffamily\fontsize{9.000000}{10.800000}\selectfont 7}%
\end{pgfscope}%
\begin{pgfscope}%
\definecolor{textcolor}{rgb}{0.150000,0.150000,0.150000}%
\pgfsetstrokecolor{textcolor}%
\pgfsetfillcolor{textcolor}%
\pgftext[x=3.521168in,y=0.319444in,,top]{\color{textcolor}\sffamily\fontsize{9.000000}{10.800000}\selectfont 8}%
\end{pgfscope}%
\begin{pgfscope}%
\definecolor{textcolor}{rgb}{0.150000,0.150000,0.150000}%
\pgfsetstrokecolor{textcolor}%
\pgfsetfillcolor{textcolor}%
\pgftext[x=3.919599in,y=0.319444in,,top]{\color{textcolor}\sffamily\fontsize{9.000000}{10.800000}\selectfont 9}%
\end{pgfscope}%
\begin{pgfscope}%
\definecolor{textcolor}{rgb}{0.150000,0.150000,0.150000}%
\pgfsetstrokecolor{textcolor}%
\pgfsetfillcolor{textcolor}%
\pgftext[x=4.318031in,y=0.319444in,,top]{\color{textcolor}\sffamily\fontsize{9.000000}{10.800000}\selectfont 10}%
\end{pgfscope}%
\begin{pgfscope}%
\definecolor{textcolor}{rgb}{0.150000,0.150000,0.150000}%
\pgfsetstrokecolor{textcolor}%
\pgfsetfillcolor{textcolor}%
\pgftext[x=4.716462in,y=0.319444in,,top]{\color{textcolor}\sffamily\fontsize{9.000000}{10.800000}\selectfont mean}%
\end{pgfscope}%
\begin{pgfscope}%
\definecolor{textcolor}{rgb}{0.150000,0.150000,0.150000}%
\pgfsetstrokecolor{textcolor}%
\pgfsetfillcolor{textcolor}%
\pgftext[x=2.724305in,y=0.125000in,,top]{\color{textcolor}\sffamily\fontsize{9.000000}{10.800000}\selectfont Routing request}%
\end{pgfscope}%
\begin{pgfscope}%
\pgfpathrectangle{\pgfqpoint{0.532932in}{0.451389in}}{\pgfqpoint{4.382746in}{1.264399in}}%
\pgfusepath{clip}%
\pgfsetroundcap%
\pgfsetroundjoin%
\pgfsetlinewidth{1.003750pt}%
\definecolor{currentstroke}{rgb}{0.800000,0.800000,0.800000}%
\pgfsetstrokecolor{currentstroke}%
\pgfsetdash{}{0pt}%
\pgfpathmoveto{\pgfqpoint{0.532932in}{0.451389in}}%
\pgfpathlineto{\pgfqpoint{4.915678in}{0.451389in}}%
\pgfusepath{stroke}%
\end{pgfscope}%
\begin{pgfscope}%
\definecolor{textcolor}{rgb}{0.150000,0.150000,0.150000}%
\pgfsetstrokecolor{textcolor}%
\pgfsetfillcolor{textcolor}%
\pgftext[x=0.332140in, y=0.403903in, left, base]{\color{textcolor}\sffamily\fontsize{9.000000}{10.800000}\selectfont 0}%
\end{pgfscope}%
\begin{pgfscope}%
\pgfpathrectangle{\pgfqpoint{0.532932in}{0.451389in}}{\pgfqpoint{4.382746in}{1.264399in}}%
\pgfusepath{clip}%
\pgfsetroundcap%
\pgfsetroundjoin%
\pgfsetlinewidth{1.003750pt}%
\definecolor{currentstroke}{rgb}{0.800000,0.800000,0.800000}%
\pgfsetstrokecolor{currentstroke}%
\pgfsetdash{}{0pt}%
\pgfpathmoveto{\pgfqpoint{0.532932in}{0.789476in}}%
\pgfpathlineto{\pgfqpoint{4.915678in}{0.789476in}}%
\pgfusepath{stroke}%
\end{pgfscope}%
\begin{pgfscope}%
\definecolor{textcolor}{rgb}{0.150000,0.150000,0.150000}%
\pgfsetstrokecolor{textcolor}%
\pgfsetfillcolor{textcolor}%
\pgftext[x=0.194444in, y=0.741991in, left, base]{\color{textcolor}\sffamily\fontsize{9.000000}{10.800000}\selectfont 200}%
\end{pgfscope}%
\begin{pgfscope}%
\pgfpathrectangle{\pgfqpoint{0.532932in}{0.451389in}}{\pgfqpoint{4.382746in}{1.264399in}}%
\pgfusepath{clip}%
\pgfsetroundcap%
\pgfsetroundjoin%
\pgfsetlinewidth{1.003750pt}%
\definecolor{currentstroke}{rgb}{0.800000,0.800000,0.800000}%
\pgfsetstrokecolor{currentstroke}%
\pgfsetdash{}{0pt}%
\pgfpathmoveto{\pgfqpoint{0.532932in}{1.127564in}}%
\pgfpathlineto{\pgfqpoint{4.915678in}{1.127564in}}%
\pgfusepath{stroke}%
\end{pgfscope}%
\begin{pgfscope}%
\definecolor{textcolor}{rgb}{0.150000,0.150000,0.150000}%
\pgfsetstrokecolor{textcolor}%
\pgfsetfillcolor{textcolor}%
\pgftext[x=0.194444in, y=1.080078in, left, base]{\color{textcolor}\sffamily\fontsize{9.000000}{10.800000}\selectfont 400}%
\end{pgfscope}%
\begin{pgfscope}%
\pgfpathrectangle{\pgfqpoint{0.532932in}{0.451389in}}{\pgfqpoint{4.382746in}{1.264399in}}%
\pgfusepath{clip}%
\pgfsetroundcap%
\pgfsetroundjoin%
\pgfsetlinewidth{1.003750pt}%
\definecolor{currentstroke}{rgb}{0.800000,0.800000,0.800000}%
\pgfsetstrokecolor{currentstroke}%
\pgfsetdash{}{0pt}%
\pgfpathmoveto{\pgfqpoint{0.532932in}{1.465651in}}%
\pgfpathlineto{\pgfqpoint{4.915678in}{1.465651in}}%
\pgfusepath{stroke}%
\end{pgfscope}%
\begin{pgfscope}%
\definecolor{textcolor}{rgb}{0.150000,0.150000,0.150000}%
\pgfsetstrokecolor{textcolor}%
\pgfsetfillcolor{textcolor}%
\pgftext[x=0.194444in, y=1.418166in, left, base]{\color{textcolor}\sffamily\fontsize{9.000000}{10.800000}\selectfont 600}%
\end{pgfscope}%
\begin{pgfscope}%
\definecolor{textcolor}{rgb}{0.150000,0.150000,0.150000}%
\pgfsetstrokecolor{textcolor}%
\pgfsetfillcolor{textcolor}%
\pgftext[x=0.125000in,y=1.083588in,,bottom,rotate=90.000000]{\color{textcolor}\sffamily\fontsize{9.000000}{10.800000}\selectfont Hausdorff distance in m}%
\end{pgfscope}%
\begin{pgfscope}%
\pgfpathrectangle{\pgfqpoint{0.532932in}{0.451389in}}{\pgfqpoint{4.382746in}{1.264399in}}%
\pgfusepath{clip}%
\pgfsetbuttcap%
\pgfsetmiterjoin%
\definecolor{currentfill}{rgb}{0.349020,0.490196,0.749020}%
\pgfsetfillcolor{currentfill}%
\pgfsetlinewidth{1.003750pt}%
\definecolor{currentstroke}{rgb}{1.000000,1.000000,1.000000}%
\pgfsetstrokecolor{currentstroke}%
\pgfsetdash{}{0pt}%
\pgfpathmoveto{\pgfqpoint{0.572775in}{0.451389in}}%
\pgfpathlineto{\pgfqpoint{0.732148in}{0.451389in}}%
\pgfpathlineto{\pgfqpoint{0.732148in}{0.604791in}}%
\pgfpathlineto{\pgfqpoint{0.572775in}{0.604791in}}%
\pgfpathlineto{\pgfqpoint{0.572775in}{0.451389in}}%
\pgfpathclose%
\pgfusepath{stroke,fill}%
\end{pgfscope}%
\begin{pgfscope}%
\pgfpathrectangle{\pgfqpoint{0.532932in}{0.451389in}}{\pgfqpoint{4.382746in}{1.264399in}}%
\pgfusepath{clip}%
\pgfsetbuttcap%
\pgfsetmiterjoin%
\definecolor{currentfill}{rgb}{0.349020,0.490196,0.749020}%
\pgfsetfillcolor{currentfill}%
\pgfsetlinewidth{1.003750pt}%
\definecolor{currentstroke}{rgb}{1.000000,1.000000,1.000000}%
\pgfsetstrokecolor{currentstroke}%
\pgfsetdash{}{0pt}%
\pgfpathmoveto{\pgfqpoint{0.971206in}{0.451389in}}%
\pgfpathlineto{\pgfqpoint{1.130579in}{0.451389in}}%
\pgfpathlineto{\pgfqpoint{1.130579in}{0.519757in}}%
\pgfpathlineto{\pgfqpoint{0.971206in}{0.519757in}}%
\pgfpathlineto{\pgfqpoint{0.971206in}{0.451389in}}%
\pgfpathclose%
\pgfusepath{stroke,fill}%
\end{pgfscope}%
\begin{pgfscope}%
\pgfpathrectangle{\pgfqpoint{0.532932in}{0.451389in}}{\pgfqpoint{4.382746in}{1.264399in}}%
\pgfusepath{clip}%
\pgfsetbuttcap%
\pgfsetmiterjoin%
\definecolor{currentfill}{rgb}{0.349020,0.490196,0.749020}%
\pgfsetfillcolor{currentfill}%
\pgfsetlinewidth{1.003750pt}%
\definecolor{currentstroke}{rgb}{1.000000,1.000000,1.000000}%
\pgfsetstrokecolor{currentstroke}%
\pgfsetdash{}{0pt}%
\pgfpathmoveto{\pgfqpoint{1.369638in}{0.451389in}}%
\pgfpathlineto{\pgfqpoint{1.529010in}{0.451389in}}%
\pgfpathlineto{\pgfqpoint{1.529010in}{0.510907in}}%
\pgfpathlineto{\pgfqpoint{1.369638in}{0.510907in}}%
\pgfpathlineto{\pgfqpoint{1.369638in}{0.451389in}}%
\pgfpathclose%
\pgfusepath{stroke,fill}%
\end{pgfscope}%
\begin{pgfscope}%
\pgfpathrectangle{\pgfqpoint{0.532932in}{0.451389in}}{\pgfqpoint{4.382746in}{1.264399in}}%
\pgfusepath{clip}%
\pgfsetbuttcap%
\pgfsetmiterjoin%
\definecolor{currentfill}{rgb}{0.349020,0.490196,0.749020}%
\pgfsetfillcolor{currentfill}%
\pgfsetlinewidth{1.003750pt}%
\definecolor{currentstroke}{rgb}{1.000000,1.000000,1.000000}%
\pgfsetstrokecolor{currentstroke}%
\pgfsetdash{}{0pt}%
\pgfpathmoveto{\pgfqpoint{1.768069in}{0.451389in}}%
\pgfpathlineto{\pgfqpoint{1.927442in}{0.451389in}}%
\pgfpathlineto{\pgfqpoint{1.927442in}{0.795808in}}%
\pgfpathlineto{\pgfqpoint{1.768069in}{0.795808in}}%
\pgfpathlineto{\pgfqpoint{1.768069in}{0.451389in}}%
\pgfpathclose%
\pgfusepath{stroke,fill}%
\end{pgfscope}%
\begin{pgfscope}%
\pgfpathrectangle{\pgfqpoint{0.532932in}{0.451389in}}{\pgfqpoint{4.382746in}{1.264399in}}%
\pgfusepath{clip}%
\pgfsetbuttcap%
\pgfsetmiterjoin%
\definecolor{currentfill}{rgb}{0.349020,0.490196,0.749020}%
\pgfsetfillcolor{currentfill}%
\pgfsetlinewidth{1.003750pt}%
\definecolor{currentstroke}{rgb}{1.000000,1.000000,1.000000}%
\pgfsetstrokecolor{currentstroke}%
\pgfsetdash{}{0pt}%
\pgfpathmoveto{\pgfqpoint{2.166501in}{0.451389in}}%
\pgfpathlineto{\pgfqpoint{2.325873in}{0.451389in}}%
\pgfpathlineto{\pgfqpoint{2.325873in}{0.543244in}}%
\pgfpathlineto{\pgfqpoint{2.166501in}{0.543244in}}%
\pgfpathlineto{\pgfqpoint{2.166501in}{0.451389in}}%
\pgfpathclose%
\pgfusepath{stroke,fill}%
\end{pgfscope}%
\begin{pgfscope}%
\pgfpathrectangle{\pgfqpoint{0.532932in}{0.451389in}}{\pgfqpoint{4.382746in}{1.264399in}}%
\pgfusepath{clip}%
\pgfsetbuttcap%
\pgfsetmiterjoin%
\definecolor{currentfill}{rgb}{0.349020,0.490196,0.749020}%
\pgfsetfillcolor{currentfill}%
\pgfsetlinewidth{1.003750pt}%
\definecolor{currentstroke}{rgb}{1.000000,1.000000,1.000000}%
\pgfsetstrokecolor{currentstroke}%
\pgfsetdash{}{0pt}%
\pgfpathmoveto{\pgfqpoint{2.564932in}{0.451389in}}%
\pgfpathlineto{\pgfqpoint{2.724305in}{0.451389in}}%
\pgfpathlineto{\pgfqpoint{2.724305in}{0.811798in}}%
\pgfpathlineto{\pgfqpoint{2.564932in}{0.811798in}}%
\pgfpathlineto{\pgfqpoint{2.564932in}{0.451389in}}%
\pgfpathclose%
\pgfusepath{stroke,fill}%
\end{pgfscope}%
\begin{pgfscope}%
\pgfpathrectangle{\pgfqpoint{0.532932in}{0.451389in}}{\pgfqpoint{4.382746in}{1.264399in}}%
\pgfusepath{clip}%
\pgfsetbuttcap%
\pgfsetmiterjoin%
\definecolor{currentfill}{rgb}{0.349020,0.490196,0.749020}%
\pgfsetfillcolor{currentfill}%
\pgfsetlinewidth{1.003750pt}%
\definecolor{currentstroke}{rgb}{1.000000,1.000000,1.000000}%
\pgfsetstrokecolor{currentstroke}%
\pgfsetdash{}{0pt}%
\pgfpathmoveto{\pgfqpoint{2.963364in}{0.451389in}}%
\pgfpathlineto{\pgfqpoint{3.122736in}{0.451389in}}%
\pgfpathlineto{\pgfqpoint{3.122736in}{0.508062in}}%
\pgfpathlineto{\pgfqpoint{2.963364in}{0.508062in}}%
\pgfpathlineto{\pgfqpoint{2.963364in}{0.451389in}}%
\pgfpathclose%
\pgfusepath{stroke,fill}%
\end{pgfscope}%
\begin{pgfscope}%
\pgfpathrectangle{\pgfqpoint{0.532932in}{0.451389in}}{\pgfqpoint{4.382746in}{1.264399in}}%
\pgfusepath{clip}%
\pgfsetbuttcap%
\pgfsetmiterjoin%
\definecolor{currentfill}{rgb}{0.349020,0.490196,0.749020}%
\pgfsetfillcolor{currentfill}%
\pgfsetlinewidth{1.003750pt}%
\definecolor{currentstroke}{rgb}{1.000000,1.000000,1.000000}%
\pgfsetstrokecolor{currentstroke}%
\pgfsetdash{}{0pt}%
\pgfpathmoveto{\pgfqpoint{3.361795in}{0.451389in}}%
\pgfpathlineto{\pgfqpoint{3.521168in}{0.451389in}}%
\pgfpathlineto{\pgfqpoint{3.521168in}{0.586595in}}%
\pgfpathlineto{\pgfqpoint{3.361795in}{0.586595in}}%
\pgfpathlineto{\pgfqpoint{3.361795in}{0.451389in}}%
\pgfpathclose%
\pgfusepath{stroke,fill}%
\end{pgfscope}%
\begin{pgfscope}%
\pgfpathrectangle{\pgfqpoint{0.532932in}{0.451389in}}{\pgfqpoint{4.382746in}{1.264399in}}%
\pgfusepath{clip}%
\pgfsetbuttcap%
\pgfsetmiterjoin%
\definecolor{currentfill}{rgb}{0.349020,0.490196,0.749020}%
\pgfsetfillcolor{currentfill}%
\pgfsetlinewidth{1.003750pt}%
\definecolor{currentstroke}{rgb}{1.000000,1.000000,1.000000}%
\pgfsetstrokecolor{currentstroke}%
\pgfsetdash{}{0pt}%
\pgfpathmoveto{\pgfqpoint{3.760227in}{0.451389in}}%
\pgfpathlineto{\pgfqpoint{3.919599in}{0.451389in}}%
\pgfpathlineto{\pgfqpoint{3.919599in}{0.586037in}}%
\pgfpathlineto{\pgfqpoint{3.760227in}{0.586037in}}%
\pgfpathlineto{\pgfqpoint{3.760227in}{0.451389in}}%
\pgfpathclose%
\pgfusepath{stroke,fill}%
\end{pgfscope}%
\begin{pgfscope}%
\pgfpathrectangle{\pgfqpoint{0.532932in}{0.451389in}}{\pgfqpoint{4.382746in}{1.264399in}}%
\pgfusepath{clip}%
\pgfsetbuttcap%
\pgfsetmiterjoin%
\definecolor{currentfill}{rgb}{0.349020,0.490196,0.749020}%
\pgfsetfillcolor{currentfill}%
\pgfsetlinewidth{1.003750pt}%
\definecolor{currentstroke}{rgb}{1.000000,1.000000,1.000000}%
\pgfsetstrokecolor{currentstroke}%
\pgfsetdash{}{0pt}%
\pgfpathmoveto{\pgfqpoint{4.158658in}{0.451389in}}%
\pgfpathlineto{\pgfqpoint{4.318031in}{0.451389in}}%
\pgfpathlineto{\pgfqpoint{4.318031in}{0.683841in}}%
\pgfpathlineto{\pgfqpoint{4.158658in}{0.683841in}}%
\pgfpathlineto{\pgfqpoint{4.158658in}{0.451389in}}%
\pgfpathclose%
\pgfusepath{stroke,fill}%
\end{pgfscope}%
\begin{pgfscope}%
\pgfpathrectangle{\pgfqpoint{0.532932in}{0.451389in}}{\pgfqpoint{4.382746in}{1.264399in}}%
\pgfusepath{clip}%
\pgfsetbuttcap%
\pgfsetmiterjoin%
\definecolor{currentfill}{rgb}{0.349020,0.490196,0.749020}%
\pgfsetfillcolor{currentfill}%
\pgfsetlinewidth{1.003750pt}%
\definecolor{currentstroke}{rgb}{1.000000,1.000000,1.000000}%
\pgfsetstrokecolor{currentstroke}%
\pgfsetdash{}{0pt}%
\pgfpathmoveto{\pgfqpoint{4.557090in}{0.451389in}}%
\pgfpathlineto{\pgfqpoint{4.716462in}{0.451389in}}%
\pgfpathlineto{\pgfqpoint{4.716462in}{0.615084in}}%
\pgfpathlineto{\pgfqpoint{4.557090in}{0.615084in}}%
\pgfpathlineto{\pgfqpoint{4.557090in}{0.451389in}}%
\pgfpathclose%
\pgfusepath{stroke,fill}%
\end{pgfscope}%
\begin{pgfscope}%
\pgfpathrectangle{\pgfqpoint{0.532932in}{0.451389in}}{\pgfqpoint{4.382746in}{1.264399in}}%
\pgfusepath{clip}%
\pgfsetbuttcap%
\pgfsetmiterjoin%
\definecolor{currentfill}{rgb}{0.852941,0.544118,0.370588}%
\pgfsetfillcolor{currentfill}%
\pgfsetlinewidth{1.003750pt}%
\definecolor{currentstroke}{rgb}{1.000000,1.000000,1.000000}%
\pgfsetstrokecolor{currentstroke}%
\pgfsetdash{}{0pt}%
\pgfpathmoveto{\pgfqpoint{0.732148in}{0.451389in}}%
\pgfpathlineto{\pgfqpoint{0.891520in}{0.451389in}}%
\pgfpathlineto{\pgfqpoint{0.891520in}{0.504906in}}%
\pgfpathlineto{\pgfqpoint{0.732148in}{0.504906in}}%
\pgfpathlineto{\pgfqpoint{0.732148in}{0.451389in}}%
\pgfpathclose%
\pgfusepath{stroke,fill}%
\end{pgfscope}%
\begin{pgfscope}%
\pgfpathrectangle{\pgfqpoint{0.532932in}{0.451389in}}{\pgfqpoint{4.382746in}{1.264399in}}%
\pgfusepath{clip}%
\pgfsetbuttcap%
\pgfsetmiterjoin%
\definecolor{currentfill}{rgb}{0.852941,0.544118,0.370588}%
\pgfsetfillcolor{currentfill}%
\pgfsetlinewidth{1.003750pt}%
\definecolor{currentstroke}{rgb}{1.000000,1.000000,1.000000}%
\pgfsetstrokecolor{currentstroke}%
\pgfsetdash{}{0pt}%
\pgfpathmoveto{\pgfqpoint{1.130579in}{0.451389in}}%
\pgfpathlineto{\pgfqpoint{1.289952in}{0.451389in}}%
\pgfpathlineto{\pgfqpoint{1.289952in}{0.491528in}}%
\pgfpathlineto{\pgfqpoint{1.130579in}{0.491528in}}%
\pgfpathlineto{\pgfqpoint{1.130579in}{0.451389in}}%
\pgfpathclose%
\pgfusepath{stroke,fill}%
\end{pgfscope}%
\begin{pgfscope}%
\pgfpathrectangle{\pgfqpoint{0.532932in}{0.451389in}}{\pgfqpoint{4.382746in}{1.264399in}}%
\pgfusepath{clip}%
\pgfsetbuttcap%
\pgfsetmiterjoin%
\definecolor{currentfill}{rgb}{0.852941,0.544118,0.370588}%
\pgfsetfillcolor{currentfill}%
\pgfsetlinewidth{1.003750pt}%
\definecolor{currentstroke}{rgb}{1.000000,1.000000,1.000000}%
\pgfsetstrokecolor{currentstroke}%
\pgfsetdash{}{0pt}%
\pgfpathmoveto{\pgfqpoint{1.529010in}{0.451389in}}%
\pgfpathlineto{\pgfqpoint{1.688383in}{0.451389in}}%
\pgfpathlineto{\pgfqpoint{1.688383in}{0.551684in}}%
\pgfpathlineto{\pgfqpoint{1.529010in}{0.551684in}}%
\pgfpathlineto{\pgfqpoint{1.529010in}{0.451389in}}%
\pgfpathclose%
\pgfusepath{stroke,fill}%
\end{pgfscope}%
\begin{pgfscope}%
\pgfpathrectangle{\pgfqpoint{0.532932in}{0.451389in}}{\pgfqpoint{4.382746in}{1.264399in}}%
\pgfusepath{clip}%
\pgfsetbuttcap%
\pgfsetmiterjoin%
\definecolor{currentfill}{rgb}{0.852941,0.544118,0.370588}%
\pgfsetfillcolor{currentfill}%
\pgfsetlinewidth{1.003750pt}%
\definecolor{currentstroke}{rgb}{1.000000,1.000000,1.000000}%
\pgfsetstrokecolor{currentstroke}%
\pgfsetdash{}{0pt}%
\pgfpathmoveto{\pgfqpoint{1.927442in}{0.451389in}}%
\pgfpathlineto{\pgfqpoint{2.086814in}{0.451389in}}%
\pgfpathlineto{\pgfqpoint{2.086814in}{0.586282in}}%
\pgfpathlineto{\pgfqpoint{1.927442in}{0.586282in}}%
\pgfpathlineto{\pgfqpoint{1.927442in}{0.451389in}}%
\pgfpathclose%
\pgfusepath{stroke,fill}%
\end{pgfscope}%
\begin{pgfscope}%
\pgfpathrectangle{\pgfqpoint{0.532932in}{0.451389in}}{\pgfqpoint{4.382746in}{1.264399in}}%
\pgfusepath{clip}%
\pgfsetbuttcap%
\pgfsetmiterjoin%
\definecolor{currentfill}{rgb}{0.852941,0.544118,0.370588}%
\pgfsetfillcolor{currentfill}%
\pgfsetlinewidth{1.003750pt}%
\definecolor{currentstroke}{rgb}{1.000000,1.000000,1.000000}%
\pgfsetstrokecolor{currentstroke}%
\pgfsetdash{}{0pt}%
\pgfpathmoveto{\pgfqpoint{2.325873in}{0.451389in}}%
\pgfpathlineto{\pgfqpoint{2.485246in}{0.451389in}}%
\pgfpathlineto{\pgfqpoint{2.485246in}{0.586282in}}%
\pgfpathlineto{\pgfqpoint{2.325873in}{0.586282in}}%
\pgfpathlineto{\pgfqpoint{2.325873in}{0.451389in}}%
\pgfpathclose%
\pgfusepath{stroke,fill}%
\end{pgfscope}%
\begin{pgfscope}%
\pgfpathrectangle{\pgfqpoint{0.532932in}{0.451389in}}{\pgfqpoint{4.382746in}{1.264399in}}%
\pgfusepath{clip}%
\pgfsetbuttcap%
\pgfsetmiterjoin%
\definecolor{currentfill}{rgb}{0.852941,0.544118,0.370588}%
\pgfsetfillcolor{currentfill}%
\pgfsetlinewidth{1.003750pt}%
\definecolor{currentstroke}{rgb}{1.000000,1.000000,1.000000}%
\pgfsetstrokecolor{currentstroke}%
\pgfsetdash{}{0pt}%
\pgfpathmoveto{\pgfqpoint{2.724305in}{0.451389in}}%
\pgfpathlineto{\pgfqpoint{2.883677in}{0.451389in}}%
\pgfpathlineto{\pgfqpoint{2.883677in}{1.655578in}}%
\pgfpathlineto{\pgfqpoint{2.724305in}{1.655578in}}%
\pgfpathlineto{\pgfqpoint{2.724305in}{0.451389in}}%
\pgfpathclose%
\pgfusepath{stroke,fill}%
\end{pgfscope}%
\begin{pgfscope}%
\pgfpathrectangle{\pgfqpoint{0.532932in}{0.451389in}}{\pgfqpoint{4.382746in}{1.264399in}}%
\pgfusepath{clip}%
\pgfsetbuttcap%
\pgfsetmiterjoin%
\definecolor{currentfill}{rgb}{0.852941,0.544118,0.370588}%
\pgfsetfillcolor{currentfill}%
\pgfsetlinewidth{1.003750pt}%
\definecolor{currentstroke}{rgb}{1.000000,1.000000,1.000000}%
\pgfsetstrokecolor{currentstroke}%
\pgfsetdash{}{0pt}%
\pgfpathmoveto{\pgfqpoint{3.122736in}{0.451389in}}%
\pgfpathlineto{\pgfqpoint{3.282109in}{0.451389in}}%
\pgfpathlineto{\pgfqpoint{3.282109in}{1.626475in}}%
\pgfpathlineto{\pgfqpoint{3.122736in}{1.626475in}}%
\pgfpathlineto{\pgfqpoint{3.122736in}{0.451389in}}%
\pgfpathclose%
\pgfusepath{stroke,fill}%
\end{pgfscope}%
\begin{pgfscope}%
\pgfpathrectangle{\pgfqpoint{0.532932in}{0.451389in}}{\pgfqpoint{4.382746in}{1.264399in}}%
\pgfusepath{clip}%
\pgfsetbuttcap%
\pgfsetmiterjoin%
\definecolor{currentfill}{rgb}{0.852941,0.544118,0.370588}%
\pgfsetfillcolor{currentfill}%
\pgfsetlinewidth{1.003750pt}%
\definecolor{currentstroke}{rgb}{1.000000,1.000000,1.000000}%
\pgfsetstrokecolor{currentstroke}%
\pgfsetdash{}{0pt}%
\pgfpathmoveto{\pgfqpoint{3.521168in}{0.451389in}}%
\pgfpathlineto{\pgfqpoint{3.680540in}{0.451389in}}%
\pgfpathlineto{\pgfqpoint{3.680540in}{0.517567in}}%
\pgfpathlineto{\pgfqpoint{3.521168in}{0.517567in}}%
\pgfpathlineto{\pgfqpoint{3.521168in}{0.451389in}}%
\pgfpathclose%
\pgfusepath{stroke,fill}%
\end{pgfscope}%
\begin{pgfscope}%
\pgfpathrectangle{\pgfqpoint{0.532932in}{0.451389in}}{\pgfqpoint{4.382746in}{1.264399in}}%
\pgfusepath{clip}%
\pgfsetbuttcap%
\pgfsetmiterjoin%
\definecolor{currentfill}{rgb}{0.852941,0.544118,0.370588}%
\pgfsetfillcolor{currentfill}%
\pgfsetlinewidth{1.003750pt}%
\definecolor{currentstroke}{rgb}{1.000000,1.000000,1.000000}%
\pgfsetstrokecolor{currentstroke}%
\pgfsetdash{}{0pt}%
\pgfpathmoveto{\pgfqpoint{3.919599in}{0.451389in}}%
\pgfpathlineto{\pgfqpoint{4.078972in}{0.451389in}}%
\pgfpathlineto{\pgfqpoint{4.078972in}{0.480472in}}%
\pgfpathlineto{\pgfqpoint{3.919599in}{0.480472in}}%
\pgfpathlineto{\pgfqpoint{3.919599in}{0.451389in}}%
\pgfpathclose%
\pgfusepath{stroke,fill}%
\end{pgfscope}%
\begin{pgfscope}%
\pgfpathrectangle{\pgfqpoint{0.532932in}{0.451389in}}{\pgfqpoint{4.382746in}{1.264399in}}%
\pgfusepath{clip}%
\pgfsetbuttcap%
\pgfsetmiterjoin%
\definecolor{currentfill}{rgb}{0.852941,0.544118,0.370588}%
\pgfsetfillcolor{currentfill}%
\pgfsetlinewidth{1.003750pt}%
\definecolor{currentstroke}{rgb}{1.000000,1.000000,1.000000}%
\pgfsetstrokecolor{currentstroke}%
\pgfsetdash{}{0pt}%
\pgfpathmoveto{\pgfqpoint{4.318031in}{0.451389in}}%
\pgfpathlineto{\pgfqpoint{4.477403in}{0.451389in}}%
\pgfpathlineto{\pgfqpoint{4.477403in}{0.569799in}}%
\pgfpathlineto{\pgfqpoint{4.318031in}{0.569799in}}%
\pgfpathlineto{\pgfqpoint{4.318031in}{0.451389in}}%
\pgfpathclose%
\pgfusepath{stroke,fill}%
\end{pgfscope}%
\begin{pgfscope}%
\pgfpathrectangle{\pgfqpoint{0.532932in}{0.451389in}}{\pgfqpoint{4.382746in}{1.264399in}}%
\pgfusepath{clip}%
\pgfsetbuttcap%
\pgfsetmiterjoin%
\definecolor{currentfill}{rgb}{0.852941,0.544118,0.370588}%
\pgfsetfillcolor{currentfill}%
\pgfsetlinewidth{1.003750pt}%
\definecolor{currentstroke}{rgb}{1.000000,1.000000,1.000000}%
\pgfsetstrokecolor{currentstroke}%
\pgfsetdash{}{0pt}%
\pgfpathmoveto{\pgfqpoint{4.716462in}{0.451389in}}%
\pgfpathlineto{\pgfqpoint{4.875835in}{0.451389in}}%
\pgfpathlineto{\pgfqpoint{4.875835in}{0.757057in}}%
\pgfpathlineto{\pgfqpoint{4.716462in}{0.757057in}}%
\pgfpathlineto{\pgfqpoint{4.716462in}{0.451389in}}%
\pgfpathclose%
\pgfusepath{stroke,fill}%
\end{pgfscope}%
\begin{pgfscope}%
\pgfsetrectcap%
\pgfsetmiterjoin%
\pgfsetlinewidth{1.254687pt}%
\definecolor{currentstroke}{rgb}{0.800000,0.800000,0.800000}%
\pgfsetstrokecolor{currentstroke}%
\pgfsetdash{}{0pt}%
\pgfpathmoveto{\pgfqpoint{0.532932in}{0.451389in}}%
\pgfpathlineto{\pgfqpoint{0.532932in}{1.715788in}}%
\pgfusepath{stroke}%
\end{pgfscope}%
\begin{pgfscope}%
\pgfsetrectcap%
\pgfsetmiterjoin%
\pgfsetlinewidth{1.254687pt}%
\definecolor{currentstroke}{rgb}{0.800000,0.800000,0.800000}%
\pgfsetstrokecolor{currentstroke}%
\pgfsetdash{}{0pt}%
\pgfpathmoveto{\pgfqpoint{4.915678in}{0.451389in}}%
\pgfpathlineto{\pgfqpoint{4.915678in}{1.715788in}}%
\pgfusepath{stroke}%
\end{pgfscope}%
\begin{pgfscope}%
\pgfsetrectcap%
\pgfsetmiterjoin%
\pgfsetlinewidth{1.254687pt}%
\definecolor{currentstroke}{rgb}{0.800000,0.800000,0.800000}%
\pgfsetstrokecolor{currentstroke}%
\pgfsetdash{}{0pt}%
\pgfpathmoveto{\pgfqpoint{0.532932in}{0.451389in}}%
\pgfpathlineto{\pgfqpoint{4.915678in}{0.451389in}}%
\pgfusepath{stroke}%
\end{pgfscope}%
\begin{pgfscope}%
\pgfsetrectcap%
\pgfsetmiterjoin%
\pgfsetlinewidth{1.254687pt}%
\definecolor{currentstroke}{rgb}{0.800000,0.800000,0.800000}%
\pgfsetstrokecolor{currentstroke}%
\pgfsetdash{}{0pt}%
\pgfpathmoveto{\pgfqpoint{0.532932in}{1.715788in}}%
\pgfpathlineto{\pgfqpoint{4.915678in}{1.715788in}}%
\pgfusepath{stroke}%
\end{pgfscope}%
\begin{pgfscope}%
\pgfsetbuttcap%
\pgfsetmiterjoin%
\definecolor{currentfill}{rgb}{1.000000,1.000000,1.000000}%
\pgfsetfillcolor{currentfill}%
\pgfsetfillopacity{0.800000}%
\pgfsetlinewidth{1.003750pt}%
\definecolor{currentstroke}{rgb}{0.800000,0.800000,0.800000}%
\pgfsetstrokecolor{currentstroke}%
\pgfsetstrokeopacity{0.800000}%
\pgfsetdash{}{0pt}%
\pgfpathmoveto{\pgfqpoint{5.112746in}{0.589350in}}%
\pgfpathlineto{\pgfqpoint{6.058523in}{0.589350in}}%
\pgfpathquadraticcurveto{\pgfqpoint{6.083523in}{0.589350in}}{\pgfqpoint{6.083523in}{0.614350in}}%
\pgfpathlineto{\pgfqpoint{6.083523in}{1.552826in}}%
\pgfpathquadraticcurveto{\pgfqpoint{6.083523in}{1.577826in}}{\pgfqpoint{6.058523in}{1.577826in}}%
\pgfpathlineto{\pgfqpoint{5.112746in}{1.577826in}}%
\pgfpathquadraticcurveto{\pgfqpoint{5.087746in}{1.577826in}}{\pgfqpoint{5.087746in}{1.552826in}}%
\pgfpathlineto{\pgfqpoint{5.087746in}{0.614350in}}%
\pgfpathquadraticcurveto{\pgfqpoint{5.087746in}{0.589350in}}{\pgfqpoint{5.112746in}{0.589350in}}%
\pgfpathlineto{\pgfqpoint{5.112746in}{0.589350in}}%
\pgfpathclose%
\pgfusepath{stroke,fill}%
\end{pgfscope}%
\begin{pgfscope}%
\pgfsetbuttcap%
\pgfsetmiterjoin%
\definecolor{currentfill}{rgb}{0.349020,0.490196,0.749020}%
\pgfsetfillcolor{currentfill}%
\pgfsetlinewidth{1.003750pt}%
\definecolor{currentstroke}{rgb}{1.000000,1.000000,1.000000}%
\pgfsetstrokecolor{currentstroke}%
\pgfsetdash{}{0pt}%
\pgfpathmoveto{\pgfqpoint{5.137746in}{1.273847in}}%
\pgfpathlineto{\pgfqpoint{5.387746in}{1.273847in}}%
\pgfpathlineto{\pgfqpoint{5.387746in}{1.361347in}}%
\pgfpathlineto{\pgfqpoint{5.137746in}{1.361347in}}%
\pgfpathlineto{\pgfqpoint{5.137746in}{1.273847in}}%
\pgfpathclose%
\pgfusepath{stroke,fill}%
\end{pgfscope}%
\begin{pgfscope}%
\definecolor{textcolor}{rgb}{0.150000,0.150000,0.150000}%
\pgfsetstrokecolor{textcolor}%
\pgfsetfillcolor{textcolor}%
\pgftext[x=5.487746in, y=1.432855in, left, base]{\color{textcolor}\sffamily\fontsize{9.000000}{10.800000}\selectfont Hybrid}%
\end{pgfscope}%
\begin{pgfscope}%
\definecolor{textcolor}{rgb}{0.150000,0.150000,0.150000}%
\pgfsetstrokecolor{textcolor}%
\pgfsetfillcolor{textcolor}%
\pgftext[x=5.487746in, y=1.288861in, left, base]{\color{textcolor}\sffamily\fontsize{9.000000}{10.800000}\selectfont routing}%
\end{pgfscope}%
\begin{pgfscope}%
\definecolor{textcolor}{rgb}{0.150000,0.150000,0.150000}%
\pgfsetstrokecolor{textcolor}%
\pgfsetfillcolor{textcolor}%
\pgftext[x=5.487746in, y=1.144867in, left, base]{\color{textcolor}\sffamily\fontsize{9.000000}{10.800000}\selectfont algorithm}%
\end{pgfscope}%
\begin{pgfscope}%
\pgfsetbuttcap%
\pgfsetmiterjoin%
\definecolor{currentfill}{rgb}{0.852941,0.544118,0.370588}%
\pgfsetfillcolor{currentfill}%
\pgfsetlinewidth{1.003750pt}%
\definecolor{currentstroke}{rgb}{1.000000,1.000000,1.000000}%
\pgfsetstrokecolor{currentstroke}%
\pgfsetdash{}{0pt}%
\pgfpathmoveto{\pgfqpoint{5.137746in}{0.798359in}}%
\pgfpathlineto{\pgfqpoint{5.387746in}{0.798359in}}%
\pgfpathlineto{\pgfqpoint{5.387746in}{0.885859in}}%
\pgfpathlineto{\pgfqpoint{5.137746in}{0.885859in}}%
\pgfpathlineto{\pgfqpoint{5.137746in}{0.798359in}}%
\pgfpathclose%
\pgfusepath{stroke,fill}%
\end{pgfscope}%
\begin{pgfscope}%
\definecolor{textcolor}{rgb}{0.150000,0.150000,0.150000}%
\pgfsetstrokecolor{textcolor}%
\pgfsetfillcolor{textcolor}%
\pgftext[x=5.487746in, y=0.957367in, left, base]{\color{textcolor}\sffamily\fontsize{9.000000}{10.800000}\selectfont Graph-}%
\end{pgfscope}%
\begin{pgfscope}%
\definecolor{textcolor}{rgb}{0.150000,0.150000,0.150000}%
\pgfsetstrokecolor{textcolor}%
\pgfsetfillcolor{textcolor}%
\pgftext[x=5.487746in, y=0.813373in, left, base]{\color{textcolor}\sffamily\fontsize{9.000000}{10.800000}\selectfont based}%
\end{pgfscope}%
\begin{pgfscope}%
\definecolor{textcolor}{rgb}{0.150000,0.150000,0.150000}%
\pgfsetstrokecolor{textcolor}%
\pgfsetfillcolor{textcolor}%
\pgftext[x=5.487746in, y=0.669379in, left, base]{\color{textcolor}\sffamily\fontsize{9.000000}{10.800000}\selectfont routing}%
\end{pgfscope}%
\end{pgfpicture}%
\makeatother%
\endgroup%

					\end{figcenter}
					\caption{\enquote{OSM rural} dataset.}
					\label{fig:eval-hausdorff-rural}
				\end{subfigure}
				\caption[Hausdorff distance comparison.]{Hausdorff distances between the expected routes and the corresponding routes of the hybrid routing algorithm and graph-based routing using the 0.5km\textsuperscript{2} OSM datasets.}
				\label{fig:eval-hausdorff}
			\end{figure}

		\subsubsection{Route quality analysis summary}
			
			The manual route analysis as well as both mathematical approaches, the comparison to the beeline distance and the Hausdorff distance, yield similar results regarding the quality of routes:
			
			\begin{itemize}
				\item The route quality heavily depends on the data quality.
				Missing data, for example even a small number of missing ditches on farmland or one fence between two larger building, result in shorter but unrealistic routes determined with the hybrid routing algorithm.
				\item Cities and densely built-up areas form corridors through which geometric- and graph-based routing results tend to be very similar.
				\item Graph-based routes between waypoints in rural or remote areas tends to create long detours.
				Such detours often are unrealistic, especially in agent-based models simulating scenarios in which pedestrians do not necessarily respect access restrictions, such as in emergency or evacuation scenarios.
			\end{itemize}
			\noindent
			In scenarios with a high quality in data, for example when using official indoor room plans of buildings or when the data quality is ensured by a manual on-ground survey, the hybrid routing algorithm will likely create high quality routes well reflecting the behavior of real pedestrians.