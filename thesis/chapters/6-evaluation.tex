% !TEX root = ../thesis.tex
% !TeX spellcheck = en_US

The previously shown design and implementation of the hybrid routing algorithm was evaluated regarding performance and usefulness.
This chapter covers the two aspects of performance and usefulness separately.
For both parts, method and design details are given followed by the respective results of the evaluations.

\section{Performance evaluation}

	The performance evaluation uses different datasets to measure graph generation and routing times.
	Each of these two steps is measured more fine-grained by measuring the execution times of separate method calls.
	The datasets have different properties and sizes and consist of artificial and real-world data.

	\subsection{Methods \& Measurements}

		% what was measured?
		\subsubsection{Collected data}
		
			The collected data consists of time measurement, many of them on the level of separate methods.
			Also, the amount of data is measures, namely the number of edges and vertices at various steps in the process.
			
			As a result, two CSV files are written per dataset containing measurement data for the import (including graph generation) and routing.
			The measurement for the routing requests also contains information about the lengths of the routes, especially beeline and actual route distances.
			
			% TODO table with all columns of the measured data including a description and example value?
		
		% datasets
		\subsubsection{Datasets}
		
			There are multiple categories of datasets that were used:
			
			\begin{description}
				\item[Maze pattern] A set of datasets containing maze like geometries, namely only connected linestrings forming a seamless pattern tile that can arbitrarily often be repeated. Many of the contained line obstacles are collinear. No roads are within these datasets.
				\item[Rectangle pattern] Also a set of pattern-based datasets, which contain simple rectangles of different sizes. No roads are within these datasets.
				\item[Circle pattern] Like the rectangle datasets but with circles. These datasets have large number of vertices.
				\item[OSM city] This is a real-world export from the OpenStreetMap database with data from the city of Hamburg, Germany. The data has been filtered to remove all over- and underground features since the hybrid routing algorithm has no handling of three dimensional data. This dataset also contains all roads and ways in the respective region.
				\item[OSM rural] Equivalent to the \enquote{OSM city} dataset, but located outside the city with more natural obstacles (lakes, ditches, forest), more open spaces and less regular buildings.
				\item[OSM export without roads] Same data from the above OSM exports, but without the roads. This is used to show the overhead of the road on the graph generation and routing times.
				\item[OSM export without obstacles] Same data from the above OSM exports, but without the obstacles, i.e. buildings, walls and water areas. This is used to show the overhead of the obstacles on the graph generation and routing times.
			\end{description}
			
			The pattern datasets exist in different sizes, since the pattern can arbitrarily often be repeated.
			
			\todo{Illustrate this as a diagram or list? Also list the number of raw vertices within the datasets.}
			
			\todo{List sizes/statistics of datasets?}
		
		% optimizations
		\subsubsection{Optimizations}
		
			As \Cref{chap:implementation} already mentioned, there were several optimizations made to the implementation.
			Some of which are on the level of data structures, some on the algorithmic level.
			The effectiveness of these optimization was also evaluated using the OSM dataset.
			Each of the following optimizations was deactivated for the evaluation:

			\begin{description}
				\item[Shadow areas] Instead, every visibility check was performed using the custom intersection check described in \Cref{subsubsec:intersection-checks}.
				\item[Custom intersection check] The custom intersection check was replaced by the \todo[inline]{probably \texttt{RobustLineIntersector}?} class from the NTS to determine intersections between line segments.
				\item[BinIndex] Instead, the \texttt{BinTree} from the NTS was used.
				\item[Convex hull] The restriction to only consider vertices on the convex hull of obstacles was removed.
				\item[Valid angle areas] Considering only potential visibility neighbors within certain angular ranges was deactivated.
				\item[$k$-NN search] The $k$ of the k-NN search was deactivated to determine all visibility neighbors in all directions.
			\end{description}			
		% how were times measured? HikerModel
		\subsubsection{Measurement method}
		
			Measuring the performance was done by a small agent-based simulation project called \texttt{HikerModel}, which consists of one agent, a list of coordinates and the input dataset.
			Each coordinate from the input list is visited by the agent using the hybrid routing algorithm to determine the path to the next location.
			The time of the graph generation as well as the time of each routing request are measured.
		
		% changes to the code
		% difficulties in C#
		\subsubsection{Technical considerations}
		
			The measurement was done by a helper class \texttt{PerformanceMeasurement} providing a method that accepts an arbitrary function which execution time should be measured.
			
			Unfortunately, C\# does not provide any method to disable the garbage collector.
			Also, the just-in-time (JIT) compilation cannot be turned of for normal .NET executions via \texttt{dotnet program.dll}.
			The alternative would be an ahead-of-time (AOT) compilation, which slows down LINQ operations.
			\todo[inline]{source on that + test it}

			Because both compilation strategies have disadvantages, the normal .NET-based execution was chosen, even though it contains JIT compilation.
			To mitigate this dynamic behavior and to generally get resilient results, three warm-up iterations were performed before measuring the times of five actual execution iterations.
			Even though only the results from the five actual iterations were part of the result, the times from the warm-up iterations showed that three iterations are enough to prepare the runtime and garbage collector.
			
			Additionally, the garbage collector was triggered during each of the eight iterations just before calling the function to measure.
			This is done by the \texttt{GC.Collect()} and \texttt{GC.WaitForPendingFinalizers()} from the \texttt{GC} class of the .NET framework.
			Using these two methods forces a garbage collection and waits for it to finish\cite{ms-gc}.
			
			To prevent the garbage collection from interfering with the execution, a 256 MiB large no-GC-region is placed around the function call via \texttt{GC.TryStartNoGCRegion(256 * 1024 * 1024)}.
			Although this only works if enough memory is available\cite{ms-no-gc-region}, the system had this memory and introduction this no-GC-region significantly stabilized the results.
			\todo[inline]{test this and maybe give an example result (standard deviation or so?)}
			
			Another step is the increase of the process priority.
			This ideally leads to an exclusive use of one CPU core on which the single threaded application runs.
			Increasing the process priority is done by settings the \texttt{PriorityClass} of the current process to \texttt{ProcessPriorityClass.High}.
		
		\subsubsection{System and hardware}
		
			The measurements were performed on an up-to-date Arch Linux operating system with .NET Core 7.0.105 and MARS framework 4.5.2.
			Apart from necessary operating system processes and the desktop environment, no other load intensive processes ran during the performance measurements.
			
			The hardware consisted of an octa core Intel\textregistered\ Xeon\textregistered\ E3-1231 v3 CPU at 3.40 GHz, a total of 16GB DDR3 1333 MHz RAM and a Samsung EVO 850 SSD.
			However, the whole algorithm and the \texttt{HikerModel} simulation are single threaded and file system operations are only performed to initially read the input data and to write the measurement results at the end.
	
	\subsection{Import and graph generation}
	
		As mentioned at the beginning of \Cref{subsec:related-work:visibility-graph}, the process of generating a visibility graph has an inherent quadratic runtime.
		This result is clearly visible in measurements of the dataset imports as seen in \Cref{fig:eval-import-city-abs} and \Cref{fig:eval-import-rural-abs}.
		
		\begin{figure}[h!]
			\begin{minipage}{.48\textwidth}
				\begin{subfigure}[t]{\linewidth}
					\begin{figcenter}
						\begingroup%
\makeatletter%
\begin{pgfpicture}%
\pgfpathrectangle{\pgfpointorigin}{\pgfqpoint{3.041406in}{1.869661in}}%
\pgfusepath{use as bounding box}%
\begin{pgfscope}%
\pgfsetbuttcap%
\pgfsetmiterjoin%
\definecolor{currentfill}{rgb}{1.000000,1.000000,1.000000}%
\pgfsetfillcolor{currentfill}%
\pgfsetlinewidth{0.000000pt}%
\definecolor{currentstroke}{rgb}{1.000000,1.000000,1.000000}%
\pgfsetstrokecolor{currentstroke}%
\pgfsetdash{}{0pt}%
\pgfpathmoveto{\pgfqpoint{0.000000in}{0.000000in}}%
\pgfpathlineto{\pgfqpoint{3.041406in}{0.000000in}}%
\pgfpathlineto{\pgfqpoint{3.041406in}{1.869661in}}%
\pgfpathlineto{\pgfqpoint{0.000000in}{1.869661in}}%
\pgfpathlineto{\pgfqpoint{0.000000in}{0.000000in}}%
\pgfpathclose%
\pgfusepath{fill}%
\end{pgfscope}%
\begin{pgfscope}%
\pgfsetbuttcap%
\pgfsetmiterjoin%
\definecolor{currentfill}{rgb}{1.000000,1.000000,1.000000}%
\pgfsetfillcolor{currentfill}%
\pgfsetlinewidth{0.000000pt}%
\definecolor{currentstroke}{rgb}{0.000000,0.000000,0.000000}%
\pgfsetstrokecolor{currentstroke}%
\pgfsetstrokeopacity{0.000000}%
\pgfsetdash{}{0pt}%
\pgfpathmoveto{\pgfqpoint{0.601779in}{0.451389in}}%
\pgfpathlineto{\pgfqpoint{3.041406in}{0.451389in}}%
\pgfpathlineto{\pgfqpoint{3.041406in}{1.848071in}}%
\pgfpathlineto{\pgfqpoint{0.601779in}{1.848071in}}%
\pgfpathlineto{\pgfqpoint{0.601779in}{0.451389in}}%
\pgfpathclose%
\pgfusepath{fill}%
\end{pgfscope}%
\begin{pgfscope}%
\pgfpathrectangle{\pgfqpoint{0.601779in}{0.451389in}}{\pgfqpoint{2.439626in}{1.396682in}}%
\pgfusepath{clip}%
\pgfsetroundcap%
\pgfsetroundjoin%
\pgfsetlinewidth{1.003750pt}%
\definecolor{currentstroke}{rgb}{0.800000,0.800000,0.800000}%
\pgfsetstrokecolor{currentstroke}%
\pgfsetdash{}{0pt}%
\pgfpathmoveto{\pgfqpoint{0.601779in}{0.451389in}}%
\pgfpathlineto{\pgfqpoint{0.601779in}{1.848071in}}%
\pgfusepath{stroke}%
\end{pgfscope}%
\begin{pgfscope}%
\definecolor{textcolor}{rgb}{0.150000,0.150000,0.150000}%
\pgfsetstrokecolor{textcolor}%
\pgfsetfillcolor{textcolor}%
\pgftext[x=0.601779in,y=0.319444in,,top]{\color{textcolor}\sffamily\fontsize{9.000000}{10.800000}\selectfont 0}%
\end{pgfscope}%
\begin{pgfscope}%
\pgfpathrectangle{\pgfqpoint{0.601779in}{0.451389in}}{\pgfqpoint{2.439626in}{1.396682in}}%
\pgfusepath{clip}%
\pgfsetroundcap%
\pgfsetroundjoin%
\pgfsetlinewidth{1.003750pt}%
\definecolor{currentstroke}{rgb}{0.800000,0.800000,0.800000}%
\pgfsetstrokecolor{currentstroke}%
\pgfsetdash{}{0pt}%
\pgfpathmoveto{\pgfqpoint{1.121799in}{0.451389in}}%
\pgfpathlineto{\pgfqpoint{1.121799in}{1.848071in}}%
\pgfusepath{stroke}%
\end{pgfscope}%
\begin{pgfscope}%
\definecolor{textcolor}{rgb}{0.150000,0.150000,0.150000}%
\pgfsetstrokecolor{textcolor}%
\pgfsetfillcolor{textcolor}%
\pgftext[x=1.121799in,y=0.319444in,,top]{\color{textcolor}\sffamily\fontsize{9.000000}{10.800000}\selectfont 10000}%
\end{pgfscope}%
\begin{pgfscope}%
\pgfpathrectangle{\pgfqpoint{0.601779in}{0.451389in}}{\pgfqpoint{2.439626in}{1.396682in}}%
\pgfusepath{clip}%
\pgfsetroundcap%
\pgfsetroundjoin%
\pgfsetlinewidth{1.003750pt}%
\definecolor{currentstroke}{rgb}{0.800000,0.800000,0.800000}%
\pgfsetstrokecolor{currentstroke}%
\pgfsetdash{}{0pt}%
\pgfpathmoveto{\pgfqpoint{1.641819in}{0.451389in}}%
\pgfpathlineto{\pgfqpoint{1.641819in}{1.848071in}}%
\pgfusepath{stroke}%
\end{pgfscope}%
\begin{pgfscope}%
\definecolor{textcolor}{rgb}{0.150000,0.150000,0.150000}%
\pgfsetstrokecolor{textcolor}%
\pgfsetfillcolor{textcolor}%
\pgftext[x=1.641819in,y=0.319444in,,top]{\color{textcolor}\sffamily\fontsize{9.000000}{10.800000}\selectfont 20000}%
\end{pgfscope}%
\begin{pgfscope}%
\pgfpathrectangle{\pgfqpoint{0.601779in}{0.451389in}}{\pgfqpoint{2.439626in}{1.396682in}}%
\pgfusepath{clip}%
\pgfsetroundcap%
\pgfsetroundjoin%
\pgfsetlinewidth{1.003750pt}%
\definecolor{currentstroke}{rgb}{0.800000,0.800000,0.800000}%
\pgfsetstrokecolor{currentstroke}%
\pgfsetdash{}{0pt}%
\pgfpathmoveto{\pgfqpoint{2.161839in}{0.451389in}}%
\pgfpathlineto{\pgfqpoint{2.161839in}{1.848071in}}%
\pgfusepath{stroke}%
\end{pgfscope}%
\begin{pgfscope}%
\definecolor{textcolor}{rgb}{0.150000,0.150000,0.150000}%
\pgfsetstrokecolor{textcolor}%
\pgfsetfillcolor{textcolor}%
\pgftext[x=2.161839in,y=0.319444in,,top]{\color{textcolor}\sffamily\fontsize{9.000000}{10.800000}\selectfont 30000}%
\end{pgfscope}%
\begin{pgfscope}%
\pgfpathrectangle{\pgfqpoint{0.601779in}{0.451389in}}{\pgfqpoint{2.439626in}{1.396682in}}%
\pgfusepath{clip}%
\pgfsetroundcap%
\pgfsetroundjoin%
\pgfsetlinewidth{1.003750pt}%
\definecolor{currentstroke}{rgb}{0.800000,0.800000,0.800000}%
\pgfsetstrokecolor{currentstroke}%
\pgfsetdash{}{0pt}%
\pgfpathmoveto{\pgfqpoint{2.681859in}{0.451389in}}%
\pgfpathlineto{\pgfqpoint{2.681859in}{1.848071in}}%
\pgfusepath{stroke}%
\end{pgfscope}%
\begin{pgfscope}%
\definecolor{textcolor}{rgb}{0.150000,0.150000,0.150000}%
\pgfsetstrokecolor{textcolor}%
\pgfsetfillcolor{textcolor}%
\pgftext[x=2.681859in,y=0.319444in,,top]{\color{textcolor}\sffamily\fontsize{9.000000}{10.800000}\selectfont 40000}%
\end{pgfscope}%
\begin{pgfscope}%
\definecolor{textcolor}{rgb}{0.150000,0.150000,0.150000}%
\pgfsetstrokecolor{textcolor}%
\pgfsetfillcolor{textcolor}%
\pgftext[x=1.821593in,y=0.125000in,,top]{\color{textcolor}\sffamily\fontsize{9.000000}{10.800000}\selectfont Input obstacle vertices}%
\end{pgfscope}%
\begin{pgfscope}%
\pgfpathrectangle{\pgfqpoint{0.601779in}{0.451389in}}{\pgfqpoint{2.439626in}{1.396682in}}%
\pgfusepath{clip}%
\pgfsetroundcap%
\pgfsetroundjoin%
\pgfsetlinewidth{1.003750pt}%
\definecolor{currentstroke}{rgb}{0.800000,0.800000,0.800000}%
\pgfsetstrokecolor{currentstroke}%
\pgfsetdash{}{0pt}%
\pgfpathmoveto{\pgfqpoint{0.601779in}{0.451389in}}%
\pgfpathlineto{\pgfqpoint{3.041406in}{0.451389in}}%
\pgfusepath{stroke}%
\end{pgfscope}%
\begin{pgfscope}%
\definecolor{textcolor}{rgb}{0.150000,0.150000,0.150000}%
\pgfsetstrokecolor{textcolor}%
\pgfsetfillcolor{textcolor}%
\pgftext[x=0.400987in, y=0.403903in, left, base]{\color{textcolor}\sffamily\fontsize{9.000000}{10.800000}\selectfont 0}%
\end{pgfscope}%
\begin{pgfscope}%
\pgfpathrectangle{\pgfqpoint{0.601779in}{0.451389in}}{\pgfqpoint{2.439626in}{1.396682in}}%
\pgfusepath{clip}%
\pgfsetroundcap%
\pgfsetroundjoin%
\pgfsetlinewidth{1.003750pt}%
\definecolor{currentstroke}{rgb}{0.800000,0.800000,0.800000}%
\pgfsetstrokecolor{currentstroke}%
\pgfsetdash{}{0pt}%
\pgfpathmoveto{\pgfqpoint{0.601779in}{0.794085in}}%
\pgfpathlineto{\pgfqpoint{3.041406in}{0.794085in}}%
\pgfusepath{stroke}%
\end{pgfscope}%
\begin{pgfscope}%
\definecolor{textcolor}{rgb}{0.150000,0.150000,0.150000}%
\pgfsetstrokecolor{textcolor}%
\pgfsetfillcolor{textcolor}%
\pgftext[x=0.263292in, y=0.746600in, left, base]{\color{textcolor}\sffamily\fontsize{9.000000}{10.800000}\selectfont 250}%
\end{pgfscope}%
\begin{pgfscope}%
\pgfpathrectangle{\pgfqpoint{0.601779in}{0.451389in}}{\pgfqpoint{2.439626in}{1.396682in}}%
\pgfusepath{clip}%
\pgfsetroundcap%
\pgfsetroundjoin%
\pgfsetlinewidth{1.003750pt}%
\definecolor{currentstroke}{rgb}{0.800000,0.800000,0.800000}%
\pgfsetstrokecolor{currentstroke}%
\pgfsetdash{}{0pt}%
\pgfpathmoveto{\pgfqpoint{0.601779in}{1.136782in}}%
\pgfpathlineto{\pgfqpoint{3.041406in}{1.136782in}}%
\pgfusepath{stroke}%
\end{pgfscope}%
\begin{pgfscope}%
\definecolor{textcolor}{rgb}{0.150000,0.150000,0.150000}%
\pgfsetstrokecolor{textcolor}%
\pgfsetfillcolor{textcolor}%
\pgftext[x=0.263292in, y=1.089297in, left, base]{\color{textcolor}\sffamily\fontsize{9.000000}{10.800000}\selectfont 500}%
\end{pgfscope}%
\begin{pgfscope}%
\pgfpathrectangle{\pgfqpoint{0.601779in}{0.451389in}}{\pgfqpoint{2.439626in}{1.396682in}}%
\pgfusepath{clip}%
\pgfsetroundcap%
\pgfsetroundjoin%
\pgfsetlinewidth{1.003750pt}%
\definecolor{currentstroke}{rgb}{0.800000,0.800000,0.800000}%
\pgfsetstrokecolor{currentstroke}%
\pgfsetdash{}{0pt}%
\pgfpathmoveto{\pgfqpoint{0.601779in}{1.479479in}}%
\pgfpathlineto{\pgfqpoint{3.041406in}{1.479479in}}%
\pgfusepath{stroke}%
\end{pgfscope}%
\begin{pgfscope}%
\definecolor{textcolor}{rgb}{0.150000,0.150000,0.150000}%
\pgfsetstrokecolor{textcolor}%
\pgfsetfillcolor{textcolor}%
\pgftext[x=0.263292in, y=1.431993in, left, base]{\color{textcolor}\sffamily\fontsize{9.000000}{10.800000}\selectfont 750}%
\end{pgfscope}%
\begin{pgfscope}%
\pgfpathrectangle{\pgfqpoint{0.601779in}{0.451389in}}{\pgfqpoint{2.439626in}{1.396682in}}%
\pgfusepath{clip}%
\pgfsetroundcap%
\pgfsetroundjoin%
\pgfsetlinewidth{1.003750pt}%
\definecolor{currentstroke}{rgb}{0.800000,0.800000,0.800000}%
\pgfsetstrokecolor{currentstroke}%
\pgfsetdash{}{0pt}%
\pgfpathmoveto{\pgfqpoint{0.601779in}{1.822175in}}%
\pgfpathlineto{\pgfqpoint{3.041406in}{1.822175in}}%
\pgfusepath{stroke}%
\end{pgfscope}%
\begin{pgfscope}%
\definecolor{textcolor}{rgb}{0.150000,0.150000,0.150000}%
\pgfsetstrokecolor{textcolor}%
\pgfsetfillcolor{textcolor}%
\pgftext[x=0.194444in, y=1.774690in, left, base]{\color{textcolor}\sffamily\fontsize{9.000000}{10.800000}\selectfont 1000}%
\end{pgfscope}%
\begin{pgfscope}%
\definecolor{textcolor}{rgb}{0.150000,0.150000,0.150000}%
\pgfsetstrokecolor{textcolor}%
\pgfsetfillcolor{textcolor}%
\pgftext[x=0.125000in,y=1.149730in,,bottom,rotate=90.000000]{\color{textcolor}\sffamily\fontsize{9.000000}{10.800000}\selectfont Time in s}%
\end{pgfscope}%
\begin{pgfscope}%
\pgfpathrectangle{\pgfqpoint{0.601779in}{0.451389in}}{\pgfqpoint{2.439626in}{1.396682in}}%
\pgfusepath{clip}%
\pgfsetbuttcap%
\pgfsetroundjoin%
\definecolor{currentfill}{rgb}{0.003922,0.450980,0.698039}%
\pgfsetfillcolor{currentfill}%
\pgfsetfillopacity{0.200000}%
\pgfsetlinewidth{1.003750pt}%
\definecolor{currentstroke}{rgb}{0.003922,0.450980,0.698039}%
\pgfsetstrokecolor{currentstroke}%
\pgfsetstrokeopacity{0.200000}%
\pgfsetdash{}{0pt}%
\pgfsys@defobject{currentmarker}{\pgfqpoint{0.970786in}{0.475103in}}{\pgfqpoint{2.942805in}{1.782692in}}{%
\pgfpathmoveto{\pgfqpoint{0.970786in}{0.475365in}}%
\pgfpathlineto{\pgfqpoint{0.970786in}{0.475103in}}%
\pgfpathlineto{\pgfqpoint{1.270421in}{0.525314in}}%
\pgfpathlineto{\pgfqpoint{1.532303in}{0.609750in}}%
\pgfpathlineto{\pgfqpoint{1.787477in}{0.734608in}}%
\pgfpathlineto{\pgfqpoint{2.380975in}{1.150136in}}%
\pgfpathlineto{\pgfqpoint{2.942805in}{1.763790in}}%
\pgfpathlineto{\pgfqpoint{2.942805in}{1.782692in}}%
\pgfpathlineto{\pgfqpoint{2.942805in}{1.782692in}}%
\pgfpathlineto{\pgfqpoint{2.380975in}{1.159429in}}%
\pgfpathlineto{\pgfqpoint{1.787477in}{0.735896in}}%
\pgfpathlineto{\pgfqpoint{1.532303in}{0.612478in}}%
\pgfpathlineto{\pgfqpoint{1.270421in}{0.526536in}}%
\pgfpathlineto{\pgfqpoint{0.970786in}{0.475365in}}%
\pgfpathlineto{\pgfqpoint{0.970786in}{0.475365in}}%
\pgfpathclose%
\pgfusepath{stroke,fill}%
}%
\begin{pgfscope}%
\pgfsys@transformshift{0.000000in}{0.000000in}%
\pgfsys@useobject{currentmarker}{}%
\end{pgfscope}%
\end{pgfscope}%
\begin{pgfscope}%
\pgfsetrectcap%
\pgfsetmiterjoin%
\pgfsetlinewidth{1.254687pt}%
\definecolor{currentstroke}{rgb}{0.800000,0.800000,0.800000}%
\pgfsetstrokecolor{currentstroke}%
\pgfsetdash{}{0pt}%
\pgfpathmoveto{\pgfqpoint{0.601779in}{0.451389in}}%
\pgfpathlineto{\pgfqpoint{0.601779in}{1.848071in}}%
\pgfusepath{stroke}%
\end{pgfscope}%
\begin{pgfscope}%
\pgfsetrectcap%
\pgfsetmiterjoin%
\pgfsetlinewidth{1.254687pt}%
\definecolor{currentstroke}{rgb}{0.800000,0.800000,0.800000}%
\pgfsetstrokecolor{currentstroke}%
\pgfsetdash{}{0pt}%
\pgfpathmoveto{\pgfqpoint{3.041406in}{0.451389in}}%
\pgfpathlineto{\pgfqpoint{3.041406in}{1.848071in}}%
\pgfusepath{stroke}%
\end{pgfscope}%
\begin{pgfscope}%
\pgfsetrectcap%
\pgfsetmiterjoin%
\pgfsetlinewidth{1.254687pt}%
\definecolor{currentstroke}{rgb}{0.800000,0.800000,0.800000}%
\pgfsetstrokecolor{currentstroke}%
\pgfsetdash{}{0pt}%
\pgfpathmoveto{\pgfqpoint{0.601779in}{0.451389in}}%
\pgfpathlineto{\pgfqpoint{3.041406in}{0.451389in}}%
\pgfusepath{stroke}%
\end{pgfscope}%
\begin{pgfscope}%
\pgfsetrectcap%
\pgfsetmiterjoin%
\pgfsetlinewidth{1.254687pt}%
\definecolor{currentstroke}{rgb}{0.800000,0.800000,0.800000}%
\pgfsetstrokecolor{currentstroke}%
\pgfsetdash{}{0pt}%
\pgfpathmoveto{\pgfqpoint{0.601779in}{1.848071in}}%
\pgfpathlineto{\pgfqpoint{3.041406in}{1.848071in}}%
\pgfusepath{stroke}%
\end{pgfscope}%
\begin{pgfscope}%
\pgfsetroundcap%
\pgfsetroundjoin%
\pgfsetlinewidth{1.003750pt}%
\definecolor{currentstroke}{rgb}{0.003922,0.450980,0.698039}%
\pgfsetstrokecolor{currentstroke}%
\pgfsetdash{}{0pt}%
\pgfpathmoveto{\pgfqpoint{0.970786in}{0.475218in}}%
\pgfpathlineto{\pgfqpoint{1.270421in}{0.525877in}}%
\pgfpathlineto{\pgfqpoint{1.532303in}{0.611282in}}%
\pgfpathlineto{\pgfqpoint{1.787477in}{0.735321in}}%
\pgfpathlineto{\pgfqpoint{2.380975in}{1.154498in}}%
\pgfpathlineto{\pgfqpoint{2.942805in}{1.771598in}}%
\pgfusepath{stroke}%
\end{pgfscope}%
\begin{pgfscope}%
\pgfsetbuttcap%
\pgfsetroundjoin%
\definecolor{currentfill}{rgb}{0.003922,0.450980,0.698039}%
\pgfsetfillcolor{currentfill}%
\pgfsetlinewidth{0.752812pt}%
\definecolor{currentstroke}{rgb}{1.000000,1.000000,1.000000}%
\pgfsetstrokecolor{currentstroke}%
\pgfsetdash{}{0pt}%
\pgfsys@defobject{currentmarker}{\pgfqpoint{-0.034722in}{-0.034722in}}{\pgfqpoint{0.034722in}{0.034722in}}{%
\pgfpathmoveto{\pgfqpoint{0.000000in}{-0.034722in}}%
\pgfpathcurveto{\pgfqpoint{0.009208in}{-0.034722in}}{\pgfqpoint{0.018041in}{-0.031064in}}{\pgfqpoint{0.024552in}{-0.024552in}}%
\pgfpathcurveto{\pgfqpoint{0.031064in}{-0.018041in}}{\pgfqpoint{0.034722in}{-0.009208in}}{\pgfqpoint{0.034722in}{0.000000in}}%
\pgfpathcurveto{\pgfqpoint{0.034722in}{0.009208in}}{\pgfqpoint{0.031064in}{0.018041in}}{\pgfqpoint{0.024552in}{0.024552in}}%
\pgfpathcurveto{\pgfqpoint{0.018041in}{0.031064in}}{\pgfqpoint{0.009208in}{0.034722in}}{\pgfqpoint{0.000000in}{0.034722in}}%
\pgfpathcurveto{\pgfqpoint{-0.009208in}{0.034722in}}{\pgfqpoint{-0.018041in}{0.031064in}}{\pgfqpoint{-0.024552in}{0.024552in}}%
\pgfpathcurveto{\pgfqpoint{-0.031064in}{0.018041in}}{\pgfqpoint{-0.034722in}{0.009208in}}{\pgfqpoint{-0.034722in}{0.000000in}}%
\pgfpathcurveto{\pgfqpoint{-0.034722in}{-0.009208in}}{\pgfqpoint{-0.031064in}{-0.018041in}}{\pgfqpoint{-0.024552in}{-0.024552in}}%
\pgfpathcurveto{\pgfqpoint{-0.018041in}{-0.031064in}}{\pgfqpoint{-0.009208in}{-0.034722in}}{\pgfqpoint{0.000000in}{-0.034722in}}%
\pgfpathlineto{\pgfqpoint{0.000000in}{-0.034722in}}%
\pgfpathclose%
\pgfusepath{stroke,fill}%
}%
\begin{pgfscope}%
\pgfsys@transformshift{0.970786in}{0.475218in}%
\pgfsys@useobject{currentmarker}{}%
\end{pgfscope}%
\begin{pgfscope}%
\pgfsys@transformshift{1.270421in}{0.525877in}%
\pgfsys@useobject{currentmarker}{}%
\end{pgfscope}%
\begin{pgfscope}%
\pgfsys@transformshift{1.532303in}{0.611282in}%
\pgfsys@useobject{currentmarker}{}%
\end{pgfscope}%
\begin{pgfscope}%
\pgfsys@transformshift{1.787477in}{0.735321in}%
\pgfsys@useobject{currentmarker}{}%
\end{pgfscope}%
\begin{pgfscope}%
\pgfsys@transformshift{2.380975in}{1.154498in}%
\pgfsys@useobject{currentmarker}{}%
\end{pgfscope}%
\begin{pgfscope}%
\pgfsys@transformshift{2.942805in}{1.771598in}%
\pgfsys@useobject{currentmarker}{}%
\end{pgfscope}%
\end{pgfscope}%
\end{pgfpicture}%
\makeatother%
\endgroup%

					\end{figcenter}
					\caption{Absolute total graph generation time for the \enquote{OSM city} dataset.}
					\label{fig:eval-import-city-abs}
				\end{subfigure}
				\\[3ex]
				\begin{subfigure}[t]{\linewidth}
					\begin{figcenter}
						\begingroup%
\makeatletter%
\begin{pgfpicture}%
\pgfpathrectangle{\pgfpointorigin}{\pgfqpoint{3.043196in}{1.867995in}}%
\pgfusepath{use as bounding box}%
\begin{pgfscope}%
\pgfsetbuttcap%
\pgfsetmiterjoin%
\definecolor{currentfill}{rgb}{1.000000,1.000000,1.000000}%
\pgfsetfillcolor{currentfill}%
\pgfsetlinewidth{0.000000pt}%
\definecolor{currentstroke}{rgb}{1.000000,1.000000,1.000000}%
\pgfsetstrokecolor{currentstroke}%
\pgfsetdash{}{0pt}%
\pgfpathmoveto{\pgfqpoint{0.000000in}{0.000000in}}%
\pgfpathlineto{\pgfqpoint{3.043196in}{0.000000in}}%
\pgfpathlineto{\pgfqpoint{3.043196in}{1.867995in}}%
\pgfpathlineto{\pgfqpoint{0.000000in}{1.867995in}}%
\pgfpathlineto{\pgfqpoint{0.000000in}{0.000000in}}%
\pgfpathclose%
\pgfusepath{fill}%
\end{pgfscope}%
\begin{pgfscope}%
\pgfsetbuttcap%
\pgfsetmiterjoin%
\definecolor{currentfill}{rgb}{1.000000,1.000000,1.000000}%
\pgfsetfillcolor{currentfill}%
\pgfsetlinewidth{0.000000pt}%
\definecolor{currentstroke}{rgb}{0.000000,0.000000,0.000000}%
\pgfsetstrokecolor{currentstroke}%
\pgfsetstrokeopacity{0.000000}%
\pgfsetdash{}{0pt}%
\pgfpathmoveto{\pgfqpoint{0.395236in}{0.451389in}}%
\pgfpathlineto{\pgfqpoint{3.043196in}{0.451389in}}%
\pgfpathlineto{\pgfqpoint{3.043196in}{1.867995in}}%
\pgfpathlineto{\pgfqpoint{0.395236in}{1.867995in}}%
\pgfpathlineto{\pgfqpoint{0.395236in}{0.451389in}}%
\pgfpathclose%
\pgfusepath{fill}%
\end{pgfscope}%
\begin{pgfscope}%
\pgfpathrectangle{\pgfqpoint{0.395236in}{0.451389in}}{\pgfqpoint{2.647959in}{1.416606in}}%
\pgfusepath{clip}%
\pgfsetroundcap%
\pgfsetroundjoin%
\pgfsetlinewidth{1.003750pt}%
\definecolor{currentstroke}{rgb}{0.800000,0.800000,0.800000}%
\pgfsetstrokecolor{currentstroke}%
\pgfsetdash{}{0pt}%
\pgfpathmoveto{\pgfqpoint{0.395236in}{0.451389in}}%
\pgfpathlineto{\pgfqpoint{0.395236in}{1.867995in}}%
\pgfusepath{stroke}%
\end{pgfscope}%
\begin{pgfscope}%
\definecolor{textcolor}{rgb}{0.150000,0.150000,0.150000}%
\pgfsetstrokecolor{textcolor}%
\pgfsetfillcolor{textcolor}%
\pgftext[x=0.395236in,y=0.319444in,,top]{\color{textcolor}\sffamily\fontsize{9.000000}{10.800000}\selectfont 0}%
\end{pgfscope}%
\begin{pgfscope}%
\pgfpathrectangle{\pgfqpoint{0.395236in}{0.451389in}}{\pgfqpoint{2.647959in}{1.416606in}}%
\pgfusepath{clip}%
\pgfsetroundcap%
\pgfsetroundjoin%
\pgfsetlinewidth{1.003750pt}%
\definecolor{currentstroke}{rgb}{0.800000,0.800000,0.800000}%
\pgfsetstrokecolor{currentstroke}%
\pgfsetdash{}{0pt}%
\pgfpathmoveto{\pgfqpoint{0.800662in}{0.451389in}}%
\pgfpathlineto{\pgfqpoint{0.800662in}{1.867995in}}%
\pgfusepath{stroke}%
\end{pgfscope}%
\begin{pgfscope}%
\definecolor{textcolor}{rgb}{0.150000,0.150000,0.150000}%
\pgfsetstrokecolor{textcolor}%
\pgfsetfillcolor{textcolor}%
\pgftext[x=0.800662in,y=0.319444in,,top]{\color{textcolor}\sffamily\fontsize{9.000000}{10.800000}\selectfont 5000}%
\end{pgfscope}%
\begin{pgfscope}%
\pgfpathrectangle{\pgfqpoint{0.395236in}{0.451389in}}{\pgfqpoint{2.647959in}{1.416606in}}%
\pgfusepath{clip}%
\pgfsetroundcap%
\pgfsetroundjoin%
\pgfsetlinewidth{1.003750pt}%
\definecolor{currentstroke}{rgb}{0.800000,0.800000,0.800000}%
\pgfsetstrokecolor{currentstroke}%
\pgfsetdash{}{0pt}%
\pgfpathmoveto{\pgfqpoint{1.206089in}{0.451389in}}%
\pgfpathlineto{\pgfqpoint{1.206089in}{1.867995in}}%
\pgfusepath{stroke}%
\end{pgfscope}%
\begin{pgfscope}%
\definecolor{textcolor}{rgb}{0.150000,0.150000,0.150000}%
\pgfsetstrokecolor{textcolor}%
\pgfsetfillcolor{textcolor}%
\pgftext[x=1.206089in,y=0.319444in,,top]{\color{textcolor}\sffamily\fontsize{9.000000}{10.800000}\selectfont 10000}%
\end{pgfscope}%
\begin{pgfscope}%
\pgfpathrectangle{\pgfqpoint{0.395236in}{0.451389in}}{\pgfqpoint{2.647959in}{1.416606in}}%
\pgfusepath{clip}%
\pgfsetroundcap%
\pgfsetroundjoin%
\pgfsetlinewidth{1.003750pt}%
\definecolor{currentstroke}{rgb}{0.800000,0.800000,0.800000}%
\pgfsetstrokecolor{currentstroke}%
\pgfsetdash{}{0pt}%
\pgfpathmoveto{\pgfqpoint{1.611515in}{0.451389in}}%
\pgfpathlineto{\pgfqpoint{1.611515in}{1.867995in}}%
\pgfusepath{stroke}%
\end{pgfscope}%
\begin{pgfscope}%
\definecolor{textcolor}{rgb}{0.150000,0.150000,0.150000}%
\pgfsetstrokecolor{textcolor}%
\pgfsetfillcolor{textcolor}%
\pgftext[x=1.611515in,y=0.319444in,,top]{\color{textcolor}\sffamily\fontsize{9.000000}{10.800000}\selectfont 15000}%
\end{pgfscope}%
\begin{pgfscope}%
\pgfpathrectangle{\pgfqpoint{0.395236in}{0.451389in}}{\pgfqpoint{2.647959in}{1.416606in}}%
\pgfusepath{clip}%
\pgfsetroundcap%
\pgfsetroundjoin%
\pgfsetlinewidth{1.003750pt}%
\definecolor{currentstroke}{rgb}{0.800000,0.800000,0.800000}%
\pgfsetstrokecolor{currentstroke}%
\pgfsetdash{}{0pt}%
\pgfpathmoveto{\pgfqpoint{2.016941in}{0.451389in}}%
\pgfpathlineto{\pgfqpoint{2.016941in}{1.867995in}}%
\pgfusepath{stroke}%
\end{pgfscope}%
\begin{pgfscope}%
\definecolor{textcolor}{rgb}{0.150000,0.150000,0.150000}%
\pgfsetstrokecolor{textcolor}%
\pgfsetfillcolor{textcolor}%
\pgftext[x=2.016941in,y=0.319444in,,top]{\color{textcolor}\sffamily\fontsize{9.000000}{10.800000}\selectfont 20000}%
\end{pgfscope}%
\begin{pgfscope}%
\pgfpathrectangle{\pgfqpoint{0.395236in}{0.451389in}}{\pgfqpoint{2.647959in}{1.416606in}}%
\pgfusepath{clip}%
\pgfsetroundcap%
\pgfsetroundjoin%
\pgfsetlinewidth{1.003750pt}%
\definecolor{currentstroke}{rgb}{0.800000,0.800000,0.800000}%
\pgfsetstrokecolor{currentstroke}%
\pgfsetdash{}{0pt}%
\pgfpathmoveto{\pgfqpoint{2.422367in}{0.451389in}}%
\pgfpathlineto{\pgfqpoint{2.422367in}{1.867995in}}%
\pgfusepath{stroke}%
\end{pgfscope}%
\begin{pgfscope}%
\definecolor{textcolor}{rgb}{0.150000,0.150000,0.150000}%
\pgfsetstrokecolor{textcolor}%
\pgfsetfillcolor{textcolor}%
\pgftext[x=2.422367in,y=0.319444in,,top]{\color{textcolor}\sffamily\fontsize{9.000000}{10.800000}\selectfont 25000}%
\end{pgfscope}%
\begin{pgfscope}%
\pgfpathrectangle{\pgfqpoint{0.395236in}{0.451389in}}{\pgfqpoint{2.647959in}{1.416606in}}%
\pgfusepath{clip}%
\pgfsetroundcap%
\pgfsetroundjoin%
\pgfsetlinewidth{1.003750pt}%
\definecolor{currentstroke}{rgb}{0.800000,0.800000,0.800000}%
\pgfsetstrokecolor{currentstroke}%
\pgfsetdash{}{0pt}%
\pgfpathmoveto{\pgfqpoint{2.827793in}{0.451389in}}%
\pgfpathlineto{\pgfqpoint{2.827793in}{1.867995in}}%
\pgfusepath{stroke}%
\end{pgfscope}%
\begin{pgfscope}%
\definecolor{textcolor}{rgb}{0.150000,0.150000,0.150000}%
\pgfsetstrokecolor{textcolor}%
\pgfsetfillcolor{textcolor}%
\pgftext[x=2.827793in,y=0.319444in,,top]{\color{textcolor}\sffamily\fontsize{9.000000}{10.800000}\selectfont 30000}%
\end{pgfscope}%
\begin{pgfscope}%
\definecolor{textcolor}{rgb}{0.150000,0.150000,0.150000}%
\pgfsetstrokecolor{textcolor}%
\pgfsetfillcolor{textcolor}%
\pgftext[x=1.719216in,y=0.125000in,,top]{\color{textcolor}\sffamily\fontsize{9.000000}{10.800000}\selectfont Input obstacle vertices}%
\end{pgfscope}%
\begin{pgfscope}%
\pgfpathrectangle{\pgfqpoint{0.395236in}{0.451389in}}{\pgfqpoint{2.647959in}{1.416606in}}%
\pgfusepath{clip}%
\pgfsetroundcap%
\pgfsetroundjoin%
\pgfsetlinewidth{1.003750pt}%
\definecolor{currentstroke}{rgb}{0.800000,0.800000,0.800000}%
\pgfsetstrokecolor{currentstroke}%
\pgfsetdash{}{0pt}%
\pgfpathmoveto{\pgfqpoint{0.395236in}{0.451389in}}%
\pgfpathlineto{\pgfqpoint{3.043196in}{0.451389in}}%
\pgfusepath{stroke}%
\end{pgfscope}%
\begin{pgfscope}%
\definecolor{textcolor}{rgb}{0.150000,0.150000,0.150000}%
\pgfsetstrokecolor{textcolor}%
\pgfsetfillcolor{textcolor}%
\pgftext[x=0.194444in, y=0.403903in, left, base]{\color{textcolor}\sffamily\fontsize{9.000000}{10.800000}\selectfont 0}%
\end{pgfscope}%
\begin{pgfscope}%
\pgfpathrectangle{\pgfqpoint{0.395236in}{0.451389in}}{\pgfqpoint{2.647959in}{1.416606in}}%
\pgfusepath{clip}%
\pgfsetroundcap%
\pgfsetroundjoin%
\pgfsetlinewidth{1.003750pt}%
\definecolor{currentstroke}{rgb}{0.800000,0.800000,0.800000}%
\pgfsetstrokecolor{currentstroke}%
\pgfsetdash{}{0pt}%
\pgfpathmoveto{\pgfqpoint{0.395236in}{0.854358in}}%
\pgfpathlineto{\pgfqpoint{3.043196in}{0.854358in}}%
\pgfusepath{stroke}%
\end{pgfscope}%
\begin{pgfscope}%
\definecolor{textcolor}{rgb}{0.150000,0.150000,0.150000}%
\pgfsetstrokecolor{textcolor}%
\pgfsetfillcolor{textcolor}%
\pgftext[x=0.194444in, y=0.806873in, left, base]{\color{textcolor}\sffamily\fontsize{9.000000}{10.800000}\selectfont 2}%
\end{pgfscope}%
\begin{pgfscope}%
\pgfpathrectangle{\pgfqpoint{0.395236in}{0.451389in}}{\pgfqpoint{2.647959in}{1.416606in}}%
\pgfusepath{clip}%
\pgfsetroundcap%
\pgfsetroundjoin%
\pgfsetlinewidth{1.003750pt}%
\definecolor{currentstroke}{rgb}{0.800000,0.800000,0.800000}%
\pgfsetstrokecolor{currentstroke}%
\pgfsetdash{}{0pt}%
\pgfpathmoveto{\pgfqpoint{0.395236in}{1.257328in}}%
\pgfpathlineto{\pgfqpoint{3.043196in}{1.257328in}}%
\pgfusepath{stroke}%
\end{pgfscope}%
\begin{pgfscope}%
\definecolor{textcolor}{rgb}{0.150000,0.150000,0.150000}%
\pgfsetstrokecolor{textcolor}%
\pgfsetfillcolor{textcolor}%
\pgftext[x=0.194444in, y=1.209842in, left, base]{\color{textcolor}\sffamily\fontsize{9.000000}{10.800000}\selectfont 4}%
\end{pgfscope}%
\begin{pgfscope}%
\pgfpathrectangle{\pgfqpoint{0.395236in}{0.451389in}}{\pgfqpoint{2.647959in}{1.416606in}}%
\pgfusepath{clip}%
\pgfsetroundcap%
\pgfsetroundjoin%
\pgfsetlinewidth{1.003750pt}%
\definecolor{currentstroke}{rgb}{0.800000,0.800000,0.800000}%
\pgfsetstrokecolor{currentstroke}%
\pgfsetdash{}{0pt}%
\pgfpathmoveto{\pgfqpoint{0.395236in}{1.660297in}}%
\pgfpathlineto{\pgfqpoint{3.043196in}{1.660297in}}%
\pgfusepath{stroke}%
\end{pgfscope}%
\begin{pgfscope}%
\definecolor{textcolor}{rgb}{0.150000,0.150000,0.150000}%
\pgfsetstrokecolor{textcolor}%
\pgfsetfillcolor{textcolor}%
\pgftext[x=0.194444in, y=1.612812in, left, base]{\color{textcolor}\sffamily\fontsize{9.000000}{10.800000}\selectfont 6}%
\end{pgfscope}%
\begin{pgfscope}%
\definecolor{textcolor}{rgb}{0.150000,0.150000,0.150000}%
\pgfsetstrokecolor{textcolor}%
\pgfsetfillcolor{textcolor}%
\pgftext[x=0.125000in,y=1.159692in,,bottom,rotate=90.000000]{\color{textcolor}\sffamily\fontsize{9.000000}{10.800000}\selectfont Time in ms}%
\end{pgfscope}%
\begin{pgfscope}%
\pgfpathrectangle{\pgfqpoint{0.395236in}{0.451389in}}{\pgfqpoint{2.647959in}{1.416606in}}%
\pgfusepath{clip}%
\pgfsetbuttcap%
\pgfsetroundjoin%
\definecolor{currentfill}{rgb}{0.003922,0.450980,0.698039}%
\pgfsetfillcolor{currentfill}%
\pgfsetfillopacity{0.200000}%
\pgfsetlinewidth{1.003750pt}%
\definecolor{currentstroke}{rgb}{0.003922,0.450980,0.698039}%
\pgfsetstrokecolor{currentstroke}%
\pgfsetstrokeopacity{0.200000}%
\pgfsetdash{}{0pt}%
\pgfsys@defobject{currentmarker}{\pgfqpoint{0.399615in}{0.472653in}}{\pgfqpoint{2.917311in}{1.801550in}}{%
\pgfpathmoveto{\pgfqpoint{0.399615in}{0.483482in}}%
\pgfpathlineto{\pgfqpoint{0.399615in}{0.472653in}}%
\pgfpathlineto{\pgfqpoint{0.412751in}{0.490586in}}%
\pgfpathlineto{\pgfqpoint{0.465294in}{0.494104in}}%
\pgfpathlineto{\pgfqpoint{0.552866in}{0.542933in}}%
\pgfpathlineto{\pgfqpoint{0.675467in}{0.592283in}}%
\pgfpathlineto{\pgfqpoint{0.833097in}{0.656559in}}%
\pgfpathlineto{\pgfqpoint{1.025755in}{0.752049in}}%
\pgfpathlineto{\pgfqpoint{1.253442in}{0.853539in}}%
\pgfpathlineto{\pgfqpoint{1.516158in}{0.976622in}}%
\pgfpathlineto{\pgfqpoint{1.813903in}{1.168476in}}%
\pgfpathlineto{\pgfqpoint{2.146677in}{1.309545in}}%
\pgfpathlineto{\pgfqpoint{2.514480in}{1.529714in}}%
\pgfpathlineto{\pgfqpoint{2.917311in}{1.794932in}}%
\pgfpathlineto{\pgfqpoint{2.917311in}{1.801550in}}%
\pgfpathlineto{\pgfqpoint{2.917311in}{1.801550in}}%
\pgfpathlineto{\pgfqpoint{2.514480in}{1.535942in}}%
\pgfpathlineto{\pgfqpoint{2.146677in}{1.320504in}}%
\pgfpathlineto{\pgfqpoint{1.813903in}{1.179577in}}%
\pgfpathlineto{\pgfqpoint{1.516158in}{0.980885in}}%
\pgfpathlineto{\pgfqpoint{1.253442in}{0.857408in}}%
\pgfpathlineto{\pgfqpoint{1.025755in}{0.753562in}}%
\pgfpathlineto{\pgfqpoint{0.833097in}{0.657935in}}%
\pgfpathlineto{\pgfqpoint{0.675467in}{0.592683in}}%
\pgfpathlineto{\pgfqpoint{0.552866in}{0.543553in}}%
\pgfpathlineto{\pgfqpoint{0.465294in}{0.495044in}}%
\pgfpathlineto{\pgfqpoint{0.412751in}{0.494366in}}%
\pgfpathlineto{\pgfqpoint{0.399615in}{0.483482in}}%
\pgfpathlineto{\pgfqpoint{0.399615in}{0.483482in}}%
\pgfpathclose%
\pgfusepath{stroke,fill}%
}%
\begin{pgfscope}%
\pgfsys@transformshift{0.000000in}{0.000000in}%
\pgfsys@useobject{currentmarker}{}%
\end{pgfscope}%
\end{pgfscope}%
\begin{pgfscope}%
\pgfsetrectcap%
\pgfsetmiterjoin%
\pgfsetlinewidth{1.254687pt}%
\definecolor{currentstroke}{rgb}{0.800000,0.800000,0.800000}%
\pgfsetstrokecolor{currentstroke}%
\pgfsetdash{}{0pt}%
\pgfpathmoveto{\pgfqpoint{0.395236in}{0.451389in}}%
\pgfpathlineto{\pgfqpoint{0.395236in}{1.867995in}}%
\pgfusepath{stroke}%
\end{pgfscope}%
\begin{pgfscope}%
\pgfsetrectcap%
\pgfsetmiterjoin%
\pgfsetlinewidth{1.254687pt}%
\definecolor{currentstroke}{rgb}{0.800000,0.800000,0.800000}%
\pgfsetstrokecolor{currentstroke}%
\pgfsetdash{}{0pt}%
\pgfpathmoveto{\pgfqpoint{3.043196in}{0.451389in}}%
\pgfpathlineto{\pgfqpoint{3.043196in}{1.867995in}}%
\pgfusepath{stroke}%
\end{pgfscope}%
\begin{pgfscope}%
\pgfsetrectcap%
\pgfsetmiterjoin%
\pgfsetlinewidth{1.254687pt}%
\definecolor{currentstroke}{rgb}{0.800000,0.800000,0.800000}%
\pgfsetstrokecolor{currentstroke}%
\pgfsetdash{}{0pt}%
\pgfpathmoveto{\pgfqpoint{0.395236in}{0.451389in}}%
\pgfpathlineto{\pgfqpoint{3.043196in}{0.451389in}}%
\pgfusepath{stroke}%
\end{pgfscope}%
\begin{pgfscope}%
\pgfsetrectcap%
\pgfsetmiterjoin%
\pgfsetlinewidth{1.254687pt}%
\definecolor{currentstroke}{rgb}{0.800000,0.800000,0.800000}%
\pgfsetstrokecolor{currentstroke}%
\pgfsetdash{}{0pt}%
\pgfpathmoveto{\pgfqpoint{0.395236in}{1.867995in}}%
\pgfpathlineto{\pgfqpoint{3.043196in}{1.867995in}}%
\pgfusepath{stroke}%
\end{pgfscope}%
\begin{pgfscope}%
\pgfsetroundcap%
\pgfsetroundjoin%
\pgfsetlinewidth{1.003750pt}%
\definecolor{currentstroke}{rgb}{0.003922,0.450980,0.698039}%
\pgfsetstrokecolor{currentstroke}%
\pgfsetdash{}{0pt}%
\pgfpathmoveto{\pgfqpoint{0.399615in}{0.475615in}}%
\pgfpathlineto{\pgfqpoint{0.412751in}{0.492303in}}%
\pgfpathlineto{\pgfqpoint{0.465294in}{0.494415in}}%
\pgfpathlineto{\pgfqpoint{0.552866in}{0.543255in}}%
\pgfpathlineto{\pgfqpoint{0.675467in}{0.592433in}}%
\pgfpathlineto{\pgfqpoint{0.833097in}{0.657188in}}%
\pgfpathlineto{\pgfqpoint{1.025755in}{0.752761in}}%
\pgfpathlineto{\pgfqpoint{1.253442in}{0.855586in}}%
\pgfpathlineto{\pgfqpoint{1.516158in}{0.978736in}}%
\pgfpathlineto{\pgfqpoint{1.813903in}{1.172913in}}%
\pgfpathlineto{\pgfqpoint{2.146677in}{1.316219in}}%
\pgfpathlineto{\pgfqpoint{2.514480in}{1.533767in}}%
\pgfpathlineto{\pgfqpoint{2.917311in}{1.797740in}}%
\pgfusepath{stroke}%
\end{pgfscope}%
\begin{pgfscope}%
\pgfsetbuttcap%
\pgfsetroundjoin%
\definecolor{currentfill}{rgb}{0.003922,0.450980,0.698039}%
\pgfsetfillcolor{currentfill}%
\pgfsetlinewidth{0.752812pt}%
\definecolor{currentstroke}{rgb}{1.000000,1.000000,1.000000}%
\pgfsetstrokecolor{currentstroke}%
\pgfsetdash{}{0pt}%
\pgfsys@defobject{currentmarker}{\pgfqpoint{-0.034722in}{-0.034722in}}{\pgfqpoint{0.034722in}{0.034722in}}{%
\pgfpathmoveto{\pgfqpoint{0.000000in}{-0.034722in}}%
\pgfpathcurveto{\pgfqpoint{0.009208in}{-0.034722in}}{\pgfqpoint{0.018041in}{-0.031064in}}{\pgfqpoint{0.024552in}{-0.024552in}}%
\pgfpathcurveto{\pgfqpoint{0.031064in}{-0.018041in}}{\pgfqpoint{0.034722in}{-0.009208in}}{\pgfqpoint{0.034722in}{0.000000in}}%
\pgfpathcurveto{\pgfqpoint{0.034722in}{0.009208in}}{\pgfqpoint{0.031064in}{0.018041in}}{\pgfqpoint{0.024552in}{0.024552in}}%
\pgfpathcurveto{\pgfqpoint{0.018041in}{0.031064in}}{\pgfqpoint{0.009208in}{0.034722in}}{\pgfqpoint{0.000000in}{0.034722in}}%
\pgfpathcurveto{\pgfqpoint{-0.009208in}{0.034722in}}{\pgfqpoint{-0.018041in}{0.031064in}}{\pgfqpoint{-0.024552in}{0.024552in}}%
\pgfpathcurveto{\pgfqpoint{-0.031064in}{0.018041in}}{\pgfqpoint{-0.034722in}{0.009208in}}{\pgfqpoint{-0.034722in}{0.000000in}}%
\pgfpathcurveto{\pgfqpoint{-0.034722in}{-0.009208in}}{\pgfqpoint{-0.031064in}{-0.018041in}}{\pgfqpoint{-0.024552in}{-0.024552in}}%
\pgfpathcurveto{\pgfqpoint{-0.018041in}{-0.031064in}}{\pgfqpoint{-0.009208in}{-0.034722in}}{\pgfqpoint{0.000000in}{-0.034722in}}%
\pgfpathlineto{\pgfqpoint{0.000000in}{-0.034722in}}%
\pgfpathclose%
\pgfusepath{stroke,fill}%
}%
\begin{pgfscope}%
\pgfsys@transformshift{0.399615in}{0.475615in}%
\pgfsys@useobject{currentmarker}{}%
\end{pgfscope}%
\begin{pgfscope}%
\pgfsys@transformshift{0.412751in}{0.492303in}%
\pgfsys@useobject{currentmarker}{}%
\end{pgfscope}%
\begin{pgfscope}%
\pgfsys@transformshift{0.465294in}{0.494415in}%
\pgfsys@useobject{currentmarker}{}%
\end{pgfscope}%
\begin{pgfscope}%
\pgfsys@transformshift{0.552866in}{0.543255in}%
\pgfsys@useobject{currentmarker}{}%
\end{pgfscope}%
\begin{pgfscope}%
\pgfsys@transformshift{0.675467in}{0.592433in}%
\pgfsys@useobject{currentmarker}{}%
\end{pgfscope}%
\begin{pgfscope}%
\pgfsys@transformshift{0.833097in}{0.657188in}%
\pgfsys@useobject{currentmarker}{}%
\end{pgfscope}%
\begin{pgfscope}%
\pgfsys@transformshift{1.025755in}{0.752761in}%
\pgfsys@useobject{currentmarker}{}%
\end{pgfscope}%
\begin{pgfscope}%
\pgfsys@transformshift{1.253442in}{0.855586in}%
\pgfsys@useobject{currentmarker}{}%
\end{pgfscope}%
\begin{pgfscope}%
\pgfsys@transformshift{1.516158in}{0.978736in}%
\pgfsys@useobject{currentmarker}{}%
\end{pgfscope}%
\begin{pgfscope}%
\pgfsys@transformshift{1.813903in}{1.172913in}%
\pgfsys@useobject{currentmarker}{}%
\end{pgfscope}%
\begin{pgfscope}%
\pgfsys@transformshift{2.146677in}{1.316219in}%
\pgfsys@useobject{currentmarker}{}%
\end{pgfscope}%
\begin{pgfscope}%
\pgfsys@transformshift{2.514480in}{1.533767in}%
\pgfsys@useobject{currentmarker}{}%
\end{pgfscope}%
\begin{pgfscope}%
\pgfsys@transformshift{2.917311in}{1.797740in}%
\pgfsys@useobject{currentmarker}{}%
\end{pgfscope}%
\end{pgfscope}%
\end{pgfpicture}%
\makeatother%
\endgroup%

					\end{figcenter}
					\caption{Time per input vertex for the \enquote{OSM city} dataset.}
				\end{subfigure}
				\\[3ex]
				\begin{subfigure}[t]{\linewidth}
					\begin{figcenter}
						\begingroup%
\makeatletter%
\begin{pgfpicture}%
\pgfpathrectangle{\pgfpointorigin}{\pgfqpoint{3.042427in}{1.867995in}}%
\pgfusepath{use as bounding box}%
\begin{pgfscope}%
\pgfsetbuttcap%
\pgfsetmiterjoin%
\definecolor{currentfill}{rgb}{1.000000,1.000000,1.000000}%
\pgfsetfillcolor{currentfill}%
\pgfsetlinewidth{0.000000pt}%
\definecolor{currentstroke}{rgb}{1.000000,1.000000,1.000000}%
\pgfsetstrokecolor{currentstroke}%
\pgfsetdash{}{0pt}%
\pgfpathmoveto{\pgfqpoint{0.000000in}{0.000000in}}%
\pgfpathlineto{\pgfqpoint{3.042427in}{0.000000in}}%
\pgfpathlineto{\pgfqpoint{3.042427in}{1.867995in}}%
\pgfpathlineto{\pgfqpoint{0.000000in}{1.867995in}}%
\pgfpathlineto{\pgfqpoint{0.000000in}{0.000000in}}%
\pgfpathclose%
\pgfusepath{fill}%
\end{pgfscope}%
\begin{pgfscope}%
\pgfsetbuttcap%
\pgfsetmiterjoin%
\definecolor{currentfill}{rgb}{1.000000,1.000000,1.000000}%
\pgfsetfillcolor{currentfill}%
\pgfsetlinewidth{0.000000pt}%
\definecolor{currentstroke}{rgb}{0.000000,0.000000,0.000000}%
\pgfsetstrokecolor{currentstroke}%
\pgfsetstrokeopacity{0.000000}%
\pgfsetdash{}{0pt}%
\pgfpathmoveto{\pgfqpoint{0.497592in}{0.451389in}}%
\pgfpathlineto{\pgfqpoint{3.042427in}{0.451389in}}%
\pgfpathlineto{\pgfqpoint{3.042427in}{1.867995in}}%
\pgfpathlineto{\pgfqpoint{0.497592in}{1.867995in}}%
\pgfpathlineto{\pgfqpoint{0.497592in}{0.451389in}}%
\pgfpathclose%
\pgfusepath{fill}%
\end{pgfscope}%
\begin{pgfscope}%
\pgfpathrectangle{\pgfqpoint{0.497592in}{0.451389in}}{\pgfqpoint{2.544834in}{1.416606in}}%
\pgfusepath{clip}%
\pgfsetroundcap%
\pgfsetroundjoin%
\pgfsetlinewidth{1.003750pt}%
\definecolor{currentstroke}{rgb}{0.800000,0.800000,0.800000}%
\pgfsetstrokecolor{currentstroke}%
\pgfsetdash{}{0pt}%
\pgfpathmoveto{\pgfqpoint{0.497592in}{0.451389in}}%
\pgfpathlineto{\pgfqpoint{0.497592in}{1.867995in}}%
\pgfusepath{stroke}%
\end{pgfscope}%
\begin{pgfscope}%
\definecolor{textcolor}{rgb}{0.150000,0.150000,0.150000}%
\pgfsetstrokecolor{textcolor}%
\pgfsetfillcolor{textcolor}%
\pgftext[x=0.497592in,y=0.319444in,,top]{\color{textcolor}\sffamily\fontsize{9.000000}{10.800000}\selectfont 0}%
\end{pgfscope}%
\begin{pgfscope}%
\pgfpathrectangle{\pgfqpoint{0.497592in}{0.451389in}}{\pgfqpoint{2.544834in}{1.416606in}}%
\pgfusepath{clip}%
\pgfsetroundcap%
\pgfsetroundjoin%
\pgfsetlinewidth{1.003750pt}%
\definecolor{currentstroke}{rgb}{0.800000,0.800000,0.800000}%
\pgfsetstrokecolor{currentstroke}%
\pgfsetdash{}{0pt}%
\pgfpathmoveto{\pgfqpoint{1.043720in}{0.451389in}}%
\pgfpathlineto{\pgfqpoint{1.043720in}{1.867995in}}%
\pgfusepath{stroke}%
\end{pgfscope}%
\begin{pgfscope}%
\definecolor{textcolor}{rgb}{0.150000,0.150000,0.150000}%
\pgfsetstrokecolor{textcolor}%
\pgfsetfillcolor{textcolor}%
\pgftext[x=1.043720in,y=0.319444in,,top]{\color{textcolor}\sffamily\fontsize{9.000000}{10.800000}\selectfont 10000}%
\end{pgfscope}%
\begin{pgfscope}%
\pgfpathrectangle{\pgfqpoint{0.497592in}{0.451389in}}{\pgfqpoint{2.544834in}{1.416606in}}%
\pgfusepath{clip}%
\pgfsetroundcap%
\pgfsetroundjoin%
\pgfsetlinewidth{1.003750pt}%
\definecolor{currentstroke}{rgb}{0.800000,0.800000,0.800000}%
\pgfsetstrokecolor{currentstroke}%
\pgfsetdash{}{0pt}%
\pgfpathmoveto{\pgfqpoint{1.589848in}{0.451389in}}%
\pgfpathlineto{\pgfqpoint{1.589848in}{1.867995in}}%
\pgfusepath{stroke}%
\end{pgfscope}%
\begin{pgfscope}%
\definecolor{textcolor}{rgb}{0.150000,0.150000,0.150000}%
\pgfsetstrokecolor{textcolor}%
\pgfsetfillcolor{textcolor}%
\pgftext[x=1.589848in,y=0.319444in,,top]{\color{textcolor}\sffamily\fontsize{9.000000}{10.800000}\selectfont 20000}%
\end{pgfscope}%
\begin{pgfscope}%
\pgfpathrectangle{\pgfqpoint{0.497592in}{0.451389in}}{\pgfqpoint{2.544834in}{1.416606in}}%
\pgfusepath{clip}%
\pgfsetroundcap%
\pgfsetroundjoin%
\pgfsetlinewidth{1.003750pt}%
\definecolor{currentstroke}{rgb}{0.800000,0.800000,0.800000}%
\pgfsetstrokecolor{currentstroke}%
\pgfsetdash{}{0pt}%
\pgfpathmoveto{\pgfqpoint{2.135975in}{0.451389in}}%
\pgfpathlineto{\pgfqpoint{2.135975in}{1.867995in}}%
\pgfusepath{stroke}%
\end{pgfscope}%
\begin{pgfscope}%
\definecolor{textcolor}{rgb}{0.150000,0.150000,0.150000}%
\pgfsetstrokecolor{textcolor}%
\pgfsetfillcolor{textcolor}%
\pgftext[x=2.135975in,y=0.319444in,,top]{\color{textcolor}\sffamily\fontsize{9.000000}{10.800000}\selectfont 30000}%
\end{pgfscope}%
\begin{pgfscope}%
\pgfpathrectangle{\pgfqpoint{0.497592in}{0.451389in}}{\pgfqpoint{2.544834in}{1.416606in}}%
\pgfusepath{clip}%
\pgfsetroundcap%
\pgfsetroundjoin%
\pgfsetlinewidth{1.003750pt}%
\definecolor{currentstroke}{rgb}{0.800000,0.800000,0.800000}%
\pgfsetstrokecolor{currentstroke}%
\pgfsetdash{}{0pt}%
\pgfpathmoveto{\pgfqpoint{2.682103in}{0.451389in}}%
\pgfpathlineto{\pgfqpoint{2.682103in}{1.867995in}}%
\pgfusepath{stroke}%
\end{pgfscope}%
\begin{pgfscope}%
\definecolor{textcolor}{rgb}{0.150000,0.150000,0.150000}%
\pgfsetstrokecolor{textcolor}%
\pgfsetfillcolor{textcolor}%
\pgftext[x=2.682103in,y=0.319444in,,top]{\color{textcolor}\sffamily\fontsize{9.000000}{10.800000}\selectfont 40000}%
\end{pgfscope}%
\begin{pgfscope}%
\definecolor{textcolor}{rgb}{0.150000,0.150000,0.150000}%
\pgfsetstrokecolor{textcolor}%
\pgfsetfillcolor{textcolor}%
\pgftext[x=1.770010in,y=0.125000in,,top]{\color{textcolor}\sffamily\fontsize{9.000000}{10.800000}\selectfont Input obstacle vertices}%
\end{pgfscope}%
\begin{pgfscope}%
\pgfpathrectangle{\pgfqpoint{0.497592in}{0.451389in}}{\pgfqpoint{2.544834in}{1.416606in}}%
\pgfusepath{clip}%
\pgfsetroundcap%
\pgfsetroundjoin%
\pgfsetlinewidth{1.003750pt}%
\definecolor{currentstroke}{rgb}{0.800000,0.800000,0.800000}%
\pgfsetstrokecolor{currentstroke}%
\pgfsetdash{}{0pt}%
\pgfpathmoveto{\pgfqpoint{0.497592in}{0.451389in}}%
\pgfpathlineto{\pgfqpoint{3.042427in}{0.451389in}}%
\pgfusepath{stroke}%
\end{pgfscope}%
\begin{pgfscope}%
\definecolor{textcolor}{rgb}{0.150000,0.150000,0.150000}%
\pgfsetstrokecolor{textcolor}%
\pgfsetfillcolor{textcolor}%
\pgftext[x=0.194444in, y=0.403903in, left, base]{\color{textcolor}\sffamily\fontsize{9.000000}{10.800000}\selectfont 0.0}%
\end{pgfscope}%
\begin{pgfscope}%
\pgfpathrectangle{\pgfqpoint{0.497592in}{0.451389in}}{\pgfqpoint{2.544834in}{1.416606in}}%
\pgfusepath{clip}%
\pgfsetroundcap%
\pgfsetroundjoin%
\pgfsetlinewidth{1.003750pt}%
\definecolor{currentstroke}{rgb}{0.800000,0.800000,0.800000}%
\pgfsetstrokecolor{currentstroke}%
\pgfsetdash{}{0pt}%
\pgfpathmoveto{\pgfqpoint{0.497592in}{0.757507in}}%
\pgfpathlineto{\pgfqpoint{3.042427in}{0.757507in}}%
\pgfusepath{stroke}%
\end{pgfscope}%
\begin{pgfscope}%
\definecolor{textcolor}{rgb}{0.150000,0.150000,0.150000}%
\pgfsetstrokecolor{textcolor}%
\pgfsetfillcolor{textcolor}%
\pgftext[x=0.194444in, y=0.710021in, left, base]{\color{textcolor}\sffamily\fontsize{9.000000}{10.800000}\selectfont 0.1}%
\end{pgfscope}%
\begin{pgfscope}%
\pgfpathrectangle{\pgfqpoint{0.497592in}{0.451389in}}{\pgfqpoint{2.544834in}{1.416606in}}%
\pgfusepath{clip}%
\pgfsetroundcap%
\pgfsetroundjoin%
\pgfsetlinewidth{1.003750pt}%
\definecolor{currentstroke}{rgb}{0.800000,0.800000,0.800000}%
\pgfsetstrokecolor{currentstroke}%
\pgfsetdash{}{0pt}%
\pgfpathmoveto{\pgfqpoint{0.497592in}{1.063625in}}%
\pgfpathlineto{\pgfqpoint{3.042427in}{1.063625in}}%
\pgfusepath{stroke}%
\end{pgfscope}%
\begin{pgfscope}%
\definecolor{textcolor}{rgb}{0.150000,0.150000,0.150000}%
\pgfsetstrokecolor{textcolor}%
\pgfsetfillcolor{textcolor}%
\pgftext[x=0.194444in, y=1.016140in, left, base]{\color{textcolor}\sffamily\fontsize{9.000000}{10.800000}\selectfont 0.2}%
\end{pgfscope}%
\begin{pgfscope}%
\pgfpathrectangle{\pgfqpoint{0.497592in}{0.451389in}}{\pgfqpoint{2.544834in}{1.416606in}}%
\pgfusepath{clip}%
\pgfsetroundcap%
\pgfsetroundjoin%
\pgfsetlinewidth{1.003750pt}%
\definecolor{currentstroke}{rgb}{0.800000,0.800000,0.800000}%
\pgfsetstrokecolor{currentstroke}%
\pgfsetdash{}{0pt}%
\pgfpathmoveto{\pgfqpoint{0.497592in}{1.369743in}}%
\pgfpathlineto{\pgfqpoint{3.042427in}{1.369743in}}%
\pgfusepath{stroke}%
\end{pgfscope}%
\begin{pgfscope}%
\definecolor{textcolor}{rgb}{0.150000,0.150000,0.150000}%
\pgfsetstrokecolor{textcolor}%
\pgfsetfillcolor{textcolor}%
\pgftext[x=0.194444in, y=1.322258in, left, base]{\color{textcolor}\sffamily\fontsize{9.000000}{10.800000}\selectfont 0.3}%
\end{pgfscope}%
\begin{pgfscope}%
\pgfpathrectangle{\pgfqpoint{0.497592in}{0.451389in}}{\pgfqpoint{2.544834in}{1.416606in}}%
\pgfusepath{clip}%
\pgfsetroundcap%
\pgfsetroundjoin%
\pgfsetlinewidth{1.003750pt}%
\definecolor{currentstroke}{rgb}{0.800000,0.800000,0.800000}%
\pgfsetstrokecolor{currentstroke}%
\pgfsetdash{}{0pt}%
\pgfpathmoveto{\pgfqpoint{0.497592in}{1.675862in}}%
\pgfpathlineto{\pgfqpoint{3.042427in}{1.675862in}}%
\pgfusepath{stroke}%
\end{pgfscope}%
\begin{pgfscope}%
\definecolor{textcolor}{rgb}{0.150000,0.150000,0.150000}%
\pgfsetstrokecolor{textcolor}%
\pgfsetfillcolor{textcolor}%
\pgftext[x=0.194444in, y=1.628376in, left, base]{\color{textcolor}\sffamily\fontsize{9.000000}{10.800000}\selectfont 0.4}%
\end{pgfscope}%
\begin{pgfscope}%
\definecolor{textcolor}{rgb}{0.150000,0.150000,0.150000}%
\pgfsetstrokecolor{textcolor}%
\pgfsetfillcolor{textcolor}%
\pgftext[x=0.125000in,y=1.159692in,,bottom,rotate=90.000000]{\color{textcolor}\sffamily\fontsize{9.000000}{10.800000}\selectfont Time in µs}%
\end{pgfscope}%
\begin{pgfscope}%
\pgfpathrectangle{\pgfqpoint{0.497592in}{0.451389in}}{\pgfqpoint{2.544834in}{1.416606in}}%
\pgfusepath{clip}%
\pgfsetbuttcap%
\pgfsetroundjoin%
\definecolor{currentfill}{rgb}{0.003922,0.450980,0.698039}%
\pgfsetfillcolor{currentfill}%
\pgfsetfillopacity{0.200000}%
\pgfsetlinewidth{1.003750pt}%
\definecolor{currentstroke}{rgb}{0.003922,0.450980,0.698039}%
\pgfsetstrokecolor{currentstroke}%
\pgfsetstrokeopacity{0.200000}%
\pgfsetdash{}{0pt}%
\pgfsys@defobject{currentmarker}{\pgfqpoint{0.521840in}{1.283530in}}{\pgfqpoint{2.922399in}{1.840163in}}{%
\pgfpathmoveto{\pgfqpoint{0.521840in}{1.840163in}}%
\pgfpathlineto{\pgfqpoint{0.521840in}{1.685377in}}%
\pgfpathlineto{\pgfqpoint{0.594585in}{1.489045in}}%
\pgfpathlineto{\pgfqpoint{0.715825in}{1.382130in}}%
\pgfpathlineto{\pgfqpoint{0.885561in}{1.312715in}}%
\pgfpathlineto{\pgfqpoint{1.103794in}{1.283530in}}%
\pgfpathlineto{\pgfqpoint{1.370523in}{1.372552in}}%
\pgfpathlineto{\pgfqpoint{1.685748in}{1.323127in}}%
\pgfpathlineto{\pgfqpoint{2.049469in}{1.472716in}}%
\pgfpathlineto{\pgfqpoint{2.461686in}{1.423765in}}%
\pgfpathlineto{\pgfqpoint{2.922399in}{1.670304in}}%
\pgfpathlineto{\pgfqpoint{2.922399in}{1.686277in}}%
\pgfpathlineto{\pgfqpoint{2.922399in}{1.686277in}}%
\pgfpathlineto{\pgfqpoint{2.461686in}{1.430446in}}%
\pgfpathlineto{\pgfqpoint{2.049469in}{1.506497in}}%
\pgfpathlineto{\pgfqpoint{1.685748in}{1.335177in}}%
\pgfpathlineto{\pgfqpoint{1.370523in}{1.386770in}}%
\pgfpathlineto{\pgfqpoint{1.103794in}{1.297994in}}%
\pgfpathlineto{\pgfqpoint{0.885561in}{1.316776in}}%
\pgfpathlineto{\pgfqpoint{0.715825in}{1.405375in}}%
\pgfpathlineto{\pgfqpoint{0.594585in}{1.502386in}}%
\pgfpathlineto{\pgfqpoint{0.521840in}{1.840163in}}%
\pgfpathlineto{\pgfqpoint{0.521840in}{1.840163in}}%
\pgfpathclose%
\pgfusepath{stroke,fill}%
}%
\begin{pgfscope}%
\pgfsys@transformshift{0.000000in}{0.000000in}%
\pgfsys@useobject{currentmarker}{}%
\end{pgfscope}%
\end{pgfscope}%
\begin{pgfscope}%
\pgfsetrectcap%
\pgfsetmiterjoin%
\pgfsetlinewidth{1.254687pt}%
\definecolor{currentstroke}{rgb}{0.800000,0.800000,0.800000}%
\pgfsetstrokecolor{currentstroke}%
\pgfsetdash{}{0pt}%
\pgfpathmoveto{\pgfqpoint{0.497592in}{0.451389in}}%
\pgfpathlineto{\pgfqpoint{0.497592in}{1.867995in}}%
\pgfusepath{stroke}%
\end{pgfscope}%
\begin{pgfscope}%
\pgfsetrectcap%
\pgfsetmiterjoin%
\pgfsetlinewidth{1.254687pt}%
\definecolor{currentstroke}{rgb}{0.800000,0.800000,0.800000}%
\pgfsetstrokecolor{currentstroke}%
\pgfsetdash{}{0pt}%
\pgfpathmoveto{\pgfqpoint{3.042427in}{0.451389in}}%
\pgfpathlineto{\pgfqpoint{3.042427in}{1.867995in}}%
\pgfusepath{stroke}%
\end{pgfscope}%
\begin{pgfscope}%
\pgfsetrectcap%
\pgfsetmiterjoin%
\pgfsetlinewidth{1.254687pt}%
\definecolor{currentstroke}{rgb}{0.800000,0.800000,0.800000}%
\pgfsetstrokecolor{currentstroke}%
\pgfsetdash{}{0pt}%
\pgfpathmoveto{\pgfqpoint{0.497592in}{0.451389in}}%
\pgfpathlineto{\pgfqpoint{3.042427in}{0.451389in}}%
\pgfusepath{stroke}%
\end{pgfscope}%
\begin{pgfscope}%
\pgfsetrectcap%
\pgfsetmiterjoin%
\pgfsetlinewidth{1.254687pt}%
\definecolor{currentstroke}{rgb}{0.800000,0.800000,0.800000}%
\pgfsetstrokecolor{currentstroke}%
\pgfsetdash{}{0pt}%
\pgfpathmoveto{\pgfqpoint{0.497592in}{1.867995in}}%
\pgfpathlineto{\pgfqpoint{3.042427in}{1.867995in}}%
\pgfusepath{stroke}%
\end{pgfscope}%
\begin{pgfscope}%
\pgfsetroundcap%
\pgfsetroundjoin%
\pgfsetlinewidth{1.003750pt}%
\definecolor{currentstroke}{rgb}{0.003922,0.450980,0.698039}%
\pgfsetstrokecolor{currentstroke}%
\pgfsetdash{}{0pt}%
\pgfpathmoveto{\pgfqpoint{0.521840in}{1.763783in}}%
\pgfpathlineto{\pgfqpoint{0.594585in}{1.493641in}}%
\pgfpathlineto{\pgfqpoint{0.715825in}{1.395792in}}%
\pgfpathlineto{\pgfqpoint{0.885561in}{1.314789in}}%
\pgfpathlineto{\pgfqpoint{1.103794in}{1.289377in}}%
\pgfpathlineto{\pgfqpoint{1.370523in}{1.379765in}}%
\pgfpathlineto{\pgfqpoint{1.685748in}{1.330658in}}%
\pgfpathlineto{\pgfqpoint{2.049469in}{1.489351in}}%
\pgfpathlineto{\pgfqpoint{2.461686in}{1.426582in}}%
\pgfpathlineto{\pgfqpoint{2.922399in}{1.676096in}}%
\pgfusepath{stroke}%
\end{pgfscope}%
\begin{pgfscope}%
\pgfsetbuttcap%
\pgfsetroundjoin%
\definecolor{currentfill}{rgb}{0.003922,0.450980,0.698039}%
\pgfsetfillcolor{currentfill}%
\pgfsetlinewidth{0.752812pt}%
\definecolor{currentstroke}{rgb}{1.000000,1.000000,1.000000}%
\pgfsetstrokecolor{currentstroke}%
\pgfsetdash{}{0pt}%
\pgfsys@defobject{currentmarker}{\pgfqpoint{-0.034722in}{-0.034722in}}{\pgfqpoint{0.034722in}{0.034722in}}{%
\pgfpathmoveto{\pgfqpoint{0.000000in}{-0.034722in}}%
\pgfpathcurveto{\pgfqpoint{0.009208in}{-0.034722in}}{\pgfqpoint{0.018041in}{-0.031064in}}{\pgfqpoint{0.024552in}{-0.024552in}}%
\pgfpathcurveto{\pgfqpoint{0.031064in}{-0.018041in}}{\pgfqpoint{0.034722in}{-0.009208in}}{\pgfqpoint{0.034722in}{0.000000in}}%
\pgfpathcurveto{\pgfqpoint{0.034722in}{0.009208in}}{\pgfqpoint{0.031064in}{0.018041in}}{\pgfqpoint{0.024552in}{0.024552in}}%
\pgfpathcurveto{\pgfqpoint{0.018041in}{0.031064in}}{\pgfqpoint{0.009208in}{0.034722in}}{\pgfqpoint{0.000000in}{0.034722in}}%
\pgfpathcurveto{\pgfqpoint{-0.009208in}{0.034722in}}{\pgfqpoint{-0.018041in}{0.031064in}}{\pgfqpoint{-0.024552in}{0.024552in}}%
\pgfpathcurveto{\pgfqpoint{-0.031064in}{0.018041in}}{\pgfqpoint{-0.034722in}{0.009208in}}{\pgfqpoint{-0.034722in}{0.000000in}}%
\pgfpathcurveto{\pgfqpoint{-0.034722in}{-0.009208in}}{\pgfqpoint{-0.031064in}{-0.018041in}}{\pgfqpoint{-0.024552in}{-0.024552in}}%
\pgfpathcurveto{\pgfqpoint{-0.018041in}{-0.031064in}}{\pgfqpoint{-0.009208in}{-0.034722in}}{\pgfqpoint{0.000000in}{-0.034722in}}%
\pgfpathlineto{\pgfqpoint{0.000000in}{-0.034722in}}%
\pgfpathclose%
\pgfusepath{stroke,fill}%
}%
\begin{pgfscope}%
\pgfsys@transformshift{0.521840in}{1.763783in}%
\pgfsys@useobject{currentmarker}{}%
\end{pgfscope}%
\begin{pgfscope}%
\pgfsys@transformshift{0.594585in}{1.493641in}%
\pgfsys@useobject{currentmarker}{}%
\end{pgfscope}%
\begin{pgfscope}%
\pgfsys@transformshift{0.715825in}{1.395792in}%
\pgfsys@useobject{currentmarker}{}%
\end{pgfscope}%
\begin{pgfscope}%
\pgfsys@transformshift{0.885561in}{1.314789in}%
\pgfsys@useobject{currentmarker}{}%
\end{pgfscope}%
\begin{pgfscope}%
\pgfsys@transformshift{1.103794in}{1.289377in}%
\pgfsys@useobject{currentmarker}{}%
\end{pgfscope}%
\begin{pgfscope}%
\pgfsys@transformshift{1.370523in}{1.379765in}%
\pgfsys@useobject{currentmarker}{}%
\end{pgfscope}%
\begin{pgfscope}%
\pgfsys@transformshift{1.685748in}{1.330658in}%
\pgfsys@useobject{currentmarker}{}%
\end{pgfscope}%
\begin{pgfscope}%
\pgfsys@transformshift{2.049469in}{1.489351in}%
\pgfsys@useobject{currentmarker}{}%
\end{pgfscope}%
\begin{pgfscope}%
\pgfsys@transformshift{2.461686in}{1.426582in}%
\pgfsys@useobject{currentmarker}{}%
\end{pgfscope}%
\begin{pgfscope}%
\pgfsys@transformshift{2.922399in}{1.676096in}%
\pgfsys@useobject{currentmarker}{}%
\end{pgfscope}%
\end{pgfscope}%
\end{pgfpicture}%
\makeatother%
\endgroup%

					\end{figcenter}
					\caption{Increase in processing time per vertex when an additional vertex is added.}
					\label{fig:eval-import-city-rel-increase}
				\end{subfigure}
				\caption{Graph generation times using the \enquote{OSM city} dataset.}
				\label{fig:eval-import-city}
			\end{minipage}
			\hfill
			\begin{minipage}{.48\textwidth}
				\begin{subfigure}[t]{\linewidth}
					\begin{figcenter}
						\begingroup%
\makeatletter%
\begin{pgfpicture}%
\pgfpathrectangle{\pgfpointorigin}{\pgfqpoint{3.041406in}{1.869661in}}%
\pgfusepath{use as bounding box}%
\begin{pgfscope}%
\pgfsetbuttcap%
\pgfsetmiterjoin%
\definecolor{currentfill}{rgb}{1.000000,1.000000,1.000000}%
\pgfsetfillcolor{currentfill}%
\pgfsetlinewidth{0.000000pt}%
\definecolor{currentstroke}{rgb}{1.000000,1.000000,1.000000}%
\pgfsetstrokecolor{currentstroke}%
\pgfsetdash{}{0pt}%
\pgfpathmoveto{\pgfqpoint{0.000000in}{0.000000in}}%
\pgfpathlineto{\pgfqpoint{3.041406in}{0.000000in}}%
\pgfpathlineto{\pgfqpoint{3.041406in}{1.869661in}}%
\pgfpathlineto{\pgfqpoint{0.000000in}{1.869661in}}%
\pgfpathlineto{\pgfqpoint{0.000000in}{0.000000in}}%
\pgfpathclose%
\pgfusepath{fill}%
\end{pgfscope}%
\begin{pgfscope}%
\pgfsetbuttcap%
\pgfsetmiterjoin%
\definecolor{currentfill}{rgb}{1.000000,1.000000,1.000000}%
\pgfsetfillcolor{currentfill}%
\pgfsetlinewidth{0.000000pt}%
\definecolor{currentstroke}{rgb}{0.000000,0.000000,0.000000}%
\pgfsetstrokecolor{currentstroke}%
\pgfsetstrokeopacity{0.000000}%
\pgfsetdash{}{0pt}%
\pgfpathmoveto{\pgfqpoint{0.601779in}{0.451389in}}%
\pgfpathlineto{\pgfqpoint{3.041406in}{0.451389in}}%
\pgfpathlineto{\pgfqpoint{3.041406in}{1.848071in}}%
\pgfpathlineto{\pgfqpoint{0.601779in}{1.848071in}}%
\pgfpathlineto{\pgfqpoint{0.601779in}{0.451389in}}%
\pgfpathclose%
\pgfusepath{fill}%
\end{pgfscope}%
\begin{pgfscope}%
\pgfpathrectangle{\pgfqpoint{0.601779in}{0.451389in}}{\pgfqpoint{2.439626in}{1.396682in}}%
\pgfusepath{clip}%
\pgfsetroundcap%
\pgfsetroundjoin%
\pgfsetlinewidth{1.003750pt}%
\definecolor{currentstroke}{rgb}{0.800000,0.800000,0.800000}%
\pgfsetstrokecolor{currentstroke}%
\pgfsetdash{}{0pt}%
\pgfpathmoveto{\pgfqpoint{0.601779in}{0.451389in}}%
\pgfpathlineto{\pgfqpoint{0.601779in}{1.848071in}}%
\pgfusepath{stroke}%
\end{pgfscope}%
\begin{pgfscope}%
\definecolor{textcolor}{rgb}{0.150000,0.150000,0.150000}%
\pgfsetstrokecolor{textcolor}%
\pgfsetfillcolor{textcolor}%
\pgftext[x=0.601779in,y=0.319444in,,top]{\color{textcolor}\sffamily\fontsize{9.000000}{10.800000}\selectfont 0}%
\end{pgfscope}%
\begin{pgfscope}%
\pgfpathrectangle{\pgfqpoint{0.601779in}{0.451389in}}{\pgfqpoint{2.439626in}{1.396682in}}%
\pgfusepath{clip}%
\pgfsetroundcap%
\pgfsetroundjoin%
\pgfsetlinewidth{1.003750pt}%
\definecolor{currentstroke}{rgb}{0.800000,0.800000,0.800000}%
\pgfsetstrokecolor{currentstroke}%
\pgfsetdash{}{0pt}%
\pgfpathmoveto{\pgfqpoint{1.121799in}{0.451389in}}%
\pgfpathlineto{\pgfqpoint{1.121799in}{1.848071in}}%
\pgfusepath{stroke}%
\end{pgfscope}%
\begin{pgfscope}%
\definecolor{textcolor}{rgb}{0.150000,0.150000,0.150000}%
\pgfsetstrokecolor{textcolor}%
\pgfsetfillcolor{textcolor}%
\pgftext[x=1.121799in,y=0.319444in,,top]{\color{textcolor}\sffamily\fontsize{9.000000}{10.800000}\selectfont 10000}%
\end{pgfscope}%
\begin{pgfscope}%
\pgfpathrectangle{\pgfqpoint{0.601779in}{0.451389in}}{\pgfqpoint{2.439626in}{1.396682in}}%
\pgfusepath{clip}%
\pgfsetroundcap%
\pgfsetroundjoin%
\pgfsetlinewidth{1.003750pt}%
\definecolor{currentstroke}{rgb}{0.800000,0.800000,0.800000}%
\pgfsetstrokecolor{currentstroke}%
\pgfsetdash{}{0pt}%
\pgfpathmoveto{\pgfqpoint{1.641819in}{0.451389in}}%
\pgfpathlineto{\pgfqpoint{1.641819in}{1.848071in}}%
\pgfusepath{stroke}%
\end{pgfscope}%
\begin{pgfscope}%
\definecolor{textcolor}{rgb}{0.150000,0.150000,0.150000}%
\pgfsetstrokecolor{textcolor}%
\pgfsetfillcolor{textcolor}%
\pgftext[x=1.641819in,y=0.319444in,,top]{\color{textcolor}\sffamily\fontsize{9.000000}{10.800000}\selectfont 20000}%
\end{pgfscope}%
\begin{pgfscope}%
\pgfpathrectangle{\pgfqpoint{0.601779in}{0.451389in}}{\pgfqpoint{2.439626in}{1.396682in}}%
\pgfusepath{clip}%
\pgfsetroundcap%
\pgfsetroundjoin%
\pgfsetlinewidth{1.003750pt}%
\definecolor{currentstroke}{rgb}{0.800000,0.800000,0.800000}%
\pgfsetstrokecolor{currentstroke}%
\pgfsetdash{}{0pt}%
\pgfpathmoveto{\pgfqpoint{2.161839in}{0.451389in}}%
\pgfpathlineto{\pgfqpoint{2.161839in}{1.848071in}}%
\pgfusepath{stroke}%
\end{pgfscope}%
\begin{pgfscope}%
\definecolor{textcolor}{rgb}{0.150000,0.150000,0.150000}%
\pgfsetstrokecolor{textcolor}%
\pgfsetfillcolor{textcolor}%
\pgftext[x=2.161839in,y=0.319444in,,top]{\color{textcolor}\sffamily\fontsize{9.000000}{10.800000}\selectfont 30000}%
\end{pgfscope}%
\begin{pgfscope}%
\pgfpathrectangle{\pgfqpoint{0.601779in}{0.451389in}}{\pgfqpoint{2.439626in}{1.396682in}}%
\pgfusepath{clip}%
\pgfsetroundcap%
\pgfsetroundjoin%
\pgfsetlinewidth{1.003750pt}%
\definecolor{currentstroke}{rgb}{0.800000,0.800000,0.800000}%
\pgfsetstrokecolor{currentstroke}%
\pgfsetdash{}{0pt}%
\pgfpathmoveto{\pgfqpoint{2.681859in}{0.451389in}}%
\pgfpathlineto{\pgfqpoint{2.681859in}{1.848071in}}%
\pgfusepath{stroke}%
\end{pgfscope}%
\begin{pgfscope}%
\definecolor{textcolor}{rgb}{0.150000,0.150000,0.150000}%
\pgfsetstrokecolor{textcolor}%
\pgfsetfillcolor{textcolor}%
\pgftext[x=2.681859in,y=0.319444in,,top]{\color{textcolor}\sffamily\fontsize{9.000000}{10.800000}\selectfont 40000}%
\end{pgfscope}%
\begin{pgfscope}%
\definecolor{textcolor}{rgb}{0.150000,0.150000,0.150000}%
\pgfsetstrokecolor{textcolor}%
\pgfsetfillcolor{textcolor}%
\pgftext[x=1.821593in,y=0.125000in,,top]{\color{textcolor}\sffamily\fontsize{9.000000}{10.800000}\selectfont Input obstacle vertices}%
\end{pgfscope}%
\begin{pgfscope}%
\pgfpathrectangle{\pgfqpoint{0.601779in}{0.451389in}}{\pgfqpoint{2.439626in}{1.396682in}}%
\pgfusepath{clip}%
\pgfsetroundcap%
\pgfsetroundjoin%
\pgfsetlinewidth{1.003750pt}%
\definecolor{currentstroke}{rgb}{0.800000,0.800000,0.800000}%
\pgfsetstrokecolor{currentstroke}%
\pgfsetdash{}{0pt}%
\pgfpathmoveto{\pgfqpoint{0.601779in}{0.451389in}}%
\pgfpathlineto{\pgfqpoint{3.041406in}{0.451389in}}%
\pgfusepath{stroke}%
\end{pgfscope}%
\begin{pgfscope}%
\definecolor{textcolor}{rgb}{0.150000,0.150000,0.150000}%
\pgfsetstrokecolor{textcolor}%
\pgfsetfillcolor{textcolor}%
\pgftext[x=0.400987in, y=0.403903in, left, base]{\color{textcolor}\sffamily\fontsize{9.000000}{10.800000}\selectfont 0}%
\end{pgfscope}%
\begin{pgfscope}%
\pgfpathrectangle{\pgfqpoint{0.601779in}{0.451389in}}{\pgfqpoint{2.439626in}{1.396682in}}%
\pgfusepath{clip}%
\pgfsetroundcap%
\pgfsetroundjoin%
\pgfsetlinewidth{1.003750pt}%
\definecolor{currentstroke}{rgb}{0.800000,0.800000,0.800000}%
\pgfsetstrokecolor{currentstroke}%
\pgfsetdash{}{0pt}%
\pgfpathmoveto{\pgfqpoint{0.601779in}{0.794085in}}%
\pgfpathlineto{\pgfqpoint{3.041406in}{0.794085in}}%
\pgfusepath{stroke}%
\end{pgfscope}%
\begin{pgfscope}%
\definecolor{textcolor}{rgb}{0.150000,0.150000,0.150000}%
\pgfsetstrokecolor{textcolor}%
\pgfsetfillcolor{textcolor}%
\pgftext[x=0.263292in, y=0.746600in, left, base]{\color{textcolor}\sffamily\fontsize{9.000000}{10.800000}\selectfont 250}%
\end{pgfscope}%
\begin{pgfscope}%
\pgfpathrectangle{\pgfqpoint{0.601779in}{0.451389in}}{\pgfqpoint{2.439626in}{1.396682in}}%
\pgfusepath{clip}%
\pgfsetroundcap%
\pgfsetroundjoin%
\pgfsetlinewidth{1.003750pt}%
\definecolor{currentstroke}{rgb}{0.800000,0.800000,0.800000}%
\pgfsetstrokecolor{currentstroke}%
\pgfsetdash{}{0pt}%
\pgfpathmoveto{\pgfqpoint{0.601779in}{1.136782in}}%
\pgfpathlineto{\pgfqpoint{3.041406in}{1.136782in}}%
\pgfusepath{stroke}%
\end{pgfscope}%
\begin{pgfscope}%
\definecolor{textcolor}{rgb}{0.150000,0.150000,0.150000}%
\pgfsetstrokecolor{textcolor}%
\pgfsetfillcolor{textcolor}%
\pgftext[x=0.263292in, y=1.089297in, left, base]{\color{textcolor}\sffamily\fontsize{9.000000}{10.800000}\selectfont 500}%
\end{pgfscope}%
\begin{pgfscope}%
\pgfpathrectangle{\pgfqpoint{0.601779in}{0.451389in}}{\pgfqpoint{2.439626in}{1.396682in}}%
\pgfusepath{clip}%
\pgfsetroundcap%
\pgfsetroundjoin%
\pgfsetlinewidth{1.003750pt}%
\definecolor{currentstroke}{rgb}{0.800000,0.800000,0.800000}%
\pgfsetstrokecolor{currentstroke}%
\pgfsetdash{}{0pt}%
\pgfpathmoveto{\pgfqpoint{0.601779in}{1.479479in}}%
\pgfpathlineto{\pgfqpoint{3.041406in}{1.479479in}}%
\pgfusepath{stroke}%
\end{pgfscope}%
\begin{pgfscope}%
\definecolor{textcolor}{rgb}{0.150000,0.150000,0.150000}%
\pgfsetstrokecolor{textcolor}%
\pgfsetfillcolor{textcolor}%
\pgftext[x=0.263292in, y=1.431993in, left, base]{\color{textcolor}\sffamily\fontsize{9.000000}{10.800000}\selectfont 750}%
\end{pgfscope}%
\begin{pgfscope}%
\pgfpathrectangle{\pgfqpoint{0.601779in}{0.451389in}}{\pgfqpoint{2.439626in}{1.396682in}}%
\pgfusepath{clip}%
\pgfsetroundcap%
\pgfsetroundjoin%
\pgfsetlinewidth{1.003750pt}%
\definecolor{currentstroke}{rgb}{0.800000,0.800000,0.800000}%
\pgfsetstrokecolor{currentstroke}%
\pgfsetdash{}{0pt}%
\pgfpathmoveto{\pgfqpoint{0.601779in}{1.822175in}}%
\pgfpathlineto{\pgfqpoint{3.041406in}{1.822175in}}%
\pgfusepath{stroke}%
\end{pgfscope}%
\begin{pgfscope}%
\definecolor{textcolor}{rgb}{0.150000,0.150000,0.150000}%
\pgfsetstrokecolor{textcolor}%
\pgfsetfillcolor{textcolor}%
\pgftext[x=0.194444in, y=1.774690in, left, base]{\color{textcolor}\sffamily\fontsize{9.000000}{10.800000}\selectfont 1000}%
\end{pgfscope}%
\begin{pgfscope}%
\definecolor{textcolor}{rgb}{0.150000,0.150000,0.150000}%
\pgfsetstrokecolor{textcolor}%
\pgfsetfillcolor{textcolor}%
\pgftext[x=0.125000in,y=1.149730in,,bottom,rotate=90.000000]{\color{textcolor}\sffamily\fontsize{9.000000}{10.800000}\selectfont Time in s}%
\end{pgfscope}%
\begin{pgfscope}%
\pgfpathrectangle{\pgfqpoint{0.601779in}{0.451389in}}{\pgfqpoint{2.439626in}{1.396682in}}%
\pgfusepath{clip}%
\pgfsetbuttcap%
\pgfsetroundjoin%
\definecolor{currentfill}{rgb}{0.003922,0.450980,0.698039}%
\pgfsetfillcolor{currentfill}%
\pgfsetfillopacity{0.200000}%
\pgfsetlinewidth{1.003750pt}%
\definecolor{currentstroke}{rgb}{0.003922,0.450980,0.698039}%
\pgfsetstrokecolor{currentstroke}%
\pgfsetstrokeopacity{0.200000}%
\pgfsetdash{}{0pt}%
\pgfsys@defobject{currentmarker}{\pgfqpoint{0.970786in}{0.475103in}}{\pgfqpoint{2.942805in}{1.782692in}}{%
\pgfpathmoveto{\pgfqpoint{0.970786in}{0.475365in}}%
\pgfpathlineto{\pgfqpoint{0.970786in}{0.475103in}}%
\pgfpathlineto{\pgfqpoint{1.270421in}{0.525314in}}%
\pgfpathlineto{\pgfqpoint{1.532303in}{0.609750in}}%
\pgfpathlineto{\pgfqpoint{1.787477in}{0.734608in}}%
\pgfpathlineto{\pgfqpoint{2.380975in}{1.150136in}}%
\pgfpathlineto{\pgfqpoint{2.942805in}{1.763790in}}%
\pgfpathlineto{\pgfqpoint{2.942805in}{1.782692in}}%
\pgfpathlineto{\pgfqpoint{2.942805in}{1.782692in}}%
\pgfpathlineto{\pgfqpoint{2.380975in}{1.159429in}}%
\pgfpathlineto{\pgfqpoint{1.787477in}{0.735896in}}%
\pgfpathlineto{\pgfqpoint{1.532303in}{0.612478in}}%
\pgfpathlineto{\pgfqpoint{1.270421in}{0.526536in}}%
\pgfpathlineto{\pgfqpoint{0.970786in}{0.475365in}}%
\pgfpathlineto{\pgfqpoint{0.970786in}{0.475365in}}%
\pgfpathclose%
\pgfusepath{stroke,fill}%
}%
\begin{pgfscope}%
\pgfsys@transformshift{0.000000in}{0.000000in}%
\pgfsys@useobject{currentmarker}{}%
\end{pgfscope}%
\end{pgfscope}%
\begin{pgfscope}%
\pgfsetrectcap%
\pgfsetmiterjoin%
\pgfsetlinewidth{1.254687pt}%
\definecolor{currentstroke}{rgb}{0.800000,0.800000,0.800000}%
\pgfsetstrokecolor{currentstroke}%
\pgfsetdash{}{0pt}%
\pgfpathmoveto{\pgfqpoint{0.601779in}{0.451389in}}%
\pgfpathlineto{\pgfqpoint{0.601779in}{1.848071in}}%
\pgfusepath{stroke}%
\end{pgfscope}%
\begin{pgfscope}%
\pgfsetrectcap%
\pgfsetmiterjoin%
\pgfsetlinewidth{1.254687pt}%
\definecolor{currentstroke}{rgb}{0.800000,0.800000,0.800000}%
\pgfsetstrokecolor{currentstroke}%
\pgfsetdash{}{0pt}%
\pgfpathmoveto{\pgfqpoint{3.041406in}{0.451389in}}%
\pgfpathlineto{\pgfqpoint{3.041406in}{1.848071in}}%
\pgfusepath{stroke}%
\end{pgfscope}%
\begin{pgfscope}%
\pgfsetrectcap%
\pgfsetmiterjoin%
\pgfsetlinewidth{1.254687pt}%
\definecolor{currentstroke}{rgb}{0.800000,0.800000,0.800000}%
\pgfsetstrokecolor{currentstroke}%
\pgfsetdash{}{0pt}%
\pgfpathmoveto{\pgfqpoint{0.601779in}{0.451389in}}%
\pgfpathlineto{\pgfqpoint{3.041406in}{0.451389in}}%
\pgfusepath{stroke}%
\end{pgfscope}%
\begin{pgfscope}%
\pgfsetrectcap%
\pgfsetmiterjoin%
\pgfsetlinewidth{1.254687pt}%
\definecolor{currentstroke}{rgb}{0.800000,0.800000,0.800000}%
\pgfsetstrokecolor{currentstroke}%
\pgfsetdash{}{0pt}%
\pgfpathmoveto{\pgfqpoint{0.601779in}{1.848071in}}%
\pgfpathlineto{\pgfqpoint{3.041406in}{1.848071in}}%
\pgfusepath{stroke}%
\end{pgfscope}%
\begin{pgfscope}%
\pgfsetroundcap%
\pgfsetroundjoin%
\pgfsetlinewidth{1.003750pt}%
\definecolor{currentstroke}{rgb}{0.003922,0.450980,0.698039}%
\pgfsetstrokecolor{currentstroke}%
\pgfsetdash{}{0pt}%
\pgfpathmoveto{\pgfqpoint{0.970786in}{0.475218in}}%
\pgfpathlineto{\pgfqpoint{1.270421in}{0.525877in}}%
\pgfpathlineto{\pgfqpoint{1.532303in}{0.611282in}}%
\pgfpathlineto{\pgfqpoint{1.787477in}{0.735321in}}%
\pgfpathlineto{\pgfqpoint{2.380975in}{1.154498in}}%
\pgfpathlineto{\pgfqpoint{2.942805in}{1.771598in}}%
\pgfusepath{stroke}%
\end{pgfscope}%
\begin{pgfscope}%
\pgfsetbuttcap%
\pgfsetroundjoin%
\definecolor{currentfill}{rgb}{0.003922,0.450980,0.698039}%
\pgfsetfillcolor{currentfill}%
\pgfsetlinewidth{0.752812pt}%
\definecolor{currentstroke}{rgb}{1.000000,1.000000,1.000000}%
\pgfsetstrokecolor{currentstroke}%
\pgfsetdash{}{0pt}%
\pgfsys@defobject{currentmarker}{\pgfqpoint{-0.034722in}{-0.034722in}}{\pgfqpoint{0.034722in}{0.034722in}}{%
\pgfpathmoveto{\pgfqpoint{0.000000in}{-0.034722in}}%
\pgfpathcurveto{\pgfqpoint{0.009208in}{-0.034722in}}{\pgfqpoint{0.018041in}{-0.031064in}}{\pgfqpoint{0.024552in}{-0.024552in}}%
\pgfpathcurveto{\pgfqpoint{0.031064in}{-0.018041in}}{\pgfqpoint{0.034722in}{-0.009208in}}{\pgfqpoint{0.034722in}{0.000000in}}%
\pgfpathcurveto{\pgfqpoint{0.034722in}{0.009208in}}{\pgfqpoint{0.031064in}{0.018041in}}{\pgfqpoint{0.024552in}{0.024552in}}%
\pgfpathcurveto{\pgfqpoint{0.018041in}{0.031064in}}{\pgfqpoint{0.009208in}{0.034722in}}{\pgfqpoint{0.000000in}{0.034722in}}%
\pgfpathcurveto{\pgfqpoint{-0.009208in}{0.034722in}}{\pgfqpoint{-0.018041in}{0.031064in}}{\pgfqpoint{-0.024552in}{0.024552in}}%
\pgfpathcurveto{\pgfqpoint{-0.031064in}{0.018041in}}{\pgfqpoint{-0.034722in}{0.009208in}}{\pgfqpoint{-0.034722in}{0.000000in}}%
\pgfpathcurveto{\pgfqpoint{-0.034722in}{-0.009208in}}{\pgfqpoint{-0.031064in}{-0.018041in}}{\pgfqpoint{-0.024552in}{-0.024552in}}%
\pgfpathcurveto{\pgfqpoint{-0.018041in}{-0.031064in}}{\pgfqpoint{-0.009208in}{-0.034722in}}{\pgfqpoint{0.000000in}{-0.034722in}}%
\pgfpathlineto{\pgfqpoint{0.000000in}{-0.034722in}}%
\pgfpathclose%
\pgfusepath{stroke,fill}%
}%
\begin{pgfscope}%
\pgfsys@transformshift{0.970786in}{0.475218in}%
\pgfsys@useobject{currentmarker}{}%
\end{pgfscope}%
\begin{pgfscope}%
\pgfsys@transformshift{1.270421in}{0.525877in}%
\pgfsys@useobject{currentmarker}{}%
\end{pgfscope}%
\begin{pgfscope}%
\pgfsys@transformshift{1.532303in}{0.611282in}%
\pgfsys@useobject{currentmarker}{}%
\end{pgfscope}%
\begin{pgfscope}%
\pgfsys@transformshift{1.787477in}{0.735321in}%
\pgfsys@useobject{currentmarker}{}%
\end{pgfscope}%
\begin{pgfscope}%
\pgfsys@transformshift{2.380975in}{1.154498in}%
\pgfsys@useobject{currentmarker}{}%
\end{pgfscope}%
\begin{pgfscope}%
\pgfsys@transformshift{2.942805in}{1.771598in}%
\pgfsys@useobject{currentmarker}{}%
\end{pgfscope}%
\end{pgfscope}%
\end{pgfpicture}%
\makeatother%
\endgroup%

					\end{figcenter}
					\caption{Absolute total graph generation time for the \enquote{OSM rural} dataset.}
					\label{fig:eval-import-rural-abs}
				\end{subfigure}
				\\[3ex]
				\begin{subfigure}[t]{\linewidth}
					\begin{figcenter}
						\begingroup%
\makeatletter%
\begin{pgfpicture}%
\pgfpathrectangle{\pgfpointorigin}{\pgfqpoint{3.043196in}{1.867995in}}%
\pgfusepath{use as bounding box}%
\begin{pgfscope}%
\pgfsetbuttcap%
\pgfsetmiterjoin%
\definecolor{currentfill}{rgb}{1.000000,1.000000,1.000000}%
\pgfsetfillcolor{currentfill}%
\pgfsetlinewidth{0.000000pt}%
\definecolor{currentstroke}{rgb}{1.000000,1.000000,1.000000}%
\pgfsetstrokecolor{currentstroke}%
\pgfsetdash{}{0pt}%
\pgfpathmoveto{\pgfqpoint{0.000000in}{0.000000in}}%
\pgfpathlineto{\pgfqpoint{3.043196in}{0.000000in}}%
\pgfpathlineto{\pgfqpoint{3.043196in}{1.867995in}}%
\pgfpathlineto{\pgfqpoint{0.000000in}{1.867995in}}%
\pgfpathlineto{\pgfqpoint{0.000000in}{0.000000in}}%
\pgfpathclose%
\pgfusepath{fill}%
\end{pgfscope}%
\begin{pgfscope}%
\pgfsetbuttcap%
\pgfsetmiterjoin%
\definecolor{currentfill}{rgb}{1.000000,1.000000,1.000000}%
\pgfsetfillcolor{currentfill}%
\pgfsetlinewidth{0.000000pt}%
\definecolor{currentstroke}{rgb}{0.000000,0.000000,0.000000}%
\pgfsetstrokecolor{currentstroke}%
\pgfsetstrokeopacity{0.000000}%
\pgfsetdash{}{0pt}%
\pgfpathmoveto{\pgfqpoint{0.395236in}{0.451389in}}%
\pgfpathlineto{\pgfqpoint{3.043196in}{0.451389in}}%
\pgfpathlineto{\pgfqpoint{3.043196in}{1.867995in}}%
\pgfpathlineto{\pgfqpoint{0.395236in}{1.867995in}}%
\pgfpathlineto{\pgfqpoint{0.395236in}{0.451389in}}%
\pgfpathclose%
\pgfusepath{fill}%
\end{pgfscope}%
\begin{pgfscope}%
\pgfpathrectangle{\pgfqpoint{0.395236in}{0.451389in}}{\pgfqpoint{2.647959in}{1.416606in}}%
\pgfusepath{clip}%
\pgfsetroundcap%
\pgfsetroundjoin%
\pgfsetlinewidth{1.003750pt}%
\definecolor{currentstroke}{rgb}{0.800000,0.800000,0.800000}%
\pgfsetstrokecolor{currentstroke}%
\pgfsetdash{}{0pt}%
\pgfpathmoveto{\pgfqpoint{0.395236in}{0.451389in}}%
\pgfpathlineto{\pgfqpoint{0.395236in}{1.867995in}}%
\pgfusepath{stroke}%
\end{pgfscope}%
\begin{pgfscope}%
\definecolor{textcolor}{rgb}{0.150000,0.150000,0.150000}%
\pgfsetstrokecolor{textcolor}%
\pgfsetfillcolor{textcolor}%
\pgftext[x=0.395236in,y=0.319444in,,top]{\color{textcolor}\sffamily\fontsize{9.000000}{10.800000}\selectfont 0}%
\end{pgfscope}%
\begin{pgfscope}%
\pgfpathrectangle{\pgfqpoint{0.395236in}{0.451389in}}{\pgfqpoint{2.647959in}{1.416606in}}%
\pgfusepath{clip}%
\pgfsetroundcap%
\pgfsetroundjoin%
\pgfsetlinewidth{1.003750pt}%
\definecolor{currentstroke}{rgb}{0.800000,0.800000,0.800000}%
\pgfsetstrokecolor{currentstroke}%
\pgfsetdash{}{0pt}%
\pgfpathmoveto{\pgfqpoint{0.800662in}{0.451389in}}%
\pgfpathlineto{\pgfqpoint{0.800662in}{1.867995in}}%
\pgfusepath{stroke}%
\end{pgfscope}%
\begin{pgfscope}%
\definecolor{textcolor}{rgb}{0.150000,0.150000,0.150000}%
\pgfsetstrokecolor{textcolor}%
\pgfsetfillcolor{textcolor}%
\pgftext[x=0.800662in,y=0.319444in,,top]{\color{textcolor}\sffamily\fontsize{9.000000}{10.800000}\selectfont 5000}%
\end{pgfscope}%
\begin{pgfscope}%
\pgfpathrectangle{\pgfqpoint{0.395236in}{0.451389in}}{\pgfqpoint{2.647959in}{1.416606in}}%
\pgfusepath{clip}%
\pgfsetroundcap%
\pgfsetroundjoin%
\pgfsetlinewidth{1.003750pt}%
\definecolor{currentstroke}{rgb}{0.800000,0.800000,0.800000}%
\pgfsetstrokecolor{currentstroke}%
\pgfsetdash{}{0pt}%
\pgfpathmoveto{\pgfqpoint{1.206089in}{0.451389in}}%
\pgfpathlineto{\pgfqpoint{1.206089in}{1.867995in}}%
\pgfusepath{stroke}%
\end{pgfscope}%
\begin{pgfscope}%
\definecolor{textcolor}{rgb}{0.150000,0.150000,0.150000}%
\pgfsetstrokecolor{textcolor}%
\pgfsetfillcolor{textcolor}%
\pgftext[x=1.206089in,y=0.319444in,,top]{\color{textcolor}\sffamily\fontsize{9.000000}{10.800000}\selectfont 10000}%
\end{pgfscope}%
\begin{pgfscope}%
\pgfpathrectangle{\pgfqpoint{0.395236in}{0.451389in}}{\pgfqpoint{2.647959in}{1.416606in}}%
\pgfusepath{clip}%
\pgfsetroundcap%
\pgfsetroundjoin%
\pgfsetlinewidth{1.003750pt}%
\definecolor{currentstroke}{rgb}{0.800000,0.800000,0.800000}%
\pgfsetstrokecolor{currentstroke}%
\pgfsetdash{}{0pt}%
\pgfpathmoveto{\pgfqpoint{1.611515in}{0.451389in}}%
\pgfpathlineto{\pgfqpoint{1.611515in}{1.867995in}}%
\pgfusepath{stroke}%
\end{pgfscope}%
\begin{pgfscope}%
\definecolor{textcolor}{rgb}{0.150000,0.150000,0.150000}%
\pgfsetstrokecolor{textcolor}%
\pgfsetfillcolor{textcolor}%
\pgftext[x=1.611515in,y=0.319444in,,top]{\color{textcolor}\sffamily\fontsize{9.000000}{10.800000}\selectfont 15000}%
\end{pgfscope}%
\begin{pgfscope}%
\pgfpathrectangle{\pgfqpoint{0.395236in}{0.451389in}}{\pgfqpoint{2.647959in}{1.416606in}}%
\pgfusepath{clip}%
\pgfsetroundcap%
\pgfsetroundjoin%
\pgfsetlinewidth{1.003750pt}%
\definecolor{currentstroke}{rgb}{0.800000,0.800000,0.800000}%
\pgfsetstrokecolor{currentstroke}%
\pgfsetdash{}{0pt}%
\pgfpathmoveto{\pgfqpoint{2.016941in}{0.451389in}}%
\pgfpathlineto{\pgfqpoint{2.016941in}{1.867995in}}%
\pgfusepath{stroke}%
\end{pgfscope}%
\begin{pgfscope}%
\definecolor{textcolor}{rgb}{0.150000,0.150000,0.150000}%
\pgfsetstrokecolor{textcolor}%
\pgfsetfillcolor{textcolor}%
\pgftext[x=2.016941in,y=0.319444in,,top]{\color{textcolor}\sffamily\fontsize{9.000000}{10.800000}\selectfont 20000}%
\end{pgfscope}%
\begin{pgfscope}%
\pgfpathrectangle{\pgfqpoint{0.395236in}{0.451389in}}{\pgfqpoint{2.647959in}{1.416606in}}%
\pgfusepath{clip}%
\pgfsetroundcap%
\pgfsetroundjoin%
\pgfsetlinewidth{1.003750pt}%
\definecolor{currentstroke}{rgb}{0.800000,0.800000,0.800000}%
\pgfsetstrokecolor{currentstroke}%
\pgfsetdash{}{0pt}%
\pgfpathmoveto{\pgfqpoint{2.422367in}{0.451389in}}%
\pgfpathlineto{\pgfqpoint{2.422367in}{1.867995in}}%
\pgfusepath{stroke}%
\end{pgfscope}%
\begin{pgfscope}%
\definecolor{textcolor}{rgb}{0.150000,0.150000,0.150000}%
\pgfsetstrokecolor{textcolor}%
\pgfsetfillcolor{textcolor}%
\pgftext[x=2.422367in,y=0.319444in,,top]{\color{textcolor}\sffamily\fontsize{9.000000}{10.800000}\selectfont 25000}%
\end{pgfscope}%
\begin{pgfscope}%
\pgfpathrectangle{\pgfqpoint{0.395236in}{0.451389in}}{\pgfqpoint{2.647959in}{1.416606in}}%
\pgfusepath{clip}%
\pgfsetroundcap%
\pgfsetroundjoin%
\pgfsetlinewidth{1.003750pt}%
\definecolor{currentstroke}{rgb}{0.800000,0.800000,0.800000}%
\pgfsetstrokecolor{currentstroke}%
\pgfsetdash{}{0pt}%
\pgfpathmoveto{\pgfqpoint{2.827793in}{0.451389in}}%
\pgfpathlineto{\pgfqpoint{2.827793in}{1.867995in}}%
\pgfusepath{stroke}%
\end{pgfscope}%
\begin{pgfscope}%
\definecolor{textcolor}{rgb}{0.150000,0.150000,0.150000}%
\pgfsetstrokecolor{textcolor}%
\pgfsetfillcolor{textcolor}%
\pgftext[x=2.827793in,y=0.319444in,,top]{\color{textcolor}\sffamily\fontsize{9.000000}{10.800000}\selectfont 30000}%
\end{pgfscope}%
\begin{pgfscope}%
\definecolor{textcolor}{rgb}{0.150000,0.150000,0.150000}%
\pgfsetstrokecolor{textcolor}%
\pgfsetfillcolor{textcolor}%
\pgftext[x=1.719216in,y=0.125000in,,top]{\color{textcolor}\sffamily\fontsize{9.000000}{10.800000}\selectfont Input obstacle vertices}%
\end{pgfscope}%
\begin{pgfscope}%
\pgfpathrectangle{\pgfqpoint{0.395236in}{0.451389in}}{\pgfqpoint{2.647959in}{1.416606in}}%
\pgfusepath{clip}%
\pgfsetroundcap%
\pgfsetroundjoin%
\pgfsetlinewidth{1.003750pt}%
\definecolor{currentstroke}{rgb}{0.800000,0.800000,0.800000}%
\pgfsetstrokecolor{currentstroke}%
\pgfsetdash{}{0pt}%
\pgfpathmoveto{\pgfqpoint{0.395236in}{0.451389in}}%
\pgfpathlineto{\pgfqpoint{3.043196in}{0.451389in}}%
\pgfusepath{stroke}%
\end{pgfscope}%
\begin{pgfscope}%
\definecolor{textcolor}{rgb}{0.150000,0.150000,0.150000}%
\pgfsetstrokecolor{textcolor}%
\pgfsetfillcolor{textcolor}%
\pgftext[x=0.194444in, y=0.403903in, left, base]{\color{textcolor}\sffamily\fontsize{9.000000}{10.800000}\selectfont 0}%
\end{pgfscope}%
\begin{pgfscope}%
\pgfpathrectangle{\pgfqpoint{0.395236in}{0.451389in}}{\pgfqpoint{2.647959in}{1.416606in}}%
\pgfusepath{clip}%
\pgfsetroundcap%
\pgfsetroundjoin%
\pgfsetlinewidth{1.003750pt}%
\definecolor{currentstroke}{rgb}{0.800000,0.800000,0.800000}%
\pgfsetstrokecolor{currentstroke}%
\pgfsetdash{}{0pt}%
\pgfpathmoveto{\pgfqpoint{0.395236in}{0.854358in}}%
\pgfpathlineto{\pgfqpoint{3.043196in}{0.854358in}}%
\pgfusepath{stroke}%
\end{pgfscope}%
\begin{pgfscope}%
\definecolor{textcolor}{rgb}{0.150000,0.150000,0.150000}%
\pgfsetstrokecolor{textcolor}%
\pgfsetfillcolor{textcolor}%
\pgftext[x=0.194444in, y=0.806873in, left, base]{\color{textcolor}\sffamily\fontsize{9.000000}{10.800000}\selectfont 2}%
\end{pgfscope}%
\begin{pgfscope}%
\pgfpathrectangle{\pgfqpoint{0.395236in}{0.451389in}}{\pgfqpoint{2.647959in}{1.416606in}}%
\pgfusepath{clip}%
\pgfsetroundcap%
\pgfsetroundjoin%
\pgfsetlinewidth{1.003750pt}%
\definecolor{currentstroke}{rgb}{0.800000,0.800000,0.800000}%
\pgfsetstrokecolor{currentstroke}%
\pgfsetdash{}{0pt}%
\pgfpathmoveto{\pgfqpoint{0.395236in}{1.257328in}}%
\pgfpathlineto{\pgfqpoint{3.043196in}{1.257328in}}%
\pgfusepath{stroke}%
\end{pgfscope}%
\begin{pgfscope}%
\definecolor{textcolor}{rgb}{0.150000,0.150000,0.150000}%
\pgfsetstrokecolor{textcolor}%
\pgfsetfillcolor{textcolor}%
\pgftext[x=0.194444in, y=1.209842in, left, base]{\color{textcolor}\sffamily\fontsize{9.000000}{10.800000}\selectfont 4}%
\end{pgfscope}%
\begin{pgfscope}%
\pgfpathrectangle{\pgfqpoint{0.395236in}{0.451389in}}{\pgfqpoint{2.647959in}{1.416606in}}%
\pgfusepath{clip}%
\pgfsetroundcap%
\pgfsetroundjoin%
\pgfsetlinewidth{1.003750pt}%
\definecolor{currentstroke}{rgb}{0.800000,0.800000,0.800000}%
\pgfsetstrokecolor{currentstroke}%
\pgfsetdash{}{0pt}%
\pgfpathmoveto{\pgfqpoint{0.395236in}{1.660297in}}%
\pgfpathlineto{\pgfqpoint{3.043196in}{1.660297in}}%
\pgfusepath{stroke}%
\end{pgfscope}%
\begin{pgfscope}%
\definecolor{textcolor}{rgb}{0.150000,0.150000,0.150000}%
\pgfsetstrokecolor{textcolor}%
\pgfsetfillcolor{textcolor}%
\pgftext[x=0.194444in, y=1.612812in, left, base]{\color{textcolor}\sffamily\fontsize{9.000000}{10.800000}\selectfont 6}%
\end{pgfscope}%
\begin{pgfscope}%
\definecolor{textcolor}{rgb}{0.150000,0.150000,0.150000}%
\pgfsetstrokecolor{textcolor}%
\pgfsetfillcolor{textcolor}%
\pgftext[x=0.125000in,y=1.159692in,,bottom,rotate=90.000000]{\color{textcolor}\sffamily\fontsize{9.000000}{10.800000}\selectfont Time in ms}%
\end{pgfscope}%
\begin{pgfscope}%
\pgfpathrectangle{\pgfqpoint{0.395236in}{0.451389in}}{\pgfqpoint{2.647959in}{1.416606in}}%
\pgfusepath{clip}%
\pgfsetbuttcap%
\pgfsetroundjoin%
\definecolor{currentfill}{rgb}{0.003922,0.450980,0.698039}%
\pgfsetfillcolor{currentfill}%
\pgfsetfillopacity{0.200000}%
\pgfsetlinewidth{1.003750pt}%
\definecolor{currentstroke}{rgb}{0.003922,0.450980,0.698039}%
\pgfsetstrokecolor{currentstroke}%
\pgfsetstrokeopacity{0.200000}%
\pgfsetdash{}{0pt}%
\pgfsys@defobject{currentmarker}{\pgfqpoint{0.399615in}{0.472653in}}{\pgfqpoint{2.917311in}{1.801550in}}{%
\pgfpathmoveto{\pgfqpoint{0.399615in}{0.483482in}}%
\pgfpathlineto{\pgfqpoint{0.399615in}{0.472653in}}%
\pgfpathlineto{\pgfqpoint{0.412751in}{0.490586in}}%
\pgfpathlineto{\pgfqpoint{0.465294in}{0.494104in}}%
\pgfpathlineto{\pgfqpoint{0.552866in}{0.542933in}}%
\pgfpathlineto{\pgfqpoint{0.675467in}{0.592283in}}%
\pgfpathlineto{\pgfqpoint{0.833097in}{0.656559in}}%
\pgfpathlineto{\pgfqpoint{1.025755in}{0.752049in}}%
\pgfpathlineto{\pgfqpoint{1.253442in}{0.853539in}}%
\pgfpathlineto{\pgfqpoint{1.516158in}{0.976622in}}%
\pgfpathlineto{\pgfqpoint{1.813903in}{1.168476in}}%
\pgfpathlineto{\pgfqpoint{2.146677in}{1.309545in}}%
\pgfpathlineto{\pgfqpoint{2.514480in}{1.529714in}}%
\pgfpathlineto{\pgfqpoint{2.917311in}{1.794932in}}%
\pgfpathlineto{\pgfqpoint{2.917311in}{1.801550in}}%
\pgfpathlineto{\pgfqpoint{2.917311in}{1.801550in}}%
\pgfpathlineto{\pgfqpoint{2.514480in}{1.535942in}}%
\pgfpathlineto{\pgfqpoint{2.146677in}{1.320504in}}%
\pgfpathlineto{\pgfqpoint{1.813903in}{1.179577in}}%
\pgfpathlineto{\pgfqpoint{1.516158in}{0.980885in}}%
\pgfpathlineto{\pgfqpoint{1.253442in}{0.857408in}}%
\pgfpathlineto{\pgfqpoint{1.025755in}{0.753562in}}%
\pgfpathlineto{\pgfqpoint{0.833097in}{0.657935in}}%
\pgfpathlineto{\pgfqpoint{0.675467in}{0.592683in}}%
\pgfpathlineto{\pgfqpoint{0.552866in}{0.543553in}}%
\pgfpathlineto{\pgfqpoint{0.465294in}{0.495044in}}%
\pgfpathlineto{\pgfqpoint{0.412751in}{0.494366in}}%
\pgfpathlineto{\pgfqpoint{0.399615in}{0.483482in}}%
\pgfpathlineto{\pgfqpoint{0.399615in}{0.483482in}}%
\pgfpathclose%
\pgfusepath{stroke,fill}%
}%
\begin{pgfscope}%
\pgfsys@transformshift{0.000000in}{0.000000in}%
\pgfsys@useobject{currentmarker}{}%
\end{pgfscope}%
\end{pgfscope}%
\begin{pgfscope}%
\pgfsetrectcap%
\pgfsetmiterjoin%
\pgfsetlinewidth{1.254687pt}%
\definecolor{currentstroke}{rgb}{0.800000,0.800000,0.800000}%
\pgfsetstrokecolor{currentstroke}%
\pgfsetdash{}{0pt}%
\pgfpathmoveto{\pgfqpoint{0.395236in}{0.451389in}}%
\pgfpathlineto{\pgfqpoint{0.395236in}{1.867995in}}%
\pgfusepath{stroke}%
\end{pgfscope}%
\begin{pgfscope}%
\pgfsetrectcap%
\pgfsetmiterjoin%
\pgfsetlinewidth{1.254687pt}%
\definecolor{currentstroke}{rgb}{0.800000,0.800000,0.800000}%
\pgfsetstrokecolor{currentstroke}%
\pgfsetdash{}{0pt}%
\pgfpathmoveto{\pgfqpoint{3.043196in}{0.451389in}}%
\pgfpathlineto{\pgfqpoint{3.043196in}{1.867995in}}%
\pgfusepath{stroke}%
\end{pgfscope}%
\begin{pgfscope}%
\pgfsetrectcap%
\pgfsetmiterjoin%
\pgfsetlinewidth{1.254687pt}%
\definecolor{currentstroke}{rgb}{0.800000,0.800000,0.800000}%
\pgfsetstrokecolor{currentstroke}%
\pgfsetdash{}{0pt}%
\pgfpathmoveto{\pgfqpoint{0.395236in}{0.451389in}}%
\pgfpathlineto{\pgfqpoint{3.043196in}{0.451389in}}%
\pgfusepath{stroke}%
\end{pgfscope}%
\begin{pgfscope}%
\pgfsetrectcap%
\pgfsetmiterjoin%
\pgfsetlinewidth{1.254687pt}%
\definecolor{currentstroke}{rgb}{0.800000,0.800000,0.800000}%
\pgfsetstrokecolor{currentstroke}%
\pgfsetdash{}{0pt}%
\pgfpathmoveto{\pgfqpoint{0.395236in}{1.867995in}}%
\pgfpathlineto{\pgfqpoint{3.043196in}{1.867995in}}%
\pgfusepath{stroke}%
\end{pgfscope}%
\begin{pgfscope}%
\pgfsetroundcap%
\pgfsetroundjoin%
\pgfsetlinewidth{1.003750pt}%
\definecolor{currentstroke}{rgb}{0.003922,0.450980,0.698039}%
\pgfsetstrokecolor{currentstroke}%
\pgfsetdash{}{0pt}%
\pgfpathmoveto{\pgfqpoint{0.399615in}{0.475615in}}%
\pgfpathlineto{\pgfqpoint{0.412751in}{0.492303in}}%
\pgfpathlineto{\pgfqpoint{0.465294in}{0.494415in}}%
\pgfpathlineto{\pgfqpoint{0.552866in}{0.543255in}}%
\pgfpathlineto{\pgfqpoint{0.675467in}{0.592433in}}%
\pgfpathlineto{\pgfqpoint{0.833097in}{0.657188in}}%
\pgfpathlineto{\pgfqpoint{1.025755in}{0.752761in}}%
\pgfpathlineto{\pgfqpoint{1.253442in}{0.855586in}}%
\pgfpathlineto{\pgfqpoint{1.516158in}{0.978736in}}%
\pgfpathlineto{\pgfqpoint{1.813903in}{1.172913in}}%
\pgfpathlineto{\pgfqpoint{2.146677in}{1.316219in}}%
\pgfpathlineto{\pgfqpoint{2.514480in}{1.533767in}}%
\pgfpathlineto{\pgfqpoint{2.917311in}{1.797740in}}%
\pgfusepath{stroke}%
\end{pgfscope}%
\begin{pgfscope}%
\pgfsetbuttcap%
\pgfsetroundjoin%
\definecolor{currentfill}{rgb}{0.003922,0.450980,0.698039}%
\pgfsetfillcolor{currentfill}%
\pgfsetlinewidth{0.752812pt}%
\definecolor{currentstroke}{rgb}{1.000000,1.000000,1.000000}%
\pgfsetstrokecolor{currentstroke}%
\pgfsetdash{}{0pt}%
\pgfsys@defobject{currentmarker}{\pgfqpoint{-0.034722in}{-0.034722in}}{\pgfqpoint{0.034722in}{0.034722in}}{%
\pgfpathmoveto{\pgfqpoint{0.000000in}{-0.034722in}}%
\pgfpathcurveto{\pgfqpoint{0.009208in}{-0.034722in}}{\pgfqpoint{0.018041in}{-0.031064in}}{\pgfqpoint{0.024552in}{-0.024552in}}%
\pgfpathcurveto{\pgfqpoint{0.031064in}{-0.018041in}}{\pgfqpoint{0.034722in}{-0.009208in}}{\pgfqpoint{0.034722in}{0.000000in}}%
\pgfpathcurveto{\pgfqpoint{0.034722in}{0.009208in}}{\pgfqpoint{0.031064in}{0.018041in}}{\pgfqpoint{0.024552in}{0.024552in}}%
\pgfpathcurveto{\pgfqpoint{0.018041in}{0.031064in}}{\pgfqpoint{0.009208in}{0.034722in}}{\pgfqpoint{0.000000in}{0.034722in}}%
\pgfpathcurveto{\pgfqpoint{-0.009208in}{0.034722in}}{\pgfqpoint{-0.018041in}{0.031064in}}{\pgfqpoint{-0.024552in}{0.024552in}}%
\pgfpathcurveto{\pgfqpoint{-0.031064in}{0.018041in}}{\pgfqpoint{-0.034722in}{0.009208in}}{\pgfqpoint{-0.034722in}{0.000000in}}%
\pgfpathcurveto{\pgfqpoint{-0.034722in}{-0.009208in}}{\pgfqpoint{-0.031064in}{-0.018041in}}{\pgfqpoint{-0.024552in}{-0.024552in}}%
\pgfpathcurveto{\pgfqpoint{-0.018041in}{-0.031064in}}{\pgfqpoint{-0.009208in}{-0.034722in}}{\pgfqpoint{0.000000in}{-0.034722in}}%
\pgfpathlineto{\pgfqpoint{0.000000in}{-0.034722in}}%
\pgfpathclose%
\pgfusepath{stroke,fill}%
}%
\begin{pgfscope}%
\pgfsys@transformshift{0.399615in}{0.475615in}%
\pgfsys@useobject{currentmarker}{}%
\end{pgfscope}%
\begin{pgfscope}%
\pgfsys@transformshift{0.412751in}{0.492303in}%
\pgfsys@useobject{currentmarker}{}%
\end{pgfscope}%
\begin{pgfscope}%
\pgfsys@transformshift{0.465294in}{0.494415in}%
\pgfsys@useobject{currentmarker}{}%
\end{pgfscope}%
\begin{pgfscope}%
\pgfsys@transformshift{0.552866in}{0.543255in}%
\pgfsys@useobject{currentmarker}{}%
\end{pgfscope}%
\begin{pgfscope}%
\pgfsys@transformshift{0.675467in}{0.592433in}%
\pgfsys@useobject{currentmarker}{}%
\end{pgfscope}%
\begin{pgfscope}%
\pgfsys@transformshift{0.833097in}{0.657188in}%
\pgfsys@useobject{currentmarker}{}%
\end{pgfscope}%
\begin{pgfscope}%
\pgfsys@transformshift{1.025755in}{0.752761in}%
\pgfsys@useobject{currentmarker}{}%
\end{pgfscope}%
\begin{pgfscope}%
\pgfsys@transformshift{1.253442in}{0.855586in}%
\pgfsys@useobject{currentmarker}{}%
\end{pgfscope}%
\begin{pgfscope}%
\pgfsys@transformshift{1.516158in}{0.978736in}%
\pgfsys@useobject{currentmarker}{}%
\end{pgfscope}%
\begin{pgfscope}%
\pgfsys@transformshift{1.813903in}{1.172913in}%
\pgfsys@useobject{currentmarker}{}%
\end{pgfscope}%
\begin{pgfscope}%
\pgfsys@transformshift{2.146677in}{1.316219in}%
\pgfsys@useobject{currentmarker}{}%
\end{pgfscope}%
\begin{pgfscope}%
\pgfsys@transformshift{2.514480in}{1.533767in}%
\pgfsys@useobject{currentmarker}{}%
\end{pgfscope}%
\begin{pgfscope}%
\pgfsys@transformshift{2.917311in}{1.797740in}%
\pgfsys@useobject{currentmarker}{}%
\end{pgfscope}%
\end{pgfscope}%
\end{pgfpicture}%
\makeatother%
\endgroup%

					\end{figcenter}
					\caption{Time per input vertex for the \enquote{OSM rural} dataset.}
				\end{subfigure}
				\\[3ex]
				\begin{subfigure}[t]{\linewidth}
					\begin{figcenter}
						\begingroup%
\makeatletter%
\begin{pgfpicture}%
\pgfpathrectangle{\pgfpointorigin}{\pgfqpoint{3.042427in}{1.867995in}}%
\pgfusepath{use as bounding box}%
\begin{pgfscope}%
\pgfsetbuttcap%
\pgfsetmiterjoin%
\definecolor{currentfill}{rgb}{1.000000,1.000000,1.000000}%
\pgfsetfillcolor{currentfill}%
\pgfsetlinewidth{0.000000pt}%
\definecolor{currentstroke}{rgb}{1.000000,1.000000,1.000000}%
\pgfsetstrokecolor{currentstroke}%
\pgfsetdash{}{0pt}%
\pgfpathmoveto{\pgfqpoint{0.000000in}{0.000000in}}%
\pgfpathlineto{\pgfqpoint{3.042427in}{0.000000in}}%
\pgfpathlineto{\pgfqpoint{3.042427in}{1.867995in}}%
\pgfpathlineto{\pgfqpoint{0.000000in}{1.867995in}}%
\pgfpathlineto{\pgfqpoint{0.000000in}{0.000000in}}%
\pgfpathclose%
\pgfusepath{fill}%
\end{pgfscope}%
\begin{pgfscope}%
\pgfsetbuttcap%
\pgfsetmiterjoin%
\definecolor{currentfill}{rgb}{1.000000,1.000000,1.000000}%
\pgfsetfillcolor{currentfill}%
\pgfsetlinewidth{0.000000pt}%
\definecolor{currentstroke}{rgb}{0.000000,0.000000,0.000000}%
\pgfsetstrokecolor{currentstroke}%
\pgfsetstrokeopacity{0.000000}%
\pgfsetdash{}{0pt}%
\pgfpathmoveto{\pgfqpoint{0.497592in}{0.451389in}}%
\pgfpathlineto{\pgfqpoint{3.042427in}{0.451389in}}%
\pgfpathlineto{\pgfqpoint{3.042427in}{1.867995in}}%
\pgfpathlineto{\pgfqpoint{0.497592in}{1.867995in}}%
\pgfpathlineto{\pgfqpoint{0.497592in}{0.451389in}}%
\pgfpathclose%
\pgfusepath{fill}%
\end{pgfscope}%
\begin{pgfscope}%
\pgfpathrectangle{\pgfqpoint{0.497592in}{0.451389in}}{\pgfqpoint{2.544834in}{1.416606in}}%
\pgfusepath{clip}%
\pgfsetroundcap%
\pgfsetroundjoin%
\pgfsetlinewidth{1.003750pt}%
\definecolor{currentstroke}{rgb}{0.800000,0.800000,0.800000}%
\pgfsetstrokecolor{currentstroke}%
\pgfsetdash{}{0pt}%
\pgfpathmoveto{\pgfqpoint{0.497592in}{0.451389in}}%
\pgfpathlineto{\pgfqpoint{0.497592in}{1.867995in}}%
\pgfusepath{stroke}%
\end{pgfscope}%
\begin{pgfscope}%
\definecolor{textcolor}{rgb}{0.150000,0.150000,0.150000}%
\pgfsetstrokecolor{textcolor}%
\pgfsetfillcolor{textcolor}%
\pgftext[x=0.497592in,y=0.319444in,,top]{\color{textcolor}\sffamily\fontsize{9.000000}{10.800000}\selectfont 0}%
\end{pgfscope}%
\begin{pgfscope}%
\pgfpathrectangle{\pgfqpoint{0.497592in}{0.451389in}}{\pgfqpoint{2.544834in}{1.416606in}}%
\pgfusepath{clip}%
\pgfsetroundcap%
\pgfsetroundjoin%
\pgfsetlinewidth{1.003750pt}%
\definecolor{currentstroke}{rgb}{0.800000,0.800000,0.800000}%
\pgfsetstrokecolor{currentstroke}%
\pgfsetdash{}{0pt}%
\pgfpathmoveto{\pgfqpoint{1.043720in}{0.451389in}}%
\pgfpathlineto{\pgfqpoint{1.043720in}{1.867995in}}%
\pgfusepath{stroke}%
\end{pgfscope}%
\begin{pgfscope}%
\definecolor{textcolor}{rgb}{0.150000,0.150000,0.150000}%
\pgfsetstrokecolor{textcolor}%
\pgfsetfillcolor{textcolor}%
\pgftext[x=1.043720in,y=0.319444in,,top]{\color{textcolor}\sffamily\fontsize{9.000000}{10.800000}\selectfont 10000}%
\end{pgfscope}%
\begin{pgfscope}%
\pgfpathrectangle{\pgfqpoint{0.497592in}{0.451389in}}{\pgfqpoint{2.544834in}{1.416606in}}%
\pgfusepath{clip}%
\pgfsetroundcap%
\pgfsetroundjoin%
\pgfsetlinewidth{1.003750pt}%
\definecolor{currentstroke}{rgb}{0.800000,0.800000,0.800000}%
\pgfsetstrokecolor{currentstroke}%
\pgfsetdash{}{0pt}%
\pgfpathmoveto{\pgfqpoint{1.589848in}{0.451389in}}%
\pgfpathlineto{\pgfqpoint{1.589848in}{1.867995in}}%
\pgfusepath{stroke}%
\end{pgfscope}%
\begin{pgfscope}%
\definecolor{textcolor}{rgb}{0.150000,0.150000,0.150000}%
\pgfsetstrokecolor{textcolor}%
\pgfsetfillcolor{textcolor}%
\pgftext[x=1.589848in,y=0.319444in,,top]{\color{textcolor}\sffamily\fontsize{9.000000}{10.800000}\selectfont 20000}%
\end{pgfscope}%
\begin{pgfscope}%
\pgfpathrectangle{\pgfqpoint{0.497592in}{0.451389in}}{\pgfqpoint{2.544834in}{1.416606in}}%
\pgfusepath{clip}%
\pgfsetroundcap%
\pgfsetroundjoin%
\pgfsetlinewidth{1.003750pt}%
\definecolor{currentstroke}{rgb}{0.800000,0.800000,0.800000}%
\pgfsetstrokecolor{currentstroke}%
\pgfsetdash{}{0pt}%
\pgfpathmoveto{\pgfqpoint{2.135975in}{0.451389in}}%
\pgfpathlineto{\pgfqpoint{2.135975in}{1.867995in}}%
\pgfusepath{stroke}%
\end{pgfscope}%
\begin{pgfscope}%
\definecolor{textcolor}{rgb}{0.150000,0.150000,0.150000}%
\pgfsetstrokecolor{textcolor}%
\pgfsetfillcolor{textcolor}%
\pgftext[x=2.135975in,y=0.319444in,,top]{\color{textcolor}\sffamily\fontsize{9.000000}{10.800000}\selectfont 30000}%
\end{pgfscope}%
\begin{pgfscope}%
\pgfpathrectangle{\pgfqpoint{0.497592in}{0.451389in}}{\pgfqpoint{2.544834in}{1.416606in}}%
\pgfusepath{clip}%
\pgfsetroundcap%
\pgfsetroundjoin%
\pgfsetlinewidth{1.003750pt}%
\definecolor{currentstroke}{rgb}{0.800000,0.800000,0.800000}%
\pgfsetstrokecolor{currentstroke}%
\pgfsetdash{}{0pt}%
\pgfpathmoveto{\pgfqpoint{2.682103in}{0.451389in}}%
\pgfpathlineto{\pgfqpoint{2.682103in}{1.867995in}}%
\pgfusepath{stroke}%
\end{pgfscope}%
\begin{pgfscope}%
\definecolor{textcolor}{rgb}{0.150000,0.150000,0.150000}%
\pgfsetstrokecolor{textcolor}%
\pgfsetfillcolor{textcolor}%
\pgftext[x=2.682103in,y=0.319444in,,top]{\color{textcolor}\sffamily\fontsize{9.000000}{10.800000}\selectfont 40000}%
\end{pgfscope}%
\begin{pgfscope}%
\definecolor{textcolor}{rgb}{0.150000,0.150000,0.150000}%
\pgfsetstrokecolor{textcolor}%
\pgfsetfillcolor{textcolor}%
\pgftext[x=1.770010in,y=0.125000in,,top]{\color{textcolor}\sffamily\fontsize{9.000000}{10.800000}\selectfont Input obstacle vertices}%
\end{pgfscope}%
\begin{pgfscope}%
\pgfpathrectangle{\pgfqpoint{0.497592in}{0.451389in}}{\pgfqpoint{2.544834in}{1.416606in}}%
\pgfusepath{clip}%
\pgfsetroundcap%
\pgfsetroundjoin%
\pgfsetlinewidth{1.003750pt}%
\definecolor{currentstroke}{rgb}{0.800000,0.800000,0.800000}%
\pgfsetstrokecolor{currentstroke}%
\pgfsetdash{}{0pt}%
\pgfpathmoveto{\pgfqpoint{0.497592in}{0.451389in}}%
\pgfpathlineto{\pgfqpoint{3.042427in}{0.451389in}}%
\pgfusepath{stroke}%
\end{pgfscope}%
\begin{pgfscope}%
\definecolor{textcolor}{rgb}{0.150000,0.150000,0.150000}%
\pgfsetstrokecolor{textcolor}%
\pgfsetfillcolor{textcolor}%
\pgftext[x=0.194444in, y=0.403903in, left, base]{\color{textcolor}\sffamily\fontsize{9.000000}{10.800000}\selectfont 0.0}%
\end{pgfscope}%
\begin{pgfscope}%
\pgfpathrectangle{\pgfqpoint{0.497592in}{0.451389in}}{\pgfqpoint{2.544834in}{1.416606in}}%
\pgfusepath{clip}%
\pgfsetroundcap%
\pgfsetroundjoin%
\pgfsetlinewidth{1.003750pt}%
\definecolor{currentstroke}{rgb}{0.800000,0.800000,0.800000}%
\pgfsetstrokecolor{currentstroke}%
\pgfsetdash{}{0pt}%
\pgfpathmoveto{\pgfqpoint{0.497592in}{0.757507in}}%
\pgfpathlineto{\pgfqpoint{3.042427in}{0.757507in}}%
\pgfusepath{stroke}%
\end{pgfscope}%
\begin{pgfscope}%
\definecolor{textcolor}{rgb}{0.150000,0.150000,0.150000}%
\pgfsetstrokecolor{textcolor}%
\pgfsetfillcolor{textcolor}%
\pgftext[x=0.194444in, y=0.710021in, left, base]{\color{textcolor}\sffamily\fontsize{9.000000}{10.800000}\selectfont 0.1}%
\end{pgfscope}%
\begin{pgfscope}%
\pgfpathrectangle{\pgfqpoint{0.497592in}{0.451389in}}{\pgfqpoint{2.544834in}{1.416606in}}%
\pgfusepath{clip}%
\pgfsetroundcap%
\pgfsetroundjoin%
\pgfsetlinewidth{1.003750pt}%
\definecolor{currentstroke}{rgb}{0.800000,0.800000,0.800000}%
\pgfsetstrokecolor{currentstroke}%
\pgfsetdash{}{0pt}%
\pgfpathmoveto{\pgfqpoint{0.497592in}{1.063625in}}%
\pgfpathlineto{\pgfqpoint{3.042427in}{1.063625in}}%
\pgfusepath{stroke}%
\end{pgfscope}%
\begin{pgfscope}%
\definecolor{textcolor}{rgb}{0.150000,0.150000,0.150000}%
\pgfsetstrokecolor{textcolor}%
\pgfsetfillcolor{textcolor}%
\pgftext[x=0.194444in, y=1.016140in, left, base]{\color{textcolor}\sffamily\fontsize{9.000000}{10.800000}\selectfont 0.2}%
\end{pgfscope}%
\begin{pgfscope}%
\pgfpathrectangle{\pgfqpoint{0.497592in}{0.451389in}}{\pgfqpoint{2.544834in}{1.416606in}}%
\pgfusepath{clip}%
\pgfsetroundcap%
\pgfsetroundjoin%
\pgfsetlinewidth{1.003750pt}%
\definecolor{currentstroke}{rgb}{0.800000,0.800000,0.800000}%
\pgfsetstrokecolor{currentstroke}%
\pgfsetdash{}{0pt}%
\pgfpathmoveto{\pgfqpoint{0.497592in}{1.369743in}}%
\pgfpathlineto{\pgfqpoint{3.042427in}{1.369743in}}%
\pgfusepath{stroke}%
\end{pgfscope}%
\begin{pgfscope}%
\definecolor{textcolor}{rgb}{0.150000,0.150000,0.150000}%
\pgfsetstrokecolor{textcolor}%
\pgfsetfillcolor{textcolor}%
\pgftext[x=0.194444in, y=1.322258in, left, base]{\color{textcolor}\sffamily\fontsize{9.000000}{10.800000}\selectfont 0.3}%
\end{pgfscope}%
\begin{pgfscope}%
\pgfpathrectangle{\pgfqpoint{0.497592in}{0.451389in}}{\pgfqpoint{2.544834in}{1.416606in}}%
\pgfusepath{clip}%
\pgfsetroundcap%
\pgfsetroundjoin%
\pgfsetlinewidth{1.003750pt}%
\definecolor{currentstroke}{rgb}{0.800000,0.800000,0.800000}%
\pgfsetstrokecolor{currentstroke}%
\pgfsetdash{}{0pt}%
\pgfpathmoveto{\pgfqpoint{0.497592in}{1.675862in}}%
\pgfpathlineto{\pgfqpoint{3.042427in}{1.675862in}}%
\pgfusepath{stroke}%
\end{pgfscope}%
\begin{pgfscope}%
\definecolor{textcolor}{rgb}{0.150000,0.150000,0.150000}%
\pgfsetstrokecolor{textcolor}%
\pgfsetfillcolor{textcolor}%
\pgftext[x=0.194444in, y=1.628376in, left, base]{\color{textcolor}\sffamily\fontsize{9.000000}{10.800000}\selectfont 0.4}%
\end{pgfscope}%
\begin{pgfscope}%
\definecolor{textcolor}{rgb}{0.150000,0.150000,0.150000}%
\pgfsetstrokecolor{textcolor}%
\pgfsetfillcolor{textcolor}%
\pgftext[x=0.125000in,y=1.159692in,,bottom,rotate=90.000000]{\color{textcolor}\sffamily\fontsize{9.000000}{10.800000}\selectfont Time in µs}%
\end{pgfscope}%
\begin{pgfscope}%
\pgfpathrectangle{\pgfqpoint{0.497592in}{0.451389in}}{\pgfqpoint{2.544834in}{1.416606in}}%
\pgfusepath{clip}%
\pgfsetbuttcap%
\pgfsetroundjoin%
\definecolor{currentfill}{rgb}{0.003922,0.450980,0.698039}%
\pgfsetfillcolor{currentfill}%
\pgfsetfillopacity{0.200000}%
\pgfsetlinewidth{1.003750pt}%
\definecolor{currentstroke}{rgb}{0.003922,0.450980,0.698039}%
\pgfsetstrokecolor{currentstroke}%
\pgfsetstrokeopacity{0.200000}%
\pgfsetdash{}{0pt}%
\pgfsys@defobject{currentmarker}{\pgfqpoint{0.521840in}{1.283530in}}{\pgfqpoint{2.922399in}{1.840163in}}{%
\pgfpathmoveto{\pgfqpoint{0.521840in}{1.840163in}}%
\pgfpathlineto{\pgfqpoint{0.521840in}{1.685377in}}%
\pgfpathlineto{\pgfqpoint{0.594585in}{1.489045in}}%
\pgfpathlineto{\pgfqpoint{0.715825in}{1.382130in}}%
\pgfpathlineto{\pgfqpoint{0.885561in}{1.312715in}}%
\pgfpathlineto{\pgfqpoint{1.103794in}{1.283530in}}%
\pgfpathlineto{\pgfqpoint{1.370523in}{1.372552in}}%
\pgfpathlineto{\pgfqpoint{1.685748in}{1.323127in}}%
\pgfpathlineto{\pgfqpoint{2.049469in}{1.472716in}}%
\pgfpathlineto{\pgfqpoint{2.461686in}{1.423765in}}%
\pgfpathlineto{\pgfqpoint{2.922399in}{1.670304in}}%
\pgfpathlineto{\pgfqpoint{2.922399in}{1.686277in}}%
\pgfpathlineto{\pgfqpoint{2.922399in}{1.686277in}}%
\pgfpathlineto{\pgfqpoint{2.461686in}{1.430446in}}%
\pgfpathlineto{\pgfqpoint{2.049469in}{1.506497in}}%
\pgfpathlineto{\pgfqpoint{1.685748in}{1.335177in}}%
\pgfpathlineto{\pgfqpoint{1.370523in}{1.386770in}}%
\pgfpathlineto{\pgfqpoint{1.103794in}{1.297994in}}%
\pgfpathlineto{\pgfqpoint{0.885561in}{1.316776in}}%
\pgfpathlineto{\pgfqpoint{0.715825in}{1.405375in}}%
\pgfpathlineto{\pgfqpoint{0.594585in}{1.502386in}}%
\pgfpathlineto{\pgfqpoint{0.521840in}{1.840163in}}%
\pgfpathlineto{\pgfqpoint{0.521840in}{1.840163in}}%
\pgfpathclose%
\pgfusepath{stroke,fill}%
}%
\begin{pgfscope}%
\pgfsys@transformshift{0.000000in}{0.000000in}%
\pgfsys@useobject{currentmarker}{}%
\end{pgfscope}%
\end{pgfscope}%
\begin{pgfscope}%
\pgfsetrectcap%
\pgfsetmiterjoin%
\pgfsetlinewidth{1.254687pt}%
\definecolor{currentstroke}{rgb}{0.800000,0.800000,0.800000}%
\pgfsetstrokecolor{currentstroke}%
\pgfsetdash{}{0pt}%
\pgfpathmoveto{\pgfqpoint{0.497592in}{0.451389in}}%
\pgfpathlineto{\pgfqpoint{0.497592in}{1.867995in}}%
\pgfusepath{stroke}%
\end{pgfscope}%
\begin{pgfscope}%
\pgfsetrectcap%
\pgfsetmiterjoin%
\pgfsetlinewidth{1.254687pt}%
\definecolor{currentstroke}{rgb}{0.800000,0.800000,0.800000}%
\pgfsetstrokecolor{currentstroke}%
\pgfsetdash{}{0pt}%
\pgfpathmoveto{\pgfqpoint{3.042427in}{0.451389in}}%
\pgfpathlineto{\pgfqpoint{3.042427in}{1.867995in}}%
\pgfusepath{stroke}%
\end{pgfscope}%
\begin{pgfscope}%
\pgfsetrectcap%
\pgfsetmiterjoin%
\pgfsetlinewidth{1.254687pt}%
\definecolor{currentstroke}{rgb}{0.800000,0.800000,0.800000}%
\pgfsetstrokecolor{currentstroke}%
\pgfsetdash{}{0pt}%
\pgfpathmoveto{\pgfqpoint{0.497592in}{0.451389in}}%
\pgfpathlineto{\pgfqpoint{3.042427in}{0.451389in}}%
\pgfusepath{stroke}%
\end{pgfscope}%
\begin{pgfscope}%
\pgfsetrectcap%
\pgfsetmiterjoin%
\pgfsetlinewidth{1.254687pt}%
\definecolor{currentstroke}{rgb}{0.800000,0.800000,0.800000}%
\pgfsetstrokecolor{currentstroke}%
\pgfsetdash{}{0pt}%
\pgfpathmoveto{\pgfqpoint{0.497592in}{1.867995in}}%
\pgfpathlineto{\pgfqpoint{3.042427in}{1.867995in}}%
\pgfusepath{stroke}%
\end{pgfscope}%
\begin{pgfscope}%
\pgfsetroundcap%
\pgfsetroundjoin%
\pgfsetlinewidth{1.003750pt}%
\definecolor{currentstroke}{rgb}{0.003922,0.450980,0.698039}%
\pgfsetstrokecolor{currentstroke}%
\pgfsetdash{}{0pt}%
\pgfpathmoveto{\pgfqpoint{0.521840in}{1.763783in}}%
\pgfpathlineto{\pgfqpoint{0.594585in}{1.493641in}}%
\pgfpathlineto{\pgfqpoint{0.715825in}{1.395792in}}%
\pgfpathlineto{\pgfqpoint{0.885561in}{1.314789in}}%
\pgfpathlineto{\pgfqpoint{1.103794in}{1.289377in}}%
\pgfpathlineto{\pgfqpoint{1.370523in}{1.379765in}}%
\pgfpathlineto{\pgfqpoint{1.685748in}{1.330658in}}%
\pgfpathlineto{\pgfqpoint{2.049469in}{1.489351in}}%
\pgfpathlineto{\pgfqpoint{2.461686in}{1.426582in}}%
\pgfpathlineto{\pgfqpoint{2.922399in}{1.676096in}}%
\pgfusepath{stroke}%
\end{pgfscope}%
\begin{pgfscope}%
\pgfsetbuttcap%
\pgfsetroundjoin%
\definecolor{currentfill}{rgb}{0.003922,0.450980,0.698039}%
\pgfsetfillcolor{currentfill}%
\pgfsetlinewidth{0.752812pt}%
\definecolor{currentstroke}{rgb}{1.000000,1.000000,1.000000}%
\pgfsetstrokecolor{currentstroke}%
\pgfsetdash{}{0pt}%
\pgfsys@defobject{currentmarker}{\pgfqpoint{-0.034722in}{-0.034722in}}{\pgfqpoint{0.034722in}{0.034722in}}{%
\pgfpathmoveto{\pgfqpoint{0.000000in}{-0.034722in}}%
\pgfpathcurveto{\pgfqpoint{0.009208in}{-0.034722in}}{\pgfqpoint{0.018041in}{-0.031064in}}{\pgfqpoint{0.024552in}{-0.024552in}}%
\pgfpathcurveto{\pgfqpoint{0.031064in}{-0.018041in}}{\pgfqpoint{0.034722in}{-0.009208in}}{\pgfqpoint{0.034722in}{0.000000in}}%
\pgfpathcurveto{\pgfqpoint{0.034722in}{0.009208in}}{\pgfqpoint{0.031064in}{0.018041in}}{\pgfqpoint{0.024552in}{0.024552in}}%
\pgfpathcurveto{\pgfqpoint{0.018041in}{0.031064in}}{\pgfqpoint{0.009208in}{0.034722in}}{\pgfqpoint{0.000000in}{0.034722in}}%
\pgfpathcurveto{\pgfqpoint{-0.009208in}{0.034722in}}{\pgfqpoint{-0.018041in}{0.031064in}}{\pgfqpoint{-0.024552in}{0.024552in}}%
\pgfpathcurveto{\pgfqpoint{-0.031064in}{0.018041in}}{\pgfqpoint{-0.034722in}{0.009208in}}{\pgfqpoint{-0.034722in}{0.000000in}}%
\pgfpathcurveto{\pgfqpoint{-0.034722in}{-0.009208in}}{\pgfqpoint{-0.031064in}{-0.018041in}}{\pgfqpoint{-0.024552in}{-0.024552in}}%
\pgfpathcurveto{\pgfqpoint{-0.018041in}{-0.031064in}}{\pgfqpoint{-0.009208in}{-0.034722in}}{\pgfqpoint{0.000000in}{-0.034722in}}%
\pgfpathlineto{\pgfqpoint{0.000000in}{-0.034722in}}%
\pgfpathclose%
\pgfusepath{stroke,fill}%
}%
\begin{pgfscope}%
\pgfsys@transformshift{0.521840in}{1.763783in}%
\pgfsys@useobject{currentmarker}{}%
\end{pgfscope}%
\begin{pgfscope}%
\pgfsys@transformshift{0.594585in}{1.493641in}%
\pgfsys@useobject{currentmarker}{}%
\end{pgfscope}%
\begin{pgfscope}%
\pgfsys@transformshift{0.715825in}{1.395792in}%
\pgfsys@useobject{currentmarker}{}%
\end{pgfscope}%
\begin{pgfscope}%
\pgfsys@transformshift{0.885561in}{1.314789in}%
\pgfsys@useobject{currentmarker}{}%
\end{pgfscope}%
\begin{pgfscope}%
\pgfsys@transformshift{1.103794in}{1.289377in}%
\pgfsys@useobject{currentmarker}{}%
\end{pgfscope}%
\begin{pgfscope}%
\pgfsys@transformshift{1.370523in}{1.379765in}%
\pgfsys@useobject{currentmarker}{}%
\end{pgfscope}%
\begin{pgfscope}%
\pgfsys@transformshift{1.685748in}{1.330658in}%
\pgfsys@useobject{currentmarker}{}%
\end{pgfscope}%
\begin{pgfscope}%
\pgfsys@transformshift{2.049469in}{1.489351in}%
\pgfsys@useobject{currentmarker}{}%
\end{pgfscope}%
\begin{pgfscope}%
\pgfsys@transformshift{2.461686in}{1.426582in}%
\pgfsys@useobject{currentmarker}{}%
\end{pgfscope}%
\begin{pgfscope}%
\pgfsys@transformshift{2.922399in}{1.676096in}%
\pgfsys@useobject{currentmarker}{}%
\end{pgfscope}%
\end{pgfscope}%
\end{pgfpicture}%
\makeatother%
\endgroup%

						% TODO same y-scale in both figures
					\end{figcenter}
					\caption{Increase in processing time per vertex when an additional vertex is added.}
					\label{fig:eval-import-rural-rel-increase}
				\end{subfigure}
				\caption{Graph generation times using the \enquote{OSM rural} dataset.}
				\label{fig:eval-import-rural}
			\end{minipage}
		\end{figure}
		
		A few aspects regarding the runtime behavior can already be inferred from \Cref{fig:eval-import-city} and \Cref{fig:eval-import-rural}.
		
		First, as mentioned above, the inherent quadratic runtime of the visibility graph creation, which is shown in detail in the following \Cref{fig:eval-import-details}.
		
		Second, the linear increase in the processing time per vertex, which shows, that the total runtime is in fact quadratic and not, for example, exponential.
		
		And third, the additional effort added to each vertex when a new additional vertex is added to the dataset is very small (below one microsecond) and decreases with the size of the dataset.
		Taking the largest dataset of about 88.000 vertices in \Cref{fig:eval-import-city-rel-increase} as an example, adding one new vertex increases the processing time of every vertex by around 0.12 µs.
		At least for smaller datasets, as \Cref{fig:eval-import-rural-rel-increase} shows, this extra time for each additional vertex decreases with the size of the dataset.
		This indicates a slower growing processing time for larger datasets, however, this effect is negligible as it is not directly visible by the measured data in figures \ref{fig:eval-import-city-abs} and \ref{fig:eval-import-rural-abs}.
		
		\begin{figure}[h!]
			\begin{figcenter}
				\begin{subfigure}[t]{\textwidth}
					\begin{figcenter}
						%% Creator: Matplotlib, PGF backend
%%
%% To include the figure in your LaTeX document, write
%%   \input{<filename>.pgf}
%%
%% Make sure the required packages are loaded in your preamble
%%   \usepackage{pgf}
%%
%% Also ensure that all the required font packages are loaded; for instance,
%% the lmodern package is sometimes necessary when using math font.
%%   \usepackage{lmodern}
%%
%% Figures using additional raster images can only be included by \input if
%% they are in the same directory as the main LaTeX file. For loading figures
%% from other directories you can use the `import` package
%%   \usepackage{import}
%%
%% and then include the figures with
%%   \import{<path to file>}{<filename>.pgf}
%%
%% Matplotlib used the following preamble
%%   
%%   \usepackage{fontspec}
%%   \setmainfont{DejaVuSerif.ttf}[Path=\detokenize{/home/hauke/.local/lib/python3.11/site-packages/matplotlib/mpl-data/fonts/ttf/}]
%%   \setsansfont{DroidSans.ttf}[Path=\detokenize{/usr/share/fonts/droid/}]
%%   \setmonofont{DejaVuSansMono.ttf}[Path=\detokenize{/home/hauke/.local/lib/python3.11/site-packages/matplotlib/mpl-data/fonts/ttf/}]
%%   \makeatletter\@ifpackageloaded{underscore}{}{\usepackage[strings]{underscore}}\makeatother
%%
\begingroup%
\makeatletter%
\begin{pgfpicture}%
\pgfpathrectangle{\pgfpointorigin}{\pgfqpoint{5.697678in}{2.407400in}}%
\pgfusepath{use as bounding box, clip}%
\begin{pgfscope}%
\pgfsetbuttcap%
\pgfsetmiterjoin%
\definecolor{currentfill}{rgb}{1.000000,1.000000,1.000000}%
\pgfsetfillcolor{currentfill}%
\pgfsetlinewidth{0.000000pt}%
\definecolor{currentstroke}{rgb}{1.000000,1.000000,1.000000}%
\pgfsetstrokecolor{currentstroke}%
\pgfsetdash{}{0pt}%
\pgfpathmoveto{\pgfqpoint{0.000000in}{0.000000in}}%
\pgfpathlineto{\pgfqpoint{5.697678in}{0.000000in}}%
\pgfpathlineto{\pgfqpoint{5.697678in}{2.407400in}}%
\pgfpathlineto{\pgfqpoint{0.000000in}{2.407400in}}%
\pgfpathlineto{\pgfqpoint{0.000000in}{0.000000in}}%
\pgfpathclose%
\pgfusepath{fill}%
\end{pgfscope}%
\begin{pgfscope}%
\pgfsetbuttcap%
\pgfsetmiterjoin%
\definecolor{currentfill}{rgb}{1.000000,1.000000,1.000000}%
\pgfsetfillcolor{currentfill}%
\pgfsetlinewidth{0.000000pt}%
\definecolor{currentstroke}{rgb}{0.000000,0.000000,0.000000}%
\pgfsetstrokecolor{currentstroke}%
\pgfsetstrokeopacity{0.000000}%
\pgfsetdash{}{0pt}%
\pgfpathmoveto{\pgfqpoint{0.592976in}{0.451389in}}%
\pgfpathlineto{\pgfqpoint{4.297997in}{0.451389in}}%
\pgfpathlineto{\pgfqpoint{4.297997in}{2.407400in}}%
\pgfpathlineto{\pgfqpoint{0.592976in}{2.407400in}}%
\pgfpathlineto{\pgfqpoint{0.592976in}{0.451389in}}%
\pgfpathclose%
\pgfusepath{fill}%
\end{pgfscope}%
\begin{pgfscope}%
\pgfpathrectangle{\pgfqpoint{0.592976in}{0.451389in}}{\pgfqpoint{3.705021in}{1.956011in}}%
\pgfusepath{clip}%
\pgfsetroundcap%
\pgfsetroundjoin%
\pgfsetlinewidth{1.003750pt}%
\definecolor{currentstroke}{rgb}{0.800000,0.800000,0.800000}%
\pgfsetstrokecolor{currentstroke}%
\pgfsetdash{}{0pt}%
\pgfpathmoveto{\pgfqpoint{1.019317in}{0.451389in}}%
\pgfpathlineto{\pgfqpoint{1.019317in}{2.407400in}}%
\pgfusepath{stroke}%
\end{pgfscope}%
\begin{pgfscope}%
\definecolor{textcolor}{rgb}{0.150000,0.150000,0.150000}%
\pgfsetstrokecolor{textcolor}%
\pgfsetfillcolor{textcolor}%
\pgftext[x=1.019317in,y=0.319444in,,top]{\color{textcolor}\sffamily\fontsize{9.000000}{10.800000}\selectfont 10000}%
\end{pgfscope}%
\begin{pgfscope}%
\pgfpathrectangle{\pgfqpoint{0.592976in}{0.451389in}}{\pgfqpoint{3.705021in}{1.956011in}}%
\pgfusepath{clip}%
\pgfsetroundcap%
\pgfsetroundjoin%
\pgfsetlinewidth{1.003750pt}%
\definecolor{currentstroke}{rgb}{0.800000,0.800000,0.800000}%
\pgfsetstrokecolor{currentstroke}%
\pgfsetdash{}{0pt}%
\pgfpathmoveto{\pgfqpoint{1.463413in}{0.451389in}}%
\pgfpathlineto{\pgfqpoint{1.463413in}{2.407400in}}%
\pgfusepath{stroke}%
\end{pgfscope}%
\begin{pgfscope}%
\definecolor{textcolor}{rgb}{0.150000,0.150000,0.150000}%
\pgfsetstrokecolor{textcolor}%
\pgfsetfillcolor{textcolor}%
\pgftext[x=1.463413in,y=0.319444in,,top]{\color{textcolor}\sffamily\fontsize{9.000000}{10.800000}\selectfont 15000}%
\end{pgfscope}%
\begin{pgfscope}%
\pgfpathrectangle{\pgfqpoint{0.592976in}{0.451389in}}{\pgfqpoint{3.705021in}{1.956011in}}%
\pgfusepath{clip}%
\pgfsetroundcap%
\pgfsetroundjoin%
\pgfsetlinewidth{1.003750pt}%
\definecolor{currentstroke}{rgb}{0.800000,0.800000,0.800000}%
\pgfsetstrokecolor{currentstroke}%
\pgfsetdash{}{0pt}%
\pgfpathmoveto{\pgfqpoint{1.907509in}{0.451389in}}%
\pgfpathlineto{\pgfqpoint{1.907509in}{2.407400in}}%
\pgfusepath{stroke}%
\end{pgfscope}%
\begin{pgfscope}%
\definecolor{textcolor}{rgb}{0.150000,0.150000,0.150000}%
\pgfsetstrokecolor{textcolor}%
\pgfsetfillcolor{textcolor}%
\pgftext[x=1.907509in,y=0.319444in,,top]{\color{textcolor}\sffamily\fontsize{9.000000}{10.800000}\selectfont 20000}%
\end{pgfscope}%
\begin{pgfscope}%
\pgfpathrectangle{\pgfqpoint{0.592976in}{0.451389in}}{\pgfqpoint{3.705021in}{1.956011in}}%
\pgfusepath{clip}%
\pgfsetroundcap%
\pgfsetroundjoin%
\pgfsetlinewidth{1.003750pt}%
\definecolor{currentstroke}{rgb}{0.800000,0.800000,0.800000}%
\pgfsetstrokecolor{currentstroke}%
\pgfsetdash{}{0pt}%
\pgfpathmoveto{\pgfqpoint{2.351605in}{0.451389in}}%
\pgfpathlineto{\pgfqpoint{2.351605in}{2.407400in}}%
\pgfusepath{stroke}%
\end{pgfscope}%
\begin{pgfscope}%
\definecolor{textcolor}{rgb}{0.150000,0.150000,0.150000}%
\pgfsetstrokecolor{textcolor}%
\pgfsetfillcolor{textcolor}%
\pgftext[x=2.351605in,y=0.319444in,,top]{\color{textcolor}\sffamily\fontsize{9.000000}{10.800000}\selectfont 25000}%
\end{pgfscope}%
\begin{pgfscope}%
\pgfpathrectangle{\pgfqpoint{0.592976in}{0.451389in}}{\pgfqpoint{3.705021in}{1.956011in}}%
\pgfusepath{clip}%
\pgfsetroundcap%
\pgfsetroundjoin%
\pgfsetlinewidth{1.003750pt}%
\definecolor{currentstroke}{rgb}{0.800000,0.800000,0.800000}%
\pgfsetstrokecolor{currentstroke}%
\pgfsetdash{}{0pt}%
\pgfpathmoveto{\pgfqpoint{2.795701in}{0.451389in}}%
\pgfpathlineto{\pgfqpoint{2.795701in}{2.407400in}}%
\pgfusepath{stroke}%
\end{pgfscope}%
\begin{pgfscope}%
\definecolor{textcolor}{rgb}{0.150000,0.150000,0.150000}%
\pgfsetstrokecolor{textcolor}%
\pgfsetfillcolor{textcolor}%
\pgftext[x=2.795701in,y=0.319444in,,top]{\color{textcolor}\sffamily\fontsize{9.000000}{10.800000}\selectfont 30000}%
\end{pgfscope}%
\begin{pgfscope}%
\pgfpathrectangle{\pgfqpoint{0.592976in}{0.451389in}}{\pgfqpoint{3.705021in}{1.956011in}}%
\pgfusepath{clip}%
\pgfsetroundcap%
\pgfsetroundjoin%
\pgfsetlinewidth{1.003750pt}%
\definecolor{currentstroke}{rgb}{0.800000,0.800000,0.800000}%
\pgfsetstrokecolor{currentstroke}%
\pgfsetdash{}{0pt}%
\pgfpathmoveto{\pgfqpoint{3.239797in}{0.451389in}}%
\pgfpathlineto{\pgfqpoint{3.239797in}{2.407400in}}%
\pgfusepath{stroke}%
\end{pgfscope}%
\begin{pgfscope}%
\definecolor{textcolor}{rgb}{0.150000,0.150000,0.150000}%
\pgfsetstrokecolor{textcolor}%
\pgfsetfillcolor{textcolor}%
\pgftext[x=3.239797in,y=0.319444in,,top]{\color{textcolor}\sffamily\fontsize{9.000000}{10.800000}\selectfont 35000}%
\end{pgfscope}%
\begin{pgfscope}%
\pgfpathrectangle{\pgfqpoint{0.592976in}{0.451389in}}{\pgfqpoint{3.705021in}{1.956011in}}%
\pgfusepath{clip}%
\pgfsetroundcap%
\pgfsetroundjoin%
\pgfsetlinewidth{1.003750pt}%
\definecolor{currentstroke}{rgb}{0.800000,0.800000,0.800000}%
\pgfsetstrokecolor{currentstroke}%
\pgfsetdash{}{0pt}%
\pgfpathmoveto{\pgfqpoint{3.683892in}{0.451389in}}%
\pgfpathlineto{\pgfqpoint{3.683892in}{2.407400in}}%
\pgfusepath{stroke}%
\end{pgfscope}%
\begin{pgfscope}%
\definecolor{textcolor}{rgb}{0.150000,0.150000,0.150000}%
\pgfsetstrokecolor{textcolor}%
\pgfsetfillcolor{textcolor}%
\pgftext[x=3.683892in,y=0.319444in,,top]{\color{textcolor}\sffamily\fontsize{9.000000}{10.800000}\selectfont 40000}%
\end{pgfscope}%
\begin{pgfscope}%
\pgfpathrectangle{\pgfqpoint{0.592976in}{0.451389in}}{\pgfqpoint{3.705021in}{1.956011in}}%
\pgfusepath{clip}%
\pgfsetroundcap%
\pgfsetroundjoin%
\pgfsetlinewidth{1.003750pt}%
\definecolor{currentstroke}{rgb}{0.800000,0.800000,0.800000}%
\pgfsetstrokecolor{currentstroke}%
\pgfsetdash{}{0pt}%
\pgfpathmoveto{\pgfqpoint{4.127988in}{0.451389in}}%
\pgfpathlineto{\pgfqpoint{4.127988in}{2.407400in}}%
\pgfusepath{stroke}%
\end{pgfscope}%
\begin{pgfscope}%
\definecolor{textcolor}{rgb}{0.150000,0.150000,0.150000}%
\pgfsetstrokecolor{textcolor}%
\pgfsetfillcolor{textcolor}%
\pgftext[x=4.127988in,y=0.319444in,,top]{\color{textcolor}\sffamily\fontsize{9.000000}{10.800000}\selectfont 45000}%
\end{pgfscope}%
\begin{pgfscope}%
\definecolor{textcolor}{rgb}{0.150000,0.150000,0.150000}%
\pgfsetstrokecolor{textcolor}%
\pgfsetfillcolor{textcolor}%
\pgftext[x=2.445487in,y=0.125000in,,top]{\color{textcolor}\sffamily\fontsize{9.000000}{10.800000}\selectfont Input obstacle vertices}%
\end{pgfscope}%
\begin{pgfscope}%
\pgfpathrectangle{\pgfqpoint{0.592976in}{0.451389in}}{\pgfqpoint{3.705021in}{1.956011in}}%
\pgfusepath{clip}%
\pgfsetroundcap%
\pgfsetroundjoin%
\pgfsetlinewidth{1.003750pt}%
\definecolor{currentstroke}{rgb}{0.800000,0.800000,0.800000}%
\pgfsetstrokecolor{currentstroke}%
\pgfsetdash{}{0pt}%
\pgfpathmoveto{\pgfqpoint{0.592976in}{0.644245in}}%
\pgfpathlineto{\pgfqpoint{4.297997in}{0.644245in}}%
\pgfusepath{stroke}%
\end{pgfscope}%
\begin{pgfscope}%
\definecolor{textcolor}{rgb}{0.150000,0.150000,0.150000}%
\pgfsetstrokecolor{textcolor}%
\pgfsetfillcolor{textcolor}%
\pgftext[x=0.194444in, y=0.596759in, left, base]{\color{textcolor}\sffamily\fontsize{9.000000}{10.800000}\selectfont \(\displaystyle {10^{-2}}\)}%
\end{pgfscope}%
\begin{pgfscope}%
\pgfpathrectangle{\pgfqpoint{0.592976in}{0.451389in}}{\pgfqpoint{3.705021in}{1.956011in}}%
\pgfusepath{clip}%
\pgfsetroundcap%
\pgfsetroundjoin%
\pgfsetlinewidth{1.003750pt}%
\definecolor{currentstroke}{rgb}{0.800000,0.800000,0.800000}%
\pgfsetstrokecolor{currentstroke}%
\pgfsetdash{}{0pt}%
\pgfpathmoveto{\pgfqpoint{0.592976in}{0.966921in}}%
\pgfpathlineto{\pgfqpoint{4.297997in}{0.966921in}}%
\pgfusepath{stroke}%
\end{pgfscope}%
\begin{pgfscope}%
\definecolor{textcolor}{rgb}{0.150000,0.150000,0.150000}%
\pgfsetstrokecolor{textcolor}%
\pgfsetfillcolor{textcolor}%
\pgftext[x=0.194444in, y=0.919436in, left, base]{\color{textcolor}\sffamily\fontsize{9.000000}{10.800000}\selectfont \(\displaystyle {10^{-1}}\)}%
\end{pgfscope}%
\begin{pgfscope}%
\pgfpathrectangle{\pgfqpoint{0.592976in}{0.451389in}}{\pgfqpoint{3.705021in}{1.956011in}}%
\pgfusepath{clip}%
\pgfsetroundcap%
\pgfsetroundjoin%
\pgfsetlinewidth{1.003750pt}%
\definecolor{currentstroke}{rgb}{0.800000,0.800000,0.800000}%
\pgfsetstrokecolor{currentstroke}%
\pgfsetdash{}{0pt}%
\pgfpathmoveto{\pgfqpoint{0.592976in}{1.289598in}}%
\pgfpathlineto{\pgfqpoint{4.297997in}{1.289598in}}%
\pgfusepath{stroke}%
\end{pgfscope}%
\begin{pgfscope}%
\definecolor{textcolor}{rgb}{0.150000,0.150000,0.150000}%
\pgfsetstrokecolor{textcolor}%
\pgfsetfillcolor{textcolor}%
\pgftext[x=0.274690in, y=1.242113in, left, base]{\color{textcolor}\sffamily\fontsize{9.000000}{10.800000}\selectfont \(\displaystyle {10^{0}}\)}%
\end{pgfscope}%
\begin{pgfscope}%
\pgfpathrectangle{\pgfqpoint{0.592976in}{0.451389in}}{\pgfqpoint{3.705021in}{1.956011in}}%
\pgfusepath{clip}%
\pgfsetroundcap%
\pgfsetroundjoin%
\pgfsetlinewidth{1.003750pt}%
\definecolor{currentstroke}{rgb}{0.800000,0.800000,0.800000}%
\pgfsetstrokecolor{currentstroke}%
\pgfsetdash{}{0pt}%
\pgfpathmoveto{\pgfqpoint{0.592976in}{1.612275in}}%
\pgfpathlineto{\pgfqpoint{4.297997in}{1.612275in}}%
\pgfusepath{stroke}%
\end{pgfscope}%
\begin{pgfscope}%
\definecolor{textcolor}{rgb}{0.150000,0.150000,0.150000}%
\pgfsetstrokecolor{textcolor}%
\pgfsetfillcolor{textcolor}%
\pgftext[x=0.274690in, y=1.564790in, left, base]{\color{textcolor}\sffamily\fontsize{9.000000}{10.800000}\selectfont \(\displaystyle {10^{1}}\)}%
\end{pgfscope}%
\begin{pgfscope}%
\pgfpathrectangle{\pgfqpoint{0.592976in}{0.451389in}}{\pgfqpoint{3.705021in}{1.956011in}}%
\pgfusepath{clip}%
\pgfsetroundcap%
\pgfsetroundjoin%
\pgfsetlinewidth{1.003750pt}%
\definecolor{currentstroke}{rgb}{0.800000,0.800000,0.800000}%
\pgfsetstrokecolor{currentstroke}%
\pgfsetdash{}{0pt}%
\pgfpathmoveto{\pgfqpoint{0.592976in}{1.934952in}}%
\pgfpathlineto{\pgfqpoint{4.297997in}{1.934952in}}%
\pgfusepath{stroke}%
\end{pgfscope}%
\begin{pgfscope}%
\definecolor{textcolor}{rgb}{0.150000,0.150000,0.150000}%
\pgfsetstrokecolor{textcolor}%
\pgfsetfillcolor{textcolor}%
\pgftext[x=0.274690in, y=1.887467in, left, base]{\color{textcolor}\sffamily\fontsize{9.000000}{10.800000}\selectfont \(\displaystyle {10^{2}}\)}%
\end{pgfscope}%
\begin{pgfscope}%
\pgfpathrectangle{\pgfqpoint{0.592976in}{0.451389in}}{\pgfqpoint{3.705021in}{1.956011in}}%
\pgfusepath{clip}%
\pgfsetroundcap%
\pgfsetroundjoin%
\pgfsetlinewidth{1.003750pt}%
\definecolor{currentstroke}{rgb}{0.800000,0.800000,0.800000}%
\pgfsetstrokecolor{currentstroke}%
\pgfsetdash{}{0pt}%
\pgfpathmoveto{\pgfqpoint{0.592976in}{2.257629in}}%
\pgfpathlineto{\pgfqpoint{4.297997in}{2.257629in}}%
\pgfusepath{stroke}%
\end{pgfscope}%
\begin{pgfscope}%
\definecolor{textcolor}{rgb}{0.150000,0.150000,0.150000}%
\pgfsetstrokecolor{textcolor}%
\pgfsetfillcolor{textcolor}%
\pgftext[x=0.274690in, y=2.210144in, left, base]{\color{textcolor}\sffamily\fontsize{9.000000}{10.800000}\selectfont \(\displaystyle {10^{3}}\)}%
\end{pgfscope}%
\begin{pgfscope}%
\definecolor{textcolor}{rgb}{0.150000,0.150000,0.150000}%
\pgfsetstrokecolor{textcolor}%
\pgfsetfillcolor{textcolor}%
\pgftext[x=0.125000in,y=1.429394in,,bottom,rotate=90.000000]{\color{textcolor}\sffamily\fontsize{9.000000}{10.800000}\selectfont Time in s}%
\end{pgfscope}%
\begin{pgfscope}%
\pgfpathrectangle{\pgfqpoint{0.592976in}{0.451389in}}{\pgfqpoint{3.705021in}{1.956011in}}%
\pgfusepath{clip}%
\pgfsetbuttcap%
\pgfsetroundjoin%
\definecolor{currentfill}{rgb}{0.003922,0.450980,0.698039}%
\pgfsetfillcolor{currentfill}%
\pgfsetfillopacity{0.200000}%
\pgfsetlinewidth{1.003750pt}%
\definecolor{currentstroke}{rgb}{0.003922,0.450980,0.698039}%
\pgfsetstrokecolor{currentstroke}%
\pgfsetstrokeopacity{0.200000}%
\pgfsetdash{}{0pt}%
\pgfsys@defobject{currentmarker}{\pgfqpoint{0.761386in}{1.754091in}}{\pgfqpoint{4.129587in}{2.318490in}}{%
\pgfpathmoveto{\pgfqpoint{0.761386in}{1.754619in}}%
\pgfpathlineto{\pgfqpoint{0.761386in}{1.754091in}}%
\pgfpathlineto{\pgfqpoint{1.273162in}{1.926439in}}%
\pgfpathlineto{\pgfqpoint{1.720455in}{2.046433in}}%
\pgfpathlineto{\pgfqpoint{2.156291in}{2.132598in}}%
\pgfpathlineto{\pgfqpoint{3.169985in}{2.249715in}}%
\pgfpathlineto{\pgfqpoint{4.129587in}{2.316449in}}%
\pgfpathlineto{\pgfqpoint{4.129587in}{2.318490in}}%
\pgfpathlineto{\pgfqpoint{4.129587in}{2.318490in}}%
\pgfpathlineto{\pgfqpoint{3.169985in}{2.251777in}}%
\pgfpathlineto{\pgfqpoint{2.156291in}{2.134478in}}%
\pgfpathlineto{\pgfqpoint{1.720455in}{2.049240in}}%
\pgfpathlineto{\pgfqpoint{1.273162in}{1.930317in}}%
\pgfpathlineto{\pgfqpoint{0.761386in}{1.754619in}}%
\pgfpathlineto{\pgfqpoint{0.761386in}{1.754619in}}%
\pgfpathclose%
\pgfusepath{stroke,fill}%
}%
\begin{pgfscope}%
\pgfsys@transformshift{0.000000in}{0.000000in}%
\pgfsys@useobject{currentmarker}{}%
\end{pgfscope}%
\end{pgfscope}%
\begin{pgfscope}%
\pgfpathrectangle{\pgfqpoint{0.592976in}{0.451389in}}{\pgfqpoint{3.705021in}{1.956011in}}%
\pgfusepath{clip}%
\pgfsetbuttcap%
\pgfsetroundjoin%
\definecolor{currentfill}{rgb}{0.870588,0.560784,0.019608}%
\pgfsetfillcolor{currentfill}%
\pgfsetfillopacity{0.200000}%
\pgfsetlinewidth{1.003750pt}%
\definecolor{currentstroke}{rgb}{0.870588,0.560784,0.019608}%
\pgfsetstrokecolor{currentstroke}%
\pgfsetstrokeopacity{0.200000}%
\pgfsetdash{}{0pt}%
\pgfsys@defobject{currentmarker}{\pgfqpoint{0.761386in}{1.694946in}}{\pgfqpoint{4.129587in}{2.251360in}}{%
\pgfpathmoveto{\pgfqpoint{0.761386in}{1.696190in}}%
\pgfpathlineto{\pgfqpoint{0.761386in}{1.694946in}}%
\pgfpathlineto{\pgfqpoint{1.273162in}{1.868929in}}%
\pgfpathlineto{\pgfqpoint{1.720455in}{1.984345in}}%
\pgfpathlineto{\pgfqpoint{2.156291in}{2.067871in}}%
\pgfpathlineto{\pgfqpoint{3.169985in}{2.187680in}}%
\pgfpathlineto{\pgfqpoint{4.129587in}{2.250044in}}%
\pgfpathlineto{\pgfqpoint{4.129587in}{2.251360in}}%
\pgfpathlineto{\pgfqpoint{4.129587in}{2.251360in}}%
\pgfpathlineto{\pgfqpoint{3.169985in}{2.190306in}}%
\pgfpathlineto{\pgfqpoint{2.156291in}{2.070284in}}%
\pgfpathlineto{\pgfqpoint{1.720455in}{1.987618in}}%
\pgfpathlineto{\pgfqpoint{1.273162in}{1.874042in}}%
\pgfpathlineto{\pgfqpoint{0.761386in}{1.696190in}}%
\pgfpathlineto{\pgfqpoint{0.761386in}{1.696190in}}%
\pgfpathclose%
\pgfusepath{stroke,fill}%
}%
\begin{pgfscope}%
\pgfsys@transformshift{0.000000in}{0.000000in}%
\pgfsys@useobject{currentmarker}{}%
\end{pgfscope}%
\end{pgfscope}%
\begin{pgfscope}%
\pgfpathrectangle{\pgfqpoint{0.592976in}{0.451389in}}{\pgfqpoint{3.705021in}{1.956011in}}%
\pgfusepath{clip}%
\pgfsetbuttcap%
\pgfsetroundjoin%
\definecolor{currentfill}{rgb}{0.007843,0.619608,0.450980}%
\pgfsetfillcolor{currentfill}%
\pgfsetfillopacity{0.200000}%
\pgfsetlinewidth{1.003750pt}%
\definecolor{currentstroke}{rgb}{0.007843,0.619608,0.450980}%
\pgfsetstrokecolor{currentstroke}%
\pgfsetstrokeopacity{0.200000}%
\pgfsetdash{}{0pt}%
\pgfsys@defobject{currentmarker}{\pgfqpoint{0.761386in}{0.963073in}}{\pgfqpoint{4.129587in}{1.256810in}}{%
\pgfpathmoveto{\pgfqpoint{0.761386in}{0.975032in}}%
\pgfpathlineto{\pgfqpoint{0.761386in}{0.963073in}}%
\pgfpathlineto{\pgfqpoint{1.273162in}{1.060138in}}%
\pgfpathlineto{\pgfqpoint{1.720455in}{1.109000in}}%
\pgfpathlineto{\pgfqpoint{2.156291in}{1.146830in}}%
\pgfpathlineto{\pgfqpoint{3.169985in}{1.206855in}}%
\pgfpathlineto{\pgfqpoint{4.129587in}{1.252957in}}%
\pgfpathlineto{\pgfqpoint{4.129587in}{1.256810in}}%
\pgfpathlineto{\pgfqpoint{4.129587in}{1.256810in}}%
\pgfpathlineto{\pgfqpoint{3.169985in}{1.209741in}}%
\pgfpathlineto{\pgfqpoint{2.156291in}{1.149290in}}%
\pgfpathlineto{\pgfqpoint{1.720455in}{1.113438in}}%
\pgfpathlineto{\pgfqpoint{1.273162in}{1.062539in}}%
\pgfpathlineto{\pgfqpoint{0.761386in}{0.975032in}}%
\pgfpathlineto{\pgfqpoint{0.761386in}{0.975032in}}%
\pgfpathclose%
\pgfusepath{stroke,fill}%
}%
\begin{pgfscope}%
\pgfsys@transformshift{0.000000in}{0.000000in}%
\pgfsys@useobject{currentmarker}{}%
\end{pgfscope}%
\end{pgfscope}%
\begin{pgfscope}%
\pgfpathrectangle{\pgfqpoint{0.592976in}{0.451389in}}{\pgfqpoint{3.705021in}{1.956011in}}%
\pgfusepath{clip}%
\pgfsetbuttcap%
\pgfsetroundjoin%
\definecolor{currentfill}{rgb}{0.835294,0.368627,0.000000}%
\pgfsetfillcolor{currentfill}%
\pgfsetfillopacity{0.200000}%
\pgfsetlinewidth{1.003750pt}%
\definecolor{currentstroke}{rgb}{0.835294,0.368627,0.000000}%
\pgfsetstrokecolor{currentstroke}%
\pgfsetstrokeopacity{0.200000}%
\pgfsetdash{}{0pt}%
\pgfsys@defobject{currentmarker}{\pgfqpoint{0.761386in}{0.855215in}}{\pgfqpoint{4.129587in}{1.117644in}}{%
\pgfpathmoveto{\pgfqpoint{0.761386in}{0.882401in}}%
\pgfpathlineto{\pgfqpoint{0.761386in}{0.855215in}}%
\pgfpathlineto{\pgfqpoint{1.273162in}{0.930147in}}%
\pgfpathlineto{\pgfqpoint{1.720455in}{0.984135in}}%
\pgfpathlineto{\pgfqpoint{2.156291in}{1.008993in}}%
\pgfpathlineto{\pgfqpoint{3.169985in}{1.068481in}}%
\pgfpathlineto{\pgfqpoint{4.129587in}{1.107762in}}%
\pgfpathlineto{\pgfqpoint{4.129587in}{1.117644in}}%
\pgfpathlineto{\pgfqpoint{4.129587in}{1.117644in}}%
\pgfpathlineto{\pgfqpoint{3.169985in}{1.077992in}}%
\pgfpathlineto{\pgfqpoint{2.156291in}{1.023400in}}%
\pgfpathlineto{\pgfqpoint{1.720455in}{0.998796in}}%
\pgfpathlineto{\pgfqpoint{1.273162in}{0.949141in}}%
\pgfpathlineto{\pgfqpoint{0.761386in}{0.882401in}}%
\pgfpathlineto{\pgfqpoint{0.761386in}{0.882401in}}%
\pgfpathclose%
\pgfusepath{stroke,fill}%
}%
\begin{pgfscope}%
\pgfsys@transformshift{0.000000in}{0.000000in}%
\pgfsys@useobject{currentmarker}{}%
\end{pgfscope}%
\end{pgfscope}%
\begin{pgfscope}%
\pgfpathrectangle{\pgfqpoint{0.592976in}{0.451389in}}{\pgfqpoint{3.705021in}{1.956011in}}%
\pgfusepath{clip}%
\pgfsetbuttcap%
\pgfsetroundjoin%
\definecolor{currentfill}{rgb}{0.800000,0.470588,0.737255}%
\pgfsetfillcolor{currentfill}%
\pgfsetfillopacity{0.200000}%
\pgfsetlinewidth{1.003750pt}%
\definecolor{currentstroke}{rgb}{0.800000,0.470588,0.737255}%
\pgfsetstrokecolor{currentstroke}%
\pgfsetstrokeopacity{0.200000}%
\pgfsetdash{}{0pt}%
\pgfsys@defobject{currentmarker}{\pgfqpoint{0.761386in}{1.599559in}}{\pgfqpoint{4.129587in}{2.183831in}}{%
\pgfpathmoveto{\pgfqpoint{0.761386in}{1.601838in}}%
\pgfpathlineto{\pgfqpoint{0.761386in}{1.599559in}}%
\pgfpathlineto{\pgfqpoint{1.273162in}{1.771506in}}%
\pgfpathlineto{\pgfqpoint{1.720455in}{1.900887in}}%
\pgfpathlineto{\pgfqpoint{2.156291in}{1.991884in}}%
\pgfpathlineto{\pgfqpoint{3.169985in}{2.105118in}}%
\pgfpathlineto{\pgfqpoint{4.129587in}{2.179243in}}%
\pgfpathlineto{\pgfqpoint{4.129587in}{2.183831in}}%
\pgfpathlineto{\pgfqpoint{4.129587in}{2.183831in}}%
\pgfpathlineto{\pgfqpoint{3.169985in}{2.106364in}}%
\pgfpathlineto{\pgfqpoint{2.156291in}{1.993660in}}%
\pgfpathlineto{\pgfqpoint{1.720455in}{1.903842in}}%
\pgfpathlineto{\pgfqpoint{1.273162in}{1.774169in}}%
\pgfpathlineto{\pgfqpoint{0.761386in}{1.601838in}}%
\pgfpathlineto{\pgfqpoint{0.761386in}{1.601838in}}%
\pgfpathclose%
\pgfusepath{stroke,fill}%
}%
\begin{pgfscope}%
\pgfsys@transformshift{0.000000in}{0.000000in}%
\pgfsys@useobject{currentmarker}{}%
\end{pgfscope}%
\end{pgfscope}%
\begin{pgfscope}%
\pgfpathrectangle{\pgfqpoint{0.592976in}{0.451389in}}{\pgfqpoint{3.705021in}{1.956011in}}%
\pgfusepath{clip}%
\pgfsetbuttcap%
\pgfsetroundjoin%
\definecolor{currentfill}{rgb}{0.792157,0.568627,0.380392}%
\pgfsetfillcolor{currentfill}%
\pgfsetfillopacity{0.200000}%
\pgfsetlinewidth{1.003750pt}%
\definecolor{currentstroke}{rgb}{0.792157,0.568627,0.380392}%
\pgfsetstrokecolor{currentstroke}%
\pgfsetstrokeopacity{0.200000}%
\pgfsetdash{}{0pt}%
\pgfsys@defobject{currentmarker}{\pgfqpoint{0.761386in}{0.540298in}}{\pgfqpoint{4.129587in}{0.919634in}}{%
\pgfpathmoveto{\pgfqpoint{0.761386in}{0.626728in}}%
\pgfpathlineto{\pgfqpoint{0.761386in}{0.540298in}}%
\pgfpathlineto{\pgfqpoint{1.273162in}{0.684487in}}%
\pgfpathlineto{\pgfqpoint{1.720455in}{0.705066in}}%
\pgfpathlineto{\pgfqpoint{2.156291in}{0.738446in}}%
\pgfpathlineto{\pgfqpoint{3.169985in}{0.785244in}}%
\pgfpathlineto{\pgfqpoint{4.129587in}{0.917653in}}%
\pgfpathlineto{\pgfqpoint{4.129587in}{0.919634in}}%
\pgfpathlineto{\pgfqpoint{4.129587in}{0.919634in}}%
\pgfpathlineto{\pgfqpoint{3.169985in}{0.794153in}}%
\pgfpathlineto{\pgfqpoint{2.156291in}{0.792682in}}%
\pgfpathlineto{\pgfqpoint{1.720455in}{0.708029in}}%
\pgfpathlineto{\pgfqpoint{1.273162in}{0.690907in}}%
\pgfpathlineto{\pgfqpoint{0.761386in}{0.626728in}}%
\pgfpathlineto{\pgfqpoint{0.761386in}{0.626728in}}%
\pgfpathclose%
\pgfusepath{stroke,fill}%
}%
\begin{pgfscope}%
\pgfsys@transformshift{0.000000in}{0.000000in}%
\pgfsys@useobject{currentmarker}{}%
\end{pgfscope}%
\end{pgfscope}%
\begin{pgfscope}%
\pgfsetrectcap%
\pgfsetmiterjoin%
\pgfsetlinewidth{1.254687pt}%
\definecolor{currentstroke}{rgb}{0.800000,0.800000,0.800000}%
\pgfsetstrokecolor{currentstroke}%
\pgfsetdash{}{0pt}%
\pgfpathmoveto{\pgfqpoint{0.592976in}{0.451389in}}%
\pgfpathlineto{\pgfqpoint{0.592976in}{2.407400in}}%
\pgfusepath{stroke}%
\end{pgfscope}%
\begin{pgfscope}%
\pgfsetrectcap%
\pgfsetmiterjoin%
\pgfsetlinewidth{1.254687pt}%
\definecolor{currentstroke}{rgb}{0.800000,0.800000,0.800000}%
\pgfsetstrokecolor{currentstroke}%
\pgfsetdash{}{0pt}%
\pgfpathmoveto{\pgfqpoint{4.297997in}{0.451389in}}%
\pgfpathlineto{\pgfqpoint{4.297997in}{2.407400in}}%
\pgfusepath{stroke}%
\end{pgfscope}%
\begin{pgfscope}%
\pgfsetrectcap%
\pgfsetmiterjoin%
\pgfsetlinewidth{1.254687pt}%
\definecolor{currentstroke}{rgb}{0.800000,0.800000,0.800000}%
\pgfsetstrokecolor{currentstroke}%
\pgfsetdash{}{0pt}%
\pgfpathmoveto{\pgfqpoint{0.592976in}{0.451389in}}%
\pgfpathlineto{\pgfqpoint{4.297997in}{0.451389in}}%
\pgfusepath{stroke}%
\end{pgfscope}%
\begin{pgfscope}%
\pgfsetrectcap%
\pgfsetmiterjoin%
\pgfsetlinewidth{1.254687pt}%
\definecolor{currentstroke}{rgb}{0.800000,0.800000,0.800000}%
\pgfsetstrokecolor{currentstroke}%
\pgfsetdash{}{0pt}%
\pgfpathmoveto{\pgfqpoint{0.592976in}{2.407400in}}%
\pgfpathlineto{\pgfqpoint{4.297997in}{2.407400in}}%
\pgfusepath{stroke}%
\end{pgfscope}%
\begin{pgfscope}%
\pgfsetbuttcap%
\pgfsetmiterjoin%
\definecolor{currentfill}{rgb}{1.000000,1.000000,1.000000}%
\pgfsetfillcolor{currentfill}%
\pgfsetfillopacity{0.800000}%
\pgfsetlinewidth{1.003750pt}%
\definecolor{currentstroke}{rgb}{0.800000,0.800000,0.800000}%
\pgfsetstrokecolor{currentstroke}%
\pgfsetstrokeopacity{0.800000}%
\pgfsetdash{}{0pt}%
\pgfpathmoveto{\pgfqpoint{4.478123in}{0.538404in}}%
\pgfpathlineto{\pgfqpoint{5.672678in}{0.538404in}}%
\pgfpathquadraticcurveto{\pgfqpoint{5.697678in}{0.538404in}}{\pgfqpoint{5.697678in}{0.563404in}}%
\pgfpathlineto{\pgfqpoint{5.697678in}{2.295385in}}%
\pgfpathquadraticcurveto{\pgfqpoint{5.697678in}{2.320385in}}{\pgfqpoint{5.672678in}{2.320385in}}%
\pgfpathlineto{\pgfqpoint{4.478123in}{2.320385in}}%
\pgfpathquadraticcurveto{\pgfqpoint{4.453123in}{2.320385in}}{\pgfqpoint{4.453123in}{2.295385in}}%
\pgfpathlineto{\pgfqpoint{4.453123in}{0.563404in}}%
\pgfpathquadraticcurveto{\pgfqpoint{4.453123in}{0.538404in}}{\pgfqpoint{4.478123in}{0.538404in}}%
\pgfpathlineto{\pgfqpoint{4.478123in}{0.538404in}}%
\pgfpathclose%
\pgfusepath{stroke,fill}%
\end{pgfscope}%
\begin{pgfscope}%
\definecolor{textcolor}{rgb}{0.150000,0.150000,0.150000}%
\pgfsetstrokecolor{textcolor}%
\pgfsetfillcolor{textcolor}%
\pgftext[x=4.872001in,y=2.175414in,left,base]{\color{textcolor}\sffamily\fontsize{9.000000}{10.800000}\selectfont Legend}%
\end{pgfscope}%
\begin{pgfscope}%
\pgfsetroundcap%
\pgfsetroundjoin%
\pgfsetlinewidth{1.505625pt}%
\definecolor{currentstroke}{rgb}{0.003922,0.450980,0.698039}%
\pgfsetstrokecolor{currentstroke}%
\pgfsetdash{}{0pt}%
\pgfpathmoveto{\pgfqpoint{4.503123in}{2.031664in}}%
\pgfpathlineto{\pgfqpoint{4.628123in}{2.031664in}}%
\pgfpathlineto{\pgfqpoint{4.753123in}{2.031664in}}%
\pgfusepath{stroke}%
\end{pgfscope}%
\begin{pgfscope}%
\definecolor{textcolor}{rgb}{0.150000,0.150000,0.150000}%
\pgfsetstrokecolor{textcolor}%
\pgfsetfillcolor{textcolor}%
\pgftext[x=4.853123in,y=1.987914in,left,base]{\color{textcolor}\sffamily\fontsize{9.000000}{10.800000}\selectfont Total time}%
\end{pgfscope}%
\begin{pgfscope}%
\pgfsetroundcap%
\pgfsetroundjoin%
\pgfsetlinewidth{1.505625pt}%
\definecolor{currentstroke}{rgb}{0.870588,0.560784,0.019608}%
\pgfsetstrokecolor{currentstroke}%
\pgfsetdash{}{0pt}%
\pgfpathmoveto{\pgfqpoint{4.503123in}{1.844164in}}%
\pgfpathlineto{\pgfqpoint{4.628123in}{1.844164in}}%
\pgfpathlineto{\pgfqpoint{4.753123in}{1.844164in}}%
\pgfusepath{stroke}%
\end{pgfscope}%
\begin{pgfscope}%
\definecolor{textcolor}{rgb}{0.150000,0.150000,0.150000}%
\pgfsetstrokecolor{textcolor}%
\pgfsetfillcolor{textcolor}%
\pgftext[x=4.853123in,y=1.800414in,left,base]{\color{textcolor}\sffamily\fontsize{9.000000}{10.800000}\selectfont kNN search}%
\end{pgfscope}%
\begin{pgfscope}%
\pgfsetroundcap%
\pgfsetroundjoin%
\pgfsetlinewidth{1.505625pt}%
\definecolor{currentstroke}{rgb}{0.007843,0.619608,0.450980}%
\pgfsetstrokecolor{currentstroke}%
\pgfsetdash{}{0pt}%
\pgfpathmoveto{\pgfqpoint{4.503123in}{1.656664in}}%
\pgfpathlineto{\pgfqpoint{4.628123in}{1.656664in}}%
\pgfpathlineto{\pgfqpoint{4.753123in}{1.656664in}}%
\pgfusepath{stroke}%
\end{pgfscope}%
\begin{pgfscope}%
\definecolor{textcolor}{rgb}{0.150000,0.150000,0.150000}%
\pgfsetstrokecolor{textcolor}%
\pgfsetfillcolor{textcolor}%
\pgftext[x=4.853123in,y=1.612914in,left,base]{\color{textcolor}\sffamily\fontsize{9.000000}{10.800000}\selectfont Create graph}%
\end{pgfscope}%
\begin{pgfscope}%
\pgfsetroundcap%
\pgfsetroundjoin%
\pgfsetlinewidth{1.505625pt}%
\definecolor{currentstroke}{rgb}{0.835294,0.368627,0.000000}%
\pgfsetstrokecolor{currentstroke}%
\pgfsetdash{}{0pt}%
\pgfpathmoveto{\pgfqpoint{4.503123in}{1.382153in}}%
\pgfpathlineto{\pgfqpoint{4.628123in}{1.382153in}}%
\pgfpathlineto{\pgfqpoint{4.753123in}{1.382153in}}%
\pgfusepath{stroke}%
\end{pgfscope}%
\begin{pgfscope}%
\definecolor{textcolor}{rgb}{0.150000,0.150000,0.150000}%
\pgfsetstrokecolor{textcolor}%
\pgfsetfillcolor{textcolor}%
\pgftext[x=4.853123in, y=1.425415in, left, base]{\color{textcolor}\sffamily\fontsize{9.000000}{10.800000}\selectfont Get \& prepare}%
\end{pgfscope}%
\begin{pgfscope}%
\definecolor{textcolor}{rgb}{0.150000,0.150000,0.150000}%
\pgfsetstrokecolor{textcolor}%
\pgfsetfillcolor{textcolor}%
\pgftext[x=4.853123in, y=1.281421in, left, base]{\color{textcolor}\sffamily\fontsize{9.000000}{10.800000}\selectfont obstacles}%
\end{pgfscope}%
\begin{pgfscope}%
\pgfsetroundcap%
\pgfsetroundjoin%
\pgfsetlinewidth{1.505625pt}%
\definecolor{currentstroke}{rgb}{0.800000,0.470588,0.737255}%
\pgfsetstrokecolor{currentstroke}%
\pgfsetdash{}{0pt}%
\pgfpathmoveto{\pgfqpoint{4.503123in}{1.050659in}}%
\pgfpathlineto{\pgfqpoint{4.628123in}{1.050659in}}%
\pgfpathlineto{\pgfqpoint{4.753123in}{1.050659in}}%
\pgfusepath{stroke}%
\end{pgfscope}%
\begin{pgfscope}%
\definecolor{textcolor}{rgb}{0.150000,0.150000,0.150000}%
\pgfsetstrokecolor{textcolor}%
\pgfsetfillcolor{textcolor}%
\pgftext[x=4.853123in, y=1.093921in, left, base]{\color{textcolor}\sffamily\fontsize{9.000000}{10.800000}\selectfont Merge road}%
\end{pgfscope}%
\begin{pgfscope}%
\definecolor{textcolor}{rgb}{0.150000,0.150000,0.150000}%
\pgfsetstrokecolor{textcolor}%
\pgfsetfillcolor{textcolor}%
\pgftext[x=4.853123in, y=0.949927in, left, base]{\color{textcolor}\sffamily\fontsize{9.000000}{10.800000}\selectfont edges}%
\end{pgfscope}%
\begin{pgfscope}%
\pgfsetroundcap%
\pgfsetroundjoin%
\pgfsetlinewidth{1.505625pt}%
\definecolor{currentstroke}{rgb}{0.792157,0.568627,0.380392}%
\pgfsetstrokecolor{currentstroke}%
\pgfsetdash{}{0pt}%
\pgfpathmoveto{\pgfqpoint{4.503123in}{0.719165in}}%
\pgfpathlineto{\pgfqpoint{4.628123in}{0.719165in}}%
\pgfpathlineto{\pgfqpoint{4.753123in}{0.719165in}}%
\pgfusepath{stroke}%
\end{pgfscope}%
\begin{pgfscope}%
\definecolor{textcolor}{rgb}{0.150000,0.150000,0.150000}%
\pgfsetstrokecolor{textcolor}%
\pgfsetfillcolor{textcolor}%
\pgftext[x=4.853123in, y=0.762427in, left, base]{\color{textcolor}\sffamily\fontsize{9.000000}{10.800000}\selectfont Add POI}%
\end{pgfscope}%
\begin{pgfscope}%
\definecolor{textcolor}{rgb}{0.150000,0.150000,0.150000}%
\pgfsetstrokecolor{textcolor}%
\pgfsetfillcolor{textcolor}%
\pgftext[x=4.853123in, y=0.618433in, left, base]{\color{textcolor}\sffamily\fontsize{9.000000}{10.800000}\selectfont attributes}%
\end{pgfscope}%
\begin{pgfscope}%
\pgfsetroundcap%
\pgfsetroundjoin%
\pgfsetlinewidth{1.003750pt}%
\definecolor{currentstroke}{rgb}{0.003922,0.450980,0.698039}%
\pgfsetstrokecolor{currentstroke}%
\pgfsetdash{}{0pt}%
\pgfpathmoveto{\pgfqpoint{0.761386in}{1.754300in}}%
\pgfpathlineto{\pgfqpoint{1.273162in}{1.928456in}}%
\pgfpathlineto{\pgfqpoint{1.720455in}{2.048083in}}%
\pgfpathlineto{\pgfqpoint{2.156291in}{2.133620in}}%
\pgfpathlineto{\pgfqpoint{3.169985in}{2.250574in}}%
\pgfpathlineto{\pgfqpoint{4.129587in}{2.317319in}}%
\pgfusepath{stroke}%
\end{pgfscope}%
\begin{pgfscope}%
\pgfsetbuttcap%
\pgfsetroundjoin%
\definecolor{currentfill}{rgb}{0.003922,0.450980,0.698039}%
\pgfsetfillcolor{currentfill}%
\pgfsetlinewidth{0.752812pt}%
\definecolor{currentstroke}{rgb}{1.000000,1.000000,1.000000}%
\pgfsetstrokecolor{currentstroke}%
\pgfsetdash{}{0pt}%
\pgfsys@defobject{currentmarker}{\pgfqpoint{-0.034722in}{-0.034722in}}{\pgfqpoint{0.034722in}{0.034722in}}{%
\pgfpathmoveto{\pgfqpoint{0.000000in}{-0.034722in}}%
\pgfpathcurveto{\pgfqpoint{0.009208in}{-0.034722in}}{\pgfqpoint{0.018041in}{-0.031064in}}{\pgfqpoint{0.024552in}{-0.024552in}}%
\pgfpathcurveto{\pgfqpoint{0.031064in}{-0.018041in}}{\pgfqpoint{0.034722in}{-0.009208in}}{\pgfqpoint{0.034722in}{0.000000in}}%
\pgfpathcurveto{\pgfqpoint{0.034722in}{0.009208in}}{\pgfqpoint{0.031064in}{0.018041in}}{\pgfqpoint{0.024552in}{0.024552in}}%
\pgfpathcurveto{\pgfqpoint{0.018041in}{0.031064in}}{\pgfqpoint{0.009208in}{0.034722in}}{\pgfqpoint{0.000000in}{0.034722in}}%
\pgfpathcurveto{\pgfqpoint{-0.009208in}{0.034722in}}{\pgfqpoint{-0.018041in}{0.031064in}}{\pgfqpoint{-0.024552in}{0.024552in}}%
\pgfpathcurveto{\pgfqpoint{-0.031064in}{0.018041in}}{\pgfqpoint{-0.034722in}{0.009208in}}{\pgfqpoint{-0.034722in}{0.000000in}}%
\pgfpathcurveto{\pgfqpoint{-0.034722in}{-0.009208in}}{\pgfqpoint{-0.031064in}{-0.018041in}}{\pgfqpoint{-0.024552in}{-0.024552in}}%
\pgfpathcurveto{\pgfqpoint{-0.018041in}{-0.031064in}}{\pgfqpoint{-0.009208in}{-0.034722in}}{\pgfqpoint{0.000000in}{-0.034722in}}%
\pgfpathlineto{\pgfqpoint{0.000000in}{-0.034722in}}%
\pgfpathclose%
\pgfusepath{stroke,fill}%
}%
\begin{pgfscope}%
\pgfsys@transformshift{0.761386in}{1.754300in}%
\pgfsys@useobject{currentmarker}{}%
\end{pgfscope}%
\begin{pgfscope}%
\pgfsys@transformshift{1.273162in}{1.928456in}%
\pgfsys@useobject{currentmarker}{}%
\end{pgfscope}%
\begin{pgfscope}%
\pgfsys@transformshift{1.720455in}{2.048083in}%
\pgfsys@useobject{currentmarker}{}%
\end{pgfscope}%
\begin{pgfscope}%
\pgfsys@transformshift{2.156291in}{2.133620in}%
\pgfsys@useobject{currentmarker}{}%
\end{pgfscope}%
\begin{pgfscope}%
\pgfsys@transformshift{3.169985in}{2.250574in}%
\pgfsys@useobject{currentmarker}{}%
\end{pgfscope}%
\begin{pgfscope}%
\pgfsys@transformshift{4.129587in}{2.317319in}%
\pgfsys@useobject{currentmarker}{}%
\end{pgfscope}%
\end{pgfscope}%
\begin{pgfscope}%
\pgfsetroundcap%
\pgfsetroundjoin%
\pgfsetlinewidth{1.003750pt}%
\definecolor{currentstroke}{rgb}{0.870588,0.560784,0.019608}%
\pgfsetstrokecolor{currentstroke}%
\pgfsetdash{}{0pt}%
\pgfpathmoveto{\pgfqpoint{0.761386in}{1.695789in}}%
\pgfpathlineto{\pgfqpoint{1.273162in}{1.871511in}}%
\pgfpathlineto{\pgfqpoint{1.720455in}{1.986561in}}%
\pgfpathlineto{\pgfqpoint{2.156291in}{2.069337in}}%
\pgfpathlineto{\pgfqpoint{3.169985in}{2.188547in}}%
\pgfpathlineto{\pgfqpoint{4.129587in}{2.250417in}}%
\pgfusepath{stroke}%
\end{pgfscope}%
\begin{pgfscope}%
\pgfsetbuttcap%
\pgfsetroundjoin%
\definecolor{currentfill}{rgb}{0.870588,0.560784,0.019608}%
\pgfsetfillcolor{currentfill}%
\pgfsetlinewidth{0.752812pt}%
\definecolor{currentstroke}{rgb}{1.000000,1.000000,1.000000}%
\pgfsetstrokecolor{currentstroke}%
\pgfsetdash{}{0pt}%
\pgfsys@defobject{currentmarker}{\pgfqpoint{-0.034722in}{-0.034722in}}{\pgfqpoint{0.034722in}{0.034722in}}{%
\pgfpathmoveto{\pgfqpoint{0.000000in}{-0.034722in}}%
\pgfpathcurveto{\pgfqpoint{0.009208in}{-0.034722in}}{\pgfqpoint{0.018041in}{-0.031064in}}{\pgfqpoint{0.024552in}{-0.024552in}}%
\pgfpathcurveto{\pgfqpoint{0.031064in}{-0.018041in}}{\pgfqpoint{0.034722in}{-0.009208in}}{\pgfqpoint{0.034722in}{0.000000in}}%
\pgfpathcurveto{\pgfqpoint{0.034722in}{0.009208in}}{\pgfqpoint{0.031064in}{0.018041in}}{\pgfqpoint{0.024552in}{0.024552in}}%
\pgfpathcurveto{\pgfqpoint{0.018041in}{0.031064in}}{\pgfqpoint{0.009208in}{0.034722in}}{\pgfqpoint{0.000000in}{0.034722in}}%
\pgfpathcurveto{\pgfqpoint{-0.009208in}{0.034722in}}{\pgfqpoint{-0.018041in}{0.031064in}}{\pgfqpoint{-0.024552in}{0.024552in}}%
\pgfpathcurveto{\pgfqpoint{-0.031064in}{0.018041in}}{\pgfqpoint{-0.034722in}{0.009208in}}{\pgfqpoint{-0.034722in}{0.000000in}}%
\pgfpathcurveto{\pgfqpoint{-0.034722in}{-0.009208in}}{\pgfqpoint{-0.031064in}{-0.018041in}}{\pgfqpoint{-0.024552in}{-0.024552in}}%
\pgfpathcurveto{\pgfqpoint{-0.018041in}{-0.031064in}}{\pgfqpoint{-0.009208in}{-0.034722in}}{\pgfqpoint{0.000000in}{-0.034722in}}%
\pgfpathlineto{\pgfqpoint{0.000000in}{-0.034722in}}%
\pgfpathclose%
\pgfusepath{stroke,fill}%
}%
\begin{pgfscope}%
\pgfsys@transformshift{0.761386in}{1.695789in}%
\pgfsys@useobject{currentmarker}{}%
\end{pgfscope}%
\begin{pgfscope}%
\pgfsys@transformshift{1.273162in}{1.871511in}%
\pgfsys@useobject{currentmarker}{}%
\end{pgfscope}%
\begin{pgfscope}%
\pgfsys@transformshift{1.720455in}{1.986561in}%
\pgfsys@useobject{currentmarker}{}%
\end{pgfscope}%
\begin{pgfscope}%
\pgfsys@transformshift{2.156291in}{2.069337in}%
\pgfsys@useobject{currentmarker}{}%
\end{pgfscope}%
\begin{pgfscope}%
\pgfsys@transformshift{3.169985in}{2.188547in}%
\pgfsys@useobject{currentmarker}{}%
\end{pgfscope}%
\begin{pgfscope}%
\pgfsys@transformshift{4.129587in}{2.250417in}%
\pgfsys@useobject{currentmarker}{}%
\end{pgfscope}%
\end{pgfscope}%
\begin{pgfscope}%
\pgfsetroundcap%
\pgfsetroundjoin%
\pgfsetlinewidth{1.003750pt}%
\definecolor{currentstroke}{rgb}{0.007843,0.619608,0.450980}%
\pgfsetstrokecolor{currentstroke}%
\pgfsetdash{}{0pt}%
\pgfpathmoveto{\pgfqpoint{0.761386in}{0.968721in}}%
\pgfpathlineto{\pgfqpoint{1.273162in}{1.061540in}}%
\pgfpathlineto{\pgfqpoint{1.720455in}{1.111983in}}%
\pgfpathlineto{\pgfqpoint{2.156291in}{1.147880in}}%
\pgfpathlineto{\pgfqpoint{3.169985in}{1.208227in}}%
\pgfpathlineto{\pgfqpoint{4.129587in}{1.254891in}}%
\pgfusepath{stroke}%
\end{pgfscope}%
\begin{pgfscope}%
\pgfsetbuttcap%
\pgfsetroundjoin%
\definecolor{currentfill}{rgb}{0.007843,0.619608,0.450980}%
\pgfsetfillcolor{currentfill}%
\pgfsetlinewidth{0.752812pt}%
\definecolor{currentstroke}{rgb}{1.000000,1.000000,1.000000}%
\pgfsetstrokecolor{currentstroke}%
\pgfsetdash{}{0pt}%
\pgfsys@defobject{currentmarker}{\pgfqpoint{-0.034722in}{-0.034722in}}{\pgfqpoint{0.034722in}{0.034722in}}{%
\pgfpathmoveto{\pgfqpoint{0.000000in}{-0.034722in}}%
\pgfpathcurveto{\pgfqpoint{0.009208in}{-0.034722in}}{\pgfqpoint{0.018041in}{-0.031064in}}{\pgfqpoint{0.024552in}{-0.024552in}}%
\pgfpathcurveto{\pgfqpoint{0.031064in}{-0.018041in}}{\pgfqpoint{0.034722in}{-0.009208in}}{\pgfqpoint{0.034722in}{0.000000in}}%
\pgfpathcurveto{\pgfqpoint{0.034722in}{0.009208in}}{\pgfqpoint{0.031064in}{0.018041in}}{\pgfqpoint{0.024552in}{0.024552in}}%
\pgfpathcurveto{\pgfqpoint{0.018041in}{0.031064in}}{\pgfqpoint{0.009208in}{0.034722in}}{\pgfqpoint{0.000000in}{0.034722in}}%
\pgfpathcurveto{\pgfqpoint{-0.009208in}{0.034722in}}{\pgfqpoint{-0.018041in}{0.031064in}}{\pgfqpoint{-0.024552in}{0.024552in}}%
\pgfpathcurveto{\pgfqpoint{-0.031064in}{0.018041in}}{\pgfqpoint{-0.034722in}{0.009208in}}{\pgfqpoint{-0.034722in}{0.000000in}}%
\pgfpathcurveto{\pgfqpoint{-0.034722in}{-0.009208in}}{\pgfqpoint{-0.031064in}{-0.018041in}}{\pgfqpoint{-0.024552in}{-0.024552in}}%
\pgfpathcurveto{\pgfqpoint{-0.018041in}{-0.031064in}}{\pgfqpoint{-0.009208in}{-0.034722in}}{\pgfqpoint{0.000000in}{-0.034722in}}%
\pgfpathlineto{\pgfqpoint{0.000000in}{-0.034722in}}%
\pgfpathclose%
\pgfusepath{stroke,fill}%
}%
\begin{pgfscope}%
\pgfsys@transformshift{0.761386in}{0.968721in}%
\pgfsys@useobject{currentmarker}{}%
\end{pgfscope}%
\begin{pgfscope}%
\pgfsys@transformshift{1.273162in}{1.061540in}%
\pgfsys@useobject{currentmarker}{}%
\end{pgfscope}%
\begin{pgfscope}%
\pgfsys@transformshift{1.720455in}{1.111983in}%
\pgfsys@useobject{currentmarker}{}%
\end{pgfscope}%
\begin{pgfscope}%
\pgfsys@transformshift{2.156291in}{1.147880in}%
\pgfsys@useobject{currentmarker}{}%
\end{pgfscope}%
\begin{pgfscope}%
\pgfsys@transformshift{3.169985in}{1.208227in}%
\pgfsys@useobject{currentmarker}{}%
\end{pgfscope}%
\begin{pgfscope}%
\pgfsys@transformshift{4.129587in}{1.254891in}%
\pgfsys@useobject{currentmarker}{}%
\end{pgfscope}%
\end{pgfscope}%
\begin{pgfscope}%
\pgfsetroundcap%
\pgfsetroundjoin%
\pgfsetlinewidth{1.003750pt}%
\definecolor{currentstroke}{rgb}{0.835294,0.368627,0.000000}%
\pgfsetstrokecolor{currentstroke}%
\pgfsetdash{}{0pt}%
\pgfpathmoveto{\pgfqpoint{0.761386in}{0.863459in}}%
\pgfpathlineto{\pgfqpoint{1.273162in}{0.937231in}}%
\pgfpathlineto{\pgfqpoint{1.720455in}{0.989444in}}%
\pgfpathlineto{\pgfqpoint{2.156291in}{1.014798in}}%
\pgfpathlineto{\pgfqpoint{3.169985in}{1.072922in}}%
\pgfpathlineto{\pgfqpoint{4.129587in}{1.111729in}}%
\pgfusepath{stroke}%
\end{pgfscope}%
\begin{pgfscope}%
\pgfsetbuttcap%
\pgfsetroundjoin%
\definecolor{currentfill}{rgb}{0.835294,0.368627,0.000000}%
\pgfsetfillcolor{currentfill}%
\pgfsetlinewidth{0.752812pt}%
\definecolor{currentstroke}{rgb}{1.000000,1.000000,1.000000}%
\pgfsetstrokecolor{currentstroke}%
\pgfsetdash{}{0pt}%
\pgfsys@defobject{currentmarker}{\pgfqpoint{-0.034722in}{-0.034722in}}{\pgfqpoint{0.034722in}{0.034722in}}{%
\pgfpathmoveto{\pgfqpoint{0.000000in}{-0.034722in}}%
\pgfpathcurveto{\pgfqpoint{0.009208in}{-0.034722in}}{\pgfqpoint{0.018041in}{-0.031064in}}{\pgfqpoint{0.024552in}{-0.024552in}}%
\pgfpathcurveto{\pgfqpoint{0.031064in}{-0.018041in}}{\pgfqpoint{0.034722in}{-0.009208in}}{\pgfqpoint{0.034722in}{0.000000in}}%
\pgfpathcurveto{\pgfqpoint{0.034722in}{0.009208in}}{\pgfqpoint{0.031064in}{0.018041in}}{\pgfqpoint{0.024552in}{0.024552in}}%
\pgfpathcurveto{\pgfqpoint{0.018041in}{0.031064in}}{\pgfqpoint{0.009208in}{0.034722in}}{\pgfqpoint{0.000000in}{0.034722in}}%
\pgfpathcurveto{\pgfqpoint{-0.009208in}{0.034722in}}{\pgfqpoint{-0.018041in}{0.031064in}}{\pgfqpoint{-0.024552in}{0.024552in}}%
\pgfpathcurveto{\pgfqpoint{-0.031064in}{0.018041in}}{\pgfqpoint{-0.034722in}{0.009208in}}{\pgfqpoint{-0.034722in}{0.000000in}}%
\pgfpathcurveto{\pgfqpoint{-0.034722in}{-0.009208in}}{\pgfqpoint{-0.031064in}{-0.018041in}}{\pgfqpoint{-0.024552in}{-0.024552in}}%
\pgfpathcurveto{\pgfqpoint{-0.018041in}{-0.031064in}}{\pgfqpoint{-0.009208in}{-0.034722in}}{\pgfqpoint{0.000000in}{-0.034722in}}%
\pgfpathlineto{\pgfqpoint{0.000000in}{-0.034722in}}%
\pgfpathclose%
\pgfusepath{stroke,fill}%
}%
\begin{pgfscope}%
\pgfsys@transformshift{0.761386in}{0.863459in}%
\pgfsys@useobject{currentmarker}{}%
\end{pgfscope}%
\begin{pgfscope}%
\pgfsys@transformshift{1.273162in}{0.937231in}%
\pgfsys@useobject{currentmarker}{}%
\end{pgfscope}%
\begin{pgfscope}%
\pgfsys@transformshift{1.720455in}{0.989444in}%
\pgfsys@useobject{currentmarker}{}%
\end{pgfscope}%
\begin{pgfscope}%
\pgfsys@transformshift{2.156291in}{1.014798in}%
\pgfsys@useobject{currentmarker}{}%
\end{pgfscope}%
\begin{pgfscope}%
\pgfsys@transformshift{3.169985in}{1.072922in}%
\pgfsys@useobject{currentmarker}{}%
\end{pgfscope}%
\begin{pgfscope}%
\pgfsys@transformshift{4.129587in}{1.111729in}%
\pgfsys@useobject{currentmarker}{}%
\end{pgfscope}%
\end{pgfscope}%
\begin{pgfscope}%
\pgfsetroundcap%
\pgfsetroundjoin%
\pgfsetlinewidth{1.003750pt}%
\definecolor{currentstroke}{rgb}{0.800000,0.470588,0.737255}%
\pgfsetstrokecolor{currentstroke}%
\pgfsetdash{}{0pt}%
\pgfpathmoveto{\pgfqpoint{0.761386in}{1.600530in}}%
\pgfpathlineto{\pgfqpoint{1.273162in}{1.772979in}}%
\pgfpathlineto{\pgfqpoint{1.720455in}{1.902010in}}%
\pgfpathlineto{\pgfqpoint{2.156291in}{1.992756in}}%
\pgfpathlineto{\pgfqpoint{3.169985in}{2.105987in}}%
\pgfpathlineto{\pgfqpoint{4.129587in}{2.181174in}}%
\pgfusepath{stroke}%
\end{pgfscope}%
\begin{pgfscope}%
\pgfsetbuttcap%
\pgfsetroundjoin%
\definecolor{currentfill}{rgb}{0.800000,0.470588,0.737255}%
\pgfsetfillcolor{currentfill}%
\pgfsetlinewidth{0.752812pt}%
\definecolor{currentstroke}{rgb}{1.000000,1.000000,1.000000}%
\pgfsetstrokecolor{currentstroke}%
\pgfsetdash{}{0pt}%
\pgfsys@defobject{currentmarker}{\pgfqpoint{-0.034722in}{-0.034722in}}{\pgfqpoint{0.034722in}{0.034722in}}{%
\pgfpathmoveto{\pgfqpoint{0.000000in}{-0.034722in}}%
\pgfpathcurveto{\pgfqpoint{0.009208in}{-0.034722in}}{\pgfqpoint{0.018041in}{-0.031064in}}{\pgfqpoint{0.024552in}{-0.024552in}}%
\pgfpathcurveto{\pgfqpoint{0.031064in}{-0.018041in}}{\pgfqpoint{0.034722in}{-0.009208in}}{\pgfqpoint{0.034722in}{0.000000in}}%
\pgfpathcurveto{\pgfqpoint{0.034722in}{0.009208in}}{\pgfqpoint{0.031064in}{0.018041in}}{\pgfqpoint{0.024552in}{0.024552in}}%
\pgfpathcurveto{\pgfqpoint{0.018041in}{0.031064in}}{\pgfqpoint{0.009208in}{0.034722in}}{\pgfqpoint{0.000000in}{0.034722in}}%
\pgfpathcurveto{\pgfqpoint{-0.009208in}{0.034722in}}{\pgfqpoint{-0.018041in}{0.031064in}}{\pgfqpoint{-0.024552in}{0.024552in}}%
\pgfpathcurveto{\pgfqpoint{-0.031064in}{0.018041in}}{\pgfqpoint{-0.034722in}{0.009208in}}{\pgfqpoint{-0.034722in}{0.000000in}}%
\pgfpathcurveto{\pgfqpoint{-0.034722in}{-0.009208in}}{\pgfqpoint{-0.031064in}{-0.018041in}}{\pgfqpoint{-0.024552in}{-0.024552in}}%
\pgfpathcurveto{\pgfqpoint{-0.018041in}{-0.031064in}}{\pgfqpoint{-0.009208in}{-0.034722in}}{\pgfqpoint{0.000000in}{-0.034722in}}%
\pgfpathlineto{\pgfqpoint{0.000000in}{-0.034722in}}%
\pgfpathclose%
\pgfusepath{stroke,fill}%
}%
\begin{pgfscope}%
\pgfsys@transformshift{0.761386in}{1.600530in}%
\pgfsys@useobject{currentmarker}{}%
\end{pgfscope}%
\begin{pgfscope}%
\pgfsys@transformshift{1.273162in}{1.772979in}%
\pgfsys@useobject{currentmarker}{}%
\end{pgfscope}%
\begin{pgfscope}%
\pgfsys@transformshift{1.720455in}{1.902010in}%
\pgfsys@useobject{currentmarker}{}%
\end{pgfscope}%
\begin{pgfscope}%
\pgfsys@transformshift{2.156291in}{1.992756in}%
\pgfsys@useobject{currentmarker}{}%
\end{pgfscope}%
\begin{pgfscope}%
\pgfsys@transformshift{3.169985in}{2.105987in}%
\pgfsys@useobject{currentmarker}{}%
\end{pgfscope}%
\begin{pgfscope}%
\pgfsys@transformshift{4.129587in}{2.181174in}%
\pgfsys@useobject{currentmarker}{}%
\end{pgfscope}%
\end{pgfscope}%
\begin{pgfscope}%
\pgfsetroundcap%
\pgfsetroundjoin%
\pgfsetlinewidth{1.003750pt}%
\definecolor{currentstroke}{rgb}{0.792157,0.568627,0.380392}%
\pgfsetstrokecolor{currentstroke}%
\pgfsetdash{}{0pt}%
\pgfpathmoveto{\pgfqpoint{0.761386in}{0.575732in}}%
\pgfpathlineto{\pgfqpoint{1.273162in}{0.688864in}}%
\pgfpathlineto{\pgfqpoint{1.720455in}{0.706895in}}%
\pgfpathlineto{\pgfqpoint{2.156291in}{0.779042in}}%
\pgfpathlineto{\pgfqpoint{3.169985in}{0.789995in}}%
\pgfpathlineto{\pgfqpoint{4.129587in}{0.918804in}}%
\pgfusepath{stroke}%
\end{pgfscope}%
\begin{pgfscope}%
\pgfsetbuttcap%
\pgfsetroundjoin%
\definecolor{currentfill}{rgb}{0.792157,0.568627,0.380392}%
\pgfsetfillcolor{currentfill}%
\pgfsetlinewidth{0.752812pt}%
\definecolor{currentstroke}{rgb}{1.000000,1.000000,1.000000}%
\pgfsetstrokecolor{currentstroke}%
\pgfsetdash{}{0pt}%
\pgfsys@defobject{currentmarker}{\pgfqpoint{-0.034722in}{-0.034722in}}{\pgfqpoint{0.034722in}{0.034722in}}{%
\pgfpathmoveto{\pgfqpoint{0.000000in}{-0.034722in}}%
\pgfpathcurveto{\pgfqpoint{0.009208in}{-0.034722in}}{\pgfqpoint{0.018041in}{-0.031064in}}{\pgfqpoint{0.024552in}{-0.024552in}}%
\pgfpathcurveto{\pgfqpoint{0.031064in}{-0.018041in}}{\pgfqpoint{0.034722in}{-0.009208in}}{\pgfqpoint{0.034722in}{0.000000in}}%
\pgfpathcurveto{\pgfqpoint{0.034722in}{0.009208in}}{\pgfqpoint{0.031064in}{0.018041in}}{\pgfqpoint{0.024552in}{0.024552in}}%
\pgfpathcurveto{\pgfqpoint{0.018041in}{0.031064in}}{\pgfqpoint{0.009208in}{0.034722in}}{\pgfqpoint{0.000000in}{0.034722in}}%
\pgfpathcurveto{\pgfqpoint{-0.009208in}{0.034722in}}{\pgfqpoint{-0.018041in}{0.031064in}}{\pgfqpoint{-0.024552in}{0.024552in}}%
\pgfpathcurveto{\pgfqpoint{-0.031064in}{0.018041in}}{\pgfqpoint{-0.034722in}{0.009208in}}{\pgfqpoint{-0.034722in}{0.000000in}}%
\pgfpathcurveto{\pgfqpoint{-0.034722in}{-0.009208in}}{\pgfqpoint{-0.031064in}{-0.018041in}}{\pgfqpoint{-0.024552in}{-0.024552in}}%
\pgfpathcurveto{\pgfqpoint{-0.018041in}{-0.031064in}}{\pgfqpoint{-0.009208in}{-0.034722in}}{\pgfqpoint{0.000000in}{-0.034722in}}%
\pgfpathlineto{\pgfqpoint{0.000000in}{-0.034722in}}%
\pgfpathclose%
\pgfusepath{stroke,fill}%
}%
\begin{pgfscope}%
\pgfsys@transformshift{0.761386in}{0.575732in}%
\pgfsys@useobject{currentmarker}{}%
\end{pgfscope}%
\begin{pgfscope}%
\pgfsys@transformshift{1.273162in}{0.688864in}%
\pgfsys@useobject{currentmarker}{}%
\end{pgfscope}%
\begin{pgfscope}%
\pgfsys@transformshift{1.720455in}{0.706895in}%
\pgfsys@useobject{currentmarker}{}%
\end{pgfscope}%
\begin{pgfscope}%
\pgfsys@transformshift{2.156291in}{0.779042in}%
\pgfsys@useobject{currentmarker}{}%
\end{pgfscope}%
\begin{pgfscope}%
\pgfsys@transformshift{3.169985in}{0.789995in}%
\pgfsys@useobject{currentmarker}{}%
\end{pgfscope}%
\begin{pgfscope}%
\pgfsys@transformshift{4.129587in}{0.918804in}%
\pgfsys@useobject{currentmarker}{}%
\end{pgfscope}%
\end{pgfscope}%
\end{pgfpicture}%
\makeatother%
\endgroup%

					\end{figcenter}
					\caption{Import time of the \enquote{OSM city} dataset by tasks.}
				\end{subfigure}
%				\hfill
%				\begin{subfigure}[t]{.48\textwidth}
%					\begin{figcenter}
%						%% Creator: Matplotlib, PGF backend
%%
%% To include the figure in your LaTeX document, write
%%   \input{<filename>.pgf}
%%
%% Make sure the required packages are loaded in your preamble
%%   \usepackage{pgf}
%%
%% Also ensure that all the required font packages are loaded; for instance,
%% the lmodern package is sometimes necessary when using math font.
%%   \usepackage{lmodern}
%%
%% Figures using additional raster images can only be included by \input if
%% they are in the same directory as the main LaTeX file. For loading figures
%% from other directories you can use the `import` package
%%   \usepackage{import}
%%
%% and then include the figures with
%%   \import{<path to file>}{<filename>.pgf}
%%
%% Matplotlib used the following preamble
%%   
%%   \usepackage{fontspec}
%%   \setmainfont{DejaVuSerif.ttf}[Path=\detokenize{/home/hauke/.local/lib/python3.11/site-packages/matplotlib/mpl-data/fonts/ttf/}]
%%   \setsansfont{DroidSans.ttf}[Path=\detokenize{/usr/share/fonts/droid/}]
%%   \setmonofont{DejaVuSansMono.ttf}[Path=\detokenize{/home/hauke/.local/lib/python3.11/site-packages/matplotlib/mpl-data/fonts/ttf/}]
%%   \makeatletter\@ifpackageloaded{underscore}{}{\usepackage[strings]{underscore}}\makeatother
%%
\begingroup%
\makeatletter%
\begin{pgfpicture}%
\pgfpathrectangle{\pgfpointorigin}{\pgfqpoint{5.675893in}{2.407400in}}%
\pgfusepath{use as bounding box, clip}%
\begin{pgfscope}%
\pgfsetbuttcap%
\pgfsetmiterjoin%
\definecolor{currentfill}{rgb}{1.000000,1.000000,1.000000}%
\pgfsetfillcolor{currentfill}%
\pgfsetlinewidth{0.000000pt}%
\definecolor{currentstroke}{rgb}{1.000000,1.000000,1.000000}%
\pgfsetstrokecolor{currentstroke}%
\pgfsetdash{}{0pt}%
\pgfpathmoveto{\pgfqpoint{0.000000in}{0.000000in}}%
\pgfpathlineto{\pgfqpoint{5.675893in}{0.000000in}}%
\pgfpathlineto{\pgfqpoint{5.675893in}{2.407400in}}%
\pgfpathlineto{\pgfqpoint{0.000000in}{2.407400in}}%
\pgfpathlineto{\pgfqpoint{0.000000in}{0.000000in}}%
\pgfpathclose%
\pgfusepath{fill}%
\end{pgfscope}%
\begin{pgfscope}%
\pgfsetbuttcap%
\pgfsetmiterjoin%
\definecolor{currentfill}{rgb}{1.000000,1.000000,1.000000}%
\pgfsetfillcolor{currentfill}%
\pgfsetlinewidth{0.000000pt}%
\definecolor{currentstroke}{rgb}{0.000000,0.000000,0.000000}%
\pgfsetstrokecolor{currentstroke}%
\pgfsetstrokeopacity{0.000000}%
\pgfsetdash{}{0pt}%
\pgfpathmoveto{\pgfqpoint{0.497592in}{0.451389in}}%
\pgfpathlineto{\pgfqpoint{4.274417in}{0.451389in}}%
\pgfpathlineto{\pgfqpoint{4.274417in}{2.407400in}}%
\pgfpathlineto{\pgfqpoint{0.497592in}{2.407400in}}%
\pgfpathlineto{\pgfqpoint{0.497592in}{0.451389in}}%
\pgfpathclose%
\pgfusepath{fill}%
\end{pgfscope}%
\begin{pgfscope}%
\pgfpathrectangle{\pgfqpoint{0.497592in}{0.451389in}}{\pgfqpoint{3.776824in}{1.956011in}}%
\pgfusepath{clip}%
\pgfsetroundcap%
\pgfsetroundjoin%
\pgfsetlinewidth{1.003750pt}%
\definecolor{currentstroke}{rgb}{0.800000,0.800000,0.800000}%
\pgfsetstrokecolor{currentstroke}%
\pgfsetdash{}{0pt}%
\pgfpathmoveto{\pgfqpoint{0.497592in}{0.451389in}}%
\pgfpathlineto{\pgfqpoint{0.497592in}{2.407400in}}%
\pgfusepath{stroke}%
\end{pgfscope}%
\begin{pgfscope}%
\definecolor{textcolor}{rgb}{0.150000,0.150000,0.150000}%
\pgfsetstrokecolor{textcolor}%
\pgfsetfillcolor{textcolor}%
\pgftext[x=0.497592in,y=0.319444in,,top]{\color{textcolor}\sffamily\fontsize{9.000000}{10.800000}\selectfont 0}%
\end{pgfscope}%
\begin{pgfscope}%
\pgfpathrectangle{\pgfqpoint{0.497592in}{0.451389in}}{\pgfqpoint{3.776824in}{1.956011in}}%
\pgfusepath{clip}%
\pgfsetroundcap%
\pgfsetroundjoin%
\pgfsetlinewidth{1.003750pt}%
\definecolor{currentstroke}{rgb}{0.800000,0.800000,0.800000}%
\pgfsetstrokecolor{currentstroke}%
\pgfsetdash{}{0pt}%
\pgfpathmoveto{\pgfqpoint{1.308108in}{0.451389in}}%
\pgfpathlineto{\pgfqpoint{1.308108in}{2.407400in}}%
\pgfusepath{stroke}%
\end{pgfscope}%
\begin{pgfscope}%
\definecolor{textcolor}{rgb}{0.150000,0.150000,0.150000}%
\pgfsetstrokecolor{textcolor}%
\pgfsetfillcolor{textcolor}%
\pgftext[x=1.308108in,y=0.319444in,,top]{\color{textcolor}\sffamily\fontsize{9.000000}{10.800000}\selectfont 10000}%
\end{pgfscope}%
\begin{pgfscope}%
\pgfpathrectangle{\pgfqpoint{0.497592in}{0.451389in}}{\pgfqpoint{3.776824in}{1.956011in}}%
\pgfusepath{clip}%
\pgfsetroundcap%
\pgfsetroundjoin%
\pgfsetlinewidth{1.003750pt}%
\definecolor{currentstroke}{rgb}{0.800000,0.800000,0.800000}%
\pgfsetstrokecolor{currentstroke}%
\pgfsetdash{}{0pt}%
\pgfpathmoveto{\pgfqpoint{2.118623in}{0.451389in}}%
\pgfpathlineto{\pgfqpoint{2.118623in}{2.407400in}}%
\pgfusepath{stroke}%
\end{pgfscope}%
\begin{pgfscope}%
\definecolor{textcolor}{rgb}{0.150000,0.150000,0.150000}%
\pgfsetstrokecolor{textcolor}%
\pgfsetfillcolor{textcolor}%
\pgftext[x=2.118623in,y=0.319444in,,top]{\color{textcolor}\sffamily\fontsize{9.000000}{10.800000}\selectfont 20000}%
\end{pgfscope}%
\begin{pgfscope}%
\pgfpathrectangle{\pgfqpoint{0.497592in}{0.451389in}}{\pgfqpoint{3.776824in}{1.956011in}}%
\pgfusepath{clip}%
\pgfsetroundcap%
\pgfsetroundjoin%
\pgfsetlinewidth{1.003750pt}%
\definecolor{currentstroke}{rgb}{0.800000,0.800000,0.800000}%
\pgfsetstrokecolor{currentstroke}%
\pgfsetdash{}{0pt}%
\pgfpathmoveto{\pgfqpoint{2.929139in}{0.451389in}}%
\pgfpathlineto{\pgfqpoint{2.929139in}{2.407400in}}%
\pgfusepath{stroke}%
\end{pgfscope}%
\begin{pgfscope}%
\definecolor{textcolor}{rgb}{0.150000,0.150000,0.150000}%
\pgfsetstrokecolor{textcolor}%
\pgfsetfillcolor{textcolor}%
\pgftext[x=2.929139in,y=0.319444in,,top]{\color{textcolor}\sffamily\fontsize{9.000000}{10.800000}\selectfont 30000}%
\end{pgfscope}%
\begin{pgfscope}%
\pgfpathrectangle{\pgfqpoint{0.497592in}{0.451389in}}{\pgfqpoint{3.776824in}{1.956011in}}%
\pgfusepath{clip}%
\pgfsetroundcap%
\pgfsetroundjoin%
\pgfsetlinewidth{1.003750pt}%
\definecolor{currentstroke}{rgb}{0.800000,0.800000,0.800000}%
\pgfsetstrokecolor{currentstroke}%
\pgfsetdash{}{0pt}%
\pgfpathmoveto{\pgfqpoint{3.739655in}{0.451389in}}%
\pgfpathlineto{\pgfqpoint{3.739655in}{2.407400in}}%
\pgfusepath{stroke}%
\end{pgfscope}%
\begin{pgfscope}%
\definecolor{textcolor}{rgb}{0.150000,0.150000,0.150000}%
\pgfsetstrokecolor{textcolor}%
\pgfsetfillcolor{textcolor}%
\pgftext[x=3.739655in,y=0.319444in,,top]{\color{textcolor}\sffamily\fontsize{9.000000}{10.800000}\selectfont 40000}%
\end{pgfscope}%
\begin{pgfscope}%
\definecolor{textcolor}{rgb}{0.150000,0.150000,0.150000}%
\pgfsetstrokecolor{textcolor}%
\pgfsetfillcolor{textcolor}%
\pgftext[x=2.386004in,y=0.125000in,,top]{\color{textcolor}\sffamily\fontsize{9.000000}{10.800000}\selectfont Input obstacle vertices}%
\end{pgfscope}%
\begin{pgfscope}%
\pgfpathrectangle{\pgfqpoint{0.497592in}{0.451389in}}{\pgfqpoint{3.776824in}{1.956011in}}%
\pgfusepath{clip}%
\pgfsetroundcap%
\pgfsetroundjoin%
\pgfsetlinewidth{1.003750pt}%
\definecolor{currentstroke}{rgb}{0.800000,0.800000,0.800000}%
\pgfsetstrokecolor{currentstroke}%
\pgfsetdash{}{0pt}%
\pgfpathmoveto{\pgfqpoint{0.497592in}{0.451389in}}%
\pgfpathlineto{\pgfqpoint{4.274417in}{0.451389in}}%
\pgfusepath{stroke}%
\end{pgfscope}%
\begin{pgfscope}%
\definecolor{textcolor}{rgb}{0.150000,0.150000,0.150000}%
\pgfsetstrokecolor{textcolor}%
\pgfsetfillcolor{textcolor}%
\pgftext[x=0.194444in, y=0.403903in, left, base]{\color{textcolor}\sffamily\fontsize{9.000000}{10.800000}\selectfont 0.0}%
\end{pgfscope}%
\begin{pgfscope}%
\pgfpathrectangle{\pgfqpoint{0.497592in}{0.451389in}}{\pgfqpoint{3.776824in}{1.956011in}}%
\pgfusepath{clip}%
\pgfsetroundcap%
\pgfsetroundjoin%
\pgfsetlinewidth{1.003750pt}%
\definecolor{currentstroke}{rgb}{0.800000,0.800000,0.800000}%
\pgfsetstrokecolor{currentstroke}%
\pgfsetdash{}{0pt}%
\pgfpathmoveto{\pgfqpoint{0.497592in}{0.823962in}}%
\pgfpathlineto{\pgfqpoint{4.274417in}{0.823962in}}%
\pgfusepath{stroke}%
\end{pgfscope}%
\begin{pgfscope}%
\definecolor{textcolor}{rgb}{0.150000,0.150000,0.150000}%
\pgfsetstrokecolor{textcolor}%
\pgfsetfillcolor{textcolor}%
\pgftext[x=0.194444in, y=0.776477in, left, base]{\color{textcolor}\sffamily\fontsize{9.000000}{10.800000}\selectfont 0.2}%
\end{pgfscope}%
\begin{pgfscope}%
\pgfpathrectangle{\pgfqpoint{0.497592in}{0.451389in}}{\pgfqpoint{3.776824in}{1.956011in}}%
\pgfusepath{clip}%
\pgfsetroundcap%
\pgfsetroundjoin%
\pgfsetlinewidth{1.003750pt}%
\definecolor{currentstroke}{rgb}{0.800000,0.800000,0.800000}%
\pgfsetstrokecolor{currentstroke}%
\pgfsetdash{}{0pt}%
\pgfpathmoveto{\pgfqpoint{0.497592in}{1.196536in}}%
\pgfpathlineto{\pgfqpoint{4.274417in}{1.196536in}}%
\pgfusepath{stroke}%
\end{pgfscope}%
\begin{pgfscope}%
\definecolor{textcolor}{rgb}{0.150000,0.150000,0.150000}%
\pgfsetstrokecolor{textcolor}%
\pgfsetfillcolor{textcolor}%
\pgftext[x=0.194444in, y=1.149050in, left, base]{\color{textcolor}\sffamily\fontsize{9.000000}{10.800000}\selectfont 0.4}%
\end{pgfscope}%
\begin{pgfscope}%
\pgfpathrectangle{\pgfqpoint{0.497592in}{0.451389in}}{\pgfqpoint{3.776824in}{1.956011in}}%
\pgfusepath{clip}%
\pgfsetroundcap%
\pgfsetroundjoin%
\pgfsetlinewidth{1.003750pt}%
\definecolor{currentstroke}{rgb}{0.800000,0.800000,0.800000}%
\pgfsetstrokecolor{currentstroke}%
\pgfsetdash{}{0pt}%
\pgfpathmoveto{\pgfqpoint{0.497592in}{1.569109in}}%
\pgfpathlineto{\pgfqpoint{4.274417in}{1.569109in}}%
\pgfusepath{stroke}%
\end{pgfscope}%
\begin{pgfscope}%
\definecolor{textcolor}{rgb}{0.150000,0.150000,0.150000}%
\pgfsetstrokecolor{textcolor}%
\pgfsetfillcolor{textcolor}%
\pgftext[x=0.194444in, y=1.521624in, left, base]{\color{textcolor}\sffamily\fontsize{9.000000}{10.800000}\selectfont 0.6}%
\end{pgfscope}%
\begin{pgfscope}%
\pgfpathrectangle{\pgfqpoint{0.497592in}{0.451389in}}{\pgfqpoint{3.776824in}{1.956011in}}%
\pgfusepath{clip}%
\pgfsetroundcap%
\pgfsetroundjoin%
\pgfsetlinewidth{1.003750pt}%
\definecolor{currentstroke}{rgb}{0.800000,0.800000,0.800000}%
\pgfsetstrokecolor{currentstroke}%
\pgfsetdash{}{0pt}%
\pgfpathmoveto{\pgfqpoint{0.497592in}{1.941683in}}%
\pgfpathlineto{\pgfqpoint{4.274417in}{1.941683in}}%
\pgfusepath{stroke}%
\end{pgfscope}%
\begin{pgfscope}%
\definecolor{textcolor}{rgb}{0.150000,0.150000,0.150000}%
\pgfsetstrokecolor{textcolor}%
\pgfsetfillcolor{textcolor}%
\pgftext[x=0.194444in, y=1.894197in, left, base]{\color{textcolor}\sffamily\fontsize{9.000000}{10.800000}\selectfont 0.8}%
\end{pgfscope}%
\begin{pgfscope}%
\pgfpathrectangle{\pgfqpoint{0.497592in}{0.451389in}}{\pgfqpoint{3.776824in}{1.956011in}}%
\pgfusepath{clip}%
\pgfsetroundcap%
\pgfsetroundjoin%
\pgfsetlinewidth{1.003750pt}%
\definecolor{currentstroke}{rgb}{0.800000,0.800000,0.800000}%
\pgfsetstrokecolor{currentstroke}%
\pgfsetdash{}{0pt}%
\pgfpathmoveto{\pgfqpoint{0.497592in}{2.314256in}}%
\pgfpathlineto{\pgfqpoint{4.274417in}{2.314256in}}%
\pgfusepath{stroke}%
\end{pgfscope}%
\begin{pgfscope}%
\definecolor{textcolor}{rgb}{0.150000,0.150000,0.150000}%
\pgfsetstrokecolor{textcolor}%
\pgfsetfillcolor{textcolor}%
\pgftext[x=0.194444in, y=2.266771in, left, base]{\color{textcolor}\sffamily\fontsize{9.000000}{10.800000}\selectfont 1.0}%
\end{pgfscope}%
\begin{pgfscope}%
\definecolor{textcolor}{rgb}{0.150000,0.150000,0.150000}%
\pgfsetstrokecolor{textcolor}%
\pgfsetfillcolor{textcolor}%
\pgftext[x=0.125000in,y=1.429394in,,bottom,rotate=90.000000]{\color{textcolor}\sffamily\fontsize{9.000000}{10.800000}\selectfont Share of total time}%
\end{pgfscope}%
\begin{pgfscope}%
\pgfpathrectangle{\pgfqpoint{0.497592in}{0.451389in}}{\pgfqpoint{3.776824in}{1.956011in}}%
\pgfusepath{clip}%
\pgfsetbuttcap%
\pgfsetroundjoin%
\definecolor{currentfill}{rgb}{0.003922,0.450980,0.698039}%
\pgfsetfillcolor{currentfill}%
\pgfsetfillopacity{0.200000}%
\pgfsetlinewidth{1.003750pt}%
\definecolor{currentstroke}{rgb}{0.003922,0.450980,0.698039}%
\pgfsetstrokecolor{currentstroke}%
\pgfsetstrokeopacity{0.200000}%
\pgfsetdash{}{0pt}%
\pgfsys@defobject{currentmarker}{\pgfqpoint{0.533579in}{2.314256in}}{\pgfqpoint{4.096281in}{2.314256in}}{%
\pgfpathmoveto{\pgfqpoint{0.533579in}{2.314256in}}%
\pgfpathlineto{\pgfqpoint{0.533579in}{2.314256in}}%
\pgfpathlineto{\pgfqpoint{0.641540in}{2.314256in}}%
\pgfpathlineto{\pgfqpoint{0.821474in}{2.314256in}}%
\pgfpathlineto{\pgfqpoint{1.073383in}{2.314256in}}%
\pgfpathlineto{\pgfqpoint{1.397265in}{2.314256in}}%
\pgfpathlineto{\pgfqpoint{1.793120in}{2.314256in}}%
\pgfpathlineto{\pgfqpoint{2.260950in}{2.314256in}}%
\pgfpathlineto{\pgfqpoint{2.800753in}{2.314256in}}%
\pgfpathlineto{\pgfqpoint{3.412530in}{2.314256in}}%
\pgfpathlineto{\pgfqpoint{4.096281in}{2.314256in}}%
\pgfpathlineto{\pgfqpoint{4.096281in}{2.314256in}}%
\pgfpathlineto{\pgfqpoint{4.096281in}{2.314256in}}%
\pgfpathlineto{\pgfqpoint{3.412530in}{2.314256in}}%
\pgfpathlineto{\pgfqpoint{2.800753in}{2.314256in}}%
\pgfpathlineto{\pgfqpoint{2.260950in}{2.314256in}}%
\pgfpathlineto{\pgfqpoint{1.793120in}{2.314256in}}%
\pgfpathlineto{\pgfqpoint{1.397265in}{2.314256in}}%
\pgfpathlineto{\pgfqpoint{1.073383in}{2.314256in}}%
\pgfpathlineto{\pgfqpoint{0.821474in}{2.314256in}}%
\pgfpathlineto{\pgfqpoint{0.641540in}{2.314256in}}%
\pgfpathlineto{\pgfqpoint{0.533579in}{2.314256in}}%
\pgfpathlineto{\pgfqpoint{0.533579in}{2.314256in}}%
\pgfpathclose%
\pgfusepath{stroke,fill}%
}%
\begin{pgfscope}%
\pgfsys@transformshift{0.000000in}{0.000000in}%
\pgfsys@useobject{currentmarker}{}%
\end{pgfscope}%
\end{pgfscope}%
\begin{pgfscope}%
\pgfpathrectangle{\pgfqpoint{0.497592in}{0.451389in}}{\pgfqpoint{3.776824in}{1.956011in}}%
\pgfusepath{clip}%
\pgfsetbuttcap%
\pgfsetroundjoin%
\definecolor{currentfill}{rgb}{0.870588,0.560784,0.019608}%
\pgfsetfillcolor{currentfill}%
\pgfsetfillopacity{0.200000}%
\pgfsetlinewidth{1.003750pt}%
\definecolor{currentstroke}{rgb}{0.870588,0.560784,0.019608}%
\pgfsetstrokecolor{currentstroke}%
\pgfsetstrokeopacity{0.200000}%
\pgfsetdash{}{0pt}%
\pgfsys@defobject{currentmarker}{\pgfqpoint{0.533579in}{2.056794in}}{\pgfqpoint{4.096281in}{2.311127in}}{%
\pgfpathmoveto{\pgfqpoint{0.533579in}{2.080290in}}%
\pgfpathlineto{\pgfqpoint{0.533579in}{2.056794in}}%
\pgfpathlineto{\pgfqpoint{0.641540in}{2.226562in}}%
\pgfpathlineto{\pgfqpoint{0.821474in}{2.267800in}}%
\pgfpathlineto{\pgfqpoint{1.073383in}{2.286772in}}%
\pgfpathlineto{\pgfqpoint{1.397265in}{2.296163in}}%
\pgfpathlineto{\pgfqpoint{1.793120in}{2.301990in}}%
\pgfpathlineto{\pgfqpoint{2.260950in}{2.306006in}}%
\pgfpathlineto{\pgfqpoint{2.800753in}{2.309213in}}%
\pgfpathlineto{\pgfqpoint{3.412530in}{2.310233in}}%
\pgfpathlineto{\pgfqpoint{4.096281in}{2.311071in}}%
\pgfpathlineto{\pgfqpoint{4.096281in}{2.311127in}}%
\pgfpathlineto{\pgfqpoint{4.096281in}{2.311127in}}%
\pgfpathlineto{\pgfqpoint{3.412530in}{2.310300in}}%
\pgfpathlineto{\pgfqpoint{2.800753in}{2.309282in}}%
\pgfpathlineto{\pgfqpoint{2.260950in}{2.306195in}}%
\pgfpathlineto{\pgfqpoint{1.793120in}{2.303049in}}%
\pgfpathlineto{\pgfqpoint{1.397265in}{2.298060in}}%
\pgfpathlineto{\pgfqpoint{1.073383in}{2.288914in}}%
\pgfpathlineto{\pgfqpoint{0.821474in}{2.271300in}}%
\pgfpathlineto{\pgfqpoint{0.641540in}{2.230016in}}%
\pgfpathlineto{\pgfqpoint{0.533579in}{2.080290in}}%
\pgfpathlineto{\pgfqpoint{0.533579in}{2.080290in}}%
\pgfpathclose%
\pgfusepath{stroke,fill}%
}%
\begin{pgfscope}%
\pgfsys@transformshift{0.000000in}{0.000000in}%
\pgfsys@useobject{currentmarker}{}%
\end{pgfscope}%
\end{pgfscope}%
\begin{pgfscope}%
\pgfpathrectangle{\pgfqpoint{0.497592in}{0.451389in}}{\pgfqpoint{3.776824in}{1.956011in}}%
\pgfusepath{clip}%
\pgfsetbuttcap%
\pgfsetroundjoin%
\definecolor{currentfill}{rgb}{0.007843,0.619608,0.450980}%
\pgfsetfillcolor{currentfill}%
\pgfsetfillopacity{0.200000}%
\pgfsetlinewidth{1.003750pt}%
\definecolor{currentstroke}{rgb}{0.007843,0.619608,0.450980}%
\pgfsetstrokecolor{currentstroke}%
\pgfsetstrokeopacity{0.200000}%
\pgfsetdash{}{0pt}%
\pgfsys@defobject{currentmarker}{\pgfqpoint{0.533579in}{0.452692in}}{\pgfqpoint{4.096281in}{0.505645in}}{%
\pgfpathmoveto{\pgfqpoint{0.533579in}{0.505645in}}%
\pgfpathlineto{\pgfqpoint{0.533579in}{0.502415in}}%
\pgfpathlineto{\pgfqpoint{0.641540in}{0.483825in}}%
\pgfpathlineto{\pgfqpoint{0.821474in}{0.467869in}}%
\pgfpathlineto{\pgfqpoint{1.073383in}{0.461671in}}%
\pgfpathlineto{\pgfqpoint{1.397265in}{0.458849in}}%
\pgfpathlineto{\pgfqpoint{1.793120in}{0.455833in}}%
\pgfpathlineto{\pgfqpoint{2.260950in}{0.455117in}}%
\pgfpathlineto{\pgfqpoint{2.800753in}{0.453433in}}%
\pgfpathlineto{\pgfqpoint{3.412530in}{0.453025in}}%
\pgfpathlineto{\pgfqpoint{4.096281in}{0.452692in}}%
\pgfpathlineto{\pgfqpoint{4.096281in}{0.452767in}}%
\pgfpathlineto{\pgfqpoint{4.096281in}{0.452767in}}%
\pgfpathlineto{\pgfqpoint{3.412530in}{0.453084in}}%
\pgfpathlineto{\pgfqpoint{2.800753in}{0.453473in}}%
\pgfpathlineto{\pgfqpoint{2.260950in}{0.455233in}}%
\pgfpathlineto{\pgfqpoint{1.793120in}{0.456984in}}%
\pgfpathlineto{\pgfqpoint{1.397265in}{0.460478in}}%
\pgfpathlineto{\pgfqpoint{1.073383in}{0.464623in}}%
\pgfpathlineto{\pgfqpoint{0.821474in}{0.468499in}}%
\pgfpathlineto{\pgfqpoint{0.641540in}{0.484324in}}%
\pgfpathlineto{\pgfqpoint{0.533579in}{0.505645in}}%
\pgfpathlineto{\pgfqpoint{0.533579in}{0.505645in}}%
\pgfpathclose%
\pgfusepath{stroke,fill}%
}%
\begin{pgfscope}%
\pgfsys@transformshift{0.000000in}{0.000000in}%
\pgfsys@useobject{currentmarker}{}%
\end{pgfscope}%
\end{pgfscope}%
\begin{pgfscope}%
\pgfpathrectangle{\pgfqpoint{0.497592in}{0.451389in}}{\pgfqpoint{3.776824in}{1.956011in}}%
\pgfusepath{clip}%
\pgfsetbuttcap%
\pgfsetroundjoin%
\definecolor{currentfill}{rgb}{0.835294,0.368627,0.000000}%
\pgfsetfillcolor{currentfill}%
\pgfsetfillopacity{0.200000}%
\pgfsetlinewidth{1.003750pt}%
\definecolor{currentstroke}{rgb}{0.835294,0.368627,0.000000}%
\pgfsetstrokecolor{currentstroke}%
\pgfsetstrokeopacity{0.200000}%
\pgfsetdash{}{0pt}%
\pgfsys@defobject{currentmarker}{\pgfqpoint{0.533579in}{0.453086in}}{\pgfqpoint{4.096281in}{0.625034in}}{%
\pgfpathmoveto{\pgfqpoint{0.533579in}{0.625034in}}%
\pgfpathlineto{\pgfqpoint{0.533579in}{0.617535in}}%
\pgfpathlineto{\pgfqpoint{0.641540in}{0.500452in}}%
\pgfpathlineto{\pgfqpoint{0.821474in}{0.475598in}}%
\pgfpathlineto{\pgfqpoint{1.073383in}{0.464589in}}%
\pgfpathlineto{\pgfqpoint{1.397265in}{0.459409in}}%
\pgfpathlineto{\pgfqpoint{1.793120in}{0.457627in}}%
\pgfpathlineto{\pgfqpoint{2.260950in}{0.455455in}}%
\pgfpathlineto{\pgfqpoint{2.800753in}{0.454131in}}%
\pgfpathlineto{\pgfqpoint{3.412530in}{0.453539in}}%
\pgfpathlineto{\pgfqpoint{4.096281in}{0.453086in}}%
\pgfpathlineto{\pgfqpoint{4.096281in}{0.453125in}}%
\pgfpathlineto{\pgfqpoint{4.096281in}{0.453125in}}%
\pgfpathlineto{\pgfqpoint{3.412530in}{0.453605in}}%
\pgfpathlineto{\pgfqpoint{2.800753in}{0.454171in}}%
\pgfpathlineto{\pgfqpoint{2.260950in}{0.455513in}}%
\pgfpathlineto{\pgfqpoint{1.793120in}{0.457777in}}%
\pgfpathlineto{\pgfqpoint{1.397265in}{0.459707in}}%
\pgfpathlineto{\pgfqpoint{1.073383in}{0.465563in}}%
\pgfpathlineto{\pgfqpoint{0.821474in}{0.478979in}}%
\pgfpathlineto{\pgfqpoint{0.641540in}{0.501257in}}%
\pgfpathlineto{\pgfqpoint{0.533579in}{0.625034in}}%
\pgfpathlineto{\pgfqpoint{0.533579in}{0.625034in}}%
\pgfpathclose%
\pgfusepath{stroke,fill}%
}%
\begin{pgfscope}%
\pgfsys@transformshift{0.000000in}{0.000000in}%
\pgfsys@useobject{currentmarker}{}%
\end{pgfscope}%
\end{pgfscope}%
\begin{pgfscope}%
\pgfpathrectangle{\pgfqpoint{0.497592in}{0.451389in}}{\pgfqpoint{3.776824in}{1.956011in}}%
\pgfusepath{clip}%
\pgfsetbuttcap%
\pgfsetroundjoin%
\definecolor{currentfill}{rgb}{0.800000,0.470588,0.737255}%
\pgfsetfillcolor{currentfill}%
\pgfsetfillopacity{0.200000}%
\pgfsetlinewidth{1.003750pt}%
\definecolor{currentstroke}{rgb}{0.800000,0.470588,0.737255}%
\pgfsetstrokecolor{currentstroke}%
\pgfsetstrokeopacity{0.200000}%
\pgfsetdash{}{0pt}%
\pgfsys@defobject{currentmarker}{\pgfqpoint{0.533579in}{0.451389in}}{\pgfqpoint{4.096281in}{0.451643in}}{%
\pgfpathmoveto{\pgfqpoint{0.533579in}{0.451643in}}%
\pgfpathlineto{\pgfqpoint{0.533579in}{0.451521in}}%
\pgfpathlineto{\pgfqpoint{0.641540in}{0.451404in}}%
\pgfpathlineto{\pgfqpoint{0.821474in}{0.451394in}}%
\pgfpathlineto{\pgfqpoint{1.073383in}{0.451391in}}%
\pgfpathlineto{\pgfqpoint{1.397265in}{0.451390in}}%
\pgfpathlineto{\pgfqpoint{1.793120in}{0.451390in}}%
\pgfpathlineto{\pgfqpoint{2.260950in}{0.451389in}}%
\pgfpathlineto{\pgfqpoint{2.800753in}{0.451389in}}%
\pgfpathlineto{\pgfqpoint{3.412530in}{0.451389in}}%
\pgfpathlineto{\pgfqpoint{4.096281in}{0.451389in}}%
\pgfpathlineto{\pgfqpoint{4.096281in}{0.451389in}}%
\pgfpathlineto{\pgfqpoint{4.096281in}{0.451389in}}%
\pgfpathlineto{\pgfqpoint{3.412530in}{0.451390in}}%
\pgfpathlineto{\pgfqpoint{2.800753in}{0.451390in}}%
\pgfpathlineto{\pgfqpoint{2.260950in}{0.451391in}}%
\pgfpathlineto{\pgfqpoint{1.793120in}{0.451393in}}%
\pgfpathlineto{\pgfqpoint{1.397265in}{0.451397in}}%
\pgfpathlineto{\pgfqpoint{1.073383in}{0.451408in}}%
\pgfpathlineto{\pgfqpoint{0.821474in}{0.451565in}}%
\pgfpathlineto{\pgfqpoint{0.641540in}{0.451413in}}%
\pgfpathlineto{\pgfqpoint{0.533579in}{0.451643in}}%
\pgfpathlineto{\pgfqpoint{0.533579in}{0.451643in}}%
\pgfpathclose%
\pgfusepath{stroke,fill}%
}%
\begin{pgfscope}%
\pgfsys@transformshift{0.000000in}{0.000000in}%
\pgfsys@useobject{currentmarker}{}%
\end{pgfscope}%
\end{pgfscope}%
\begin{pgfscope}%
\pgfpathrectangle{\pgfqpoint{0.497592in}{0.451389in}}{\pgfqpoint{3.776824in}{1.956011in}}%
\pgfusepath{clip}%
\pgfsetbuttcap%
\pgfsetroundjoin%
\definecolor{currentfill}{rgb}{0.792157,0.568627,0.380392}%
\pgfsetfillcolor{currentfill}%
\pgfsetfillopacity{0.200000}%
\pgfsetlinewidth{1.003750pt}%
\definecolor{currentstroke}{rgb}{0.792157,0.568627,0.380392}%
\pgfsetstrokecolor{currentstroke}%
\pgfsetstrokeopacity{0.200000}%
\pgfsetdash{}{0pt}%
\pgfsys@defobject{currentmarker}{\pgfqpoint{0.533579in}{0.451389in}}{\pgfqpoint{4.096281in}{0.451711in}}{%
\pgfpathmoveto{\pgfqpoint{0.533579in}{0.451711in}}%
\pgfpathlineto{\pgfqpoint{0.533579in}{0.451478in}}%
\pgfpathlineto{\pgfqpoint{0.641540in}{0.451402in}}%
\pgfpathlineto{\pgfqpoint{0.821474in}{0.451397in}}%
\pgfpathlineto{\pgfqpoint{1.073383in}{0.451392in}}%
\pgfpathlineto{\pgfqpoint{1.397265in}{0.451391in}}%
\pgfpathlineto{\pgfqpoint{1.793120in}{0.451390in}}%
\pgfpathlineto{\pgfqpoint{2.260950in}{0.451390in}}%
\pgfpathlineto{\pgfqpoint{2.800753in}{0.451389in}}%
\pgfpathlineto{\pgfqpoint{3.412530in}{0.451389in}}%
\pgfpathlineto{\pgfqpoint{4.096281in}{0.451389in}}%
\pgfpathlineto{\pgfqpoint{4.096281in}{0.451389in}}%
\pgfpathlineto{\pgfqpoint{4.096281in}{0.451389in}}%
\pgfpathlineto{\pgfqpoint{3.412530in}{0.451389in}}%
\pgfpathlineto{\pgfqpoint{2.800753in}{0.451390in}}%
\pgfpathlineto{\pgfqpoint{2.260950in}{0.451390in}}%
\pgfpathlineto{\pgfqpoint{1.793120in}{0.451391in}}%
\pgfpathlineto{\pgfqpoint{1.397265in}{0.451391in}}%
\pgfpathlineto{\pgfqpoint{1.073383in}{0.451394in}}%
\pgfpathlineto{\pgfqpoint{0.821474in}{0.451403in}}%
\pgfpathlineto{\pgfqpoint{0.641540in}{0.451430in}}%
\pgfpathlineto{\pgfqpoint{0.533579in}{0.451711in}}%
\pgfpathlineto{\pgfqpoint{0.533579in}{0.451711in}}%
\pgfpathclose%
\pgfusepath{stroke,fill}%
}%
\begin{pgfscope}%
\pgfsys@transformshift{0.000000in}{0.000000in}%
\pgfsys@useobject{currentmarker}{}%
\end{pgfscope}%
\end{pgfscope}%
\begin{pgfscope}%
\pgfsetrectcap%
\pgfsetmiterjoin%
\pgfsetlinewidth{1.254687pt}%
\definecolor{currentstroke}{rgb}{0.800000,0.800000,0.800000}%
\pgfsetstrokecolor{currentstroke}%
\pgfsetdash{}{0pt}%
\pgfpathmoveto{\pgfqpoint{0.497592in}{0.451389in}}%
\pgfpathlineto{\pgfqpoint{0.497592in}{2.407400in}}%
\pgfusepath{stroke}%
\end{pgfscope}%
\begin{pgfscope}%
\pgfsetrectcap%
\pgfsetmiterjoin%
\pgfsetlinewidth{1.254687pt}%
\definecolor{currentstroke}{rgb}{0.800000,0.800000,0.800000}%
\pgfsetstrokecolor{currentstroke}%
\pgfsetdash{}{0pt}%
\pgfpathmoveto{\pgfqpoint{4.274417in}{0.451389in}}%
\pgfpathlineto{\pgfqpoint{4.274417in}{2.407400in}}%
\pgfusepath{stroke}%
\end{pgfscope}%
\begin{pgfscope}%
\pgfsetrectcap%
\pgfsetmiterjoin%
\pgfsetlinewidth{1.254687pt}%
\definecolor{currentstroke}{rgb}{0.800000,0.800000,0.800000}%
\pgfsetstrokecolor{currentstroke}%
\pgfsetdash{}{0pt}%
\pgfpathmoveto{\pgfqpoint{0.497592in}{0.451389in}}%
\pgfpathlineto{\pgfqpoint{4.274417in}{0.451389in}}%
\pgfusepath{stroke}%
\end{pgfscope}%
\begin{pgfscope}%
\pgfsetrectcap%
\pgfsetmiterjoin%
\pgfsetlinewidth{1.254687pt}%
\definecolor{currentstroke}{rgb}{0.800000,0.800000,0.800000}%
\pgfsetstrokecolor{currentstroke}%
\pgfsetdash{}{0pt}%
\pgfpathmoveto{\pgfqpoint{0.497592in}{2.407400in}}%
\pgfpathlineto{\pgfqpoint{4.274417in}{2.407400in}}%
\pgfusepath{stroke}%
\end{pgfscope}%
\begin{pgfscope}%
\pgfsetbuttcap%
\pgfsetmiterjoin%
\definecolor{currentfill}{rgb}{1.000000,1.000000,1.000000}%
\pgfsetfillcolor{currentfill}%
\pgfsetfillopacity{0.800000}%
\pgfsetlinewidth{1.003750pt}%
\definecolor{currentstroke}{rgb}{0.800000,0.800000,0.800000}%
\pgfsetstrokecolor{currentstroke}%
\pgfsetstrokeopacity{0.800000}%
\pgfsetdash{}{0pt}%
\pgfpathmoveto{\pgfqpoint{4.456337in}{0.538404in}}%
\pgfpathlineto{\pgfqpoint{5.650893in}{0.538404in}}%
\pgfpathquadraticcurveto{\pgfqpoint{5.675893in}{0.538404in}}{\pgfqpoint{5.675893in}{0.563404in}}%
\pgfpathlineto{\pgfqpoint{5.675893in}{2.295385in}}%
\pgfpathquadraticcurveto{\pgfqpoint{5.675893in}{2.320385in}}{\pgfqpoint{5.650893in}{2.320385in}}%
\pgfpathlineto{\pgfqpoint{4.456337in}{2.320385in}}%
\pgfpathquadraticcurveto{\pgfqpoint{4.431337in}{2.320385in}}{\pgfqpoint{4.431337in}{2.295385in}}%
\pgfpathlineto{\pgfqpoint{4.431337in}{0.563404in}}%
\pgfpathquadraticcurveto{\pgfqpoint{4.431337in}{0.538404in}}{\pgfqpoint{4.456337in}{0.538404in}}%
\pgfpathlineto{\pgfqpoint{4.456337in}{0.538404in}}%
\pgfpathclose%
\pgfusepath{stroke,fill}%
\end{pgfscope}%
\begin{pgfscope}%
\definecolor{textcolor}{rgb}{0.150000,0.150000,0.150000}%
\pgfsetstrokecolor{textcolor}%
\pgfsetfillcolor{textcolor}%
\pgftext[x=4.850215in,y=2.175414in,left,base]{\color{textcolor}\sffamily\fontsize{9.000000}{10.800000}\selectfont Legend}%
\end{pgfscope}%
\begin{pgfscope}%
\pgfsetroundcap%
\pgfsetroundjoin%
\pgfsetlinewidth{1.505625pt}%
\definecolor{currentstroke}{rgb}{0.003922,0.450980,0.698039}%
\pgfsetstrokecolor{currentstroke}%
\pgfsetdash{}{0pt}%
\pgfpathmoveto{\pgfqpoint{4.481337in}{2.031664in}}%
\pgfpathlineto{\pgfqpoint{4.606337in}{2.031664in}}%
\pgfpathlineto{\pgfqpoint{4.731337in}{2.031664in}}%
\pgfusepath{stroke}%
\end{pgfscope}%
\begin{pgfscope}%
\definecolor{textcolor}{rgb}{0.150000,0.150000,0.150000}%
\pgfsetstrokecolor{textcolor}%
\pgfsetfillcolor{textcolor}%
\pgftext[x=4.831337in,y=1.987914in,left,base]{\color{textcolor}\sffamily\fontsize{9.000000}{10.800000}\selectfont Total time}%
\end{pgfscope}%
\begin{pgfscope}%
\pgfsetroundcap%
\pgfsetroundjoin%
\pgfsetlinewidth{1.505625pt}%
\definecolor{currentstroke}{rgb}{0.870588,0.560784,0.019608}%
\pgfsetstrokecolor{currentstroke}%
\pgfsetdash{}{0pt}%
\pgfpathmoveto{\pgfqpoint{4.481337in}{1.844164in}}%
\pgfpathlineto{\pgfqpoint{4.606337in}{1.844164in}}%
\pgfpathlineto{\pgfqpoint{4.731337in}{1.844164in}}%
\pgfusepath{stroke}%
\end{pgfscope}%
\begin{pgfscope}%
\definecolor{textcolor}{rgb}{0.150000,0.150000,0.150000}%
\pgfsetstrokecolor{textcolor}%
\pgfsetfillcolor{textcolor}%
\pgftext[x=4.831337in,y=1.800414in,left,base]{\color{textcolor}\sffamily\fontsize{9.000000}{10.800000}\selectfont kNN search}%
\end{pgfscope}%
\begin{pgfscope}%
\pgfsetroundcap%
\pgfsetroundjoin%
\pgfsetlinewidth{1.505625pt}%
\definecolor{currentstroke}{rgb}{0.007843,0.619608,0.450980}%
\pgfsetstrokecolor{currentstroke}%
\pgfsetdash{}{0pt}%
\pgfpathmoveto{\pgfqpoint{4.481337in}{1.656664in}}%
\pgfpathlineto{\pgfqpoint{4.606337in}{1.656664in}}%
\pgfpathlineto{\pgfqpoint{4.731337in}{1.656664in}}%
\pgfusepath{stroke}%
\end{pgfscope}%
\begin{pgfscope}%
\definecolor{textcolor}{rgb}{0.150000,0.150000,0.150000}%
\pgfsetstrokecolor{textcolor}%
\pgfsetfillcolor{textcolor}%
\pgftext[x=4.831337in,y=1.612914in,left,base]{\color{textcolor}\sffamily\fontsize{9.000000}{10.800000}\selectfont Create graph}%
\end{pgfscope}%
\begin{pgfscope}%
\pgfsetroundcap%
\pgfsetroundjoin%
\pgfsetlinewidth{1.505625pt}%
\definecolor{currentstroke}{rgb}{0.835294,0.368627,0.000000}%
\pgfsetstrokecolor{currentstroke}%
\pgfsetdash{}{0pt}%
\pgfpathmoveto{\pgfqpoint{4.481337in}{1.382153in}}%
\pgfpathlineto{\pgfqpoint{4.606337in}{1.382153in}}%
\pgfpathlineto{\pgfqpoint{4.731337in}{1.382153in}}%
\pgfusepath{stroke}%
\end{pgfscope}%
\begin{pgfscope}%
\definecolor{textcolor}{rgb}{0.150000,0.150000,0.150000}%
\pgfsetstrokecolor{textcolor}%
\pgfsetfillcolor{textcolor}%
\pgftext[x=4.831337in, y=1.425415in, left, base]{\color{textcolor}\sffamily\fontsize{9.000000}{10.800000}\selectfont Get \& prepare}%
\end{pgfscope}%
\begin{pgfscope}%
\definecolor{textcolor}{rgb}{0.150000,0.150000,0.150000}%
\pgfsetstrokecolor{textcolor}%
\pgfsetfillcolor{textcolor}%
\pgftext[x=4.831337in, y=1.281421in, left, base]{\color{textcolor}\sffamily\fontsize{9.000000}{10.800000}\selectfont obstacles}%
\end{pgfscope}%
\begin{pgfscope}%
\pgfsetroundcap%
\pgfsetroundjoin%
\pgfsetlinewidth{1.505625pt}%
\definecolor{currentstroke}{rgb}{0.800000,0.470588,0.737255}%
\pgfsetstrokecolor{currentstroke}%
\pgfsetdash{}{0pt}%
\pgfpathmoveto{\pgfqpoint{4.481337in}{1.050659in}}%
\pgfpathlineto{\pgfqpoint{4.606337in}{1.050659in}}%
\pgfpathlineto{\pgfqpoint{4.731337in}{1.050659in}}%
\pgfusepath{stroke}%
\end{pgfscope}%
\begin{pgfscope}%
\definecolor{textcolor}{rgb}{0.150000,0.150000,0.150000}%
\pgfsetstrokecolor{textcolor}%
\pgfsetfillcolor{textcolor}%
\pgftext[x=4.831337in, y=1.093921in, left, base]{\color{textcolor}\sffamily\fontsize{9.000000}{10.800000}\selectfont Merge road}%
\end{pgfscope}%
\begin{pgfscope}%
\definecolor{textcolor}{rgb}{0.150000,0.150000,0.150000}%
\pgfsetstrokecolor{textcolor}%
\pgfsetfillcolor{textcolor}%
\pgftext[x=4.831337in, y=0.949927in, left, base]{\color{textcolor}\sffamily\fontsize{9.000000}{10.800000}\selectfont edges}%
\end{pgfscope}%
\begin{pgfscope}%
\pgfsetroundcap%
\pgfsetroundjoin%
\pgfsetlinewidth{1.505625pt}%
\definecolor{currentstroke}{rgb}{0.792157,0.568627,0.380392}%
\pgfsetstrokecolor{currentstroke}%
\pgfsetdash{}{0pt}%
\pgfpathmoveto{\pgfqpoint{4.481337in}{0.719165in}}%
\pgfpathlineto{\pgfqpoint{4.606337in}{0.719165in}}%
\pgfpathlineto{\pgfqpoint{4.731337in}{0.719165in}}%
\pgfusepath{stroke}%
\end{pgfscope}%
\begin{pgfscope}%
\definecolor{textcolor}{rgb}{0.150000,0.150000,0.150000}%
\pgfsetstrokecolor{textcolor}%
\pgfsetfillcolor{textcolor}%
\pgftext[x=4.831337in, y=0.762427in, left, base]{\color{textcolor}\sffamily\fontsize{9.000000}{10.800000}\selectfont Add POI}%
\end{pgfscope}%
\begin{pgfscope}%
\definecolor{textcolor}{rgb}{0.150000,0.150000,0.150000}%
\pgfsetstrokecolor{textcolor}%
\pgfsetfillcolor{textcolor}%
\pgftext[x=4.831337in, y=0.618433in, left, base]{\color{textcolor}\sffamily\fontsize{9.000000}{10.800000}\selectfont attributes}%
\end{pgfscope}%
\begin{pgfscope}%
\pgfsetroundcap%
\pgfsetroundjoin%
\pgfsetlinewidth{1.003750pt}%
\definecolor{currentstroke}{rgb}{0.003922,0.450980,0.698039}%
\pgfsetstrokecolor{currentstroke}%
\pgfsetdash{}{0pt}%
\pgfpathmoveto{\pgfqpoint{0.533579in}{2.314256in}}%
\pgfpathlineto{\pgfqpoint{0.641540in}{2.314256in}}%
\pgfpathlineto{\pgfqpoint{0.821474in}{2.314256in}}%
\pgfpathlineto{\pgfqpoint{1.073383in}{2.314256in}}%
\pgfpathlineto{\pgfqpoint{1.397265in}{2.314256in}}%
\pgfpathlineto{\pgfqpoint{1.793120in}{2.314256in}}%
\pgfpathlineto{\pgfqpoint{2.260950in}{2.314256in}}%
\pgfpathlineto{\pgfqpoint{2.800753in}{2.314256in}}%
\pgfpathlineto{\pgfqpoint{3.412530in}{2.314256in}}%
\pgfpathlineto{\pgfqpoint{4.096281in}{2.314256in}}%
\pgfusepath{stroke}%
\end{pgfscope}%
\begin{pgfscope}%
\pgfsetbuttcap%
\pgfsetroundjoin%
\definecolor{currentfill}{rgb}{0.003922,0.450980,0.698039}%
\pgfsetfillcolor{currentfill}%
\pgfsetlinewidth{0.752812pt}%
\definecolor{currentstroke}{rgb}{1.000000,1.000000,1.000000}%
\pgfsetstrokecolor{currentstroke}%
\pgfsetdash{}{0pt}%
\pgfsys@defobject{currentmarker}{\pgfqpoint{-0.034722in}{-0.034722in}}{\pgfqpoint{0.034722in}{0.034722in}}{%
\pgfpathmoveto{\pgfqpoint{0.000000in}{-0.034722in}}%
\pgfpathcurveto{\pgfqpoint{0.009208in}{-0.034722in}}{\pgfqpoint{0.018041in}{-0.031064in}}{\pgfqpoint{0.024552in}{-0.024552in}}%
\pgfpathcurveto{\pgfqpoint{0.031064in}{-0.018041in}}{\pgfqpoint{0.034722in}{-0.009208in}}{\pgfqpoint{0.034722in}{0.000000in}}%
\pgfpathcurveto{\pgfqpoint{0.034722in}{0.009208in}}{\pgfqpoint{0.031064in}{0.018041in}}{\pgfqpoint{0.024552in}{0.024552in}}%
\pgfpathcurveto{\pgfqpoint{0.018041in}{0.031064in}}{\pgfqpoint{0.009208in}{0.034722in}}{\pgfqpoint{0.000000in}{0.034722in}}%
\pgfpathcurveto{\pgfqpoint{-0.009208in}{0.034722in}}{\pgfqpoint{-0.018041in}{0.031064in}}{\pgfqpoint{-0.024552in}{0.024552in}}%
\pgfpathcurveto{\pgfqpoint{-0.031064in}{0.018041in}}{\pgfqpoint{-0.034722in}{0.009208in}}{\pgfqpoint{-0.034722in}{0.000000in}}%
\pgfpathcurveto{\pgfqpoint{-0.034722in}{-0.009208in}}{\pgfqpoint{-0.031064in}{-0.018041in}}{\pgfqpoint{-0.024552in}{-0.024552in}}%
\pgfpathcurveto{\pgfqpoint{-0.018041in}{-0.031064in}}{\pgfqpoint{-0.009208in}{-0.034722in}}{\pgfqpoint{0.000000in}{-0.034722in}}%
\pgfpathlineto{\pgfqpoint{0.000000in}{-0.034722in}}%
\pgfpathclose%
\pgfusepath{stroke,fill}%
}%
\begin{pgfscope}%
\pgfsys@transformshift{0.533579in}{2.314256in}%
\pgfsys@useobject{currentmarker}{}%
\end{pgfscope}%
\begin{pgfscope}%
\pgfsys@transformshift{0.641540in}{2.314256in}%
\pgfsys@useobject{currentmarker}{}%
\end{pgfscope}%
\begin{pgfscope}%
\pgfsys@transformshift{0.821474in}{2.314256in}%
\pgfsys@useobject{currentmarker}{}%
\end{pgfscope}%
\begin{pgfscope}%
\pgfsys@transformshift{1.073383in}{2.314256in}%
\pgfsys@useobject{currentmarker}{}%
\end{pgfscope}%
\begin{pgfscope}%
\pgfsys@transformshift{1.397265in}{2.314256in}%
\pgfsys@useobject{currentmarker}{}%
\end{pgfscope}%
\begin{pgfscope}%
\pgfsys@transformshift{1.793120in}{2.314256in}%
\pgfsys@useobject{currentmarker}{}%
\end{pgfscope}%
\begin{pgfscope}%
\pgfsys@transformshift{2.260950in}{2.314256in}%
\pgfsys@useobject{currentmarker}{}%
\end{pgfscope}%
\begin{pgfscope}%
\pgfsys@transformshift{2.800753in}{2.314256in}%
\pgfsys@useobject{currentmarker}{}%
\end{pgfscope}%
\begin{pgfscope}%
\pgfsys@transformshift{3.412530in}{2.314256in}%
\pgfsys@useobject{currentmarker}{}%
\end{pgfscope}%
\begin{pgfscope}%
\pgfsys@transformshift{4.096281in}{2.314256in}%
\pgfsys@useobject{currentmarker}{}%
\end{pgfscope}%
\end{pgfscope}%
\begin{pgfscope}%
\pgfsetroundcap%
\pgfsetroundjoin%
\pgfsetlinewidth{1.003750pt}%
\definecolor{currentstroke}{rgb}{0.870588,0.560784,0.019608}%
\pgfsetstrokecolor{currentstroke}%
\pgfsetdash{}{0pt}%
\pgfpathmoveto{\pgfqpoint{0.533579in}{2.073143in}}%
\pgfpathlineto{\pgfqpoint{0.641540in}{2.228902in}}%
\pgfpathlineto{\pgfqpoint{0.821474in}{2.269753in}}%
\pgfpathlineto{\pgfqpoint{1.073383in}{2.288048in}}%
\pgfpathlineto{\pgfqpoint{1.397265in}{2.296942in}}%
\pgfpathlineto{\pgfqpoint{1.793120in}{2.302609in}}%
\pgfpathlineto{\pgfqpoint{2.260950in}{2.306107in}}%
\pgfpathlineto{\pgfqpoint{2.800753in}{2.309254in}}%
\pgfpathlineto{\pgfqpoint{3.412530in}{2.310265in}}%
\pgfpathlineto{\pgfqpoint{4.096281in}{2.311098in}}%
\pgfusepath{stroke}%
\end{pgfscope}%
\begin{pgfscope}%
\pgfsetbuttcap%
\pgfsetroundjoin%
\definecolor{currentfill}{rgb}{0.870588,0.560784,0.019608}%
\pgfsetfillcolor{currentfill}%
\pgfsetlinewidth{0.752812pt}%
\definecolor{currentstroke}{rgb}{1.000000,1.000000,1.000000}%
\pgfsetstrokecolor{currentstroke}%
\pgfsetdash{}{0pt}%
\pgfsys@defobject{currentmarker}{\pgfqpoint{-0.034722in}{-0.034722in}}{\pgfqpoint{0.034722in}{0.034722in}}{%
\pgfpathmoveto{\pgfqpoint{0.000000in}{-0.034722in}}%
\pgfpathcurveto{\pgfqpoint{0.009208in}{-0.034722in}}{\pgfqpoint{0.018041in}{-0.031064in}}{\pgfqpoint{0.024552in}{-0.024552in}}%
\pgfpathcurveto{\pgfqpoint{0.031064in}{-0.018041in}}{\pgfqpoint{0.034722in}{-0.009208in}}{\pgfqpoint{0.034722in}{0.000000in}}%
\pgfpathcurveto{\pgfqpoint{0.034722in}{0.009208in}}{\pgfqpoint{0.031064in}{0.018041in}}{\pgfqpoint{0.024552in}{0.024552in}}%
\pgfpathcurveto{\pgfqpoint{0.018041in}{0.031064in}}{\pgfqpoint{0.009208in}{0.034722in}}{\pgfqpoint{0.000000in}{0.034722in}}%
\pgfpathcurveto{\pgfqpoint{-0.009208in}{0.034722in}}{\pgfqpoint{-0.018041in}{0.031064in}}{\pgfqpoint{-0.024552in}{0.024552in}}%
\pgfpathcurveto{\pgfqpoint{-0.031064in}{0.018041in}}{\pgfqpoint{-0.034722in}{0.009208in}}{\pgfqpoint{-0.034722in}{0.000000in}}%
\pgfpathcurveto{\pgfqpoint{-0.034722in}{-0.009208in}}{\pgfqpoint{-0.031064in}{-0.018041in}}{\pgfqpoint{-0.024552in}{-0.024552in}}%
\pgfpathcurveto{\pgfqpoint{-0.018041in}{-0.031064in}}{\pgfqpoint{-0.009208in}{-0.034722in}}{\pgfqpoint{0.000000in}{-0.034722in}}%
\pgfpathlineto{\pgfqpoint{0.000000in}{-0.034722in}}%
\pgfpathclose%
\pgfusepath{stroke,fill}%
}%
\begin{pgfscope}%
\pgfsys@transformshift{0.533579in}{2.073143in}%
\pgfsys@useobject{currentmarker}{}%
\end{pgfscope}%
\begin{pgfscope}%
\pgfsys@transformshift{0.641540in}{2.228902in}%
\pgfsys@useobject{currentmarker}{}%
\end{pgfscope}%
\begin{pgfscope}%
\pgfsys@transformshift{0.821474in}{2.269753in}%
\pgfsys@useobject{currentmarker}{}%
\end{pgfscope}%
\begin{pgfscope}%
\pgfsys@transformshift{1.073383in}{2.288048in}%
\pgfsys@useobject{currentmarker}{}%
\end{pgfscope}%
\begin{pgfscope}%
\pgfsys@transformshift{1.397265in}{2.296942in}%
\pgfsys@useobject{currentmarker}{}%
\end{pgfscope}%
\begin{pgfscope}%
\pgfsys@transformshift{1.793120in}{2.302609in}%
\pgfsys@useobject{currentmarker}{}%
\end{pgfscope}%
\begin{pgfscope}%
\pgfsys@transformshift{2.260950in}{2.306107in}%
\pgfsys@useobject{currentmarker}{}%
\end{pgfscope}%
\begin{pgfscope}%
\pgfsys@transformshift{2.800753in}{2.309254in}%
\pgfsys@useobject{currentmarker}{}%
\end{pgfscope}%
\begin{pgfscope}%
\pgfsys@transformshift{3.412530in}{2.310265in}%
\pgfsys@useobject{currentmarker}{}%
\end{pgfscope}%
\begin{pgfscope}%
\pgfsys@transformshift{4.096281in}{2.311098in}%
\pgfsys@useobject{currentmarker}{}%
\end{pgfscope}%
\end{pgfscope}%
\begin{pgfscope}%
\pgfsetroundcap%
\pgfsetroundjoin%
\pgfsetlinewidth{1.003750pt}%
\definecolor{currentstroke}{rgb}{0.007843,0.619608,0.450980}%
\pgfsetstrokecolor{currentstroke}%
\pgfsetdash{}{0pt}%
\pgfpathmoveto{\pgfqpoint{0.533579in}{0.503812in}}%
\pgfpathlineto{\pgfqpoint{0.641540in}{0.484028in}}%
\pgfpathlineto{\pgfqpoint{0.821474in}{0.468064in}}%
\pgfpathlineto{\pgfqpoint{1.073383in}{0.462826in}}%
\pgfpathlineto{\pgfqpoint{1.397265in}{0.459864in}}%
\pgfpathlineto{\pgfqpoint{1.793120in}{0.456305in}}%
\pgfpathlineto{\pgfqpoint{2.260950in}{0.455172in}}%
\pgfpathlineto{\pgfqpoint{2.800753in}{0.453449in}}%
\pgfpathlineto{\pgfqpoint{3.412530in}{0.453049in}}%
\pgfpathlineto{\pgfqpoint{4.096281in}{0.452726in}}%
\pgfusepath{stroke}%
\end{pgfscope}%
\begin{pgfscope}%
\pgfsetbuttcap%
\pgfsetroundjoin%
\definecolor{currentfill}{rgb}{0.007843,0.619608,0.450980}%
\pgfsetfillcolor{currentfill}%
\pgfsetlinewidth{0.752812pt}%
\definecolor{currentstroke}{rgb}{1.000000,1.000000,1.000000}%
\pgfsetstrokecolor{currentstroke}%
\pgfsetdash{}{0pt}%
\pgfsys@defobject{currentmarker}{\pgfqpoint{-0.034722in}{-0.034722in}}{\pgfqpoint{0.034722in}{0.034722in}}{%
\pgfpathmoveto{\pgfqpoint{0.000000in}{-0.034722in}}%
\pgfpathcurveto{\pgfqpoint{0.009208in}{-0.034722in}}{\pgfqpoint{0.018041in}{-0.031064in}}{\pgfqpoint{0.024552in}{-0.024552in}}%
\pgfpathcurveto{\pgfqpoint{0.031064in}{-0.018041in}}{\pgfqpoint{0.034722in}{-0.009208in}}{\pgfqpoint{0.034722in}{0.000000in}}%
\pgfpathcurveto{\pgfqpoint{0.034722in}{0.009208in}}{\pgfqpoint{0.031064in}{0.018041in}}{\pgfqpoint{0.024552in}{0.024552in}}%
\pgfpathcurveto{\pgfqpoint{0.018041in}{0.031064in}}{\pgfqpoint{0.009208in}{0.034722in}}{\pgfqpoint{0.000000in}{0.034722in}}%
\pgfpathcurveto{\pgfqpoint{-0.009208in}{0.034722in}}{\pgfqpoint{-0.018041in}{0.031064in}}{\pgfqpoint{-0.024552in}{0.024552in}}%
\pgfpathcurveto{\pgfqpoint{-0.031064in}{0.018041in}}{\pgfqpoint{-0.034722in}{0.009208in}}{\pgfqpoint{-0.034722in}{0.000000in}}%
\pgfpathcurveto{\pgfqpoint{-0.034722in}{-0.009208in}}{\pgfqpoint{-0.031064in}{-0.018041in}}{\pgfqpoint{-0.024552in}{-0.024552in}}%
\pgfpathcurveto{\pgfqpoint{-0.018041in}{-0.031064in}}{\pgfqpoint{-0.009208in}{-0.034722in}}{\pgfqpoint{0.000000in}{-0.034722in}}%
\pgfpathlineto{\pgfqpoint{0.000000in}{-0.034722in}}%
\pgfpathclose%
\pgfusepath{stroke,fill}%
}%
\begin{pgfscope}%
\pgfsys@transformshift{0.533579in}{0.503812in}%
\pgfsys@useobject{currentmarker}{}%
\end{pgfscope}%
\begin{pgfscope}%
\pgfsys@transformshift{0.641540in}{0.484028in}%
\pgfsys@useobject{currentmarker}{}%
\end{pgfscope}%
\begin{pgfscope}%
\pgfsys@transformshift{0.821474in}{0.468064in}%
\pgfsys@useobject{currentmarker}{}%
\end{pgfscope}%
\begin{pgfscope}%
\pgfsys@transformshift{1.073383in}{0.462826in}%
\pgfsys@useobject{currentmarker}{}%
\end{pgfscope}%
\begin{pgfscope}%
\pgfsys@transformshift{1.397265in}{0.459864in}%
\pgfsys@useobject{currentmarker}{}%
\end{pgfscope}%
\begin{pgfscope}%
\pgfsys@transformshift{1.793120in}{0.456305in}%
\pgfsys@useobject{currentmarker}{}%
\end{pgfscope}%
\begin{pgfscope}%
\pgfsys@transformshift{2.260950in}{0.455172in}%
\pgfsys@useobject{currentmarker}{}%
\end{pgfscope}%
\begin{pgfscope}%
\pgfsys@transformshift{2.800753in}{0.453449in}%
\pgfsys@useobject{currentmarker}{}%
\end{pgfscope}%
\begin{pgfscope}%
\pgfsys@transformshift{3.412530in}{0.453049in}%
\pgfsys@useobject{currentmarker}{}%
\end{pgfscope}%
\begin{pgfscope}%
\pgfsys@transformshift{4.096281in}{0.452726in}%
\pgfsys@useobject{currentmarker}{}%
\end{pgfscope}%
\end{pgfscope}%
\begin{pgfscope}%
\pgfsetroundcap%
\pgfsetroundjoin%
\pgfsetlinewidth{1.003750pt}%
\definecolor{currentstroke}{rgb}{0.835294,0.368627,0.000000}%
\pgfsetstrokecolor{currentstroke}%
\pgfsetdash{}{0pt}%
\pgfpathmoveto{\pgfqpoint{0.533579in}{0.622179in}}%
\pgfpathlineto{\pgfqpoint{0.641540in}{0.500814in}}%
\pgfpathlineto{\pgfqpoint{0.821474in}{0.476821in}}%
\pgfpathlineto{\pgfqpoint{1.073383in}{0.465080in}}%
\pgfpathlineto{\pgfqpoint{1.397265in}{0.459562in}}%
\pgfpathlineto{\pgfqpoint{1.793120in}{0.457704in}}%
\pgfpathlineto{\pgfqpoint{2.260950in}{0.455484in}}%
\pgfpathlineto{\pgfqpoint{2.800753in}{0.454153in}}%
\pgfpathlineto{\pgfqpoint{3.412530in}{0.453574in}}%
\pgfpathlineto{\pgfqpoint{4.096281in}{0.453103in}}%
\pgfusepath{stroke}%
\end{pgfscope}%
\begin{pgfscope}%
\pgfsetbuttcap%
\pgfsetroundjoin%
\definecolor{currentfill}{rgb}{0.835294,0.368627,0.000000}%
\pgfsetfillcolor{currentfill}%
\pgfsetlinewidth{0.752812pt}%
\definecolor{currentstroke}{rgb}{1.000000,1.000000,1.000000}%
\pgfsetstrokecolor{currentstroke}%
\pgfsetdash{}{0pt}%
\pgfsys@defobject{currentmarker}{\pgfqpoint{-0.034722in}{-0.034722in}}{\pgfqpoint{0.034722in}{0.034722in}}{%
\pgfpathmoveto{\pgfqpoint{0.000000in}{-0.034722in}}%
\pgfpathcurveto{\pgfqpoint{0.009208in}{-0.034722in}}{\pgfqpoint{0.018041in}{-0.031064in}}{\pgfqpoint{0.024552in}{-0.024552in}}%
\pgfpathcurveto{\pgfqpoint{0.031064in}{-0.018041in}}{\pgfqpoint{0.034722in}{-0.009208in}}{\pgfqpoint{0.034722in}{0.000000in}}%
\pgfpathcurveto{\pgfqpoint{0.034722in}{0.009208in}}{\pgfqpoint{0.031064in}{0.018041in}}{\pgfqpoint{0.024552in}{0.024552in}}%
\pgfpathcurveto{\pgfqpoint{0.018041in}{0.031064in}}{\pgfqpoint{0.009208in}{0.034722in}}{\pgfqpoint{0.000000in}{0.034722in}}%
\pgfpathcurveto{\pgfqpoint{-0.009208in}{0.034722in}}{\pgfqpoint{-0.018041in}{0.031064in}}{\pgfqpoint{-0.024552in}{0.024552in}}%
\pgfpathcurveto{\pgfqpoint{-0.031064in}{0.018041in}}{\pgfqpoint{-0.034722in}{0.009208in}}{\pgfqpoint{-0.034722in}{0.000000in}}%
\pgfpathcurveto{\pgfqpoint{-0.034722in}{-0.009208in}}{\pgfqpoint{-0.031064in}{-0.018041in}}{\pgfqpoint{-0.024552in}{-0.024552in}}%
\pgfpathcurveto{\pgfqpoint{-0.018041in}{-0.031064in}}{\pgfqpoint{-0.009208in}{-0.034722in}}{\pgfqpoint{0.000000in}{-0.034722in}}%
\pgfpathlineto{\pgfqpoint{0.000000in}{-0.034722in}}%
\pgfpathclose%
\pgfusepath{stroke,fill}%
}%
\begin{pgfscope}%
\pgfsys@transformshift{0.533579in}{0.622179in}%
\pgfsys@useobject{currentmarker}{}%
\end{pgfscope}%
\begin{pgfscope}%
\pgfsys@transformshift{0.641540in}{0.500814in}%
\pgfsys@useobject{currentmarker}{}%
\end{pgfscope}%
\begin{pgfscope}%
\pgfsys@transformshift{0.821474in}{0.476821in}%
\pgfsys@useobject{currentmarker}{}%
\end{pgfscope}%
\begin{pgfscope}%
\pgfsys@transformshift{1.073383in}{0.465080in}%
\pgfsys@useobject{currentmarker}{}%
\end{pgfscope}%
\begin{pgfscope}%
\pgfsys@transformshift{1.397265in}{0.459562in}%
\pgfsys@useobject{currentmarker}{}%
\end{pgfscope}%
\begin{pgfscope}%
\pgfsys@transformshift{1.793120in}{0.457704in}%
\pgfsys@useobject{currentmarker}{}%
\end{pgfscope}%
\begin{pgfscope}%
\pgfsys@transformshift{2.260950in}{0.455484in}%
\pgfsys@useobject{currentmarker}{}%
\end{pgfscope}%
\begin{pgfscope}%
\pgfsys@transformshift{2.800753in}{0.454153in}%
\pgfsys@useobject{currentmarker}{}%
\end{pgfscope}%
\begin{pgfscope}%
\pgfsys@transformshift{3.412530in}{0.453574in}%
\pgfsys@useobject{currentmarker}{}%
\end{pgfscope}%
\begin{pgfscope}%
\pgfsys@transformshift{4.096281in}{0.453103in}%
\pgfsys@useobject{currentmarker}{}%
\end{pgfscope}%
\end{pgfscope}%
\begin{pgfscope}%
\pgfsetroundcap%
\pgfsetroundjoin%
\pgfsetlinewidth{1.003750pt}%
\definecolor{currentstroke}{rgb}{0.800000,0.470588,0.737255}%
\pgfsetstrokecolor{currentstroke}%
\pgfsetdash{}{0pt}%
\pgfpathmoveto{\pgfqpoint{0.533579in}{0.451581in}}%
\pgfpathlineto{\pgfqpoint{0.641540in}{0.451410in}}%
\pgfpathlineto{\pgfqpoint{0.821474in}{0.451450in}}%
\pgfpathlineto{\pgfqpoint{1.073383in}{0.451398in}}%
\pgfpathlineto{\pgfqpoint{1.397265in}{0.451393in}}%
\pgfpathlineto{\pgfqpoint{1.793120in}{0.451391in}}%
\pgfpathlineto{\pgfqpoint{2.260950in}{0.451390in}}%
\pgfpathlineto{\pgfqpoint{2.800753in}{0.451389in}}%
\pgfpathlineto{\pgfqpoint{3.412530in}{0.451389in}}%
\pgfpathlineto{\pgfqpoint{4.096281in}{0.451389in}}%
\pgfusepath{stroke}%
\end{pgfscope}%
\begin{pgfscope}%
\pgfsetbuttcap%
\pgfsetroundjoin%
\definecolor{currentfill}{rgb}{0.800000,0.470588,0.737255}%
\pgfsetfillcolor{currentfill}%
\pgfsetlinewidth{0.752812pt}%
\definecolor{currentstroke}{rgb}{1.000000,1.000000,1.000000}%
\pgfsetstrokecolor{currentstroke}%
\pgfsetdash{}{0pt}%
\pgfsys@defobject{currentmarker}{\pgfqpoint{-0.034722in}{-0.034722in}}{\pgfqpoint{0.034722in}{0.034722in}}{%
\pgfpathmoveto{\pgfqpoint{0.000000in}{-0.034722in}}%
\pgfpathcurveto{\pgfqpoint{0.009208in}{-0.034722in}}{\pgfqpoint{0.018041in}{-0.031064in}}{\pgfqpoint{0.024552in}{-0.024552in}}%
\pgfpathcurveto{\pgfqpoint{0.031064in}{-0.018041in}}{\pgfqpoint{0.034722in}{-0.009208in}}{\pgfqpoint{0.034722in}{0.000000in}}%
\pgfpathcurveto{\pgfqpoint{0.034722in}{0.009208in}}{\pgfqpoint{0.031064in}{0.018041in}}{\pgfqpoint{0.024552in}{0.024552in}}%
\pgfpathcurveto{\pgfqpoint{0.018041in}{0.031064in}}{\pgfqpoint{0.009208in}{0.034722in}}{\pgfqpoint{0.000000in}{0.034722in}}%
\pgfpathcurveto{\pgfqpoint{-0.009208in}{0.034722in}}{\pgfqpoint{-0.018041in}{0.031064in}}{\pgfqpoint{-0.024552in}{0.024552in}}%
\pgfpathcurveto{\pgfqpoint{-0.031064in}{0.018041in}}{\pgfqpoint{-0.034722in}{0.009208in}}{\pgfqpoint{-0.034722in}{0.000000in}}%
\pgfpathcurveto{\pgfqpoint{-0.034722in}{-0.009208in}}{\pgfqpoint{-0.031064in}{-0.018041in}}{\pgfqpoint{-0.024552in}{-0.024552in}}%
\pgfpathcurveto{\pgfqpoint{-0.018041in}{-0.031064in}}{\pgfqpoint{-0.009208in}{-0.034722in}}{\pgfqpoint{0.000000in}{-0.034722in}}%
\pgfpathlineto{\pgfqpoint{0.000000in}{-0.034722in}}%
\pgfpathclose%
\pgfusepath{stroke,fill}%
}%
\begin{pgfscope}%
\pgfsys@transformshift{0.533579in}{0.451581in}%
\pgfsys@useobject{currentmarker}{}%
\end{pgfscope}%
\begin{pgfscope}%
\pgfsys@transformshift{0.641540in}{0.451410in}%
\pgfsys@useobject{currentmarker}{}%
\end{pgfscope}%
\begin{pgfscope}%
\pgfsys@transformshift{0.821474in}{0.451450in}%
\pgfsys@useobject{currentmarker}{}%
\end{pgfscope}%
\begin{pgfscope}%
\pgfsys@transformshift{1.073383in}{0.451398in}%
\pgfsys@useobject{currentmarker}{}%
\end{pgfscope}%
\begin{pgfscope}%
\pgfsys@transformshift{1.397265in}{0.451393in}%
\pgfsys@useobject{currentmarker}{}%
\end{pgfscope}%
\begin{pgfscope}%
\pgfsys@transformshift{1.793120in}{0.451391in}%
\pgfsys@useobject{currentmarker}{}%
\end{pgfscope}%
\begin{pgfscope}%
\pgfsys@transformshift{2.260950in}{0.451390in}%
\pgfsys@useobject{currentmarker}{}%
\end{pgfscope}%
\begin{pgfscope}%
\pgfsys@transformshift{2.800753in}{0.451389in}%
\pgfsys@useobject{currentmarker}{}%
\end{pgfscope}%
\begin{pgfscope}%
\pgfsys@transformshift{3.412530in}{0.451389in}%
\pgfsys@useobject{currentmarker}{}%
\end{pgfscope}%
\begin{pgfscope}%
\pgfsys@transformshift{4.096281in}{0.451389in}%
\pgfsys@useobject{currentmarker}{}%
\end{pgfscope}%
\end{pgfscope}%
\begin{pgfscope}%
\pgfsetroundcap%
\pgfsetroundjoin%
\pgfsetlinewidth{1.003750pt}%
\definecolor{currentstroke}{rgb}{0.792157,0.568627,0.380392}%
\pgfsetstrokecolor{currentstroke}%
\pgfsetdash{}{0pt}%
\pgfpathmoveto{\pgfqpoint{0.533579in}{0.451576in}}%
\pgfpathlineto{\pgfqpoint{0.641540in}{0.451415in}}%
\pgfpathlineto{\pgfqpoint{0.821474in}{0.451400in}}%
\pgfpathlineto{\pgfqpoint{1.073383in}{0.451393in}}%
\pgfpathlineto{\pgfqpoint{1.397265in}{0.451391in}}%
\pgfpathlineto{\pgfqpoint{1.793120in}{0.451390in}}%
\pgfpathlineto{\pgfqpoint{2.260950in}{0.451390in}}%
\pgfpathlineto{\pgfqpoint{2.800753in}{0.451389in}}%
\pgfpathlineto{\pgfqpoint{3.412530in}{0.451389in}}%
\pgfpathlineto{\pgfqpoint{4.096281in}{0.451389in}}%
\pgfusepath{stroke}%
\end{pgfscope}%
\begin{pgfscope}%
\pgfsetbuttcap%
\pgfsetroundjoin%
\definecolor{currentfill}{rgb}{0.792157,0.568627,0.380392}%
\pgfsetfillcolor{currentfill}%
\pgfsetlinewidth{0.752812pt}%
\definecolor{currentstroke}{rgb}{1.000000,1.000000,1.000000}%
\pgfsetstrokecolor{currentstroke}%
\pgfsetdash{}{0pt}%
\pgfsys@defobject{currentmarker}{\pgfqpoint{-0.034722in}{-0.034722in}}{\pgfqpoint{0.034722in}{0.034722in}}{%
\pgfpathmoveto{\pgfqpoint{0.000000in}{-0.034722in}}%
\pgfpathcurveto{\pgfqpoint{0.009208in}{-0.034722in}}{\pgfqpoint{0.018041in}{-0.031064in}}{\pgfqpoint{0.024552in}{-0.024552in}}%
\pgfpathcurveto{\pgfqpoint{0.031064in}{-0.018041in}}{\pgfqpoint{0.034722in}{-0.009208in}}{\pgfqpoint{0.034722in}{0.000000in}}%
\pgfpathcurveto{\pgfqpoint{0.034722in}{0.009208in}}{\pgfqpoint{0.031064in}{0.018041in}}{\pgfqpoint{0.024552in}{0.024552in}}%
\pgfpathcurveto{\pgfqpoint{0.018041in}{0.031064in}}{\pgfqpoint{0.009208in}{0.034722in}}{\pgfqpoint{0.000000in}{0.034722in}}%
\pgfpathcurveto{\pgfqpoint{-0.009208in}{0.034722in}}{\pgfqpoint{-0.018041in}{0.031064in}}{\pgfqpoint{-0.024552in}{0.024552in}}%
\pgfpathcurveto{\pgfqpoint{-0.031064in}{0.018041in}}{\pgfqpoint{-0.034722in}{0.009208in}}{\pgfqpoint{-0.034722in}{0.000000in}}%
\pgfpathcurveto{\pgfqpoint{-0.034722in}{-0.009208in}}{\pgfqpoint{-0.031064in}{-0.018041in}}{\pgfqpoint{-0.024552in}{-0.024552in}}%
\pgfpathcurveto{\pgfqpoint{-0.018041in}{-0.031064in}}{\pgfqpoint{-0.009208in}{-0.034722in}}{\pgfqpoint{0.000000in}{-0.034722in}}%
\pgfpathlineto{\pgfqpoint{0.000000in}{-0.034722in}}%
\pgfpathclose%
\pgfusepath{stroke,fill}%
}%
\begin{pgfscope}%
\pgfsys@transformshift{0.533579in}{0.451576in}%
\pgfsys@useobject{currentmarker}{}%
\end{pgfscope}%
\begin{pgfscope}%
\pgfsys@transformshift{0.641540in}{0.451415in}%
\pgfsys@useobject{currentmarker}{}%
\end{pgfscope}%
\begin{pgfscope}%
\pgfsys@transformshift{0.821474in}{0.451400in}%
\pgfsys@useobject{currentmarker}{}%
\end{pgfscope}%
\begin{pgfscope}%
\pgfsys@transformshift{1.073383in}{0.451393in}%
\pgfsys@useobject{currentmarker}{}%
\end{pgfscope}%
\begin{pgfscope}%
\pgfsys@transformshift{1.397265in}{0.451391in}%
\pgfsys@useobject{currentmarker}{}%
\end{pgfscope}%
\begin{pgfscope}%
\pgfsys@transformshift{1.793120in}{0.451390in}%
\pgfsys@useobject{currentmarker}{}%
\end{pgfscope}%
\begin{pgfscope}%
\pgfsys@transformshift{2.260950in}{0.451390in}%
\pgfsys@useobject{currentmarker}{}%
\end{pgfscope}%
\begin{pgfscope}%
\pgfsys@transformshift{2.800753in}{0.451389in}%
\pgfsys@useobject{currentmarker}{}%
\end{pgfscope}%
\begin{pgfscope}%
\pgfsys@transformshift{3.412530in}{0.451389in}%
\pgfsys@useobject{currentmarker}{}%
\end{pgfscope}%
\begin{pgfscope}%
\pgfsys@transformshift{4.096281in}{0.451389in}%
\pgfsys@useobject{currentmarker}{}%
\end{pgfscope}%
\end{pgfscope}%
\end{pgfpicture}%
\makeatother%
\endgroup%

%					\end{figcenter}
%					\caption{Relative share of each task on the total import time.}
%				\end{subfigure}
			\end{figcenter}
			\\
			\begin{figcenter}
				\begin{subfigure}[t]{\textwidth}
					\begin{figcenter}
						%% Creator: Matplotlib, PGF backend
%%
%% To include the figure in your LaTeX document, write
%%   \input{<filename>.pgf}
%%
%% Make sure the required packages are loaded in your preamble
%%   \usepackage{pgf}
%%
%% Also ensure that all the required font packages are loaded; for instance,
%% the lmodern package is sometimes necessary when using math font.
%%   \usepackage{lmodern}
%%
%% Figures using additional raster images can only be included by \input if
%% they are in the same directory as the main LaTeX file. For loading figures
%% from other directories you can use the `import` package
%%   \usepackage{import}
%%
%% and then include the figures with
%%   \import{<path to file>}{<filename>.pgf}
%%
%% Matplotlib used the following preamble
%%   
%%   \usepackage{fontspec}
%%   \setmainfont{DejaVuSerif.ttf}[Path=\detokenize{/home/hauke/.local/lib/python3.11/site-packages/matplotlib/mpl-data/fonts/ttf/}]
%%   \setsansfont{DroidSans.ttf}[Path=\detokenize{/usr/share/fonts/droid/}]
%%   \setmonofont{DejaVuSansMono.ttf}[Path=\detokenize{/home/hauke/.local/lib/python3.11/site-packages/matplotlib/mpl-data/fonts/ttf/}]
%%   \makeatletter\@ifpackageloaded{underscore}{}{\usepackage[strings]{underscore}}\makeatother
%%
\begingroup%
\makeatletter%
\begin{pgfpicture}%
\pgfpathrectangle{\pgfpointorigin}{\pgfqpoint{5.697678in}{2.407400in}}%
\pgfusepath{use as bounding box, clip}%
\begin{pgfscope}%
\pgfsetbuttcap%
\pgfsetmiterjoin%
\definecolor{currentfill}{rgb}{1.000000,1.000000,1.000000}%
\pgfsetfillcolor{currentfill}%
\pgfsetlinewidth{0.000000pt}%
\definecolor{currentstroke}{rgb}{1.000000,1.000000,1.000000}%
\pgfsetstrokecolor{currentstroke}%
\pgfsetdash{}{0pt}%
\pgfpathmoveto{\pgfqpoint{0.000000in}{0.000000in}}%
\pgfpathlineto{\pgfqpoint{5.697678in}{0.000000in}}%
\pgfpathlineto{\pgfqpoint{5.697678in}{2.407400in}}%
\pgfpathlineto{\pgfqpoint{0.000000in}{2.407400in}}%
\pgfpathlineto{\pgfqpoint{0.000000in}{0.000000in}}%
\pgfpathclose%
\pgfusepath{fill}%
\end{pgfscope}%
\begin{pgfscope}%
\pgfsetbuttcap%
\pgfsetmiterjoin%
\definecolor{currentfill}{rgb}{1.000000,1.000000,1.000000}%
\pgfsetfillcolor{currentfill}%
\pgfsetlinewidth{0.000000pt}%
\definecolor{currentstroke}{rgb}{0.000000,0.000000,0.000000}%
\pgfsetstrokecolor{currentstroke}%
\pgfsetstrokeopacity{0.000000}%
\pgfsetdash{}{0pt}%
\pgfpathmoveto{\pgfqpoint{0.592976in}{0.451389in}}%
\pgfpathlineto{\pgfqpoint{4.297997in}{0.451389in}}%
\pgfpathlineto{\pgfqpoint{4.297997in}{2.407400in}}%
\pgfpathlineto{\pgfqpoint{0.592976in}{2.407400in}}%
\pgfpathlineto{\pgfqpoint{0.592976in}{0.451389in}}%
\pgfpathclose%
\pgfusepath{fill}%
\end{pgfscope}%
\begin{pgfscope}%
\pgfpathrectangle{\pgfqpoint{0.592976in}{0.451389in}}{\pgfqpoint{3.705021in}{1.956011in}}%
\pgfusepath{clip}%
\pgfsetroundcap%
\pgfsetroundjoin%
\pgfsetlinewidth{1.003750pt}%
\definecolor{currentstroke}{rgb}{0.800000,0.800000,0.800000}%
\pgfsetstrokecolor{currentstroke}%
\pgfsetdash{}{0pt}%
\pgfpathmoveto{\pgfqpoint{1.019317in}{0.451389in}}%
\pgfpathlineto{\pgfqpoint{1.019317in}{2.407400in}}%
\pgfusepath{stroke}%
\end{pgfscope}%
\begin{pgfscope}%
\definecolor{textcolor}{rgb}{0.150000,0.150000,0.150000}%
\pgfsetstrokecolor{textcolor}%
\pgfsetfillcolor{textcolor}%
\pgftext[x=1.019317in,y=0.319444in,,top]{\color{textcolor}\sffamily\fontsize{9.000000}{10.800000}\selectfont 10000}%
\end{pgfscope}%
\begin{pgfscope}%
\pgfpathrectangle{\pgfqpoint{0.592976in}{0.451389in}}{\pgfqpoint{3.705021in}{1.956011in}}%
\pgfusepath{clip}%
\pgfsetroundcap%
\pgfsetroundjoin%
\pgfsetlinewidth{1.003750pt}%
\definecolor{currentstroke}{rgb}{0.800000,0.800000,0.800000}%
\pgfsetstrokecolor{currentstroke}%
\pgfsetdash{}{0pt}%
\pgfpathmoveto{\pgfqpoint{1.463413in}{0.451389in}}%
\pgfpathlineto{\pgfqpoint{1.463413in}{2.407400in}}%
\pgfusepath{stroke}%
\end{pgfscope}%
\begin{pgfscope}%
\definecolor{textcolor}{rgb}{0.150000,0.150000,0.150000}%
\pgfsetstrokecolor{textcolor}%
\pgfsetfillcolor{textcolor}%
\pgftext[x=1.463413in,y=0.319444in,,top]{\color{textcolor}\sffamily\fontsize{9.000000}{10.800000}\selectfont 15000}%
\end{pgfscope}%
\begin{pgfscope}%
\pgfpathrectangle{\pgfqpoint{0.592976in}{0.451389in}}{\pgfqpoint{3.705021in}{1.956011in}}%
\pgfusepath{clip}%
\pgfsetroundcap%
\pgfsetroundjoin%
\pgfsetlinewidth{1.003750pt}%
\definecolor{currentstroke}{rgb}{0.800000,0.800000,0.800000}%
\pgfsetstrokecolor{currentstroke}%
\pgfsetdash{}{0pt}%
\pgfpathmoveto{\pgfqpoint{1.907509in}{0.451389in}}%
\pgfpathlineto{\pgfqpoint{1.907509in}{2.407400in}}%
\pgfusepath{stroke}%
\end{pgfscope}%
\begin{pgfscope}%
\definecolor{textcolor}{rgb}{0.150000,0.150000,0.150000}%
\pgfsetstrokecolor{textcolor}%
\pgfsetfillcolor{textcolor}%
\pgftext[x=1.907509in,y=0.319444in,,top]{\color{textcolor}\sffamily\fontsize{9.000000}{10.800000}\selectfont 20000}%
\end{pgfscope}%
\begin{pgfscope}%
\pgfpathrectangle{\pgfqpoint{0.592976in}{0.451389in}}{\pgfqpoint{3.705021in}{1.956011in}}%
\pgfusepath{clip}%
\pgfsetroundcap%
\pgfsetroundjoin%
\pgfsetlinewidth{1.003750pt}%
\definecolor{currentstroke}{rgb}{0.800000,0.800000,0.800000}%
\pgfsetstrokecolor{currentstroke}%
\pgfsetdash{}{0pt}%
\pgfpathmoveto{\pgfqpoint{2.351605in}{0.451389in}}%
\pgfpathlineto{\pgfqpoint{2.351605in}{2.407400in}}%
\pgfusepath{stroke}%
\end{pgfscope}%
\begin{pgfscope}%
\definecolor{textcolor}{rgb}{0.150000,0.150000,0.150000}%
\pgfsetstrokecolor{textcolor}%
\pgfsetfillcolor{textcolor}%
\pgftext[x=2.351605in,y=0.319444in,,top]{\color{textcolor}\sffamily\fontsize{9.000000}{10.800000}\selectfont 25000}%
\end{pgfscope}%
\begin{pgfscope}%
\pgfpathrectangle{\pgfqpoint{0.592976in}{0.451389in}}{\pgfqpoint{3.705021in}{1.956011in}}%
\pgfusepath{clip}%
\pgfsetroundcap%
\pgfsetroundjoin%
\pgfsetlinewidth{1.003750pt}%
\definecolor{currentstroke}{rgb}{0.800000,0.800000,0.800000}%
\pgfsetstrokecolor{currentstroke}%
\pgfsetdash{}{0pt}%
\pgfpathmoveto{\pgfqpoint{2.795701in}{0.451389in}}%
\pgfpathlineto{\pgfqpoint{2.795701in}{2.407400in}}%
\pgfusepath{stroke}%
\end{pgfscope}%
\begin{pgfscope}%
\definecolor{textcolor}{rgb}{0.150000,0.150000,0.150000}%
\pgfsetstrokecolor{textcolor}%
\pgfsetfillcolor{textcolor}%
\pgftext[x=2.795701in,y=0.319444in,,top]{\color{textcolor}\sffamily\fontsize{9.000000}{10.800000}\selectfont 30000}%
\end{pgfscope}%
\begin{pgfscope}%
\pgfpathrectangle{\pgfqpoint{0.592976in}{0.451389in}}{\pgfqpoint{3.705021in}{1.956011in}}%
\pgfusepath{clip}%
\pgfsetroundcap%
\pgfsetroundjoin%
\pgfsetlinewidth{1.003750pt}%
\definecolor{currentstroke}{rgb}{0.800000,0.800000,0.800000}%
\pgfsetstrokecolor{currentstroke}%
\pgfsetdash{}{0pt}%
\pgfpathmoveto{\pgfqpoint{3.239797in}{0.451389in}}%
\pgfpathlineto{\pgfqpoint{3.239797in}{2.407400in}}%
\pgfusepath{stroke}%
\end{pgfscope}%
\begin{pgfscope}%
\definecolor{textcolor}{rgb}{0.150000,0.150000,0.150000}%
\pgfsetstrokecolor{textcolor}%
\pgfsetfillcolor{textcolor}%
\pgftext[x=3.239797in,y=0.319444in,,top]{\color{textcolor}\sffamily\fontsize{9.000000}{10.800000}\selectfont 35000}%
\end{pgfscope}%
\begin{pgfscope}%
\pgfpathrectangle{\pgfqpoint{0.592976in}{0.451389in}}{\pgfqpoint{3.705021in}{1.956011in}}%
\pgfusepath{clip}%
\pgfsetroundcap%
\pgfsetroundjoin%
\pgfsetlinewidth{1.003750pt}%
\definecolor{currentstroke}{rgb}{0.800000,0.800000,0.800000}%
\pgfsetstrokecolor{currentstroke}%
\pgfsetdash{}{0pt}%
\pgfpathmoveto{\pgfqpoint{3.683892in}{0.451389in}}%
\pgfpathlineto{\pgfqpoint{3.683892in}{2.407400in}}%
\pgfusepath{stroke}%
\end{pgfscope}%
\begin{pgfscope}%
\definecolor{textcolor}{rgb}{0.150000,0.150000,0.150000}%
\pgfsetstrokecolor{textcolor}%
\pgfsetfillcolor{textcolor}%
\pgftext[x=3.683892in,y=0.319444in,,top]{\color{textcolor}\sffamily\fontsize{9.000000}{10.800000}\selectfont 40000}%
\end{pgfscope}%
\begin{pgfscope}%
\pgfpathrectangle{\pgfqpoint{0.592976in}{0.451389in}}{\pgfqpoint{3.705021in}{1.956011in}}%
\pgfusepath{clip}%
\pgfsetroundcap%
\pgfsetroundjoin%
\pgfsetlinewidth{1.003750pt}%
\definecolor{currentstroke}{rgb}{0.800000,0.800000,0.800000}%
\pgfsetstrokecolor{currentstroke}%
\pgfsetdash{}{0pt}%
\pgfpathmoveto{\pgfqpoint{4.127988in}{0.451389in}}%
\pgfpathlineto{\pgfqpoint{4.127988in}{2.407400in}}%
\pgfusepath{stroke}%
\end{pgfscope}%
\begin{pgfscope}%
\definecolor{textcolor}{rgb}{0.150000,0.150000,0.150000}%
\pgfsetstrokecolor{textcolor}%
\pgfsetfillcolor{textcolor}%
\pgftext[x=4.127988in,y=0.319444in,,top]{\color{textcolor}\sffamily\fontsize{9.000000}{10.800000}\selectfont 45000}%
\end{pgfscope}%
\begin{pgfscope}%
\definecolor{textcolor}{rgb}{0.150000,0.150000,0.150000}%
\pgfsetstrokecolor{textcolor}%
\pgfsetfillcolor{textcolor}%
\pgftext[x=2.445487in,y=0.125000in,,top]{\color{textcolor}\sffamily\fontsize{9.000000}{10.800000}\selectfont Input obstacle vertices}%
\end{pgfscope}%
\begin{pgfscope}%
\pgfpathrectangle{\pgfqpoint{0.592976in}{0.451389in}}{\pgfqpoint{3.705021in}{1.956011in}}%
\pgfusepath{clip}%
\pgfsetroundcap%
\pgfsetroundjoin%
\pgfsetlinewidth{1.003750pt}%
\definecolor{currentstroke}{rgb}{0.800000,0.800000,0.800000}%
\pgfsetstrokecolor{currentstroke}%
\pgfsetdash{}{0pt}%
\pgfpathmoveto{\pgfqpoint{0.592976in}{0.644245in}}%
\pgfpathlineto{\pgfqpoint{4.297997in}{0.644245in}}%
\pgfusepath{stroke}%
\end{pgfscope}%
\begin{pgfscope}%
\definecolor{textcolor}{rgb}{0.150000,0.150000,0.150000}%
\pgfsetstrokecolor{textcolor}%
\pgfsetfillcolor{textcolor}%
\pgftext[x=0.194444in, y=0.596759in, left, base]{\color{textcolor}\sffamily\fontsize{9.000000}{10.800000}\selectfont \(\displaystyle {10^{-2}}\)}%
\end{pgfscope}%
\begin{pgfscope}%
\pgfpathrectangle{\pgfqpoint{0.592976in}{0.451389in}}{\pgfqpoint{3.705021in}{1.956011in}}%
\pgfusepath{clip}%
\pgfsetroundcap%
\pgfsetroundjoin%
\pgfsetlinewidth{1.003750pt}%
\definecolor{currentstroke}{rgb}{0.800000,0.800000,0.800000}%
\pgfsetstrokecolor{currentstroke}%
\pgfsetdash{}{0pt}%
\pgfpathmoveto{\pgfqpoint{0.592976in}{0.966921in}}%
\pgfpathlineto{\pgfqpoint{4.297997in}{0.966921in}}%
\pgfusepath{stroke}%
\end{pgfscope}%
\begin{pgfscope}%
\definecolor{textcolor}{rgb}{0.150000,0.150000,0.150000}%
\pgfsetstrokecolor{textcolor}%
\pgfsetfillcolor{textcolor}%
\pgftext[x=0.194444in, y=0.919436in, left, base]{\color{textcolor}\sffamily\fontsize{9.000000}{10.800000}\selectfont \(\displaystyle {10^{-1}}\)}%
\end{pgfscope}%
\begin{pgfscope}%
\pgfpathrectangle{\pgfqpoint{0.592976in}{0.451389in}}{\pgfqpoint{3.705021in}{1.956011in}}%
\pgfusepath{clip}%
\pgfsetroundcap%
\pgfsetroundjoin%
\pgfsetlinewidth{1.003750pt}%
\definecolor{currentstroke}{rgb}{0.800000,0.800000,0.800000}%
\pgfsetstrokecolor{currentstroke}%
\pgfsetdash{}{0pt}%
\pgfpathmoveto{\pgfqpoint{0.592976in}{1.289598in}}%
\pgfpathlineto{\pgfqpoint{4.297997in}{1.289598in}}%
\pgfusepath{stroke}%
\end{pgfscope}%
\begin{pgfscope}%
\definecolor{textcolor}{rgb}{0.150000,0.150000,0.150000}%
\pgfsetstrokecolor{textcolor}%
\pgfsetfillcolor{textcolor}%
\pgftext[x=0.274690in, y=1.242113in, left, base]{\color{textcolor}\sffamily\fontsize{9.000000}{10.800000}\selectfont \(\displaystyle {10^{0}}\)}%
\end{pgfscope}%
\begin{pgfscope}%
\pgfpathrectangle{\pgfqpoint{0.592976in}{0.451389in}}{\pgfqpoint{3.705021in}{1.956011in}}%
\pgfusepath{clip}%
\pgfsetroundcap%
\pgfsetroundjoin%
\pgfsetlinewidth{1.003750pt}%
\definecolor{currentstroke}{rgb}{0.800000,0.800000,0.800000}%
\pgfsetstrokecolor{currentstroke}%
\pgfsetdash{}{0pt}%
\pgfpathmoveto{\pgfqpoint{0.592976in}{1.612275in}}%
\pgfpathlineto{\pgfqpoint{4.297997in}{1.612275in}}%
\pgfusepath{stroke}%
\end{pgfscope}%
\begin{pgfscope}%
\definecolor{textcolor}{rgb}{0.150000,0.150000,0.150000}%
\pgfsetstrokecolor{textcolor}%
\pgfsetfillcolor{textcolor}%
\pgftext[x=0.274690in, y=1.564790in, left, base]{\color{textcolor}\sffamily\fontsize{9.000000}{10.800000}\selectfont \(\displaystyle {10^{1}}\)}%
\end{pgfscope}%
\begin{pgfscope}%
\pgfpathrectangle{\pgfqpoint{0.592976in}{0.451389in}}{\pgfqpoint{3.705021in}{1.956011in}}%
\pgfusepath{clip}%
\pgfsetroundcap%
\pgfsetroundjoin%
\pgfsetlinewidth{1.003750pt}%
\definecolor{currentstroke}{rgb}{0.800000,0.800000,0.800000}%
\pgfsetstrokecolor{currentstroke}%
\pgfsetdash{}{0pt}%
\pgfpathmoveto{\pgfqpoint{0.592976in}{1.934952in}}%
\pgfpathlineto{\pgfqpoint{4.297997in}{1.934952in}}%
\pgfusepath{stroke}%
\end{pgfscope}%
\begin{pgfscope}%
\definecolor{textcolor}{rgb}{0.150000,0.150000,0.150000}%
\pgfsetstrokecolor{textcolor}%
\pgfsetfillcolor{textcolor}%
\pgftext[x=0.274690in, y=1.887467in, left, base]{\color{textcolor}\sffamily\fontsize{9.000000}{10.800000}\selectfont \(\displaystyle {10^{2}}\)}%
\end{pgfscope}%
\begin{pgfscope}%
\pgfpathrectangle{\pgfqpoint{0.592976in}{0.451389in}}{\pgfqpoint{3.705021in}{1.956011in}}%
\pgfusepath{clip}%
\pgfsetroundcap%
\pgfsetroundjoin%
\pgfsetlinewidth{1.003750pt}%
\definecolor{currentstroke}{rgb}{0.800000,0.800000,0.800000}%
\pgfsetstrokecolor{currentstroke}%
\pgfsetdash{}{0pt}%
\pgfpathmoveto{\pgfqpoint{0.592976in}{2.257629in}}%
\pgfpathlineto{\pgfqpoint{4.297997in}{2.257629in}}%
\pgfusepath{stroke}%
\end{pgfscope}%
\begin{pgfscope}%
\definecolor{textcolor}{rgb}{0.150000,0.150000,0.150000}%
\pgfsetstrokecolor{textcolor}%
\pgfsetfillcolor{textcolor}%
\pgftext[x=0.274690in, y=2.210144in, left, base]{\color{textcolor}\sffamily\fontsize{9.000000}{10.800000}\selectfont \(\displaystyle {10^{3}}\)}%
\end{pgfscope}%
\begin{pgfscope}%
\definecolor{textcolor}{rgb}{0.150000,0.150000,0.150000}%
\pgfsetstrokecolor{textcolor}%
\pgfsetfillcolor{textcolor}%
\pgftext[x=0.125000in,y=1.429394in,,bottom,rotate=90.000000]{\color{textcolor}\sffamily\fontsize{9.000000}{10.800000}\selectfont Time in s}%
\end{pgfscope}%
\begin{pgfscope}%
\pgfpathrectangle{\pgfqpoint{0.592976in}{0.451389in}}{\pgfqpoint{3.705021in}{1.956011in}}%
\pgfusepath{clip}%
\pgfsetbuttcap%
\pgfsetroundjoin%
\definecolor{currentfill}{rgb}{0.003922,0.450980,0.698039}%
\pgfsetfillcolor{currentfill}%
\pgfsetfillopacity{0.200000}%
\pgfsetlinewidth{1.003750pt}%
\definecolor{currentstroke}{rgb}{0.003922,0.450980,0.698039}%
\pgfsetstrokecolor{currentstroke}%
\pgfsetstrokeopacity{0.200000}%
\pgfsetdash{}{0pt}%
\pgfsys@defobject{currentmarker}{\pgfqpoint{0.761386in}{1.754091in}}{\pgfqpoint{4.129587in}{2.318490in}}{%
\pgfpathmoveto{\pgfqpoint{0.761386in}{1.754619in}}%
\pgfpathlineto{\pgfqpoint{0.761386in}{1.754091in}}%
\pgfpathlineto{\pgfqpoint{1.273162in}{1.926439in}}%
\pgfpathlineto{\pgfqpoint{1.720455in}{2.046433in}}%
\pgfpathlineto{\pgfqpoint{2.156291in}{2.132598in}}%
\pgfpathlineto{\pgfqpoint{3.169985in}{2.249715in}}%
\pgfpathlineto{\pgfqpoint{4.129587in}{2.316449in}}%
\pgfpathlineto{\pgfqpoint{4.129587in}{2.318490in}}%
\pgfpathlineto{\pgfqpoint{4.129587in}{2.318490in}}%
\pgfpathlineto{\pgfqpoint{3.169985in}{2.251777in}}%
\pgfpathlineto{\pgfqpoint{2.156291in}{2.134478in}}%
\pgfpathlineto{\pgfqpoint{1.720455in}{2.049240in}}%
\pgfpathlineto{\pgfqpoint{1.273162in}{1.930317in}}%
\pgfpathlineto{\pgfqpoint{0.761386in}{1.754619in}}%
\pgfpathlineto{\pgfqpoint{0.761386in}{1.754619in}}%
\pgfpathclose%
\pgfusepath{stroke,fill}%
}%
\begin{pgfscope}%
\pgfsys@transformshift{0.000000in}{0.000000in}%
\pgfsys@useobject{currentmarker}{}%
\end{pgfscope}%
\end{pgfscope}%
\begin{pgfscope}%
\pgfpathrectangle{\pgfqpoint{0.592976in}{0.451389in}}{\pgfqpoint{3.705021in}{1.956011in}}%
\pgfusepath{clip}%
\pgfsetbuttcap%
\pgfsetroundjoin%
\definecolor{currentfill}{rgb}{0.870588,0.560784,0.019608}%
\pgfsetfillcolor{currentfill}%
\pgfsetfillopacity{0.200000}%
\pgfsetlinewidth{1.003750pt}%
\definecolor{currentstroke}{rgb}{0.870588,0.560784,0.019608}%
\pgfsetstrokecolor{currentstroke}%
\pgfsetstrokeopacity{0.200000}%
\pgfsetdash{}{0pt}%
\pgfsys@defobject{currentmarker}{\pgfqpoint{0.761386in}{1.694946in}}{\pgfqpoint{4.129587in}{2.251360in}}{%
\pgfpathmoveto{\pgfqpoint{0.761386in}{1.696190in}}%
\pgfpathlineto{\pgfqpoint{0.761386in}{1.694946in}}%
\pgfpathlineto{\pgfqpoint{1.273162in}{1.868929in}}%
\pgfpathlineto{\pgfqpoint{1.720455in}{1.984345in}}%
\pgfpathlineto{\pgfqpoint{2.156291in}{2.067871in}}%
\pgfpathlineto{\pgfqpoint{3.169985in}{2.187680in}}%
\pgfpathlineto{\pgfqpoint{4.129587in}{2.250044in}}%
\pgfpathlineto{\pgfqpoint{4.129587in}{2.251360in}}%
\pgfpathlineto{\pgfqpoint{4.129587in}{2.251360in}}%
\pgfpathlineto{\pgfqpoint{3.169985in}{2.190306in}}%
\pgfpathlineto{\pgfqpoint{2.156291in}{2.070284in}}%
\pgfpathlineto{\pgfqpoint{1.720455in}{1.987618in}}%
\pgfpathlineto{\pgfqpoint{1.273162in}{1.874042in}}%
\pgfpathlineto{\pgfqpoint{0.761386in}{1.696190in}}%
\pgfpathlineto{\pgfqpoint{0.761386in}{1.696190in}}%
\pgfpathclose%
\pgfusepath{stroke,fill}%
}%
\begin{pgfscope}%
\pgfsys@transformshift{0.000000in}{0.000000in}%
\pgfsys@useobject{currentmarker}{}%
\end{pgfscope}%
\end{pgfscope}%
\begin{pgfscope}%
\pgfpathrectangle{\pgfqpoint{0.592976in}{0.451389in}}{\pgfqpoint{3.705021in}{1.956011in}}%
\pgfusepath{clip}%
\pgfsetbuttcap%
\pgfsetroundjoin%
\definecolor{currentfill}{rgb}{0.007843,0.619608,0.450980}%
\pgfsetfillcolor{currentfill}%
\pgfsetfillopacity{0.200000}%
\pgfsetlinewidth{1.003750pt}%
\definecolor{currentstroke}{rgb}{0.007843,0.619608,0.450980}%
\pgfsetstrokecolor{currentstroke}%
\pgfsetstrokeopacity{0.200000}%
\pgfsetdash{}{0pt}%
\pgfsys@defobject{currentmarker}{\pgfqpoint{0.761386in}{0.963073in}}{\pgfqpoint{4.129587in}{1.256810in}}{%
\pgfpathmoveto{\pgfqpoint{0.761386in}{0.975032in}}%
\pgfpathlineto{\pgfqpoint{0.761386in}{0.963073in}}%
\pgfpathlineto{\pgfqpoint{1.273162in}{1.060138in}}%
\pgfpathlineto{\pgfqpoint{1.720455in}{1.109000in}}%
\pgfpathlineto{\pgfqpoint{2.156291in}{1.146830in}}%
\pgfpathlineto{\pgfqpoint{3.169985in}{1.206855in}}%
\pgfpathlineto{\pgfqpoint{4.129587in}{1.252957in}}%
\pgfpathlineto{\pgfqpoint{4.129587in}{1.256810in}}%
\pgfpathlineto{\pgfqpoint{4.129587in}{1.256810in}}%
\pgfpathlineto{\pgfqpoint{3.169985in}{1.209741in}}%
\pgfpathlineto{\pgfqpoint{2.156291in}{1.149290in}}%
\pgfpathlineto{\pgfqpoint{1.720455in}{1.113438in}}%
\pgfpathlineto{\pgfqpoint{1.273162in}{1.062539in}}%
\pgfpathlineto{\pgfqpoint{0.761386in}{0.975032in}}%
\pgfpathlineto{\pgfqpoint{0.761386in}{0.975032in}}%
\pgfpathclose%
\pgfusepath{stroke,fill}%
}%
\begin{pgfscope}%
\pgfsys@transformshift{0.000000in}{0.000000in}%
\pgfsys@useobject{currentmarker}{}%
\end{pgfscope}%
\end{pgfscope}%
\begin{pgfscope}%
\pgfpathrectangle{\pgfqpoint{0.592976in}{0.451389in}}{\pgfqpoint{3.705021in}{1.956011in}}%
\pgfusepath{clip}%
\pgfsetbuttcap%
\pgfsetroundjoin%
\definecolor{currentfill}{rgb}{0.835294,0.368627,0.000000}%
\pgfsetfillcolor{currentfill}%
\pgfsetfillopacity{0.200000}%
\pgfsetlinewidth{1.003750pt}%
\definecolor{currentstroke}{rgb}{0.835294,0.368627,0.000000}%
\pgfsetstrokecolor{currentstroke}%
\pgfsetstrokeopacity{0.200000}%
\pgfsetdash{}{0pt}%
\pgfsys@defobject{currentmarker}{\pgfqpoint{0.761386in}{0.855215in}}{\pgfqpoint{4.129587in}{1.117644in}}{%
\pgfpathmoveto{\pgfqpoint{0.761386in}{0.882401in}}%
\pgfpathlineto{\pgfqpoint{0.761386in}{0.855215in}}%
\pgfpathlineto{\pgfqpoint{1.273162in}{0.930147in}}%
\pgfpathlineto{\pgfqpoint{1.720455in}{0.984135in}}%
\pgfpathlineto{\pgfqpoint{2.156291in}{1.008993in}}%
\pgfpathlineto{\pgfqpoint{3.169985in}{1.068481in}}%
\pgfpathlineto{\pgfqpoint{4.129587in}{1.107762in}}%
\pgfpathlineto{\pgfqpoint{4.129587in}{1.117644in}}%
\pgfpathlineto{\pgfqpoint{4.129587in}{1.117644in}}%
\pgfpathlineto{\pgfqpoint{3.169985in}{1.077992in}}%
\pgfpathlineto{\pgfqpoint{2.156291in}{1.023400in}}%
\pgfpathlineto{\pgfqpoint{1.720455in}{0.998796in}}%
\pgfpathlineto{\pgfqpoint{1.273162in}{0.949141in}}%
\pgfpathlineto{\pgfqpoint{0.761386in}{0.882401in}}%
\pgfpathlineto{\pgfqpoint{0.761386in}{0.882401in}}%
\pgfpathclose%
\pgfusepath{stroke,fill}%
}%
\begin{pgfscope}%
\pgfsys@transformshift{0.000000in}{0.000000in}%
\pgfsys@useobject{currentmarker}{}%
\end{pgfscope}%
\end{pgfscope}%
\begin{pgfscope}%
\pgfpathrectangle{\pgfqpoint{0.592976in}{0.451389in}}{\pgfqpoint{3.705021in}{1.956011in}}%
\pgfusepath{clip}%
\pgfsetbuttcap%
\pgfsetroundjoin%
\definecolor{currentfill}{rgb}{0.800000,0.470588,0.737255}%
\pgfsetfillcolor{currentfill}%
\pgfsetfillopacity{0.200000}%
\pgfsetlinewidth{1.003750pt}%
\definecolor{currentstroke}{rgb}{0.800000,0.470588,0.737255}%
\pgfsetstrokecolor{currentstroke}%
\pgfsetstrokeopacity{0.200000}%
\pgfsetdash{}{0pt}%
\pgfsys@defobject{currentmarker}{\pgfqpoint{0.761386in}{1.599559in}}{\pgfqpoint{4.129587in}{2.183831in}}{%
\pgfpathmoveto{\pgfqpoint{0.761386in}{1.601838in}}%
\pgfpathlineto{\pgfqpoint{0.761386in}{1.599559in}}%
\pgfpathlineto{\pgfqpoint{1.273162in}{1.771506in}}%
\pgfpathlineto{\pgfqpoint{1.720455in}{1.900887in}}%
\pgfpathlineto{\pgfqpoint{2.156291in}{1.991884in}}%
\pgfpathlineto{\pgfqpoint{3.169985in}{2.105118in}}%
\pgfpathlineto{\pgfqpoint{4.129587in}{2.179243in}}%
\pgfpathlineto{\pgfqpoint{4.129587in}{2.183831in}}%
\pgfpathlineto{\pgfqpoint{4.129587in}{2.183831in}}%
\pgfpathlineto{\pgfqpoint{3.169985in}{2.106364in}}%
\pgfpathlineto{\pgfqpoint{2.156291in}{1.993660in}}%
\pgfpathlineto{\pgfqpoint{1.720455in}{1.903842in}}%
\pgfpathlineto{\pgfqpoint{1.273162in}{1.774169in}}%
\pgfpathlineto{\pgfqpoint{0.761386in}{1.601838in}}%
\pgfpathlineto{\pgfqpoint{0.761386in}{1.601838in}}%
\pgfpathclose%
\pgfusepath{stroke,fill}%
}%
\begin{pgfscope}%
\pgfsys@transformshift{0.000000in}{0.000000in}%
\pgfsys@useobject{currentmarker}{}%
\end{pgfscope}%
\end{pgfscope}%
\begin{pgfscope}%
\pgfpathrectangle{\pgfqpoint{0.592976in}{0.451389in}}{\pgfqpoint{3.705021in}{1.956011in}}%
\pgfusepath{clip}%
\pgfsetbuttcap%
\pgfsetroundjoin%
\definecolor{currentfill}{rgb}{0.792157,0.568627,0.380392}%
\pgfsetfillcolor{currentfill}%
\pgfsetfillopacity{0.200000}%
\pgfsetlinewidth{1.003750pt}%
\definecolor{currentstroke}{rgb}{0.792157,0.568627,0.380392}%
\pgfsetstrokecolor{currentstroke}%
\pgfsetstrokeopacity{0.200000}%
\pgfsetdash{}{0pt}%
\pgfsys@defobject{currentmarker}{\pgfqpoint{0.761386in}{0.540298in}}{\pgfqpoint{4.129587in}{0.919634in}}{%
\pgfpathmoveto{\pgfqpoint{0.761386in}{0.626728in}}%
\pgfpathlineto{\pgfqpoint{0.761386in}{0.540298in}}%
\pgfpathlineto{\pgfqpoint{1.273162in}{0.684487in}}%
\pgfpathlineto{\pgfqpoint{1.720455in}{0.705066in}}%
\pgfpathlineto{\pgfqpoint{2.156291in}{0.738446in}}%
\pgfpathlineto{\pgfqpoint{3.169985in}{0.785244in}}%
\pgfpathlineto{\pgfqpoint{4.129587in}{0.917653in}}%
\pgfpathlineto{\pgfqpoint{4.129587in}{0.919634in}}%
\pgfpathlineto{\pgfqpoint{4.129587in}{0.919634in}}%
\pgfpathlineto{\pgfqpoint{3.169985in}{0.794153in}}%
\pgfpathlineto{\pgfqpoint{2.156291in}{0.792682in}}%
\pgfpathlineto{\pgfqpoint{1.720455in}{0.708029in}}%
\pgfpathlineto{\pgfqpoint{1.273162in}{0.690907in}}%
\pgfpathlineto{\pgfqpoint{0.761386in}{0.626728in}}%
\pgfpathlineto{\pgfqpoint{0.761386in}{0.626728in}}%
\pgfpathclose%
\pgfusepath{stroke,fill}%
}%
\begin{pgfscope}%
\pgfsys@transformshift{0.000000in}{0.000000in}%
\pgfsys@useobject{currentmarker}{}%
\end{pgfscope}%
\end{pgfscope}%
\begin{pgfscope}%
\pgfsetrectcap%
\pgfsetmiterjoin%
\pgfsetlinewidth{1.254687pt}%
\definecolor{currentstroke}{rgb}{0.800000,0.800000,0.800000}%
\pgfsetstrokecolor{currentstroke}%
\pgfsetdash{}{0pt}%
\pgfpathmoveto{\pgfqpoint{0.592976in}{0.451389in}}%
\pgfpathlineto{\pgfqpoint{0.592976in}{2.407400in}}%
\pgfusepath{stroke}%
\end{pgfscope}%
\begin{pgfscope}%
\pgfsetrectcap%
\pgfsetmiterjoin%
\pgfsetlinewidth{1.254687pt}%
\definecolor{currentstroke}{rgb}{0.800000,0.800000,0.800000}%
\pgfsetstrokecolor{currentstroke}%
\pgfsetdash{}{0pt}%
\pgfpathmoveto{\pgfqpoint{4.297997in}{0.451389in}}%
\pgfpathlineto{\pgfqpoint{4.297997in}{2.407400in}}%
\pgfusepath{stroke}%
\end{pgfscope}%
\begin{pgfscope}%
\pgfsetrectcap%
\pgfsetmiterjoin%
\pgfsetlinewidth{1.254687pt}%
\definecolor{currentstroke}{rgb}{0.800000,0.800000,0.800000}%
\pgfsetstrokecolor{currentstroke}%
\pgfsetdash{}{0pt}%
\pgfpathmoveto{\pgfqpoint{0.592976in}{0.451389in}}%
\pgfpathlineto{\pgfqpoint{4.297997in}{0.451389in}}%
\pgfusepath{stroke}%
\end{pgfscope}%
\begin{pgfscope}%
\pgfsetrectcap%
\pgfsetmiterjoin%
\pgfsetlinewidth{1.254687pt}%
\definecolor{currentstroke}{rgb}{0.800000,0.800000,0.800000}%
\pgfsetstrokecolor{currentstroke}%
\pgfsetdash{}{0pt}%
\pgfpathmoveto{\pgfqpoint{0.592976in}{2.407400in}}%
\pgfpathlineto{\pgfqpoint{4.297997in}{2.407400in}}%
\pgfusepath{stroke}%
\end{pgfscope}%
\begin{pgfscope}%
\pgfsetbuttcap%
\pgfsetmiterjoin%
\definecolor{currentfill}{rgb}{1.000000,1.000000,1.000000}%
\pgfsetfillcolor{currentfill}%
\pgfsetfillopacity{0.800000}%
\pgfsetlinewidth{1.003750pt}%
\definecolor{currentstroke}{rgb}{0.800000,0.800000,0.800000}%
\pgfsetstrokecolor{currentstroke}%
\pgfsetstrokeopacity{0.800000}%
\pgfsetdash{}{0pt}%
\pgfpathmoveto{\pgfqpoint{4.478123in}{0.538404in}}%
\pgfpathlineto{\pgfqpoint{5.672678in}{0.538404in}}%
\pgfpathquadraticcurveto{\pgfqpoint{5.697678in}{0.538404in}}{\pgfqpoint{5.697678in}{0.563404in}}%
\pgfpathlineto{\pgfqpoint{5.697678in}{2.295385in}}%
\pgfpathquadraticcurveto{\pgfqpoint{5.697678in}{2.320385in}}{\pgfqpoint{5.672678in}{2.320385in}}%
\pgfpathlineto{\pgfqpoint{4.478123in}{2.320385in}}%
\pgfpathquadraticcurveto{\pgfqpoint{4.453123in}{2.320385in}}{\pgfqpoint{4.453123in}{2.295385in}}%
\pgfpathlineto{\pgfqpoint{4.453123in}{0.563404in}}%
\pgfpathquadraticcurveto{\pgfqpoint{4.453123in}{0.538404in}}{\pgfqpoint{4.478123in}{0.538404in}}%
\pgfpathlineto{\pgfqpoint{4.478123in}{0.538404in}}%
\pgfpathclose%
\pgfusepath{stroke,fill}%
\end{pgfscope}%
\begin{pgfscope}%
\definecolor{textcolor}{rgb}{0.150000,0.150000,0.150000}%
\pgfsetstrokecolor{textcolor}%
\pgfsetfillcolor{textcolor}%
\pgftext[x=4.872001in,y=2.175414in,left,base]{\color{textcolor}\sffamily\fontsize{9.000000}{10.800000}\selectfont Legend}%
\end{pgfscope}%
\begin{pgfscope}%
\pgfsetroundcap%
\pgfsetroundjoin%
\pgfsetlinewidth{1.505625pt}%
\definecolor{currentstroke}{rgb}{0.003922,0.450980,0.698039}%
\pgfsetstrokecolor{currentstroke}%
\pgfsetdash{}{0pt}%
\pgfpathmoveto{\pgfqpoint{4.503123in}{2.031664in}}%
\pgfpathlineto{\pgfqpoint{4.628123in}{2.031664in}}%
\pgfpathlineto{\pgfqpoint{4.753123in}{2.031664in}}%
\pgfusepath{stroke}%
\end{pgfscope}%
\begin{pgfscope}%
\definecolor{textcolor}{rgb}{0.150000,0.150000,0.150000}%
\pgfsetstrokecolor{textcolor}%
\pgfsetfillcolor{textcolor}%
\pgftext[x=4.853123in,y=1.987914in,left,base]{\color{textcolor}\sffamily\fontsize{9.000000}{10.800000}\selectfont Total time}%
\end{pgfscope}%
\begin{pgfscope}%
\pgfsetroundcap%
\pgfsetroundjoin%
\pgfsetlinewidth{1.505625pt}%
\definecolor{currentstroke}{rgb}{0.870588,0.560784,0.019608}%
\pgfsetstrokecolor{currentstroke}%
\pgfsetdash{}{0pt}%
\pgfpathmoveto{\pgfqpoint{4.503123in}{1.844164in}}%
\pgfpathlineto{\pgfqpoint{4.628123in}{1.844164in}}%
\pgfpathlineto{\pgfqpoint{4.753123in}{1.844164in}}%
\pgfusepath{stroke}%
\end{pgfscope}%
\begin{pgfscope}%
\definecolor{textcolor}{rgb}{0.150000,0.150000,0.150000}%
\pgfsetstrokecolor{textcolor}%
\pgfsetfillcolor{textcolor}%
\pgftext[x=4.853123in,y=1.800414in,left,base]{\color{textcolor}\sffamily\fontsize{9.000000}{10.800000}\selectfont kNN search}%
\end{pgfscope}%
\begin{pgfscope}%
\pgfsetroundcap%
\pgfsetroundjoin%
\pgfsetlinewidth{1.505625pt}%
\definecolor{currentstroke}{rgb}{0.007843,0.619608,0.450980}%
\pgfsetstrokecolor{currentstroke}%
\pgfsetdash{}{0pt}%
\pgfpathmoveto{\pgfqpoint{4.503123in}{1.656664in}}%
\pgfpathlineto{\pgfqpoint{4.628123in}{1.656664in}}%
\pgfpathlineto{\pgfqpoint{4.753123in}{1.656664in}}%
\pgfusepath{stroke}%
\end{pgfscope}%
\begin{pgfscope}%
\definecolor{textcolor}{rgb}{0.150000,0.150000,0.150000}%
\pgfsetstrokecolor{textcolor}%
\pgfsetfillcolor{textcolor}%
\pgftext[x=4.853123in,y=1.612914in,left,base]{\color{textcolor}\sffamily\fontsize{9.000000}{10.800000}\selectfont Create graph}%
\end{pgfscope}%
\begin{pgfscope}%
\pgfsetroundcap%
\pgfsetroundjoin%
\pgfsetlinewidth{1.505625pt}%
\definecolor{currentstroke}{rgb}{0.835294,0.368627,0.000000}%
\pgfsetstrokecolor{currentstroke}%
\pgfsetdash{}{0pt}%
\pgfpathmoveto{\pgfqpoint{4.503123in}{1.382153in}}%
\pgfpathlineto{\pgfqpoint{4.628123in}{1.382153in}}%
\pgfpathlineto{\pgfqpoint{4.753123in}{1.382153in}}%
\pgfusepath{stroke}%
\end{pgfscope}%
\begin{pgfscope}%
\definecolor{textcolor}{rgb}{0.150000,0.150000,0.150000}%
\pgfsetstrokecolor{textcolor}%
\pgfsetfillcolor{textcolor}%
\pgftext[x=4.853123in, y=1.425415in, left, base]{\color{textcolor}\sffamily\fontsize{9.000000}{10.800000}\selectfont Get \& prepare}%
\end{pgfscope}%
\begin{pgfscope}%
\definecolor{textcolor}{rgb}{0.150000,0.150000,0.150000}%
\pgfsetstrokecolor{textcolor}%
\pgfsetfillcolor{textcolor}%
\pgftext[x=4.853123in, y=1.281421in, left, base]{\color{textcolor}\sffamily\fontsize{9.000000}{10.800000}\selectfont obstacles}%
\end{pgfscope}%
\begin{pgfscope}%
\pgfsetroundcap%
\pgfsetroundjoin%
\pgfsetlinewidth{1.505625pt}%
\definecolor{currentstroke}{rgb}{0.800000,0.470588,0.737255}%
\pgfsetstrokecolor{currentstroke}%
\pgfsetdash{}{0pt}%
\pgfpathmoveto{\pgfqpoint{4.503123in}{1.050659in}}%
\pgfpathlineto{\pgfqpoint{4.628123in}{1.050659in}}%
\pgfpathlineto{\pgfqpoint{4.753123in}{1.050659in}}%
\pgfusepath{stroke}%
\end{pgfscope}%
\begin{pgfscope}%
\definecolor{textcolor}{rgb}{0.150000,0.150000,0.150000}%
\pgfsetstrokecolor{textcolor}%
\pgfsetfillcolor{textcolor}%
\pgftext[x=4.853123in, y=1.093921in, left, base]{\color{textcolor}\sffamily\fontsize{9.000000}{10.800000}\selectfont Merge road}%
\end{pgfscope}%
\begin{pgfscope}%
\definecolor{textcolor}{rgb}{0.150000,0.150000,0.150000}%
\pgfsetstrokecolor{textcolor}%
\pgfsetfillcolor{textcolor}%
\pgftext[x=4.853123in, y=0.949927in, left, base]{\color{textcolor}\sffamily\fontsize{9.000000}{10.800000}\selectfont edges}%
\end{pgfscope}%
\begin{pgfscope}%
\pgfsetroundcap%
\pgfsetroundjoin%
\pgfsetlinewidth{1.505625pt}%
\definecolor{currentstroke}{rgb}{0.792157,0.568627,0.380392}%
\pgfsetstrokecolor{currentstroke}%
\pgfsetdash{}{0pt}%
\pgfpathmoveto{\pgfqpoint{4.503123in}{0.719165in}}%
\pgfpathlineto{\pgfqpoint{4.628123in}{0.719165in}}%
\pgfpathlineto{\pgfqpoint{4.753123in}{0.719165in}}%
\pgfusepath{stroke}%
\end{pgfscope}%
\begin{pgfscope}%
\definecolor{textcolor}{rgb}{0.150000,0.150000,0.150000}%
\pgfsetstrokecolor{textcolor}%
\pgfsetfillcolor{textcolor}%
\pgftext[x=4.853123in, y=0.762427in, left, base]{\color{textcolor}\sffamily\fontsize{9.000000}{10.800000}\selectfont Add POI}%
\end{pgfscope}%
\begin{pgfscope}%
\definecolor{textcolor}{rgb}{0.150000,0.150000,0.150000}%
\pgfsetstrokecolor{textcolor}%
\pgfsetfillcolor{textcolor}%
\pgftext[x=4.853123in, y=0.618433in, left, base]{\color{textcolor}\sffamily\fontsize{9.000000}{10.800000}\selectfont attributes}%
\end{pgfscope}%
\begin{pgfscope}%
\pgfsetroundcap%
\pgfsetroundjoin%
\pgfsetlinewidth{1.003750pt}%
\definecolor{currentstroke}{rgb}{0.003922,0.450980,0.698039}%
\pgfsetstrokecolor{currentstroke}%
\pgfsetdash{}{0pt}%
\pgfpathmoveto{\pgfqpoint{0.761386in}{1.754300in}}%
\pgfpathlineto{\pgfqpoint{1.273162in}{1.928456in}}%
\pgfpathlineto{\pgfqpoint{1.720455in}{2.048083in}}%
\pgfpathlineto{\pgfqpoint{2.156291in}{2.133620in}}%
\pgfpathlineto{\pgfqpoint{3.169985in}{2.250574in}}%
\pgfpathlineto{\pgfqpoint{4.129587in}{2.317319in}}%
\pgfusepath{stroke}%
\end{pgfscope}%
\begin{pgfscope}%
\pgfsetbuttcap%
\pgfsetroundjoin%
\definecolor{currentfill}{rgb}{0.003922,0.450980,0.698039}%
\pgfsetfillcolor{currentfill}%
\pgfsetlinewidth{0.752812pt}%
\definecolor{currentstroke}{rgb}{1.000000,1.000000,1.000000}%
\pgfsetstrokecolor{currentstroke}%
\pgfsetdash{}{0pt}%
\pgfsys@defobject{currentmarker}{\pgfqpoint{-0.034722in}{-0.034722in}}{\pgfqpoint{0.034722in}{0.034722in}}{%
\pgfpathmoveto{\pgfqpoint{0.000000in}{-0.034722in}}%
\pgfpathcurveto{\pgfqpoint{0.009208in}{-0.034722in}}{\pgfqpoint{0.018041in}{-0.031064in}}{\pgfqpoint{0.024552in}{-0.024552in}}%
\pgfpathcurveto{\pgfqpoint{0.031064in}{-0.018041in}}{\pgfqpoint{0.034722in}{-0.009208in}}{\pgfqpoint{0.034722in}{0.000000in}}%
\pgfpathcurveto{\pgfqpoint{0.034722in}{0.009208in}}{\pgfqpoint{0.031064in}{0.018041in}}{\pgfqpoint{0.024552in}{0.024552in}}%
\pgfpathcurveto{\pgfqpoint{0.018041in}{0.031064in}}{\pgfqpoint{0.009208in}{0.034722in}}{\pgfqpoint{0.000000in}{0.034722in}}%
\pgfpathcurveto{\pgfqpoint{-0.009208in}{0.034722in}}{\pgfqpoint{-0.018041in}{0.031064in}}{\pgfqpoint{-0.024552in}{0.024552in}}%
\pgfpathcurveto{\pgfqpoint{-0.031064in}{0.018041in}}{\pgfqpoint{-0.034722in}{0.009208in}}{\pgfqpoint{-0.034722in}{0.000000in}}%
\pgfpathcurveto{\pgfqpoint{-0.034722in}{-0.009208in}}{\pgfqpoint{-0.031064in}{-0.018041in}}{\pgfqpoint{-0.024552in}{-0.024552in}}%
\pgfpathcurveto{\pgfqpoint{-0.018041in}{-0.031064in}}{\pgfqpoint{-0.009208in}{-0.034722in}}{\pgfqpoint{0.000000in}{-0.034722in}}%
\pgfpathlineto{\pgfqpoint{0.000000in}{-0.034722in}}%
\pgfpathclose%
\pgfusepath{stroke,fill}%
}%
\begin{pgfscope}%
\pgfsys@transformshift{0.761386in}{1.754300in}%
\pgfsys@useobject{currentmarker}{}%
\end{pgfscope}%
\begin{pgfscope}%
\pgfsys@transformshift{1.273162in}{1.928456in}%
\pgfsys@useobject{currentmarker}{}%
\end{pgfscope}%
\begin{pgfscope}%
\pgfsys@transformshift{1.720455in}{2.048083in}%
\pgfsys@useobject{currentmarker}{}%
\end{pgfscope}%
\begin{pgfscope}%
\pgfsys@transformshift{2.156291in}{2.133620in}%
\pgfsys@useobject{currentmarker}{}%
\end{pgfscope}%
\begin{pgfscope}%
\pgfsys@transformshift{3.169985in}{2.250574in}%
\pgfsys@useobject{currentmarker}{}%
\end{pgfscope}%
\begin{pgfscope}%
\pgfsys@transformshift{4.129587in}{2.317319in}%
\pgfsys@useobject{currentmarker}{}%
\end{pgfscope}%
\end{pgfscope}%
\begin{pgfscope}%
\pgfsetroundcap%
\pgfsetroundjoin%
\pgfsetlinewidth{1.003750pt}%
\definecolor{currentstroke}{rgb}{0.870588,0.560784,0.019608}%
\pgfsetstrokecolor{currentstroke}%
\pgfsetdash{}{0pt}%
\pgfpathmoveto{\pgfqpoint{0.761386in}{1.695789in}}%
\pgfpathlineto{\pgfqpoint{1.273162in}{1.871511in}}%
\pgfpathlineto{\pgfqpoint{1.720455in}{1.986561in}}%
\pgfpathlineto{\pgfqpoint{2.156291in}{2.069337in}}%
\pgfpathlineto{\pgfqpoint{3.169985in}{2.188547in}}%
\pgfpathlineto{\pgfqpoint{4.129587in}{2.250417in}}%
\pgfusepath{stroke}%
\end{pgfscope}%
\begin{pgfscope}%
\pgfsetbuttcap%
\pgfsetroundjoin%
\definecolor{currentfill}{rgb}{0.870588,0.560784,0.019608}%
\pgfsetfillcolor{currentfill}%
\pgfsetlinewidth{0.752812pt}%
\definecolor{currentstroke}{rgb}{1.000000,1.000000,1.000000}%
\pgfsetstrokecolor{currentstroke}%
\pgfsetdash{}{0pt}%
\pgfsys@defobject{currentmarker}{\pgfqpoint{-0.034722in}{-0.034722in}}{\pgfqpoint{0.034722in}{0.034722in}}{%
\pgfpathmoveto{\pgfqpoint{0.000000in}{-0.034722in}}%
\pgfpathcurveto{\pgfqpoint{0.009208in}{-0.034722in}}{\pgfqpoint{0.018041in}{-0.031064in}}{\pgfqpoint{0.024552in}{-0.024552in}}%
\pgfpathcurveto{\pgfqpoint{0.031064in}{-0.018041in}}{\pgfqpoint{0.034722in}{-0.009208in}}{\pgfqpoint{0.034722in}{0.000000in}}%
\pgfpathcurveto{\pgfqpoint{0.034722in}{0.009208in}}{\pgfqpoint{0.031064in}{0.018041in}}{\pgfqpoint{0.024552in}{0.024552in}}%
\pgfpathcurveto{\pgfqpoint{0.018041in}{0.031064in}}{\pgfqpoint{0.009208in}{0.034722in}}{\pgfqpoint{0.000000in}{0.034722in}}%
\pgfpathcurveto{\pgfqpoint{-0.009208in}{0.034722in}}{\pgfqpoint{-0.018041in}{0.031064in}}{\pgfqpoint{-0.024552in}{0.024552in}}%
\pgfpathcurveto{\pgfqpoint{-0.031064in}{0.018041in}}{\pgfqpoint{-0.034722in}{0.009208in}}{\pgfqpoint{-0.034722in}{0.000000in}}%
\pgfpathcurveto{\pgfqpoint{-0.034722in}{-0.009208in}}{\pgfqpoint{-0.031064in}{-0.018041in}}{\pgfqpoint{-0.024552in}{-0.024552in}}%
\pgfpathcurveto{\pgfqpoint{-0.018041in}{-0.031064in}}{\pgfqpoint{-0.009208in}{-0.034722in}}{\pgfqpoint{0.000000in}{-0.034722in}}%
\pgfpathlineto{\pgfqpoint{0.000000in}{-0.034722in}}%
\pgfpathclose%
\pgfusepath{stroke,fill}%
}%
\begin{pgfscope}%
\pgfsys@transformshift{0.761386in}{1.695789in}%
\pgfsys@useobject{currentmarker}{}%
\end{pgfscope}%
\begin{pgfscope}%
\pgfsys@transformshift{1.273162in}{1.871511in}%
\pgfsys@useobject{currentmarker}{}%
\end{pgfscope}%
\begin{pgfscope}%
\pgfsys@transformshift{1.720455in}{1.986561in}%
\pgfsys@useobject{currentmarker}{}%
\end{pgfscope}%
\begin{pgfscope}%
\pgfsys@transformshift{2.156291in}{2.069337in}%
\pgfsys@useobject{currentmarker}{}%
\end{pgfscope}%
\begin{pgfscope}%
\pgfsys@transformshift{3.169985in}{2.188547in}%
\pgfsys@useobject{currentmarker}{}%
\end{pgfscope}%
\begin{pgfscope}%
\pgfsys@transformshift{4.129587in}{2.250417in}%
\pgfsys@useobject{currentmarker}{}%
\end{pgfscope}%
\end{pgfscope}%
\begin{pgfscope}%
\pgfsetroundcap%
\pgfsetroundjoin%
\pgfsetlinewidth{1.003750pt}%
\definecolor{currentstroke}{rgb}{0.007843,0.619608,0.450980}%
\pgfsetstrokecolor{currentstroke}%
\pgfsetdash{}{0pt}%
\pgfpathmoveto{\pgfqpoint{0.761386in}{0.968721in}}%
\pgfpathlineto{\pgfqpoint{1.273162in}{1.061540in}}%
\pgfpathlineto{\pgfqpoint{1.720455in}{1.111983in}}%
\pgfpathlineto{\pgfqpoint{2.156291in}{1.147880in}}%
\pgfpathlineto{\pgfqpoint{3.169985in}{1.208227in}}%
\pgfpathlineto{\pgfqpoint{4.129587in}{1.254891in}}%
\pgfusepath{stroke}%
\end{pgfscope}%
\begin{pgfscope}%
\pgfsetbuttcap%
\pgfsetroundjoin%
\definecolor{currentfill}{rgb}{0.007843,0.619608,0.450980}%
\pgfsetfillcolor{currentfill}%
\pgfsetlinewidth{0.752812pt}%
\definecolor{currentstroke}{rgb}{1.000000,1.000000,1.000000}%
\pgfsetstrokecolor{currentstroke}%
\pgfsetdash{}{0pt}%
\pgfsys@defobject{currentmarker}{\pgfqpoint{-0.034722in}{-0.034722in}}{\pgfqpoint{0.034722in}{0.034722in}}{%
\pgfpathmoveto{\pgfqpoint{0.000000in}{-0.034722in}}%
\pgfpathcurveto{\pgfqpoint{0.009208in}{-0.034722in}}{\pgfqpoint{0.018041in}{-0.031064in}}{\pgfqpoint{0.024552in}{-0.024552in}}%
\pgfpathcurveto{\pgfqpoint{0.031064in}{-0.018041in}}{\pgfqpoint{0.034722in}{-0.009208in}}{\pgfqpoint{0.034722in}{0.000000in}}%
\pgfpathcurveto{\pgfqpoint{0.034722in}{0.009208in}}{\pgfqpoint{0.031064in}{0.018041in}}{\pgfqpoint{0.024552in}{0.024552in}}%
\pgfpathcurveto{\pgfqpoint{0.018041in}{0.031064in}}{\pgfqpoint{0.009208in}{0.034722in}}{\pgfqpoint{0.000000in}{0.034722in}}%
\pgfpathcurveto{\pgfqpoint{-0.009208in}{0.034722in}}{\pgfqpoint{-0.018041in}{0.031064in}}{\pgfqpoint{-0.024552in}{0.024552in}}%
\pgfpathcurveto{\pgfqpoint{-0.031064in}{0.018041in}}{\pgfqpoint{-0.034722in}{0.009208in}}{\pgfqpoint{-0.034722in}{0.000000in}}%
\pgfpathcurveto{\pgfqpoint{-0.034722in}{-0.009208in}}{\pgfqpoint{-0.031064in}{-0.018041in}}{\pgfqpoint{-0.024552in}{-0.024552in}}%
\pgfpathcurveto{\pgfqpoint{-0.018041in}{-0.031064in}}{\pgfqpoint{-0.009208in}{-0.034722in}}{\pgfqpoint{0.000000in}{-0.034722in}}%
\pgfpathlineto{\pgfqpoint{0.000000in}{-0.034722in}}%
\pgfpathclose%
\pgfusepath{stroke,fill}%
}%
\begin{pgfscope}%
\pgfsys@transformshift{0.761386in}{0.968721in}%
\pgfsys@useobject{currentmarker}{}%
\end{pgfscope}%
\begin{pgfscope}%
\pgfsys@transformshift{1.273162in}{1.061540in}%
\pgfsys@useobject{currentmarker}{}%
\end{pgfscope}%
\begin{pgfscope}%
\pgfsys@transformshift{1.720455in}{1.111983in}%
\pgfsys@useobject{currentmarker}{}%
\end{pgfscope}%
\begin{pgfscope}%
\pgfsys@transformshift{2.156291in}{1.147880in}%
\pgfsys@useobject{currentmarker}{}%
\end{pgfscope}%
\begin{pgfscope}%
\pgfsys@transformshift{3.169985in}{1.208227in}%
\pgfsys@useobject{currentmarker}{}%
\end{pgfscope}%
\begin{pgfscope}%
\pgfsys@transformshift{4.129587in}{1.254891in}%
\pgfsys@useobject{currentmarker}{}%
\end{pgfscope}%
\end{pgfscope}%
\begin{pgfscope}%
\pgfsetroundcap%
\pgfsetroundjoin%
\pgfsetlinewidth{1.003750pt}%
\definecolor{currentstroke}{rgb}{0.835294,0.368627,0.000000}%
\pgfsetstrokecolor{currentstroke}%
\pgfsetdash{}{0pt}%
\pgfpathmoveto{\pgfqpoint{0.761386in}{0.863459in}}%
\pgfpathlineto{\pgfqpoint{1.273162in}{0.937231in}}%
\pgfpathlineto{\pgfqpoint{1.720455in}{0.989444in}}%
\pgfpathlineto{\pgfqpoint{2.156291in}{1.014798in}}%
\pgfpathlineto{\pgfqpoint{3.169985in}{1.072922in}}%
\pgfpathlineto{\pgfqpoint{4.129587in}{1.111729in}}%
\pgfusepath{stroke}%
\end{pgfscope}%
\begin{pgfscope}%
\pgfsetbuttcap%
\pgfsetroundjoin%
\definecolor{currentfill}{rgb}{0.835294,0.368627,0.000000}%
\pgfsetfillcolor{currentfill}%
\pgfsetlinewidth{0.752812pt}%
\definecolor{currentstroke}{rgb}{1.000000,1.000000,1.000000}%
\pgfsetstrokecolor{currentstroke}%
\pgfsetdash{}{0pt}%
\pgfsys@defobject{currentmarker}{\pgfqpoint{-0.034722in}{-0.034722in}}{\pgfqpoint{0.034722in}{0.034722in}}{%
\pgfpathmoveto{\pgfqpoint{0.000000in}{-0.034722in}}%
\pgfpathcurveto{\pgfqpoint{0.009208in}{-0.034722in}}{\pgfqpoint{0.018041in}{-0.031064in}}{\pgfqpoint{0.024552in}{-0.024552in}}%
\pgfpathcurveto{\pgfqpoint{0.031064in}{-0.018041in}}{\pgfqpoint{0.034722in}{-0.009208in}}{\pgfqpoint{0.034722in}{0.000000in}}%
\pgfpathcurveto{\pgfqpoint{0.034722in}{0.009208in}}{\pgfqpoint{0.031064in}{0.018041in}}{\pgfqpoint{0.024552in}{0.024552in}}%
\pgfpathcurveto{\pgfqpoint{0.018041in}{0.031064in}}{\pgfqpoint{0.009208in}{0.034722in}}{\pgfqpoint{0.000000in}{0.034722in}}%
\pgfpathcurveto{\pgfqpoint{-0.009208in}{0.034722in}}{\pgfqpoint{-0.018041in}{0.031064in}}{\pgfqpoint{-0.024552in}{0.024552in}}%
\pgfpathcurveto{\pgfqpoint{-0.031064in}{0.018041in}}{\pgfqpoint{-0.034722in}{0.009208in}}{\pgfqpoint{-0.034722in}{0.000000in}}%
\pgfpathcurveto{\pgfqpoint{-0.034722in}{-0.009208in}}{\pgfqpoint{-0.031064in}{-0.018041in}}{\pgfqpoint{-0.024552in}{-0.024552in}}%
\pgfpathcurveto{\pgfqpoint{-0.018041in}{-0.031064in}}{\pgfqpoint{-0.009208in}{-0.034722in}}{\pgfqpoint{0.000000in}{-0.034722in}}%
\pgfpathlineto{\pgfqpoint{0.000000in}{-0.034722in}}%
\pgfpathclose%
\pgfusepath{stroke,fill}%
}%
\begin{pgfscope}%
\pgfsys@transformshift{0.761386in}{0.863459in}%
\pgfsys@useobject{currentmarker}{}%
\end{pgfscope}%
\begin{pgfscope}%
\pgfsys@transformshift{1.273162in}{0.937231in}%
\pgfsys@useobject{currentmarker}{}%
\end{pgfscope}%
\begin{pgfscope}%
\pgfsys@transformshift{1.720455in}{0.989444in}%
\pgfsys@useobject{currentmarker}{}%
\end{pgfscope}%
\begin{pgfscope}%
\pgfsys@transformshift{2.156291in}{1.014798in}%
\pgfsys@useobject{currentmarker}{}%
\end{pgfscope}%
\begin{pgfscope}%
\pgfsys@transformshift{3.169985in}{1.072922in}%
\pgfsys@useobject{currentmarker}{}%
\end{pgfscope}%
\begin{pgfscope}%
\pgfsys@transformshift{4.129587in}{1.111729in}%
\pgfsys@useobject{currentmarker}{}%
\end{pgfscope}%
\end{pgfscope}%
\begin{pgfscope}%
\pgfsetroundcap%
\pgfsetroundjoin%
\pgfsetlinewidth{1.003750pt}%
\definecolor{currentstroke}{rgb}{0.800000,0.470588,0.737255}%
\pgfsetstrokecolor{currentstroke}%
\pgfsetdash{}{0pt}%
\pgfpathmoveto{\pgfqpoint{0.761386in}{1.600530in}}%
\pgfpathlineto{\pgfqpoint{1.273162in}{1.772979in}}%
\pgfpathlineto{\pgfqpoint{1.720455in}{1.902010in}}%
\pgfpathlineto{\pgfqpoint{2.156291in}{1.992756in}}%
\pgfpathlineto{\pgfqpoint{3.169985in}{2.105987in}}%
\pgfpathlineto{\pgfqpoint{4.129587in}{2.181174in}}%
\pgfusepath{stroke}%
\end{pgfscope}%
\begin{pgfscope}%
\pgfsetbuttcap%
\pgfsetroundjoin%
\definecolor{currentfill}{rgb}{0.800000,0.470588,0.737255}%
\pgfsetfillcolor{currentfill}%
\pgfsetlinewidth{0.752812pt}%
\definecolor{currentstroke}{rgb}{1.000000,1.000000,1.000000}%
\pgfsetstrokecolor{currentstroke}%
\pgfsetdash{}{0pt}%
\pgfsys@defobject{currentmarker}{\pgfqpoint{-0.034722in}{-0.034722in}}{\pgfqpoint{0.034722in}{0.034722in}}{%
\pgfpathmoveto{\pgfqpoint{0.000000in}{-0.034722in}}%
\pgfpathcurveto{\pgfqpoint{0.009208in}{-0.034722in}}{\pgfqpoint{0.018041in}{-0.031064in}}{\pgfqpoint{0.024552in}{-0.024552in}}%
\pgfpathcurveto{\pgfqpoint{0.031064in}{-0.018041in}}{\pgfqpoint{0.034722in}{-0.009208in}}{\pgfqpoint{0.034722in}{0.000000in}}%
\pgfpathcurveto{\pgfqpoint{0.034722in}{0.009208in}}{\pgfqpoint{0.031064in}{0.018041in}}{\pgfqpoint{0.024552in}{0.024552in}}%
\pgfpathcurveto{\pgfqpoint{0.018041in}{0.031064in}}{\pgfqpoint{0.009208in}{0.034722in}}{\pgfqpoint{0.000000in}{0.034722in}}%
\pgfpathcurveto{\pgfqpoint{-0.009208in}{0.034722in}}{\pgfqpoint{-0.018041in}{0.031064in}}{\pgfqpoint{-0.024552in}{0.024552in}}%
\pgfpathcurveto{\pgfqpoint{-0.031064in}{0.018041in}}{\pgfqpoint{-0.034722in}{0.009208in}}{\pgfqpoint{-0.034722in}{0.000000in}}%
\pgfpathcurveto{\pgfqpoint{-0.034722in}{-0.009208in}}{\pgfqpoint{-0.031064in}{-0.018041in}}{\pgfqpoint{-0.024552in}{-0.024552in}}%
\pgfpathcurveto{\pgfqpoint{-0.018041in}{-0.031064in}}{\pgfqpoint{-0.009208in}{-0.034722in}}{\pgfqpoint{0.000000in}{-0.034722in}}%
\pgfpathlineto{\pgfqpoint{0.000000in}{-0.034722in}}%
\pgfpathclose%
\pgfusepath{stroke,fill}%
}%
\begin{pgfscope}%
\pgfsys@transformshift{0.761386in}{1.600530in}%
\pgfsys@useobject{currentmarker}{}%
\end{pgfscope}%
\begin{pgfscope}%
\pgfsys@transformshift{1.273162in}{1.772979in}%
\pgfsys@useobject{currentmarker}{}%
\end{pgfscope}%
\begin{pgfscope}%
\pgfsys@transformshift{1.720455in}{1.902010in}%
\pgfsys@useobject{currentmarker}{}%
\end{pgfscope}%
\begin{pgfscope}%
\pgfsys@transformshift{2.156291in}{1.992756in}%
\pgfsys@useobject{currentmarker}{}%
\end{pgfscope}%
\begin{pgfscope}%
\pgfsys@transformshift{3.169985in}{2.105987in}%
\pgfsys@useobject{currentmarker}{}%
\end{pgfscope}%
\begin{pgfscope}%
\pgfsys@transformshift{4.129587in}{2.181174in}%
\pgfsys@useobject{currentmarker}{}%
\end{pgfscope}%
\end{pgfscope}%
\begin{pgfscope}%
\pgfsetroundcap%
\pgfsetroundjoin%
\pgfsetlinewidth{1.003750pt}%
\definecolor{currentstroke}{rgb}{0.792157,0.568627,0.380392}%
\pgfsetstrokecolor{currentstroke}%
\pgfsetdash{}{0pt}%
\pgfpathmoveto{\pgfqpoint{0.761386in}{0.575732in}}%
\pgfpathlineto{\pgfqpoint{1.273162in}{0.688864in}}%
\pgfpathlineto{\pgfqpoint{1.720455in}{0.706895in}}%
\pgfpathlineto{\pgfqpoint{2.156291in}{0.779042in}}%
\pgfpathlineto{\pgfqpoint{3.169985in}{0.789995in}}%
\pgfpathlineto{\pgfqpoint{4.129587in}{0.918804in}}%
\pgfusepath{stroke}%
\end{pgfscope}%
\begin{pgfscope}%
\pgfsetbuttcap%
\pgfsetroundjoin%
\definecolor{currentfill}{rgb}{0.792157,0.568627,0.380392}%
\pgfsetfillcolor{currentfill}%
\pgfsetlinewidth{0.752812pt}%
\definecolor{currentstroke}{rgb}{1.000000,1.000000,1.000000}%
\pgfsetstrokecolor{currentstroke}%
\pgfsetdash{}{0pt}%
\pgfsys@defobject{currentmarker}{\pgfqpoint{-0.034722in}{-0.034722in}}{\pgfqpoint{0.034722in}{0.034722in}}{%
\pgfpathmoveto{\pgfqpoint{0.000000in}{-0.034722in}}%
\pgfpathcurveto{\pgfqpoint{0.009208in}{-0.034722in}}{\pgfqpoint{0.018041in}{-0.031064in}}{\pgfqpoint{0.024552in}{-0.024552in}}%
\pgfpathcurveto{\pgfqpoint{0.031064in}{-0.018041in}}{\pgfqpoint{0.034722in}{-0.009208in}}{\pgfqpoint{0.034722in}{0.000000in}}%
\pgfpathcurveto{\pgfqpoint{0.034722in}{0.009208in}}{\pgfqpoint{0.031064in}{0.018041in}}{\pgfqpoint{0.024552in}{0.024552in}}%
\pgfpathcurveto{\pgfqpoint{0.018041in}{0.031064in}}{\pgfqpoint{0.009208in}{0.034722in}}{\pgfqpoint{0.000000in}{0.034722in}}%
\pgfpathcurveto{\pgfqpoint{-0.009208in}{0.034722in}}{\pgfqpoint{-0.018041in}{0.031064in}}{\pgfqpoint{-0.024552in}{0.024552in}}%
\pgfpathcurveto{\pgfqpoint{-0.031064in}{0.018041in}}{\pgfqpoint{-0.034722in}{0.009208in}}{\pgfqpoint{-0.034722in}{0.000000in}}%
\pgfpathcurveto{\pgfqpoint{-0.034722in}{-0.009208in}}{\pgfqpoint{-0.031064in}{-0.018041in}}{\pgfqpoint{-0.024552in}{-0.024552in}}%
\pgfpathcurveto{\pgfqpoint{-0.018041in}{-0.031064in}}{\pgfqpoint{-0.009208in}{-0.034722in}}{\pgfqpoint{0.000000in}{-0.034722in}}%
\pgfpathlineto{\pgfqpoint{0.000000in}{-0.034722in}}%
\pgfpathclose%
\pgfusepath{stroke,fill}%
}%
\begin{pgfscope}%
\pgfsys@transformshift{0.761386in}{0.575732in}%
\pgfsys@useobject{currentmarker}{}%
\end{pgfscope}%
\begin{pgfscope}%
\pgfsys@transformshift{1.273162in}{0.688864in}%
\pgfsys@useobject{currentmarker}{}%
\end{pgfscope}%
\begin{pgfscope}%
\pgfsys@transformshift{1.720455in}{0.706895in}%
\pgfsys@useobject{currentmarker}{}%
\end{pgfscope}%
\begin{pgfscope}%
\pgfsys@transformshift{2.156291in}{0.779042in}%
\pgfsys@useobject{currentmarker}{}%
\end{pgfscope}%
\begin{pgfscope}%
\pgfsys@transformshift{3.169985in}{0.789995in}%
\pgfsys@useobject{currentmarker}{}%
\end{pgfscope}%
\begin{pgfscope}%
\pgfsys@transformshift{4.129587in}{0.918804in}%
\pgfsys@useobject{currentmarker}{}%
\end{pgfscope}%
\end{pgfscope}%
\end{pgfpicture}%
\makeatother%
\endgroup%

					\end{figcenter}
					\caption{Import time of the \enquote{OSM rural} dataset by tasks.}
				\end{subfigure}
%				\hfill
%				\begin{subfigure}[t]{.48\textwidth}
%					\begin{figcenter}
%						%% Creator: Matplotlib, PGF backend
%%
%% To include the figure in your LaTeX document, write
%%   \input{<filename>.pgf}
%%
%% Make sure the required packages are loaded in your preamble
%%   \usepackage{pgf}
%%
%% Also ensure that all the required font packages are loaded; for instance,
%% the lmodern package is sometimes necessary when using math font.
%%   \usepackage{lmodern}
%%
%% Figures using additional raster images can only be included by \input if
%% they are in the same directory as the main LaTeX file. For loading figures
%% from other directories you can use the `import` package
%%   \usepackage{import}
%%
%% and then include the figures with
%%   \import{<path to file>}{<filename>.pgf}
%%
%% Matplotlib used the following preamble
%%   
%%   \usepackage{fontspec}
%%   \setmainfont{DejaVuSerif.ttf}[Path=\detokenize{/home/hauke/.local/lib/python3.11/site-packages/matplotlib/mpl-data/fonts/ttf/}]
%%   \setsansfont{DroidSans.ttf}[Path=\detokenize{/usr/share/fonts/droid/}]
%%   \setmonofont{DejaVuSansMono.ttf}[Path=\detokenize{/home/hauke/.local/lib/python3.11/site-packages/matplotlib/mpl-data/fonts/ttf/}]
%%   \makeatletter\@ifpackageloaded{underscore}{}{\usepackage[strings]{underscore}}\makeatother
%%
\begingroup%
\makeatletter%
\begin{pgfpicture}%
\pgfpathrectangle{\pgfpointorigin}{\pgfqpoint{5.675893in}{2.407400in}}%
\pgfusepath{use as bounding box, clip}%
\begin{pgfscope}%
\pgfsetbuttcap%
\pgfsetmiterjoin%
\definecolor{currentfill}{rgb}{1.000000,1.000000,1.000000}%
\pgfsetfillcolor{currentfill}%
\pgfsetlinewidth{0.000000pt}%
\definecolor{currentstroke}{rgb}{1.000000,1.000000,1.000000}%
\pgfsetstrokecolor{currentstroke}%
\pgfsetdash{}{0pt}%
\pgfpathmoveto{\pgfqpoint{0.000000in}{0.000000in}}%
\pgfpathlineto{\pgfqpoint{5.675893in}{0.000000in}}%
\pgfpathlineto{\pgfqpoint{5.675893in}{2.407400in}}%
\pgfpathlineto{\pgfqpoint{0.000000in}{2.407400in}}%
\pgfpathlineto{\pgfqpoint{0.000000in}{0.000000in}}%
\pgfpathclose%
\pgfusepath{fill}%
\end{pgfscope}%
\begin{pgfscope}%
\pgfsetbuttcap%
\pgfsetmiterjoin%
\definecolor{currentfill}{rgb}{1.000000,1.000000,1.000000}%
\pgfsetfillcolor{currentfill}%
\pgfsetlinewidth{0.000000pt}%
\definecolor{currentstroke}{rgb}{0.000000,0.000000,0.000000}%
\pgfsetstrokecolor{currentstroke}%
\pgfsetstrokeopacity{0.000000}%
\pgfsetdash{}{0pt}%
\pgfpathmoveto{\pgfqpoint{0.497592in}{0.451389in}}%
\pgfpathlineto{\pgfqpoint{4.274417in}{0.451389in}}%
\pgfpathlineto{\pgfqpoint{4.274417in}{2.407400in}}%
\pgfpathlineto{\pgfqpoint{0.497592in}{2.407400in}}%
\pgfpathlineto{\pgfqpoint{0.497592in}{0.451389in}}%
\pgfpathclose%
\pgfusepath{fill}%
\end{pgfscope}%
\begin{pgfscope}%
\pgfpathrectangle{\pgfqpoint{0.497592in}{0.451389in}}{\pgfqpoint{3.776824in}{1.956011in}}%
\pgfusepath{clip}%
\pgfsetroundcap%
\pgfsetroundjoin%
\pgfsetlinewidth{1.003750pt}%
\definecolor{currentstroke}{rgb}{0.800000,0.800000,0.800000}%
\pgfsetstrokecolor{currentstroke}%
\pgfsetdash{}{0pt}%
\pgfpathmoveto{\pgfqpoint{0.497592in}{0.451389in}}%
\pgfpathlineto{\pgfqpoint{0.497592in}{2.407400in}}%
\pgfusepath{stroke}%
\end{pgfscope}%
\begin{pgfscope}%
\definecolor{textcolor}{rgb}{0.150000,0.150000,0.150000}%
\pgfsetstrokecolor{textcolor}%
\pgfsetfillcolor{textcolor}%
\pgftext[x=0.497592in,y=0.319444in,,top]{\color{textcolor}\sffamily\fontsize{9.000000}{10.800000}\selectfont 0}%
\end{pgfscope}%
\begin{pgfscope}%
\pgfpathrectangle{\pgfqpoint{0.497592in}{0.451389in}}{\pgfqpoint{3.776824in}{1.956011in}}%
\pgfusepath{clip}%
\pgfsetroundcap%
\pgfsetroundjoin%
\pgfsetlinewidth{1.003750pt}%
\definecolor{currentstroke}{rgb}{0.800000,0.800000,0.800000}%
\pgfsetstrokecolor{currentstroke}%
\pgfsetdash{}{0pt}%
\pgfpathmoveto{\pgfqpoint{1.308108in}{0.451389in}}%
\pgfpathlineto{\pgfqpoint{1.308108in}{2.407400in}}%
\pgfusepath{stroke}%
\end{pgfscope}%
\begin{pgfscope}%
\definecolor{textcolor}{rgb}{0.150000,0.150000,0.150000}%
\pgfsetstrokecolor{textcolor}%
\pgfsetfillcolor{textcolor}%
\pgftext[x=1.308108in,y=0.319444in,,top]{\color{textcolor}\sffamily\fontsize{9.000000}{10.800000}\selectfont 10000}%
\end{pgfscope}%
\begin{pgfscope}%
\pgfpathrectangle{\pgfqpoint{0.497592in}{0.451389in}}{\pgfqpoint{3.776824in}{1.956011in}}%
\pgfusepath{clip}%
\pgfsetroundcap%
\pgfsetroundjoin%
\pgfsetlinewidth{1.003750pt}%
\definecolor{currentstroke}{rgb}{0.800000,0.800000,0.800000}%
\pgfsetstrokecolor{currentstroke}%
\pgfsetdash{}{0pt}%
\pgfpathmoveto{\pgfqpoint{2.118623in}{0.451389in}}%
\pgfpathlineto{\pgfqpoint{2.118623in}{2.407400in}}%
\pgfusepath{stroke}%
\end{pgfscope}%
\begin{pgfscope}%
\definecolor{textcolor}{rgb}{0.150000,0.150000,0.150000}%
\pgfsetstrokecolor{textcolor}%
\pgfsetfillcolor{textcolor}%
\pgftext[x=2.118623in,y=0.319444in,,top]{\color{textcolor}\sffamily\fontsize{9.000000}{10.800000}\selectfont 20000}%
\end{pgfscope}%
\begin{pgfscope}%
\pgfpathrectangle{\pgfqpoint{0.497592in}{0.451389in}}{\pgfqpoint{3.776824in}{1.956011in}}%
\pgfusepath{clip}%
\pgfsetroundcap%
\pgfsetroundjoin%
\pgfsetlinewidth{1.003750pt}%
\definecolor{currentstroke}{rgb}{0.800000,0.800000,0.800000}%
\pgfsetstrokecolor{currentstroke}%
\pgfsetdash{}{0pt}%
\pgfpathmoveto{\pgfqpoint{2.929139in}{0.451389in}}%
\pgfpathlineto{\pgfqpoint{2.929139in}{2.407400in}}%
\pgfusepath{stroke}%
\end{pgfscope}%
\begin{pgfscope}%
\definecolor{textcolor}{rgb}{0.150000,0.150000,0.150000}%
\pgfsetstrokecolor{textcolor}%
\pgfsetfillcolor{textcolor}%
\pgftext[x=2.929139in,y=0.319444in,,top]{\color{textcolor}\sffamily\fontsize{9.000000}{10.800000}\selectfont 30000}%
\end{pgfscope}%
\begin{pgfscope}%
\pgfpathrectangle{\pgfqpoint{0.497592in}{0.451389in}}{\pgfqpoint{3.776824in}{1.956011in}}%
\pgfusepath{clip}%
\pgfsetroundcap%
\pgfsetroundjoin%
\pgfsetlinewidth{1.003750pt}%
\definecolor{currentstroke}{rgb}{0.800000,0.800000,0.800000}%
\pgfsetstrokecolor{currentstroke}%
\pgfsetdash{}{0pt}%
\pgfpathmoveto{\pgfqpoint{3.739655in}{0.451389in}}%
\pgfpathlineto{\pgfqpoint{3.739655in}{2.407400in}}%
\pgfusepath{stroke}%
\end{pgfscope}%
\begin{pgfscope}%
\definecolor{textcolor}{rgb}{0.150000,0.150000,0.150000}%
\pgfsetstrokecolor{textcolor}%
\pgfsetfillcolor{textcolor}%
\pgftext[x=3.739655in,y=0.319444in,,top]{\color{textcolor}\sffamily\fontsize{9.000000}{10.800000}\selectfont 40000}%
\end{pgfscope}%
\begin{pgfscope}%
\definecolor{textcolor}{rgb}{0.150000,0.150000,0.150000}%
\pgfsetstrokecolor{textcolor}%
\pgfsetfillcolor{textcolor}%
\pgftext[x=2.386004in,y=0.125000in,,top]{\color{textcolor}\sffamily\fontsize{9.000000}{10.800000}\selectfont Input obstacle vertices}%
\end{pgfscope}%
\begin{pgfscope}%
\pgfpathrectangle{\pgfqpoint{0.497592in}{0.451389in}}{\pgfqpoint{3.776824in}{1.956011in}}%
\pgfusepath{clip}%
\pgfsetroundcap%
\pgfsetroundjoin%
\pgfsetlinewidth{1.003750pt}%
\definecolor{currentstroke}{rgb}{0.800000,0.800000,0.800000}%
\pgfsetstrokecolor{currentstroke}%
\pgfsetdash{}{0pt}%
\pgfpathmoveto{\pgfqpoint{0.497592in}{0.451389in}}%
\pgfpathlineto{\pgfqpoint{4.274417in}{0.451389in}}%
\pgfusepath{stroke}%
\end{pgfscope}%
\begin{pgfscope}%
\definecolor{textcolor}{rgb}{0.150000,0.150000,0.150000}%
\pgfsetstrokecolor{textcolor}%
\pgfsetfillcolor{textcolor}%
\pgftext[x=0.194444in, y=0.403903in, left, base]{\color{textcolor}\sffamily\fontsize{9.000000}{10.800000}\selectfont 0.0}%
\end{pgfscope}%
\begin{pgfscope}%
\pgfpathrectangle{\pgfqpoint{0.497592in}{0.451389in}}{\pgfqpoint{3.776824in}{1.956011in}}%
\pgfusepath{clip}%
\pgfsetroundcap%
\pgfsetroundjoin%
\pgfsetlinewidth{1.003750pt}%
\definecolor{currentstroke}{rgb}{0.800000,0.800000,0.800000}%
\pgfsetstrokecolor{currentstroke}%
\pgfsetdash{}{0pt}%
\pgfpathmoveto{\pgfqpoint{0.497592in}{0.823962in}}%
\pgfpathlineto{\pgfqpoint{4.274417in}{0.823962in}}%
\pgfusepath{stroke}%
\end{pgfscope}%
\begin{pgfscope}%
\definecolor{textcolor}{rgb}{0.150000,0.150000,0.150000}%
\pgfsetstrokecolor{textcolor}%
\pgfsetfillcolor{textcolor}%
\pgftext[x=0.194444in, y=0.776477in, left, base]{\color{textcolor}\sffamily\fontsize{9.000000}{10.800000}\selectfont 0.2}%
\end{pgfscope}%
\begin{pgfscope}%
\pgfpathrectangle{\pgfqpoint{0.497592in}{0.451389in}}{\pgfqpoint{3.776824in}{1.956011in}}%
\pgfusepath{clip}%
\pgfsetroundcap%
\pgfsetroundjoin%
\pgfsetlinewidth{1.003750pt}%
\definecolor{currentstroke}{rgb}{0.800000,0.800000,0.800000}%
\pgfsetstrokecolor{currentstroke}%
\pgfsetdash{}{0pt}%
\pgfpathmoveto{\pgfqpoint{0.497592in}{1.196536in}}%
\pgfpathlineto{\pgfqpoint{4.274417in}{1.196536in}}%
\pgfusepath{stroke}%
\end{pgfscope}%
\begin{pgfscope}%
\definecolor{textcolor}{rgb}{0.150000,0.150000,0.150000}%
\pgfsetstrokecolor{textcolor}%
\pgfsetfillcolor{textcolor}%
\pgftext[x=0.194444in, y=1.149050in, left, base]{\color{textcolor}\sffamily\fontsize{9.000000}{10.800000}\selectfont 0.4}%
\end{pgfscope}%
\begin{pgfscope}%
\pgfpathrectangle{\pgfqpoint{0.497592in}{0.451389in}}{\pgfqpoint{3.776824in}{1.956011in}}%
\pgfusepath{clip}%
\pgfsetroundcap%
\pgfsetroundjoin%
\pgfsetlinewidth{1.003750pt}%
\definecolor{currentstroke}{rgb}{0.800000,0.800000,0.800000}%
\pgfsetstrokecolor{currentstroke}%
\pgfsetdash{}{0pt}%
\pgfpathmoveto{\pgfqpoint{0.497592in}{1.569109in}}%
\pgfpathlineto{\pgfqpoint{4.274417in}{1.569109in}}%
\pgfusepath{stroke}%
\end{pgfscope}%
\begin{pgfscope}%
\definecolor{textcolor}{rgb}{0.150000,0.150000,0.150000}%
\pgfsetstrokecolor{textcolor}%
\pgfsetfillcolor{textcolor}%
\pgftext[x=0.194444in, y=1.521624in, left, base]{\color{textcolor}\sffamily\fontsize{9.000000}{10.800000}\selectfont 0.6}%
\end{pgfscope}%
\begin{pgfscope}%
\pgfpathrectangle{\pgfqpoint{0.497592in}{0.451389in}}{\pgfqpoint{3.776824in}{1.956011in}}%
\pgfusepath{clip}%
\pgfsetroundcap%
\pgfsetroundjoin%
\pgfsetlinewidth{1.003750pt}%
\definecolor{currentstroke}{rgb}{0.800000,0.800000,0.800000}%
\pgfsetstrokecolor{currentstroke}%
\pgfsetdash{}{0pt}%
\pgfpathmoveto{\pgfqpoint{0.497592in}{1.941683in}}%
\pgfpathlineto{\pgfqpoint{4.274417in}{1.941683in}}%
\pgfusepath{stroke}%
\end{pgfscope}%
\begin{pgfscope}%
\definecolor{textcolor}{rgb}{0.150000,0.150000,0.150000}%
\pgfsetstrokecolor{textcolor}%
\pgfsetfillcolor{textcolor}%
\pgftext[x=0.194444in, y=1.894197in, left, base]{\color{textcolor}\sffamily\fontsize{9.000000}{10.800000}\selectfont 0.8}%
\end{pgfscope}%
\begin{pgfscope}%
\pgfpathrectangle{\pgfqpoint{0.497592in}{0.451389in}}{\pgfqpoint{3.776824in}{1.956011in}}%
\pgfusepath{clip}%
\pgfsetroundcap%
\pgfsetroundjoin%
\pgfsetlinewidth{1.003750pt}%
\definecolor{currentstroke}{rgb}{0.800000,0.800000,0.800000}%
\pgfsetstrokecolor{currentstroke}%
\pgfsetdash{}{0pt}%
\pgfpathmoveto{\pgfqpoint{0.497592in}{2.314256in}}%
\pgfpathlineto{\pgfqpoint{4.274417in}{2.314256in}}%
\pgfusepath{stroke}%
\end{pgfscope}%
\begin{pgfscope}%
\definecolor{textcolor}{rgb}{0.150000,0.150000,0.150000}%
\pgfsetstrokecolor{textcolor}%
\pgfsetfillcolor{textcolor}%
\pgftext[x=0.194444in, y=2.266771in, left, base]{\color{textcolor}\sffamily\fontsize{9.000000}{10.800000}\selectfont 1.0}%
\end{pgfscope}%
\begin{pgfscope}%
\definecolor{textcolor}{rgb}{0.150000,0.150000,0.150000}%
\pgfsetstrokecolor{textcolor}%
\pgfsetfillcolor{textcolor}%
\pgftext[x=0.125000in,y=1.429394in,,bottom,rotate=90.000000]{\color{textcolor}\sffamily\fontsize{9.000000}{10.800000}\selectfont Share of total time}%
\end{pgfscope}%
\begin{pgfscope}%
\pgfpathrectangle{\pgfqpoint{0.497592in}{0.451389in}}{\pgfqpoint{3.776824in}{1.956011in}}%
\pgfusepath{clip}%
\pgfsetbuttcap%
\pgfsetroundjoin%
\definecolor{currentfill}{rgb}{0.003922,0.450980,0.698039}%
\pgfsetfillcolor{currentfill}%
\pgfsetfillopacity{0.200000}%
\pgfsetlinewidth{1.003750pt}%
\definecolor{currentstroke}{rgb}{0.003922,0.450980,0.698039}%
\pgfsetstrokecolor{currentstroke}%
\pgfsetstrokeopacity{0.200000}%
\pgfsetdash{}{0pt}%
\pgfsys@defobject{currentmarker}{\pgfqpoint{0.533579in}{2.314256in}}{\pgfqpoint{4.096281in}{2.314256in}}{%
\pgfpathmoveto{\pgfqpoint{0.533579in}{2.314256in}}%
\pgfpathlineto{\pgfqpoint{0.533579in}{2.314256in}}%
\pgfpathlineto{\pgfqpoint{0.641540in}{2.314256in}}%
\pgfpathlineto{\pgfqpoint{0.821474in}{2.314256in}}%
\pgfpathlineto{\pgfqpoint{1.073383in}{2.314256in}}%
\pgfpathlineto{\pgfqpoint{1.397265in}{2.314256in}}%
\pgfpathlineto{\pgfqpoint{1.793120in}{2.314256in}}%
\pgfpathlineto{\pgfqpoint{2.260950in}{2.314256in}}%
\pgfpathlineto{\pgfqpoint{2.800753in}{2.314256in}}%
\pgfpathlineto{\pgfqpoint{3.412530in}{2.314256in}}%
\pgfpathlineto{\pgfqpoint{4.096281in}{2.314256in}}%
\pgfpathlineto{\pgfqpoint{4.096281in}{2.314256in}}%
\pgfpathlineto{\pgfqpoint{4.096281in}{2.314256in}}%
\pgfpathlineto{\pgfqpoint{3.412530in}{2.314256in}}%
\pgfpathlineto{\pgfqpoint{2.800753in}{2.314256in}}%
\pgfpathlineto{\pgfqpoint{2.260950in}{2.314256in}}%
\pgfpathlineto{\pgfqpoint{1.793120in}{2.314256in}}%
\pgfpathlineto{\pgfqpoint{1.397265in}{2.314256in}}%
\pgfpathlineto{\pgfqpoint{1.073383in}{2.314256in}}%
\pgfpathlineto{\pgfqpoint{0.821474in}{2.314256in}}%
\pgfpathlineto{\pgfqpoint{0.641540in}{2.314256in}}%
\pgfpathlineto{\pgfqpoint{0.533579in}{2.314256in}}%
\pgfpathlineto{\pgfqpoint{0.533579in}{2.314256in}}%
\pgfpathclose%
\pgfusepath{stroke,fill}%
}%
\begin{pgfscope}%
\pgfsys@transformshift{0.000000in}{0.000000in}%
\pgfsys@useobject{currentmarker}{}%
\end{pgfscope}%
\end{pgfscope}%
\begin{pgfscope}%
\pgfpathrectangle{\pgfqpoint{0.497592in}{0.451389in}}{\pgfqpoint{3.776824in}{1.956011in}}%
\pgfusepath{clip}%
\pgfsetbuttcap%
\pgfsetroundjoin%
\definecolor{currentfill}{rgb}{0.870588,0.560784,0.019608}%
\pgfsetfillcolor{currentfill}%
\pgfsetfillopacity{0.200000}%
\pgfsetlinewidth{1.003750pt}%
\definecolor{currentstroke}{rgb}{0.870588,0.560784,0.019608}%
\pgfsetstrokecolor{currentstroke}%
\pgfsetstrokeopacity{0.200000}%
\pgfsetdash{}{0pt}%
\pgfsys@defobject{currentmarker}{\pgfqpoint{0.533579in}{2.056794in}}{\pgfqpoint{4.096281in}{2.311127in}}{%
\pgfpathmoveto{\pgfqpoint{0.533579in}{2.080290in}}%
\pgfpathlineto{\pgfqpoint{0.533579in}{2.056794in}}%
\pgfpathlineto{\pgfqpoint{0.641540in}{2.226562in}}%
\pgfpathlineto{\pgfqpoint{0.821474in}{2.267800in}}%
\pgfpathlineto{\pgfqpoint{1.073383in}{2.286772in}}%
\pgfpathlineto{\pgfqpoint{1.397265in}{2.296163in}}%
\pgfpathlineto{\pgfqpoint{1.793120in}{2.301990in}}%
\pgfpathlineto{\pgfqpoint{2.260950in}{2.306006in}}%
\pgfpathlineto{\pgfqpoint{2.800753in}{2.309213in}}%
\pgfpathlineto{\pgfqpoint{3.412530in}{2.310233in}}%
\pgfpathlineto{\pgfqpoint{4.096281in}{2.311071in}}%
\pgfpathlineto{\pgfqpoint{4.096281in}{2.311127in}}%
\pgfpathlineto{\pgfqpoint{4.096281in}{2.311127in}}%
\pgfpathlineto{\pgfqpoint{3.412530in}{2.310300in}}%
\pgfpathlineto{\pgfqpoint{2.800753in}{2.309282in}}%
\pgfpathlineto{\pgfqpoint{2.260950in}{2.306195in}}%
\pgfpathlineto{\pgfqpoint{1.793120in}{2.303049in}}%
\pgfpathlineto{\pgfqpoint{1.397265in}{2.298060in}}%
\pgfpathlineto{\pgfqpoint{1.073383in}{2.288914in}}%
\pgfpathlineto{\pgfqpoint{0.821474in}{2.271300in}}%
\pgfpathlineto{\pgfqpoint{0.641540in}{2.230016in}}%
\pgfpathlineto{\pgfqpoint{0.533579in}{2.080290in}}%
\pgfpathlineto{\pgfqpoint{0.533579in}{2.080290in}}%
\pgfpathclose%
\pgfusepath{stroke,fill}%
}%
\begin{pgfscope}%
\pgfsys@transformshift{0.000000in}{0.000000in}%
\pgfsys@useobject{currentmarker}{}%
\end{pgfscope}%
\end{pgfscope}%
\begin{pgfscope}%
\pgfpathrectangle{\pgfqpoint{0.497592in}{0.451389in}}{\pgfqpoint{3.776824in}{1.956011in}}%
\pgfusepath{clip}%
\pgfsetbuttcap%
\pgfsetroundjoin%
\definecolor{currentfill}{rgb}{0.007843,0.619608,0.450980}%
\pgfsetfillcolor{currentfill}%
\pgfsetfillopacity{0.200000}%
\pgfsetlinewidth{1.003750pt}%
\definecolor{currentstroke}{rgb}{0.007843,0.619608,0.450980}%
\pgfsetstrokecolor{currentstroke}%
\pgfsetstrokeopacity{0.200000}%
\pgfsetdash{}{0pt}%
\pgfsys@defobject{currentmarker}{\pgfqpoint{0.533579in}{0.452692in}}{\pgfqpoint{4.096281in}{0.505645in}}{%
\pgfpathmoveto{\pgfqpoint{0.533579in}{0.505645in}}%
\pgfpathlineto{\pgfqpoint{0.533579in}{0.502415in}}%
\pgfpathlineto{\pgfqpoint{0.641540in}{0.483825in}}%
\pgfpathlineto{\pgfqpoint{0.821474in}{0.467869in}}%
\pgfpathlineto{\pgfqpoint{1.073383in}{0.461671in}}%
\pgfpathlineto{\pgfqpoint{1.397265in}{0.458849in}}%
\pgfpathlineto{\pgfqpoint{1.793120in}{0.455833in}}%
\pgfpathlineto{\pgfqpoint{2.260950in}{0.455117in}}%
\pgfpathlineto{\pgfqpoint{2.800753in}{0.453433in}}%
\pgfpathlineto{\pgfqpoint{3.412530in}{0.453025in}}%
\pgfpathlineto{\pgfqpoint{4.096281in}{0.452692in}}%
\pgfpathlineto{\pgfqpoint{4.096281in}{0.452767in}}%
\pgfpathlineto{\pgfqpoint{4.096281in}{0.452767in}}%
\pgfpathlineto{\pgfqpoint{3.412530in}{0.453084in}}%
\pgfpathlineto{\pgfqpoint{2.800753in}{0.453473in}}%
\pgfpathlineto{\pgfqpoint{2.260950in}{0.455233in}}%
\pgfpathlineto{\pgfqpoint{1.793120in}{0.456984in}}%
\pgfpathlineto{\pgfqpoint{1.397265in}{0.460478in}}%
\pgfpathlineto{\pgfqpoint{1.073383in}{0.464623in}}%
\pgfpathlineto{\pgfqpoint{0.821474in}{0.468499in}}%
\pgfpathlineto{\pgfqpoint{0.641540in}{0.484324in}}%
\pgfpathlineto{\pgfqpoint{0.533579in}{0.505645in}}%
\pgfpathlineto{\pgfqpoint{0.533579in}{0.505645in}}%
\pgfpathclose%
\pgfusepath{stroke,fill}%
}%
\begin{pgfscope}%
\pgfsys@transformshift{0.000000in}{0.000000in}%
\pgfsys@useobject{currentmarker}{}%
\end{pgfscope}%
\end{pgfscope}%
\begin{pgfscope}%
\pgfpathrectangle{\pgfqpoint{0.497592in}{0.451389in}}{\pgfqpoint{3.776824in}{1.956011in}}%
\pgfusepath{clip}%
\pgfsetbuttcap%
\pgfsetroundjoin%
\definecolor{currentfill}{rgb}{0.835294,0.368627,0.000000}%
\pgfsetfillcolor{currentfill}%
\pgfsetfillopacity{0.200000}%
\pgfsetlinewidth{1.003750pt}%
\definecolor{currentstroke}{rgb}{0.835294,0.368627,0.000000}%
\pgfsetstrokecolor{currentstroke}%
\pgfsetstrokeopacity{0.200000}%
\pgfsetdash{}{0pt}%
\pgfsys@defobject{currentmarker}{\pgfqpoint{0.533579in}{0.453086in}}{\pgfqpoint{4.096281in}{0.625034in}}{%
\pgfpathmoveto{\pgfqpoint{0.533579in}{0.625034in}}%
\pgfpathlineto{\pgfqpoint{0.533579in}{0.617535in}}%
\pgfpathlineto{\pgfqpoint{0.641540in}{0.500452in}}%
\pgfpathlineto{\pgfqpoint{0.821474in}{0.475598in}}%
\pgfpathlineto{\pgfqpoint{1.073383in}{0.464589in}}%
\pgfpathlineto{\pgfqpoint{1.397265in}{0.459409in}}%
\pgfpathlineto{\pgfqpoint{1.793120in}{0.457627in}}%
\pgfpathlineto{\pgfqpoint{2.260950in}{0.455455in}}%
\pgfpathlineto{\pgfqpoint{2.800753in}{0.454131in}}%
\pgfpathlineto{\pgfqpoint{3.412530in}{0.453539in}}%
\pgfpathlineto{\pgfqpoint{4.096281in}{0.453086in}}%
\pgfpathlineto{\pgfqpoint{4.096281in}{0.453125in}}%
\pgfpathlineto{\pgfqpoint{4.096281in}{0.453125in}}%
\pgfpathlineto{\pgfqpoint{3.412530in}{0.453605in}}%
\pgfpathlineto{\pgfqpoint{2.800753in}{0.454171in}}%
\pgfpathlineto{\pgfqpoint{2.260950in}{0.455513in}}%
\pgfpathlineto{\pgfqpoint{1.793120in}{0.457777in}}%
\pgfpathlineto{\pgfqpoint{1.397265in}{0.459707in}}%
\pgfpathlineto{\pgfqpoint{1.073383in}{0.465563in}}%
\pgfpathlineto{\pgfqpoint{0.821474in}{0.478979in}}%
\pgfpathlineto{\pgfqpoint{0.641540in}{0.501257in}}%
\pgfpathlineto{\pgfqpoint{0.533579in}{0.625034in}}%
\pgfpathlineto{\pgfqpoint{0.533579in}{0.625034in}}%
\pgfpathclose%
\pgfusepath{stroke,fill}%
}%
\begin{pgfscope}%
\pgfsys@transformshift{0.000000in}{0.000000in}%
\pgfsys@useobject{currentmarker}{}%
\end{pgfscope}%
\end{pgfscope}%
\begin{pgfscope}%
\pgfpathrectangle{\pgfqpoint{0.497592in}{0.451389in}}{\pgfqpoint{3.776824in}{1.956011in}}%
\pgfusepath{clip}%
\pgfsetbuttcap%
\pgfsetroundjoin%
\definecolor{currentfill}{rgb}{0.800000,0.470588,0.737255}%
\pgfsetfillcolor{currentfill}%
\pgfsetfillopacity{0.200000}%
\pgfsetlinewidth{1.003750pt}%
\definecolor{currentstroke}{rgb}{0.800000,0.470588,0.737255}%
\pgfsetstrokecolor{currentstroke}%
\pgfsetstrokeopacity{0.200000}%
\pgfsetdash{}{0pt}%
\pgfsys@defobject{currentmarker}{\pgfqpoint{0.533579in}{0.451389in}}{\pgfqpoint{4.096281in}{0.451643in}}{%
\pgfpathmoveto{\pgfqpoint{0.533579in}{0.451643in}}%
\pgfpathlineto{\pgfqpoint{0.533579in}{0.451521in}}%
\pgfpathlineto{\pgfqpoint{0.641540in}{0.451404in}}%
\pgfpathlineto{\pgfqpoint{0.821474in}{0.451394in}}%
\pgfpathlineto{\pgfqpoint{1.073383in}{0.451391in}}%
\pgfpathlineto{\pgfqpoint{1.397265in}{0.451390in}}%
\pgfpathlineto{\pgfqpoint{1.793120in}{0.451390in}}%
\pgfpathlineto{\pgfqpoint{2.260950in}{0.451389in}}%
\pgfpathlineto{\pgfqpoint{2.800753in}{0.451389in}}%
\pgfpathlineto{\pgfqpoint{3.412530in}{0.451389in}}%
\pgfpathlineto{\pgfqpoint{4.096281in}{0.451389in}}%
\pgfpathlineto{\pgfqpoint{4.096281in}{0.451389in}}%
\pgfpathlineto{\pgfqpoint{4.096281in}{0.451389in}}%
\pgfpathlineto{\pgfqpoint{3.412530in}{0.451390in}}%
\pgfpathlineto{\pgfqpoint{2.800753in}{0.451390in}}%
\pgfpathlineto{\pgfqpoint{2.260950in}{0.451391in}}%
\pgfpathlineto{\pgfqpoint{1.793120in}{0.451393in}}%
\pgfpathlineto{\pgfqpoint{1.397265in}{0.451397in}}%
\pgfpathlineto{\pgfqpoint{1.073383in}{0.451408in}}%
\pgfpathlineto{\pgfqpoint{0.821474in}{0.451565in}}%
\pgfpathlineto{\pgfqpoint{0.641540in}{0.451413in}}%
\pgfpathlineto{\pgfqpoint{0.533579in}{0.451643in}}%
\pgfpathlineto{\pgfqpoint{0.533579in}{0.451643in}}%
\pgfpathclose%
\pgfusepath{stroke,fill}%
}%
\begin{pgfscope}%
\pgfsys@transformshift{0.000000in}{0.000000in}%
\pgfsys@useobject{currentmarker}{}%
\end{pgfscope}%
\end{pgfscope}%
\begin{pgfscope}%
\pgfpathrectangle{\pgfqpoint{0.497592in}{0.451389in}}{\pgfqpoint{3.776824in}{1.956011in}}%
\pgfusepath{clip}%
\pgfsetbuttcap%
\pgfsetroundjoin%
\definecolor{currentfill}{rgb}{0.792157,0.568627,0.380392}%
\pgfsetfillcolor{currentfill}%
\pgfsetfillopacity{0.200000}%
\pgfsetlinewidth{1.003750pt}%
\definecolor{currentstroke}{rgb}{0.792157,0.568627,0.380392}%
\pgfsetstrokecolor{currentstroke}%
\pgfsetstrokeopacity{0.200000}%
\pgfsetdash{}{0pt}%
\pgfsys@defobject{currentmarker}{\pgfqpoint{0.533579in}{0.451389in}}{\pgfqpoint{4.096281in}{0.451711in}}{%
\pgfpathmoveto{\pgfqpoint{0.533579in}{0.451711in}}%
\pgfpathlineto{\pgfqpoint{0.533579in}{0.451478in}}%
\pgfpathlineto{\pgfqpoint{0.641540in}{0.451402in}}%
\pgfpathlineto{\pgfqpoint{0.821474in}{0.451397in}}%
\pgfpathlineto{\pgfqpoint{1.073383in}{0.451392in}}%
\pgfpathlineto{\pgfqpoint{1.397265in}{0.451391in}}%
\pgfpathlineto{\pgfqpoint{1.793120in}{0.451390in}}%
\pgfpathlineto{\pgfqpoint{2.260950in}{0.451390in}}%
\pgfpathlineto{\pgfqpoint{2.800753in}{0.451389in}}%
\pgfpathlineto{\pgfqpoint{3.412530in}{0.451389in}}%
\pgfpathlineto{\pgfqpoint{4.096281in}{0.451389in}}%
\pgfpathlineto{\pgfqpoint{4.096281in}{0.451389in}}%
\pgfpathlineto{\pgfqpoint{4.096281in}{0.451389in}}%
\pgfpathlineto{\pgfqpoint{3.412530in}{0.451389in}}%
\pgfpathlineto{\pgfqpoint{2.800753in}{0.451390in}}%
\pgfpathlineto{\pgfqpoint{2.260950in}{0.451390in}}%
\pgfpathlineto{\pgfqpoint{1.793120in}{0.451391in}}%
\pgfpathlineto{\pgfqpoint{1.397265in}{0.451391in}}%
\pgfpathlineto{\pgfqpoint{1.073383in}{0.451394in}}%
\pgfpathlineto{\pgfqpoint{0.821474in}{0.451403in}}%
\pgfpathlineto{\pgfqpoint{0.641540in}{0.451430in}}%
\pgfpathlineto{\pgfqpoint{0.533579in}{0.451711in}}%
\pgfpathlineto{\pgfqpoint{0.533579in}{0.451711in}}%
\pgfpathclose%
\pgfusepath{stroke,fill}%
}%
\begin{pgfscope}%
\pgfsys@transformshift{0.000000in}{0.000000in}%
\pgfsys@useobject{currentmarker}{}%
\end{pgfscope}%
\end{pgfscope}%
\begin{pgfscope}%
\pgfsetrectcap%
\pgfsetmiterjoin%
\pgfsetlinewidth{1.254687pt}%
\definecolor{currentstroke}{rgb}{0.800000,0.800000,0.800000}%
\pgfsetstrokecolor{currentstroke}%
\pgfsetdash{}{0pt}%
\pgfpathmoveto{\pgfqpoint{0.497592in}{0.451389in}}%
\pgfpathlineto{\pgfqpoint{0.497592in}{2.407400in}}%
\pgfusepath{stroke}%
\end{pgfscope}%
\begin{pgfscope}%
\pgfsetrectcap%
\pgfsetmiterjoin%
\pgfsetlinewidth{1.254687pt}%
\definecolor{currentstroke}{rgb}{0.800000,0.800000,0.800000}%
\pgfsetstrokecolor{currentstroke}%
\pgfsetdash{}{0pt}%
\pgfpathmoveto{\pgfqpoint{4.274417in}{0.451389in}}%
\pgfpathlineto{\pgfqpoint{4.274417in}{2.407400in}}%
\pgfusepath{stroke}%
\end{pgfscope}%
\begin{pgfscope}%
\pgfsetrectcap%
\pgfsetmiterjoin%
\pgfsetlinewidth{1.254687pt}%
\definecolor{currentstroke}{rgb}{0.800000,0.800000,0.800000}%
\pgfsetstrokecolor{currentstroke}%
\pgfsetdash{}{0pt}%
\pgfpathmoveto{\pgfqpoint{0.497592in}{0.451389in}}%
\pgfpathlineto{\pgfqpoint{4.274417in}{0.451389in}}%
\pgfusepath{stroke}%
\end{pgfscope}%
\begin{pgfscope}%
\pgfsetrectcap%
\pgfsetmiterjoin%
\pgfsetlinewidth{1.254687pt}%
\definecolor{currentstroke}{rgb}{0.800000,0.800000,0.800000}%
\pgfsetstrokecolor{currentstroke}%
\pgfsetdash{}{0pt}%
\pgfpathmoveto{\pgfqpoint{0.497592in}{2.407400in}}%
\pgfpathlineto{\pgfqpoint{4.274417in}{2.407400in}}%
\pgfusepath{stroke}%
\end{pgfscope}%
\begin{pgfscope}%
\pgfsetbuttcap%
\pgfsetmiterjoin%
\definecolor{currentfill}{rgb}{1.000000,1.000000,1.000000}%
\pgfsetfillcolor{currentfill}%
\pgfsetfillopacity{0.800000}%
\pgfsetlinewidth{1.003750pt}%
\definecolor{currentstroke}{rgb}{0.800000,0.800000,0.800000}%
\pgfsetstrokecolor{currentstroke}%
\pgfsetstrokeopacity{0.800000}%
\pgfsetdash{}{0pt}%
\pgfpathmoveto{\pgfqpoint{4.456337in}{0.538404in}}%
\pgfpathlineto{\pgfqpoint{5.650893in}{0.538404in}}%
\pgfpathquadraticcurveto{\pgfqpoint{5.675893in}{0.538404in}}{\pgfqpoint{5.675893in}{0.563404in}}%
\pgfpathlineto{\pgfqpoint{5.675893in}{2.295385in}}%
\pgfpathquadraticcurveto{\pgfqpoint{5.675893in}{2.320385in}}{\pgfqpoint{5.650893in}{2.320385in}}%
\pgfpathlineto{\pgfqpoint{4.456337in}{2.320385in}}%
\pgfpathquadraticcurveto{\pgfqpoint{4.431337in}{2.320385in}}{\pgfqpoint{4.431337in}{2.295385in}}%
\pgfpathlineto{\pgfqpoint{4.431337in}{0.563404in}}%
\pgfpathquadraticcurveto{\pgfqpoint{4.431337in}{0.538404in}}{\pgfqpoint{4.456337in}{0.538404in}}%
\pgfpathlineto{\pgfqpoint{4.456337in}{0.538404in}}%
\pgfpathclose%
\pgfusepath{stroke,fill}%
\end{pgfscope}%
\begin{pgfscope}%
\definecolor{textcolor}{rgb}{0.150000,0.150000,0.150000}%
\pgfsetstrokecolor{textcolor}%
\pgfsetfillcolor{textcolor}%
\pgftext[x=4.850215in,y=2.175414in,left,base]{\color{textcolor}\sffamily\fontsize{9.000000}{10.800000}\selectfont Legend}%
\end{pgfscope}%
\begin{pgfscope}%
\pgfsetroundcap%
\pgfsetroundjoin%
\pgfsetlinewidth{1.505625pt}%
\definecolor{currentstroke}{rgb}{0.003922,0.450980,0.698039}%
\pgfsetstrokecolor{currentstroke}%
\pgfsetdash{}{0pt}%
\pgfpathmoveto{\pgfqpoint{4.481337in}{2.031664in}}%
\pgfpathlineto{\pgfqpoint{4.606337in}{2.031664in}}%
\pgfpathlineto{\pgfqpoint{4.731337in}{2.031664in}}%
\pgfusepath{stroke}%
\end{pgfscope}%
\begin{pgfscope}%
\definecolor{textcolor}{rgb}{0.150000,0.150000,0.150000}%
\pgfsetstrokecolor{textcolor}%
\pgfsetfillcolor{textcolor}%
\pgftext[x=4.831337in,y=1.987914in,left,base]{\color{textcolor}\sffamily\fontsize{9.000000}{10.800000}\selectfont Total time}%
\end{pgfscope}%
\begin{pgfscope}%
\pgfsetroundcap%
\pgfsetroundjoin%
\pgfsetlinewidth{1.505625pt}%
\definecolor{currentstroke}{rgb}{0.870588,0.560784,0.019608}%
\pgfsetstrokecolor{currentstroke}%
\pgfsetdash{}{0pt}%
\pgfpathmoveto{\pgfqpoint{4.481337in}{1.844164in}}%
\pgfpathlineto{\pgfqpoint{4.606337in}{1.844164in}}%
\pgfpathlineto{\pgfqpoint{4.731337in}{1.844164in}}%
\pgfusepath{stroke}%
\end{pgfscope}%
\begin{pgfscope}%
\definecolor{textcolor}{rgb}{0.150000,0.150000,0.150000}%
\pgfsetstrokecolor{textcolor}%
\pgfsetfillcolor{textcolor}%
\pgftext[x=4.831337in,y=1.800414in,left,base]{\color{textcolor}\sffamily\fontsize{9.000000}{10.800000}\selectfont kNN search}%
\end{pgfscope}%
\begin{pgfscope}%
\pgfsetroundcap%
\pgfsetroundjoin%
\pgfsetlinewidth{1.505625pt}%
\definecolor{currentstroke}{rgb}{0.007843,0.619608,0.450980}%
\pgfsetstrokecolor{currentstroke}%
\pgfsetdash{}{0pt}%
\pgfpathmoveto{\pgfqpoint{4.481337in}{1.656664in}}%
\pgfpathlineto{\pgfqpoint{4.606337in}{1.656664in}}%
\pgfpathlineto{\pgfqpoint{4.731337in}{1.656664in}}%
\pgfusepath{stroke}%
\end{pgfscope}%
\begin{pgfscope}%
\definecolor{textcolor}{rgb}{0.150000,0.150000,0.150000}%
\pgfsetstrokecolor{textcolor}%
\pgfsetfillcolor{textcolor}%
\pgftext[x=4.831337in,y=1.612914in,left,base]{\color{textcolor}\sffamily\fontsize{9.000000}{10.800000}\selectfont Create graph}%
\end{pgfscope}%
\begin{pgfscope}%
\pgfsetroundcap%
\pgfsetroundjoin%
\pgfsetlinewidth{1.505625pt}%
\definecolor{currentstroke}{rgb}{0.835294,0.368627,0.000000}%
\pgfsetstrokecolor{currentstroke}%
\pgfsetdash{}{0pt}%
\pgfpathmoveto{\pgfqpoint{4.481337in}{1.382153in}}%
\pgfpathlineto{\pgfqpoint{4.606337in}{1.382153in}}%
\pgfpathlineto{\pgfqpoint{4.731337in}{1.382153in}}%
\pgfusepath{stroke}%
\end{pgfscope}%
\begin{pgfscope}%
\definecolor{textcolor}{rgb}{0.150000,0.150000,0.150000}%
\pgfsetstrokecolor{textcolor}%
\pgfsetfillcolor{textcolor}%
\pgftext[x=4.831337in, y=1.425415in, left, base]{\color{textcolor}\sffamily\fontsize{9.000000}{10.800000}\selectfont Get \& prepare}%
\end{pgfscope}%
\begin{pgfscope}%
\definecolor{textcolor}{rgb}{0.150000,0.150000,0.150000}%
\pgfsetstrokecolor{textcolor}%
\pgfsetfillcolor{textcolor}%
\pgftext[x=4.831337in, y=1.281421in, left, base]{\color{textcolor}\sffamily\fontsize{9.000000}{10.800000}\selectfont obstacles}%
\end{pgfscope}%
\begin{pgfscope}%
\pgfsetroundcap%
\pgfsetroundjoin%
\pgfsetlinewidth{1.505625pt}%
\definecolor{currentstroke}{rgb}{0.800000,0.470588,0.737255}%
\pgfsetstrokecolor{currentstroke}%
\pgfsetdash{}{0pt}%
\pgfpathmoveto{\pgfqpoint{4.481337in}{1.050659in}}%
\pgfpathlineto{\pgfqpoint{4.606337in}{1.050659in}}%
\pgfpathlineto{\pgfqpoint{4.731337in}{1.050659in}}%
\pgfusepath{stroke}%
\end{pgfscope}%
\begin{pgfscope}%
\definecolor{textcolor}{rgb}{0.150000,0.150000,0.150000}%
\pgfsetstrokecolor{textcolor}%
\pgfsetfillcolor{textcolor}%
\pgftext[x=4.831337in, y=1.093921in, left, base]{\color{textcolor}\sffamily\fontsize{9.000000}{10.800000}\selectfont Merge road}%
\end{pgfscope}%
\begin{pgfscope}%
\definecolor{textcolor}{rgb}{0.150000,0.150000,0.150000}%
\pgfsetstrokecolor{textcolor}%
\pgfsetfillcolor{textcolor}%
\pgftext[x=4.831337in, y=0.949927in, left, base]{\color{textcolor}\sffamily\fontsize{9.000000}{10.800000}\selectfont edges}%
\end{pgfscope}%
\begin{pgfscope}%
\pgfsetroundcap%
\pgfsetroundjoin%
\pgfsetlinewidth{1.505625pt}%
\definecolor{currentstroke}{rgb}{0.792157,0.568627,0.380392}%
\pgfsetstrokecolor{currentstroke}%
\pgfsetdash{}{0pt}%
\pgfpathmoveto{\pgfqpoint{4.481337in}{0.719165in}}%
\pgfpathlineto{\pgfqpoint{4.606337in}{0.719165in}}%
\pgfpathlineto{\pgfqpoint{4.731337in}{0.719165in}}%
\pgfusepath{stroke}%
\end{pgfscope}%
\begin{pgfscope}%
\definecolor{textcolor}{rgb}{0.150000,0.150000,0.150000}%
\pgfsetstrokecolor{textcolor}%
\pgfsetfillcolor{textcolor}%
\pgftext[x=4.831337in, y=0.762427in, left, base]{\color{textcolor}\sffamily\fontsize{9.000000}{10.800000}\selectfont Add POI}%
\end{pgfscope}%
\begin{pgfscope}%
\definecolor{textcolor}{rgb}{0.150000,0.150000,0.150000}%
\pgfsetstrokecolor{textcolor}%
\pgfsetfillcolor{textcolor}%
\pgftext[x=4.831337in, y=0.618433in, left, base]{\color{textcolor}\sffamily\fontsize{9.000000}{10.800000}\selectfont attributes}%
\end{pgfscope}%
\begin{pgfscope}%
\pgfsetroundcap%
\pgfsetroundjoin%
\pgfsetlinewidth{1.003750pt}%
\definecolor{currentstroke}{rgb}{0.003922,0.450980,0.698039}%
\pgfsetstrokecolor{currentstroke}%
\pgfsetdash{}{0pt}%
\pgfpathmoveto{\pgfqpoint{0.533579in}{2.314256in}}%
\pgfpathlineto{\pgfqpoint{0.641540in}{2.314256in}}%
\pgfpathlineto{\pgfqpoint{0.821474in}{2.314256in}}%
\pgfpathlineto{\pgfqpoint{1.073383in}{2.314256in}}%
\pgfpathlineto{\pgfqpoint{1.397265in}{2.314256in}}%
\pgfpathlineto{\pgfqpoint{1.793120in}{2.314256in}}%
\pgfpathlineto{\pgfqpoint{2.260950in}{2.314256in}}%
\pgfpathlineto{\pgfqpoint{2.800753in}{2.314256in}}%
\pgfpathlineto{\pgfqpoint{3.412530in}{2.314256in}}%
\pgfpathlineto{\pgfqpoint{4.096281in}{2.314256in}}%
\pgfusepath{stroke}%
\end{pgfscope}%
\begin{pgfscope}%
\pgfsetbuttcap%
\pgfsetroundjoin%
\definecolor{currentfill}{rgb}{0.003922,0.450980,0.698039}%
\pgfsetfillcolor{currentfill}%
\pgfsetlinewidth{0.752812pt}%
\definecolor{currentstroke}{rgb}{1.000000,1.000000,1.000000}%
\pgfsetstrokecolor{currentstroke}%
\pgfsetdash{}{0pt}%
\pgfsys@defobject{currentmarker}{\pgfqpoint{-0.034722in}{-0.034722in}}{\pgfqpoint{0.034722in}{0.034722in}}{%
\pgfpathmoveto{\pgfqpoint{0.000000in}{-0.034722in}}%
\pgfpathcurveto{\pgfqpoint{0.009208in}{-0.034722in}}{\pgfqpoint{0.018041in}{-0.031064in}}{\pgfqpoint{0.024552in}{-0.024552in}}%
\pgfpathcurveto{\pgfqpoint{0.031064in}{-0.018041in}}{\pgfqpoint{0.034722in}{-0.009208in}}{\pgfqpoint{0.034722in}{0.000000in}}%
\pgfpathcurveto{\pgfqpoint{0.034722in}{0.009208in}}{\pgfqpoint{0.031064in}{0.018041in}}{\pgfqpoint{0.024552in}{0.024552in}}%
\pgfpathcurveto{\pgfqpoint{0.018041in}{0.031064in}}{\pgfqpoint{0.009208in}{0.034722in}}{\pgfqpoint{0.000000in}{0.034722in}}%
\pgfpathcurveto{\pgfqpoint{-0.009208in}{0.034722in}}{\pgfqpoint{-0.018041in}{0.031064in}}{\pgfqpoint{-0.024552in}{0.024552in}}%
\pgfpathcurveto{\pgfqpoint{-0.031064in}{0.018041in}}{\pgfqpoint{-0.034722in}{0.009208in}}{\pgfqpoint{-0.034722in}{0.000000in}}%
\pgfpathcurveto{\pgfqpoint{-0.034722in}{-0.009208in}}{\pgfqpoint{-0.031064in}{-0.018041in}}{\pgfqpoint{-0.024552in}{-0.024552in}}%
\pgfpathcurveto{\pgfqpoint{-0.018041in}{-0.031064in}}{\pgfqpoint{-0.009208in}{-0.034722in}}{\pgfqpoint{0.000000in}{-0.034722in}}%
\pgfpathlineto{\pgfqpoint{0.000000in}{-0.034722in}}%
\pgfpathclose%
\pgfusepath{stroke,fill}%
}%
\begin{pgfscope}%
\pgfsys@transformshift{0.533579in}{2.314256in}%
\pgfsys@useobject{currentmarker}{}%
\end{pgfscope}%
\begin{pgfscope}%
\pgfsys@transformshift{0.641540in}{2.314256in}%
\pgfsys@useobject{currentmarker}{}%
\end{pgfscope}%
\begin{pgfscope}%
\pgfsys@transformshift{0.821474in}{2.314256in}%
\pgfsys@useobject{currentmarker}{}%
\end{pgfscope}%
\begin{pgfscope}%
\pgfsys@transformshift{1.073383in}{2.314256in}%
\pgfsys@useobject{currentmarker}{}%
\end{pgfscope}%
\begin{pgfscope}%
\pgfsys@transformshift{1.397265in}{2.314256in}%
\pgfsys@useobject{currentmarker}{}%
\end{pgfscope}%
\begin{pgfscope}%
\pgfsys@transformshift{1.793120in}{2.314256in}%
\pgfsys@useobject{currentmarker}{}%
\end{pgfscope}%
\begin{pgfscope}%
\pgfsys@transformshift{2.260950in}{2.314256in}%
\pgfsys@useobject{currentmarker}{}%
\end{pgfscope}%
\begin{pgfscope}%
\pgfsys@transformshift{2.800753in}{2.314256in}%
\pgfsys@useobject{currentmarker}{}%
\end{pgfscope}%
\begin{pgfscope}%
\pgfsys@transformshift{3.412530in}{2.314256in}%
\pgfsys@useobject{currentmarker}{}%
\end{pgfscope}%
\begin{pgfscope}%
\pgfsys@transformshift{4.096281in}{2.314256in}%
\pgfsys@useobject{currentmarker}{}%
\end{pgfscope}%
\end{pgfscope}%
\begin{pgfscope}%
\pgfsetroundcap%
\pgfsetroundjoin%
\pgfsetlinewidth{1.003750pt}%
\definecolor{currentstroke}{rgb}{0.870588,0.560784,0.019608}%
\pgfsetstrokecolor{currentstroke}%
\pgfsetdash{}{0pt}%
\pgfpathmoveto{\pgfqpoint{0.533579in}{2.073143in}}%
\pgfpathlineto{\pgfqpoint{0.641540in}{2.228902in}}%
\pgfpathlineto{\pgfqpoint{0.821474in}{2.269753in}}%
\pgfpathlineto{\pgfqpoint{1.073383in}{2.288048in}}%
\pgfpathlineto{\pgfqpoint{1.397265in}{2.296942in}}%
\pgfpathlineto{\pgfqpoint{1.793120in}{2.302609in}}%
\pgfpathlineto{\pgfqpoint{2.260950in}{2.306107in}}%
\pgfpathlineto{\pgfqpoint{2.800753in}{2.309254in}}%
\pgfpathlineto{\pgfqpoint{3.412530in}{2.310265in}}%
\pgfpathlineto{\pgfqpoint{4.096281in}{2.311098in}}%
\pgfusepath{stroke}%
\end{pgfscope}%
\begin{pgfscope}%
\pgfsetbuttcap%
\pgfsetroundjoin%
\definecolor{currentfill}{rgb}{0.870588,0.560784,0.019608}%
\pgfsetfillcolor{currentfill}%
\pgfsetlinewidth{0.752812pt}%
\definecolor{currentstroke}{rgb}{1.000000,1.000000,1.000000}%
\pgfsetstrokecolor{currentstroke}%
\pgfsetdash{}{0pt}%
\pgfsys@defobject{currentmarker}{\pgfqpoint{-0.034722in}{-0.034722in}}{\pgfqpoint{0.034722in}{0.034722in}}{%
\pgfpathmoveto{\pgfqpoint{0.000000in}{-0.034722in}}%
\pgfpathcurveto{\pgfqpoint{0.009208in}{-0.034722in}}{\pgfqpoint{0.018041in}{-0.031064in}}{\pgfqpoint{0.024552in}{-0.024552in}}%
\pgfpathcurveto{\pgfqpoint{0.031064in}{-0.018041in}}{\pgfqpoint{0.034722in}{-0.009208in}}{\pgfqpoint{0.034722in}{0.000000in}}%
\pgfpathcurveto{\pgfqpoint{0.034722in}{0.009208in}}{\pgfqpoint{0.031064in}{0.018041in}}{\pgfqpoint{0.024552in}{0.024552in}}%
\pgfpathcurveto{\pgfqpoint{0.018041in}{0.031064in}}{\pgfqpoint{0.009208in}{0.034722in}}{\pgfqpoint{0.000000in}{0.034722in}}%
\pgfpathcurveto{\pgfqpoint{-0.009208in}{0.034722in}}{\pgfqpoint{-0.018041in}{0.031064in}}{\pgfqpoint{-0.024552in}{0.024552in}}%
\pgfpathcurveto{\pgfqpoint{-0.031064in}{0.018041in}}{\pgfqpoint{-0.034722in}{0.009208in}}{\pgfqpoint{-0.034722in}{0.000000in}}%
\pgfpathcurveto{\pgfqpoint{-0.034722in}{-0.009208in}}{\pgfqpoint{-0.031064in}{-0.018041in}}{\pgfqpoint{-0.024552in}{-0.024552in}}%
\pgfpathcurveto{\pgfqpoint{-0.018041in}{-0.031064in}}{\pgfqpoint{-0.009208in}{-0.034722in}}{\pgfqpoint{0.000000in}{-0.034722in}}%
\pgfpathlineto{\pgfqpoint{0.000000in}{-0.034722in}}%
\pgfpathclose%
\pgfusepath{stroke,fill}%
}%
\begin{pgfscope}%
\pgfsys@transformshift{0.533579in}{2.073143in}%
\pgfsys@useobject{currentmarker}{}%
\end{pgfscope}%
\begin{pgfscope}%
\pgfsys@transformshift{0.641540in}{2.228902in}%
\pgfsys@useobject{currentmarker}{}%
\end{pgfscope}%
\begin{pgfscope}%
\pgfsys@transformshift{0.821474in}{2.269753in}%
\pgfsys@useobject{currentmarker}{}%
\end{pgfscope}%
\begin{pgfscope}%
\pgfsys@transformshift{1.073383in}{2.288048in}%
\pgfsys@useobject{currentmarker}{}%
\end{pgfscope}%
\begin{pgfscope}%
\pgfsys@transformshift{1.397265in}{2.296942in}%
\pgfsys@useobject{currentmarker}{}%
\end{pgfscope}%
\begin{pgfscope}%
\pgfsys@transformshift{1.793120in}{2.302609in}%
\pgfsys@useobject{currentmarker}{}%
\end{pgfscope}%
\begin{pgfscope}%
\pgfsys@transformshift{2.260950in}{2.306107in}%
\pgfsys@useobject{currentmarker}{}%
\end{pgfscope}%
\begin{pgfscope}%
\pgfsys@transformshift{2.800753in}{2.309254in}%
\pgfsys@useobject{currentmarker}{}%
\end{pgfscope}%
\begin{pgfscope}%
\pgfsys@transformshift{3.412530in}{2.310265in}%
\pgfsys@useobject{currentmarker}{}%
\end{pgfscope}%
\begin{pgfscope}%
\pgfsys@transformshift{4.096281in}{2.311098in}%
\pgfsys@useobject{currentmarker}{}%
\end{pgfscope}%
\end{pgfscope}%
\begin{pgfscope}%
\pgfsetroundcap%
\pgfsetroundjoin%
\pgfsetlinewidth{1.003750pt}%
\definecolor{currentstroke}{rgb}{0.007843,0.619608,0.450980}%
\pgfsetstrokecolor{currentstroke}%
\pgfsetdash{}{0pt}%
\pgfpathmoveto{\pgfqpoint{0.533579in}{0.503812in}}%
\pgfpathlineto{\pgfqpoint{0.641540in}{0.484028in}}%
\pgfpathlineto{\pgfqpoint{0.821474in}{0.468064in}}%
\pgfpathlineto{\pgfqpoint{1.073383in}{0.462826in}}%
\pgfpathlineto{\pgfqpoint{1.397265in}{0.459864in}}%
\pgfpathlineto{\pgfqpoint{1.793120in}{0.456305in}}%
\pgfpathlineto{\pgfqpoint{2.260950in}{0.455172in}}%
\pgfpathlineto{\pgfqpoint{2.800753in}{0.453449in}}%
\pgfpathlineto{\pgfqpoint{3.412530in}{0.453049in}}%
\pgfpathlineto{\pgfqpoint{4.096281in}{0.452726in}}%
\pgfusepath{stroke}%
\end{pgfscope}%
\begin{pgfscope}%
\pgfsetbuttcap%
\pgfsetroundjoin%
\definecolor{currentfill}{rgb}{0.007843,0.619608,0.450980}%
\pgfsetfillcolor{currentfill}%
\pgfsetlinewidth{0.752812pt}%
\definecolor{currentstroke}{rgb}{1.000000,1.000000,1.000000}%
\pgfsetstrokecolor{currentstroke}%
\pgfsetdash{}{0pt}%
\pgfsys@defobject{currentmarker}{\pgfqpoint{-0.034722in}{-0.034722in}}{\pgfqpoint{0.034722in}{0.034722in}}{%
\pgfpathmoveto{\pgfqpoint{0.000000in}{-0.034722in}}%
\pgfpathcurveto{\pgfqpoint{0.009208in}{-0.034722in}}{\pgfqpoint{0.018041in}{-0.031064in}}{\pgfqpoint{0.024552in}{-0.024552in}}%
\pgfpathcurveto{\pgfqpoint{0.031064in}{-0.018041in}}{\pgfqpoint{0.034722in}{-0.009208in}}{\pgfqpoint{0.034722in}{0.000000in}}%
\pgfpathcurveto{\pgfqpoint{0.034722in}{0.009208in}}{\pgfqpoint{0.031064in}{0.018041in}}{\pgfqpoint{0.024552in}{0.024552in}}%
\pgfpathcurveto{\pgfqpoint{0.018041in}{0.031064in}}{\pgfqpoint{0.009208in}{0.034722in}}{\pgfqpoint{0.000000in}{0.034722in}}%
\pgfpathcurveto{\pgfqpoint{-0.009208in}{0.034722in}}{\pgfqpoint{-0.018041in}{0.031064in}}{\pgfqpoint{-0.024552in}{0.024552in}}%
\pgfpathcurveto{\pgfqpoint{-0.031064in}{0.018041in}}{\pgfqpoint{-0.034722in}{0.009208in}}{\pgfqpoint{-0.034722in}{0.000000in}}%
\pgfpathcurveto{\pgfqpoint{-0.034722in}{-0.009208in}}{\pgfqpoint{-0.031064in}{-0.018041in}}{\pgfqpoint{-0.024552in}{-0.024552in}}%
\pgfpathcurveto{\pgfqpoint{-0.018041in}{-0.031064in}}{\pgfqpoint{-0.009208in}{-0.034722in}}{\pgfqpoint{0.000000in}{-0.034722in}}%
\pgfpathlineto{\pgfqpoint{0.000000in}{-0.034722in}}%
\pgfpathclose%
\pgfusepath{stroke,fill}%
}%
\begin{pgfscope}%
\pgfsys@transformshift{0.533579in}{0.503812in}%
\pgfsys@useobject{currentmarker}{}%
\end{pgfscope}%
\begin{pgfscope}%
\pgfsys@transformshift{0.641540in}{0.484028in}%
\pgfsys@useobject{currentmarker}{}%
\end{pgfscope}%
\begin{pgfscope}%
\pgfsys@transformshift{0.821474in}{0.468064in}%
\pgfsys@useobject{currentmarker}{}%
\end{pgfscope}%
\begin{pgfscope}%
\pgfsys@transformshift{1.073383in}{0.462826in}%
\pgfsys@useobject{currentmarker}{}%
\end{pgfscope}%
\begin{pgfscope}%
\pgfsys@transformshift{1.397265in}{0.459864in}%
\pgfsys@useobject{currentmarker}{}%
\end{pgfscope}%
\begin{pgfscope}%
\pgfsys@transformshift{1.793120in}{0.456305in}%
\pgfsys@useobject{currentmarker}{}%
\end{pgfscope}%
\begin{pgfscope}%
\pgfsys@transformshift{2.260950in}{0.455172in}%
\pgfsys@useobject{currentmarker}{}%
\end{pgfscope}%
\begin{pgfscope}%
\pgfsys@transformshift{2.800753in}{0.453449in}%
\pgfsys@useobject{currentmarker}{}%
\end{pgfscope}%
\begin{pgfscope}%
\pgfsys@transformshift{3.412530in}{0.453049in}%
\pgfsys@useobject{currentmarker}{}%
\end{pgfscope}%
\begin{pgfscope}%
\pgfsys@transformshift{4.096281in}{0.452726in}%
\pgfsys@useobject{currentmarker}{}%
\end{pgfscope}%
\end{pgfscope}%
\begin{pgfscope}%
\pgfsetroundcap%
\pgfsetroundjoin%
\pgfsetlinewidth{1.003750pt}%
\definecolor{currentstroke}{rgb}{0.835294,0.368627,0.000000}%
\pgfsetstrokecolor{currentstroke}%
\pgfsetdash{}{0pt}%
\pgfpathmoveto{\pgfqpoint{0.533579in}{0.622179in}}%
\pgfpathlineto{\pgfqpoint{0.641540in}{0.500814in}}%
\pgfpathlineto{\pgfqpoint{0.821474in}{0.476821in}}%
\pgfpathlineto{\pgfqpoint{1.073383in}{0.465080in}}%
\pgfpathlineto{\pgfqpoint{1.397265in}{0.459562in}}%
\pgfpathlineto{\pgfqpoint{1.793120in}{0.457704in}}%
\pgfpathlineto{\pgfqpoint{2.260950in}{0.455484in}}%
\pgfpathlineto{\pgfqpoint{2.800753in}{0.454153in}}%
\pgfpathlineto{\pgfqpoint{3.412530in}{0.453574in}}%
\pgfpathlineto{\pgfqpoint{4.096281in}{0.453103in}}%
\pgfusepath{stroke}%
\end{pgfscope}%
\begin{pgfscope}%
\pgfsetbuttcap%
\pgfsetroundjoin%
\definecolor{currentfill}{rgb}{0.835294,0.368627,0.000000}%
\pgfsetfillcolor{currentfill}%
\pgfsetlinewidth{0.752812pt}%
\definecolor{currentstroke}{rgb}{1.000000,1.000000,1.000000}%
\pgfsetstrokecolor{currentstroke}%
\pgfsetdash{}{0pt}%
\pgfsys@defobject{currentmarker}{\pgfqpoint{-0.034722in}{-0.034722in}}{\pgfqpoint{0.034722in}{0.034722in}}{%
\pgfpathmoveto{\pgfqpoint{0.000000in}{-0.034722in}}%
\pgfpathcurveto{\pgfqpoint{0.009208in}{-0.034722in}}{\pgfqpoint{0.018041in}{-0.031064in}}{\pgfqpoint{0.024552in}{-0.024552in}}%
\pgfpathcurveto{\pgfqpoint{0.031064in}{-0.018041in}}{\pgfqpoint{0.034722in}{-0.009208in}}{\pgfqpoint{0.034722in}{0.000000in}}%
\pgfpathcurveto{\pgfqpoint{0.034722in}{0.009208in}}{\pgfqpoint{0.031064in}{0.018041in}}{\pgfqpoint{0.024552in}{0.024552in}}%
\pgfpathcurveto{\pgfqpoint{0.018041in}{0.031064in}}{\pgfqpoint{0.009208in}{0.034722in}}{\pgfqpoint{0.000000in}{0.034722in}}%
\pgfpathcurveto{\pgfqpoint{-0.009208in}{0.034722in}}{\pgfqpoint{-0.018041in}{0.031064in}}{\pgfqpoint{-0.024552in}{0.024552in}}%
\pgfpathcurveto{\pgfqpoint{-0.031064in}{0.018041in}}{\pgfqpoint{-0.034722in}{0.009208in}}{\pgfqpoint{-0.034722in}{0.000000in}}%
\pgfpathcurveto{\pgfqpoint{-0.034722in}{-0.009208in}}{\pgfqpoint{-0.031064in}{-0.018041in}}{\pgfqpoint{-0.024552in}{-0.024552in}}%
\pgfpathcurveto{\pgfqpoint{-0.018041in}{-0.031064in}}{\pgfqpoint{-0.009208in}{-0.034722in}}{\pgfqpoint{0.000000in}{-0.034722in}}%
\pgfpathlineto{\pgfqpoint{0.000000in}{-0.034722in}}%
\pgfpathclose%
\pgfusepath{stroke,fill}%
}%
\begin{pgfscope}%
\pgfsys@transformshift{0.533579in}{0.622179in}%
\pgfsys@useobject{currentmarker}{}%
\end{pgfscope}%
\begin{pgfscope}%
\pgfsys@transformshift{0.641540in}{0.500814in}%
\pgfsys@useobject{currentmarker}{}%
\end{pgfscope}%
\begin{pgfscope}%
\pgfsys@transformshift{0.821474in}{0.476821in}%
\pgfsys@useobject{currentmarker}{}%
\end{pgfscope}%
\begin{pgfscope}%
\pgfsys@transformshift{1.073383in}{0.465080in}%
\pgfsys@useobject{currentmarker}{}%
\end{pgfscope}%
\begin{pgfscope}%
\pgfsys@transformshift{1.397265in}{0.459562in}%
\pgfsys@useobject{currentmarker}{}%
\end{pgfscope}%
\begin{pgfscope}%
\pgfsys@transformshift{1.793120in}{0.457704in}%
\pgfsys@useobject{currentmarker}{}%
\end{pgfscope}%
\begin{pgfscope}%
\pgfsys@transformshift{2.260950in}{0.455484in}%
\pgfsys@useobject{currentmarker}{}%
\end{pgfscope}%
\begin{pgfscope}%
\pgfsys@transformshift{2.800753in}{0.454153in}%
\pgfsys@useobject{currentmarker}{}%
\end{pgfscope}%
\begin{pgfscope}%
\pgfsys@transformshift{3.412530in}{0.453574in}%
\pgfsys@useobject{currentmarker}{}%
\end{pgfscope}%
\begin{pgfscope}%
\pgfsys@transformshift{4.096281in}{0.453103in}%
\pgfsys@useobject{currentmarker}{}%
\end{pgfscope}%
\end{pgfscope}%
\begin{pgfscope}%
\pgfsetroundcap%
\pgfsetroundjoin%
\pgfsetlinewidth{1.003750pt}%
\definecolor{currentstroke}{rgb}{0.800000,0.470588,0.737255}%
\pgfsetstrokecolor{currentstroke}%
\pgfsetdash{}{0pt}%
\pgfpathmoveto{\pgfqpoint{0.533579in}{0.451581in}}%
\pgfpathlineto{\pgfqpoint{0.641540in}{0.451410in}}%
\pgfpathlineto{\pgfqpoint{0.821474in}{0.451450in}}%
\pgfpathlineto{\pgfqpoint{1.073383in}{0.451398in}}%
\pgfpathlineto{\pgfqpoint{1.397265in}{0.451393in}}%
\pgfpathlineto{\pgfqpoint{1.793120in}{0.451391in}}%
\pgfpathlineto{\pgfqpoint{2.260950in}{0.451390in}}%
\pgfpathlineto{\pgfqpoint{2.800753in}{0.451389in}}%
\pgfpathlineto{\pgfqpoint{3.412530in}{0.451389in}}%
\pgfpathlineto{\pgfqpoint{4.096281in}{0.451389in}}%
\pgfusepath{stroke}%
\end{pgfscope}%
\begin{pgfscope}%
\pgfsetbuttcap%
\pgfsetroundjoin%
\definecolor{currentfill}{rgb}{0.800000,0.470588,0.737255}%
\pgfsetfillcolor{currentfill}%
\pgfsetlinewidth{0.752812pt}%
\definecolor{currentstroke}{rgb}{1.000000,1.000000,1.000000}%
\pgfsetstrokecolor{currentstroke}%
\pgfsetdash{}{0pt}%
\pgfsys@defobject{currentmarker}{\pgfqpoint{-0.034722in}{-0.034722in}}{\pgfqpoint{0.034722in}{0.034722in}}{%
\pgfpathmoveto{\pgfqpoint{0.000000in}{-0.034722in}}%
\pgfpathcurveto{\pgfqpoint{0.009208in}{-0.034722in}}{\pgfqpoint{0.018041in}{-0.031064in}}{\pgfqpoint{0.024552in}{-0.024552in}}%
\pgfpathcurveto{\pgfqpoint{0.031064in}{-0.018041in}}{\pgfqpoint{0.034722in}{-0.009208in}}{\pgfqpoint{0.034722in}{0.000000in}}%
\pgfpathcurveto{\pgfqpoint{0.034722in}{0.009208in}}{\pgfqpoint{0.031064in}{0.018041in}}{\pgfqpoint{0.024552in}{0.024552in}}%
\pgfpathcurveto{\pgfqpoint{0.018041in}{0.031064in}}{\pgfqpoint{0.009208in}{0.034722in}}{\pgfqpoint{0.000000in}{0.034722in}}%
\pgfpathcurveto{\pgfqpoint{-0.009208in}{0.034722in}}{\pgfqpoint{-0.018041in}{0.031064in}}{\pgfqpoint{-0.024552in}{0.024552in}}%
\pgfpathcurveto{\pgfqpoint{-0.031064in}{0.018041in}}{\pgfqpoint{-0.034722in}{0.009208in}}{\pgfqpoint{-0.034722in}{0.000000in}}%
\pgfpathcurveto{\pgfqpoint{-0.034722in}{-0.009208in}}{\pgfqpoint{-0.031064in}{-0.018041in}}{\pgfqpoint{-0.024552in}{-0.024552in}}%
\pgfpathcurveto{\pgfqpoint{-0.018041in}{-0.031064in}}{\pgfqpoint{-0.009208in}{-0.034722in}}{\pgfqpoint{0.000000in}{-0.034722in}}%
\pgfpathlineto{\pgfqpoint{0.000000in}{-0.034722in}}%
\pgfpathclose%
\pgfusepath{stroke,fill}%
}%
\begin{pgfscope}%
\pgfsys@transformshift{0.533579in}{0.451581in}%
\pgfsys@useobject{currentmarker}{}%
\end{pgfscope}%
\begin{pgfscope}%
\pgfsys@transformshift{0.641540in}{0.451410in}%
\pgfsys@useobject{currentmarker}{}%
\end{pgfscope}%
\begin{pgfscope}%
\pgfsys@transformshift{0.821474in}{0.451450in}%
\pgfsys@useobject{currentmarker}{}%
\end{pgfscope}%
\begin{pgfscope}%
\pgfsys@transformshift{1.073383in}{0.451398in}%
\pgfsys@useobject{currentmarker}{}%
\end{pgfscope}%
\begin{pgfscope}%
\pgfsys@transformshift{1.397265in}{0.451393in}%
\pgfsys@useobject{currentmarker}{}%
\end{pgfscope}%
\begin{pgfscope}%
\pgfsys@transformshift{1.793120in}{0.451391in}%
\pgfsys@useobject{currentmarker}{}%
\end{pgfscope}%
\begin{pgfscope}%
\pgfsys@transformshift{2.260950in}{0.451390in}%
\pgfsys@useobject{currentmarker}{}%
\end{pgfscope}%
\begin{pgfscope}%
\pgfsys@transformshift{2.800753in}{0.451389in}%
\pgfsys@useobject{currentmarker}{}%
\end{pgfscope}%
\begin{pgfscope}%
\pgfsys@transformshift{3.412530in}{0.451389in}%
\pgfsys@useobject{currentmarker}{}%
\end{pgfscope}%
\begin{pgfscope}%
\pgfsys@transformshift{4.096281in}{0.451389in}%
\pgfsys@useobject{currentmarker}{}%
\end{pgfscope}%
\end{pgfscope}%
\begin{pgfscope}%
\pgfsetroundcap%
\pgfsetroundjoin%
\pgfsetlinewidth{1.003750pt}%
\definecolor{currentstroke}{rgb}{0.792157,0.568627,0.380392}%
\pgfsetstrokecolor{currentstroke}%
\pgfsetdash{}{0pt}%
\pgfpathmoveto{\pgfqpoint{0.533579in}{0.451576in}}%
\pgfpathlineto{\pgfqpoint{0.641540in}{0.451415in}}%
\pgfpathlineto{\pgfqpoint{0.821474in}{0.451400in}}%
\pgfpathlineto{\pgfqpoint{1.073383in}{0.451393in}}%
\pgfpathlineto{\pgfqpoint{1.397265in}{0.451391in}}%
\pgfpathlineto{\pgfqpoint{1.793120in}{0.451390in}}%
\pgfpathlineto{\pgfqpoint{2.260950in}{0.451390in}}%
\pgfpathlineto{\pgfqpoint{2.800753in}{0.451389in}}%
\pgfpathlineto{\pgfqpoint{3.412530in}{0.451389in}}%
\pgfpathlineto{\pgfqpoint{4.096281in}{0.451389in}}%
\pgfusepath{stroke}%
\end{pgfscope}%
\begin{pgfscope}%
\pgfsetbuttcap%
\pgfsetroundjoin%
\definecolor{currentfill}{rgb}{0.792157,0.568627,0.380392}%
\pgfsetfillcolor{currentfill}%
\pgfsetlinewidth{0.752812pt}%
\definecolor{currentstroke}{rgb}{1.000000,1.000000,1.000000}%
\pgfsetstrokecolor{currentstroke}%
\pgfsetdash{}{0pt}%
\pgfsys@defobject{currentmarker}{\pgfqpoint{-0.034722in}{-0.034722in}}{\pgfqpoint{0.034722in}{0.034722in}}{%
\pgfpathmoveto{\pgfqpoint{0.000000in}{-0.034722in}}%
\pgfpathcurveto{\pgfqpoint{0.009208in}{-0.034722in}}{\pgfqpoint{0.018041in}{-0.031064in}}{\pgfqpoint{0.024552in}{-0.024552in}}%
\pgfpathcurveto{\pgfqpoint{0.031064in}{-0.018041in}}{\pgfqpoint{0.034722in}{-0.009208in}}{\pgfqpoint{0.034722in}{0.000000in}}%
\pgfpathcurveto{\pgfqpoint{0.034722in}{0.009208in}}{\pgfqpoint{0.031064in}{0.018041in}}{\pgfqpoint{0.024552in}{0.024552in}}%
\pgfpathcurveto{\pgfqpoint{0.018041in}{0.031064in}}{\pgfqpoint{0.009208in}{0.034722in}}{\pgfqpoint{0.000000in}{0.034722in}}%
\pgfpathcurveto{\pgfqpoint{-0.009208in}{0.034722in}}{\pgfqpoint{-0.018041in}{0.031064in}}{\pgfqpoint{-0.024552in}{0.024552in}}%
\pgfpathcurveto{\pgfqpoint{-0.031064in}{0.018041in}}{\pgfqpoint{-0.034722in}{0.009208in}}{\pgfqpoint{-0.034722in}{0.000000in}}%
\pgfpathcurveto{\pgfqpoint{-0.034722in}{-0.009208in}}{\pgfqpoint{-0.031064in}{-0.018041in}}{\pgfqpoint{-0.024552in}{-0.024552in}}%
\pgfpathcurveto{\pgfqpoint{-0.018041in}{-0.031064in}}{\pgfqpoint{-0.009208in}{-0.034722in}}{\pgfqpoint{0.000000in}{-0.034722in}}%
\pgfpathlineto{\pgfqpoint{0.000000in}{-0.034722in}}%
\pgfpathclose%
\pgfusepath{stroke,fill}%
}%
\begin{pgfscope}%
\pgfsys@transformshift{0.533579in}{0.451576in}%
\pgfsys@useobject{currentmarker}{}%
\end{pgfscope}%
\begin{pgfscope}%
\pgfsys@transformshift{0.641540in}{0.451415in}%
\pgfsys@useobject{currentmarker}{}%
\end{pgfscope}%
\begin{pgfscope}%
\pgfsys@transformshift{0.821474in}{0.451400in}%
\pgfsys@useobject{currentmarker}{}%
\end{pgfscope}%
\begin{pgfscope}%
\pgfsys@transformshift{1.073383in}{0.451393in}%
\pgfsys@useobject{currentmarker}{}%
\end{pgfscope}%
\begin{pgfscope}%
\pgfsys@transformshift{1.397265in}{0.451391in}%
\pgfsys@useobject{currentmarker}{}%
\end{pgfscope}%
\begin{pgfscope}%
\pgfsys@transformshift{1.793120in}{0.451390in}%
\pgfsys@useobject{currentmarker}{}%
\end{pgfscope}%
\begin{pgfscope}%
\pgfsys@transformshift{2.260950in}{0.451390in}%
\pgfsys@useobject{currentmarker}{}%
\end{pgfscope}%
\begin{pgfscope}%
\pgfsys@transformshift{2.800753in}{0.451389in}%
\pgfsys@useobject{currentmarker}{}%
\end{pgfscope}%
\begin{pgfscope}%
\pgfsys@transformshift{3.412530in}{0.451389in}%
\pgfsys@useobject{currentmarker}{}%
\end{pgfscope}%
\begin{pgfscope}%
\pgfsys@transformshift{4.096281in}{0.451389in}%
\pgfsys@useobject{currentmarker}{}%
\end{pgfscope}%
\end{pgfscope}%
\end{pgfpicture}%
\makeatother%
\endgroup%

%					\end{figcenter}
%					\caption{Relative share of each task on the total import time.}
%				\end{subfigure}
			\end{figcenter}
			\caption{Details of graph generation times for the two datasets \enquote{OSM city} (above) and \enquote{OSM rural} (below).}
			\label{fig:eval-import-details}
		\end{figure}
		
		In \Cref{fig:eval-import-details} both OSM dataset import times are split into the performed tasks.
		The task with the largest effort in terms of the required time is the $k$ nearest neighbor (kNN) search.
		This step takes, even for the smallest dataset, at least 50\% of the overall hybrid visibility graph generation time.
		For the largest dataset in the \enquote{OSM city} set, the share of the kNN search on the total processing time was over 97\%.
		The second most time consuming step is the merge of the road network with the visibility graph which has a share of not more than 3.5\% in any of the \enquote{OSM city} datasets.
		All other steps have a negligible effect on the processing performance with a combined share of less than 0.8\% on the total processing time of the \enquote{OSM city} dataset.
		
		Very small datasets, meaning only a few thousand vertices, show a different distribution.
		Even though the kNN search is still the most expensive task, the relative shares of other tasks, especially the marge operation, are significantly higher.
		\todo[inline]{reasons for this}
		
		\todo{Table with values? Absolute or relative?}
	
	\subsection{Routing}
		
		\todo{General routing times}
		
		\todo{Detailed view (as above)}
		
		\todo{Times per meter?}
	
\section{Routing evaluation}

	\subsection{Correctness}
	
		% is the shortest route really the shortest

	\subsection{Usefulness of routes}
	
		% How realistic are the routes (= can I go there in real life)? If not: Why not?
		% How much shorter are the routes?