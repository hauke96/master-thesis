% !TEX root = ../thesis.tex
% !TeX spellcheck = en_US

The previously shown design and implementation of the hybrid routing algorithm was evaluated regarding performance and usefulness.
This chapter covers the two aspects of performance and usefulness separately.
For both parts, method and design details are given followed by the respective results of the evaluations.

\section{Performance evaluation}

	The performance evaluation uses different datasets to measure graph generation and routing times.
	Each of these two steps is measured more fine-grained by measuring the execution times of separate method calls.
	The datasets have different properties and sizes and consist of artificial and real world data.

	\subsection{Methods \& Measurements}

		% what was measured?
		\subsubsection{Collected data}
		
			The collected data consists of time measurement, many of them on the level of separate methods.
			Also, the amount of data is measures, namely the number of edges and vertices at various steps in the process.
			
			As a result, two CSV files are written per dataset containing measurement data for the import (including graph generation) and routing.
			The measurement for the routing requests also contains information about the lengths of the routes, especially beeline and actual route distances.
			
			% TODO table with all columns of the measured data including a description and example value?
		
		% datasets
		\subsubsection{Datasets}
		
			There are multiple categories of datasets that were used:
			
			\begin{description}
				\item[Maze pattern] A set of datasets containing maze like geometries, namely only connected linestrings forming a seamless pattern tile that can arbitrarily often be repeated. Many of the contained line obstacles are collinear. No roads are within these datasets.
				\item[Rectangle pattern] Also a set of pattern-based datasets, which contain simple rectangles of different sizes. No roads are within these datasets.
				\item[Circle pattern] Like the rectangle datasets but with circles. These datasets have large number of vertices.
				\item[OSM city] This is a real world export from the OpenStreetMap database with data from the city of Hamburg, Germany. The data has been filtered to remove all over- and underground features since the hybrid routing algorithm has no handling of three dimensional data. This dataset also contains all roads and ways in the respective region.
				\item[OSM rural] Equivalent to the \enquote{OSM city} dataset, but located outside the city with more natural obstacles (lakes, ditches, forest), more open spaces and less regular buildings.
				\item[OSM export without roads] Same data from the above OSM exports, but without the roads. This is used to show the overhead of the road on the graph generation and routing times.
				\item[OSM export without obstacles] Same data from the above OSM exports, but without the obstacles, i.e. buildings, walls and water areas. This is used to show the overhead of the obstacles on the graph generation and routing times.
			\end{description}
			
			The pattern datasets exist in different sizes, since the pattern can arbitrarily often be repeated.
			
			\todo{Illustrate this as a diagram or list? Also list the number of raw vertices within the datasets.}
			
			\todo{List sizes/statistics of datasets?}
		
		% optimizations
		\subsubsection{Optimizations}
		
			As \cref{chap:implementation} already mentioned, there were several optimizations made to the implementation.
			Some of which are on the level of data structures, some on the algorithmic level.
			The effectiveness of these optimization was also evaluated using the OSM dataset.
			Each of the following optimizations was deactivated for the evaluation:

			\begin{description}
				\item[Shadow areas] Instead, every visibility check was performed using the custom intersection check described in \cref{subsubsec:intersection-checks}.
				\item[Custom intersection check] The custom intersection check was replaced by the \todo[inline]{probably \texttt{RobustLineIntersector}?} class from the NTS to determine intersections between line segments.
				\item[BinIndex] Instead, the \texttt{BinTree} from the NTS was used.
				\item[Convex hull] The restriction to only consider vertices on the convex hull of obstacles was removed.
				\item[Valid angle areas] Considering only potential visibility neighbors within certain angular ranges was deactivated.
				\item[$k$-NN search] The $k$ of the k-NN search was deactivated to determine all visibility neighbors in all directions.
			\end{description}			
		% how were times measured? HikerModel
		\subsubsection{Measurement method}
		
			Measuring the performance was done by a small agent-based simulation project called \texttt{HikerModel}, which consists of one agent, a list of coordinates and the input dataset.
			Each coordinate from the input list is visited by the agent using the hybrid routing algorithm to determine the path to the next location.
			The time of the graph generation as well as the time of each routing request are measured.
		
		% changes to the code
		% difficulties in C#
		\subsubsection{Technical considerations}
		
			The measurement was done by a helper class \texttt{PerformanceMeasurement} providing a method that accepts an arbitrary function which execution time should be measured.
			
			Unfortunately, C\# does not provide any method to disable the garbage collector.
			Also, the just-in-time (JIT) compilation cannot be turned of for normal .NET executions via \texttt{dotnet program.dll}.
			The alternative would be an ahead-of-time (AOT) compilation, which slows down LINQ operations.
			\todo[inline]{source on that + test it}

			Because both compilation strategies have disadvantages, the normal .NET-based execution was chosen, even though it contains JIT compilation.
			To mitigate this dynamic behavior and to generally get resilient results, three warm-up iterations were performed before measuring the times of five actual execution iterations.
			Even though only the results from the five actual iterations were part of the result, the times from the warm-up iterations showed that three iterations are enough to prepare the runtime and garbage collector.
			
			Additionally, the garbage collector was triggered during each of the eight iterations just before calling the function to measure.
			This is done by the \texttt{GC.Collect()} and \texttt{GC.WaitForPendingFinalizers()} from the \texttt{GC} class of the .NET framework.
			Using these two methods forces a garbage collection and waits for it to finish \cite{ms-gc}.
			
			To prevent the garbage collection from interfering with the execution, a 256 MiB large no-GC-region is placed around the function call via \texttt{GC.TryStartNoGCRegion(256 * 1024 * 1024)}.
			Although this only works if enough memory is available \cite{ms-no-gc-region}, the system had this memory and introduction this no-GC-region significantly stabilized the results.
			\todo[inline]{test this and maybe give an example result (standard deviation or so?)}
			
			Another step is the increase of the process priority.
			This ideally leads to an exclusive use of one CPU core on which the single threaded application runs.
			Increasing the process priority is done by settings the \texttt{PriorityClass} of the current process to \texttt{ProcessPriorityClass.High}.
		
		\subsubsection{System and hardware}
		
			The measurements were performed on an up-to-date Arch Linux operating system with .NET Core 7.0.105 and MARS framework 4.5.2.
			Apart from necessary operating system processes and the desktop environment, no other load intensive processes ran during the performance measurements.
			
			The hardware consisted of an octa core Intel\textregistered\ Xeon\textregistered\ E3-1231 v3 CPU at 3.40 GHz, a total of 16GB DDR3 1333 MHz RAM and a Samsung EVO 850 SSD.
			However, the whole algorithm and the \texttt{HikerModel} simulation are single threaded and file system operations are only performed to initially read the input data and to write the measurement results at the end.
	
	\subsection{Import and graph generation}
	
		As mentioned at the beginning of \cref{subsec:related-work:visibility-graph}, the process of generating a visibility graph has an inherent quadratic runtime.
		This result is clearly visible in measurements of the dataset imports as seen in \cref{fig:eval-import-city-abs} and \cref{fig:eval-import-rural-abs}.
		
		\begin{figure}[h]
			\begin{minipage}{.48\textwidth}
				\begin{subfigure}[t]{\linewidth}
					\begin{figcenter}
						\input{images/evaluation/osm-based-city/plot-iteration-time-per-vertices.py_absolute.pgf}
					\end{figcenter}
					\caption{Absolute total graph generation time for the \enquote{OSM city} dataset.}
					\label{fig:eval-import-city-abs}
				\end{subfigure}
				\\[3ex]
				\begin{subfigure}[t]{\linewidth}
					\begin{figcenter}
						%% Creator: Matplotlib, PGF backend
%%
%% To include the figure in your LaTeX document, write
%%   \input{<filename>.pgf}
%%
%% Make sure the required packages are loaded in your preamble
%%   \usepackage{pgf}
%%
%% Also ensure that all the required font packages are loaded; for instance,
%% the lmodern package is sometimes necessary when using math font.
%%   \usepackage{lmodern}
%%
%% Figures using additional raster images can only be included by \input if
%% they are in the same directory as the main LaTeX file. For loading figures
%% from other directories you can use the `import` package
%%   \usepackage{import}
%%
%% and then include the figures with
%%   \import{<path to file>}{<filename>.pgf}
%%
%% Matplotlib used the following preamble
%%   
%%   \usepackage{fontspec}
%%   \setmainfont{DejaVuSerif.ttf}[Path=\detokenize{/home/hauke/.local/lib/python3.11/site-packages/matplotlib/mpl-data/fonts/ttf/}]
%%   \setsansfont{DroidSans.ttf}[Path=\detokenize{/usr/share/fonts/droid/}]
%%   \setmonofont{DejaVuSansMono.ttf}[Path=\detokenize{/home/hauke/.local/lib/python3.11/site-packages/matplotlib/mpl-data/fonts/ttf/}]
%%   \makeatletter\@ifpackageloaded{underscore}{}{\usepackage[strings]{underscore}}\makeatother
%%
\begingroup%
\makeatletter%
\begin{pgfpicture}%
\pgfpathrectangle{\pgfpointorigin}{\pgfqpoint{2.682524in}{1.770898in}}%
\pgfusepath{use as bounding box, clip}%
\begin{pgfscope}%
\pgfsetbuttcap%
\pgfsetmiterjoin%
\definecolor{currentfill}{rgb}{1.000000,1.000000,1.000000}%
\pgfsetfillcolor{currentfill}%
\pgfsetlinewidth{0.000000pt}%
\definecolor{currentstroke}{rgb}{1.000000,1.000000,1.000000}%
\pgfsetstrokecolor{currentstroke}%
\pgfsetdash{}{0pt}%
\pgfpathmoveto{\pgfqpoint{0.000000in}{0.000000in}}%
\pgfpathlineto{\pgfqpoint{2.682524in}{0.000000in}}%
\pgfpathlineto{\pgfqpoint{2.682524in}{1.770898in}}%
\pgfpathlineto{\pgfqpoint{0.000000in}{1.770898in}}%
\pgfpathlineto{\pgfqpoint{0.000000in}{0.000000in}}%
\pgfpathclose%
\pgfusepath{fill}%
\end{pgfscope}%
\begin{pgfscope}%
\pgfsetbuttcap%
\pgfsetmiterjoin%
\definecolor{currentfill}{rgb}{1.000000,1.000000,1.000000}%
\pgfsetfillcolor{currentfill}%
\pgfsetlinewidth{0.000000pt}%
\definecolor{currentstroke}{rgb}{0.000000,0.000000,0.000000}%
\pgfsetstrokecolor{currentstroke}%
\pgfsetstrokeopacity{0.000000}%
\pgfsetdash{}{0pt}%
\pgfpathmoveto{\pgfqpoint{0.566440in}{0.451389in}}%
\pgfpathlineto{\pgfqpoint{2.682524in}{0.451389in}}%
\pgfpathlineto{\pgfqpoint{2.682524in}{1.770898in}}%
\pgfpathlineto{\pgfqpoint{0.566440in}{1.770898in}}%
\pgfpathlineto{\pgfqpoint{0.566440in}{0.451389in}}%
\pgfpathclose%
\pgfusepath{fill}%
\end{pgfscope}%
\begin{pgfscope}%
\pgfpathrectangle{\pgfqpoint{0.566440in}{0.451389in}}{\pgfqpoint{2.116084in}{1.319509in}}%
\pgfusepath{clip}%
\pgfsetroundcap%
\pgfsetroundjoin%
\pgfsetlinewidth{1.003750pt}%
\definecolor{currentstroke}{rgb}{0.800000,0.800000,0.800000}%
\pgfsetstrokecolor{currentstroke}%
\pgfsetdash{}{0pt}%
\pgfpathmoveto{\pgfqpoint{0.566440in}{0.451389in}}%
\pgfpathlineto{\pgfqpoint{0.566440in}{1.770898in}}%
\pgfusepath{stroke}%
\end{pgfscope}%
\begin{pgfscope}%
\definecolor{textcolor}{rgb}{0.150000,0.150000,0.150000}%
\pgfsetstrokecolor{textcolor}%
\pgfsetfillcolor{textcolor}%
\pgftext[x=0.566440in,y=0.319444in,,top]{\color{textcolor}\sffamily\fontsize{9.000000}{10.800000}\selectfont 0}%
\end{pgfscope}%
\begin{pgfscope}%
\pgfpathrectangle{\pgfqpoint{0.566440in}{0.451389in}}{\pgfqpoint{2.116084in}{1.319509in}}%
\pgfusepath{clip}%
\pgfsetroundcap%
\pgfsetroundjoin%
\pgfsetlinewidth{1.003750pt}%
\definecolor{currentstroke}{rgb}{0.800000,0.800000,0.800000}%
\pgfsetstrokecolor{currentstroke}%
\pgfsetdash{}{0pt}%
\pgfpathmoveto{\pgfqpoint{1.025441in}{0.451389in}}%
\pgfpathlineto{\pgfqpoint{1.025441in}{1.770898in}}%
\pgfusepath{stroke}%
\end{pgfscope}%
\begin{pgfscope}%
\definecolor{textcolor}{rgb}{0.150000,0.150000,0.150000}%
\pgfsetstrokecolor{textcolor}%
\pgfsetfillcolor{textcolor}%
\pgftext[x=1.025441in,y=0.319444in,,top]{\color{textcolor}\sffamily\fontsize{9.000000}{10.800000}\selectfont 20000}%
\end{pgfscope}%
\begin{pgfscope}%
\pgfpathrectangle{\pgfqpoint{0.566440in}{0.451389in}}{\pgfqpoint{2.116084in}{1.319509in}}%
\pgfusepath{clip}%
\pgfsetroundcap%
\pgfsetroundjoin%
\pgfsetlinewidth{1.003750pt}%
\definecolor{currentstroke}{rgb}{0.800000,0.800000,0.800000}%
\pgfsetstrokecolor{currentstroke}%
\pgfsetdash{}{0pt}%
\pgfpathmoveto{\pgfqpoint{1.484442in}{0.451389in}}%
\pgfpathlineto{\pgfqpoint{1.484442in}{1.770898in}}%
\pgfusepath{stroke}%
\end{pgfscope}%
\begin{pgfscope}%
\definecolor{textcolor}{rgb}{0.150000,0.150000,0.150000}%
\pgfsetstrokecolor{textcolor}%
\pgfsetfillcolor{textcolor}%
\pgftext[x=1.484442in,y=0.319444in,,top]{\color{textcolor}\sffamily\fontsize{9.000000}{10.800000}\selectfont 40000}%
\end{pgfscope}%
\begin{pgfscope}%
\pgfpathrectangle{\pgfqpoint{0.566440in}{0.451389in}}{\pgfqpoint{2.116084in}{1.319509in}}%
\pgfusepath{clip}%
\pgfsetroundcap%
\pgfsetroundjoin%
\pgfsetlinewidth{1.003750pt}%
\definecolor{currentstroke}{rgb}{0.800000,0.800000,0.800000}%
\pgfsetstrokecolor{currentstroke}%
\pgfsetdash{}{0pt}%
\pgfpathmoveto{\pgfqpoint{1.943442in}{0.451389in}}%
\pgfpathlineto{\pgfqpoint{1.943442in}{1.770898in}}%
\pgfusepath{stroke}%
\end{pgfscope}%
\begin{pgfscope}%
\definecolor{textcolor}{rgb}{0.150000,0.150000,0.150000}%
\pgfsetstrokecolor{textcolor}%
\pgfsetfillcolor{textcolor}%
\pgftext[x=1.943442in,y=0.319444in,,top]{\color{textcolor}\sffamily\fontsize{9.000000}{10.800000}\selectfont 60000}%
\end{pgfscope}%
\begin{pgfscope}%
\pgfpathrectangle{\pgfqpoint{0.566440in}{0.451389in}}{\pgfqpoint{2.116084in}{1.319509in}}%
\pgfusepath{clip}%
\pgfsetroundcap%
\pgfsetroundjoin%
\pgfsetlinewidth{1.003750pt}%
\definecolor{currentstroke}{rgb}{0.800000,0.800000,0.800000}%
\pgfsetstrokecolor{currentstroke}%
\pgfsetdash{}{0pt}%
\pgfpathmoveto{\pgfqpoint{2.402443in}{0.451389in}}%
\pgfpathlineto{\pgfqpoint{2.402443in}{1.770898in}}%
\pgfusepath{stroke}%
\end{pgfscope}%
\begin{pgfscope}%
\definecolor{textcolor}{rgb}{0.150000,0.150000,0.150000}%
\pgfsetstrokecolor{textcolor}%
\pgfsetfillcolor{textcolor}%
\pgftext[x=2.402443in,y=0.319444in,,top]{\color{textcolor}\sffamily\fontsize{9.000000}{10.800000}\selectfont 80000}%
\end{pgfscope}%
\begin{pgfscope}%
\definecolor{textcolor}{rgb}{0.150000,0.150000,0.150000}%
\pgfsetstrokecolor{textcolor}%
\pgfsetfillcolor{textcolor}%
\pgftext[x=1.624482in,y=0.125000in,,top]{\color{textcolor}\sffamily\fontsize{9.000000}{10.800000}\selectfont Input obstacle vertices}%
\end{pgfscope}%
\begin{pgfscope}%
\pgfpathrectangle{\pgfqpoint{0.566440in}{0.451389in}}{\pgfqpoint{2.116084in}{1.319509in}}%
\pgfusepath{clip}%
\pgfsetroundcap%
\pgfsetroundjoin%
\pgfsetlinewidth{1.003750pt}%
\definecolor{currentstroke}{rgb}{0.800000,0.800000,0.800000}%
\pgfsetstrokecolor{currentstroke}%
\pgfsetdash{}{0pt}%
\pgfpathmoveto{\pgfqpoint{0.566440in}{0.451389in}}%
\pgfpathlineto{\pgfqpoint{2.682524in}{0.451389in}}%
\pgfusepath{stroke}%
\end{pgfscope}%
\begin{pgfscope}%
\definecolor{textcolor}{rgb}{0.150000,0.150000,0.150000}%
\pgfsetstrokecolor{textcolor}%
\pgfsetfillcolor{textcolor}%
\pgftext[x=0.263292in, y=0.403903in, left, base]{\color{textcolor}\sffamily\fontsize{9.000000}{10.800000}\selectfont 0.0}%
\end{pgfscope}%
\begin{pgfscope}%
\pgfpathrectangle{\pgfqpoint{0.566440in}{0.451389in}}{\pgfqpoint{2.116084in}{1.319509in}}%
\pgfusepath{clip}%
\pgfsetroundcap%
\pgfsetroundjoin%
\pgfsetlinewidth{1.003750pt}%
\definecolor{currentstroke}{rgb}{0.800000,0.800000,0.800000}%
\pgfsetstrokecolor{currentstroke}%
\pgfsetdash{}{0pt}%
\pgfpathmoveto{\pgfqpoint{0.566440in}{0.744703in}}%
\pgfpathlineto{\pgfqpoint{2.682524in}{0.744703in}}%
\pgfusepath{stroke}%
\end{pgfscope}%
\begin{pgfscope}%
\definecolor{textcolor}{rgb}{0.150000,0.150000,0.150000}%
\pgfsetstrokecolor{textcolor}%
\pgfsetfillcolor{textcolor}%
\pgftext[x=0.263292in, y=0.697217in, left, base]{\color{textcolor}\sffamily\fontsize{9.000000}{10.800000}\selectfont 2.5}%
\end{pgfscope}%
\begin{pgfscope}%
\pgfpathrectangle{\pgfqpoint{0.566440in}{0.451389in}}{\pgfqpoint{2.116084in}{1.319509in}}%
\pgfusepath{clip}%
\pgfsetroundcap%
\pgfsetroundjoin%
\pgfsetlinewidth{1.003750pt}%
\definecolor{currentstroke}{rgb}{0.800000,0.800000,0.800000}%
\pgfsetstrokecolor{currentstroke}%
\pgfsetdash{}{0pt}%
\pgfpathmoveto{\pgfqpoint{0.566440in}{1.038017in}}%
\pgfpathlineto{\pgfqpoint{2.682524in}{1.038017in}}%
\pgfusepath{stroke}%
\end{pgfscope}%
\begin{pgfscope}%
\definecolor{textcolor}{rgb}{0.150000,0.150000,0.150000}%
\pgfsetstrokecolor{textcolor}%
\pgfsetfillcolor{textcolor}%
\pgftext[x=0.263292in, y=0.990532in, left, base]{\color{textcolor}\sffamily\fontsize{9.000000}{10.800000}\selectfont 5.0}%
\end{pgfscope}%
\begin{pgfscope}%
\pgfpathrectangle{\pgfqpoint{0.566440in}{0.451389in}}{\pgfqpoint{2.116084in}{1.319509in}}%
\pgfusepath{clip}%
\pgfsetroundcap%
\pgfsetroundjoin%
\pgfsetlinewidth{1.003750pt}%
\definecolor{currentstroke}{rgb}{0.800000,0.800000,0.800000}%
\pgfsetstrokecolor{currentstroke}%
\pgfsetdash{}{0pt}%
\pgfpathmoveto{\pgfqpoint{0.566440in}{1.331331in}}%
\pgfpathlineto{\pgfqpoint{2.682524in}{1.331331in}}%
\pgfusepath{stroke}%
\end{pgfscope}%
\begin{pgfscope}%
\definecolor{textcolor}{rgb}{0.150000,0.150000,0.150000}%
\pgfsetstrokecolor{textcolor}%
\pgfsetfillcolor{textcolor}%
\pgftext[x=0.263292in, y=1.283846in, left, base]{\color{textcolor}\sffamily\fontsize{9.000000}{10.800000}\selectfont 7.5}%
\end{pgfscope}%
\begin{pgfscope}%
\pgfpathrectangle{\pgfqpoint{0.566440in}{0.451389in}}{\pgfqpoint{2.116084in}{1.319509in}}%
\pgfusepath{clip}%
\pgfsetroundcap%
\pgfsetroundjoin%
\pgfsetlinewidth{1.003750pt}%
\definecolor{currentstroke}{rgb}{0.800000,0.800000,0.800000}%
\pgfsetstrokecolor{currentstroke}%
\pgfsetdash{}{0pt}%
\pgfpathmoveto{\pgfqpoint{0.566440in}{1.624645in}}%
\pgfpathlineto{\pgfqpoint{2.682524in}{1.624645in}}%
\pgfusepath{stroke}%
\end{pgfscope}%
\begin{pgfscope}%
\definecolor{textcolor}{rgb}{0.150000,0.150000,0.150000}%
\pgfsetstrokecolor{textcolor}%
\pgfsetfillcolor{textcolor}%
\pgftext[x=0.194444in, y=1.577160in, left, base]{\color{textcolor}\sffamily\fontsize{9.000000}{10.800000}\selectfont 10.0}%
\end{pgfscope}%
\begin{pgfscope}%
\definecolor{textcolor}{rgb}{0.150000,0.150000,0.150000}%
\pgfsetstrokecolor{textcolor}%
\pgfsetfillcolor{textcolor}%
\pgftext[x=0.125000in,y=1.111143in,,bottom,rotate=90.000000]{\color{textcolor}\sffamily\fontsize{9.000000}{10.800000}\selectfont Time in ms}%
\end{pgfscope}%
\begin{pgfscope}%
\pgfpathrectangle{\pgfqpoint{0.566440in}{0.451389in}}{\pgfqpoint{2.116084in}{1.319509in}}%
\pgfusepath{clip}%
\pgfsetbuttcap%
\pgfsetroundjoin%
\definecolor{currentfill}{rgb}{0.003922,0.450980,0.698039}%
\pgfsetfillcolor{currentfill}%
\pgfsetfillopacity{0.200000}%
\pgfsetlinewidth{1.003750pt}%
\definecolor{currentstroke}{rgb}{0.003922,0.450980,0.698039}%
\pgfsetstrokecolor{currentstroke}%
\pgfsetstrokeopacity{0.200000}%
\pgfsetdash{}{0pt}%
\pgfsys@defobject{currentmarker}{\pgfqpoint{0.887764in}{0.609974in}}{\pgfqpoint{2.597060in}{1.715616in}}{%
\pgfpathmoveto{\pgfqpoint{0.887764in}{0.613973in}}%
\pgfpathlineto{\pgfqpoint{0.887764in}{0.609974in}}%
\pgfpathlineto{\pgfqpoint{1.147283in}{0.767000in}}%
\pgfpathlineto{\pgfqpoint{1.373249in}{0.935656in}}%
\pgfpathlineto{\pgfqpoint{1.593431in}{1.152457in}}%
\pgfpathlineto{\pgfqpoint{2.110175in}{1.426811in}}%
\pgfpathlineto{\pgfqpoint{2.597060in}{1.680963in}}%
\pgfpathlineto{\pgfqpoint{2.597060in}{1.715616in}}%
\pgfpathlineto{\pgfqpoint{2.597060in}{1.715616in}}%
\pgfpathlineto{\pgfqpoint{2.110175in}{1.439807in}}%
\pgfpathlineto{\pgfqpoint{1.593431in}{1.157148in}}%
\pgfpathlineto{\pgfqpoint{1.373249in}{0.947175in}}%
\pgfpathlineto{\pgfqpoint{1.147283in}{0.772201in}}%
\pgfpathlineto{\pgfqpoint{0.887764in}{0.613973in}}%
\pgfpathlineto{\pgfqpoint{0.887764in}{0.613973in}}%
\pgfpathclose%
\pgfusepath{stroke,fill}%
}%
\begin{pgfscope}%
\pgfsys@transformshift{0.000000in}{0.000000in}%
\pgfsys@useobject{currentmarker}{}%
\end{pgfscope}%
\end{pgfscope}%
\begin{pgfscope}%
\pgfsetrectcap%
\pgfsetmiterjoin%
\pgfsetlinewidth{1.254687pt}%
\definecolor{currentstroke}{rgb}{0.800000,0.800000,0.800000}%
\pgfsetstrokecolor{currentstroke}%
\pgfsetdash{}{0pt}%
\pgfpathmoveto{\pgfqpoint{0.566440in}{0.451389in}}%
\pgfpathlineto{\pgfqpoint{0.566440in}{1.770898in}}%
\pgfusepath{stroke}%
\end{pgfscope}%
\begin{pgfscope}%
\pgfsetrectcap%
\pgfsetmiterjoin%
\pgfsetlinewidth{1.254687pt}%
\definecolor{currentstroke}{rgb}{0.800000,0.800000,0.800000}%
\pgfsetstrokecolor{currentstroke}%
\pgfsetdash{}{0pt}%
\pgfpathmoveto{\pgfqpoint{2.682524in}{0.451389in}}%
\pgfpathlineto{\pgfqpoint{2.682524in}{1.770898in}}%
\pgfusepath{stroke}%
\end{pgfscope}%
\begin{pgfscope}%
\pgfsetrectcap%
\pgfsetmiterjoin%
\pgfsetlinewidth{1.254687pt}%
\definecolor{currentstroke}{rgb}{0.800000,0.800000,0.800000}%
\pgfsetstrokecolor{currentstroke}%
\pgfsetdash{}{0pt}%
\pgfpathmoveto{\pgfqpoint{0.566440in}{0.451389in}}%
\pgfpathlineto{\pgfqpoint{2.682524in}{0.451389in}}%
\pgfusepath{stroke}%
\end{pgfscope}%
\begin{pgfscope}%
\pgfsetrectcap%
\pgfsetmiterjoin%
\pgfsetlinewidth{1.254687pt}%
\definecolor{currentstroke}{rgb}{0.800000,0.800000,0.800000}%
\pgfsetstrokecolor{currentstroke}%
\pgfsetdash{}{0pt}%
\pgfpathmoveto{\pgfqpoint{0.566440in}{1.770898in}}%
\pgfpathlineto{\pgfqpoint{2.682524in}{1.770898in}}%
\pgfusepath{stroke}%
\end{pgfscope}%
\begin{pgfscope}%
\pgfsetroundcap%
\pgfsetroundjoin%
\pgfsetlinewidth{1.003750pt}%
\definecolor{currentstroke}{rgb}{0.003922,0.450980,0.698039}%
\pgfsetstrokecolor{currentstroke}%
\pgfsetdash{}{0pt}%
\pgfpathmoveto{\pgfqpoint{0.887764in}{0.611728in}}%
\pgfpathlineto{\pgfqpoint{1.147283in}{0.769536in}}%
\pgfpathlineto{\pgfqpoint{1.373249in}{0.942104in}}%
\pgfpathlineto{\pgfqpoint{1.593431in}{1.155204in}}%
\pgfpathlineto{\pgfqpoint{2.110175in}{1.434253in}}%
\pgfpathlineto{\pgfqpoint{2.597060in}{1.698659in}}%
\pgfusepath{stroke}%
\end{pgfscope}%
\begin{pgfscope}%
\pgfsetbuttcap%
\pgfsetroundjoin%
\definecolor{currentfill}{rgb}{0.003922,0.450980,0.698039}%
\pgfsetfillcolor{currentfill}%
\pgfsetlinewidth{0.752812pt}%
\definecolor{currentstroke}{rgb}{1.000000,1.000000,1.000000}%
\pgfsetstrokecolor{currentstroke}%
\pgfsetdash{}{0pt}%
\pgfsys@defobject{currentmarker}{\pgfqpoint{-0.034722in}{-0.034722in}}{\pgfqpoint{0.034722in}{0.034722in}}{%
\pgfpathmoveto{\pgfqpoint{0.000000in}{-0.034722in}}%
\pgfpathcurveto{\pgfqpoint{0.009208in}{-0.034722in}}{\pgfqpoint{0.018041in}{-0.031064in}}{\pgfqpoint{0.024552in}{-0.024552in}}%
\pgfpathcurveto{\pgfqpoint{0.031064in}{-0.018041in}}{\pgfqpoint{0.034722in}{-0.009208in}}{\pgfqpoint{0.034722in}{0.000000in}}%
\pgfpathcurveto{\pgfqpoint{0.034722in}{0.009208in}}{\pgfqpoint{0.031064in}{0.018041in}}{\pgfqpoint{0.024552in}{0.024552in}}%
\pgfpathcurveto{\pgfqpoint{0.018041in}{0.031064in}}{\pgfqpoint{0.009208in}{0.034722in}}{\pgfqpoint{0.000000in}{0.034722in}}%
\pgfpathcurveto{\pgfqpoint{-0.009208in}{0.034722in}}{\pgfqpoint{-0.018041in}{0.031064in}}{\pgfqpoint{-0.024552in}{0.024552in}}%
\pgfpathcurveto{\pgfqpoint{-0.031064in}{0.018041in}}{\pgfqpoint{-0.034722in}{0.009208in}}{\pgfqpoint{-0.034722in}{0.000000in}}%
\pgfpathcurveto{\pgfqpoint{-0.034722in}{-0.009208in}}{\pgfqpoint{-0.031064in}{-0.018041in}}{\pgfqpoint{-0.024552in}{-0.024552in}}%
\pgfpathcurveto{\pgfqpoint{-0.018041in}{-0.031064in}}{\pgfqpoint{-0.009208in}{-0.034722in}}{\pgfqpoint{0.000000in}{-0.034722in}}%
\pgfpathlineto{\pgfqpoint{0.000000in}{-0.034722in}}%
\pgfpathclose%
\pgfusepath{stroke,fill}%
}%
\begin{pgfscope}%
\pgfsys@transformshift{0.887764in}{0.611728in}%
\pgfsys@useobject{currentmarker}{}%
\end{pgfscope}%
\begin{pgfscope}%
\pgfsys@transformshift{1.147283in}{0.769536in}%
\pgfsys@useobject{currentmarker}{}%
\end{pgfscope}%
\begin{pgfscope}%
\pgfsys@transformshift{1.373249in}{0.942104in}%
\pgfsys@useobject{currentmarker}{}%
\end{pgfscope}%
\begin{pgfscope}%
\pgfsys@transformshift{1.593431in}{1.155204in}%
\pgfsys@useobject{currentmarker}{}%
\end{pgfscope}%
\begin{pgfscope}%
\pgfsys@transformshift{2.110175in}{1.434253in}%
\pgfsys@useobject{currentmarker}{}%
\end{pgfscope}%
\begin{pgfscope}%
\pgfsys@transformshift{2.597060in}{1.698659in}%
\pgfsys@useobject{currentmarker}{}%
\end{pgfscope}%
\end{pgfscope}%
\end{pgfpicture}%
\makeatother%
\endgroup%

					\end{figcenter}
					\caption{Time per input vertex for the \enquote{OSM city} dataset.}
				\end{subfigure}
				\\[3ex]
				\begin{subfigure}[t]{\linewidth}
					\begin{figcenter}
						\input{images/evaluation/osm-based-city/plot-iteration-time-per-vertices.py_per-vertex-added.pgf}
					\end{figcenter}
					\caption{Increase in processing time per vertex when an additional vertex is added.}
					\label{fig:eval-import-city-rel-increase}
				\end{subfigure}
				\caption{Graph generation times using the \enquote{OSM city} dataset.}
				\label{fig:eval-import-city}
			\end{minipage}
			\hfill
			\begin{minipage}{.48\textwidth}
				\begin{subfigure}[t]{\linewidth}
					\begin{figcenter}
						\input{images/evaluation/osm-based-rural/plot-iteration-time-per-vertices.py_absolute.pgf}
					\end{figcenter}
					\caption{Absolute total graph generation time for the \enquote{OSM rural} dataset.}
					\label{fig:eval-import-rural-abs}
				\end{subfigure}
				\\[3ex]
				\begin{subfigure}[t]{\linewidth}
					\begin{figcenter}
						%% Creator: Matplotlib, PGF backend
%%
%% To include the figure in your LaTeX document, write
%%   \input{<filename>.pgf}
%%
%% Make sure the required packages are loaded in your preamble
%%   \usepackage{pgf}
%%
%% Also ensure that all the required font packages are loaded; for instance,
%% the lmodern package is sometimes necessary when using math font.
%%   \usepackage{lmodern}
%%
%% Figures using additional raster images can only be included by \input if
%% they are in the same directory as the main LaTeX file. For loading figures
%% from other directories you can use the `import` package
%%   \usepackage{import}
%%
%% and then include the figures with
%%   \import{<path to file>}{<filename>.pgf}
%%
%% Matplotlib used the following preamble
%%   
%%   \usepackage{fontspec}
%%   \setmainfont{DejaVuSerif.ttf}[Path=\detokenize{/home/hauke/.local/lib/python3.11/site-packages/matplotlib/mpl-data/fonts/ttf/}]
%%   \setsansfont{DroidSans.ttf}[Path=\detokenize{/usr/share/fonts/droid/}]
%%   \setmonofont{DejaVuSansMono.ttf}[Path=\detokenize{/home/hauke/.local/lib/python3.11/site-packages/matplotlib/mpl-data/fonts/ttf/}]
%%   \makeatletter\@ifpackageloaded{underscore}{}{\usepackage[strings]{underscore}}\makeatother
%%
\begingroup%
\makeatletter%
\begin{pgfpicture}%
\pgfpathrectangle{\pgfpointorigin}{\pgfqpoint{2.682524in}{1.770898in}}%
\pgfusepath{use as bounding box, clip}%
\begin{pgfscope}%
\pgfsetbuttcap%
\pgfsetmiterjoin%
\definecolor{currentfill}{rgb}{1.000000,1.000000,1.000000}%
\pgfsetfillcolor{currentfill}%
\pgfsetlinewidth{0.000000pt}%
\definecolor{currentstroke}{rgb}{1.000000,1.000000,1.000000}%
\pgfsetstrokecolor{currentstroke}%
\pgfsetdash{}{0pt}%
\pgfpathmoveto{\pgfqpoint{0.000000in}{0.000000in}}%
\pgfpathlineto{\pgfqpoint{2.682524in}{0.000000in}}%
\pgfpathlineto{\pgfqpoint{2.682524in}{1.770898in}}%
\pgfpathlineto{\pgfqpoint{0.000000in}{1.770898in}}%
\pgfpathlineto{\pgfqpoint{0.000000in}{0.000000in}}%
\pgfpathclose%
\pgfusepath{fill}%
\end{pgfscope}%
\begin{pgfscope}%
\pgfsetbuttcap%
\pgfsetmiterjoin%
\definecolor{currentfill}{rgb}{1.000000,1.000000,1.000000}%
\pgfsetfillcolor{currentfill}%
\pgfsetlinewidth{0.000000pt}%
\definecolor{currentstroke}{rgb}{0.000000,0.000000,0.000000}%
\pgfsetstrokecolor{currentstroke}%
\pgfsetstrokeopacity{0.000000}%
\pgfsetdash{}{0pt}%
\pgfpathmoveto{\pgfqpoint{0.566440in}{0.451389in}}%
\pgfpathlineto{\pgfqpoint{2.682524in}{0.451389in}}%
\pgfpathlineto{\pgfqpoint{2.682524in}{1.770898in}}%
\pgfpathlineto{\pgfqpoint{0.566440in}{1.770898in}}%
\pgfpathlineto{\pgfqpoint{0.566440in}{0.451389in}}%
\pgfpathclose%
\pgfusepath{fill}%
\end{pgfscope}%
\begin{pgfscope}%
\pgfpathrectangle{\pgfqpoint{0.566440in}{0.451389in}}{\pgfqpoint{2.116084in}{1.319509in}}%
\pgfusepath{clip}%
\pgfsetroundcap%
\pgfsetroundjoin%
\pgfsetlinewidth{1.003750pt}%
\definecolor{currentstroke}{rgb}{0.800000,0.800000,0.800000}%
\pgfsetstrokecolor{currentstroke}%
\pgfsetdash{}{0pt}%
\pgfpathmoveto{\pgfqpoint{0.566440in}{0.451389in}}%
\pgfpathlineto{\pgfqpoint{0.566440in}{1.770898in}}%
\pgfusepath{stroke}%
\end{pgfscope}%
\begin{pgfscope}%
\definecolor{textcolor}{rgb}{0.150000,0.150000,0.150000}%
\pgfsetstrokecolor{textcolor}%
\pgfsetfillcolor{textcolor}%
\pgftext[x=0.566440in,y=0.319444in,,top]{\color{textcolor}\sffamily\fontsize{9.000000}{10.800000}\selectfont 0}%
\end{pgfscope}%
\begin{pgfscope}%
\pgfpathrectangle{\pgfqpoint{0.566440in}{0.451389in}}{\pgfqpoint{2.116084in}{1.319509in}}%
\pgfusepath{clip}%
\pgfsetroundcap%
\pgfsetroundjoin%
\pgfsetlinewidth{1.003750pt}%
\definecolor{currentstroke}{rgb}{0.800000,0.800000,0.800000}%
\pgfsetstrokecolor{currentstroke}%
\pgfsetdash{}{0pt}%
\pgfpathmoveto{\pgfqpoint{1.025441in}{0.451389in}}%
\pgfpathlineto{\pgfqpoint{1.025441in}{1.770898in}}%
\pgfusepath{stroke}%
\end{pgfscope}%
\begin{pgfscope}%
\definecolor{textcolor}{rgb}{0.150000,0.150000,0.150000}%
\pgfsetstrokecolor{textcolor}%
\pgfsetfillcolor{textcolor}%
\pgftext[x=1.025441in,y=0.319444in,,top]{\color{textcolor}\sffamily\fontsize{9.000000}{10.800000}\selectfont 20000}%
\end{pgfscope}%
\begin{pgfscope}%
\pgfpathrectangle{\pgfqpoint{0.566440in}{0.451389in}}{\pgfqpoint{2.116084in}{1.319509in}}%
\pgfusepath{clip}%
\pgfsetroundcap%
\pgfsetroundjoin%
\pgfsetlinewidth{1.003750pt}%
\definecolor{currentstroke}{rgb}{0.800000,0.800000,0.800000}%
\pgfsetstrokecolor{currentstroke}%
\pgfsetdash{}{0pt}%
\pgfpathmoveto{\pgfqpoint{1.484442in}{0.451389in}}%
\pgfpathlineto{\pgfqpoint{1.484442in}{1.770898in}}%
\pgfusepath{stroke}%
\end{pgfscope}%
\begin{pgfscope}%
\definecolor{textcolor}{rgb}{0.150000,0.150000,0.150000}%
\pgfsetstrokecolor{textcolor}%
\pgfsetfillcolor{textcolor}%
\pgftext[x=1.484442in,y=0.319444in,,top]{\color{textcolor}\sffamily\fontsize{9.000000}{10.800000}\selectfont 40000}%
\end{pgfscope}%
\begin{pgfscope}%
\pgfpathrectangle{\pgfqpoint{0.566440in}{0.451389in}}{\pgfqpoint{2.116084in}{1.319509in}}%
\pgfusepath{clip}%
\pgfsetroundcap%
\pgfsetroundjoin%
\pgfsetlinewidth{1.003750pt}%
\definecolor{currentstroke}{rgb}{0.800000,0.800000,0.800000}%
\pgfsetstrokecolor{currentstroke}%
\pgfsetdash{}{0pt}%
\pgfpathmoveto{\pgfqpoint{1.943442in}{0.451389in}}%
\pgfpathlineto{\pgfqpoint{1.943442in}{1.770898in}}%
\pgfusepath{stroke}%
\end{pgfscope}%
\begin{pgfscope}%
\definecolor{textcolor}{rgb}{0.150000,0.150000,0.150000}%
\pgfsetstrokecolor{textcolor}%
\pgfsetfillcolor{textcolor}%
\pgftext[x=1.943442in,y=0.319444in,,top]{\color{textcolor}\sffamily\fontsize{9.000000}{10.800000}\selectfont 60000}%
\end{pgfscope}%
\begin{pgfscope}%
\pgfpathrectangle{\pgfqpoint{0.566440in}{0.451389in}}{\pgfqpoint{2.116084in}{1.319509in}}%
\pgfusepath{clip}%
\pgfsetroundcap%
\pgfsetroundjoin%
\pgfsetlinewidth{1.003750pt}%
\definecolor{currentstroke}{rgb}{0.800000,0.800000,0.800000}%
\pgfsetstrokecolor{currentstroke}%
\pgfsetdash{}{0pt}%
\pgfpathmoveto{\pgfqpoint{2.402443in}{0.451389in}}%
\pgfpathlineto{\pgfqpoint{2.402443in}{1.770898in}}%
\pgfusepath{stroke}%
\end{pgfscope}%
\begin{pgfscope}%
\definecolor{textcolor}{rgb}{0.150000,0.150000,0.150000}%
\pgfsetstrokecolor{textcolor}%
\pgfsetfillcolor{textcolor}%
\pgftext[x=2.402443in,y=0.319444in,,top]{\color{textcolor}\sffamily\fontsize{9.000000}{10.800000}\selectfont 80000}%
\end{pgfscope}%
\begin{pgfscope}%
\definecolor{textcolor}{rgb}{0.150000,0.150000,0.150000}%
\pgfsetstrokecolor{textcolor}%
\pgfsetfillcolor{textcolor}%
\pgftext[x=1.624482in,y=0.125000in,,top]{\color{textcolor}\sffamily\fontsize{9.000000}{10.800000}\selectfont Input obstacle vertices}%
\end{pgfscope}%
\begin{pgfscope}%
\pgfpathrectangle{\pgfqpoint{0.566440in}{0.451389in}}{\pgfqpoint{2.116084in}{1.319509in}}%
\pgfusepath{clip}%
\pgfsetroundcap%
\pgfsetroundjoin%
\pgfsetlinewidth{1.003750pt}%
\definecolor{currentstroke}{rgb}{0.800000,0.800000,0.800000}%
\pgfsetstrokecolor{currentstroke}%
\pgfsetdash{}{0pt}%
\pgfpathmoveto{\pgfqpoint{0.566440in}{0.451389in}}%
\pgfpathlineto{\pgfqpoint{2.682524in}{0.451389in}}%
\pgfusepath{stroke}%
\end{pgfscope}%
\begin{pgfscope}%
\definecolor{textcolor}{rgb}{0.150000,0.150000,0.150000}%
\pgfsetstrokecolor{textcolor}%
\pgfsetfillcolor{textcolor}%
\pgftext[x=0.263292in, y=0.403903in, left, base]{\color{textcolor}\sffamily\fontsize{9.000000}{10.800000}\selectfont 0.0}%
\end{pgfscope}%
\begin{pgfscope}%
\pgfpathrectangle{\pgfqpoint{0.566440in}{0.451389in}}{\pgfqpoint{2.116084in}{1.319509in}}%
\pgfusepath{clip}%
\pgfsetroundcap%
\pgfsetroundjoin%
\pgfsetlinewidth{1.003750pt}%
\definecolor{currentstroke}{rgb}{0.800000,0.800000,0.800000}%
\pgfsetstrokecolor{currentstroke}%
\pgfsetdash{}{0pt}%
\pgfpathmoveto{\pgfqpoint{0.566440in}{0.744703in}}%
\pgfpathlineto{\pgfqpoint{2.682524in}{0.744703in}}%
\pgfusepath{stroke}%
\end{pgfscope}%
\begin{pgfscope}%
\definecolor{textcolor}{rgb}{0.150000,0.150000,0.150000}%
\pgfsetstrokecolor{textcolor}%
\pgfsetfillcolor{textcolor}%
\pgftext[x=0.263292in, y=0.697217in, left, base]{\color{textcolor}\sffamily\fontsize{9.000000}{10.800000}\selectfont 2.5}%
\end{pgfscope}%
\begin{pgfscope}%
\pgfpathrectangle{\pgfqpoint{0.566440in}{0.451389in}}{\pgfqpoint{2.116084in}{1.319509in}}%
\pgfusepath{clip}%
\pgfsetroundcap%
\pgfsetroundjoin%
\pgfsetlinewidth{1.003750pt}%
\definecolor{currentstroke}{rgb}{0.800000,0.800000,0.800000}%
\pgfsetstrokecolor{currentstroke}%
\pgfsetdash{}{0pt}%
\pgfpathmoveto{\pgfqpoint{0.566440in}{1.038017in}}%
\pgfpathlineto{\pgfqpoint{2.682524in}{1.038017in}}%
\pgfusepath{stroke}%
\end{pgfscope}%
\begin{pgfscope}%
\definecolor{textcolor}{rgb}{0.150000,0.150000,0.150000}%
\pgfsetstrokecolor{textcolor}%
\pgfsetfillcolor{textcolor}%
\pgftext[x=0.263292in, y=0.990532in, left, base]{\color{textcolor}\sffamily\fontsize{9.000000}{10.800000}\selectfont 5.0}%
\end{pgfscope}%
\begin{pgfscope}%
\pgfpathrectangle{\pgfqpoint{0.566440in}{0.451389in}}{\pgfqpoint{2.116084in}{1.319509in}}%
\pgfusepath{clip}%
\pgfsetroundcap%
\pgfsetroundjoin%
\pgfsetlinewidth{1.003750pt}%
\definecolor{currentstroke}{rgb}{0.800000,0.800000,0.800000}%
\pgfsetstrokecolor{currentstroke}%
\pgfsetdash{}{0pt}%
\pgfpathmoveto{\pgfqpoint{0.566440in}{1.331331in}}%
\pgfpathlineto{\pgfqpoint{2.682524in}{1.331331in}}%
\pgfusepath{stroke}%
\end{pgfscope}%
\begin{pgfscope}%
\definecolor{textcolor}{rgb}{0.150000,0.150000,0.150000}%
\pgfsetstrokecolor{textcolor}%
\pgfsetfillcolor{textcolor}%
\pgftext[x=0.263292in, y=1.283846in, left, base]{\color{textcolor}\sffamily\fontsize{9.000000}{10.800000}\selectfont 7.5}%
\end{pgfscope}%
\begin{pgfscope}%
\pgfpathrectangle{\pgfqpoint{0.566440in}{0.451389in}}{\pgfqpoint{2.116084in}{1.319509in}}%
\pgfusepath{clip}%
\pgfsetroundcap%
\pgfsetroundjoin%
\pgfsetlinewidth{1.003750pt}%
\definecolor{currentstroke}{rgb}{0.800000,0.800000,0.800000}%
\pgfsetstrokecolor{currentstroke}%
\pgfsetdash{}{0pt}%
\pgfpathmoveto{\pgfqpoint{0.566440in}{1.624645in}}%
\pgfpathlineto{\pgfqpoint{2.682524in}{1.624645in}}%
\pgfusepath{stroke}%
\end{pgfscope}%
\begin{pgfscope}%
\definecolor{textcolor}{rgb}{0.150000,0.150000,0.150000}%
\pgfsetstrokecolor{textcolor}%
\pgfsetfillcolor{textcolor}%
\pgftext[x=0.194444in, y=1.577160in, left, base]{\color{textcolor}\sffamily\fontsize{9.000000}{10.800000}\selectfont 10.0}%
\end{pgfscope}%
\begin{pgfscope}%
\definecolor{textcolor}{rgb}{0.150000,0.150000,0.150000}%
\pgfsetstrokecolor{textcolor}%
\pgfsetfillcolor{textcolor}%
\pgftext[x=0.125000in,y=1.111143in,,bottom,rotate=90.000000]{\color{textcolor}\sffamily\fontsize{9.000000}{10.800000}\selectfont Time in ms}%
\end{pgfscope}%
\begin{pgfscope}%
\pgfpathrectangle{\pgfqpoint{0.566440in}{0.451389in}}{\pgfqpoint{2.116084in}{1.319509in}}%
\pgfusepath{clip}%
\pgfsetbuttcap%
\pgfsetroundjoin%
\definecolor{currentfill}{rgb}{0.003922,0.450980,0.698039}%
\pgfsetfillcolor{currentfill}%
\pgfsetfillopacity{0.200000}%
\pgfsetlinewidth{1.003750pt}%
\definecolor{currentstroke}{rgb}{0.003922,0.450980,0.698039}%
\pgfsetstrokecolor{currentstroke}%
\pgfsetstrokeopacity{0.200000}%
\pgfsetdash{}{0pt}%
\pgfsys@defobject{currentmarker}{\pgfqpoint{0.887764in}{0.609974in}}{\pgfqpoint{2.597060in}{1.715616in}}{%
\pgfpathmoveto{\pgfqpoint{0.887764in}{0.613973in}}%
\pgfpathlineto{\pgfqpoint{0.887764in}{0.609974in}}%
\pgfpathlineto{\pgfqpoint{1.147283in}{0.767000in}}%
\pgfpathlineto{\pgfqpoint{1.373249in}{0.935656in}}%
\pgfpathlineto{\pgfqpoint{1.593431in}{1.152457in}}%
\pgfpathlineto{\pgfqpoint{2.110175in}{1.426811in}}%
\pgfpathlineto{\pgfqpoint{2.597060in}{1.680963in}}%
\pgfpathlineto{\pgfqpoint{2.597060in}{1.715616in}}%
\pgfpathlineto{\pgfqpoint{2.597060in}{1.715616in}}%
\pgfpathlineto{\pgfqpoint{2.110175in}{1.439807in}}%
\pgfpathlineto{\pgfqpoint{1.593431in}{1.157148in}}%
\pgfpathlineto{\pgfqpoint{1.373249in}{0.947175in}}%
\pgfpathlineto{\pgfqpoint{1.147283in}{0.772201in}}%
\pgfpathlineto{\pgfqpoint{0.887764in}{0.613973in}}%
\pgfpathlineto{\pgfqpoint{0.887764in}{0.613973in}}%
\pgfpathclose%
\pgfusepath{stroke,fill}%
}%
\begin{pgfscope}%
\pgfsys@transformshift{0.000000in}{0.000000in}%
\pgfsys@useobject{currentmarker}{}%
\end{pgfscope}%
\end{pgfscope}%
\begin{pgfscope}%
\pgfsetrectcap%
\pgfsetmiterjoin%
\pgfsetlinewidth{1.254687pt}%
\definecolor{currentstroke}{rgb}{0.800000,0.800000,0.800000}%
\pgfsetstrokecolor{currentstroke}%
\pgfsetdash{}{0pt}%
\pgfpathmoveto{\pgfqpoint{0.566440in}{0.451389in}}%
\pgfpathlineto{\pgfqpoint{0.566440in}{1.770898in}}%
\pgfusepath{stroke}%
\end{pgfscope}%
\begin{pgfscope}%
\pgfsetrectcap%
\pgfsetmiterjoin%
\pgfsetlinewidth{1.254687pt}%
\definecolor{currentstroke}{rgb}{0.800000,0.800000,0.800000}%
\pgfsetstrokecolor{currentstroke}%
\pgfsetdash{}{0pt}%
\pgfpathmoveto{\pgfqpoint{2.682524in}{0.451389in}}%
\pgfpathlineto{\pgfqpoint{2.682524in}{1.770898in}}%
\pgfusepath{stroke}%
\end{pgfscope}%
\begin{pgfscope}%
\pgfsetrectcap%
\pgfsetmiterjoin%
\pgfsetlinewidth{1.254687pt}%
\definecolor{currentstroke}{rgb}{0.800000,0.800000,0.800000}%
\pgfsetstrokecolor{currentstroke}%
\pgfsetdash{}{0pt}%
\pgfpathmoveto{\pgfqpoint{0.566440in}{0.451389in}}%
\pgfpathlineto{\pgfqpoint{2.682524in}{0.451389in}}%
\pgfusepath{stroke}%
\end{pgfscope}%
\begin{pgfscope}%
\pgfsetrectcap%
\pgfsetmiterjoin%
\pgfsetlinewidth{1.254687pt}%
\definecolor{currentstroke}{rgb}{0.800000,0.800000,0.800000}%
\pgfsetstrokecolor{currentstroke}%
\pgfsetdash{}{0pt}%
\pgfpathmoveto{\pgfqpoint{0.566440in}{1.770898in}}%
\pgfpathlineto{\pgfqpoint{2.682524in}{1.770898in}}%
\pgfusepath{stroke}%
\end{pgfscope}%
\begin{pgfscope}%
\pgfsetroundcap%
\pgfsetroundjoin%
\pgfsetlinewidth{1.003750pt}%
\definecolor{currentstroke}{rgb}{0.003922,0.450980,0.698039}%
\pgfsetstrokecolor{currentstroke}%
\pgfsetdash{}{0pt}%
\pgfpathmoveto{\pgfqpoint{0.887764in}{0.611728in}}%
\pgfpathlineto{\pgfqpoint{1.147283in}{0.769536in}}%
\pgfpathlineto{\pgfqpoint{1.373249in}{0.942104in}}%
\pgfpathlineto{\pgfqpoint{1.593431in}{1.155204in}}%
\pgfpathlineto{\pgfqpoint{2.110175in}{1.434253in}}%
\pgfpathlineto{\pgfqpoint{2.597060in}{1.698659in}}%
\pgfusepath{stroke}%
\end{pgfscope}%
\begin{pgfscope}%
\pgfsetbuttcap%
\pgfsetroundjoin%
\definecolor{currentfill}{rgb}{0.003922,0.450980,0.698039}%
\pgfsetfillcolor{currentfill}%
\pgfsetlinewidth{0.752812pt}%
\definecolor{currentstroke}{rgb}{1.000000,1.000000,1.000000}%
\pgfsetstrokecolor{currentstroke}%
\pgfsetdash{}{0pt}%
\pgfsys@defobject{currentmarker}{\pgfqpoint{-0.034722in}{-0.034722in}}{\pgfqpoint{0.034722in}{0.034722in}}{%
\pgfpathmoveto{\pgfqpoint{0.000000in}{-0.034722in}}%
\pgfpathcurveto{\pgfqpoint{0.009208in}{-0.034722in}}{\pgfqpoint{0.018041in}{-0.031064in}}{\pgfqpoint{0.024552in}{-0.024552in}}%
\pgfpathcurveto{\pgfqpoint{0.031064in}{-0.018041in}}{\pgfqpoint{0.034722in}{-0.009208in}}{\pgfqpoint{0.034722in}{0.000000in}}%
\pgfpathcurveto{\pgfqpoint{0.034722in}{0.009208in}}{\pgfqpoint{0.031064in}{0.018041in}}{\pgfqpoint{0.024552in}{0.024552in}}%
\pgfpathcurveto{\pgfqpoint{0.018041in}{0.031064in}}{\pgfqpoint{0.009208in}{0.034722in}}{\pgfqpoint{0.000000in}{0.034722in}}%
\pgfpathcurveto{\pgfqpoint{-0.009208in}{0.034722in}}{\pgfqpoint{-0.018041in}{0.031064in}}{\pgfqpoint{-0.024552in}{0.024552in}}%
\pgfpathcurveto{\pgfqpoint{-0.031064in}{0.018041in}}{\pgfqpoint{-0.034722in}{0.009208in}}{\pgfqpoint{-0.034722in}{0.000000in}}%
\pgfpathcurveto{\pgfqpoint{-0.034722in}{-0.009208in}}{\pgfqpoint{-0.031064in}{-0.018041in}}{\pgfqpoint{-0.024552in}{-0.024552in}}%
\pgfpathcurveto{\pgfqpoint{-0.018041in}{-0.031064in}}{\pgfqpoint{-0.009208in}{-0.034722in}}{\pgfqpoint{0.000000in}{-0.034722in}}%
\pgfpathlineto{\pgfqpoint{0.000000in}{-0.034722in}}%
\pgfpathclose%
\pgfusepath{stroke,fill}%
}%
\begin{pgfscope}%
\pgfsys@transformshift{0.887764in}{0.611728in}%
\pgfsys@useobject{currentmarker}{}%
\end{pgfscope}%
\begin{pgfscope}%
\pgfsys@transformshift{1.147283in}{0.769536in}%
\pgfsys@useobject{currentmarker}{}%
\end{pgfscope}%
\begin{pgfscope}%
\pgfsys@transformshift{1.373249in}{0.942104in}%
\pgfsys@useobject{currentmarker}{}%
\end{pgfscope}%
\begin{pgfscope}%
\pgfsys@transformshift{1.593431in}{1.155204in}%
\pgfsys@useobject{currentmarker}{}%
\end{pgfscope}%
\begin{pgfscope}%
\pgfsys@transformshift{2.110175in}{1.434253in}%
\pgfsys@useobject{currentmarker}{}%
\end{pgfscope}%
\begin{pgfscope}%
\pgfsys@transformshift{2.597060in}{1.698659in}%
\pgfsys@useobject{currentmarker}{}%
\end{pgfscope}%
\end{pgfscope}%
\end{pgfpicture}%
\makeatother%
\endgroup%

					\end{figcenter}
					\caption{Time per input vertex for the \enquote{OSM rural} dataset.}
				\end{subfigure}
				\\[3ex]
				\begin{subfigure}[t]{\linewidth}
					\begin{figcenter}
						\input{images/evaluation/osm-based-rural/plot-iteration-time-per-vertices.py_per-vertex-added.pgf}
						% TODO same y-scale in both figures
					\end{figcenter}
					\caption{Increase in processing time per vertex when an additional vertex is added.}
					\label{fig:eval-import-rural-rel-increase}
				\end{subfigure}
				\caption{Graph generation times using the \enquote{OSM rural} dataset.}
				\label{fig:eval-import-rural}
			\end{minipage}
		\end{figure}
		
		A few aspects regarding the runtime behavior can already be inferred from \cref{fig:eval-import-city} and \cref{fig:eval-import-rural}.
		
		First, as mentioned above, the inherent quadratic runtime of the visibility graph creation, which is shown in detail in the following \cref{fig:eval-import-details}.
		
		Second, the linear increase in the processing time per vertex, which shows, that the total runtime is in fact quadratic and not, for example, exponential.
		
		And third, the additional effort added to each vertex when a new additional vertex is added to the dataset is very small (below one microsecond) and decreases with the size of the dataset.
		Taking the largest dataset of about 88.000 vertices in \cref{fig:eval-import-city-rel-increase} as an example, adding one new vertex increases the processing time of every vertex by around 0.12 µs.
		At least for smaller datasets, as \cref{fig:eval-import-rural-rel-increase} shows, this extra time for each additional vertex decreases with the size of the dataset.
		This indicates a slower growing processing time for larger datasets, however, this effect is negligible as it is not directly visible by the measured data in \cref{fig:eval-import-city-abs} and \cref{fig:eval-import-rural-abs}.
		
		\begin{figure}[h]
			\begin{figcenter}
				\begin{subfigure}[t]{\textwidth}
					\begin{figcenter}
						%% Creator: Matplotlib, PGF backend
%%
%% To include the figure in your LaTeX document, write
%%   \input{<filename>.pgf}
%%
%% Make sure the required packages are loaded in your preamble
%%   \usepackage{pgf}
%%
%% Also ensure that all the required font packages are loaded; for instance,
%% the lmodern package is sometimes necessary when using math font.
%%   \usepackage{lmodern}
%%
%% Figures using additional raster images can only be included by \input if
%% they are in the same directory as the main LaTeX file. For loading figures
%% from other directories you can use the `import` package
%%   \usepackage{import}
%%
%% and then include the figures with
%%   \import{<path to file>}{<filename>.pgf}
%%
%% Matplotlib used the following preamble
%%   
%%   \usepackage{fontspec}
%%   \setmainfont{DejaVuSerif.ttf}[Path=\detokenize{/home/hauke/.local/lib/python3.11/site-packages/matplotlib/mpl-data/fonts/ttf/}]
%%   \setsansfont{DroidSans.ttf}[Path=\detokenize{/usr/share/fonts/droid/}]
%%   \setmonofont{DejaVuSansMono.ttf}[Path=\detokenize{/home/hauke/.local/lib/python3.11/site-packages/matplotlib/mpl-data/fonts/ttf/}]
%%   \makeatletter\@ifpackageloaded{underscore}{}{\usepackage[strings]{underscore}}\makeatother
%%
\begingroup%
\makeatletter%
\begin{pgfpicture}%
\pgfpathrectangle{\pgfpointorigin}{\pgfqpoint{5.745359in}{3.237620in}}%
\pgfusepath{use as bounding box, clip}%
\begin{pgfscope}%
\pgfsetbuttcap%
\pgfsetmiterjoin%
\definecolor{currentfill}{rgb}{1.000000,1.000000,1.000000}%
\pgfsetfillcolor{currentfill}%
\pgfsetlinewidth{0.000000pt}%
\definecolor{currentstroke}{rgb}{1.000000,1.000000,1.000000}%
\pgfsetstrokecolor{currentstroke}%
\pgfsetdash{}{0pt}%
\pgfpathmoveto{\pgfqpoint{0.000000in}{0.000000in}}%
\pgfpathlineto{\pgfqpoint{5.745359in}{0.000000in}}%
\pgfpathlineto{\pgfqpoint{5.745359in}{3.237620in}}%
\pgfpathlineto{\pgfqpoint{0.000000in}{3.237620in}}%
\pgfpathlineto{\pgfqpoint{0.000000in}{0.000000in}}%
\pgfpathclose%
\pgfusepath{fill}%
\end{pgfscope}%
\begin{pgfscope}%
\pgfsetbuttcap%
\pgfsetmiterjoin%
\definecolor{currentfill}{rgb}{1.000000,1.000000,1.000000}%
\pgfsetfillcolor{currentfill}%
\pgfsetlinewidth{0.000000pt}%
\definecolor{currentstroke}{rgb}{0.000000,0.000000,0.000000}%
\pgfsetstrokecolor{currentstroke}%
\pgfsetstrokeopacity{0.000000}%
\pgfsetdash{}{0pt}%
\pgfpathmoveto{\pgfqpoint{0.592976in}{0.451389in}}%
\pgfpathlineto{\pgfqpoint{4.095253in}{0.451389in}}%
\pgfpathlineto{\pgfqpoint{4.095253in}{3.237620in}}%
\pgfpathlineto{\pgfqpoint{0.592976in}{3.237620in}}%
\pgfpathlineto{\pgfqpoint{0.592976in}{0.451389in}}%
\pgfpathclose%
\pgfusepath{fill}%
\end{pgfscope}%
\begin{pgfscope}%
\pgfpathrectangle{\pgfqpoint{0.592976in}{0.451389in}}{\pgfqpoint{3.502277in}{2.786231in}}%
\pgfusepath{clip}%
\pgfsetroundcap%
\pgfsetroundjoin%
\pgfsetlinewidth{1.003750pt}%
\definecolor{currentstroke}{rgb}{0.800000,0.800000,0.800000}%
\pgfsetstrokecolor{currentstroke}%
\pgfsetdash{}{0pt}%
\pgfpathmoveto{\pgfqpoint{0.744191in}{0.451389in}}%
\pgfpathlineto{\pgfqpoint{0.744191in}{3.237620in}}%
\pgfusepath{stroke}%
\end{pgfscope}%
\begin{pgfscope}%
\definecolor{textcolor}{rgb}{0.150000,0.150000,0.150000}%
\pgfsetstrokecolor{textcolor}%
\pgfsetfillcolor{textcolor}%
\pgftext[x=0.744191in,y=0.319444in,,top]{\color{textcolor}\sffamily\fontsize{9.000000}{10.800000}\selectfont 0}%
\end{pgfscope}%
\begin{pgfscope}%
\pgfpathrectangle{\pgfqpoint{0.592976in}{0.451389in}}{\pgfqpoint{3.502277in}{2.786231in}}%
\pgfusepath{clip}%
\pgfsetroundcap%
\pgfsetroundjoin%
\pgfsetlinewidth{1.003750pt}%
\definecolor{currentstroke}{rgb}{0.800000,0.800000,0.800000}%
\pgfsetstrokecolor{currentstroke}%
\pgfsetdash{}{0pt}%
\pgfpathmoveto{\pgfqpoint{1.276169in}{0.451389in}}%
\pgfpathlineto{\pgfqpoint{1.276169in}{3.237620in}}%
\pgfusepath{stroke}%
\end{pgfscope}%
\begin{pgfscope}%
\definecolor{textcolor}{rgb}{0.150000,0.150000,0.150000}%
\pgfsetstrokecolor{textcolor}%
\pgfsetfillcolor{textcolor}%
\pgftext[x=1.276169in,y=0.319444in,,top]{\color{textcolor}\sffamily\fontsize{9.000000}{10.800000}\selectfont 5000}%
\end{pgfscope}%
\begin{pgfscope}%
\pgfpathrectangle{\pgfqpoint{0.592976in}{0.451389in}}{\pgfqpoint{3.502277in}{2.786231in}}%
\pgfusepath{clip}%
\pgfsetroundcap%
\pgfsetroundjoin%
\pgfsetlinewidth{1.003750pt}%
\definecolor{currentstroke}{rgb}{0.800000,0.800000,0.800000}%
\pgfsetstrokecolor{currentstroke}%
\pgfsetdash{}{0pt}%
\pgfpathmoveto{\pgfqpoint{1.808147in}{0.451389in}}%
\pgfpathlineto{\pgfqpoint{1.808147in}{3.237620in}}%
\pgfusepath{stroke}%
\end{pgfscope}%
\begin{pgfscope}%
\definecolor{textcolor}{rgb}{0.150000,0.150000,0.150000}%
\pgfsetstrokecolor{textcolor}%
\pgfsetfillcolor{textcolor}%
\pgftext[x=1.808147in,y=0.319444in,,top]{\color{textcolor}\sffamily\fontsize{9.000000}{10.800000}\selectfont 10000}%
\end{pgfscope}%
\begin{pgfscope}%
\pgfpathrectangle{\pgfqpoint{0.592976in}{0.451389in}}{\pgfqpoint{3.502277in}{2.786231in}}%
\pgfusepath{clip}%
\pgfsetroundcap%
\pgfsetroundjoin%
\pgfsetlinewidth{1.003750pt}%
\definecolor{currentstroke}{rgb}{0.800000,0.800000,0.800000}%
\pgfsetstrokecolor{currentstroke}%
\pgfsetdash{}{0pt}%
\pgfpathmoveto{\pgfqpoint{2.340125in}{0.451389in}}%
\pgfpathlineto{\pgfqpoint{2.340125in}{3.237620in}}%
\pgfusepath{stroke}%
\end{pgfscope}%
\begin{pgfscope}%
\definecolor{textcolor}{rgb}{0.150000,0.150000,0.150000}%
\pgfsetstrokecolor{textcolor}%
\pgfsetfillcolor{textcolor}%
\pgftext[x=2.340125in,y=0.319444in,,top]{\color{textcolor}\sffamily\fontsize{9.000000}{10.800000}\selectfont 15000}%
\end{pgfscope}%
\begin{pgfscope}%
\pgfpathrectangle{\pgfqpoint{0.592976in}{0.451389in}}{\pgfqpoint{3.502277in}{2.786231in}}%
\pgfusepath{clip}%
\pgfsetroundcap%
\pgfsetroundjoin%
\pgfsetlinewidth{1.003750pt}%
\definecolor{currentstroke}{rgb}{0.800000,0.800000,0.800000}%
\pgfsetstrokecolor{currentstroke}%
\pgfsetdash{}{0pt}%
\pgfpathmoveto{\pgfqpoint{2.872103in}{0.451389in}}%
\pgfpathlineto{\pgfqpoint{2.872103in}{3.237620in}}%
\pgfusepath{stroke}%
\end{pgfscope}%
\begin{pgfscope}%
\definecolor{textcolor}{rgb}{0.150000,0.150000,0.150000}%
\pgfsetstrokecolor{textcolor}%
\pgfsetfillcolor{textcolor}%
\pgftext[x=2.872103in,y=0.319444in,,top]{\color{textcolor}\sffamily\fontsize{9.000000}{10.800000}\selectfont 20000}%
\end{pgfscope}%
\begin{pgfscope}%
\pgfpathrectangle{\pgfqpoint{0.592976in}{0.451389in}}{\pgfqpoint{3.502277in}{2.786231in}}%
\pgfusepath{clip}%
\pgfsetroundcap%
\pgfsetroundjoin%
\pgfsetlinewidth{1.003750pt}%
\definecolor{currentstroke}{rgb}{0.800000,0.800000,0.800000}%
\pgfsetstrokecolor{currentstroke}%
\pgfsetdash{}{0pt}%
\pgfpathmoveto{\pgfqpoint{3.404081in}{0.451389in}}%
\pgfpathlineto{\pgfqpoint{3.404081in}{3.237620in}}%
\pgfusepath{stroke}%
\end{pgfscope}%
\begin{pgfscope}%
\definecolor{textcolor}{rgb}{0.150000,0.150000,0.150000}%
\pgfsetstrokecolor{textcolor}%
\pgfsetfillcolor{textcolor}%
\pgftext[x=3.404081in,y=0.319444in,,top]{\color{textcolor}\sffamily\fontsize{9.000000}{10.800000}\selectfont 25000}%
\end{pgfscope}%
\begin{pgfscope}%
\pgfpathrectangle{\pgfqpoint{0.592976in}{0.451389in}}{\pgfqpoint{3.502277in}{2.786231in}}%
\pgfusepath{clip}%
\pgfsetroundcap%
\pgfsetroundjoin%
\pgfsetlinewidth{1.003750pt}%
\definecolor{currentstroke}{rgb}{0.800000,0.800000,0.800000}%
\pgfsetstrokecolor{currentstroke}%
\pgfsetdash{}{0pt}%
\pgfpathmoveto{\pgfqpoint{3.936059in}{0.451389in}}%
\pgfpathlineto{\pgfqpoint{3.936059in}{3.237620in}}%
\pgfusepath{stroke}%
\end{pgfscope}%
\begin{pgfscope}%
\definecolor{textcolor}{rgb}{0.150000,0.150000,0.150000}%
\pgfsetstrokecolor{textcolor}%
\pgfsetfillcolor{textcolor}%
\pgftext[x=3.936059in,y=0.319444in,,top]{\color{textcolor}\sffamily\fontsize{9.000000}{10.800000}\selectfont 30000}%
\end{pgfscope}%
\begin{pgfscope}%
\definecolor{textcolor}{rgb}{0.150000,0.150000,0.150000}%
\pgfsetstrokecolor{textcolor}%
\pgfsetfillcolor{textcolor}%
\pgftext[x=2.344114in,y=0.125000in,,top]{\color{textcolor}\sffamily\fontsize{9.000000}{10.800000}\selectfont Input obstacle vertices}%
\end{pgfscope}%
\begin{pgfscope}%
\pgfpathrectangle{\pgfqpoint{0.592976in}{0.451389in}}{\pgfqpoint{3.502277in}{2.786231in}}%
\pgfusepath{clip}%
\pgfsetroundcap%
\pgfsetroundjoin%
\pgfsetlinewidth{1.003750pt}%
\definecolor{currentstroke}{rgb}{0.800000,0.800000,0.800000}%
\pgfsetstrokecolor{currentstroke}%
\pgfsetdash{}{0pt}%
\pgfpathmoveto{\pgfqpoint{0.592976in}{0.489539in}}%
\pgfpathlineto{\pgfqpoint{4.095253in}{0.489539in}}%
\pgfusepath{stroke}%
\end{pgfscope}%
\begin{pgfscope}%
\definecolor{textcolor}{rgb}{0.150000,0.150000,0.150000}%
\pgfsetstrokecolor{textcolor}%
\pgfsetfillcolor{textcolor}%
\pgftext[x=0.194444in, y=0.442053in, left, base]{\color{textcolor}\sffamily\fontsize{9.000000}{10.800000}\selectfont \(\displaystyle {10^{-5}}\)}%
\end{pgfscope}%
\begin{pgfscope}%
\pgfpathrectangle{\pgfqpoint{0.592976in}{0.451389in}}{\pgfqpoint{3.502277in}{2.786231in}}%
\pgfusepath{clip}%
\pgfsetroundcap%
\pgfsetroundjoin%
\pgfsetlinewidth{1.003750pt}%
\definecolor{currentstroke}{rgb}{0.800000,0.800000,0.800000}%
\pgfsetstrokecolor{currentstroke}%
\pgfsetdash{}{0pt}%
\pgfpathmoveto{\pgfqpoint{0.592976in}{0.836215in}}%
\pgfpathlineto{\pgfqpoint{4.095253in}{0.836215in}}%
\pgfusepath{stroke}%
\end{pgfscope}%
\begin{pgfscope}%
\definecolor{textcolor}{rgb}{0.150000,0.150000,0.150000}%
\pgfsetstrokecolor{textcolor}%
\pgfsetfillcolor{textcolor}%
\pgftext[x=0.194444in, y=0.788729in, left, base]{\color{textcolor}\sffamily\fontsize{9.000000}{10.800000}\selectfont \(\displaystyle {10^{-4}}\)}%
\end{pgfscope}%
\begin{pgfscope}%
\pgfpathrectangle{\pgfqpoint{0.592976in}{0.451389in}}{\pgfqpoint{3.502277in}{2.786231in}}%
\pgfusepath{clip}%
\pgfsetroundcap%
\pgfsetroundjoin%
\pgfsetlinewidth{1.003750pt}%
\definecolor{currentstroke}{rgb}{0.800000,0.800000,0.800000}%
\pgfsetstrokecolor{currentstroke}%
\pgfsetdash{}{0pt}%
\pgfpathmoveto{\pgfqpoint{0.592976in}{1.182890in}}%
\pgfpathlineto{\pgfqpoint{4.095253in}{1.182890in}}%
\pgfusepath{stroke}%
\end{pgfscope}%
\begin{pgfscope}%
\definecolor{textcolor}{rgb}{0.150000,0.150000,0.150000}%
\pgfsetstrokecolor{textcolor}%
\pgfsetfillcolor{textcolor}%
\pgftext[x=0.194444in, y=1.135405in, left, base]{\color{textcolor}\sffamily\fontsize{9.000000}{10.800000}\selectfont \(\displaystyle {10^{-3}}\)}%
\end{pgfscope}%
\begin{pgfscope}%
\pgfpathrectangle{\pgfqpoint{0.592976in}{0.451389in}}{\pgfqpoint{3.502277in}{2.786231in}}%
\pgfusepath{clip}%
\pgfsetroundcap%
\pgfsetroundjoin%
\pgfsetlinewidth{1.003750pt}%
\definecolor{currentstroke}{rgb}{0.800000,0.800000,0.800000}%
\pgfsetstrokecolor{currentstroke}%
\pgfsetdash{}{0pt}%
\pgfpathmoveto{\pgfqpoint{0.592976in}{1.529566in}}%
\pgfpathlineto{\pgfqpoint{4.095253in}{1.529566in}}%
\pgfusepath{stroke}%
\end{pgfscope}%
\begin{pgfscope}%
\definecolor{textcolor}{rgb}{0.150000,0.150000,0.150000}%
\pgfsetstrokecolor{textcolor}%
\pgfsetfillcolor{textcolor}%
\pgftext[x=0.194444in, y=1.482081in, left, base]{\color{textcolor}\sffamily\fontsize{9.000000}{10.800000}\selectfont \(\displaystyle {10^{-2}}\)}%
\end{pgfscope}%
\begin{pgfscope}%
\pgfpathrectangle{\pgfqpoint{0.592976in}{0.451389in}}{\pgfqpoint{3.502277in}{2.786231in}}%
\pgfusepath{clip}%
\pgfsetroundcap%
\pgfsetroundjoin%
\pgfsetlinewidth{1.003750pt}%
\definecolor{currentstroke}{rgb}{0.800000,0.800000,0.800000}%
\pgfsetstrokecolor{currentstroke}%
\pgfsetdash{}{0pt}%
\pgfpathmoveto{\pgfqpoint{0.592976in}{1.876242in}}%
\pgfpathlineto{\pgfqpoint{4.095253in}{1.876242in}}%
\pgfusepath{stroke}%
\end{pgfscope}%
\begin{pgfscope}%
\definecolor{textcolor}{rgb}{0.150000,0.150000,0.150000}%
\pgfsetstrokecolor{textcolor}%
\pgfsetfillcolor{textcolor}%
\pgftext[x=0.194444in, y=1.828757in, left, base]{\color{textcolor}\sffamily\fontsize{9.000000}{10.800000}\selectfont \(\displaystyle {10^{-1}}\)}%
\end{pgfscope}%
\begin{pgfscope}%
\pgfpathrectangle{\pgfqpoint{0.592976in}{0.451389in}}{\pgfqpoint{3.502277in}{2.786231in}}%
\pgfusepath{clip}%
\pgfsetroundcap%
\pgfsetroundjoin%
\pgfsetlinewidth{1.003750pt}%
\definecolor{currentstroke}{rgb}{0.800000,0.800000,0.800000}%
\pgfsetstrokecolor{currentstroke}%
\pgfsetdash{}{0pt}%
\pgfpathmoveto{\pgfqpoint{0.592976in}{2.222918in}}%
\pgfpathlineto{\pgfqpoint{4.095253in}{2.222918in}}%
\pgfusepath{stroke}%
\end{pgfscope}%
\begin{pgfscope}%
\definecolor{textcolor}{rgb}{0.150000,0.150000,0.150000}%
\pgfsetstrokecolor{textcolor}%
\pgfsetfillcolor{textcolor}%
\pgftext[x=0.274690in, y=2.175433in, left, base]{\color{textcolor}\sffamily\fontsize{9.000000}{10.800000}\selectfont \(\displaystyle {10^{0}}\)}%
\end{pgfscope}%
\begin{pgfscope}%
\pgfpathrectangle{\pgfqpoint{0.592976in}{0.451389in}}{\pgfqpoint{3.502277in}{2.786231in}}%
\pgfusepath{clip}%
\pgfsetroundcap%
\pgfsetroundjoin%
\pgfsetlinewidth{1.003750pt}%
\definecolor{currentstroke}{rgb}{0.800000,0.800000,0.800000}%
\pgfsetstrokecolor{currentstroke}%
\pgfsetdash{}{0pt}%
\pgfpathmoveto{\pgfqpoint{0.592976in}{2.569594in}}%
\pgfpathlineto{\pgfqpoint{4.095253in}{2.569594in}}%
\pgfusepath{stroke}%
\end{pgfscope}%
\begin{pgfscope}%
\definecolor{textcolor}{rgb}{0.150000,0.150000,0.150000}%
\pgfsetstrokecolor{textcolor}%
\pgfsetfillcolor{textcolor}%
\pgftext[x=0.274690in, y=2.522109in, left, base]{\color{textcolor}\sffamily\fontsize{9.000000}{10.800000}\selectfont \(\displaystyle {10^{1}}\)}%
\end{pgfscope}%
\begin{pgfscope}%
\pgfpathrectangle{\pgfqpoint{0.592976in}{0.451389in}}{\pgfqpoint{3.502277in}{2.786231in}}%
\pgfusepath{clip}%
\pgfsetroundcap%
\pgfsetroundjoin%
\pgfsetlinewidth{1.003750pt}%
\definecolor{currentstroke}{rgb}{0.800000,0.800000,0.800000}%
\pgfsetstrokecolor{currentstroke}%
\pgfsetdash{}{0pt}%
\pgfpathmoveto{\pgfqpoint{0.592976in}{2.916270in}}%
\pgfpathlineto{\pgfqpoint{4.095253in}{2.916270in}}%
\pgfusepath{stroke}%
\end{pgfscope}%
\begin{pgfscope}%
\definecolor{textcolor}{rgb}{0.150000,0.150000,0.150000}%
\pgfsetstrokecolor{textcolor}%
\pgfsetfillcolor{textcolor}%
\pgftext[x=0.274690in, y=2.868784in, left, base]{\color{textcolor}\sffamily\fontsize{9.000000}{10.800000}\selectfont \(\displaystyle {10^{2}}\)}%
\end{pgfscope}%
\begin{pgfscope}%
\definecolor{textcolor}{rgb}{0.150000,0.150000,0.150000}%
\pgfsetstrokecolor{textcolor}%
\pgfsetfillcolor{textcolor}%
\pgftext[x=0.125000in,y=1.844504in,,bottom,rotate=90.000000]{\color{textcolor}\sffamily\fontsize{9.000000}{10.800000}\selectfont Time in s}%
\end{pgfscope}%
\begin{pgfscope}%
\pgfpathrectangle{\pgfqpoint{0.592976in}{0.451389in}}{\pgfqpoint{3.502277in}{2.786231in}}%
\pgfusepath{clip}%
\pgfsetbuttcap%
\pgfsetroundjoin%
\definecolor{currentfill}{rgb}{0.003922,0.450980,0.698039}%
\pgfsetfillcolor{currentfill}%
\pgfsetfillopacity{0.200000}%
\pgfsetlinewidth{1.003750pt}%
\definecolor{currentstroke}{rgb}{0.003922,0.450980,0.698039}%
\pgfsetstrokecolor{currentstroke}%
\pgfsetstrokeopacity{0.200000}%
\pgfsetdash{}{0pt}%
\pgfsys@defobject{currentmarker}{\pgfqpoint{0.752170in}{1.539725in}}{\pgfqpoint{3.936059in}{3.110973in}}{%
\pgfpathmoveto{\pgfqpoint{0.752170in}{1.574131in}}%
\pgfpathlineto{\pgfqpoint{0.752170in}{1.539725in}}%
\pgfpathlineto{\pgfqpoint{0.776109in}{1.748589in}}%
\pgfpathlineto{\pgfqpoint{0.816008in}{1.935627in}}%
\pgfpathlineto{\pgfqpoint{0.871865in}{2.107138in}}%
\pgfpathlineto{\pgfqpoint{0.943682in}{2.230496in}}%
\pgfpathlineto{\pgfqpoint{1.031459in}{2.341192in}}%
\pgfpathlineto{\pgfqpoint{1.135194in}{2.417909in}}%
\pgfpathlineto{\pgfqpoint{1.254889in}{2.521621in}}%
\pgfpathlineto{\pgfqpoint{1.390544in}{2.560815in}}%
\pgfpathlineto{\pgfqpoint{1.542158in}{2.634818in}}%
\pgfpathlineto{\pgfqpoint{1.893263in}{2.765225in}}%
\pgfpathlineto{\pgfqpoint{2.308206in}{2.842439in}}%
\pgfpathlineto{\pgfqpoint{2.786986in}{2.980611in}}%
\pgfpathlineto{\pgfqpoint{3.329604in}{3.009306in}}%
\pgfpathlineto{\pgfqpoint{3.936059in}{3.110007in}}%
\pgfpathlineto{\pgfqpoint{3.936059in}{3.110973in}}%
\pgfpathlineto{\pgfqpoint{3.936059in}{3.110973in}}%
\pgfpathlineto{\pgfqpoint{3.329604in}{3.013752in}}%
\pgfpathlineto{\pgfqpoint{2.786986in}{2.990069in}}%
\pgfpathlineto{\pgfqpoint{2.308206in}{2.848349in}}%
\pgfpathlineto{\pgfqpoint{1.893263in}{2.767325in}}%
\pgfpathlineto{\pgfqpoint{1.542158in}{2.637988in}}%
\pgfpathlineto{\pgfqpoint{1.390544in}{2.564571in}}%
\pgfpathlineto{\pgfqpoint{1.254889in}{2.525128in}}%
\pgfpathlineto{\pgfqpoint{1.135194in}{2.419417in}}%
\pgfpathlineto{\pgfqpoint{1.031459in}{2.342323in}}%
\pgfpathlineto{\pgfqpoint{0.943682in}{2.231636in}}%
\pgfpathlineto{\pgfqpoint{0.871865in}{2.110017in}}%
\pgfpathlineto{\pgfqpoint{0.816008in}{1.939399in}}%
\pgfpathlineto{\pgfqpoint{0.776109in}{1.780735in}}%
\pgfpathlineto{\pgfqpoint{0.752170in}{1.574131in}}%
\pgfpathlineto{\pgfqpoint{0.752170in}{1.574131in}}%
\pgfpathclose%
\pgfusepath{stroke,fill}%
}%
\begin{pgfscope}%
\pgfsys@transformshift{0.000000in}{0.000000in}%
\pgfsys@useobject{currentmarker}{}%
\end{pgfscope}%
\end{pgfscope}%
\begin{pgfscope}%
\pgfpathrectangle{\pgfqpoint{0.592976in}{0.451389in}}{\pgfqpoint{3.502277in}{2.786231in}}%
\pgfusepath{clip}%
\pgfsetbuttcap%
\pgfsetroundjoin%
\definecolor{currentfill}{rgb}{0.870588,0.560784,0.019608}%
\pgfsetfillcolor{currentfill}%
\pgfsetfillopacity{0.200000}%
\pgfsetlinewidth{1.003750pt}%
\definecolor{currentstroke}{rgb}{0.870588,0.560784,0.019608}%
\pgfsetstrokecolor{currentstroke}%
\pgfsetstrokeopacity{0.200000}%
\pgfsetdash{}{0pt}%
\pgfsys@defobject{currentmarker}{\pgfqpoint{0.752170in}{1.501955in}}{\pgfqpoint{3.936059in}{3.110389in}}{%
\pgfpathmoveto{\pgfqpoint{0.752170in}{1.504771in}}%
\pgfpathlineto{\pgfqpoint{0.752170in}{1.501955in}}%
\pgfpathlineto{\pgfqpoint{0.776109in}{1.720774in}}%
\pgfpathlineto{\pgfqpoint{0.816008in}{1.923716in}}%
\pgfpathlineto{\pgfqpoint{0.871865in}{2.099865in}}%
\pgfpathlineto{\pgfqpoint{0.943682in}{2.222718in}}%
\pgfpathlineto{\pgfqpoint{1.031459in}{2.335152in}}%
\pgfpathlineto{\pgfqpoint{1.135194in}{2.412068in}}%
\pgfpathlineto{\pgfqpoint{1.254889in}{2.517935in}}%
\pgfpathlineto{\pgfqpoint{1.390544in}{2.557177in}}%
\pgfpathlineto{\pgfqpoint{1.542158in}{2.631969in}}%
\pgfpathlineto{\pgfqpoint{1.893263in}{2.763416in}}%
\pgfpathlineto{\pgfqpoint{2.308206in}{2.840830in}}%
\pgfpathlineto{\pgfqpoint{2.786986in}{2.979769in}}%
\pgfpathlineto{\pgfqpoint{3.329604in}{3.008517in}}%
\pgfpathlineto{\pgfqpoint{3.936059in}{3.109419in}}%
\pgfpathlineto{\pgfqpoint{3.936059in}{3.110389in}}%
\pgfpathlineto{\pgfqpoint{3.936059in}{3.110389in}}%
\pgfpathlineto{\pgfqpoint{3.329604in}{3.012967in}}%
\pgfpathlineto{\pgfqpoint{2.786986in}{2.989246in}}%
\pgfpathlineto{\pgfqpoint{2.308206in}{2.846717in}}%
\pgfpathlineto{\pgfqpoint{1.893263in}{2.765335in}}%
\pgfpathlineto{\pgfqpoint{1.542158in}{2.634490in}}%
\pgfpathlineto{\pgfqpoint{1.390544in}{2.560051in}}%
\pgfpathlineto{\pgfqpoint{1.254889in}{2.520634in}}%
\pgfpathlineto{\pgfqpoint{1.135194in}{2.412685in}}%
\pgfpathlineto{\pgfqpoint{1.031459in}{2.335843in}}%
\pgfpathlineto{\pgfqpoint{0.943682in}{2.223530in}}%
\pgfpathlineto{\pgfqpoint{0.871865in}{2.102489in}}%
\pgfpathlineto{\pgfqpoint{0.816008in}{1.924428in}}%
\pgfpathlineto{\pgfqpoint{0.776109in}{1.742032in}}%
\pgfpathlineto{\pgfqpoint{0.752170in}{1.504771in}}%
\pgfpathlineto{\pgfqpoint{0.752170in}{1.504771in}}%
\pgfpathclose%
\pgfusepath{stroke,fill}%
}%
\begin{pgfscope}%
\pgfsys@transformshift{0.000000in}{0.000000in}%
\pgfsys@useobject{currentmarker}{}%
\end{pgfscope}%
\end{pgfscope}%
\begin{pgfscope}%
\pgfpathrectangle{\pgfqpoint{0.592976in}{0.451389in}}{\pgfqpoint{3.502277in}{2.786231in}}%
\pgfusepath{clip}%
\pgfsetbuttcap%
\pgfsetroundjoin%
\definecolor{currentfill}{rgb}{0.007843,0.619608,0.450980}%
\pgfsetfillcolor{currentfill}%
\pgfsetfillopacity{0.200000}%
\pgfsetlinewidth{1.003750pt}%
\definecolor{currentstroke}{rgb}{0.007843,0.619608,0.450980}%
\pgfsetstrokecolor{currentstroke}%
\pgfsetstrokeopacity{0.200000}%
\pgfsetdash{}{0pt}%
\pgfsys@defobject{currentmarker}{\pgfqpoint{0.752170in}{1.240464in}}{\pgfqpoint{3.936059in}{2.250245in}}{%
\pgfpathmoveto{\pgfqpoint{0.752170in}{1.244378in}}%
\pgfpathlineto{\pgfqpoint{0.752170in}{1.240464in}}%
\pgfpathlineto{\pgfqpoint{0.776109in}{1.390249in}}%
\pgfpathlineto{\pgfqpoint{0.816008in}{1.491380in}}%
\pgfpathlineto{\pgfqpoint{0.871865in}{1.591541in}}%
\pgfpathlineto{\pgfqpoint{0.943682in}{1.743216in}}%
\pgfpathlineto{\pgfqpoint{1.031459in}{1.814064in}}%
\pgfpathlineto{\pgfqpoint{1.135194in}{1.882849in}}%
\pgfpathlineto{\pgfqpoint{1.254889in}{1.924982in}}%
\pgfpathlineto{\pgfqpoint{1.390544in}{1.966511in}}%
\pgfpathlineto{\pgfqpoint{1.542158in}{2.006307in}}%
\pgfpathlineto{\pgfqpoint{1.893263in}{2.061371in}}%
\pgfpathlineto{\pgfqpoint{2.308206in}{2.117734in}}%
\pgfpathlineto{\pgfqpoint{2.786986in}{2.174612in}}%
\pgfpathlineto{\pgfqpoint{3.329604in}{2.182936in}}%
\pgfpathlineto{\pgfqpoint{3.936059in}{2.247667in}}%
\pgfpathlineto{\pgfqpoint{3.936059in}{2.250245in}}%
\pgfpathlineto{\pgfqpoint{3.936059in}{2.250245in}}%
\pgfpathlineto{\pgfqpoint{3.329604in}{2.215296in}}%
\pgfpathlineto{\pgfqpoint{2.786986in}{2.182201in}}%
\pgfpathlineto{\pgfqpoint{2.308206in}{2.140917in}}%
\pgfpathlineto{\pgfqpoint{1.893263in}{2.092039in}}%
\pgfpathlineto{\pgfqpoint{1.542158in}{2.040175in}}%
\pgfpathlineto{\pgfqpoint{1.390544in}{2.002330in}}%
\pgfpathlineto{\pgfqpoint{1.254889in}{1.933040in}}%
\pgfpathlineto{\pgfqpoint{1.135194in}{1.885418in}}%
\pgfpathlineto{\pgfqpoint{1.031459in}{1.840603in}}%
\pgfpathlineto{\pgfqpoint{0.943682in}{1.748603in}}%
\pgfpathlineto{\pgfqpoint{0.871865in}{1.597169in}}%
\pgfpathlineto{\pgfqpoint{0.816008in}{1.495533in}}%
\pgfpathlineto{\pgfqpoint{0.776109in}{1.467739in}}%
\pgfpathlineto{\pgfqpoint{0.752170in}{1.244378in}}%
\pgfpathlineto{\pgfqpoint{0.752170in}{1.244378in}}%
\pgfpathclose%
\pgfusepath{stroke,fill}%
}%
\begin{pgfscope}%
\pgfsys@transformshift{0.000000in}{0.000000in}%
\pgfsys@useobject{currentmarker}{}%
\end{pgfscope}%
\end{pgfscope}%
\begin{pgfscope}%
\pgfpathrectangle{\pgfqpoint{0.592976in}{0.451389in}}{\pgfqpoint{3.502277in}{2.786231in}}%
\pgfusepath{clip}%
\pgfsetbuttcap%
\pgfsetroundjoin%
\definecolor{currentfill}{rgb}{0.835294,0.368627,0.000000}%
\pgfsetfillcolor{currentfill}%
\pgfsetfillopacity{0.200000}%
\pgfsetlinewidth{1.003750pt}%
\definecolor{currentstroke}{rgb}{0.835294,0.368627,0.000000}%
\pgfsetstrokecolor{currentstroke}%
\pgfsetstrokeopacity{0.200000}%
\pgfsetdash{}{0pt}%
\pgfsys@defobject{currentmarker}{\pgfqpoint{0.752170in}{1.101498in}}{\pgfqpoint{3.936059in}{1.870320in}}{%
\pgfpathmoveto{\pgfqpoint{0.752170in}{1.113851in}}%
\pgfpathlineto{\pgfqpoint{0.752170in}{1.101498in}}%
\pgfpathlineto{\pgfqpoint{0.776109in}{1.270313in}}%
\pgfpathlineto{\pgfqpoint{0.816008in}{1.285082in}}%
\pgfpathlineto{\pgfqpoint{0.871865in}{1.353428in}}%
\pgfpathlineto{\pgfqpoint{0.943682in}{1.420079in}}%
\pgfpathlineto{\pgfqpoint{1.031459in}{1.470738in}}%
\pgfpathlineto{\pgfqpoint{1.135194in}{1.517682in}}%
\pgfpathlineto{\pgfqpoint{1.254889in}{1.572514in}}%
\pgfpathlineto{\pgfqpoint{1.390544in}{1.602365in}}%
\pgfpathlineto{\pgfqpoint{1.542158in}{1.642583in}}%
\pgfpathlineto{\pgfqpoint{1.893263in}{1.693285in}}%
\pgfpathlineto{\pgfqpoint{2.308206in}{1.743149in}}%
\pgfpathlineto{\pgfqpoint{2.786986in}{1.786978in}}%
\pgfpathlineto{\pgfqpoint{3.329604in}{1.818403in}}%
\pgfpathlineto{\pgfqpoint{3.936059in}{1.854850in}}%
\pgfpathlineto{\pgfqpoint{3.936059in}{1.870320in}}%
\pgfpathlineto{\pgfqpoint{3.936059in}{1.870320in}}%
\pgfpathlineto{\pgfqpoint{3.329604in}{1.840316in}}%
\pgfpathlineto{\pgfqpoint{2.786986in}{1.804826in}}%
\pgfpathlineto{\pgfqpoint{2.308206in}{1.772294in}}%
\pgfpathlineto{\pgfqpoint{1.893263in}{1.728224in}}%
\pgfpathlineto{\pgfqpoint{1.542158in}{1.673505in}}%
\pgfpathlineto{\pgfqpoint{1.390544in}{1.651889in}}%
\pgfpathlineto{\pgfqpoint{1.254889in}{1.621941in}}%
\pgfpathlineto{\pgfqpoint{1.135194in}{1.531338in}}%
\pgfpathlineto{\pgfqpoint{1.031459in}{1.491969in}}%
\pgfpathlineto{\pgfqpoint{0.943682in}{1.429663in}}%
\pgfpathlineto{\pgfqpoint{0.871865in}{1.358333in}}%
\pgfpathlineto{\pgfqpoint{0.816008in}{1.294682in}}%
\pgfpathlineto{\pgfqpoint{0.776109in}{1.306289in}}%
\pgfpathlineto{\pgfqpoint{0.752170in}{1.113851in}}%
\pgfpathlineto{\pgfqpoint{0.752170in}{1.113851in}}%
\pgfpathclose%
\pgfusepath{stroke,fill}%
}%
\begin{pgfscope}%
\pgfsys@transformshift{0.000000in}{0.000000in}%
\pgfsys@useobject{currentmarker}{}%
\end{pgfscope}%
\end{pgfscope}%
\begin{pgfscope}%
\pgfpathrectangle{\pgfqpoint{0.592976in}{0.451389in}}{\pgfqpoint{3.502277in}{2.786231in}}%
\pgfusepath{clip}%
\pgfsetbuttcap%
\pgfsetroundjoin%
\definecolor{currentfill}{rgb}{0.800000,0.470588,0.737255}%
\pgfsetfillcolor{currentfill}%
\pgfsetfillopacity{0.200000}%
\pgfsetlinewidth{1.003750pt}%
\definecolor{currentstroke}{rgb}{0.800000,0.470588,0.737255}%
\pgfsetstrokecolor{currentstroke}%
\pgfsetstrokeopacity{0.200000}%
\pgfsetdash{}{0pt}%
\pgfsys@defobject{currentmarker}{\pgfqpoint{0.752170in}{0.621349in}}{\pgfqpoint{3.936059in}{1.678637in}}{%
\pgfpathmoveto{\pgfqpoint{0.752170in}{0.669295in}}%
\pgfpathlineto{\pgfqpoint{0.752170in}{0.621349in}}%
\pgfpathlineto{\pgfqpoint{0.776109in}{0.718645in}}%
\pgfpathlineto{\pgfqpoint{0.816008in}{0.726335in}}%
\pgfpathlineto{\pgfqpoint{0.871865in}{0.802618in}}%
\pgfpathlineto{\pgfqpoint{0.943682in}{0.861389in}}%
\pgfpathlineto{\pgfqpoint{1.031459in}{0.910879in}}%
\pgfpathlineto{\pgfqpoint{1.135194in}{1.512827in}}%
\pgfpathlineto{\pgfqpoint{1.254889in}{1.445367in}}%
\pgfpathlineto{\pgfqpoint{1.390544in}{1.062935in}}%
\pgfpathlineto{\pgfqpoint{1.542158in}{1.059284in}}%
\pgfpathlineto{\pgfqpoint{1.893263in}{1.114042in}}%
\pgfpathlineto{\pgfqpoint{2.308206in}{1.175706in}}%
\pgfpathlineto{\pgfqpoint{2.786986in}{1.244678in}}%
\pgfpathlineto{\pgfqpoint{3.329604in}{1.297226in}}%
\pgfpathlineto{\pgfqpoint{3.936059in}{1.382902in}}%
\pgfpathlineto{\pgfqpoint{3.936059in}{1.503956in}}%
\pgfpathlineto{\pgfqpoint{3.936059in}{1.503956in}}%
\pgfpathlineto{\pgfqpoint{3.329604in}{1.533249in}}%
\pgfpathlineto{\pgfqpoint{2.786986in}{1.309635in}}%
\pgfpathlineto{\pgfqpoint{2.308206in}{1.380473in}}%
\pgfpathlineto{\pgfqpoint{1.893263in}{1.172061in}}%
\pgfpathlineto{\pgfqpoint{1.542158in}{1.318651in}}%
\pgfpathlineto{\pgfqpoint{1.390544in}{1.185783in}}%
\pgfpathlineto{\pgfqpoint{1.254889in}{1.678637in}}%
\pgfpathlineto{\pgfqpoint{1.135194in}{1.651839in}}%
\pgfpathlineto{\pgfqpoint{1.031459in}{0.953411in}}%
\pgfpathlineto{\pgfqpoint{0.943682in}{0.918042in}}%
\pgfpathlineto{\pgfqpoint{0.871865in}{0.824964in}}%
\pgfpathlineto{\pgfqpoint{0.816008in}{0.764727in}}%
\pgfpathlineto{\pgfqpoint{0.776109in}{0.783372in}}%
\pgfpathlineto{\pgfqpoint{0.752170in}{0.669295in}}%
\pgfpathlineto{\pgfqpoint{0.752170in}{0.669295in}}%
\pgfpathclose%
\pgfusepath{stroke,fill}%
}%
\begin{pgfscope}%
\pgfsys@transformshift{0.000000in}{0.000000in}%
\pgfsys@useobject{currentmarker}{}%
\end{pgfscope}%
\end{pgfscope}%
\begin{pgfscope}%
\pgfpathrectangle{\pgfqpoint{0.592976in}{0.451389in}}{\pgfqpoint{3.502277in}{2.786231in}}%
\pgfusepath{clip}%
\pgfsetbuttcap%
\pgfsetroundjoin%
\definecolor{currentfill}{rgb}{0.792157,0.568627,0.380392}%
\pgfsetfillcolor{currentfill}%
\pgfsetfillopacity{0.200000}%
\pgfsetlinewidth{1.003750pt}%
\definecolor{currentstroke}{rgb}{0.792157,0.568627,0.380392}%
\pgfsetstrokecolor{currentstroke}%
\pgfsetstrokeopacity{0.200000}%
\pgfsetdash{}{0pt}%
\pgfsys@defobject{currentmarker}{\pgfqpoint{0.752170in}{0.578035in}}{\pgfqpoint{3.936059in}{1.580442in}}{%
\pgfpathmoveto{\pgfqpoint{0.752170in}{0.643478in}}%
\pgfpathlineto{\pgfqpoint{0.752170in}{0.578035in}}%
\pgfpathlineto{\pgfqpoint{0.776109in}{0.716659in}}%
\pgfpathlineto{\pgfqpoint{0.816008in}{0.713971in}}%
\pgfpathlineto{\pgfqpoint{0.871865in}{0.808523in}}%
\pgfpathlineto{\pgfqpoint{0.943682in}{0.905275in}}%
\pgfpathlineto{\pgfqpoint{1.031459in}{0.977152in}}%
\pgfpathlineto{\pgfqpoint{1.135194in}{1.018325in}}%
\pgfpathlineto{\pgfqpoint{1.254889in}{1.068830in}}%
\pgfpathlineto{\pgfqpoint{1.390544in}{1.094622in}}%
\pgfpathlineto{\pgfqpoint{1.542158in}{1.124249in}}%
\pgfpathlineto{\pgfqpoint{1.893263in}{1.177807in}}%
\pgfpathlineto{\pgfqpoint{2.308206in}{1.227426in}}%
\pgfpathlineto{\pgfqpoint{2.786986in}{1.268106in}}%
\pgfpathlineto{\pgfqpoint{3.329604in}{1.316496in}}%
\pgfpathlineto{\pgfqpoint{3.936059in}{1.359120in}}%
\pgfpathlineto{\pgfqpoint{3.936059in}{1.580442in}}%
\pgfpathlineto{\pgfqpoint{3.936059in}{1.580442in}}%
\pgfpathlineto{\pgfqpoint{3.329604in}{1.554519in}}%
\pgfpathlineto{\pgfqpoint{2.786986in}{1.488978in}}%
\pgfpathlineto{\pgfqpoint{2.308206in}{1.442961in}}%
\pgfpathlineto{\pgfqpoint{1.893263in}{1.194338in}}%
\pgfpathlineto{\pgfqpoint{1.542158in}{1.362929in}}%
\pgfpathlineto{\pgfqpoint{1.390544in}{1.254049in}}%
\pgfpathlineto{\pgfqpoint{1.254889in}{1.101705in}}%
\pgfpathlineto{\pgfqpoint{1.135194in}{1.038474in}}%
\pgfpathlineto{\pgfqpoint{1.031459in}{0.988630in}}%
\pgfpathlineto{\pgfqpoint{0.943682in}{0.942072in}}%
\pgfpathlineto{\pgfqpoint{0.871865in}{0.868602in}}%
\pgfpathlineto{\pgfqpoint{0.816008in}{0.786755in}}%
\pgfpathlineto{\pgfqpoint{0.776109in}{0.765692in}}%
\pgfpathlineto{\pgfqpoint{0.752170in}{0.643478in}}%
\pgfpathlineto{\pgfqpoint{0.752170in}{0.643478in}}%
\pgfpathclose%
\pgfusepath{stroke,fill}%
}%
\begin{pgfscope}%
\pgfsys@transformshift{0.000000in}{0.000000in}%
\pgfsys@useobject{currentmarker}{}%
\end{pgfscope}%
\end{pgfscope}%
\begin{pgfscope}%
\pgfsetrectcap%
\pgfsetmiterjoin%
\pgfsetlinewidth{1.254687pt}%
\definecolor{currentstroke}{rgb}{0.800000,0.800000,0.800000}%
\pgfsetstrokecolor{currentstroke}%
\pgfsetdash{}{0pt}%
\pgfpathmoveto{\pgfqpoint{0.592976in}{0.451389in}}%
\pgfpathlineto{\pgfqpoint{0.592976in}{3.237620in}}%
\pgfusepath{stroke}%
\end{pgfscope}%
\begin{pgfscope}%
\pgfsetrectcap%
\pgfsetmiterjoin%
\pgfsetlinewidth{1.254687pt}%
\definecolor{currentstroke}{rgb}{0.800000,0.800000,0.800000}%
\pgfsetstrokecolor{currentstroke}%
\pgfsetdash{}{0pt}%
\pgfpathmoveto{\pgfqpoint{4.095253in}{0.451389in}}%
\pgfpathlineto{\pgfqpoint{4.095253in}{3.237620in}}%
\pgfusepath{stroke}%
\end{pgfscope}%
\begin{pgfscope}%
\pgfsetrectcap%
\pgfsetmiterjoin%
\pgfsetlinewidth{1.254687pt}%
\definecolor{currentstroke}{rgb}{0.800000,0.800000,0.800000}%
\pgfsetstrokecolor{currentstroke}%
\pgfsetdash{}{0pt}%
\pgfpathmoveto{\pgfqpoint{0.592976in}{0.451389in}}%
\pgfpathlineto{\pgfqpoint{4.095253in}{0.451389in}}%
\pgfusepath{stroke}%
\end{pgfscope}%
\begin{pgfscope}%
\pgfsetrectcap%
\pgfsetmiterjoin%
\pgfsetlinewidth{1.254687pt}%
\definecolor{currentstroke}{rgb}{0.800000,0.800000,0.800000}%
\pgfsetstrokecolor{currentstroke}%
\pgfsetdash{}{0pt}%
\pgfpathmoveto{\pgfqpoint{0.592976in}{3.237620in}}%
\pgfpathlineto{\pgfqpoint{4.095253in}{3.237620in}}%
\pgfusepath{stroke}%
\end{pgfscope}%
\begin{pgfscope}%
\pgfsetbuttcap%
\pgfsetmiterjoin%
\definecolor{currentfill}{rgb}{1.000000,1.000000,1.000000}%
\pgfsetfillcolor{currentfill}%
\pgfsetfillopacity{0.800000}%
\pgfsetlinewidth{1.003750pt}%
\definecolor{currentstroke}{rgb}{0.800000,0.800000,0.800000}%
\pgfsetstrokecolor{currentstroke}%
\pgfsetstrokeopacity{0.800000}%
\pgfsetdash{}{0pt}%
\pgfpathmoveto{\pgfqpoint{4.270310in}{0.521129in}}%
\pgfpathlineto{\pgfqpoint{5.720359in}{0.521129in}}%
\pgfpathquadraticcurveto{\pgfqpoint{5.745359in}{0.521129in}}{\pgfqpoint{5.745359in}{0.546129in}}%
\pgfpathlineto{\pgfqpoint{5.745359in}{3.142880in}}%
\pgfpathquadraticcurveto{\pgfqpoint{5.745359in}{3.167880in}}{\pgfqpoint{5.720359in}{3.167880in}}%
\pgfpathlineto{\pgfqpoint{4.270310in}{3.167880in}}%
\pgfpathquadraticcurveto{\pgfqpoint{4.245310in}{3.167880in}}{\pgfqpoint{4.245310in}{3.142880in}}%
\pgfpathlineto{\pgfqpoint{4.245310in}{0.546129in}}%
\pgfpathquadraticcurveto{\pgfqpoint{4.245310in}{0.521129in}}{\pgfqpoint{4.270310in}{0.521129in}}%
\pgfpathlineto{\pgfqpoint{4.270310in}{0.521129in}}%
\pgfpathclose%
\pgfusepath{stroke,fill}%
\end{pgfscope}%
\begin{pgfscope}%
\definecolor{textcolor}{rgb}{0.150000,0.150000,0.150000}%
\pgfsetstrokecolor{textcolor}%
\pgfsetfillcolor{textcolor}%
\pgftext[x=4.791935in,y=3.022909in,left,base]{\color{textcolor}\sffamily\fontsize{9.000000}{10.800000}\selectfont Legend}%
\end{pgfscope}%
\begin{pgfscope}%
\pgfsetroundcap%
\pgfsetroundjoin%
\pgfsetlinewidth{1.505625pt}%
\definecolor{currentstroke}{rgb}{0.003922,0.450980,0.698039}%
\pgfsetstrokecolor{currentstroke}%
\pgfsetdash{}{0pt}%
\pgfpathmoveto{\pgfqpoint{4.295310in}{2.879159in}}%
\pgfpathlineto{\pgfqpoint{4.420310in}{2.879159in}}%
\pgfpathlineto{\pgfqpoint{4.545310in}{2.879159in}}%
\pgfusepath{stroke}%
\end{pgfscope}%
\begin{pgfscope}%
\definecolor{textcolor}{rgb}{0.150000,0.150000,0.150000}%
\pgfsetstrokecolor{textcolor}%
\pgfsetfillcolor{textcolor}%
\pgftext[x=4.645310in,y=2.835409in,left,base]{\color{textcolor}\sffamily\fontsize{9.000000}{10.800000}\selectfont Total time}%
\end{pgfscope}%
\begin{pgfscope}%
\pgfsetroundcap%
\pgfsetroundjoin%
\pgfsetlinewidth{1.505625pt}%
\definecolor{currentstroke}{rgb}{0.870588,0.560784,0.019608}%
\pgfsetstrokecolor{currentstroke}%
\pgfsetdash{}{0pt}%
\pgfpathmoveto{\pgfqpoint{4.295310in}{2.532651in}}%
\pgfpathlineto{\pgfqpoint{4.420310in}{2.532651in}}%
\pgfpathlineto{\pgfqpoint{4.545310in}{2.532651in}}%
\pgfusepath{stroke}%
\end{pgfscope}%
\begin{pgfscope}%
\definecolor{textcolor}{rgb}{0.150000,0.150000,0.150000}%
\pgfsetstrokecolor{textcolor}%
\pgfsetfillcolor{textcolor}%
\pgftext[x=4.645310in, y=2.647909in, left, base]{\color{textcolor}\sffamily\fontsize{9.000000}{10.800000}\selectfont Find all \(\displaystyle k\)}%
\end{pgfscope}%
\begin{pgfscope}%
\definecolor{textcolor}{rgb}{0.150000,0.150000,0.150000}%
\pgfsetstrokecolor{textcolor}%
\pgfsetfillcolor{textcolor}%
\pgftext[x=4.645310in, y=2.503915in, left, base]{\color{textcolor}\sffamily\fontsize{9.000000}{10.800000}\selectfont nearest visibility}%
\end{pgfscope}%
\begin{pgfscope}%
\definecolor{textcolor}{rgb}{0.150000,0.150000,0.150000}%
\pgfsetstrokecolor{textcolor}%
\pgfsetfillcolor{textcolor}%
\pgftext[x=4.645310in, y=2.359921in, left, base]{\color{textcolor}\sffamily\fontsize{9.000000}{10.800000}\selectfont neighbors}%
\end{pgfscope}%
\begin{pgfscope}%
\pgfsetroundcap%
\pgfsetroundjoin%
\pgfsetlinewidth{1.505625pt}%
\definecolor{currentstroke}{rgb}{0.007843,0.619608,0.450980}%
\pgfsetstrokecolor{currentstroke}%
\pgfsetdash{}{0pt}%
\pgfpathmoveto{\pgfqpoint{4.295310in}{2.056760in}}%
\pgfpathlineto{\pgfqpoint{4.420310in}{2.056760in}}%
\pgfpathlineto{\pgfqpoint{4.545310in}{2.056760in}}%
\pgfusepath{stroke}%
\end{pgfscope}%
\begin{pgfscope}%
\definecolor{textcolor}{rgb}{0.150000,0.150000,0.150000}%
\pgfsetstrokecolor{textcolor}%
\pgfsetfillcolor{textcolor}%
\pgftext[x=4.645310in, y=2.172421in, left, base]{\color{textcolor}\sffamily\fontsize{9.000000}{10.800000}\selectfont Building the}%
\end{pgfscope}%
\begin{pgfscope}%
\definecolor{textcolor}{rgb}{0.150000,0.150000,0.150000}%
\pgfsetstrokecolor{textcolor}%
\pgfsetfillcolor{textcolor}%
\pgftext[x=4.645310in, y=2.027622in, left, base]{\color{textcolor}\sffamily\fontsize{9.000000}{10.800000}\selectfont graph from}%
\end{pgfscope}%
\begin{pgfscope}%
\definecolor{textcolor}{rgb}{0.150000,0.150000,0.150000}%
\pgfsetstrokecolor{textcolor}%
\pgfsetfillcolor{textcolor}%
\pgftext[x=4.645310in, y=1.883628in, left, base]{\color{textcolor}\sffamily\fontsize{9.000000}{10.800000}\selectfont visibility neighbors}%
\end{pgfscope}%
\begin{pgfscope}%
\pgfsetroundcap%
\pgfsetroundjoin%
\pgfsetlinewidth{1.505625pt}%
\definecolor{currentstroke}{rgb}{0.835294,0.368627,0.000000}%
\pgfsetstrokecolor{currentstroke}%
\pgfsetdash{}{0pt}%
\pgfpathmoveto{\pgfqpoint{4.295310in}{1.580869in}}%
\pgfpathlineto{\pgfqpoint{4.420310in}{1.580869in}}%
\pgfpathlineto{\pgfqpoint{4.545310in}{1.580869in}}%
\pgfusepath{stroke}%
\end{pgfscope}%
\begin{pgfscope}%
\definecolor{textcolor}{rgb}{0.150000,0.150000,0.150000}%
\pgfsetstrokecolor{textcolor}%
\pgfsetfillcolor{textcolor}%
\pgftext[x=4.645310in, y=1.696128in, left, base]{\color{textcolor}\sffamily\fontsize{9.000000}{10.800000}\selectfont Getting and}%
\end{pgfscope}%
\begin{pgfscope}%
\definecolor{textcolor}{rgb}{0.150000,0.150000,0.150000}%
\pgfsetstrokecolor{textcolor}%
\pgfsetfillcolor{textcolor}%
\pgftext[x=4.645310in, y=1.552134in, left, base]{\color{textcolor}\sffamily\fontsize{9.000000}{10.800000}\selectfont preparing all}%
\end{pgfscope}%
\begin{pgfscope}%
\definecolor{textcolor}{rgb}{0.150000,0.150000,0.150000}%
\pgfsetstrokecolor{textcolor}%
\pgfsetfillcolor{textcolor}%
\pgftext[x=4.645310in, y=1.408140in, left, base]{\color{textcolor}\sffamily\fontsize{9.000000}{10.800000}\selectfont obstacles}%
\end{pgfscope}%
\begin{pgfscope}%
\pgfsetroundcap%
\pgfsetroundjoin%
\pgfsetlinewidth{1.505625pt}%
\definecolor{currentstroke}{rgb}{0.800000,0.470588,0.737255}%
\pgfsetstrokecolor{currentstroke}%
\pgfsetdash{}{0pt}%
\pgfpathmoveto{\pgfqpoint{4.295310in}{1.105381in}}%
\pgfpathlineto{\pgfqpoint{4.420310in}{1.105381in}}%
\pgfpathlineto{\pgfqpoint{4.545310in}{1.105381in}}%
\pgfusepath{stroke}%
\end{pgfscope}%
\begin{pgfscope}%
\definecolor{textcolor}{rgb}{0.150000,0.150000,0.150000}%
\pgfsetstrokecolor{textcolor}%
\pgfsetfillcolor{textcolor}%
\pgftext[x=4.645310in, y=1.220640in, left, base]{\color{textcolor}\sffamily\fontsize{9.000000}{10.800000}\selectfont Merge V-Graph}%
\end{pgfscope}%
\begin{pgfscope}%
\definecolor{textcolor}{rgb}{0.150000,0.150000,0.150000}%
\pgfsetstrokecolor{textcolor}%
\pgfsetfillcolor{textcolor}%
\pgftext[x=4.645310in, y=1.076646in, left, base]{\color{textcolor}\sffamily\fontsize{9.000000}{10.800000}\selectfont into the}%
\end{pgfscope}%
\begin{pgfscope}%
\definecolor{textcolor}{rgb}{0.150000,0.150000,0.150000}%
\pgfsetstrokecolor{textcolor}%
\pgfsetfillcolor{textcolor}%
\pgftext[x=4.645310in, y=0.932652in, left, base]{\color{textcolor}\sffamily\fontsize{9.000000}{10.800000}\selectfont road graph}%
\end{pgfscope}%
\begin{pgfscope}%
\pgfsetroundcap%
\pgfsetroundjoin%
\pgfsetlinewidth{1.505625pt}%
\definecolor{currentstroke}{rgb}{0.792157,0.568627,0.380392}%
\pgfsetstrokecolor{currentstroke}%
\pgfsetdash{}{0pt}%
\pgfpathmoveto{\pgfqpoint{4.295310in}{0.701890in}}%
\pgfpathlineto{\pgfqpoint{4.420310in}{0.701890in}}%
\pgfpathlineto{\pgfqpoint{4.545310in}{0.701890in}}%
\pgfusepath{stroke}%
\end{pgfscope}%
\begin{pgfscope}%
\definecolor{textcolor}{rgb}{0.150000,0.150000,0.150000}%
\pgfsetstrokecolor{textcolor}%
\pgfsetfillcolor{textcolor}%
\pgftext[x=4.645310in, y=0.745152in, left, base]{\color{textcolor}\sffamily\fontsize{9.000000}{10.800000}\selectfont Add attributed}%
\end{pgfscope}%
\begin{pgfscope}%
\definecolor{textcolor}{rgb}{0.150000,0.150000,0.150000}%
\pgfsetstrokecolor{textcolor}%
\pgfsetfillcolor{textcolor}%
\pgftext[x=4.645310in, y=0.601158in, left, base]{\color{textcolor}\sffamily\fontsize{9.000000}{10.800000}\selectfont to POIs}%
\end{pgfscope}%
\begin{pgfscope}%
\pgfsetroundcap%
\pgfsetroundjoin%
\pgfsetlinewidth{1.003750pt}%
\definecolor{currentstroke}{rgb}{0.003922,0.450980,0.698039}%
\pgfsetstrokecolor{currentstroke}%
\pgfsetdash{}{0pt}%
\pgfpathmoveto{\pgfqpoint{0.752170in}{1.549605in}}%
\pgfpathlineto{\pgfqpoint{0.776109in}{1.766958in}}%
\pgfpathlineto{\pgfqpoint{0.816008in}{1.936892in}}%
\pgfpathlineto{\pgfqpoint{0.871865in}{2.108379in}}%
\pgfpathlineto{\pgfqpoint{0.943682in}{2.230863in}}%
\pgfpathlineto{\pgfqpoint{1.031459in}{2.341881in}}%
\pgfpathlineto{\pgfqpoint{1.135194in}{2.418664in}}%
\pgfpathlineto{\pgfqpoint{1.254889in}{2.523568in}}%
\pgfpathlineto{\pgfqpoint{1.390544in}{2.563022in}}%
\pgfpathlineto{\pgfqpoint{1.542158in}{2.636859in}}%
\pgfpathlineto{\pgfqpoint{1.893263in}{2.766384in}}%
\pgfpathlineto{\pgfqpoint{2.308206in}{2.845198in}}%
\pgfpathlineto{\pgfqpoint{2.786986in}{2.984851in}}%
\pgfpathlineto{\pgfqpoint{3.329604in}{3.012255in}}%
\pgfpathlineto{\pgfqpoint{3.936059in}{3.110472in}}%
\pgfusepath{stroke}%
\end{pgfscope}%
\begin{pgfscope}%
\pgfsetbuttcap%
\pgfsetroundjoin%
\definecolor{currentfill}{rgb}{0.003922,0.450980,0.698039}%
\pgfsetfillcolor{currentfill}%
\pgfsetlinewidth{0.752812pt}%
\definecolor{currentstroke}{rgb}{1.000000,1.000000,1.000000}%
\pgfsetstrokecolor{currentstroke}%
\pgfsetdash{}{0pt}%
\pgfsys@defobject{currentmarker}{\pgfqpoint{-0.034722in}{-0.034722in}}{\pgfqpoint{0.034722in}{0.034722in}}{%
\pgfpathmoveto{\pgfqpoint{0.000000in}{-0.034722in}}%
\pgfpathcurveto{\pgfqpoint{0.009208in}{-0.034722in}}{\pgfqpoint{0.018041in}{-0.031064in}}{\pgfqpoint{0.024552in}{-0.024552in}}%
\pgfpathcurveto{\pgfqpoint{0.031064in}{-0.018041in}}{\pgfqpoint{0.034722in}{-0.009208in}}{\pgfqpoint{0.034722in}{0.000000in}}%
\pgfpathcurveto{\pgfqpoint{0.034722in}{0.009208in}}{\pgfqpoint{0.031064in}{0.018041in}}{\pgfqpoint{0.024552in}{0.024552in}}%
\pgfpathcurveto{\pgfqpoint{0.018041in}{0.031064in}}{\pgfqpoint{0.009208in}{0.034722in}}{\pgfqpoint{0.000000in}{0.034722in}}%
\pgfpathcurveto{\pgfqpoint{-0.009208in}{0.034722in}}{\pgfqpoint{-0.018041in}{0.031064in}}{\pgfqpoint{-0.024552in}{0.024552in}}%
\pgfpathcurveto{\pgfqpoint{-0.031064in}{0.018041in}}{\pgfqpoint{-0.034722in}{0.009208in}}{\pgfqpoint{-0.034722in}{0.000000in}}%
\pgfpathcurveto{\pgfqpoint{-0.034722in}{-0.009208in}}{\pgfqpoint{-0.031064in}{-0.018041in}}{\pgfqpoint{-0.024552in}{-0.024552in}}%
\pgfpathcurveto{\pgfqpoint{-0.018041in}{-0.031064in}}{\pgfqpoint{-0.009208in}{-0.034722in}}{\pgfqpoint{0.000000in}{-0.034722in}}%
\pgfpathlineto{\pgfqpoint{0.000000in}{-0.034722in}}%
\pgfpathclose%
\pgfusepath{stroke,fill}%
}%
\begin{pgfscope}%
\pgfsys@transformshift{0.752170in}{1.549605in}%
\pgfsys@useobject{currentmarker}{}%
\end{pgfscope}%
\begin{pgfscope}%
\pgfsys@transformshift{0.776109in}{1.766958in}%
\pgfsys@useobject{currentmarker}{}%
\end{pgfscope}%
\begin{pgfscope}%
\pgfsys@transformshift{0.816008in}{1.936892in}%
\pgfsys@useobject{currentmarker}{}%
\end{pgfscope}%
\begin{pgfscope}%
\pgfsys@transformshift{0.871865in}{2.108379in}%
\pgfsys@useobject{currentmarker}{}%
\end{pgfscope}%
\begin{pgfscope}%
\pgfsys@transformshift{0.943682in}{2.230863in}%
\pgfsys@useobject{currentmarker}{}%
\end{pgfscope}%
\begin{pgfscope}%
\pgfsys@transformshift{1.031459in}{2.341881in}%
\pgfsys@useobject{currentmarker}{}%
\end{pgfscope}%
\begin{pgfscope}%
\pgfsys@transformshift{1.135194in}{2.418664in}%
\pgfsys@useobject{currentmarker}{}%
\end{pgfscope}%
\begin{pgfscope}%
\pgfsys@transformshift{1.254889in}{2.523568in}%
\pgfsys@useobject{currentmarker}{}%
\end{pgfscope}%
\begin{pgfscope}%
\pgfsys@transformshift{1.390544in}{2.563022in}%
\pgfsys@useobject{currentmarker}{}%
\end{pgfscope}%
\begin{pgfscope}%
\pgfsys@transformshift{1.542158in}{2.636859in}%
\pgfsys@useobject{currentmarker}{}%
\end{pgfscope}%
\begin{pgfscope}%
\pgfsys@transformshift{1.893263in}{2.766384in}%
\pgfsys@useobject{currentmarker}{}%
\end{pgfscope}%
\begin{pgfscope}%
\pgfsys@transformshift{2.308206in}{2.845198in}%
\pgfsys@useobject{currentmarker}{}%
\end{pgfscope}%
\begin{pgfscope}%
\pgfsys@transformshift{2.786986in}{2.984851in}%
\pgfsys@useobject{currentmarker}{}%
\end{pgfscope}%
\begin{pgfscope}%
\pgfsys@transformshift{3.329604in}{3.012255in}%
\pgfsys@useobject{currentmarker}{}%
\end{pgfscope}%
\begin{pgfscope}%
\pgfsys@transformshift{3.936059in}{3.110472in}%
\pgfsys@useobject{currentmarker}{}%
\end{pgfscope}%
\end{pgfscope}%
\begin{pgfscope}%
\pgfsetroundcap%
\pgfsetroundjoin%
\pgfsetlinewidth{1.003750pt}%
\definecolor{currentstroke}{rgb}{0.870588,0.560784,0.019608}%
\pgfsetstrokecolor{currentstroke}%
\pgfsetdash{}{0pt}%
\pgfpathmoveto{\pgfqpoint{0.752170in}{1.503219in}}%
\pgfpathlineto{\pgfqpoint{0.776109in}{1.733234in}}%
\pgfpathlineto{\pgfqpoint{0.816008in}{1.923975in}}%
\pgfpathlineto{\pgfqpoint{0.871865in}{2.100919in}}%
\pgfpathlineto{\pgfqpoint{0.943682in}{2.223038in}}%
\pgfpathlineto{\pgfqpoint{1.031459in}{2.335470in}}%
\pgfpathlineto{\pgfqpoint{1.135194in}{2.412366in}}%
\pgfpathlineto{\pgfqpoint{1.254889in}{2.519408in}}%
\pgfpathlineto{\pgfqpoint{1.390544in}{2.558856in}}%
\pgfpathlineto{\pgfqpoint{1.542158in}{2.633705in}}%
\pgfpathlineto{\pgfqpoint{1.893263in}{2.764493in}}%
\pgfpathlineto{\pgfqpoint{2.308206in}{2.843588in}}%
\pgfpathlineto{\pgfqpoint{2.786986in}{2.984003in}}%
\pgfpathlineto{\pgfqpoint{3.329604in}{3.011419in}}%
\pgfpathlineto{\pgfqpoint{3.936059in}{3.109888in}}%
\pgfusepath{stroke}%
\end{pgfscope}%
\begin{pgfscope}%
\pgfsetbuttcap%
\pgfsetroundjoin%
\definecolor{currentfill}{rgb}{0.870588,0.560784,0.019608}%
\pgfsetfillcolor{currentfill}%
\pgfsetlinewidth{0.752812pt}%
\definecolor{currentstroke}{rgb}{1.000000,1.000000,1.000000}%
\pgfsetstrokecolor{currentstroke}%
\pgfsetdash{}{0pt}%
\pgfsys@defobject{currentmarker}{\pgfqpoint{-0.034722in}{-0.034722in}}{\pgfqpoint{0.034722in}{0.034722in}}{%
\pgfpathmoveto{\pgfqpoint{0.000000in}{-0.034722in}}%
\pgfpathcurveto{\pgfqpoint{0.009208in}{-0.034722in}}{\pgfqpoint{0.018041in}{-0.031064in}}{\pgfqpoint{0.024552in}{-0.024552in}}%
\pgfpathcurveto{\pgfqpoint{0.031064in}{-0.018041in}}{\pgfqpoint{0.034722in}{-0.009208in}}{\pgfqpoint{0.034722in}{0.000000in}}%
\pgfpathcurveto{\pgfqpoint{0.034722in}{0.009208in}}{\pgfqpoint{0.031064in}{0.018041in}}{\pgfqpoint{0.024552in}{0.024552in}}%
\pgfpathcurveto{\pgfqpoint{0.018041in}{0.031064in}}{\pgfqpoint{0.009208in}{0.034722in}}{\pgfqpoint{0.000000in}{0.034722in}}%
\pgfpathcurveto{\pgfqpoint{-0.009208in}{0.034722in}}{\pgfqpoint{-0.018041in}{0.031064in}}{\pgfqpoint{-0.024552in}{0.024552in}}%
\pgfpathcurveto{\pgfqpoint{-0.031064in}{0.018041in}}{\pgfqpoint{-0.034722in}{0.009208in}}{\pgfqpoint{-0.034722in}{0.000000in}}%
\pgfpathcurveto{\pgfqpoint{-0.034722in}{-0.009208in}}{\pgfqpoint{-0.031064in}{-0.018041in}}{\pgfqpoint{-0.024552in}{-0.024552in}}%
\pgfpathcurveto{\pgfqpoint{-0.018041in}{-0.031064in}}{\pgfqpoint{-0.009208in}{-0.034722in}}{\pgfqpoint{0.000000in}{-0.034722in}}%
\pgfpathlineto{\pgfqpoint{0.000000in}{-0.034722in}}%
\pgfpathclose%
\pgfusepath{stroke,fill}%
}%
\begin{pgfscope}%
\pgfsys@transformshift{0.752170in}{1.503219in}%
\pgfsys@useobject{currentmarker}{}%
\end{pgfscope}%
\begin{pgfscope}%
\pgfsys@transformshift{0.776109in}{1.733234in}%
\pgfsys@useobject{currentmarker}{}%
\end{pgfscope}%
\begin{pgfscope}%
\pgfsys@transformshift{0.816008in}{1.923975in}%
\pgfsys@useobject{currentmarker}{}%
\end{pgfscope}%
\begin{pgfscope}%
\pgfsys@transformshift{0.871865in}{2.100919in}%
\pgfsys@useobject{currentmarker}{}%
\end{pgfscope}%
\begin{pgfscope}%
\pgfsys@transformshift{0.943682in}{2.223038in}%
\pgfsys@useobject{currentmarker}{}%
\end{pgfscope}%
\begin{pgfscope}%
\pgfsys@transformshift{1.031459in}{2.335470in}%
\pgfsys@useobject{currentmarker}{}%
\end{pgfscope}%
\begin{pgfscope}%
\pgfsys@transformshift{1.135194in}{2.412366in}%
\pgfsys@useobject{currentmarker}{}%
\end{pgfscope}%
\begin{pgfscope}%
\pgfsys@transformshift{1.254889in}{2.519408in}%
\pgfsys@useobject{currentmarker}{}%
\end{pgfscope}%
\begin{pgfscope}%
\pgfsys@transformshift{1.390544in}{2.558856in}%
\pgfsys@useobject{currentmarker}{}%
\end{pgfscope}%
\begin{pgfscope}%
\pgfsys@transformshift{1.542158in}{2.633705in}%
\pgfsys@useobject{currentmarker}{}%
\end{pgfscope}%
\begin{pgfscope}%
\pgfsys@transformshift{1.893263in}{2.764493in}%
\pgfsys@useobject{currentmarker}{}%
\end{pgfscope}%
\begin{pgfscope}%
\pgfsys@transformshift{2.308206in}{2.843588in}%
\pgfsys@useobject{currentmarker}{}%
\end{pgfscope}%
\begin{pgfscope}%
\pgfsys@transformshift{2.786986in}{2.984003in}%
\pgfsys@useobject{currentmarker}{}%
\end{pgfscope}%
\begin{pgfscope}%
\pgfsys@transformshift{3.329604in}{3.011419in}%
\pgfsys@useobject{currentmarker}{}%
\end{pgfscope}%
\begin{pgfscope}%
\pgfsys@transformshift{3.936059in}{3.109888in}%
\pgfsys@useobject{currentmarker}{}%
\end{pgfscope}%
\end{pgfscope}%
\begin{pgfscope}%
\pgfsetroundcap%
\pgfsetroundjoin%
\pgfsetlinewidth{1.003750pt}%
\definecolor{currentstroke}{rgb}{0.007843,0.619608,0.450980}%
\pgfsetstrokecolor{currentstroke}%
\pgfsetdash{}{0pt}%
\pgfpathmoveto{\pgfqpoint{0.752170in}{1.242383in}}%
\pgfpathlineto{\pgfqpoint{0.776109in}{1.445955in}}%
\pgfpathlineto{\pgfqpoint{0.816008in}{1.493767in}}%
\pgfpathlineto{\pgfqpoint{0.871865in}{1.594417in}}%
\pgfpathlineto{\pgfqpoint{0.943682in}{1.746145in}}%
\pgfpathlineto{\pgfqpoint{1.031459in}{1.833412in}}%
\pgfpathlineto{\pgfqpoint{1.135194in}{1.884095in}}%
\pgfpathlineto{\pgfqpoint{1.254889in}{1.928281in}}%
\pgfpathlineto{\pgfqpoint{1.390544in}{1.988111in}}%
\pgfpathlineto{\pgfqpoint{1.542158in}{2.024397in}}%
\pgfpathlineto{\pgfqpoint{1.893263in}{2.079852in}}%
\pgfpathlineto{\pgfqpoint{2.308206in}{2.134182in}}%
\pgfpathlineto{\pgfqpoint{2.786986in}{2.180151in}}%
\pgfpathlineto{\pgfqpoint{3.329604in}{2.202290in}}%
\pgfpathlineto{\pgfqpoint{3.936059in}{2.248868in}}%
\pgfusepath{stroke}%
\end{pgfscope}%
\begin{pgfscope}%
\pgfsetbuttcap%
\pgfsetroundjoin%
\definecolor{currentfill}{rgb}{0.007843,0.619608,0.450980}%
\pgfsetfillcolor{currentfill}%
\pgfsetlinewidth{0.752812pt}%
\definecolor{currentstroke}{rgb}{1.000000,1.000000,1.000000}%
\pgfsetstrokecolor{currentstroke}%
\pgfsetdash{}{0pt}%
\pgfsys@defobject{currentmarker}{\pgfqpoint{-0.034722in}{-0.034722in}}{\pgfqpoint{0.034722in}{0.034722in}}{%
\pgfpathmoveto{\pgfqpoint{0.000000in}{-0.034722in}}%
\pgfpathcurveto{\pgfqpoint{0.009208in}{-0.034722in}}{\pgfqpoint{0.018041in}{-0.031064in}}{\pgfqpoint{0.024552in}{-0.024552in}}%
\pgfpathcurveto{\pgfqpoint{0.031064in}{-0.018041in}}{\pgfqpoint{0.034722in}{-0.009208in}}{\pgfqpoint{0.034722in}{0.000000in}}%
\pgfpathcurveto{\pgfqpoint{0.034722in}{0.009208in}}{\pgfqpoint{0.031064in}{0.018041in}}{\pgfqpoint{0.024552in}{0.024552in}}%
\pgfpathcurveto{\pgfqpoint{0.018041in}{0.031064in}}{\pgfqpoint{0.009208in}{0.034722in}}{\pgfqpoint{0.000000in}{0.034722in}}%
\pgfpathcurveto{\pgfqpoint{-0.009208in}{0.034722in}}{\pgfqpoint{-0.018041in}{0.031064in}}{\pgfqpoint{-0.024552in}{0.024552in}}%
\pgfpathcurveto{\pgfqpoint{-0.031064in}{0.018041in}}{\pgfqpoint{-0.034722in}{0.009208in}}{\pgfqpoint{-0.034722in}{0.000000in}}%
\pgfpathcurveto{\pgfqpoint{-0.034722in}{-0.009208in}}{\pgfqpoint{-0.031064in}{-0.018041in}}{\pgfqpoint{-0.024552in}{-0.024552in}}%
\pgfpathcurveto{\pgfqpoint{-0.018041in}{-0.031064in}}{\pgfqpoint{-0.009208in}{-0.034722in}}{\pgfqpoint{0.000000in}{-0.034722in}}%
\pgfpathlineto{\pgfqpoint{0.000000in}{-0.034722in}}%
\pgfpathclose%
\pgfusepath{stroke,fill}%
}%
\begin{pgfscope}%
\pgfsys@transformshift{0.752170in}{1.242383in}%
\pgfsys@useobject{currentmarker}{}%
\end{pgfscope}%
\begin{pgfscope}%
\pgfsys@transformshift{0.776109in}{1.445955in}%
\pgfsys@useobject{currentmarker}{}%
\end{pgfscope}%
\begin{pgfscope}%
\pgfsys@transformshift{0.816008in}{1.493767in}%
\pgfsys@useobject{currentmarker}{}%
\end{pgfscope}%
\begin{pgfscope}%
\pgfsys@transformshift{0.871865in}{1.594417in}%
\pgfsys@useobject{currentmarker}{}%
\end{pgfscope}%
\begin{pgfscope}%
\pgfsys@transformshift{0.943682in}{1.746145in}%
\pgfsys@useobject{currentmarker}{}%
\end{pgfscope}%
\begin{pgfscope}%
\pgfsys@transformshift{1.031459in}{1.833412in}%
\pgfsys@useobject{currentmarker}{}%
\end{pgfscope}%
\begin{pgfscope}%
\pgfsys@transformshift{1.135194in}{1.884095in}%
\pgfsys@useobject{currentmarker}{}%
\end{pgfscope}%
\begin{pgfscope}%
\pgfsys@transformshift{1.254889in}{1.928281in}%
\pgfsys@useobject{currentmarker}{}%
\end{pgfscope}%
\begin{pgfscope}%
\pgfsys@transformshift{1.390544in}{1.988111in}%
\pgfsys@useobject{currentmarker}{}%
\end{pgfscope}%
\begin{pgfscope}%
\pgfsys@transformshift{1.542158in}{2.024397in}%
\pgfsys@useobject{currentmarker}{}%
\end{pgfscope}%
\begin{pgfscope}%
\pgfsys@transformshift{1.893263in}{2.079852in}%
\pgfsys@useobject{currentmarker}{}%
\end{pgfscope}%
\begin{pgfscope}%
\pgfsys@transformshift{2.308206in}{2.134182in}%
\pgfsys@useobject{currentmarker}{}%
\end{pgfscope}%
\begin{pgfscope}%
\pgfsys@transformshift{2.786986in}{2.180151in}%
\pgfsys@useobject{currentmarker}{}%
\end{pgfscope}%
\begin{pgfscope}%
\pgfsys@transformshift{3.329604in}{2.202290in}%
\pgfsys@useobject{currentmarker}{}%
\end{pgfscope}%
\begin{pgfscope}%
\pgfsys@transformshift{3.936059in}{2.248868in}%
\pgfsys@useobject{currentmarker}{}%
\end{pgfscope}%
\end{pgfscope}%
\begin{pgfscope}%
\pgfsetroundcap%
\pgfsetroundjoin%
\pgfsetlinewidth{1.003750pt}%
\definecolor{currentstroke}{rgb}{0.835294,0.368627,0.000000}%
\pgfsetstrokecolor{currentstroke}%
\pgfsetdash{}{0pt}%
\pgfpathmoveto{\pgfqpoint{0.752170in}{1.107628in}}%
\pgfpathlineto{\pgfqpoint{0.776109in}{1.290644in}}%
\pgfpathlineto{\pgfqpoint{0.816008in}{1.289714in}}%
\pgfpathlineto{\pgfqpoint{0.871865in}{1.355020in}}%
\pgfpathlineto{\pgfqpoint{0.943682in}{1.424046in}}%
\pgfpathlineto{\pgfqpoint{1.031459in}{1.480011in}}%
\pgfpathlineto{\pgfqpoint{1.135194in}{1.523865in}}%
\pgfpathlineto{\pgfqpoint{1.254889in}{1.599029in}}%
\pgfpathlineto{\pgfqpoint{1.390544in}{1.630113in}}%
\pgfpathlineto{\pgfqpoint{1.542158in}{1.656316in}}%
\pgfpathlineto{\pgfqpoint{1.893263in}{1.710110in}}%
\pgfpathlineto{\pgfqpoint{2.308206in}{1.756857in}}%
\pgfpathlineto{\pgfqpoint{2.786986in}{1.794107in}}%
\pgfpathlineto{\pgfqpoint{3.329604in}{1.827547in}}%
\pgfpathlineto{\pgfqpoint{3.936059in}{1.862448in}}%
\pgfusepath{stroke}%
\end{pgfscope}%
\begin{pgfscope}%
\pgfsetbuttcap%
\pgfsetroundjoin%
\definecolor{currentfill}{rgb}{0.835294,0.368627,0.000000}%
\pgfsetfillcolor{currentfill}%
\pgfsetlinewidth{0.752812pt}%
\definecolor{currentstroke}{rgb}{1.000000,1.000000,1.000000}%
\pgfsetstrokecolor{currentstroke}%
\pgfsetdash{}{0pt}%
\pgfsys@defobject{currentmarker}{\pgfqpoint{-0.034722in}{-0.034722in}}{\pgfqpoint{0.034722in}{0.034722in}}{%
\pgfpathmoveto{\pgfqpoint{0.000000in}{-0.034722in}}%
\pgfpathcurveto{\pgfqpoint{0.009208in}{-0.034722in}}{\pgfqpoint{0.018041in}{-0.031064in}}{\pgfqpoint{0.024552in}{-0.024552in}}%
\pgfpathcurveto{\pgfqpoint{0.031064in}{-0.018041in}}{\pgfqpoint{0.034722in}{-0.009208in}}{\pgfqpoint{0.034722in}{0.000000in}}%
\pgfpathcurveto{\pgfqpoint{0.034722in}{0.009208in}}{\pgfqpoint{0.031064in}{0.018041in}}{\pgfqpoint{0.024552in}{0.024552in}}%
\pgfpathcurveto{\pgfqpoint{0.018041in}{0.031064in}}{\pgfqpoint{0.009208in}{0.034722in}}{\pgfqpoint{0.000000in}{0.034722in}}%
\pgfpathcurveto{\pgfqpoint{-0.009208in}{0.034722in}}{\pgfqpoint{-0.018041in}{0.031064in}}{\pgfqpoint{-0.024552in}{0.024552in}}%
\pgfpathcurveto{\pgfqpoint{-0.031064in}{0.018041in}}{\pgfqpoint{-0.034722in}{0.009208in}}{\pgfqpoint{-0.034722in}{0.000000in}}%
\pgfpathcurveto{\pgfqpoint{-0.034722in}{-0.009208in}}{\pgfqpoint{-0.031064in}{-0.018041in}}{\pgfqpoint{-0.024552in}{-0.024552in}}%
\pgfpathcurveto{\pgfqpoint{-0.018041in}{-0.031064in}}{\pgfqpoint{-0.009208in}{-0.034722in}}{\pgfqpoint{0.000000in}{-0.034722in}}%
\pgfpathlineto{\pgfqpoint{0.000000in}{-0.034722in}}%
\pgfpathclose%
\pgfusepath{stroke,fill}%
}%
\begin{pgfscope}%
\pgfsys@transformshift{0.752170in}{1.107628in}%
\pgfsys@useobject{currentmarker}{}%
\end{pgfscope}%
\begin{pgfscope}%
\pgfsys@transformshift{0.776109in}{1.290644in}%
\pgfsys@useobject{currentmarker}{}%
\end{pgfscope}%
\begin{pgfscope}%
\pgfsys@transformshift{0.816008in}{1.289714in}%
\pgfsys@useobject{currentmarker}{}%
\end{pgfscope}%
\begin{pgfscope}%
\pgfsys@transformshift{0.871865in}{1.355020in}%
\pgfsys@useobject{currentmarker}{}%
\end{pgfscope}%
\begin{pgfscope}%
\pgfsys@transformshift{0.943682in}{1.424046in}%
\pgfsys@useobject{currentmarker}{}%
\end{pgfscope}%
\begin{pgfscope}%
\pgfsys@transformshift{1.031459in}{1.480011in}%
\pgfsys@useobject{currentmarker}{}%
\end{pgfscope}%
\begin{pgfscope}%
\pgfsys@transformshift{1.135194in}{1.523865in}%
\pgfsys@useobject{currentmarker}{}%
\end{pgfscope}%
\begin{pgfscope}%
\pgfsys@transformshift{1.254889in}{1.599029in}%
\pgfsys@useobject{currentmarker}{}%
\end{pgfscope}%
\begin{pgfscope}%
\pgfsys@transformshift{1.390544in}{1.630113in}%
\pgfsys@useobject{currentmarker}{}%
\end{pgfscope}%
\begin{pgfscope}%
\pgfsys@transformshift{1.542158in}{1.656316in}%
\pgfsys@useobject{currentmarker}{}%
\end{pgfscope}%
\begin{pgfscope}%
\pgfsys@transformshift{1.893263in}{1.710110in}%
\pgfsys@useobject{currentmarker}{}%
\end{pgfscope}%
\begin{pgfscope}%
\pgfsys@transformshift{2.308206in}{1.756857in}%
\pgfsys@useobject{currentmarker}{}%
\end{pgfscope}%
\begin{pgfscope}%
\pgfsys@transformshift{2.786986in}{1.794107in}%
\pgfsys@useobject{currentmarker}{}%
\end{pgfscope}%
\begin{pgfscope}%
\pgfsys@transformshift{3.329604in}{1.827547in}%
\pgfsys@useobject{currentmarker}{}%
\end{pgfscope}%
\begin{pgfscope}%
\pgfsys@transformshift{3.936059in}{1.862448in}%
\pgfsys@useobject{currentmarker}{}%
\end{pgfscope}%
\end{pgfscope}%
\begin{pgfscope}%
\pgfsetroundcap%
\pgfsetroundjoin%
\pgfsetlinewidth{1.003750pt}%
\definecolor{currentstroke}{rgb}{0.800000,0.470588,0.737255}%
\pgfsetstrokecolor{currentstroke}%
\pgfsetdash{}{0pt}%
\pgfpathmoveto{\pgfqpoint{0.752170in}{0.635699in}}%
\pgfpathlineto{\pgfqpoint{0.776109in}{0.752110in}}%
\pgfpathlineto{\pgfqpoint{0.816008in}{0.744553in}}%
\pgfpathlineto{\pgfqpoint{0.871865in}{0.812453in}}%
\pgfpathlineto{\pgfqpoint{0.943682in}{0.879829in}}%
\pgfpathlineto{\pgfqpoint{1.031459in}{0.929810in}}%
\pgfpathlineto{\pgfqpoint{1.135194in}{1.609542in}}%
\pgfpathlineto{\pgfqpoint{1.254889in}{1.619175in}}%
\pgfpathlineto{\pgfqpoint{1.390544in}{1.113851in}}%
\pgfpathlineto{\pgfqpoint{1.542158in}{1.255301in}}%
\pgfpathlineto{\pgfqpoint{1.893263in}{1.150978in}}%
\pgfpathlineto{\pgfqpoint{2.308206in}{1.275789in}}%
\pgfpathlineto{\pgfqpoint{2.786986in}{1.277180in}}%
\pgfpathlineto{\pgfqpoint{3.329604in}{1.425561in}}%
\pgfpathlineto{\pgfqpoint{3.936059in}{1.428371in}}%
\pgfusepath{stroke}%
\end{pgfscope}%
\begin{pgfscope}%
\pgfsetbuttcap%
\pgfsetroundjoin%
\definecolor{currentfill}{rgb}{0.800000,0.470588,0.737255}%
\pgfsetfillcolor{currentfill}%
\pgfsetlinewidth{0.752812pt}%
\definecolor{currentstroke}{rgb}{1.000000,1.000000,1.000000}%
\pgfsetstrokecolor{currentstroke}%
\pgfsetdash{}{0pt}%
\pgfsys@defobject{currentmarker}{\pgfqpoint{-0.034722in}{-0.034722in}}{\pgfqpoint{0.034722in}{0.034722in}}{%
\pgfpathmoveto{\pgfqpoint{0.000000in}{-0.034722in}}%
\pgfpathcurveto{\pgfqpoint{0.009208in}{-0.034722in}}{\pgfqpoint{0.018041in}{-0.031064in}}{\pgfqpoint{0.024552in}{-0.024552in}}%
\pgfpathcurveto{\pgfqpoint{0.031064in}{-0.018041in}}{\pgfqpoint{0.034722in}{-0.009208in}}{\pgfqpoint{0.034722in}{0.000000in}}%
\pgfpathcurveto{\pgfqpoint{0.034722in}{0.009208in}}{\pgfqpoint{0.031064in}{0.018041in}}{\pgfqpoint{0.024552in}{0.024552in}}%
\pgfpathcurveto{\pgfqpoint{0.018041in}{0.031064in}}{\pgfqpoint{0.009208in}{0.034722in}}{\pgfqpoint{0.000000in}{0.034722in}}%
\pgfpathcurveto{\pgfqpoint{-0.009208in}{0.034722in}}{\pgfqpoint{-0.018041in}{0.031064in}}{\pgfqpoint{-0.024552in}{0.024552in}}%
\pgfpathcurveto{\pgfqpoint{-0.031064in}{0.018041in}}{\pgfqpoint{-0.034722in}{0.009208in}}{\pgfqpoint{-0.034722in}{0.000000in}}%
\pgfpathcurveto{\pgfqpoint{-0.034722in}{-0.009208in}}{\pgfqpoint{-0.031064in}{-0.018041in}}{\pgfqpoint{-0.024552in}{-0.024552in}}%
\pgfpathcurveto{\pgfqpoint{-0.018041in}{-0.031064in}}{\pgfqpoint{-0.009208in}{-0.034722in}}{\pgfqpoint{0.000000in}{-0.034722in}}%
\pgfpathlineto{\pgfqpoint{0.000000in}{-0.034722in}}%
\pgfpathclose%
\pgfusepath{stroke,fill}%
}%
\begin{pgfscope}%
\pgfsys@transformshift{0.752170in}{0.635699in}%
\pgfsys@useobject{currentmarker}{}%
\end{pgfscope}%
\begin{pgfscope}%
\pgfsys@transformshift{0.776109in}{0.752110in}%
\pgfsys@useobject{currentmarker}{}%
\end{pgfscope}%
\begin{pgfscope}%
\pgfsys@transformshift{0.816008in}{0.744553in}%
\pgfsys@useobject{currentmarker}{}%
\end{pgfscope}%
\begin{pgfscope}%
\pgfsys@transformshift{0.871865in}{0.812453in}%
\pgfsys@useobject{currentmarker}{}%
\end{pgfscope}%
\begin{pgfscope}%
\pgfsys@transformshift{0.943682in}{0.879829in}%
\pgfsys@useobject{currentmarker}{}%
\end{pgfscope}%
\begin{pgfscope}%
\pgfsys@transformshift{1.031459in}{0.929810in}%
\pgfsys@useobject{currentmarker}{}%
\end{pgfscope}%
\begin{pgfscope}%
\pgfsys@transformshift{1.135194in}{1.609542in}%
\pgfsys@useobject{currentmarker}{}%
\end{pgfscope}%
\begin{pgfscope}%
\pgfsys@transformshift{1.254889in}{1.619175in}%
\pgfsys@useobject{currentmarker}{}%
\end{pgfscope}%
\begin{pgfscope}%
\pgfsys@transformshift{1.390544in}{1.113851in}%
\pgfsys@useobject{currentmarker}{}%
\end{pgfscope}%
\begin{pgfscope}%
\pgfsys@transformshift{1.542158in}{1.255301in}%
\pgfsys@useobject{currentmarker}{}%
\end{pgfscope}%
\begin{pgfscope}%
\pgfsys@transformshift{1.893263in}{1.150978in}%
\pgfsys@useobject{currentmarker}{}%
\end{pgfscope}%
\begin{pgfscope}%
\pgfsys@transformshift{2.308206in}{1.275789in}%
\pgfsys@useobject{currentmarker}{}%
\end{pgfscope}%
\begin{pgfscope}%
\pgfsys@transformshift{2.786986in}{1.277180in}%
\pgfsys@useobject{currentmarker}{}%
\end{pgfscope}%
\begin{pgfscope}%
\pgfsys@transformshift{3.329604in}{1.425561in}%
\pgfsys@useobject{currentmarker}{}%
\end{pgfscope}%
\begin{pgfscope}%
\pgfsys@transformshift{3.936059in}{1.428371in}%
\pgfsys@useobject{currentmarker}{}%
\end{pgfscope}%
\end{pgfscope}%
\begin{pgfscope}%
\pgfsetroundcap%
\pgfsetroundjoin%
\pgfsetlinewidth{1.003750pt}%
\definecolor{currentstroke}{rgb}{0.792157,0.568627,0.380392}%
\pgfsetstrokecolor{currentstroke}%
\pgfsetdash{}{0pt}%
\pgfpathmoveto{\pgfqpoint{0.752170in}{0.602671in}}%
\pgfpathlineto{\pgfqpoint{0.776109in}{0.745106in}}%
\pgfpathlineto{\pgfqpoint{0.816008in}{0.747296in}}%
\pgfpathlineto{\pgfqpoint{0.871865in}{0.837414in}}%
\pgfpathlineto{\pgfqpoint{0.943682in}{0.924544in}}%
\pgfpathlineto{\pgfqpoint{1.031459in}{0.980999in}}%
\pgfpathlineto{\pgfqpoint{1.135194in}{1.029238in}}%
\pgfpathlineto{\pgfqpoint{1.254889in}{1.078591in}}%
\pgfpathlineto{\pgfqpoint{1.390544in}{1.156926in}}%
\pgfpathlineto{\pgfqpoint{1.542158in}{1.283793in}}%
\pgfpathlineto{\pgfqpoint{1.893263in}{1.183071in}}%
\pgfpathlineto{\pgfqpoint{2.308206in}{1.352689in}}%
\pgfpathlineto{\pgfqpoint{2.786986in}{1.375437in}}%
\pgfpathlineto{\pgfqpoint{3.329604in}{1.486217in}}%
\pgfpathlineto{\pgfqpoint{3.936059in}{1.490379in}}%
\pgfusepath{stroke}%
\end{pgfscope}%
\begin{pgfscope}%
\pgfsetbuttcap%
\pgfsetroundjoin%
\definecolor{currentfill}{rgb}{0.792157,0.568627,0.380392}%
\pgfsetfillcolor{currentfill}%
\pgfsetlinewidth{0.752812pt}%
\definecolor{currentstroke}{rgb}{1.000000,1.000000,1.000000}%
\pgfsetstrokecolor{currentstroke}%
\pgfsetdash{}{0pt}%
\pgfsys@defobject{currentmarker}{\pgfqpoint{-0.034722in}{-0.034722in}}{\pgfqpoint{0.034722in}{0.034722in}}{%
\pgfpathmoveto{\pgfqpoint{0.000000in}{-0.034722in}}%
\pgfpathcurveto{\pgfqpoint{0.009208in}{-0.034722in}}{\pgfqpoint{0.018041in}{-0.031064in}}{\pgfqpoint{0.024552in}{-0.024552in}}%
\pgfpathcurveto{\pgfqpoint{0.031064in}{-0.018041in}}{\pgfqpoint{0.034722in}{-0.009208in}}{\pgfqpoint{0.034722in}{0.000000in}}%
\pgfpathcurveto{\pgfqpoint{0.034722in}{0.009208in}}{\pgfqpoint{0.031064in}{0.018041in}}{\pgfqpoint{0.024552in}{0.024552in}}%
\pgfpathcurveto{\pgfqpoint{0.018041in}{0.031064in}}{\pgfqpoint{0.009208in}{0.034722in}}{\pgfqpoint{0.000000in}{0.034722in}}%
\pgfpathcurveto{\pgfqpoint{-0.009208in}{0.034722in}}{\pgfqpoint{-0.018041in}{0.031064in}}{\pgfqpoint{-0.024552in}{0.024552in}}%
\pgfpathcurveto{\pgfqpoint{-0.031064in}{0.018041in}}{\pgfqpoint{-0.034722in}{0.009208in}}{\pgfqpoint{-0.034722in}{0.000000in}}%
\pgfpathcurveto{\pgfqpoint{-0.034722in}{-0.009208in}}{\pgfqpoint{-0.031064in}{-0.018041in}}{\pgfqpoint{-0.024552in}{-0.024552in}}%
\pgfpathcurveto{\pgfqpoint{-0.018041in}{-0.031064in}}{\pgfqpoint{-0.009208in}{-0.034722in}}{\pgfqpoint{0.000000in}{-0.034722in}}%
\pgfpathlineto{\pgfqpoint{0.000000in}{-0.034722in}}%
\pgfpathclose%
\pgfusepath{stroke,fill}%
}%
\begin{pgfscope}%
\pgfsys@transformshift{0.752170in}{0.602671in}%
\pgfsys@useobject{currentmarker}{}%
\end{pgfscope}%
\begin{pgfscope}%
\pgfsys@transformshift{0.776109in}{0.745106in}%
\pgfsys@useobject{currentmarker}{}%
\end{pgfscope}%
\begin{pgfscope}%
\pgfsys@transformshift{0.816008in}{0.747296in}%
\pgfsys@useobject{currentmarker}{}%
\end{pgfscope}%
\begin{pgfscope}%
\pgfsys@transformshift{0.871865in}{0.837414in}%
\pgfsys@useobject{currentmarker}{}%
\end{pgfscope}%
\begin{pgfscope}%
\pgfsys@transformshift{0.943682in}{0.924544in}%
\pgfsys@useobject{currentmarker}{}%
\end{pgfscope}%
\begin{pgfscope}%
\pgfsys@transformshift{1.031459in}{0.980999in}%
\pgfsys@useobject{currentmarker}{}%
\end{pgfscope}%
\begin{pgfscope}%
\pgfsys@transformshift{1.135194in}{1.029238in}%
\pgfsys@useobject{currentmarker}{}%
\end{pgfscope}%
\begin{pgfscope}%
\pgfsys@transformshift{1.254889in}{1.078591in}%
\pgfsys@useobject{currentmarker}{}%
\end{pgfscope}%
\begin{pgfscope}%
\pgfsys@transformshift{1.390544in}{1.156926in}%
\pgfsys@useobject{currentmarker}{}%
\end{pgfscope}%
\begin{pgfscope}%
\pgfsys@transformshift{1.542158in}{1.283793in}%
\pgfsys@useobject{currentmarker}{}%
\end{pgfscope}%
\begin{pgfscope}%
\pgfsys@transformshift{1.893263in}{1.183071in}%
\pgfsys@useobject{currentmarker}{}%
\end{pgfscope}%
\begin{pgfscope}%
\pgfsys@transformshift{2.308206in}{1.352689in}%
\pgfsys@useobject{currentmarker}{}%
\end{pgfscope}%
\begin{pgfscope}%
\pgfsys@transformshift{2.786986in}{1.375437in}%
\pgfsys@useobject{currentmarker}{}%
\end{pgfscope}%
\begin{pgfscope}%
\pgfsys@transformshift{3.329604in}{1.486217in}%
\pgfsys@useobject{currentmarker}{}%
\end{pgfscope}%
\begin{pgfscope}%
\pgfsys@transformshift{3.936059in}{1.490379in}%
\pgfsys@useobject{currentmarker}{}%
\end{pgfscope}%
\end{pgfscope}%
\end{pgfpicture}%
\makeatother%
\endgroup%

					\end{figcenter}
					\caption{Import time of the \enquote{OSM city} dataset by tasks.}
				\end{subfigure}
%				\hfill
%				\begin{subfigure}[t]{.48\textwidth}
%					\begin{figcenter}
%						%% Creator: Matplotlib, PGF backend
%%
%% To include the figure in your LaTeX document, write
%%   \input{<filename>.pgf}
%%
%% Make sure the required packages are loaded in your preamble
%%   \usepackage{pgf}
%%
%% Also ensure that all the required font packages are loaded; for instance,
%% the lmodern package is sometimes necessary when using math font.
%%   \usepackage{lmodern}
%%
%% Figures using additional raster images can only be included by \input if
%% they are in the same directory as the main LaTeX file. For loading figures
%% from other directories you can use the `import` package
%%   \usepackage{import}
%%
%% and then include the figures with
%%   \import{<path to file>}{<filename>.pgf}
%%
%% Matplotlib used the following preamble
%%   
%%   \usepackage{fontspec}
%%   \setmainfont{DejaVuSerif.ttf}[Path=\detokenize{/home/hauke/.local/lib/python3.11/site-packages/matplotlib/mpl-data/fonts/ttf/}]
%%   \setsansfont{DroidSans.ttf}[Path=\detokenize{/usr/share/fonts/droid/}]
%%   \setmonofont{DejaVuSansMono.ttf}[Path=\detokenize{/home/hauke/.local/lib/python3.11/site-packages/matplotlib/mpl-data/fonts/ttf/}]
%%   \makeatletter\@ifpackageloaded{underscore}{}{\usepackage[strings]{underscore}}\makeatother
%%
\begingroup%
\makeatletter%
\begin{pgfpicture}%
\pgfpathrectangle{\pgfpointorigin}{\pgfqpoint{5.694946in}{2.407400in}}%
\pgfusepath{use as bounding box, clip}%
\begin{pgfscope}%
\pgfsetbuttcap%
\pgfsetmiterjoin%
\definecolor{currentfill}{rgb}{1.000000,1.000000,1.000000}%
\pgfsetfillcolor{currentfill}%
\pgfsetlinewidth{0.000000pt}%
\definecolor{currentstroke}{rgb}{1.000000,1.000000,1.000000}%
\pgfsetstrokecolor{currentstroke}%
\pgfsetdash{}{0pt}%
\pgfpathmoveto{\pgfqpoint{0.000000in}{0.000000in}}%
\pgfpathlineto{\pgfqpoint{5.694946in}{0.000000in}}%
\pgfpathlineto{\pgfqpoint{5.694946in}{2.407400in}}%
\pgfpathlineto{\pgfqpoint{0.000000in}{2.407400in}}%
\pgfpathlineto{\pgfqpoint{0.000000in}{0.000000in}}%
\pgfpathclose%
\pgfusepath{fill}%
\end{pgfscope}%
\begin{pgfscope}%
\pgfsetbuttcap%
\pgfsetmiterjoin%
\definecolor{currentfill}{rgb}{1.000000,1.000000,1.000000}%
\pgfsetfillcolor{currentfill}%
\pgfsetlinewidth{0.000000pt}%
\definecolor{currentstroke}{rgb}{0.000000,0.000000,0.000000}%
\pgfsetstrokecolor{currentstroke}%
\pgfsetstrokeopacity{0.000000}%
\pgfsetdash{}{0pt}%
\pgfpathmoveto{\pgfqpoint{0.592976in}{0.451389in}}%
\pgfpathlineto{\pgfqpoint{4.295331in}{0.451389in}}%
\pgfpathlineto{\pgfqpoint{4.295331in}{2.407400in}}%
\pgfpathlineto{\pgfqpoint{0.592976in}{2.407400in}}%
\pgfpathlineto{\pgfqpoint{0.592976in}{0.451389in}}%
\pgfpathclose%
\pgfusepath{fill}%
\end{pgfscope}%
\begin{pgfscope}%
\pgfpathrectangle{\pgfqpoint{0.592976in}{0.451389in}}{\pgfqpoint{3.702355in}{1.956011in}}%
\pgfusepath{clip}%
\pgfsetroundcap%
\pgfsetroundjoin%
\pgfsetlinewidth{1.003750pt}%
\definecolor{currentstroke}{rgb}{0.800000,0.800000,0.800000}%
\pgfsetstrokecolor{currentstroke}%
\pgfsetdash{}{0pt}%
\pgfpathmoveto{\pgfqpoint{1.179728in}{0.451389in}}%
\pgfpathlineto{\pgfqpoint{1.179728in}{2.407400in}}%
\pgfusepath{stroke}%
\end{pgfscope}%
\begin{pgfscope}%
\definecolor{textcolor}{rgb}{0.150000,0.150000,0.150000}%
\pgfsetstrokecolor{textcolor}%
\pgfsetfillcolor{textcolor}%
\pgftext[x=1.179728in,y=0.319444in,,top]{\color{textcolor}\sffamily\fontsize{9.000000}{10.800000}\selectfont 2000}%
\end{pgfscope}%
\begin{pgfscope}%
\pgfpathrectangle{\pgfqpoint{0.592976in}{0.451389in}}{\pgfqpoint{3.702355in}{1.956011in}}%
\pgfusepath{clip}%
\pgfsetroundcap%
\pgfsetroundjoin%
\pgfsetlinewidth{1.003750pt}%
\definecolor{currentstroke}{rgb}{0.800000,0.800000,0.800000}%
\pgfsetstrokecolor{currentstroke}%
\pgfsetdash{}{0pt}%
\pgfpathmoveto{\pgfqpoint{1.802905in}{0.451389in}}%
\pgfpathlineto{\pgfqpoint{1.802905in}{2.407400in}}%
\pgfusepath{stroke}%
\end{pgfscope}%
\begin{pgfscope}%
\definecolor{textcolor}{rgb}{0.150000,0.150000,0.150000}%
\pgfsetstrokecolor{textcolor}%
\pgfsetfillcolor{textcolor}%
\pgftext[x=1.802905in,y=0.319444in,,top]{\color{textcolor}\sffamily\fontsize{9.000000}{10.800000}\selectfont 4000}%
\end{pgfscope}%
\begin{pgfscope}%
\pgfpathrectangle{\pgfqpoint{0.592976in}{0.451389in}}{\pgfqpoint{3.702355in}{1.956011in}}%
\pgfusepath{clip}%
\pgfsetroundcap%
\pgfsetroundjoin%
\pgfsetlinewidth{1.003750pt}%
\definecolor{currentstroke}{rgb}{0.800000,0.800000,0.800000}%
\pgfsetstrokecolor{currentstroke}%
\pgfsetdash{}{0pt}%
\pgfpathmoveto{\pgfqpoint{2.426081in}{0.451389in}}%
\pgfpathlineto{\pgfqpoint{2.426081in}{2.407400in}}%
\pgfusepath{stroke}%
\end{pgfscope}%
\begin{pgfscope}%
\definecolor{textcolor}{rgb}{0.150000,0.150000,0.150000}%
\pgfsetstrokecolor{textcolor}%
\pgfsetfillcolor{textcolor}%
\pgftext[x=2.426081in,y=0.319444in,,top]{\color{textcolor}\sffamily\fontsize{9.000000}{10.800000}\selectfont 6000}%
\end{pgfscope}%
\begin{pgfscope}%
\pgfpathrectangle{\pgfqpoint{0.592976in}{0.451389in}}{\pgfqpoint{3.702355in}{1.956011in}}%
\pgfusepath{clip}%
\pgfsetroundcap%
\pgfsetroundjoin%
\pgfsetlinewidth{1.003750pt}%
\definecolor{currentstroke}{rgb}{0.800000,0.800000,0.800000}%
\pgfsetstrokecolor{currentstroke}%
\pgfsetdash{}{0pt}%
\pgfpathmoveto{\pgfqpoint{3.049258in}{0.451389in}}%
\pgfpathlineto{\pgfqpoint{3.049258in}{2.407400in}}%
\pgfusepath{stroke}%
\end{pgfscope}%
\begin{pgfscope}%
\definecolor{textcolor}{rgb}{0.150000,0.150000,0.150000}%
\pgfsetstrokecolor{textcolor}%
\pgfsetfillcolor{textcolor}%
\pgftext[x=3.049258in,y=0.319444in,,top]{\color{textcolor}\sffamily\fontsize{9.000000}{10.800000}\selectfont 8000}%
\end{pgfscope}%
\begin{pgfscope}%
\pgfpathrectangle{\pgfqpoint{0.592976in}{0.451389in}}{\pgfqpoint{3.702355in}{1.956011in}}%
\pgfusepath{clip}%
\pgfsetroundcap%
\pgfsetroundjoin%
\pgfsetlinewidth{1.003750pt}%
\definecolor{currentstroke}{rgb}{0.800000,0.800000,0.800000}%
\pgfsetstrokecolor{currentstroke}%
\pgfsetdash{}{0pt}%
\pgfpathmoveto{\pgfqpoint{3.672435in}{0.451389in}}%
\pgfpathlineto{\pgfqpoint{3.672435in}{2.407400in}}%
\pgfusepath{stroke}%
\end{pgfscope}%
\begin{pgfscope}%
\definecolor{textcolor}{rgb}{0.150000,0.150000,0.150000}%
\pgfsetstrokecolor{textcolor}%
\pgfsetfillcolor{textcolor}%
\pgftext[x=3.672435in,y=0.319444in,,top]{\color{textcolor}\sffamily\fontsize{9.000000}{10.800000}\selectfont 10000}%
\end{pgfscope}%
\begin{pgfscope}%
\definecolor{textcolor}{rgb}{0.150000,0.150000,0.150000}%
\pgfsetstrokecolor{textcolor}%
\pgfsetfillcolor{textcolor}%
\pgftext[x=2.444154in,y=0.125000in,,top]{\color{textcolor}\sffamily\fontsize{9.000000}{10.800000}\selectfont Input obstacle vertices}%
\end{pgfscope}%
\begin{pgfscope}%
\pgfpathrectangle{\pgfqpoint{0.592976in}{0.451389in}}{\pgfqpoint{3.702355in}{1.956011in}}%
\pgfusepath{clip}%
\pgfsetroundcap%
\pgfsetroundjoin%
\pgfsetlinewidth{1.003750pt}%
\definecolor{currentstroke}{rgb}{0.800000,0.800000,0.800000}%
\pgfsetstrokecolor{currentstroke}%
\pgfsetdash{}{0pt}%
\pgfpathmoveto{\pgfqpoint{0.592976in}{0.524535in}}%
\pgfpathlineto{\pgfqpoint{4.295331in}{0.524535in}}%
\pgfusepath{stroke}%
\end{pgfscope}%
\begin{pgfscope}%
\definecolor{textcolor}{rgb}{0.150000,0.150000,0.150000}%
\pgfsetstrokecolor{textcolor}%
\pgfsetfillcolor{textcolor}%
\pgftext[x=0.194444in, y=0.477050in, left, base]{\color{textcolor}\sffamily\fontsize{9.000000}{10.800000}\selectfont \(\displaystyle {10^{-7}}\)}%
\end{pgfscope}%
\begin{pgfscope}%
\pgfpathrectangle{\pgfqpoint{0.592976in}{0.451389in}}{\pgfqpoint{3.702355in}{1.956011in}}%
\pgfusepath{clip}%
\pgfsetroundcap%
\pgfsetroundjoin%
\pgfsetlinewidth{1.003750pt}%
\definecolor{currentstroke}{rgb}{0.800000,0.800000,0.800000}%
\pgfsetstrokecolor{currentstroke}%
\pgfsetdash{}{0pt}%
\pgfpathmoveto{\pgfqpoint{0.592976in}{0.973024in}}%
\pgfpathlineto{\pgfqpoint{4.295331in}{0.973024in}}%
\pgfusepath{stroke}%
\end{pgfscope}%
\begin{pgfscope}%
\definecolor{textcolor}{rgb}{0.150000,0.150000,0.150000}%
\pgfsetstrokecolor{textcolor}%
\pgfsetfillcolor{textcolor}%
\pgftext[x=0.194444in, y=0.925538in, left, base]{\color{textcolor}\sffamily\fontsize{9.000000}{10.800000}\selectfont \(\displaystyle {10^{-6}}\)}%
\end{pgfscope}%
\begin{pgfscope}%
\pgfpathrectangle{\pgfqpoint{0.592976in}{0.451389in}}{\pgfqpoint{3.702355in}{1.956011in}}%
\pgfusepath{clip}%
\pgfsetroundcap%
\pgfsetroundjoin%
\pgfsetlinewidth{1.003750pt}%
\definecolor{currentstroke}{rgb}{0.800000,0.800000,0.800000}%
\pgfsetstrokecolor{currentstroke}%
\pgfsetdash{}{0pt}%
\pgfpathmoveto{\pgfqpoint{0.592976in}{1.421513in}}%
\pgfpathlineto{\pgfqpoint{4.295331in}{1.421513in}}%
\pgfusepath{stroke}%
\end{pgfscope}%
\begin{pgfscope}%
\definecolor{textcolor}{rgb}{0.150000,0.150000,0.150000}%
\pgfsetstrokecolor{textcolor}%
\pgfsetfillcolor{textcolor}%
\pgftext[x=0.194444in, y=1.374027in, left, base]{\color{textcolor}\sffamily\fontsize{9.000000}{10.800000}\selectfont \(\displaystyle {10^{-5}}\)}%
\end{pgfscope}%
\begin{pgfscope}%
\pgfpathrectangle{\pgfqpoint{0.592976in}{0.451389in}}{\pgfqpoint{3.702355in}{1.956011in}}%
\pgfusepath{clip}%
\pgfsetroundcap%
\pgfsetroundjoin%
\pgfsetlinewidth{1.003750pt}%
\definecolor{currentstroke}{rgb}{0.800000,0.800000,0.800000}%
\pgfsetstrokecolor{currentstroke}%
\pgfsetdash{}{0pt}%
\pgfpathmoveto{\pgfqpoint{0.592976in}{1.870001in}}%
\pgfpathlineto{\pgfqpoint{4.295331in}{1.870001in}}%
\pgfusepath{stroke}%
\end{pgfscope}%
\begin{pgfscope}%
\definecolor{textcolor}{rgb}{0.150000,0.150000,0.150000}%
\pgfsetstrokecolor{textcolor}%
\pgfsetfillcolor{textcolor}%
\pgftext[x=0.194444in, y=1.822516in, left, base]{\color{textcolor}\sffamily\fontsize{9.000000}{10.800000}\selectfont \(\displaystyle {10^{-4}}\)}%
\end{pgfscope}%
\begin{pgfscope}%
\pgfpathrectangle{\pgfqpoint{0.592976in}{0.451389in}}{\pgfqpoint{3.702355in}{1.956011in}}%
\pgfusepath{clip}%
\pgfsetroundcap%
\pgfsetroundjoin%
\pgfsetlinewidth{1.003750pt}%
\definecolor{currentstroke}{rgb}{0.800000,0.800000,0.800000}%
\pgfsetstrokecolor{currentstroke}%
\pgfsetdash{}{0pt}%
\pgfpathmoveto{\pgfqpoint{0.592976in}{2.318490in}}%
\pgfpathlineto{\pgfqpoint{4.295331in}{2.318490in}}%
\pgfusepath{stroke}%
\end{pgfscope}%
\begin{pgfscope}%
\definecolor{textcolor}{rgb}{0.150000,0.150000,0.150000}%
\pgfsetstrokecolor{textcolor}%
\pgfsetfillcolor{textcolor}%
\pgftext[x=0.194444in, y=2.271005in, left, base]{\color{textcolor}\sffamily\fontsize{9.000000}{10.800000}\selectfont \(\displaystyle {10^{-3}}\)}%
\end{pgfscope}%
\begin{pgfscope}%
\definecolor{textcolor}{rgb}{0.150000,0.150000,0.150000}%
\pgfsetstrokecolor{textcolor}%
\pgfsetfillcolor{textcolor}%
\pgftext[x=0.125000in,y=1.429394in,,bottom,rotate=90.000000]{\color{textcolor}\sffamily\fontsize{9.000000}{10.800000}\selectfont Share of total time}%
\end{pgfscope}%
\begin{pgfscope}%
\pgfpathrectangle{\pgfqpoint{0.592976in}{0.451389in}}{\pgfqpoint{3.702355in}{1.956011in}}%
\pgfusepath{clip}%
\pgfsetbuttcap%
\pgfsetroundjoin%
\definecolor{currentfill}{rgb}{0.003922,0.450980,0.698039}%
\pgfsetfillcolor{currentfill}%
\pgfsetfillopacity{0.200000}%
\pgfsetlinewidth{1.003750pt}%
\definecolor{currentstroke}{rgb}{0.003922,0.450980,0.698039}%
\pgfsetstrokecolor{currentstroke}%
\pgfsetstrokeopacity{0.200000}%
\pgfsetdash{}{0pt}%
\pgfsys@defobject{currentmarker}{\pgfqpoint{0.761265in}{2.318490in}}{\pgfqpoint{4.127042in}{2.318490in}}{%
\pgfpathmoveto{\pgfqpoint{0.761265in}{2.318490in}}%
\pgfpathlineto{\pgfqpoint{0.761265in}{2.318490in}}%
\pgfpathlineto{\pgfqpoint{1.091548in}{2.318490in}}%
\pgfpathlineto{\pgfqpoint{1.466389in}{2.318490in}}%
\pgfpathlineto{\pgfqpoint{1.890773in}{2.318490in}}%
\pgfpathlineto{\pgfqpoint{3.157691in}{2.318490in}}%
\pgfpathlineto{\pgfqpoint{4.127042in}{2.318490in}}%
\pgfpathlineto{\pgfqpoint{4.127042in}{2.318490in}}%
\pgfpathlineto{\pgfqpoint{4.127042in}{2.318490in}}%
\pgfpathlineto{\pgfqpoint{3.157691in}{2.318490in}}%
\pgfpathlineto{\pgfqpoint{1.890773in}{2.318490in}}%
\pgfpathlineto{\pgfqpoint{1.466389in}{2.318490in}}%
\pgfpathlineto{\pgfqpoint{1.091548in}{2.318490in}}%
\pgfpathlineto{\pgfqpoint{0.761265in}{2.318490in}}%
\pgfpathlineto{\pgfqpoint{0.761265in}{2.318490in}}%
\pgfpathclose%
\pgfusepath{stroke,fill}%
}%
\begin{pgfscope}%
\pgfsys@transformshift{0.000000in}{0.000000in}%
\pgfsys@useobject{currentmarker}{}%
\end{pgfscope}%
\end{pgfscope}%
\begin{pgfscope}%
\pgfpathrectangle{\pgfqpoint{0.592976in}{0.451389in}}{\pgfqpoint{3.702355in}{1.956011in}}%
\pgfusepath{clip}%
\pgfsetbuttcap%
\pgfsetroundjoin%
\definecolor{currentfill}{rgb}{0.870588,0.560784,0.019608}%
\pgfsetfillcolor{currentfill}%
\pgfsetfillopacity{0.200000}%
\pgfsetlinewidth{1.003750pt}%
\definecolor{currentstroke}{rgb}{0.870588,0.560784,0.019608}%
\pgfsetstrokecolor{currentstroke}%
\pgfsetstrokeopacity{0.200000}%
\pgfsetdash{}{0pt}%
\pgfsys@defobject{currentmarker}{\pgfqpoint{0.761265in}{2.179422in}}{\pgfqpoint{4.127042in}{2.292625in}}{%
\pgfpathmoveto{\pgfqpoint{0.761265in}{2.185196in}}%
\pgfpathlineto{\pgfqpoint{0.761265in}{2.179422in}}%
\pgfpathlineto{\pgfqpoint{1.091548in}{2.204191in}}%
\pgfpathlineto{\pgfqpoint{1.466389in}{2.256371in}}%
\pgfpathlineto{\pgfqpoint{1.890773in}{2.271872in}}%
\pgfpathlineto{\pgfqpoint{3.157691in}{2.284033in}}%
\pgfpathlineto{\pgfqpoint{4.127042in}{2.290923in}}%
\pgfpathlineto{\pgfqpoint{4.127042in}{2.292625in}}%
\pgfpathlineto{\pgfqpoint{4.127042in}{2.292625in}}%
\pgfpathlineto{\pgfqpoint{3.157691in}{2.286622in}}%
\pgfpathlineto{\pgfqpoint{1.890773in}{2.276390in}}%
\pgfpathlineto{\pgfqpoint{1.466389in}{2.260445in}}%
\pgfpathlineto{\pgfqpoint{1.091548in}{2.214138in}}%
\pgfpathlineto{\pgfqpoint{0.761265in}{2.185196in}}%
\pgfpathlineto{\pgfqpoint{0.761265in}{2.185196in}}%
\pgfpathclose%
\pgfusepath{stroke,fill}%
}%
\begin{pgfscope}%
\pgfsys@transformshift{0.000000in}{0.000000in}%
\pgfsys@useobject{currentmarker}{}%
\end{pgfscope}%
\end{pgfscope}%
\begin{pgfscope}%
\pgfpathrectangle{\pgfqpoint{0.592976in}{0.451389in}}{\pgfqpoint{3.702355in}{1.956011in}}%
\pgfusepath{clip}%
\pgfsetbuttcap%
\pgfsetroundjoin%
\definecolor{currentfill}{rgb}{0.007843,0.619608,0.450980}%
\pgfsetfillcolor{currentfill}%
\pgfsetfillopacity{0.200000}%
\pgfsetlinewidth{1.003750pt}%
\definecolor{currentstroke}{rgb}{0.007843,0.619608,0.450980}%
\pgfsetstrokecolor{currentstroke}%
\pgfsetstrokeopacity{0.200000}%
\pgfsetdash{}{0pt}%
\pgfsys@defobject{currentmarker}{\pgfqpoint{0.761265in}{1.465869in}}{\pgfqpoint{4.127042in}{1.651466in}}{%
\pgfpathmoveto{\pgfqpoint{0.761265in}{1.651466in}}%
\pgfpathlineto{\pgfqpoint{0.761265in}{1.650385in}}%
\pgfpathlineto{\pgfqpoint{1.091548in}{1.535931in}}%
\pgfpathlineto{\pgfqpoint{1.466389in}{1.579708in}}%
\pgfpathlineto{\pgfqpoint{1.890773in}{1.573612in}}%
\pgfpathlineto{\pgfqpoint{3.157691in}{1.510137in}}%
\pgfpathlineto{\pgfqpoint{4.127042in}{1.465869in}}%
\pgfpathlineto{\pgfqpoint{4.127042in}{1.476596in}}%
\pgfpathlineto{\pgfqpoint{4.127042in}{1.476596in}}%
\pgfpathlineto{\pgfqpoint{3.157691in}{1.516067in}}%
\pgfpathlineto{\pgfqpoint{1.890773in}{1.584935in}}%
\pgfpathlineto{\pgfqpoint{1.466389in}{1.587851in}}%
\pgfpathlineto{\pgfqpoint{1.091548in}{1.556795in}}%
\pgfpathlineto{\pgfqpoint{0.761265in}{1.651466in}}%
\pgfpathlineto{\pgfqpoint{0.761265in}{1.651466in}}%
\pgfpathclose%
\pgfusepath{stroke,fill}%
}%
\begin{pgfscope}%
\pgfsys@transformshift{0.000000in}{0.000000in}%
\pgfsys@useobject{currentmarker}{}%
\end{pgfscope}%
\end{pgfscope}%
\begin{pgfscope}%
\pgfpathrectangle{\pgfqpoint{0.592976in}{0.451389in}}{\pgfqpoint{3.702355in}{1.956011in}}%
\pgfusepath{clip}%
\pgfsetbuttcap%
\pgfsetroundjoin%
\definecolor{currentfill}{rgb}{0.835294,0.368627,0.000000}%
\pgfsetfillcolor{currentfill}%
\pgfsetfillopacity{0.200000}%
\pgfsetlinewidth{1.003750pt}%
\definecolor{currentstroke}{rgb}{0.835294,0.368627,0.000000}%
\pgfsetstrokecolor{currentstroke}%
\pgfsetstrokeopacity{0.200000}%
\pgfsetdash{}{0pt}%
\pgfsys@defobject{currentmarker}{\pgfqpoint{0.761265in}{1.191897in}}{\pgfqpoint{4.127042in}{1.510784in}}{%
\pgfpathmoveto{\pgfqpoint{0.761265in}{1.510784in}}%
\pgfpathlineto{\pgfqpoint{0.761265in}{1.499849in}}%
\pgfpathlineto{\pgfqpoint{1.091548in}{1.344084in}}%
\pgfpathlineto{\pgfqpoint{1.466389in}{1.316344in}}%
\pgfpathlineto{\pgfqpoint{1.890773in}{1.277472in}}%
\pgfpathlineto{\pgfqpoint{3.157691in}{1.221987in}}%
\pgfpathlineto{\pgfqpoint{4.127042in}{1.191897in}}%
\pgfpathlineto{\pgfqpoint{4.127042in}{1.226923in}}%
\pgfpathlineto{\pgfqpoint{4.127042in}{1.226923in}}%
\pgfpathlineto{\pgfqpoint{3.157691in}{1.250997in}}%
\pgfpathlineto{\pgfqpoint{1.890773in}{1.304030in}}%
\pgfpathlineto{\pgfqpoint{1.466389in}{1.332316in}}%
\pgfpathlineto{\pgfqpoint{1.091548in}{1.467128in}}%
\pgfpathlineto{\pgfqpoint{0.761265in}{1.510784in}}%
\pgfpathlineto{\pgfqpoint{0.761265in}{1.510784in}}%
\pgfpathclose%
\pgfusepath{stroke,fill}%
}%
\begin{pgfscope}%
\pgfsys@transformshift{0.000000in}{0.000000in}%
\pgfsys@useobject{currentmarker}{}%
\end{pgfscope}%
\end{pgfscope}%
\begin{pgfscope}%
\pgfpathrectangle{\pgfqpoint{0.592976in}{0.451389in}}{\pgfqpoint{3.702355in}{1.956011in}}%
\pgfusepath{clip}%
\pgfsetbuttcap%
\pgfsetroundjoin%
\definecolor{currentfill}{rgb}{0.800000,0.470588,0.737255}%
\pgfsetfillcolor{currentfill}%
\pgfsetfillopacity{0.200000}%
\pgfsetlinewidth{1.003750pt}%
\definecolor{currentstroke}{rgb}{0.800000,0.470588,0.737255}%
\pgfsetstrokecolor{currentstroke}%
\pgfsetstrokeopacity{0.200000}%
\pgfsetdash{}{0pt}%
\pgfsys@defobject{currentmarker}{\pgfqpoint{0.761265in}{1.877366in}}{\pgfqpoint{4.127042in}{2.155375in}}{%
\pgfpathmoveto{\pgfqpoint{0.761265in}{2.155375in}}%
\pgfpathlineto{\pgfqpoint{0.761265in}{2.146762in}}%
\pgfpathlineto{\pgfqpoint{1.091548in}{2.126007in}}%
\pgfpathlineto{\pgfqpoint{1.466389in}{2.022284in}}%
\pgfpathlineto{\pgfqpoint{1.890773in}{1.960169in}}%
\pgfpathlineto{\pgfqpoint{3.157691in}{1.914779in}}%
\pgfpathlineto{\pgfqpoint{4.127042in}{1.877366in}}%
\pgfpathlineto{\pgfqpoint{4.127042in}{1.891647in}}%
\pgfpathlineto{\pgfqpoint{4.127042in}{1.891647in}}%
\pgfpathlineto{\pgfqpoint{3.157691in}{1.931396in}}%
\pgfpathlineto{\pgfqpoint{1.890773in}{1.981764in}}%
\pgfpathlineto{\pgfqpoint{1.466389in}{2.036978in}}%
\pgfpathlineto{\pgfqpoint{1.091548in}{2.140481in}}%
\pgfpathlineto{\pgfqpoint{0.761265in}{2.155375in}}%
\pgfpathlineto{\pgfqpoint{0.761265in}{2.155375in}}%
\pgfpathclose%
\pgfusepath{stroke,fill}%
}%
\begin{pgfscope}%
\pgfsys@transformshift{0.000000in}{0.000000in}%
\pgfsys@useobject{currentmarker}{}%
\end{pgfscope}%
\end{pgfscope}%
\begin{pgfscope}%
\pgfpathrectangle{\pgfqpoint{0.592976in}{0.451389in}}{\pgfqpoint{3.702355in}{1.956011in}}%
\pgfusepath{clip}%
\pgfsetbuttcap%
\pgfsetroundjoin%
\definecolor{currentfill}{rgb}{0.792157,0.568627,0.380392}%
\pgfsetfillcolor{currentfill}%
\pgfsetfillopacity{0.200000}%
\pgfsetlinewidth{1.003750pt}%
\definecolor{currentstroke}{rgb}{0.792157,0.568627,0.380392}%
\pgfsetstrokecolor{currentstroke}%
\pgfsetstrokeopacity{0.200000}%
\pgfsetdash{}{0pt}%
\pgfsys@defobject{currentmarker}{\pgfqpoint{0.761265in}{0.540298in}}{\pgfqpoint{4.127042in}{0.963196in}}{%
\pgfpathmoveto{\pgfqpoint{0.761265in}{0.917466in}}%
\pgfpathlineto{\pgfqpoint{0.761265in}{0.898816in}}%
\pgfpathlineto{\pgfqpoint{1.091548in}{0.751731in}}%
\pgfpathlineto{\pgfqpoint{1.466389in}{0.709081in}}%
\pgfpathlineto{\pgfqpoint{1.890773in}{0.681996in}}%
\pgfpathlineto{\pgfqpoint{3.157691in}{0.595232in}}%
\pgfpathlineto{\pgfqpoint{4.127042in}{0.540298in}}%
\pgfpathlineto{\pgfqpoint{4.127042in}{0.680150in}}%
\pgfpathlineto{\pgfqpoint{4.127042in}{0.680150in}}%
\pgfpathlineto{\pgfqpoint{3.157691in}{0.689680in}}%
\pgfpathlineto{\pgfqpoint{1.890773in}{0.963196in}}%
\pgfpathlineto{\pgfqpoint{1.466389in}{0.729335in}}%
\pgfpathlineto{\pgfqpoint{1.091548in}{0.833740in}}%
\pgfpathlineto{\pgfqpoint{0.761265in}{0.917466in}}%
\pgfpathlineto{\pgfqpoint{0.761265in}{0.917466in}}%
\pgfpathclose%
\pgfusepath{stroke,fill}%
}%
\begin{pgfscope}%
\pgfsys@transformshift{0.000000in}{0.000000in}%
\pgfsys@useobject{currentmarker}{}%
\end{pgfscope}%
\end{pgfscope}%
\begin{pgfscope}%
\pgfsetrectcap%
\pgfsetmiterjoin%
\pgfsetlinewidth{1.254687pt}%
\definecolor{currentstroke}{rgb}{0.800000,0.800000,0.800000}%
\pgfsetstrokecolor{currentstroke}%
\pgfsetdash{}{0pt}%
\pgfpathmoveto{\pgfqpoint{0.592976in}{0.451389in}}%
\pgfpathlineto{\pgfqpoint{0.592976in}{2.407400in}}%
\pgfusepath{stroke}%
\end{pgfscope}%
\begin{pgfscope}%
\pgfsetrectcap%
\pgfsetmiterjoin%
\pgfsetlinewidth{1.254687pt}%
\definecolor{currentstroke}{rgb}{0.800000,0.800000,0.800000}%
\pgfsetstrokecolor{currentstroke}%
\pgfsetdash{}{0pt}%
\pgfpathmoveto{\pgfqpoint{4.295331in}{0.451389in}}%
\pgfpathlineto{\pgfqpoint{4.295331in}{2.407400in}}%
\pgfusepath{stroke}%
\end{pgfscope}%
\begin{pgfscope}%
\pgfsetrectcap%
\pgfsetmiterjoin%
\pgfsetlinewidth{1.254687pt}%
\definecolor{currentstroke}{rgb}{0.800000,0.800000,0.800000}%
\pgfsetstrokecolor{currentstroke}%
\pgfsetdash{}{0pt}%
\pgfpathmoveto{\pgfqpoint{0.592976in}{0.451389in}}%
\pgfpathlineto{\pgfqpoint{4.295331in}{0.451389in}}%
\pgfusepath{stroke}%
\end{pgfscope}%
\begin{pgfscope}%
\pgfsetrectcap%
\pgfsetmiterjoin%
\pgfsetlinewidth{1.254687pt}%
\definecolor{currentstroke}{rgb}{0.800000,0.800000,0.800000}%
\pgfsetstrokecolor{currentstroke}%
\pgfsetdash{}{0pt}%
\pgfpathmoveto{\pgfqpoint{0.592976in}{2.407400in}}%
\pgfpathlineto{\pgfqpoint{4.295331in}{2.407400in}}%
\pgfusepath{stroke}%
\end{pgfscope}%
\begin{pgfscope}%
\pgfsetbuttcap%
\pgfsetmiterjoin%
\definecolor{currentfill}{rgb}{1.000000,1.000000,1.000000}%
\pgfsetfillcolor{currentfill}%
\pgfsetfillopacity{0.800000}%
\pgfsetlinewidth{1.003750pt}%
\definecolor{currentstroke}{rgb}{0.800000,0.800000,0.800000}%
\pgfsetstrokecolor{currentstroke}%
\pgfsetstrokeopacity{0.800000}%
\pgfsetdash{}{0pt}%
\pgfpathmoveto{\pgfqpoint{4.475390in}{0.538404in}}%
\pgfpathlineto{\pgfqpoint{5.669946in}{0.538404in}}%
\pgfpathquadraticcurveto{\pgfqpoint{5.694946in}{0.538404in}}{\pgfqpoint{5.694946in}{0.563404in}}%
\pgfpathlineto{\pgfqpoint{5.694946in}{2.295385in}}%
\pgfpathquadraticcurveto{\pgfqpoint{5.694946in}{2.320385in}}{\pgfqpoint{5.669946in}{2.320385in}}%
\pgfpathlineto{\pgfqpoint{4.475390in}{2.320385in}}%
\pgfpathquadraticcurveto{\pgfqpoint{4.450390in}{2.320385in}}{\pgfqpoint{4.450390in}{2.295385in}}%
\pgfpathlineto{\pgfqpoint{4.450390in}{0.563404in}}%
\pgfpathquadraticcurveto{\pgfqpoint{4.450390in}{0.538404in}}{\pgfqpoint{4.475390in}{0.538404in}}%
\pgfpathlineto{\pgfqpoint{4.475390in}{0.538404in}}%
\pgfpathclose%
\pgfusepath{stroke,fill}%
\end{pgfscope}%
\begin{pgfscope}%
\definecolor{textcolor}{rgb}{0.150000,0.150000,0.150000}%
\pgfsetstrokecolor{textcolor}%
\pgfsetfillcolor{textcolor}%
\pgftext[x=4.869268in,y=2.175414in,left,base]{\color{textcolor}\sffamily\fontsize{9.000000}{10.800000}\selectfont Legend}%
\end{pgfscope}%
\begin{pgfscope}%
\pgfsetroundcap%
\pgfsetroundjoin%
\pgfsetlinewidth{1.505625pt}%
\definecolor{currentstroke}{rgb}{0.003922,0.450980,0.698039}%
\pgfsetstrokecolor{currentstroke}%
\pgfsetdash{}{0pt}%
\pgfpathmoveto{\pgfqpoint{4.500390in}{2.031664in}}%
\pgfpathlineto{\pgfqpoint{4.625390in}{2.031664in}}%
\pgfpathlineto{\pgfqpoint{4.750390in}{2.031664in}}%
\pgfusepath{stroke}%
\end{pgfscope}%
\begin{pgfscope}%
\definecolor{textcolor}{rgb}{0.150000,0.150000,0.150000}%
\pgfsetstrokecolor{textcolor}%
\pgfsetfillcolor{textcolor}%
\pgftext[x=4.850390in,y=1.987914in,left,base]{\color{textcolor}\sffamily\fontsize{9.000000}{10.800000}\selectfont Total time}%
\end{pgfscope}%
\begin{pgfscope}%
\pgfsetroundcap%
\pgfsetroundjoin%
\pgfsetlinewidth{1.505625pt}%
\definecolor{currentstroke}{rgb}{0.870588,0.560784,0.019608}%
\pgfsetstrokecolor{currentstroke}%
\pgfsetdash{}{0pt}%
\pgfpathmoveto{\pgfqpoint{4.500390in}{1.844164in}}%
\pgfpathlineto{\pgfqpoint{4.625390in}{1.844164in}}%
\pgfpathlineto{\pgfqpoint{4.750390in}{1.844164in}}%
\pgfusepath{stroke}%
\end{pgfscope}%
\begin{pgfscope}%
\definecolor{textcolor}{rgb}{0.150000,0.150000,0.150000}%
\pgfsetstrokecolor{textcolor}%
\pgfsetfillcolor{textcolor}%
\pgftext[x=4.850390in,y=1.800414in,left,base]{\color{textcolor}\sffamily\fontsize{9.000000}{10.800000}\selectfont kNN search}%
\end{pgfscope}%
\begin{pgfscope}%
\pgfsetroundcap%
\pgfsetroundjoin%
\pgfsetlinewidth{1.505625pt}%
\definecolor{currentstroke}{rgb}{0.007843,0.619608,0.450980}%
\pgfsetstrokecolor{currentstroke}%
\pgfsetdash{}{0pt}%
\pgfpathmoveto{\pgfqpoint{4.500390in}{1.656664in}}%
\pgfpathlineto{\pgfqpoint{4.625390in}{1.656664in}}%
\pgfpathlineto{\pgfqpoint{4.750390in}{1.656664in}}%
\pgfusepath{stroke}%
\end{pgfscope}%
\begin{pgfscope}%
\definecolor{textcolor}{rgb}{0.150000,0.150000,0.150000}%
\pgfsetstrokecolor{textcolor}%
\pgfsetfillcolor{textcolor}%
\pgftext[x=4.850390in,y=1.612914in,left,base]{\color{textcolor}\sffamily\fontsize{9.000000}{10.800000}\selectfont Create graph}%
\end{pgfscope}%
\begin{pgfscope}%
\pgfsetroundcap%
\pgfsetroundjoin%
\pgfsetlinewidth{1.505625pt}%
\definecolor{currentstroke}{rgb}{0.835294,0.368627,0.000000}%
\pgfsetstrokecolor{currentstroke}%
\pgfsetdash{}{0pt}%
\pgfpathmoveto{\pgfqpoint{4.500390in}{1.382153in}}%
\pgfpathlineto{\pgfqpoint{4.625390in}{1.382153in}}%
\pgfpathlineto{\pgfqpoint{4.750390in}{1.382153in}}%
\pgfusepath{stroke}%
\end{pgfscope}%
\begin{pgfscope}%
\definecolor{textcolor}{rgb}{0.150000,0.150000,0.150000}%
\pgfsetstrokecolor{textcolor}%
\pgfsetfillcolor{textcolor}%
\pgftext[x=4.850390in, y=1.425415in, left, base]{\color{textcolor}\sffamily\fontsize{9.000000}{10.800000}\selectfont Get \& prepare}%
\end{pgfscope}%
\begin{pgfscope}%
\definecolor{textcolor}{rgb}{0.150000,0.150000,0.150000}%
\pgfsetstrokecolor{textcolor}%
\pgfsetfillcolor{textcolor}%
\pgftext[x=4.850390in, y=1.281421in, left, base]{\color{textcolor}\sffamily\fontsize{9.000000}{10.800000}\selectfont obstacles}%
\end{pgfscope}%
\begin{pgfscope}%
\pgfsetroundcap%
\pgfsetroundjoin%
\pgfsetlinewidth{1.505625pt}%
\definecolor{currentstroke}{rgb}{0.800000,0.470588,0.737255}%
\pgfsetstrokecolor{currentstroke}%
\pgfsetdash{}{0pt}%
\pgfpathmoveto{\pgfqpoint{4.500390in}{1.050659in}}%
\pgfpathlineto{\pgfqpoint{4.625390in}{1.050659in}}%
\pgfpathlineto{\pgfqpoint{4.750390in}{1.050659in}}%
\pgfusepath{stroke}%
\end{pgfscope}%
\begin{pgfscope}%
\definecolor{textcolor}{rgb}{0.150000,0.150000,0.150000}%
\pgfsetstrokecolor{textcolor}%
\pgfsetfillcolor{textcolor}%
\pgftext[x=4.850390in, y=1.093921in, left, base]{\color{textcolor}\sffamily\fontsize{9.000000}{10.800000}\selectfont Merge road}%
\end{pgfscope}%
\begin{pgfscope}%
\definecolor{textcolor}{rgb}{0.150000,0.150000,0.150000}%
\pgfsetstrokecolor{textcolor}%
\pgfsetfillcolor{textcolor}%
\pgftext[x=4.850390in, y=0.949927in, left, base]{\color{textcolor}\sffamily\fontsize{9.000000}{10.800000}\selectfont edges}%
\end{pgfscope}%
\begin{pgfscope}%
\pgfsetroundcap%
\pgfsetroundjoin%
\pgfsetlinewidth{1.505625pt}%
\definecolor{currentstroke}{rgb}{0.792157,0.568627,0.380392}%
\pgfsetstrokecolor{currentstroke}%
\pgfsetdash{}{0pt}%
\pgfpathmoveto{\pgfqpoint{4.500390in}{0.719165in}}%
\pgfpathlineto{\pgfqpoint{4.625390in}{0.719165in}}%
\pgfpathlineto{\pgfqpoint{4.750390in}{0.719165in}}%
\pgfusepath{stroke}%
\end{pgfscope}%
\begin{pgfscope}%
\definecolor{textcolor}{rgb}{0.150000,0.150000,0.150000}%
\pgfsetstrokecolor{textcolor}%
\pgfsetfillcolor{textcolor}%
\pgftext[x=4.850390in, y=0.762427in, left, base]{\color{textcolor}\sffamily\fontsize{9.000000}{10.800000}\selectfont Add POI}%
\end{pgfscope}%
\begin{pgfscope}%
\definecolor{textcolor}{rgb}{0.150000,0.150000,0.150000}%
\pgfsetstrokecolor{textcolor}%
\pgfsetfillcolor{textcolor}%
\pgftext[x=4.850390in, y=0.618433in, left, base]{\color{textcolor}\sffamily\fontsize{9.000000}{10.800000}\selectfont attributes}%
\end{pgfscope}%
\begin{pgfscope}%
\pgfsetroundcap%
\pgfsetroundjoin%
\pgfsetlinewidth{1.003750pt}%
\definecolor{currentstroke}{rgb}{0.003922,0.450980,0.698039}%
\pgfsetstrokecolor{currentstroke}%
\pgfsetdash{}{0pt}%
\pgfpathmoveto{\pgfqpoint{0.761265in}{2.318490in}}%
\pgfpathlineto{\pgfqpoint{1.091548in}{2.318490in}}%
\pgfpathlineto{\pgfqpoint{1.466389in}{2.318490in}}%
\pgfpathlineto{\pgfqpoint{1.890773in}{2.318490in}}%
\pgfpathlineto{\pgfqpoint{3.157691in}{2.318490in}}%
\pgfpathlineto{\pgfqpoint{4.127042in}{2.318490in}}%
\pgfusepath{stroke}%
\end{pgfscope}%
\begin{pgfscope}%
\pgfsetbuttcap%
\pgfsetroundjoin%
\definecolor{currentfill}{rgb}{0.003922,0.450980,0.698039}%
\pgfsetfillcolor{currentfill}%
\pgfsetlinewidth{0.752812pt}%
\definecolor{currentstroke}{rgb}{1.000000,1.000000,1.000000}%
\pgfsetstrokecolor{currentstroke}%
\pgfsetdash{}{0pt}%
\pgfsys@defobject{currentmarker}{\pgfqpoint{-0.034722in}{-0.034722in}}{\pgfqpoint{0.034722in}{0.034722in}}{%
\pgfpathmoveto{\pgfqpoint{0.000000in}{-0.034722in}}%
\pgfpathcurveto{\pgfqpoint{0.009208in}{-0.034722in}}{\pgfqpoint{0.018041in}{-0.031064in}}{\pgfqpoint{0.024552in}{-0.024552in}}%
\pgfpathcurveto{\pgfqpoint{0.031064in}{-0.018041in}}{\pgfqpoint{0.034722in}{-0.009208in}}{\pgfqpoint{0.034722in}{0.000000in}}%
\pgfpathcurveto{\pgfqpoint{0.034722in}{0.009208in}}{\pgfqpoint{0.031064in}{0.018041in}}{\pgfqpoint{0.024552in}{0.024552in}}%
\pgfpathcurveto{\pgfqpoint{0.018041in}{0.031064in}}{\pgfqpoint{0.009208in}{0.034722in}}{\pgfqpoint{0.000000in}{0.034722in}}%
\pgfpathcurveto{\pgfqpoint{-0.009208in}{0.034722in}}{\pgfqpoint{-0.018041in}{0.031064in}}{\pgfqpoint{-0.024552in}{0.024552in}}%
\pgfpathcurveto{\pgfqpoint{-0.031064in}{0.018041in}}{\pgfqpoint{-0.034722in}{0.009208in}}{\pgfqpoint{-0.034722in}{0.000000in}}%
\pgfpathcurveto{\pgfqpoint{-0.034722in}{-0.009208in}}{\pgfqpoint{-0.031064in}{-0.018041in}}{\pgfqpoint{-0.024552in}{-0.024552in}}%
\pgfpathcurveto{\pgfqpoint{-0.018041in}{-0.031064in}}{\pgfqpoint{-0.009208in}{-0.034722in}}{\pgfqpoint{0.000000in}{-0.034722in}}%
\pgfpathlineto{\pgfqpoint{0.000000in}{-0.034722in}}%
\pgfpathclose%
\pgfusepath{stroke,fill}%
}%
\begin{pgfscope}%
\pgfsys@transformshift{0.761265in}{2.318490in}%
\pgfsys@useobject{currentmarker}{}%
\end{pgfscope}%
\begin{pgfscope}%
\pgfsys@transformshift{1.091548in}{2.318490in}%
\pgfsys@useobject{currentmarker}{}%
\end{pgfscope}%
\begin{pgfscope}%
\pgfsys@transformshift{1.466389in}{2.318490in}%
\pgfsys@useobject{currentmarker}{}%
\end{pgfscope}%
\begin{pgfscope}%
\pgfsys@transformshift{1.890773in}{2.318490in}%
\pgfsys@useobject{currentmarker}{}%
\end{pgfscope}%
\begin{pgfscope}%
\pgfsys@transformshift{3.157691in}{2.318490in}%
\pgfsys@useobject{currentmarker}{}%
\end{pgfscope}%
\begin{pgfscope}%
\pgfsys@transformshift{4.127042in}{2.318490in}%
\pgfsys@useobject{currentmarker}{}%
\end{pgfscope}%
\end{pgfscope}%
\begin{pgfscope}%
\pgfsetroundcap%
\pgfsetroundjoin%
\pgfsetlinewidth{1.003750pt}%
\definecolor{currentstroke}{rgb}{0.870588,0.560784,0.019608}%
\pgfsetstrokecolor{currentstroke}%
\pgfsetdash{}{0pt}%
\pgfpathmoveto{\pgfqpoint{0.761265in}{2.182346in}}%
\pgfpathlineto{\pgfqpoint{1.091548in}{2.208468in}}%
\pgfpathlineto{\pgfqpoint{1.466389in}{2.258117in}}%
\pgfpathlineto{\pgfqpoint{1.890773in}{2.273853in}}%
\pgfpathlineto{\pgfqpoint{3.157691in}{2.285173in}}%
\pgfpathlineto{\pgfqpoint{4.127042in}{2.291692in}}%
\pgfusepath{stroke}%
\end{pgfscope}%
\begin{pgfscope}%
\pgfsetbuttcap%
\pgfsetroundjoin%
\definecolor{currentfill}{rgb}{0.870588,0.560784,0.019608}%
\pgfsetfillcolor{currentfill}%
\pgfsetlinewidth{0.752812pt}%
\definecolor{currentstroke}{rgb}{1.000000,1.000000,1.000000}%
\pgfsetstrokecolor{currentstroke}%
\pgfsetdash{}{0pt}%
\pgfsys@defobject{currentmarker}{\pgfqpoint{-0.034722in}{-0.034722in}}{\pgfqpoint{0.034722in}{0.034722in}}{%
\pgfpathmoveto{\pgfqpoint{0.000000in}{-0.034722in}}%
\pgfpathcurveto{\pgfqpoint{0.009208in}{-0.034722in}}{\pgfqpoint{0.018041in}{-0.031064in}}{\pgfqpoint{0.024552in}{-0.024552in}}%
\pgfpathcurveto{\pgfqpoint{0.031064in}{-0.018041in}}{\pgfqpoint{0.034722in}{-0.009208in}}{\pgfqpoint{0.034722in}{0.000000in}}%
\pgfpathcurveto{\pgfqpoint{0.034722in}{0.009208in}}{\pgfqpoint{0.031064in}{0.018041in}}{\pgfqpoint{0.024552in}{0.024552in}}%
\pgfpathcurveto{\pgfqpoint{0.018041in}{0.031064in}}{\pgfqpoint{0.009208in}{0.034722in}}{\pgfqpoint{0.000000in}{0.034722in}}%
\pgfpathcurveto{\pgfqpoint{-0.009208in}{0.034722in}}{\pgfqpoint{-0.018041in}{0.031064in}}{\pgfqpoint{-0.024552in}{0.024552in}}%
\pgfpathcurveto{\pgfqpoint{-0.031064in}{0.018041in}}{\pgfqpoint{-0.034722in}{0.009208in}}{\pgfqpoint{-0.034722in}{0.000000in}}%
\pgfpathcurveto{\pgfqpoint{-0.034722in}{-0.009208in}}{\pgfqpoint{-0.031064in}{-0.018041in}}{\pgfqpoint{-0.024552in}{-0.024552in}}%
\pgfpathcurveto{\pgfqpoint{-0.018041in}{-0.031064in}}{\pgfqpoint{-0.009208in}{-0.034722in}}{\pgfqpoint{0.000000in}{-0.034722in}}%
\pgfpathlineto{\pgfqpoint{0.000000in}{-0.034722in}}%
\pgfpathclose%
\pgfusepath{stroke,fill}%
}%
\begin{pgfscope}%
\pgfsys@transformshift{0.761265in}{2.182346in}%
\pgfsys@useobject{currentmarker}{}%
\end{pgfscope}%
\begin{pgfscope}%
\pgfsys@transformshift{1.091548in}{2.208468in}%
\pgfsys@useobject{currentmarker}{}%
\end{pgfscope}%
\begin{pgfscope}%
\pgfsys@transformshift{1.466389in}{2.258117in}%
\pgfsys@useobject{currentmarker}{}%
\end{pgfscope}%
\begin{pgfscope}%
\pgfsys@transformshift{1.890773in}{2.273853in}%
\pgfsys@useobject{currentmarker}{}%
\end{pgfscope}%
\begin{pgfscope}%
\pgfsys@transformshift{3.157691in}{2.285173in}%
\pgfsys@useobject{currentmarker}{}%
\end{pgfscope}%
\begin{pgfscope}%
\pgfsys@transformshift{4.127042in}{2.291692in}%
\pgfsys@useobject{currentmarker}{}%
\end{pgfscope}%
\end{pgfscope}%
\begin{pgfscope}%
\pgfsetroundcap%
\pgfsetroundjoin%
\pgfsetlinewidth{1.003750pt}%
\definecolor{currentstroke}{rgb}{0.007843,0.619608,0.450980}%
\pgfsetstrokecolor{currentstroke}%
\pgfsetdash{}{0pt}%
\pgfpathmoveto{\pgfqpoint{0.761265in}{1.650967in}}%
\pgfpathlineto{\pgfqpoint{1.091548in}{1.544641in}}%
\pgfpathlineto{\pgfqpoint{1.466389in}{1.583228in}}%
\pgfpathlineto{\pgfqpoint{1.890773in}{1.580172in}}%
\pgfpathlineto{\pgfqpoint{3.157691in}{1.513270in}}%
\pgfpathlineto{\pgfqpoint{4.127042in}{1.470914in}}%
\pgfusepath{stroke}%
\end{pgfscope}%
\begin{pgfscope}%
\pgfsetbuttcap%
\pgfsetroundjoin%
\definecolor{currentfill}{rgb}{0.007843,0.619608,0.450980}%
\pgfsetfillcolor{currentfill}%
\pgfsetlinewidth{0.752812pt}%
\definecolor{currentstroke}{rgb}{1.000000,1.000000,1.000000}%
\pgfsetstrokecolor{currentstroke}%
\pgfsetdash{}{0pt}%
\pgfsys@defobject{currentmarker}{\pgfqpoint{-0.034722in}{-0.034722in}}{\pgfqpoint{0.034722in}{0.034722in}}{%
\pgfpathmoveto{\pgfqpoint{0.000000in}{-0.034722in}}%
\pgfpathcurveto{\pgfqpoint{0.009208in}{-0.034722in}}{\pgfqpoint{0.018041in}{-0.031064in}}{\pgfqpoint{0.024552in}{-0.024552in}}%
\pgfpathcurveto{\pgfqpoint{0.031064in}{-0.018041in}}{\pgfqpoint{0.034722in}{-0.009208in}}{\pgfqpoint{0.034722in}{0.000000in}}%
\pgfpathcurveto{\pgfqpoint{0.034722in}{0.009208in}}{\pgfqpoint{0.031064in}{0.018041in}}{\pgfqpoint{0.024552in}{0.024552in}}%
\pgfpathcurveto{\pgfqpoint{0.018041in}{0.031064in}}{\pgfqpoint{0.009208in}{0.034722in}}{\pgfqpoint{0.000000in}{0.034722in}}%
\pgfpathcurveto{\pgfqpoint{-0.009208in}{0.034722in}}{\pgfqpoint{-0.018041in}{0.031064in}}{\pgfqpoint{-0.024552in}{0.024552in}}%
\pgfpathcurveto{\pgfqpoint{-0.031064in}{0.018041in}}{\pgfqpoint{-0.034722in}{0.009208in}}{\pgfqpoint{-0.034722in}{0.000000in}}%
\pgfpathcurveto{\pgfqpoint{-0.034722in}{-0.009208in}}{\pgfqpoint{-0.031064in}{-0.018041in}}{\pgfqpoint{-0.024552in}{-0.024552in}}%
\pgfpathcurveto{\pgfqpoint{-0.018041in}{-0.031064in}}{\pgfqpoint{-0.009208in}{-0.034722in}}{\pgfqpoint{0.000000in}{-0.034722in}}%
\pgfpathlineto{\pgfqpoint{0.000000in}{-0.034722in}}%
\pgfpathclose%
\pgfusepath{stroke,fill}%
}%
\begin{pgfscope}%
\pgfsys@transformshift{0.761265in}{1.650967in}%
\pgfsys@useobject{currentmarker}{}%
\end{pgfscope}%
\begin{pgfscope}%
\pgfsys@transformshift{1.091548in}{1.544641in}%
\pgfsys@useobject{currentmarker}{}%
\end{pgfscope}%
\begin{pgfscope}%
\pgfsys@transformshift{1.466389in}{1.583228in}%
\pgfsys@useobject{currentmarker}{}%
\end{pgfscope}%
\begin{pgfscope}%
\pgfsys@transformshift{1.890773in}{1.580172in}%
\pgfsys@useobject{currentmarker}{}%
\end{pgfscope}%
\begin{pgfscope}%
\pgfsys@transformshift{3.157691in}{1.513270in}%
\pgfsys@useobject{currentmarker}{}%
\end{pgfscope}%
\begin{pgfscope}%
\pgfsys@transformshift{4.127042in}{1.470914in}%
\pgfsys@useobject{currentmarker}{}%
\end{pgfscope}%
\end{pgfscope}%
\begin{pgfscope}%
\pgfsetroundcap%
\pgfsetroundjoin%
\pgfsetlinewidth{1.003750pt}%
\definecolor{currentstroke}{rgb}{0.835294,0.368627,0.000000}%
\pgfsetstrokecolor{currentstroke}%
\pgfsetdash{}{0pt}%
\pgfpathmoveto{\pgfqpoint{0.761265in}{1.505470in}}%
\pgfpathlineto{\pgfqpoint{1.091548in}{1.414033in}}%
\pgfpathlineto{\pgfqpoint{1.466389in}{1.325327in}}%
\pgfpathlineto{\pgfqpoint{1.890773in}{1.290878in}}%
\pgfpathlineto{\pgfqpoint{3.157691in}{1.234099in}}%
\pgfpathlineto{\pgfqpoint{4.127042in}{1.205642in}}%
\pgfusepath{stroke}%
\end{pgfscope}%
\begin{pgfscope}%
\pgfsetbuttcap%
\pgfsetroundjoin%
\definecolor{currentfill}{rgb}{0.835294,0.368627,0.000000}%
\pgfsetfillcolor{currentfill}%
\pgfsetlinewidth{0.752812pt}%
\definecolor{currentstroke}{rgb}{1.000000,1.000000,1.000000}%
\pgfsetstrokecolor{currentstroke}%
\pgfsetdash{}{0pt}%
\pgfsys@defobject{currentmarker}{\pgfqpoint{-0.034722in}{-0.034722in}}{\pgfqpoint{0.034722in}{0.034722in}}{%
\pgfpathmoveto{\pgfqpoint{0.000000in}{-0.034722in}}%
\pgfpathcurveto{\pgfqpoint{0.009208in}{-0.034722in}}{\pgfqpoint{0.018041in}{-0.031064in}}{\pgfqpoint{0.024552in}{-0.024552in}}%
\pgfpathcurveto{\pgfqpoint{0.031064in}{-0.018041in}}{\pgfqpoint{0.034722in}{-0.009208in}}{\pgfqpoint{0.034722in}{0.000000in}}%
\pgfpathcurveto{\pgfqpoint{0.034722in}{0.009208in}}{\pgfqpoint{0.031064in}{0.018041in}}{\pgfqpoint{0.024552in}{0.024552in}}%
\pgfpathcurveto{\pgfqpoint{0.018041in}{0.031064in}}{\pgfqpoint{0.009208in}{0.034722in}}{\pgfqpoint{0.000000in}{0.034722in}}%
\pgfpathcurveto{\pgfqpoint{-0.009208in}{0.034722in}}{\pgfqpoint{-0.018041in}{0.031064in}}{\pgfqpoint{-0.024552in}{0.024552in}}%
\pgfpathcurveto{\pgfqpoint{-0.031064in}{0.018041in}}{\pgfqpoint{-0.034722in}{0.009208in}}{\pgfqpoint{-0.034722in}{0.000000in}}%
\pgfpathcurveto{\pgfqpoint{-0.034722in}{-0.009208in}}{\pgfqpoint{-0.031064in}{-0.018041in}}{\pgfqpoint{-0.024552in}{-0.024552in}}%
\pgfpathcurveto{\pgfqpoint{-0.018041in}{-0.031064in}}{\pgfqpoint{-0.009208in}{-0.034722in}}{\pgfqpoint{0.000000in}{-0.034722in}}%
\pgfpathlineto{\pgfqpoint{0.000000in}{-0.034722in}}%
\pgfpathclose%
\pgfusepath{stroke,fill}%
}%
\begin{pgfscope}%
\pgfsys@transformshift{0.761265in}{1.505470in}%
\pgfsys@useobject{currentmarker}{}%
\end{pgfscope}%
\begin{pgfscope}%
\pgfsys@transformshift{1.091548in}{1.414033in}%
\pgfsys@useobject{currentmarker}{}%
\end{pgfscope}%
\begin{pgfscope}%
\pgfsys@transformshift{1.466389in}{1.325327in}%
\pgfsys@useobject{currentmarker}{}%
\end{pgfscope}%
\begin{pgfscope}%
\pgfsys@transformshift{1.890773in}{1.290878in}%
\pgfsys@useobject{currentmarker}{}%
\end{pgfscope}%
\begin{pgfscope}%
\pgfsys@transformshift{3.157691in}{1.234099in}%
\pgfsys@useobject{currentmarker}{}%
\end{pgfscope}%
\begin{pgfscope}%
\pgfsys@transformshift{4.127042in}{1.205642in}%
\pgfsys@useobject{currentmarker}{}%
\end{pgfscope}%
\end{pgfscope}%
\begin{pgfscope}%
\pgfsetroundcap%
\pgfsetroundjoin%
\pgfsetlinewidth{1.003750pt}%
\definecolor{currentstroke}{rgb}{0.800000,0.470588,0.737255}%
\pgfsetstrokecolor{currentstroke}%
\pgfsetdash{}{0pt}%
\pgfpathmoveto{\pgfqpoint{0.761265in}{2.151934in}}%
\pgfpathlineto{\pgfqpoint{1.091548in}{2.134732in}}%
\pgfpathlineto{\pgfqpoint{1.466389in}{2.030423in}}%
\pgfpathlineto{\pgfqpoint{1.890773in}{1.971947in}}%
\pgfpathlineto{\pgfqpoint{3.157691in}{1.923719in}}%
\pgfpathlineto{\pgfqpoint{4.127042in}{1.885557in}}%
\pgfusepath{stroke}%
\end{pgfscope}%
\begin{pgfscope}%
\pgfsetbuttcap%
\pgfsetroundjoin%
\definecolor{currentfill}{rgb}{0.800000,0.470588,0.737255}%
\pgfsetfillcolor{currentfill}%
\pgfsetlinewidth{0.752812pt}%
\definecolor{currentstroke}{rgb}{1.000000,1.000000,1.000000}%
\pgfsetstrokecolor{currentstroke}%
\pgfsetdash{}{0pt}%
\pgfsys@defobject{currentmarker}{\pgfqpoint{-0.034722in}{-0.034722in}}{\pgfqpoint{0.034722in}{0.034722in}}{%
\pgfpathmoveto{\pgfqpoint{0.000000in}{-0.034722in}}%
\pgfpathcurveto{\pgfqpoint{0.009208in}{-0.034722in}}{\pgfqpoint{0.018041in}{-0.031064in}}{\pgfqpoint{0.024552in}{-0.024552in}}%
\pgfpathcurveto{\pgfqpoint{0.031064in}{-0.018041in}}{\pgfqpoint{0.034722in}{-0.009208in}}{\pgfqpoint{0.034722in}{0.000000in}}%
\pgfpathcurveto{\pgfqpoint{0.034722in}{0.009208in}}{\pgfqpoint{0.031064in}{0.018041in}}{\pgfqpoint{0.024552in}{0.024552in}}%
\pgfpathcurveto{\pgfqpoint{0.018041in}{0.031064in}}{\pgfqpoint{0.009208in}{0.034722in}}{\pgfqpoint{0.000000in}{0.034722in}}%
\pgfpathcurveto{\pgfqpoint{-0.009208in}{0.034722in}}{\pgfqpoint{-0.018041in}{0.031064in}}{\pgfqpoint{-0.024552in}{0.024552in}}%
\pgfpathcurveto{\pgfqpoint{-0.031064in}{0.018041in}}{\pgfqpoint{-0.034722in}{0.009208in}}{\pgfqpoint{-0.034722in}{0.000000in}}%
\pgfpathcurveto{\pgfqpoint{-0.034722in}{-0.009208in}}{\pgfqpoint{-0.031064in}{-0.018041in}}{\pgfqpoint{-0.024552in}{-0.024552in}}%
\pgfpathcurveto{\pgfqpoint{-0.018041in}{-0.031064in}}{\pgfqpoint{-0.009208in}{-0.034722in}}{\pgfqpoint{0.000000in}{-0.034722in}}%
\pgfpathlineto{\pgfqpoint{0.000000in}{-0.034722in}}%
\pgfpathclose%
\pgfusepath{stroke,fill}%
}%
\begin{pgfscope}%
\pgfsys@transformshift{0.761265in}{2.151934in}%
\pgfsys@useobject{currentmarker}{}%
\end{pgfscope}%
\begin{pgfscope}%
\pgfsys@transformshift{1.091548in}{2.134732in}%
\pgfsys@useobject{currentmarker}{}%
\end{pgfscope}%
\begin{pgfscope}%
\pgfsys@transformshift{1.466389in}{2.030423in}%
\pgfsys@useobject{currentmarker}{}%
\end{pgfscope}%
\begin{pgfscope}%
\pgfsys@transformshift{1.890773in}{1.971947in}%
\pgfsys@useobject{currentmarker}{}%
\end{pgfscope}%
\begin{pgfscope}%
\pgfsys@transformshift{3.157691in}{1.923719in}%
\pgfsys@useobject{currentmarker}{}%
\end{pgfscope}%
\begin{pgfscope}%
\pgfsys@transformshift{4.127042in}{1.885557in}%
\pgfsys@useobject{currentmarker}{}%
\end{pgfscope}%
\end{pgfscope}%
\begin{pgfscope}%
\pgfsetroundcap%
\pgfsetroundjoin%
\pgfsetlinewidth{1.003750pt}%
\definecolor{currentstroke}{rgb}{0.792157,0.568627,0.380392}%
\pgfsetstrokecolor{currentstroke}%
\pgfsetdash{}{0pt}%
\pgfpathmoveto{\pgfqpoint{0.761265in}{0.906338in}}%
\pgfpathlineto{\pgfqpoint{1.091548in}{0.780249in}}%
\pgfpathlineto{\pgfqpoint{1.466389in}{0.718713in}}%
\pgfpathlineto{\pgfqpoint{1.890773in}{0.802233in}}%
\pgfpathlineto{\pgfqpoint{3.157691in}{0.633730in}}%
\pgfpathlineto{\pgfqpoint{4.127042in}{0.598356in}}%
\pgfusepath{stroke}%
\end{pgfscope}%
\begin{pgfscope}%
\pgfsetbuttcap%
\pgfsetroundjoin%
\definecolor{currentfill}{rgb}{0.792157,0.568627,0.380392}%
\pgfsetfillcolor{currentfill}%
\pgfsetlinewidth{0.752812pt}%
\definecolor{currentstroke}{rgb}{1.000000,1.000000,1.000000}%
\pgfsetstrokecolor{currentstroke}%
\pgfsetdash{}{0pt}%
\pgfsys@defobject{currentmarker}{\pgfqpoint{-0.034722in}{-0.034722in}}{\pgfqpoint{0.034722in}{0.034722in}}{%
\pgfpathmoveto{\pgfqpoint{0.000000in}{-0.034722in}}%
\pgfpathcurveto{\pgfqpoint{0.009208in}{-0.034722in}}{\pgfqpoint{0.018041in}{-0.031064in}}{\pgfqpoint{0.024552in}{-0.024552in}}%
\pgfpathcurveto{\pgfqpoint{0.031064in}{-0.018041in}}{\pgfqpoint{0.034722in}{-0.009208in}}{\pgfqpoint{0.034722in}{0.000000in}}%
\pgfpathcurveto{\pgfqpoint{0.034722in}{0.009208in}}{\pgfqpoint{0.031064in}{0.018041in}}{\pgfqpoint{0.024552in}{0.024552in}}%
\pgfpathcurveto{\pgfqpoint{0.018041in}{0.031064in}}{\pgfqpoint{0.009208in}{0.034722in}}{\pgfqpoint{0.000000in}{0.034722in}}%
\pgfpathcurveto{\pgfqpoint{-0.009208in}{0.034722in}}{\pgfqpoint{-0.018041in}{0.031064in}}{\pgfqpoint{-0.024552in}{0.024552in}}%
\pgfpathcurveto{\pgfqpoint{-0.031064in}{0.018041in}}{\pgfqpoint{-0.034722in}{0.009208in}}{\pgfqpoint{-0.034722in}{0.000000in}}%
\pgfpathcurveto{\pgfqpoint{-0.034722in}{-0.009208in}}{\pgfqpoint{-0.031064in}{-0.018041in}}{\pgfqpoint{-0.024552in}{-0.024552in}}%
\pgfpathcurveto{\pgfqpoint{-0.018041in}{-0.031064in}}{\pgfqpoint{-0.009208in}{-0.034722in}}{\pgfqpoint{0.000000in}{-0.034722in}}%
\pgfpathlineto{\pgfqpoint{0.000000in}{-0.034722in}}%
\pgfpathclose%
\pgfusepath{stroke,fill}%
}%
\begin{pgfscope}%
\pgfsys@transformshift{0.761265in}{0.906338in}%
\pgfsys@useobject{currentmarker}{}%
\end{pgfscope}%
\begin{pgfscope}%
\pgfsys@transformshift{1.091548in}{0.780249in}%
\pgfsys@useobject{currentmarker}{}%
\end{pgfscope}%
\begin{pgfscope}%
\pgfsys@transformshift{1.466389in}{0.718713in}%
\pgfsys@useobject{currentmarker}{}%
\end{pgfscope}%
\begin{pgfscope}%
\pgfsys@transformshift{1.890773in}{0.802233in}%
\pgfsys@useobject{currentmarker}{}%
\end{pgfscope}%
\begin{pgfscope}%
\pgfsys@transformshift{3.157691in}{0.633730in}%
\pgfsys@useobject{currentmarker}{}%
\end{pgfscope}%
\begin{pgfscope}%
\pgfsys@transformshift{4.127042in}{0.598356in}%
\pgfsys@useobject{currentmarker}{}%
\end{pgfscope}%
\end{pgfscope}%
\end{pgfpicture}%
\makeatother%
\endgroup%

%					\end{figcenter}
%					\caption{Relative share of each task on the total import time.}
%				\end{subfigure}
			\end{figcenter}
			\\
			\begin{figcenter}
				\begin{subfigure}[t]{\textwidth}
					\begin{figcenter}
						%% Creator: Matplotlib, PGF backend
%%
%% To include the figure in your LaTeX document, write
%%   \input{<filename>.pgf}
%%
%% Make sure the required packages are loaded in your preamble
%%   \usepackage{pgf}
%%
%% Also ensure that all the required font packages are loaded; for instance,
%% the lmodern package is sometimes necessary when using math font.
%%   \usepackage{lmodern}
%%
%% Figures using additional raster images can only be included by \input if
%% they are in the same directory as the main LaTeX file. For loading figures
%% from other directories you can use the `import` package
%%   \usepackage{import}
%%
%% and then include the figures with
%%   \import{<path to file>}{<filename>.pgf}
%%
%% Matplotlib used the following preamble
%%   
%%   \usepackage{fontspec}
%%   \setmainfont{DejaVuSerif.ttf}[Path=\detokenize{/home/hauke/.local/lib/python3.11/site-packages/matplotlib/mpl-data/fonts/ttf/}]
%%   \setsansfont{DroidSans.ttf}[Path=\detokenize{/usr/share/fonts/droid/}]
%%   \setmonofont{DejaVuSansMono.ttf}[Path=\detokenize{/home/hauke/.local/lib/python3.11/site-packages/matplotlib/mpl-data/fonts/ttf/}]
%%   \makeatletter\@ifpackageloaded{underscore}{}{\usepackage[strings]{underscore}}\makeatother
%%
\begingroup%
\makeatletter%
\begin{pgfpicture}%
\pgfpathrectangle{\pgfpointorigin}{\pgfqpoint{5.745359in}{3.237620in}}%
\pgfusepath{use as bounding box, clip}%
\begin{pgfscope}%
\pgfsetbuttcap%
\pgfsetmiterjoin%
\definecolor{currentfill}{rgb}{1.000000,1.000000,1.000000}%
\pgfsetfillcolor{currentfill}%
\pgfsetlinewidth{0.000000pt}%
\definecolor{currentstroke}{rgb}{1.000000,1.000000,1.000000}%
\pgfsetstrokecolor{currentstroke}%
\pgfsetdash{}{0pt}%
\pgfpathmoveto{\pgfqpoint{0.000000in}{0.000000in}}%
\pgfpathlineto{\pgfqpoint{5.745359in}{0.000000in}}%
\pgfpathlineto{\pgfqpoint{5.745359in}{3.237620in}}%
\pgfpathlineto{\pgfqpoint{0.000000in}{3.237620in}}%
\pgfpathlineto{\pgfqpoint{0.000000in}{0.000000in}}%
\pgfpathclose%
\pgfusepath{fill}%
\end{pgfscope}%
\begin{pgfscope}%
\pgfsetbuttcap%
\pgfsetmiterjoin%
\definecolor{currentfill}{rgb}{1.000000,1.000000,1.000000}%
\pgfsetfillcolor{currentfill}%
\pgfsetlinewidth{0.000000pt}%
\definecolor{currentstroke}{rgb}{0.000000,0.000000,0.000000}%
\pgfsetstrokecolor{currentstroke}%
\pgfsetstrokeopacity{0.000000}%
\pgfsetdash{}{0pt}%
\pgfpathmoveto{\pgfqpoint{0.592976in}{0.451389in}}%
\pgfpathlineto{\pgfqpoint{4.095253in}{0.451389in}}%
\pgfpathlineto{\pgfqpoint{4.095253in}{3.237620in}}%
\pgfpathlineto{\pgfqpoint{0.592976in}{3.237620in}}%
\pgfpathlineto{\pgfqpoint{0.592976in}{0.451389in}}%
\pgfpathclose%
\pgfusepath{fill}%
\end{pgfscope}%
\begin{pgfscope}%
\pgfpathrectangle{\pgfqpoint{0.592976in}{0.451389in}}{\pgfqpoint{3.502277in}{2.786231in}}%
\pgfusepath{clip}%
\pgfsetroundcap%
\pgfsetroundjoin%
\pgfsetlinewidth{1.003750pt}%
\definecolor{currentstroke}{rgb}{0.800000,0.800000,0.800000}%
\pgfsetstrokecolor{currentstroke}%
\pgfsetdash{}{0pt}%
\pgfpathmoveto{\pgfqpoint{0.744191in}{0.451389in}}%
\pgfpathlineto{\pgfqpoint{0.744191in}{3.237620in}}%
\pgfusepath{stroke}%
\end{pgfscope}%
\begin{pgfscope}%
\definecolor{textcolor}{rgb}{0.150000,0.150000,0.150000}%
\pgfsetstrokecolor{textcolor}%
\pgfsetfillcolor{textcolor}%
\pgftext[x=0.744191in,y=0.319444in,,top]{\color{textcolor}\sffamily\fontsize{9.000000}{10.800000}\selectfont 0}%
\end{pgfscope}%
\begin{pgfscope}%
\pgfpathrectangle{\pgfqpoint{0.592976in}{0.451389in}}{\pgfqpoint{3.502277in}{2.786231in}}%
\pgfusepath{clip}%
\pgfsetroundcap%
\pgfsetroundjoin%
\pgfsetlinewidth{1.003750pt}%
\definecolor{currentstroke}{rgb}{0.800000,0.800000,0.800000}%
\pgfsetstrokecolor{currentstroke}%
\pgfsetdash{}{0pt}%
\pgfpathmoveto{\pgfqpoint{1.276169in}{0.451389in}}%
\pgfpathlineto{\pgfqpoint{1.276169in}{3.237620in}}%
\pgfusepath{stroke}%
\end{pgfscope}%
\begin{pgfscope}%
\definecolor{textcolor}{rgb}{0.150000,0.150000,0.150000}%
\pgfsetstrokecolor{textcolor}%
\pgfsetfillcolor{textcolor}%
\pgftext[x=1.276169in,y=0.319444in,,top]{\color{textcolor}\sffamily\fontsize{9.000000}{10.800000}\selectfont 5000}%
\end{pgfscope}%
\begin{pgfscope}%
\pgfpathrectangle{\pgfqpoint{0.592976in}{0.451389in}}{\pgfqpoint{3.502277in}{2.786231in}}%
\pgfusepath{clip}%
\pgfsetroundcap%
\pgfsetroundjoin%
\pgfsetlinewidth{1.003750pt}%
\definecolor{currentstroke}{rgb}{0.800000,0.800000,0.800000}%
\pgfsetstrokecolor{currentstroke}%
\pgfsetdash{}{0pt}%
\pgfpathmoveto{\pgfqpoint{1.808147in}{0.451389in}}%
\pgfpathlineto{\pgfqpoint{1.808147in}{3.237620in}}%
\pgfusepath{stroke}%
\end{pgfscope}%
\begin{pgfscope}%
\definecolor{textcolor}{rgb}{0.150000,0.150000,0.150000}%
\pgfsetstrokecolor{textcolor}%
\pgfsetfillcolor{textcolor}%
\pgftext[x=1.808147in,y=0.319444in,,top]{\color{textcolor}\sffamily\fontsize{9.000000}{10.800000}\selectfont 10000}%
\end{pgfscope}%
\begin{pgfscope}%
\pgfpathrectangle{\pgfqpoint{0.592976in}{0.451389in}}{\pgfqpoint{3.502277in}{2.786231in}}%
\pgfusepath{clip}%
\pgfsetroundcap%
\pgfsetroundjoin%
\pgfsetlinewidth{1.003750pt}%
\definecolor{currentstroke}{rgb}{0.800000,0.800000,0.800000}%
\pgfsetstrokecolor{currentstroke}%
\pgfsetdash{}{0pt}%
\pgfpathmoveto{\pgfqpoint{2.340125in}{0.451389in}}%
\pgfpathlineto{\pgfqpoint{2.340125in}{3.237620in}}%
\pgfusepath{stroke}%
\end{pgfscope}%
\begin{pgfscope}%
\definecolor{textcolor}{rgb}{0.150000,0.150000,0.150000}%
\pgfsetstrokecolor{textcolor}%
\pgfsetfillcolor{textcolor}%
\pgftext[x=2.340125in,y=0.319444in,,top]{\color{textcolor}\sffamily\fontsize{9.000000}{10.800000}\selectfont 15000}%
\end{pgfscope}%
\begin{pgfscope}%
\pgfpathrectangle{\pgfqpoint{0.592976in}{0.451389in}}{\pgfqpoint{3.502277in}{2.786231in}}%
\pgfusepath{clip}%
\pgfsetroundcap%
\pgfsetroundjoin%
\pgfsetlinewidth{1.003750pt}%
\definecolor{currentstroke}{rgb}{0.800000,0.800000,0.800000}%
\pgfsetstrokecolor{currentstroke}%
\pgfsetdash{}{0pt}%
\pgfpathmoveto{\pgfqpoint{2.872103in}{0.451389in}}%
\pgfpathlineto{\pgfqpoint{2.872103in}{3.237620in}}%
\pgfusepath{stroke}%
\end{pgfscope}%
\begin{pgfscope}%
\definecolor{textcolor}{rgb}{0.150000,0.150000,0.150000}%
\pgfsetstrokecolor{textcolor}%
\pgfsetfillcolor{textcolor}%
\pgftext[x=2.872103in,y=0.319444in,,top]{\color{textcolor}\sffamily\fontsize{9.000000}{10.800000}\selectfont 20000}%
\end{pgfscope}%
\begin{pgfscope}%
\pgfpathrectangle{\pgfqpoint{0.592976in}{0.451389in}}{\pgfqpoint{3.502277in}{2.786231in}}%
\pgfusepath{clip}%
\pgfsetroundcap%
\pgfsetroundjoin%
\pgfsetlinewidth{1.003750pt}%
\definecolor{currentstroke}{rgb}{0.800000,0.800000,0.800000}%
\pgfsetstrokecolor{currentstroke}%
\pgfsetdash{}{0pt}%
\pgfpathmoveto{\pgfqpoint{3.404081in}{0.451389in}}%
\pgfpathlineto{\pgfqpoint{3.404081in}{3.237620in}}%
\pgfusepath{stroke}%
\end{pgfscope}%
\begin{pgfscope}%
\definecolor{textcolor}{rgb}{0.150000,0.150000,0.150000}%
\pgfsetstrokecolor{textcolor}%
\pgfsetfillcolor{textcolor}%
\pgftext[x=3.404081in,y=0.319444in,,top]{\color{textcolor}\sffamily\fontsize{9.000000}{10.800000}\selectfont 25000}%
\end{pgfscope}%
\begin{pgfscope}%
\pgfpathrectangle{\pgfqpoint{0.592976in}{0.451389in}}{\pgfqpoint{3.502277in}{2.786231in}}%
\pgfusepath{clip}%
\pgfsetroundcap%
\pgfsetroundjoin%
\pgfsetlinewidth{1.003750pt}%
\definecolor{currentstroke}{rgb}{0.800000,0.800000,0.800000}%
\pgfsetstrokecolor{currentstroke}%
\pgfsetdash{}{0pt}%
\pgfpathmoveto{\pgfqpoint{3.936059in}{0.451389in}}%
\pgfpathlineto{\pgfqpoint{3.936059in}{3.237620in}}%
\pgfusepath{stroke}%
\end{pgfscope}%
\begin{pgfscope}%
\definecolor{textcolor}{rgb}{0.150000,0.150000,0.150000}%
\pgfsetstrokecolor{textcolor}%
\pgfsetfillcolor{textcolor}%
\pgftext[x=3.936059in,y=0.319444in,,top]{\color{textcolor}\sffamily\fontsize{9.000000}{10.800000}\selectfont 30000}%
\end{pgfscope}%
\begin{pgfscope}%
\definecolor{textcolor}{rgb}{0.150000,0.150000,0.150000}%
\pgfsetstrokecolor{textcolor}%
\pgfsetfillcolor{textcolor}%
\pgftext[x=2.344114in,y=0.125000in,,top]{\color{textcolor}\sffamily\fontsize{9.000000}{10.800000}\selectfont Input obstacle vertices}%
\end{pgfscope}%
\begin{pgfscope}%
\pgfpathrectangle{\pgfqpoint{0.592976in}{0.451389in}}{\pgfqpoint{3.502277in}{2.786231in}}%
\pgfusepath{clip}%
\pgfsetroundcap%
\pgfsetroundjoin%
\pgfsetlinewidth{1.003750pt}%
\definecolor{currentstroke}{rgb}{0.800000,0.800000,0.800000}%
\pgfsetstrokecolor{currentstroke}%
\pgfsetdash{}{0pt}%
\pgfpathmoveto{\pgfqpoint{0.592976in}{0.489539in}}%
\pgfpathlineto{\pgfqpoint{4.095253in}{0.489539in}}%
\pgfusepath{stroke}%
\end{pgfscope}%
\begin{pgfscope}%
\definecolor{textcolor}{rgb}{0.150000,0.150000,0.150000}%
\pgfsetstrokecolor{textcolor}%
\pgfsetfillcolor{textcolor}%
\pgftext[x=0.194444in, y=0.442053in, left, base]{\color{textcolor}\sffamily\fontsize{9.000000}{10.800000}\selectfont \(\displaystyle {10^{-5}}\)}%
\end{pgfscope}%
\begin{pgfscope}%
\pgfpathrectangle{\pgfqpoint{0.592976in}{0.451389in}}{\pgfqpoint{3.502277in}{2.786231in}}%
\pgfusepath{clip}%
\pgfsetroundcap%
\pgfsetroundjoin%
\pgfsetlinewidth{1.003750pt}%
\definecolor{currentstroke}{rgb}{0.800000,0.800000,0.800000}%
\pgfsetstrokecolor{currentstroke}%
\pgfsetdash{}{0pt}%
\pgfpathmoveto{\pgfqpoint{0.592976in}{0.836215in}}%
\pgfpathlineto{\pgfqpoint{4.095253in}{0.836215in}}%
\pgfusepath{stroke}%
\end{pgfscope}%
\begin{pgfscope}%
\definecolor{textcolor}{rgb}{0.150000,0.150000,0.150000}%
\pgfsetstrokecolor{textcolor}%
\pgfsetfillcolor{textcolor}%
\pgftext[x=0.194444in, y=0.788729in, left, base]{\color{textcolor}\sffamily\fontsize{9.000000}{10.800000}\selectfont \(\displaystyle {10^{-4}}\)}%
\end{pgfscope}%
\begin{pgfscope}%
\pgfpathrectangle{\pgfqpoint{0.592976in}{0.451389in}}{\pgfqpoint{3.502277in}{2.786231in}}%
\pgfusepath{clip}%
\pgfsetroundcap%
\pgfsetroundjoin%
\pgfsetlinewidth{1.003750pt}%
\definecolor{currentstroke}{rgb}{0.800000,0.800000,0.800000}%
\pgfsetstrokecolor{currentstroke}%
\pgfsetdash{}{0pt}%
\pgfpathmoveto{\pgfqpoint{0.592976in}{1.182890in}}%
\pgfpathlineto{\pgfqpoint{4.095253in}{1.182890in}}%
\pgfusepath{stroke}%
\end{pgfscope}%
\begin{pgfscope}%
\definecolor{textcolor}{rgb}{0.150000,0.150000,0.150000}%
\pgfsetstrokecolor{textcolor}%
\pgfsetfillcolor{textcolor}%
\pgftext[x=0.194444in, y=1.135405in, left, base]{\color{textcolor}\sffamily\fontsize{9.000000}{10.800000}\selectfont \(\displaystyle {10^{-3}}\)}%
\end{pgfscope}%
\begin{pgfscope}%
\pgfpathrectangle{\pgfqpoint{0.592976in}{0.451389in}}{\pgfqpoint{3.502277in}{2.786231in}}%
\pgfusepath{clip}%
\pgfsetroundcap%
\pgfsetroundjoin%
\pgfsetlinewidth{1.003750pt}%
\definecolor{currentstroke}{rgb}{0.800000,0.800000,0.800000}%
\pgfsetstrokecolor{currentstroke}%
\pgfsetdash{}{0pt}%
\pgfpathmoveto{\pgfqpoint{0.592976in}{1.529566in}}%
\pgfpathlineto{\pgfqpoint{4.095253in}{1.529566in}}%
\pgfusepath{stroke}%
\end{pgfscope}%
\begin{pgfscope}%
\definecolor{textcolor}{rgb}{0.150000,0.150000,0.150000}%
\pgfsetstrokecolor{textcolor}%
\pgfsetfillcolor{textcolor}%
\pgftext[x=0.194444in, y=1.482081in, left, base]{\color{textcolor}\sffamily\fontsize{9.000000}{10.800000}\selectfont \(\displaystyle {10^{-2}}\)}%
\end{pgfscope}%
\begin{pgfscope}%
\pgfpathrectangle{\pgfqpoint{0.592976in}{0.451389in}}{\pgfqpoint{3.502277in}{2.786231in}}%
\pgfusepath{clip}%
\pgfsetroundcap%
\pgfsetroundjoin%
\pgfsetlinewidth{1.003750pt}%
\definecolor{currentstroke}{rgb}{0.800000,0.800000,0.800000}%
\pgfsetstrokecolor{currentstroke}%
\pgfsetdash{}{0pt}%
\pgfpathmoveto{\pgfqpoint{0.592976in}{1.876242in}}%
\pgfpathlineto{\pgfqpoint{4.095253in}{1.876242in}}%
\pgfusepath{stroke}%
\end{pgfscope}%
\begin{pgfscope}%
\definecolor{textcolor}{rgb}{0.150000,0.150000,0.150000}%
\pgfsetstrokecolor{textcolor}%
\pgfsetfillcolor{textcolor}%
\pgftext[x=0.194444in, y=1.828757in, left, base]{\color{textcolor}\sffamily\fontsize{9.000000}{10.800000}\selectfont \(\displaystyle {10^{-1}}\)}%
\end{pgfscope}%
\begin{pgfscope}%
\pgfpathrectangle{\pgfqpoint{0.592976in}{0.451389in}}{\pgfqpoint{3.502277in}{2.786231in}}%
\pgfusepath{clip}%
\pgfsetroundcap%
\pgfsetroundjoin%
\pgfsetlinewidth{1.003750pt}%
\definecolor{currentstroke}{rgb}{0.800000,0.800000,0.800000}%
\pgfsetstrokecolor{currentstroke}%
\pgfsetdash{}{0pt}%
\pgfpathmoveto{\pgfqpoint{0.592976in}{2.222918in}}%
\pgfpathlineto{\pgfqpoint{4.095253in}{2.222918in}}%
\pgfusepath{stroke}%
\end{pgfscope}%
\begin{pgfscope}%
\definecolor{textcolor}{rgb}{0.150000,0.150000,0.150000}%
\pgfsetstrokecolor{textcolor}%
\pgfsetfillcolor{textcolor}%
\pgftext[x=0.274690in, y=2.175433in, left, base]{\color{textcolor}\sffamily\fontsize{9.000000}{10.800000}\selectfont \(\displaystyle {10^{0}}\)}%
\end{pgfscope}%
\begin{pgfscope}%
\pgfpathrectangle{\pgfqpoint{0.592976in}{0.451389in}}{\pgfqpoint{3.502277in}{2.786231in}}%
\pgfusepath{clip}%
\pgfsetroundcap%
\pgfsetroundjoin%
\pgfsetlinewidth{1.003750pt}%
\definecolor{currentstroke}{rgb}{0.800000,0.800000,0.800000}%
\pgfsetstrokecolor{currentstroke}%
\pgfsetdash{}{0pt}%
\pgfpathmoveto{\pgfqpoint{0.592976in}{2.569594in}}%
\pgfpathlineto{\pgfqpoint{4.095253in}{2.569594in}}%
\pgfusepath{stroke}%
\end{pgfscope}%
\begin{pgfscope}%
\definecolor{textcolor}{rgb}{0.150000,0.150000,0.150000}%
\pgfsetstrokecolor{textcolor}%
\pgfsetfillcolor{textcolor}%
\pgftext[x=0.274690in, y=2.522109in, left, base]{\color{textcolor}\sffamily\fontsize{9.000000}{10.800000}\selectfont \(\displaystyle {10^{1}}\)}%
\end{pgfscope}%
\begin{pgfscope}%
\pgfpathrectangle{\pgfqpoint{0.592976in}{0.451389in}}{\pgfqpoint{3.502277in}{2.786231in}}%
\pgfusepath{clip}%
\pgfsetroundcap%
\pgfsetroundjoin%
\pgfsetlinewidth{1.003750pt}%
\definecolor{currentstroke}{rgb}{0.800000,0.800000,0.800000}%
\pgfsetstrokecolor{currentstroke}%
\pgfsetdash{}{0pt}%
\pgfpathmoveto{\pgfqpoint{0.592976in}{2.916270in}}%
\pgfpathlineto{\pgfqpoint{4.095253in}{2.916270in}}%
\pgfusepath{stroke}%
\end{pgfscope}%
\begin{pgfscope}%
\definecolor{textcolor}{rgb}{0.150000,0.150000,0.150000}%
\pgfsetstrokecolor{textcolor}%
\pgfsetfillcolor{textcolor}%
\pgftext[x=0.274690in, y=2.868784in, left, base]{\color{textcolor}\sffamily\fontsize{9.000000}{10.800000}\selectfont \(\displaystyle {10^{2}}\)}%
\end{pgfscope}%
\begin{pgfscope}%
\definecolor{textcolor}{rgb}{0.150000,0.150000,0.150000}%
\pgfsetstrokecolor{textcolor}%
\pgfsetfillcolor{textcolor}%
\pgftext[x=0.125000in,y=1.844504in,,bottom,rotate=90.000000]{\color{textcolor}\sffamily\fontsize{9.000000}{10.800000}\selectfont Time in s}%
\end{pgfscope}%
\begin{pgfscope}%
\pgfpathrectangle{\pgfqpoint{0.592976in}{0.451389in}}{\pgfqpoint{3.502277in}{2.786231in}}%
\pgfusepath{clip}%
\pgfsetbuttcap%
\pgfsetroundjoin%
\definecolor{currentfill}{rgb}{0.003922,0.450980,0.698039}%
\pgfsetfillcolor{currentfill}%
\pgfsetfillopacity{0.200000}%
\pgfsetlinewidth{1.003750pt}%
\definecolor{currentstroke}{rgb}{0.003922,0.450980,0.698039}%
\pgfsetstrokecolor{currentstroke}%
\pgfsetstrokeopacity{0.200000}%
\pgfsetdash{}{0pt}%
\pgfsys@defobject{currentmarker}{\pgfqpoint{0.752170in}{1.539725in}}{\pgfqpoint{3.936059in}{3.110973in}}{%
\pgfpathmoveto{\pgfqpoint{0.752170in}{1.574131in}}%
\pgfpathlineto{\pgfqpoint{0.752170in}{1.539725in}}%
\pgfpathlineto{\pgfqpoint{0.776109in}{1.748589in}}%
\pgfpathlineto{\pgfqpoint{0.816008in}{1.935627in}}%
\pgfpathlineto{\pgfqpoint{0.871865in}{2.107138in}}%
\pgfpathlineto{\pgfqpoint{0.943682in}{2.230496in}}%
\pgfpathlineto{\pgfqpoint{1.031459in}{2.341192in}}%
\pgfpathlineto{\pgfqpoint{1.135194in}{2.417909in}}%
\pgfpathlineto{\pgfqpoint{1.254889in}{2.521621in}}%
\pgfpathlineto{\pgfqpoint{1.390544in}{2.560815in}}%
\pgfpathlineto{\pgfqpoint{1.542158in}{2.634818in}}%
\pgfpathlineto{\pgfqpoint{1.893263in}{2.765225in}}%
\pgfpathlineto{\pgfqpoint{2.308206in}{2.842439in}}%
\pgfpathlineto{\pgfqpoint{2.786986in}{2.980611in}}%
\pgfpathlineto{\pgfqpoint{3.329604in}{3.009306in}}%
\pgfpathlineto{\pgfqpoint{3.936059in}{3.110007in}}%
\pgfpathlineto{\pgfqpoint{3.936059in}{3.110973in}}%
\pgfpathlineto{\pgfqpoint{3.936059in}{3.110973in}}%
\pgfpathlineto{\pgfqpoint{3.329604in}{3.013752in}}%
\pgfpathlineto{\pgfqpoint{2.786986in}{2.990069in}}%
\pgfpathlineto{\pgfqpoint{2.308206in}{2.848349in}}%
\pgfpathlineto{\pgfqpoint{1.893263in}{2.767325in}}%
\pgfpathlineto{\pgfqpoint{1.542158in}{2.637988in}}%
\pgfpathlineto{\pgfqpoint{1.390544in}{2.564571in}}%
\pgfpathlineto{\pgfqpoint{1.254889in}{2.525128in}}%
\pgfpathlineto{\pgfqpoint{1.135194in}{2.419417in}}%
\pgfpathlineto{\pgfqpoint{1.031459in}{2.342323in}}%
\pgfpathlineto{\pgfqpoint{0.943682in}{2.231636in}}%
\pgfpathlineto{\pgfqpoint{0.871865in}{2.110017in}}%
\pgfpathlineto{\pgfqpoint{0.816008in}{1.939399in}}%
\pgfpathlineto{\pgfqpoint{0.776109in}{1.780735in}}%
\pgfpathlineto{\pgfqpoint{0.752170in}{1.574131in}}%
\pgfpathlineto{\pgfqpoint{0.752170in}{1.574131in}}%
\pgfpathclose%
\pgfusepath{stroke,fill}%
}%
\begin{pgfscope}%
\pgfsys@transformshift{0.000000in}{0.000000in}%
\pgfsys@useobject{currentmarker}{}%
\end{pgfscope}%
\end{pgfscope}%
\begin{pgfscope}%
\pgfpathrectangle{\pgfqpoint{0.592976in}{0.451389in}}{\pgfqpoint{3.502277in}{2.786231in}}%
\pgfusepath{clip}%
\pgfsetbuttcap%
\pgfsetroundjoin%
\definecolor{currentfill}{rgb}{0.870588,0.560784,0.019608}%
\pgfsetfillcolor{currentfill}%
\pgfsetfillopacity{0.200000}%
\pgfsetlinewidth{1.003750pt}%
\definecolor{currentstroke}{rgb}{0.870588,0.560784,0.019608}%
\pgfsetstrokecolor{currentstroke}%
\pgfsetstrokeopacity{0.200000}%
\pgfsetdash{}{0pt}%
\pgfsys@defobject{currentmarker}{\pgfqpoint{0.752170in}{1.501955in}}{\pgfqpoint{3.936059in}{3.110389in}}{%
\pgfpathmoveto{\pgfqpoint{0.752170in}{1.504771in}}%
\pgfpathlineto{\pgfqpoint{0.752170in}{1.501955in}}%
\pgfpathlineto{\pgfqpoint{0.776109in}{1.720774in}}%
\pgfpathlineto{\pgfqpoint{0.816008in}{1.923716in}}%
\pgfpathlineto{\pgfqpoint{0.871865in}{2.099865in}}%
\pgfpathlineto{\pgfqpoint{0.943682in}{2.222718in}}%
\pgfpathlineto{\pgfqpoint{1.031459in}{2.335152in}}%
\pgfpathlineto{\pgfqpoint{1.135194in}{2.412068in}}%
\pgfpathlineto{\pgfqpoint{1.254889in}{2.517935in}}%
\pgfpathlineto{\pgfqpoint{1.390544in}{2.557177in}}%
\pgfpathlineto{\pgfqpoint{1.542158in}{2.631969in}}%
\pgfpathlineto{\pgfqpoint{1.893263in}{2.763416in}}%
\pgfpathlineto{\pgfqpoint{2.308206in}{2.840830in}}%
\pgfpathlineto{\pgfqpoint{2.786986in}{2.979769in}}%
\pgfpathlineto{\pgfqpoint{3.329604in}{3.008517in}}%
\pgfpathlineto{\pgfqpoint{3.936059in}{3.109419in}}%
\pgfpathlineto{\pgfqpoint{3.936059in}{3.110389in}}%
\pgfpathlineto{\pgfqpoint{3.936059in}{3.110389in}}%
\pgfpathlineto{\pgfqpoint{3.329604in}{3.012967in}}%
\pgfpathlineto{\pgfqpoint{2.786986in}{2.989246in}}%
\pgfpathlineto{\pgfqpoint{2.308206in}{2.846717in}}%
\pgfpathlineto{\pgfqpoint{1.893263in}{2.765335in}}%
\pgfpathlineto{\pgfqpoint{1.542158in}{2.634490in}}%
\pgfpathlineto{\pgfqpoint{1.390544in}{2.560051in}}%
\pgfpathlineto{\pgfqpoint{1.254889in}{2.520634in}}%
\pgfpathlineto{\pgfqpoint{1.135194in}{2.412685in}}%
\pgfpathlineto{\pgfqpoint{1.031459in}{2.335843in}}%
\pgfpathlineto{\pgfqpoint{0.943682in}{2.223530in}}%
\pgfpathlineto{\pgfqpoint{0.871865in}{2.102489in}}%
\pgfpathlineto{\pgfqpoint{0.816008in}{1.924428in}}%
\pgfpathlineto{\pgfqpoint{0.776109in}{1.742032in}}%
\pgfpathlineto{\pgfqpoint{0.752170in}{1.504771in}}%
\pgfpathlineto{\pgfqpoint{0.752170in}{1.504771in}}%
\pgfpathclose%
\pgfusepath{stroke,fill}%
}%
\begin{pgfscope}%
\pgfsys@transformshift{0.000000in}{0.000000in}%
\pgfsys@useobject{currentmarker}{}%
\end{pgfscope}%
\end{pgfscope}%
\begin{pgfscope}%
\pgfpathrectangle{\pgfqpoint{0.592976in}{0.451389in}}{\pgfqpoint{3.502277in}{2.786231in}}%
\pgfusepath{clip}%
\pgfsetbuttcap%
\pgfsetroundjoin%
\definecolor{currentfill}{rgb}{0.007843,0.619608,0.450980}%
\pgfsetfillcolor{currentfill}%
\pgfsetfillopacity{0.200000}%
\pgfsetlinewidth{1.003750pt}%
\definecolor{currentstroke}{rgb}{0.007843,0.619608,0.450980}%
\pgfsetstrokecolor{currentstroke}%
\pgfsetstrokeopacity{0.200000}%
\pgfsetdash{}{0pt}%
\pgfsys@defobject{currentmarker}{\pgfqpoint{0.752170in}{1.240464in}}{\pgfqpoint{3.936059in}{2.250245in}}{%
\pgfpathmoveto{\pgfqpoint{0.752170in}{1.244378in}}%
\pgfpathlineto{\pgfqpoint{0.752170in}{1.240464in}}%
\pgfpathlineto{\pgfqpoint{0.776109in}{1.390249in}}%
\pgfpathlineto{\pgfqpoint{0.816008in}{1.491380in}}%
\pgfpathlineto{\pgfqpoint{0.871865in}{1.591541in}}%
\pgfpathlineto{\pgfqpoint{0.943682in}{1.743216in}}%
\pgfpathlineto{\pgfqpoint{1.031459in}{1.814064in}}%
\pgfpathlineto{\pgfqpoint{1.135194in}{1.882849in}}%
\pgfpathlineto{\pgfqpoint{1.254889in}{1.924982in}}%
\pgfpathlineto{\pgfqpoint{1.390544in}{1.966511in}}%
\pgfpathlineto{\pgfqpoint{1.542158in}{2.006307in}}%
\pgfpathlineto{\pgfqpoint{1.893263in}{2.061371in}}%
\pgfpathlineto{\pgfqpoint{2.308206in}{2.117734in}}%
\pgfpathlineto{\pgfqpoint{2.786986in}{2.174612in}}%
\pgfpathlineto{\pgfqpoint{3.329604in}{2.182936in}}%
\pgfpathlineto{\pgfqpoint{3.936059in}{2.247667in}}%
\pgfpathlineto{\pgfqpoint{3.936059in}{2.250245in}}%
\pgfpathlineto{\pgfqpoint{3.936059in}{2.250245in}}%
\pgfpathlineto{\pgfqpoint{3.329604in}{2.215296in}}%
\pgfpathlineto{\pgfqpoint{2.786986in}{2.182201in}}%
\pgfpathlineto{\pgfqpoint{2.308206in}{2.140917in}}%
\pgfpathlineto{\pgfqpoint{1.893263in}{2.092039in}}%
\pgfpathlineto{\pgfqpoint{1.542158in}{2.040175in}}%
\pgfpathlineto{\pgfqpoint{1.390544in}{2.002330in}}%
\pgfpathlineto{\pgfqpoint{1.254889in}{1.933040in}}%
\pgfpathlineto{\pgfqpoint{1.135194in}{1.885418in}}%
\pgfpathlineto{\pgfqpoint{1.031459in}{1.840603in}}%
\pgfpathlineto{\pgfqpoint{0.943682in}{1.748603in}}%
\pgfpathlineto{\pgfqpoint{0.871865in}{1.597169in}}%
\pgfpathlineto{\pgfqpoint{0.816008in}{1.495533in}}%
\pgfpathlineto{\pgfqpoint{0.776109in}{1.467739in}}%
\pgfpathlineto{\pgfqpoint{0.752170in}{1.244378in}}%
\pgfpathlineto{\pgfqpoint{0.752170in}{1.244378in}}%
\pgfpathclose%
\pgfusepath{stroke,fill}%
}%
\begin{pgfscope}%
\pgfsys@transformshift{0.000000in}{0.000000in}%
\pgfsys@useobject{currentmarker}{}%
\end{pgfscope}%
\end{pgfscope}%
\begin{pgfscope}%
\pgfpathrectangle{\pgfqpoint{0.592976in}{0.451389in}}{\pgfqpoint{3.502277in}{2.786231in}}%
\pgfusepath{clip}%
\pgfsetbuttcap%
\pgfsetroundjoin%
\definecolor{currentfill}{rgb}{0.835294,0.368627,0.000000}%
\pgfsetfillcolor{currentfill}%
\pgfsetfillopacity{0.200000}%
\pgfsetlinewidth{1.003750pt}%
\definecolor{currentstroke}{rgb}{0.835294,0.368627,0.000000}%
\pgfsetstrokecolor{currentstroke}%
\pgfsetstrokeopacity{0.200000}%
\pgfsetdash{}{0pt}%
\pgfsys@defobject{currentmarker}{\pgfqpoint{0.752170in}{1.101498in}}{\pgfqpoint{3.936059in}{1.870320in}}{%
\pgfpathmoveto{\pgfqpoint{0.752170in}{1.113851in}}%
\pgfpathlineto{\pgfqpoint{0.752170in}{1.101498in}}%
\pgfpathlineto{\pgfqpoint{0.776109in}{1.270313in}}%
\pgfpathlineto{\pgfqpoint{0.816008in}{1.285082in}}%
\pgfpathlineto{\pgfqpoint{0.871865in}{1.353428in}}%
\pgfpathlineto{\pgfqpoint{0.943682in}{1.420079in}}%
\pgfpathlineto{\pgfqpoint{1.031459in}{1.470738in}}%
\pgfpathlineto{\pgfqpoint{1.135194in}{1.517682in}}%
\pgfpathlineto{\pgfqpoint{1.254889in}{1.572514in}}%
\pgfpathlineto{\pgfqpoint{1.390544in}{1.602365in}}%
\pgfpathlineto{\pgfqpoint{1.542158in}{1.642583in}}%
\pgfpathlineto{\pgfqpoint{1.893263in}{1.693285in}}%
\pgfpathlineto{\pgfqpoint{2.308206in}{1.743149in}}%
\pgfpathlineto{\pgfqpoint{2.786986in}{1.786978in}}%
\pgfpathlineto{\pgfqpoint{3.329604in}{1.818403in}}%
\pgfpathlineto{\pgfqpoint{3.936059in}{1.854850in}}%
\pgfpathlineto{\pgfqpoint{3.936059in}{1.870320in}}%
\pgfpathlineto{\pgfqpoint{3.936059in}{1.870320in}}%
\pgfpathlineto{\pgfqpoint{3.329604in}{1.840316in}}%
\pgfpathlineto{\pgfqpoint{2.786986in}{1.804826in}}%
\pgfpathlineto{\pgfqpoint{2.308206in}{1.772294in}}%
\pgfpathlineto{\pgfqpoint{1.893263in}{1.728224in}}%
\pgfpathlineto{\pgfqpoint{1.542158in}{1.673505in}}%
\pgfpathlineto{\pgfqpoint{1.390544in}{1.651889in}}%
\pgfpathlineto{\pgfqpoint{1.254889in}{1.621941in}}%
\pgfpathlineto{\pgfqpoint{1.135194in}{1.531338in}}%
\pgfpathlineto{\pgfqpoint{1.031459in}{1.491969in}}%
\pgfpathlineto{\pgfqpoint{0.943682in}{1.429663in}}%
\pgfpathlineto{\pgfqpoint{0.871865in}{1.358333in}}%
\pgfpathlineto{\pgfqpoint{0.816008in}{1.294682in}}%
\pgfpathlineto{\pgfqpoint{0.776109in}{1.306289in}}%
\pgfpathlineto{\pgfqpoint{0.752170in}{1.113851in}}%
\pgfpathlineto{\pgfqpoint{0.752170in}{1.113851in}}%
\pgfpathclose%
\pgfusepath{stroke,fill}%
}%
\begin{pgfscope}%
\pgfsys@transformshift{0.000000in}{0.000000in}%
\pgfsys@useobject{currentmarker}{}%
\end{pgfscope}%
\end{pgfscope}%
\begin{pgfscope}%
\pgfpathrectangle{\pgfqpoint{0.592976in}{0.451389in}}{\pgfqpoint{3.502277in}{2.786231in}}%
\pgfusepath{clip}%
\pgfsetbuttcap%
\pgfsetroundjoin%
\definecolor{currentfill}{rgb}{0.800000,0.470588,0.737255}%
\pgfsetfillcolor{currentfill}%
\pgfsetfillopacity{0.200000}%
\pgfsetlinewidth{1.003750pt}%
\definecolor{currentstroke}{rgb}{0.800000,0.470588,0.737255}%
\pgfsetstrokecolor{currentstroke}%
\pgfsetstrokeopacity{0.200000}%
\pgfsetdash{}{0pt}%
\pgfsys@defobject{currentmarker}{\pgfqpoint{0.752170in}{0.621349in}}{\pgfqpoint{3.936059in}{1.678637in}}{%
\pgfpathmoveto{\pgfqpoint{0.752170in}{0.669295in}}%
\pgfpathlineto{\pgfqpoint{0.752170in}{0.621349in}}%
\pgfpathlineto{\pgfqpoint{0.776109in}{0.718645in}}%
\pgfpathlineto{\pgfqpoint{0.816008in}{0.726335in}}%
\pgfpathlineto{\pgfqpoint{0.871865in}{0.802618in}}%
\pgfpathlineto{\pgfqpoint{0.943682in}{0.861389in}}%
\pgfpathlineto{\pgfqpoint{1.031459in}{0.910879in}}%
\pgfpathlineto{\pgfqpoint{1.135194in}{1.512827in}}%
\pgfpathlineto{\pgfqpoint{1.254889in}{1.445367in}}%
\pgfpathlineto{\pgfqpoint{1.390544in}{1.062935in}}%
\pgfpathlineto{\pgfqpoint{1.542158in}{1.059284in}}%
\pgfpathlineto{\pgfqpoint{1.893263in}{1.114042in}}%
\pgfpathlineto{\pgfqpoint{2.308206in}{1.175706in}}%
\pgfpathlineto{\pgfqpoint{2.786986in}{1.244678in}}%
\pgfpathlineto{\pgfqpoint{3.329604in}{1.297226in}}%
\pgfpathlineto{\pgfqpoint{3.936059in}{1.382902in}}%
\pgfpathlineto{\pgfqpoint{3.936059in}{1.503956in}}%
\pgfpathlineto{\pgfqpoint{3.936059in}{1.503956in}}%
\pgfpathlineto{\pgfqpoint{3.329604in}{1.533249in}}%
\pgfpathlineto{\pgfqpoint{2.786986in}{1.309635in}}%
\pgfpathlineto{\pgfqpoint{2.308206in}{1.380473in}}%
\pgfpathlineto{\pgfqpoint{1.893263in}{1.172061in}}%
\pgfpathlineto{\pgfqpoint{1.542158in}{1.318651in}}%
\pgfpathlineto{\pgfqpoint{1.390544in}{1.185783in}}%
\pgfpathlineto{\pgfqpoint{1.254889in}{1.678637in}}%
\pgfpathlineto{\pgfqpoint{1.135194in}{1.651839in}}%
\pgfpathlineto{\pgfqpoint{1.031459in}{0.953411in}}%
\pgfpathlineto{\pgfqpoint{0.943682in}{0.918042in}}%
\pgfpathlineto{\pgfqpoint{0.871865in}{0.824964in}}%
\pgfpathlineto{\pgfqpoint{0.816008in}{0.764727in}}%
\pgfpathlineto{\pgfqpoint{0.776109in}{0.783372in}}%
\pgfpathlineto{\pgfqpoint{0.752170in}{0.669295in}}%
\pgfpathlineto{\pgfqpoint{0.752170in}{0.669295in}}%
\pgfpathclose%
\pgfusepath{stroke,fill}%
}%
\begin{pgfscope}%
\pgfsys@transformshift{0.000000in}{0.000000in}%
\pgfsys@useobject{currentmarker}{}%
\end{pgfscope}%
\end{pgfscope}%
\begin{pgfscope}%
\pgfpathrectangle{\pgfqpoint{0.592976in}{0.451389in}}{\pgfqpoint{3.502277in}{2.786231in}}%
\pgfusepath{clip}%
\pgfsetbuttcap%
\pgfsetroundjoin%
\definecolor{currentfill}{rgb}{0.792157,0.568627,0.380392}%
\pgfsetfillcolor{currentfill}%
\pgfsetfillopacity{0.200000}%
\pgfsetlinewidth{1.003750pt}%
\definecolor{currentstroke}{rgb}{0.792157,0.568627,0.380392}%
\pgfsetstrokecolor{currentstroke}%
\pgfsetstrokeopacity{0.200000}%
\pgfsetdash{}{0pt}%
\pgfsys@defobject{currentmarker}{\pgfqpoint{0.752170in}{0.578035in}}{\pgfqpoint{3.936059in}{1.580442in}}{%
\pgfpathmoveto{\pgfqpoint{0.752170in}{0.643478in}}%
\pgfpathlineto{\pgfqpoint{0.752170in}{0.578035in}}%
\pgfpathlineto{\pgfqpoint{0.776109in}{0.716659in}}%
\pgfpathlineto{\pgfqpoint{0.816008in}{0.713971in}}%
\pgfpathlineto{\pgfqpoint{0.871865in}{0.808523in}}%
\pgfpathlineto{\pgfqpoint{0.943682in}{0.905275in}}%
\pgfpathlineto{\pgfqpoint{1.031459in}{0.977152in}}%
\pgfpathlineto{\pgfqpoint{1.135194in}{1.018325in}}%
\pgfpathlineto{\pgfqpoint{1.254889in}{1.068830in}}%
\pgfpathlineto{\pgfqpoint{1.390544in}{1.094622in}}%
\pgfpathlineto{\pgfqpoint{1.542158in}{1.124249in}}%
\pgfpathlineto{\pgfqpoint{1.893263in}{1.177807in}}%
\pgfpathlineto{\pgfqpoint{2.308206in}{1.227426in}}%
\pgfpathlineto{\pgfqpoint{2.786986in}{1.268106in}}%
\pgfpathlineto{\pgfqpoint{3.329604in}{1.316496in}}%
\pgfpathlineto{\pgfqpoint{3.936059in}{1.359120in}}%
\pgfpathlineto{\pgfqpoint{3.936059in}{1.580442in}}%
\pgfpathlineto{\pgfqpoint{3.936059in}{1.580442in}}%
\pgfpathlineto{\pgfqpoint{3.329604in}{1.554519in}}%
\pgfpathlineto{\pgfqpoint{2.786986in}{1.488978in}}%
\pgfpathlineto{\pgfqpoint{2.308206in}{1.442961in}}%
\pgfpathlineto{\pgfqpoint{1.893263in}{1.194338in}}%
\pgfpathlineto{\pgfqpoint{1.542158in}{1.362929in}}%
\pgfpathlineto{\pgfqpoint{1.390544in}{1.254049in}}%
\pgfpathlineto{\pgfqpoint{1.254889in}{1.101705in}}%
\pgfpathlineto{\pgfqpoint{1.135194in}{1.038474in}}%
\pgfpathlineto{\pgfqpoint{1.031459in}{0.988630in}}%
\pgfpathlineto{\pgfqpoint{0.943682in}{0.942072in}}%
\pgfpathlineto{\pgfqpoint{0.871865in}{0.868602in}}%
\pgfpathlineto{\pgfqpoint{0.816008in}{0.786755in}}%
\pgfpathlineto{\pgfqpoint{0.776109in}{0.765692in}}%
\pgfpathlineto{\pgfqpoint{0.752170in}{0.643478in}}%
\pgfpathlineto{\pgfqpoint{0.752170in}{0.643478in}}%
\pgfpathclose%
\pgfusepath{stroke,fill}%
}%
\begin{pgfscope}%
\pgfsys@transformshift{0.000000in}{0.000000in}%
\pgfsys@useobject{currentmarker}{}%
\end{pgfscope}%
\end{pgfscope}%
\begin{pgfscope}%
\pgfsetrectcap%
\pgfsetmiterjoin%
\pgfsetlinewidth{1.254687pt}%
\definecolor{currentstroke}{rgb}{0.800000,0.800000,0.800000}%
\pgfsetstrokecolor{currentstroke}%
\pgfsetdash{}{0pt}%
\pgfpathmoveto{\pgfqpoint{0.592976in}{0.451389in}}%
\pgfpathlineto{\pgfqpoint{0.592976in}{3.237620in}}%
\pgfusepath{stroke}%
\end{pgfscope}%
\begin{pgfscope}%
\pgfsetrectcap%
\pgfsetmiterjoin%
\pgfsetlinewidth{1.254687pt}%
\definecolor{currentstroke}{rgb}{0.800000,0.800000,0.800000}%
\pgfsetstrokecolor{currentstroke}%
\pgfsetdash{}{0pt}%
\pgfpathmoveto{\pgfqpoint{4.095253in}{0.451389in}}%
\pgfpathlineto{\pgfqpoint{4.095253in}{3.237620in}}%
\pgfusepath{stroke}%
\end{pgfscope}%
\begin{pgfscope}%
\pgfsetrectcap%
\pgfsetmiterjoin%
\pgfsetlinewidth{1.254687pt}%
\definecolor{currentstroke}{rgb}{0.800000,0.800000,0.800000}%
\pgfsetstrokecolor{currentstroke}%
\pgfsetdash{}{0pt}%
\pgfpathmoveto{\pgfqpoint{0.592976in}{0.451389in}}%
\pgfpathlineto{\pgfqpoint{4.095253in}{0.451389in}}%
\pgfusepath{stroke}%
\end{pgfscope}%
\begin{pgfscope}%
\pgfsetrectcap%
\pgfsetmiterjoin%
\pgfsetlinewidth{1.254687pt}%
\definecolor{currentstroke}{rgb}{0.800000,0.800000,0.800000}%
\pgfsetstrokecolor{currentstroke}%
\pgfsetdash{}{0pt}%
\pgfpathmoveto{\pgfqpoint{0.592976in}{3.237620in}}%
\pgfpathlineto{\pgfqpoint{4.095253in}{3.237620in}}%
\pgfusepath{stroke}%
\end{pgfscope}%
\begin{pgfscope}%
\pgfsetbuttcap%
\pgfsetmiterjoin%
\definecolor{currentfill}{rgb}{1.000000,1.000000,1.000000}%
\pgfsetfillcolor{currentfill}%
\pgfsetfillopacity{0.800000}%
\pgfsetlinewidth{1.003750pt}%
\definecolor{currentstroke}{rgb}{0.800000,0.800000,0.800000}%
\pgfsetstrokecolor{currentstroke}%
\pgfsetstrokeopacity{0.800000}%
\pgfsetdash{}{0pt}%
\pgfpathmoveto{\pgfqpoint{4.270310in}{0.521129in}}%
\pgfpathlineto{\pgfqpoint{5.720359in}{0.521129in}}%
\pgfpathquadraticcurveto{\pgfqpoint{5.745359in}{0.521129in}}{\pgfqpoint{5.745359in}{0.546129in}}%
\pgfpathlineto{\pgfqpoint{5.745359in}{3.142880in}}%
\pgfpathquadraticcurveto{\pgfqpoint{5.745359in}{3.167880in}}{\pgfqpoint{5.720359in}{3.167880in}}%
\pgfpathlineto{\pgfqpoint{4.270310in}{3.167880in}}%
\pgfpathquadraticcurveto{\pgfqpoint{4.245310in}{3.167880in}}{\pgfqpoint{4.245310in}{3.142880in}}%
\pgfpathlineto{\pgfqpoint{4.245310in}{0.546129in}}%
\pgfpathquadraticcurveto{\pgfqpoint{4.245310in}{0.521129in}}{\pgfqpoint{4.270310in}{0.521129in}}%
\pgfpathlineto{\pgfqpoint{4.270310in}{0.521129in}}%
\pgfpathclose%
\pgfusepath{stroke,fill}%
\end{pgfscope}%
\begin{pgfscope}%
\definecolor{textcolor}{rgb}{0.150000,0.150000,0.150000}%
\pgfsetstrokecolor{textcolor}%
\pgfsetfillcolor{textcolor}%
\pgftext[x=4.791935in,y=3.022909in,left,base]{\color{textcolor}\sffamily\fontsize{9.000000}{10.800000}\selectfont Legend}%
\end{pgfscope}%
\begin{pgfscope}%
\pgfsetroundcap%
\pgfsetroundjoin%
\pgfsetlinewidth{1.505625pt}%
\definecolor{currentstroke}{rgb}{0.003922,0.450980,0.698039}%
\pgfsetstrokecolor{currentstroke}%
\pgfsetdash{}{0pt}%
\pgfpathmoveto{\pgfqpoint{4.295310in}{2.879159in}}%
\pgfpathlineto{\pgfqpoint{4.420310in}{2.879159in}}%
\pgfpathlineto{\pgfqpoint{4.545310in}{2.879159in}}%
\pgfusepath{stroke}%
\end{pgfscope}%
\begin{pgfscope}%
\definecolor{textcolor}{rgb}{0.150000,0.150000,0.150000}%
\pgfsetstrokecolor{textcolor}%
\pgfsetfillcolor{textcolor}%
\pgftext[x=4.645310in,y=2.835409in,left,base]{\color{textcolor}\sffamily\fontsize{9.000000}{10.800000}\selectfont Total time}%
\end{pgfscope}%
\begin{pgfscope}%
\pgfsetroundcap%
\pgfsetroundjoin%
\pgfsetlinewidth{1.505625pt}%
\definecolor{currentstroke}{rgb}{0.870588,0.560784,0.019608}%
\pgfsetstrokecolor{currentstroke}%
\pgfsetdash{}{0pt}%
\pgfpathmoveto{\pgfqpoint{4.295310in}{2.532651in}}%
\pgfpathlineto{\pgfqpoint{4.420310in}{2.532651in}}%
\pgfpathlineto{\pgfqpoint{4.545310in}{2.532651in}}%
\pgfusepath{stroke}%
\end{pgfscope}%
\begin{pgfscope}%
\definecolor{textcolor}{rgb}{0.150000,0.150000,0.150000}%
\pgfsetstrokecolor{textcolor}%
\pgfsetfillcolor{textcolor}%
\pgftext[x=4.645310in, y=2.647909in, left, base]{\color{textcolor}\sffamily\fontsize{9.000000}{10.800000}\selectfont Find all \(\displaystyle k\)}%
\end{pgfscope}%
\begin{pgfscope}%
\definecolor{textcolor}{rgb}{0.150000,0.150000,0.150000}%
\pgfsetstrokecolor{textcolor}%
\pgfsetfillcolor{textcolor}%
\pgftext[x=4.645310in, y=2.503915in, left, base]{\color{textcolor}\sffamily\fontsize{9.000000}{10.800000}\selectfont nearest visibility}%
\end{pgfscope}%
\begin{pgfscope}%
\definecolor{textcolor}{rgb}{0.150000,0.150000,0.150000}%
\pgfsetstrokecolor{textcolor}%
\pgfsetfillcolor{textcolor}%
\pgftext[x=4.645310in, y=2.359921in, left, base]{\color{textcolor}\sffamily\fontsize{9.000000}{10.800000}\selectfont neighbors}%
\end{pgfscope}%
\begin{pgfscope}%
\pgfsetroundcap%
\pgfsetroundjoin%
\pgfsetlinewidth{1.505625pt}%
\definecolor{currentstroke}{rgb}{0.007843,0.619608,0.450980}%
\pgfsetstrokecolor{currentstroke}%
\pgfsetdash{}{0pt}%
\pgfpathmoveto{\pgfqpoint{4.295310in}{2.056760in}}%
\pgfpathlineto{\pgfqpoint{4.420310in}{2.056760in}}%
\pgfpathlineto{\pgfqpoint{4.545310in}{2.056760in}}%
\pgfusepath{stroke}%
\end{pgfscope}%
\begin{pgfscope}%
\definecolor{textcolor}{rgb}{0.150000,0.150000,0.150000}%
\pgfsetstrokecolor{textcolor}%
\pgfsetfillcolor{textcolor}%
\pgftext[x=4.645310in, y=2.172421in, left, base]{\color{textcolor}\sffamily\fontsize{9.000000}{10.800000}\selectfont Building the}%
\end{pgfscope}%
\begin{pgfscope}%
\definecolor{textcolor}{rgb}{0.150000,0.150000,0.150000}%
\pgfsetstrokecolor{textcolor}%
\pgfsetfillcolor{textcolor}%
\pgftext[x=4.645310in, y=2.027622in, left, base]{\color{textcolor}\sffamily\fontsize{9.000000}{10.800000}\selectfont graph from}%
\end{pgfscope}%
\begin{pgfscope}%
\definecolor{textcolor}{rgb}{0.150000,0.150000,0.150000}%
\pgfsetstrokecolor{textcolor}%
\pgfsetfillcolor{textcolor}%
\pgftext[x=4.645310in, y=1.883628in, left, base]{\color{textcolor}\sffamily\fontsize{9.000000}{10.800000}\selectfont visibility neighbors}%
\end{pgfscope}%
\begin{pgfscope}%
\pgfsetroundcap%
\pgfsetroundjoin%
\pgfsetlinewidth{1.505625pt}%
\definecolor{currentstroke}{rgb}{0.835294,0.368627,0.000000}%
\pgfsetstrokecolor{currentstroke}%
\pgfsetdash{}{0pt}%
\pgfpathmoveto{\pgfqpoint{4.295310in}{1.580869in}}%
\pgfpathlineto{\pgfqpoint{4.420310in}{1.580869in}}%
\pgfpathlineto{\pgfqpoint{4.545310in}{1.580869in}}%
\pgfusepath{stroke}%
\end{pgfscope}%
\begin{pgfscope}%
\definecolor{textcolor}{rgb}{0.150000,0.150000,0.150000}%
\pgfsetstrokecolor{textcolor}%
\pgfsetfillcolor{textcolor}%
\pgftext[x=4.645310in, y=1.696128in, left, base]{\color{textcolor}\sffamily\fontsize{9.000000}{10.800000}\selectfont Getting and}%
\end{pgfscope}%
\begin{pgfscope}%
\definecolor{textcolor}{rgb}{0.150000,0.150000,0.150000}%
\pgfsetstrokecolor{textcolor}%
\pgfsetfillcolor{textcolor}%
\pgftext[x=4.645310in, y=1.552134in, left, base]{\color{textcolor}\sffamily\fontsize{9.000000}{10.800000}\selectfont preparing all}%
\end{pgfscope}%
\begin{pgfscope}%
\definecolor{textcolor}{rgb}{0.150000,0.150000,0.150000}%
\pgfsetstrokecolor{textcolor}%
\pgfsetfillcolor{textcolor}%
\pgftext[x=4.645310in, y=1.408140in, left, base]{\color{textcolor}\sffamily\fontsize{9.000000}{10.800000}\selectfont obstacles}%
\end{pgfscope}%
\begin{pgfscope}%
\pgfsetroundcap%
\pgfsetroundjoin%
\pgfsetlinewidth{1.505625pt}%
\definecolor{currentstroke}{rgb}{0.800000,0.470588,0.737255}%
\pgfsetstrokecolor{currentstroke}%
\pgfsetdash{}{0pt}%
\pgfpathmoveto{\pgfqpoint{4.295310in}{1.105381in}}%
\pgfpathlineto{\pgfqpoint{4.420310in}{1.105381in}}%
\pgfpathlineto{\pgfqpoint{4.545310in}{1.105381in}}%
\pgfusepath{stroke}%
\end{pgfscope}%
\begin{pgfscope}%
\definecolor{textcolor}{rgb}{0.150000,0.150000,0.150000}%
\pgfsetstrokecolor{textcolor}%
\pgfsetfillcolor{textcolor}%
\pgftext[x=4.645310in, y=1.220640in, left, base]{\color{textcolor}\sffamily\fontsize{9.000000}{10.800000}\selectfont Merge V-Graph}%
\end{pgfscope}%
\begin{pgfscope}%
\definecolor{textcolor}{rgb}{0.150000,0.150000,0.150000}%
\pgfsetstrokecolor{textcolor}%
\pgfsetfillcolor{textcolor}%
\pgftext[x=4.645310in, y=1.076646in, left, base]{\color{textcolor}\sffamily\fontsize{9.000000}{10.800000}\selectfont into the}%
\end{pgfscope}%
\begin{pgfscope}%
\definecolor{textcolor}{rgb}{0.150000,0.150000,0.150000}%
\pgfsetstrokecolor{textcolor}%
\pgfsetfillcolor{textcolor}%
\pgftext[x=4.645310in, y=0.932652in, left, base]{\color{textcolor}\sffamily\fontsize{9.000000}{10.800000}\selectfont road graph}%
\end{pgfscope}%
\begin{pgfscope}%
\pgfsetroundcap%
\pgfsetroundjoin%
\pgfsetlinewidth{1.505625pt}%
\definecolor{currentstroke}{rgb}{0.792157,0.568627,0.380392}%
\pgfsetstrokecolor{currentstroke}%
\pgfsetdash{}{0pt}%
\pgfpathmoveto{\pgfqpoint{4.295310in}{0.701890in}}%
\pgfpathlineto{\pgfqpoint{4.420310in}{0.701890in}}%
\pgfpathlineto{\pgfqpoint{4.545310in}{0.701890in}}%
\pgfusepath{stroke}%
\end{pgfscope}%
\begin{pgfscope}%
\definecolor{textcolor}{rgb}{0.150000,0.150000,0.150000}%
\pgfsetstrokecolor{textcolor}%
\pgfsetfillcolor{textcolor}%
\pgftext[x=4.645310in, y=0.745152in, left, base]{\color{textcolor}\sffamily\fontsize{9.000000}{10.800000}\selectfont Add attributed}%
\end{pgfscope}%
\begin{pgfscope}%
\definecolor{textcolor}{rgb}{0.150000,0.150000,0.150000}%
\pgfsetstrokecolor{textcolor}%
\pgfsetfillcolor{textcolor}%
\pgftext[x=4.645310in, y=0.601158in, left, base]{\color{textcolor}\sffamily\fontsize{9.000000}{10.800000}\selectfont to POIs}%
\end{pgfscope}%
\begin{pgfscope}%
\pgfsetroundcap%
\pgfsetroundjoin%
\pgfsetlinewidth{1.003750pt}%
\definecolor{currentstroke}{rgb}{0.003922,0.450980,0.698039}%
\pgfsetstrokecolor{currentstroke}%
\pgfsetdash{}{0pt}%
\pgfpathmoveto{\pgfqpoint{0.752170in}{1.549605in}}%
\pgfpathlineto{\pgfqpoint{0.776109in}{1.766958in}}%
\pgfpathlineto{\pgfqpoint{0.816008in}{1.936892in}}%
\pgfpathlineto{\pgfqpoint{0.871865in}{2.108379in}}%
\pgfpathlineto{\pgfqpoint{0.943682in}{2.230863in}}%
\pgfpathlineto{\pgfqpoint{1.031459in}{2.341881in}}%
\pgfpathlineto{\pgfqpoint{1.135194in}{2.418664in}}%
\pgfpathlineto{\pgfqpoint{1.254889in}{2.523568in}}%
\pgfpathlineto{\pgfqpoint{1.390544in}{2.563022in}}%
\pgfpathlineto{\pgfqpoint{1.542158in}{2.636859in}}%
\pgfpathlineto{\pgfqpoint{1.893263in}{2.766384in}}%
\pgfpathlineto{\pgfqpoint{2.308206in}{2.845198in}}%
\pgfpathlineto{\pgfqpoint{2.786986in}{2.984851in}}%
\pgfpathlineto{\pgfqpoint{3.329604in}{3.012255in}}%
\pgfpathlineto{\pgfqpoint{3.936059in}{3.110472in}}%
\pgfusepath{stroke}%
\end{pgfscope}%
\begin{pgfscope}%
\pgfsetbuttcap%
\pgfsetroundjoin%
\definecolor{currentfill}{rgb}{0.003922,0.450980,0.698039}%
\pgfsetfillcolor{currentfill}%
\pgfsetlinewidth{0.752812pt}%
\definecolor{currentstroke}{rgb}{1.000000,1.000000,1.000000}%
\pgfsetstrokecolor{currentstroke}%
\pgfsetdash{}{0pt}%
\pgfsys@defobject{currentmarker}{\pgfqpoint{-0.034722in}{-0.034722in}}{\pgfqpoint{0.034722in}{0.034722in}}{%
\pgfpathmoveto{\pgfqpoint{0.000000in}{-0.034722in}}%
\pgfpathcurveto{\pgfqpoint{0.009208in}{-0.034722in}}{\pgfqpoint{0.018041in}{-0.031064in}}{\pgfqpoint{0.024552in}{-0.024552in}}%
\pgfpathcurveto{\pgfqpoint{0.031064in}{-0.018041in}}{\pgfqpoint{0.034722in}{-0.009208in}}{\pgfqpoint{0.034722in}{0.000000in}}%
\pgfpathcurveto{\pgfqpoint{0.034722in}{0.009208in}}{\pgfqpoint{0.031064in}{0.018041in}}{\pgfqpoint{0.024552in}{0.024552in}}%
\pgfpathcurveto{\pgfqpoint{0.018041in}{0.031064in}}{\pgfqpoint{0.009208in}{0.034722in}}{\pgfqpoint{0.000000in}{0.034722in}}%
\pgfpathcurveto{\pgfqpoint{-0.009208in}{0.034722in}}{\pgfqpoint{-0.018041in}{0.031064in}}{\pgfqpoint{-0.024552in}{0.024552in}}%
\pgfpathcurveto{\pgfqpoint{-0.031064in}{0.018041in}}{\pgfqpoint{-0.034722in}{0.009208in}}{\pgfqpoint{-0.034722in}{0.000000in}}%
\pgfpathcurveto{\pgfqpoint{-0.034722in}{-0.009208in}}{\pgfqpoint{-0.031064in}{-0.018041in}}{\pgfqpoint{-0.024552in}{-0.024552in}}%
\pgfpathcurveto{\pgfqpoint{-0.018041in}{-0.031064in}}{\pgfqpoint{-0.009208in}{-0.034722in}}{\pgfqpoint{0.000000in}{-0.034722in}}%
\pgfpathlineto{\pgfqpoint{0.000000in}{-0.034722in}}%
\pgfpathclose%
\pgfusepath{stroke,fill}%
}%
\begin{pgfscope}%
\pgfsys@transformshift{0.752170in}{1.549605in}%
\pgfsys@useobject{currentmarker}{}%
\end{pgfscope}%
\begin{pgfscope}%
\pgfsys@transformshift{0.776109in}{1.766958in}%
\pgfsys@useobject{currentmarker}{}%
\end{pgfscope}%
\begin{pgfscope}%
\pgfsys@transformshift{0.816008in}{1.936892in}%
\pgfsys@useobject{currentmarker}{}%
\end{pgfscope}%
\begin{pgfscope}%
\pgfsys@transformshift{0.871865in}{2.108379in}%
\pgfsys@useobject{currentmarker}{}%
\end{pgfscope}%
\begin{pgfscope}%
\pgfsys@transformshift{0.943682in}{2.230863in}%
\pgfsys@useobject{currentmarker}{}%
\end{pgfscope}%
\begin{pgfscope}%
\pgfsys@transformshift{1.031459in}{2.341881in}%
\pgfsys@useobject{currentmarker}{}%
\end{pgfscope}%
\begin{pgfscope}%
\pgfsys@transformshift{1.135194in}{2.418664in}%
\pgfsys@useobject{currentmarker}{}%
\end{pgfscope}%
\begin{pgfscope}%
\pgfsys@transformshift{1.254889in}{2.523568in}%
\pgfsys@useobject{currentmarker}{}%
\end{pgfscope}%
\begin{pgfscope}%
\pgfsys@transformshift{1.390544in}{2.563022in}%
\pgfsys@useobject{currentmarker}{}%
\end{pgfscope}%
\begin{pgfscope}%
\pgfsys@transformshift{1.542158in}{2.636859in}%
\pgfsys@useobject{currentmarker}{}%
\end{pgfscope}%
\begin{pgfscope}%
\pgfsys@transformshift{1.893263in}{2.766384in}%
\pgfsys@useobject{currentmarker}{}%
\end{pgfscope}%
\begin{pgfscope}%
\pgfsys@transformshift{2.308206in}{2.845198in}%
\pgfsys@useobject{currentmarker}{}%
\end{pgfscope}%
\begin{pgfscope}%
\pgfsys@transformshift{2.786986in}{2.984851in}%
\pgfsys@useobject{currentmarker}{}%
\end{pgfscope}%
\begin{pgfscope}%
\pgfsys@transformshift{3.329604in}{3.012255in}%
\pgfsys@useobject{currentmarker}{}%
\end{pgfscope}%
\begin{pgfscope}%
\pgfsys@transformshift{3.936059in}{3.110472in}%
\pgfsys@useobject{currentmarker}{}%
\end{pgfscope}%
\end{pgfscope}%
\begin{pgfscope}%
\pgfsetroundcap%
\pgfsetroundjoin%
\pgfsetlinewidth{1.003750pt}%
\definecolor{currentstroke}{rgb}{0.870588,0.560784,0.019608}%
\pgfsetstrokecolor{currentstroke}%
\pgfsetdash{}{0pt}%
\pgfpathmoveto{\pgfqpoint{0.752170in}{1.503219in}}%
\pgfpathlineto{\pgfqpoint{0.776109in}{1.733234in}}%
\pgfpathlineto{\pgfqpoint{0.816008in}{1.923975in}}%
\pgfpathlineto{\pgfqpoint{0.871865in}{2.100919in}}%
\pgfpathlineto{\pgfqpoint{0.943682in}{2.223038in}}%
\pgfpathlineto{\pgfqpoint{1.031459in}{2.335470in}}%
\pgfpathlineto{\pgfqpoint{1.135194in}{2.412366in}}%
\pgfpathlineto{\pgfqpoint{1.254889in}{2.519408in}}%
\pgfpathlineto{\pgfqpoint{1.390544in}{2.558856in}}%
\pgfpathlineto{\pgfqpoint{1.542158in}{2.633705in}}%
\pgfpathlineto{\pgfqpoint{1.893263in}{2.764493in}}%
\pgfpathlineto{\pgfqpoint{2.308206in}{2.843588in}}%
\pgfpathlineto{\pgfqpoint{2.786986in}{2.984003in}}%
\pgfpathlineto{\pgfqpoint{3.329604in}{3.011419in}}%
\pgfpathlineto{\pgfqpoint{3.936059in}{3.109888in}}%
\pgfusepath{stroke}%
\end{pgfscope}%
\begin{pgfscope}%
\pgfsetbuttcap%
\pgfsetroundjoin%
\definecolor{currentfill}{rgb}{0.870588,0.560784,0.019608}%
\pgfsetfillcolor{currentfill}%
\pgfsetlinewidth{0.752812pt}%
\definecolor{currentstroke}{rgb}{1.000000,1.000000,1.000000}%
\pgfsetstrokecolor{currentstroke}%
\pgfsetdash{}{0pt}%
\pgfsys@defobject{currentmarker}{\pgfqpoint{-0.034722in}{-0.034722in}}{\pgfqpoint{0.034722in}{0.034722in}}{%
\pgfpathmoveto{\pgfqpoint{0.000000in}{-0.034722in}}%
\pgfpathcurveto{\pgfqpoint{0.009208in}{-0.034722in}}{\pgfqpoint{0.018041in}{-0.031064in}}{\pgfqpoint{0.024552in}{-0.024552in}}%
\pgfpathcurveto{\pgfqpoint{0.031064in}{-0.018041in}}{\pgfqpoint{0.034722in}{-0.009208in}}{\pgfqpoint{0.034722in}{0.000000in}}%
\pgfpathcurveto{\pgfqpoint{0.034722in}{0.009208in}}{\pgfqpoint{0.031064in}{0.018041in}}{\pgfqpoint{0.024552in}{0.024552in}}%
\pgfpathcurveto{\pgfqpoint{0.018041in}{0.031064in}}{\pgfqpoint{0.009208in}{0.034722in}}{\pgfqpoint{0.000000in}{0.034722in}}%
\pgfpathcurveto{\pgfqpoint{-0.009208in}{0.034722in}}{\pgfqpoint{-0.018041in}{0.031064in}}{\pgfqpoint{-0.024552in}{0.024552in}}%
\pgfpathcurveto{\pgfqpoint{-0.031064in}{0.018041in}}{\pgfqpoint{-0.034722in}{0.009208in}}{\pgfqpoint{-0.034722in}{0.000000in}}%
\pgfpathcurveto{\pgfqpoint{-0.034722in}{-0.009208in}}{\pgfqpoint{-0.031064in}{-0.018041in}}{\pgfqpoint{-0.024552in}{-0.024552in}}%
\pgfpathcurveto{\pgfqpoint{-0.018041in}{-0.031064in}}{\pgfqpoint{-0.009208in}{-0.034722in}}{\pgfqpoint{0.000000in}{-0.034722in}}%
\pgfpathlineto{\pgfqpoint{0.000000in}{-0.034722in}}%
\pgfpathclose%
\pgfusepath{stroke,fill}%
}%
\begin{pgfscope}%
\pgfsys@transformshift{0.752170in}{1.503219in}%
\pgfsys@useobject{currentmarker}{}%
\end{pgfscope}%
\begin{pgfscope}%
\pgfsys@transformshift{0.776109in}{1.733234in}%
\pgfsys@useobject{currentmarker}{}%
\end{pgfscope}%
\begin{pgfscope}%
\pgfsys@transformshift{0.816008in}{1.923975in}%
\pgfsys@useobject{currentmarker}{}%
\end{pgfscope}%
\begin{pgfscope}%
\pgfsys@transformshift{0.871865in}{2.100919in}%
\pgfsys@useobject{currentmarker}{}%
\end{pgfscope}%
\begin{pgfscope}%
\pgfsys@transformshift{0.943682in}{2.223038in}%
\pgfsys@useobject{currentmarker}{}%
\end{pgfscope}%
\begin{pgfscope}%
\pgfsys@transformshift{1.031459in}{2.335470in}%
\pgfsys@useobject{currentmarker}{}%
\end{pgfscope}%
\begin{pgfscope}%
\pgfsys@transformshift{1.135194in}{2.412366in}%
\pgfsys@useobject{currentmarker}{}%
\end{pgfscope}%
\begin{pgfscope}%
\pgfsys@transformshift{1.254889in}{2.519408in}%
\pgfsys@useobject{currentmarker}{}%
\end{pgfscope}%
\begin{pgfscope}%
\pgfsys@transformshift{1.390544in}{2.558856in}%
\pgfsys@useobject{currentmarker}{}%
\end{pgfscope}%
\begin{pgfscope}%
\pgfsys@transformshift{1.542158in}{2.633705in}%
\pgfsys@useobject{currentmarker}{}%
\end{pgfscope}%
\begin{pgfscope}%
\pgfsys@transformshift{1.893263in}{2.764493in}%
\pgfsys@useobject{currentmarker}{}%
\end{pgfscope}%
\begin{pgfscope}%
\pgfsys@transformshift{2.308206in}{2.843588in}%
\pgfsys@useobject{currentmarker}{}%
\end{pgfscope}%
\begin{pgfscope}%
\pgfsys@transformshift{2.786986in}{2.984003in}%
\pgfsys@useobject{currentmarker}{}%
\end{pgfscope}%
\begin{pgfscope}%
\pgfsys@transformshift{3.329604in}{3.011419in}%
\pgfsys@useobject{currentmarker}{}%
\end{pgfscope}%
\begin{pgfscope}%
\pgfsys@transformshift{3.936059in}{3.109888in}%
\pgfsys@useobject{currentmarker}{}%
\end{pgfscope}%
\end{pgfscope}%
\begin{pgfscope}%
\pgfsetroundcap%
\pgfsetroundjoin%
\pgfsetlinewidth{1.003750pt}%
\definecolor{currentstroke}{rgb}{0.007843,0.619608,0.450980}%
\pgfsetstrokecolor{currentstroke}%
\pgfsetdash{}{0pt}%
\pgfpathmoveto{\pgfqpoint{0.752170in}{1.242383in}}%
\pgfpathlineto{\pgfqpoint{0.776109in}{1.445955in}}%
\pgfpathlineto{\pgfqpoint{0.816008in}{1.493767in}}%
\pgfpathlineto{\pgfqpoint{0.871865in}{1.594417in}}%
\pgfpathlineto{\pgfqpoint{0.943682in}{1.746145in}}%
\pgfpathlineto{\pgfqpoint{1.031459in}{1.833412in}}%
\pgfpathlineto{\pgfqpoint{1.135194in}{1.884095in}}%
\pgfpathlineto{\pgfqpoint{1.254889in}{1.928281in}}%
\pgfpathlineto{\pgfqpoint{1.390544in}{1.988111in}}%
\pgfpathlineto{\pgfqpoint{1.542158in}{2.024397in}}%
\pgfpathlineto{\pgfqpoint{1.893263in}{2.079852in}}%
\pgfpathlineto{\pgfqpoint{2.308206in}{2.134182in}}%
\pgfpathlineto{\pgfqpoint{2.786986in}{2.180151in}}%
\pgfpathlineto{\pgfqpoint{3.329604in}{2.202290in}}%
\pgfpathlineto{\pgfqpoint{3.936059in}{2.248868in}}%
\pgfusepath{stroke}%
\end{pgfscope}%
\begin{pgfscope}%
\pgfsetbuttcap%
\pgfsetroundjoin%
\definecolor{currentfill}{rgb}{0.007843,0.619608,0.450980}%
\pgfsetfillcolor{currentfill}%
\pgfsetlinewidth{0.752812pt}%
\definecolor{currentstroke}{rgb}{1.000000,1.000000,1.000000}%
\pgfsetstrokecolor{currentstroke}%
\pgfsetdash{}{0pt}%
\pgfsys@defobject{currentmarker}{\pgfqpoint{-0.034722in}{-0.034722in}}{\pgfqpoint{0.034722in}{0.034722in}}{%
\pgfpathmoveto{\pgfqpoint{0.000000in}{-0.034722in}}%
\pgfpathcurveto{\pgfqpoint{0.009208in}{-0.034722in}}{\pgfqpoint{0.018041in}{-0.031064in}}{\pgfqpoint{0.024552in}{-0.024552in}}%
\pgfpathcurveto{\pgfqpoint{0.031064in}{-0.018041in}}{\pgfqpoint{0.034722in}{-0.009208in}}{\pgfqpoint{0.034722in}{0.000000in}}%
\pgfpathcurveto{\pgfqpoint{0.034722in}{0.009208in}}{\pgfqpoint{0.031064in}{0.018041in}}{\pgfqpoint{0.024552in}{0.024552in}}%
\pgfpathcurveto{\pgfqpoint{0.018041in}{0.031064in}}{\pgfqpoint{0.009208in}{0.034722in}}{\pgfqpoint{0.000000in}{0.034722in}}%
\pgfpathcurveto{\pgfqpoint{-0.009208in}{0.034722in}}{\pgfqpoint{-0.018041in}{0.031064in}}{\pgfqpoint{-0.024552in}{0.024552in}}%
\pgfpathcurveto{\pgfqpoint{-0.031064in}{0.018041in}}{\pgfqpoint{-0.034722in}{0.009208in}}{\pgfqpoint{-0.034722in}{0.000000in}}%
\pgfpathcurveto{\pgfqpoint{-0.034722in}{-0.009208in}}{\pgfqpoint{-0.031064in}{-0.018041in}}{\pgfqpoint{-0.024552in}{-0.024552in}}%
\pgfpathcurveto{\pgfqpoint{-0.018041in}{-0.031064in}}{\pgfqpoint{-0.009208in}{-0.034722in}}{\pgfqpoint{0.000000in}{-0.034722in}}%
\pgfpathlineto{\pgfqpoint{0.000000in}{-0.034722in}}%
\pgfpathclose%
\pgfusepath{stroke,fill}%
}%
\begin{pgfscope}%
\pgfsys@transformshift{0.752170in}{1.242383in}%
\pgfsys@useobject{currentmarker}{}%
\end{pgfscope}%
\begin{pgfscope}%
\pgfsys@transformshift{0.776109in}{1.445955in}%
\pgfsys@useobject{currentmarker}{}%
\end{pgfscope}%
\begin{pgfscope}%
\pgfsys@transformshift{0.816008in}{1.493767in}%
\pgfsys@useobject{currentmarker}{}%
\end{pgfscope}%
\begin{pgfscope}%
\pgfsys@transformshift{0.871865in}{1.594417in}%
\pgfsys@useobject{currentmarker}{}%
\end{pgfscope}%
\begin{pgfscope}%
\pgfsys@transformshift{0.943682in}{1.746145in}%
\pgfsys@useobject{currentmarker}{}%
\end{pgfscope}%
\begin{pgfscope}%
\pgfsys@transformshift{1.031459in}{1.833412in}%
\pgfsys@useobject{currentmarker}{}%
\end{pgfscope}%
\begin{pgfscope}%
\pgfsys@transformshift{1.135194in}{1.884095in}%
\pgfsys@useobject{currentmarker}{}%
\end{pgfscope}%
\begin{pgfscope}%
\pgfsys@transformshift{1.254889in}{1.928281in}%
\pgfsys@useobject{currentmarker}{}%
\end{pgfscope}%
\begin{pgfscope}%
\pgfsys@transformshift{1.390544in}{1.988111in}%
\pgfsys@useobject{currentmarker}{}%
\end{pgfscope}%
\begin{pgfscope}%
\pgfsys@transformshift{1.542158in}{2.024397in}%
\pgfsys@useobject{currentmarker}{}%
\end{pgfscope}%
\begin{pgfscope}%
\pgfsys@transformshift{1.893263in}{2.079852in}%
\pgfsys@useobject{currentmarker}{}%
\end{pgfscope}%
\begin{pgfscope}%
\pgfsys@transformshift{2.308206in}{2.134182in}%
\pgfsys@useobject{currentmarker}{}%
\end{pgfscope}%
\begin{pgfscope}%
\pgfsys@transformshift{2.786986in}{2.180151in}%
\pgfsys@useobject{currentmarker}{}%
\end{pgfscope}%
\begin{pgfscope}%
\pgfsys@transformshift{3.329604in}{2.202290in}%
\pgfsys@useobject{currentmarker}{}%
\end{pgfscope}%
\begin{pgfscope}%
\pgfsys@transformshift{3.936059in}{2.248868in}%
\pgfsys@useobject{currentmarker}{}%
\end{pgfscope}%
\end{pgfscope}%
\begin{pgfscope}%
\pgfsetroundcap%
\pgfsetroundjoin%
\pgfsetlinewidth{1.003750pt}%
\definecolor{currentstroke}{rgb}{0.835294,0.368627,0.000000}%
\pgfsetstrokecolor{currentstroke}%
\pgfsetdash{}{0pt}%
\pgfpathmoveto{\pgfqpoint{0.752170in}{1.107628in}}%
\pgfpathlineto{\pgfqpoint{0.776109in}{1.290644in}}%
\pgfpathlineto{\pgfqpoint{0.816008in}{1.289714in}}%
\pgfpathlineto{\pgfqpoint{0.871865in}{1.355020in}}%
\pgfpathlineto{\pgfqpoint{0.943682in}{1.424046in}}%
\pgfpathlineto{\pgfqpoint{1.031459in}{1.480011in}}%
\pgfpathlineto{\pgfqpoint{1.135194in}{1.523865in}}%
\pgfpathlineto{\pgfqpoint{1.254889in}{1.599029in}}%
\pgfpathlineto{\pgfqpoint{1.390544in}{1.630113in}}%
\pgfpathlineto{\pgfqpoint{1.542158in}{1.656316in}}%
\pgfpathlineto{\pgfqpoint{1.893263in}{1.710110in}}%
\pgfpathlineto{\pgfqpoint{2.308206in}{1.756857in}}%
\pgfpathlineto{\pgfqpoint{2.786986in}{1.794107in}}%
\pgfpathlineto{\pgfqpoint{3.329604in}{1.827547in}}%
\pgfpathlineto{\pgfqpoint{3.936059in}{1.862448in}}%
\pgfusepath{stroke}%
\end{pgfscope}%
\begin{pgfscope}%
\pgfsetbuttcap%
\pgfsetroundjoin%
\definecolor{currentfill}{rgb}{0.835294,0.368627,0.000000}%
\pgfsetfillcolor{currentfill}%
\pgfsetlinewidth{0.752812pt}%
\definecolor{currentstroke}{rgb}{1.000000,1.000000,1.000000}%
\pgfsetstrokecolor{currentstroke}%
\pgfsetdash{}{0pt}%
\pgfsys@defobject{currentmarker}{\pgfqpoint{-0.034722in}{-0.034722in}}{\pgfqpoint{0.034722in}{0.034722in}}{%
\pgfpathmoveto{\pgfqpoint{0.000000in}{-0.034722in}}%
\pgfpathcurveto{\pgfqpoint{0.009208in}{-0.034722in}}{\pgfqpoint{0.018041in}{-0.031064in}}{\pgfqpoint{0.024552in}{-0.024552in}}%
\pgfpathcurveto{\pgfqpoint{0.031064in}{-0.018041in}}{\pgfqpoint{0.034722in}{-0.009208in}}{\pgfqpoint{0.034722in}{0.000000in}}%
\pgfpathcurveto{\pgfqpoint{0.034722in}{0.009208in}}{\pgfqpoint{0.031064in}{0.018041in}}{\pgfqpoint{0.024552in}{0.024552in}}%
\pgfpathcurveto{\pgfqpoint{0.018041in}{0.031064in}}{\pgfqpoint{0.009208in}{0.034722in}}{\pgfqpoint{0.000000in}{0.034722in}}%
\pgfpathcurveto{\pgfqpoint{-0.009208in}{0.034722in}}{\pgfqpoint{-0.018041in}{0.031064in}}{\pgfqpoint{-0.024552in}{0.024552in}}%
\pgfpathcurveto{\pgfqpoint{-0.031064in}{0.018041in}}{\pgfqpoint{-0.034722in}{0.009208in}}{\pgfqpoint{-0.034722in}{0.000000in}}%
\pgfpathcurveto{\pgfqpoint{-0.034722in}{-0.009208in}}{\pgfqpoint{-0.031064in}{-0.018041in}}{\pgfqpoint{-0.024552in}{-0.024552in}}%
\pgfpathcurveto{\pgfqpoint{-0.018041in}{-0.031064in}}{\pgfqpoint{-0.009208in}{-0.034722in}}{\pgfqpoint{0.000000in}{-0.034722in}}%
\pgfpathlineto{\pgfqpoint{0.000000in}{-0.034722in}}%
\pgfpathclose%
\pgfusepath{stroke,fill}%
}%
\begin{pgfscope}%
\pgfsys@transformshift{0.752170in}{1.107628in}%
\pgfsys@useobject{currentmarker}{}%
\end{pgfscope}%
\begin{pgfscope}%
\pgfsys@transformshift{0.776109in}{1.290644in}%
\pgfsys@useobject{currentmarker}{}%
\end{pgfscope}%
\begin{pgfscope}%
\pgfsys@transformshift{0.816008in}{1.289714in}%
\pgfsys@useobject{currentmarker}{}%
\end{pgfscope}%
\begin{pgfscope}%
\pgfsys@transformshift{0.871865in}{1.355020in}%
\pgfsys@useobject{currentmarker}{}%
\end{pgfscope}%
\begin{pgfscope}%
\pgfsys@transformshift{0.943682in}{1.424046in}%
\pgfsys@useobject{currentmarker}{}%
\end{pgfscope}%
\begin{pgfscope}%
\pgfsys@transformshift{1.031459in}{1.480011in}%
\pgfsys@useobject{currentmarker}{}%
\end{pgfscope}%
\begin{pgfscope}%
\pgfsys@transformshift{1.135194in}{1.523865in}%
\pgfsys@useobject{currentmarker}{}%
\end{pgfscope}%
\begin{pgfscope}%
\pgfsys@transformshift{1.254889in}{1.599029in}%
\pgfsys@useobject{currentmarker}{}%
\end{pgfscope}%
\begin{pgfscope}%
\pgfsys@transformshift{1.390544in}{1.630113in}%
\pgfsys@useobject{currentmarker}{}%
\end{pgfscope}%
\begin{pgfscope}%
\pgfsys@transformshift{1.542158in}{1.656316in}%
\pgfsys@useobject{currentmarker}{}%
\end{pgfscope}%
\begin{pgfscope}%
\pgfsys@transformshift{1.893263in}{1.710110in}%
\pgfsys@useobject{currentmarker}{}%
\end{pgfscope}%
\begin{pgfscope}%
\pgfsys@transformshift{2.308206in}{1.756857in}%
\pgfsys@useobject{currentmarker}{}%
\end{pgfscope}%
\begin{pgfscope}%
\pgfsys@transformshift{2.786986in}{1.794107in}%
\pgfsys@useobject{currentmarker}{}%
\end{pgfscope}%
\begin{pgfscope}%
\pgfsys@transformshift{3.329604in}{1.827547in}%
\pgfsys@useobject{currentmarker}{}%
\end{pgfscope}%
\begin{pgfscope}%
\pgfsys@transformshift{3.936059in}{1.862448in}%
\pgfsys@useobject{currentmarker}{}%
\end{pgfscope}%
\end{pgfscope}%
\begin{pgfscope}%
\pgfsetroundcap%
\pgfsetroundjoin%
\pgfsetlinewidth{1.003750pt}%
\definecolor{currentstroke}{rgb}{0.800000,0.470588,0.737255}%
\pgfsetstrokecolor{currentstroke}%
\pgfsetdash{}{0pt}%
\pgfpathmoveto{\pgfqpoint{0.752170in}{0.635699in}}%
\pgfpathlineto{\pgfqpoint{0.776109in}{0.752110in}}%
\pgfpathlineto{\pgfqpoint{0.816008in}{0.744553in}}%
\pgfpathlineto{\pgfqpoint{0.871865in}{0.812453in}}%
\pgfpathlineto{\pgfqpoint{0.943682in}{0.879829in}}%
\pgfpathlineto{\pgfqpoint{1.031459in}{0.929810in}}%
\pgfpathlineto{\pgfqpoint{1.135194in}{1.609542in}}%
\pgfpathlineto{\pgfqpoint{1.254889in}{1.619175in}}%
\pgfpathlineto{\pgfqpoint{1.390544in}{1.113851in}}%
\pgfpathlineto{\pgfqpoint{1.542158in}{1.255301in}}%
\pgfpathlineto{\pgfqpoint{1.893263in}{1.150978in}}%
\pgfpathlineto{\pgfqpoint{2.308206in}{1.275789in}}%
\pgfpathlineto{\pgfqpoint{2.786986in}{1.277180in}}%
\pgfpathlineto{\pgfqpoint{3.329604in}{1.425561in}}%
\pgfpathlineto{\pgfqpoint{3.936059in}{1.428371in}}%
\pgfusepath{stroke}%
\end{pgfscope}%
\begin{pgfscope}%
\pgfsetbuttcap%
\pgfsetroundjoin%
\definecolor{currentfill}{rgb}{0.800000,0.470588,0.737255}%
\pgfsetfillcolor{currentfill}%
\pgfsetlinewidth{0.752812pt}%
\definecolor{currentstroke}{rgb}{1.000000,1.000000,1.000000}%
\pgfsetstrokecolor{currentstroke}%
\pgfsetdash{}{0pt}%
\pgfsys@defobject{currentmarker}{\pgfqpoint{-0.034722in}{-0.034722in}}{\pgfqpoint{0.034722in}{0.034722in}}{%
\pgfpathmoveto{\pgfqpoint{0.000000in}{-0.034722in}}%
\pgfpathcurveto{\pgfqpoint{0.009208in}{-0.034722in}}{\pgfqpoint{0.018041in}{-0.031064in}}{\pgfqpoint{0.024552in}{-0.024552in}}%
\pgfpathcurveto{\pgfqpoint{0.031064in}{-0.018041in}}{\pgfqpoint{0.034722in}{-0.009208in}}{\pgfqpoint{0.034722in}{0.000000in}}%
\pgfpathcurveto{\pgfqpoint{0.034722in}{0.009208in}}{\pgfqpoint{0.031064in}{0.018041in}}{\pgfqpoint{0.024552in}{0.024552in}}%
\pgfpathcurveto{\pgfqpoint{0.018041in}{0.031064in}}{\pgfqpoint{0.009208in}{0.034722in}}{\pgfqpoint{0.000000in}{0.034722in}}%
\pgfpathcurveto{\pgfqpoint{-0.009208in}{0.034722in}}{\pgfqpoint{-0.018041in}{0.031064in}}{\pgfqpoint{-0.024552in}{0.024552in}}%
\pgfpathcurveto{\pgfqpoint{-0.031064in}{0.018041in}}{\pgfqpoint{-0.034722in}{0.009208in}}{\pgfqpoint{-0.034722in}{0.000000in}}%
\pgfpathcurveto{\pgfqpoint{-0.034722in}{-0.009208in}}{\pgfqpoint{-0.031064in}{-0.018041in}}{\pgfqpoint{-0.024552in}{-0.024552in}}%
\pgfpathcurveto{\pgfqpoint{-0.018041in}{-0.031064in}}{\pgfqpoint{-0.009208in}{-0.034722in}}{\pgfqpoint{0.000000in}{-0.034722in}}%
\pgfpathlineto{\pgfqpoint{0.000000in}{-0.034722in}}%
\pgfpathclose%
\pgfusepath{stroke,fill}%
}%
\begin{pgfscope}%
\pgfsys@transformshift{0.752170in}{0.635699in}%
\pgfsys@useobject{currentmarker}{}%
\end{pgfscope}%
\begin{pgfscope}%
\pgfsys@transformshift{0.776109in}{0.752110in}%
\pgfsys@useobject{currentmarker}{}%
\end{pgfscope}%
\begin{pgfscope}%
\pgfsys@transformshift{0.816008in}{0.744553in}%
\pgfsys@useobject{currentmarker}{}%
\end{pgfscope}%
\begin{pgfscope}%
\pgfsys@transformshift{0.871865in}{0.812453in}%
\pgfsys@useobject{currentmarker}{}%
\end{pgfscope}%
\begin{pgfscope}%
\pgfsys@transformshift{0.943682in}{0.879829in}%
\pgfsys@useobject{currentmarker}{}%
\end{pgfscope}%
\begin{pgfscope}%
\pgfsys@transformshift{1.031459in}{0.929810in}%
\pgfsys@useobject{currentmarker}{}%
\end{pgfscope}%
\begin{pgfscope}%
\pgfsys@transformshift{1.135194in}{1.609542in}%
\pgfsys@useobject{currentmarker}{}%
\end{pgfscope}%
\begin{pgfscope}%
\pgfsys@transformshift{1.254889in}{1.619175in}%
\pgfsys@useobject{currentmarker}{}%
\end{pgfscope}%
\begin{pgfscope}%
\pgfsys@transformshift{1.390544in}{1.113851in}%
\pgfsys@useobject{currentmarker}{}%
\end{pgfscope}%
\begin{pgfscope}%
\pgfsys@transformshift{1.542158in}{1.255301in}%
\pgfsys@useobject{currentmarker}{}%
\end{pgfscope}%
\begin{pgfscope}%
\pgfsys@transformshift{1.893263in}{1.150978in}%
\pgfsys@useobject{currentmarker}{}%
\end{pgfscope}%
\begin{pgfscope}%
\pgfsys@transformshift{2.308206in}{1.275789in}%
\pgfsys@useobject{currentmarker}{}%
\end{pgfscope}%
\begin{pgfscope}%
\pgfsys@transformshift{2.786986in}{1.277180in}%
\pgfsys@useobject{currentmarker}{}%
\end{pgfscope}%
\begin{pgfscope}%
\pgfsys@transformshift{3.329604in}{1.425561in}%
\pgfsys@useobject{currentmarker}{}%
\end{pgfscope}%
\begin{pgfscope}%
\pgfsys@transformshift{3.936059in}{1.428371in}%
\pgfsys@useobject{currentmarker}{}%
\end{pgfscope}%
\end{pgfscope}%
\begin{pgfscope}%
\pgfsetroundcap%
\pgfsetroundjoin%
\pgfsetlinewidth{1.003750pt}%
\definecolor{currentstroke}{rgb}{0.792157,0.568627,0.380392}%
\pgfsetstrokecolor{currentstroke}%
\pgfsetdash{}{0pt}%
\pgfpathmoveto{\pgfqpoint{0.752170in}{0.602671in}}%
\pgfpathlineto{\pgfqpoint{0.776109in}{0.745106in}}%
\pgfpathlineto{\pgfqpoint{0.816008in}{0.747296in}}%
\pgfpathlineto{\pgfqpoint{0.871865in}{0.837414in}}%
\pgfpathlineto{\pgfqpoint{0.943682in}{0.924544in}}%
\pgfpathlineto{\pgfqpoint{1.031459in}{0.980999in}}%
\pgfpathlineto{\pgfqpoint{1.135194in}{1.029238in}}%
\pgfpathlineto{\pgfqpoint{1.254889in}{1.078591in}}%
\pgfpathlineto{\pgfqpoint{1.390544in}{1.156926in}}%
\pgfpathlineto{\pgfqpoint{1.542158in}{1.283793in}}%
\pgfpathlineto{\pgfqpoint{1.893263in}{1.183071in}}%
\pgfpathlineto{\pgfqpoint{2.308206in}{1.352689in}}%
\pgfpathlineto{\pgfqpoint{2.786986in}{1.375437in}}%
\pgfpathlineto{\pgfqpoint{3.329604in}{1.486217in}}%
\pgfpathlineto{\pgfqpoint{3.936059in}{1.490379in}}%
\pgfusepath{stroke}%
\end{pgfscope}%
\begin{pgfscope}%
\pgfsetbuttcap%
\pgfsetroundjoin%
\definecolor{currentfill}{rgb}{0.792157,0.568627,0.380392}%
\pgfsetfillcolor{currentfill}%
\pgfsetlinewidth{0.752812pt}%
\definecolor{currentstroke}{rgb}{1.000000,1.000000,1.000000}%
\pgfsetstrokecolor{currentstroke}%
\pgfsetdash{}{0pt}%
\pgfsys@defobject{currentmarker}{\pgfqpoint{-0.034722in}{-0.034722in}}{\pgfqpoint{0.034722in}{0.034722in}}{%
\pgfpathmoveto{\pgfqpoint{0.000000in}{-0.034722in}}%
\pgfpathcurveto{\pgfqpoint{0.009208in}{-0.034722in}}{\pgfqpoint{0.018041in}{-0.031064in}}{\pgfqpoint{0.024552in}{-0.024552in}}%
\pgfpathcurveto{\pgfqpoint{0.031064in}{-0.018041in}}{\pgfqpoint{0.034722in}{-0.009208in}}{\pgfqpoint{0.034722in}{0.000000in}}%
\pgfpathcurveto{\pgfqpoint{0.034722in}{0.009208in}}{\pgfqpoint{0.031064in}{0.018041in}}{\pgfqpoint{0.024552in}{0.024552in}}%
\pgfpathcurveto{\pgfqpoint{0.018041in}{0.031064in}}{\pgfqpoint{0.009208in}{0.034722in}}{\pgfqpoint{0.000000in}{0.034722in}}%
\pgfpathcurveto{\pgfqpoint{-0.009208in}{0.034722in}}{\pgfqpoint{-0.018041in}{0.031064in}}{\pgfqpoint{-0.024552in}{0.024552in}}%
\pgfpathcurveto{\pgfqpoint{-0.031064in}{0.018041in}}{\pgfqpoint{-0.034722in}{0.009208in}}{\pgfqpoint{-0.034722in}{0.000000in}}%
\pgfpathcurveto{\pgfqpoint{-0.034722in}{-0.009208in}}{\pgfqpoint{-0.031064in}{-0.018041in}}{\pgfqpoint{-0.024552in}{-0.024552in}}%
\pgfpathcurveto{\pgfqpoint{-0.018041in}{-0.031064in}}{\pgfqpoint{-0.009208in}{-0.034722in}}{\pgfqpoint{0.000000in}{-0.034722in}}%
\pgfpathlineto{\pgfqpoint{0.000000in}{-0.034722in}}%
\pgfpathclose%
\pgfusepath{stroke,fill}%
}%
\begin{pgfscope}%
\pgfsys@transformshift{0.752170in}{0.602671in}%
\pgfsys@useobject{currentmarker}{}%
\end{pgfscope}%
\begin{pgfscope}%
\pgfsys@transformshift{0.776109in}{0.745106in}%
\pgfsys@useobject{currentmarker}{}%
\end{pgfscope}%
\begin{pgfscope}%
\pgfsys@transformshift{0.816008in}{0.747296in}%
\pgfsys@useobject{currentmarker}{}%
\end{pgfscope}%
\begin{pgfscope}%
\pgfsys@transformshift{0.871865in}{0.837414in}%
\pgfsys@useobject{currentmarker}{}%
\end{pgfscope}%
\begin{pgfscope}%
\pgfsys@transformshift{0.943682in}{0.924544in}%
\pgfsys@useobject{currentmarker}{}%
\end{pgfscope}%
\begin{pgfscope}%
\pgfsys@transformshift{1.031459in}{0.980999in}%
\pgfsys@useobject{currentmarker}{}%
\end{pgfscope}%
\begin{pgfscope}%
\pgfsys@transformshift{1.135194in}{1.029238in}%
\pgfsys@useobject{currentmarker}{}%
\end{pgfscope}%
\begin{pgfscope}%
\pgfsys@transformshift{1.254889in}{1.078591in}%
\pgfsys@useobject{currentmarker}{}%
\end{pgfscope}%
\begin{pgfscope}%
\pgfsys@transformshift{1.390544in}{1.156926in}%
\pgfsys@useobject{currentmarker}{}%
\end{pgfscope}%
\begin{pgfscope}%
\pgfsys@transformshift{1.542158in}{1.283793in}%
\pgfsys@useobject{currentmarker}{}%
\end{pgfscope}%
\begin{pgfscope}%
\pgfsys@transformshift{1.893263in}{1.183071in}%
\pgfsys@useobject{currentmarker}{}%
\end{pgfscope}%
\begin{pgfscope}%
\pgfsys@transformshift{2.308206in}{1.352689in}%
\pgfsys@useobject{currentmarker}{}%
\end{pgfscope}%
\begin{pgfscope}%
\pgfsys@transformshift{2.786986in}{1.375437in}%
\pgfsys@useobject{currentmarker}{}%
\end{pgfscope}%
\begin{pgfscope}%
\pgfsys@transformshift{3.329604in}{1.486217in}%
\pgfsys@useobject{currentmarker}{}%
\end{pgfscope}%
\begin{pgfscope}%
\pgfsys@transformshift{3.936059in}{1.490379in}%
\pgfsys@useobject{currentmarker}{}%
\end{pgfscope}%
\end{pgfscope}%
\end{pgfpicture}%
\makeatother%
\endgroup%

					\end{figcenter}
					\caption{Import time of the \enquote{OSM rural} dataset by tasks.}
				\end{subfigure}
%				\hfill
%				\begin{subfigure}[t]{.48\textwidth}
%					\begin{figcenter}
%						%% Creator: Matplotlib, PGF backend
%%
%% To include the figure in your LaTeX document, write
%%   \input{<filename>.pgf}
%%
%% Make sure the required packages are loaded in your preamble
%%   \usepackage{pgf}
%%
%% Also ensure that all the required font packages are loaded; for instance,
%% the lmodern package is sometimes necessary when using math font.
%%   \usepackage{lmodern}
%%
%% Figures using additional raster images can only be included by \input if
%% they are in the same directory as the main LaTeX file. For loading figures
%% from other directories you can use the `import` package
%%   \usepackage{import}
%%
%% and then include the figures with
%%   \import{<path to file>}{<filename>.pgf}
%%
%% Matplotlib used the following preamble
%%   
%%   \usepackage{fontspec}
%%   \setmainfont{DejaVuSerif.ttf}[Path=\detokenize{/home/hauke/.local/lib/python3.11/site-packages/matplotlib/mpl-data/fonts/ttf/}]
%%   \setsansfont{DroidSans.ttf}[Path=\detokenize{/usr/share/fonts/droid/}]
%%   \setmonofont{DejaVuSansMono.ttf}[Path=\detokenize{/home/hauke/.local/lib/python3.11/site-packages/matplotlib/mpl-data/fonts/ttf/}]
%%   \makeatletter\@ifpackageloaded{underscore}{}{\usepackage[strings]{underscore}}\makeatother
%%
\begingroup%
\makeatletter%
\begin{pgfpicture}%
\pgfpathrectangle{\pgfpointorigin}{\pgfqpoint{5.694946in}{2.407400in}}%
\pgfusepath{use as bounding box, clip}%
\begin{pgfscope}%
\pgfsetbuttcap%
\pgfsetmiterjoin%
\definecolor{currentfill}{rgb}{1.000000,1.000000,1.000000}%
\pgfsetfillcolor{currentfill}%
\pgfsetlinewidth{0.000000pt}%
\definecolor{currentstroke}{rgb}{1.000000,1.000000,1.000000}%
\pgfsetstrokecolor{currentstroke}%
\pgfsetdash{}{0pt}%
\pgfpathmoveto{\pgfqpoint{0.000000in}{0.000000in}}%
\pgfpathlineto{\pgfqpoint{5.694946in}{0.000000in}}%
\pgfpathlineto{\pgfqpoint{5.694946in}{2.407400in}}%
\pgfpathlineto{\pgfqpoint{0.000000in}{2.407400in}}%
\pgfpathlineto{\pgfqpoint{0.000000in}{0.000000in}}%
\pgfpathclose%
\pgfusepath{fill}%
\end{pgfscope}%
\begin{pgfscope}%
\pgfsetbuttcap%
\pgfsetmiterjoin%
\definecolor{currentfill}{rgb}{1.000000,1.000000,1.000000}%
\pgfsetfillcolor{currentfill}%
\pgfsetlinewidth{0.000000pt}%
\definecolor{currentstroke}{rgb}{0.000000,0.000000,0.000000}%
\pgfsetstrokecolor{currentstroke}%
\pgfsetstrokeopacity{0.000000}%
\pgfsetdash{}{0pt}%
\pgfpathmoveto{\pgfqpoint{0.592976in}{0.451389in}}%
\pgfpathlineto{\pgfqpoint{4.295331in}{0.451389in}}%
\pgfpathlineto{\pgfqpoint{4.295331in}{2.407400in}}%
\pgfpathlineto{\pgfqpoint{0.592976in}{2.407400in}}%
\pgfpathlineto{\pgfqpoint{0.592976in}{0.451389in}}%
\pgfpathclose%
\pgfusepath{fill}%
\end{pgfscope}%
\begin{pgfscope}%
\pgfpathrectangle{\pgfqpoint{0.592976in}{0.451389in}}{\pgfqpoint{3.702355in}{1.956011in}}%
\pgfusepath{clip}%
\pgfsetroundcap%
\pgfsetroundjoin%
\pgfsetlinewidth{1.003750pt}%
\definecolor{currentstroke}{rgb}{0.800000,0.800000,0.800000}%
\pgfsetstrokecolor{currentstroke}%
\pgfsetdash{}{0pt}%
\pgfpathmoveto{\pgfqpoint{1.179728in}{0.451389in}}%
\pgfpathlineto{\pgfqpoint{1.179728in}{2.407400in}}%
\pgfusepath{stroke}%
\end{pgfscope}%
\begin{pgfscope}%
\definecolor{textcolor}{rgb}{0.150000,0.150000,0.150000}%
\pgfsetstrokecolor{textcolor}%
\pgfsetfillcolor{textcolor}%
\pgftext[x=1.179728in,y=0.319444in,,top]{\color{textcolor}\sffamily\fontsize{9.000000}{10.800000}\selectfont 2000}%
\end{pgfscope}%
\begin{pgfscope}%
\pgfpathrectangle{\pgfqpoint{0.592976in}{0.451389in}}{\pgfqpoint{3.702355in}{1.956011in}}%
\pgfusepath{clip}%
\pgfsetroundcap%
\pgfsetroundjoin%
\pgfsetlinewidth{1.003750pt}%
\definecolor{currentstroke}{rgb}{0.800000,0.800000,0.800000}%
\pgfsetstrokecolor{currentstroke}%
\pgfsetdash{}{0pt}%
\pgfpathmoveto{\pgfqpoint{1.802905in}{0.451389in}}%
\pgfpathlineto{\pgfqpoint{1.802905in}{2.407400in}}%
\pgfusepath{stroke}%
\end{pgfscope}%
\begin{pgfscope}%
\definecolor{textcolor}{rgb}{0.150000,0.150000,0.150000}%
\pgfsetstrokecolor{textcolor}%
\pgfsetfillcolor{textcolor}%
\pgftext[x=1.802905in,y=0.319444in,,top]{\color{textcolor}\sffamily\fontsize{9.000000}{10.800000}\selectfont 4000}%
\end{pgfscope}%
\begin{pgfscope}%
\pgfpathrectangle{\pgfqpoint{0.592976in}{0.451389in}}{\pgfqpoint{3.702355in}{1.956011in}}%
\pgfusepath{clip}%
\pgfsetroundcap%
\pgfsetroundjoin%
\pgfsetlinewidth{1.003750pt}%
\definecolor{currentstroke}{rgb}{0.800000,0.800000,0.800000}%
\pgfsetstrokecolor{currentstroke}%
\pgfsetdash{}{0pt}%
\pgfpathmoveto{\pgfqpoint{2.426081in}{0.451389in}}%
\pgfpathlineto{\pgfqpoint{2.426081in}{2.407400in}}%
\pgfusepath{stroke}%
\end{pgfscope}%
\begin{pgfscope}%
\definecolor{textcolor}{rgb}{0.150000,0.150000,0.150000}%
\pgfsetstrokecolor{textcolor}%
\pgfsetfillcolor{textcolor}%
\pgftext[x=2.426081in,y=0.319444in,,top]{\color{textcolor}\sffamily\fontsize{9.000000}{10.800000}\selectfont 6000}%
\end{pgfscope}%
\begin{pgfscope}%
\pgfpathrectangle{\pgfqpoint{0.592976in}{0.451389in}}{\pgfqpoint{3.702355in}{1.956011in}}%
\pgfusepath{clip}%
\pgfsetroundcap%
\pgfsetroundjoin%
\pgfsetlinewidth{1.003750pt}%
\definecolor{currentstroke}{rgb}{0.800000,0.800000,0.800000}%
\pgfsetstrokecolor{currentstroke}%
\pgfsetdash{}{0pt}%
\pgfpathmoveto{\pgfqpoint{3.049258in}{0.451389in}}%
\pgfpathlineto{\pgfqpoint{3.049258in}{2.407400in}}%
\pgfusepath{stroke}%
\end{pgfscope}%
\begin{pgfscope}%
\definecolor{textcolor}{rgb}{0.150000,0.150000,0.150000}%
\pgfsetstrokecolor{textcolor}%
\pgfsetfillcolor{textcolor}%
\pgftext[x=3.049258in,y=0.319444in,,top]{\color{textcolor}\sffamily\fontsize{9.000000}{10.800000}\selectfont 8000}%
\end{pgfscope}%
\begin{pgfscope}%
\pgfpathrectangle{\pgfqpoint{0.592976in}{0.451389in}}{\pgfqpoint{3.702355in}{1.956011in}}%
\pgfusepath{clip}%
\pgfsetroundcap%
\pgfsetroundjoin%
\pgfsetlinewidth{1.003750pt}%
\definecolor{currentstroke}{rgb}{0.800000,0.800000,0.800000}%
\pgfsetstrokecolor{currentstroke}%
\pgfsetdash{}{0pt}%
\pgfpathmoveto{\pgfqpoint{3.672435in}{0.451389in}}%
\pgfpathlineto{\pgfqpoint{3.672435in}{2.407400in}}%
\pgfusepath{stroke}%
\end{pgfscope}%
\begin{pgfscope}%
\definecolor{textcolor}{rgb}{0.150000,0.150000,0.150000}%
\pgfsetstrokecolor{textcolor}%
\pgfsetfillcolor{textcolor}%
\pgftext[x=3.672435in,y=0.319444in,,top]{\color{textcolor}\sffamily\fontsize{9.000000}{10.800000}\selectfont 10000}%
\end{pgfscope}%
\begin{pgfscope}%
\definecolor{textcolor}{rgb}{0.150000,0.150000,0.150000}%
\pgfsetstrokecolor{textcolor}%
\pgfsetfillcolor{textcolor}%
\pgftext[x=2.444154in,y=0.125000in,,top]{\color{textcolor}\sffamily\fontsize{9.000000}{10.800000}\selectfont Input obstacle vertices}%
\end{pgfscope}%
\begin{pgfscope}%
\pgfpathrectangle{\pgfqpoint{0.592976in}{0.451389in}}{\pgfqpoint{3.702355in}{1.956011in}}%
\pgfusepath{clip}%
\pgfsetroundcap%
\pgfsetroundjoin%
\pgfsetlinewidth{1.003750pt}%
\definecolor{currentstroke}{rgb}{0.800000,0.800000,0.800000}%
\pgfsetstrokecolor{currentstroke}%
\pgfsetdash{}{0pt}%
\pgfpathmoveto{\pgfqpoint{0.592976in}{0.524535in}}%
\pgfpathlineto{\pgfqpoint{4.295331in}{0.524535in}}%
\pgfusepath{stroke}%
\end{pgfscope}%
\begin{pgfscope}%
\definecolor{textcolor}{rgb}{0.150000,0.150000,0.150000}%
\pgfsetstrokecolor{textcolor}%
\pgfsetfillcolor{textcolor}%
\pgftext[x=0.194444in, y=0.477050in, left, base]{\color{textcolor}\sffamily\fontsize{9.000000}{10.800000}\selectfont \(\displaystyle {10^{-7}}\)}%
\end{pgfscope}%
\begin{pgfscope}%
\pgfpathrectangle{\pgfqpoint{0.592976in}{0.451389in}}{\pgfqpoint{3.702355in}{1.956011in}}%
\pgfusepath{clip}%
\pgfsetroundcap%
\pgfsetroundjoin%
\pgfsetlinewidth{1.003750pt}%
\definecolor{currentstroke}{rgb}{0.800000,0.800000,0.800000}%
\pgfsetstrokecolor{currentstroke}%
\pgfsetdash{}{0pt}%
\pgfpathmoveto{\pgfqpoint{0.592976in}{0.973024in}}%
\pgfpathlineto{\pgfqpoint{4.295331in}{0.973024in}}%
\pgfusepath{stroke}%
\end{pgfscope}%
\begin{pgfscope}%
\definecolor{textcolor}{rgb}{0.150000,0.150000,0.150000}%
\pgfsetstrokecolor{textcolor}%
\pgfsetfillcolor{textcolor}%
\pgftext[x=0.194444in, y=0.925538in, left, base]{\color{textcolor}\sffamily\fontsize{9.000000}{10.800000}\selectfont \(\displaystyle {10^{-6}}\)}%
\end{pgfscope}%
\begin{pgfscope}%
\pgfpathrectangle{\pgfqpoint{0.592976in}{0.451389in}}{\pgfqpoint{3.702355in}{1.956011in}}%
\pgfusepath{clip}%
\pgfsetroundcap%
\pgfsetroundjoin%
\pgfsetlinewidth{1.003750pt}%
\definecolor{currentstroke}{rgb}{0.800000,0.800000,0.800000}%
\pgfsetstrokecolor{currentstroke}%
\pgfsetdash{}{0pt}%
\pgfpathmoveto{\pgfqpoint{0.592976in}{1.421513in}}%
\pgfpathlineto{\pgfqpoint{4.295331in}{1.421513in}}%
\pgfusepath{stroke}%
\end{pgfscope}%
\begin{pgfscope}%
\definecolor{textcolor}{rgb}{0.150000,0.150000,0.150000}%
\pgfsetstrokecolor{textcolor}%
\pgfsetfillcolor{textcolor}%
\pgftext[x=0.194444in, y=1.374027in, left, base]{\color{textcolor}\sffamily\fontsize{9.000000}{10.800000}\selectfont \(\displaystyle {10^{-5}}\)}%
\end{pgfscope}%
\begin{pgfscope}%
\pgfpathrectangle{\pgfqpoint{0.592976in}{0.451389in}}{\pgfqpoint{3.702355in}{1.956011in}}%
\pgfusepath{clip}%
\pgfsetroundcap%
\pgfsetroundjoin%
\pgfsetlinewidth{1.003750pt}%
\definecolor{currentstroke}{rgb}{0.800000,0.800000,0.800000}%
\pgfsetstrokecolor{currentstroke}%
\pgfsetdash{}{0pt}%
\pgfpathmoveto{\pgfqpoint{0.592976in}{1.870001in}}%
\pgfpathlineto{\pgfqpoint{4.295331in}{1.870001in}}%
\pgfusepath{stroke}%
\end{pgfscope}%
\begin{pgfscope}%
\definecolor{textcolor}{rgb}{0.150000,0.150000,0.150000}%
\pgfsetstrokecolor{textcolor}%
\pgfsetfillcolor{textcolor}%
\pgftext[x=0.194444in, y=1.822516in, left, base]{\color{textcolor}\sffamily\fontsize{9.000000}{10.800000}\selectfont \(\displaystyle {10^{-4}}\)}%
\end{pgfscope}%
\begin{pgfscope}%
\pgfpathrectangle{\pgfqpoint{0.592976in}{0.451389in}}{\pgfqpoint{3.702355in}{1.956011in}}%
\pgfusepath{clip}%
\pgfsetroundcap%
\pgfsetroundjoin%
\pgfsetlinewidth{1.003750pt}%
\definecolor{currentstroke}{rgb}{0.800000,0.800000,0.800000}%
\pgfsetstrokecolor{currentstroke}%
\pgfsetdash{}{0pt}%
\pgfpathmoveto{\pgfqpoint{0.592976in}{2.318490in}}%
\pgfpathlineto{\pgfqpoint{4.295331in}{2.318490in}}%
\pgfusepath{stroke}%
\end{pgfscope}%
\begin{pgfscope}%
\definecolor{textcolor}{rgb}{0.150000,0.150000,0.150000}%
\pgfsetstrokecolor{textcolor}%
\pgfsetfillcolor{textcolor}%
\pgftext[x=0.194444in, y=2.271005in, left, base]{\color{textcolor}\sffamily\fontsize{9.000000}{10.800000}\selectfont \(\displaystyle {10^{-3}}\)}%
\end{pgfscope}%
\begin{pgfscope}%
\definecolor{textcolor}{rgb}{0.150000,0.150000,0.150000}%
\pgfsetstrokecolor{textcolor}%
\pgfsetfillcolor{textcolor}%
\pgftext[x=0.125000in,y=1.429394in,,bottom,rotate=90.000000]{\color{textcolor}\sffamily\fontsize{9.000000}{10.800000}\selectfont Share of total time}%
\end{pgfscope}%
\begin{pgfscope}%
\pgfpathrectangle{\pgfqpoint{0.592976in}{0.451389in}}{\pgfqpoint{3.702355in}{1.956011in}}%
\pgfusepath{clip}%
\pgfsetbuttcap%
\pgfsetroundjoin%
\definecolor{currentfill}{rgb}{0.003922,0.450980,0.698039}%
\pgfsetfillcolor{currentfill}%
\pgfsetfillopacity{0.200000}%
\pgfsetlinewidth{1.003750pt}%
\definecolor{currentstroke}{rgb}{0.003922,0.450980,0.698039}%
\pgfsetstrokecolor{currentstroke}%
\pgfsetstrokeopacity{0.200000}%
\pgfsetdash{}{0pt}%
\pgfsys@defobject{currentmarker}{\pgfqpoint{0.761265in}{2.318490in}}{\pgfqpoint{4.127042in}{2.318490in}}{%
\pgfpathmoveto{\pgfqpoint{0.761265in}{2.318490in}}%
\pgfpathlineto{\pgfqpoint{0.761265in}{2.318490in}}%
\pgfpathlineto{\pgfqpoint{1.091548in}{2.318490in}}%
\pgfpathlineto{\pgfqpoint{1.466389in}{2.318490in}}%
\pgfpathlineto{\pgfqpoint{1.890773in}{2.318490in}}%
\pgfpathlineto{\pgfqpoint{3.157691in}{2.318490in}}%
\pgfpathlineto{\pgfqpoint{4.127042in}{2.318490in}}%
\pgfpathlineto{\pgfqpoint{4.127042in}{2.318490in}}%
\pgfpathlineto{\pgfqpoint{4.127042in}{2.318490in}}%
\pgfpathlineto{\pgfqpoint{3.157691in}{2.318490in}}%
\pgfpathlineto{\pgfqpoint{1.890773in}{2.318490in}}%
\pgfpathlineto{\pgfqpoint{1.466389in}{2.318490in}}%
\pgfpathlineto{\pgfqpoint{1.091548in}{2.318490in}}%
\pgfpathlineto{\pgfqpoint{0.761265in}{2.318490in}}%
\pgfpathlineto{\pgfqpoint{0.761265in}{2.318490in}}%
\pgfpathclose%
\pgfusepath{stroke,fill}%
}%
\begin{pgfscope}%
\pgfsys@transformshift{0.000000in}{0.000000in}%
\pgfsys@useobject{currentmarker}{}%
\end{pgfscope}%
\end{pgfscope}%
\begin{pgfscope}%
\pgfpathrectangle{\pgfqpoint{0.592976in}{0.451389in}}{\pgfqpoint{3.702355in}{1.956011in}}%
\pgfusepath{clip}%
\pgfsetbuttcap%
\pgfsetroundjoin%
\definecolor{currentfill}{rgb}{0.870588,0.560784,0.019608}%
\pgfsetfillcolor{currentfill}%
\pgfsetfillopacity{0.200000}%
\pgfsetlinewidth{1.003750pt}%
\definecolor{currentstroke}{rgb}{0.870588,0.560784,0.019608}%
\pgfsetstrokecolor{currentstroke}%
\pgfsetstrokeopacity{0.200000}%
\pgfsetdash{}{0pt}%
\pgfsys@defobject{currentmarker}{\pgfqpoint{0.761265in}{2.179422in}}{\pgfqpoint{4.127042in}{2.292625in}}{%
\pgfpathmoveto{\pgfqpoint{0.761265in}{2.185196in}}%
\pgfpathlineto{\pgfqpoint{0.761265in}{2.179422in}}%
\pgfpathlineto{\pgfqpoint{1.091548in}{2.204191in}}%
\pgfpathlineto{\pgfqpoint{1.466389in}{2.256371in}}%
\pgfpathlineto{\pgfqpoint{1.890773in}{2.271872in}}%
\pgfpathlineto{\pgfqpoint{3.157691in}{2.284033in}}%
\pgfpathlineto{\pgfqpoint{4.127042in}{2.290923in}}%
\pgfpathlineto{\pgfqpoint{4.127042in}{2.292625in}}%
\pgfpathlineto{\pgfqpoint{4.127042in}{2.292625in}}%
\pgfpathlineto{\pgfqpoint{3.157691in}{2.286622in}}%
\pgfpathlineto{\pgfqpoint{1.890773in}{2.276390in}}%
\pgfpathlineto{\pgfqpoint{1.466389in}{2.260445in}}%
\pgfpathlineto{\pgfqpoint{1.091548in}{2.214138in}}%
\pgfpathlineto{\pgfqpoint{0.761265in}{2.185196in}}%
\pgfpathlineto{\pgfqpoint{0.761265in}{2.185196in}}%
\pgfpathclose%
\pgfusepath{stroke,fill}%
}%
\begin{pgfscope}%
\pgfsys@transformshift{0.000000in}{0.000000in}%
\pgfsys@useobject{currentmarker}{}%
\end{pgfscope}%
\end{pgfscope}%
\begin{pgfscope}%
\pgfpathrectangle{\pgfqpoint{0.592976in}{0.451389in}}{\pgfqpoint{3.702355in}{1.956011in}}%
\pgfusepath{clip}%
\pgfsetbuttcap%
\pgfsetroundjoin%
\definecolor{currentfill}{rgb}{0.007843,0.619608,0.450980}%
\pgfsetfillcolor{currentfill}%
\pgfsetfillopacity{0.200000}%
\pgfsetlinewidth{1.003750pt}%
\definecolor{currentstroke}{rgb}{0.007843,0.619608,0.450980}%
\pgfsetstrokecolor{currentstroke}%
\pgfsetstrokeopacity{0.200000}%
\pgfsetdash{}{0pt}%
\pgfsys@defobject{currentmarker}{\pgfqpoint{0.761265in}{1.465869in}}{\pgfqpoint{4.127042in}{1.651466in}}{%
\pgfpathmoveto{\pgfqpoint{0.761265in}{1.651466in}}%
\pgfpathlineto{\pgfqpoint{0.761265in}{1.650385in}}%
\pgfpathlineto{\pgfqpoint{1.091548in}{1.535931in}}%
\pgfpathlineto{\pgfqpoint{1.466389in}{1.579708in}}%
\pgfpathlineto{\pgfqpoint{1.890773in}{1.573612in}}%
\pgfpathlineto{\pgfqpoint{3.157691in}{1.510137in}}%
\pgfpathlineto{\pgfqpoint{4.127042in}{1.465869in}}%
\pgfpathlineto{\pgfqpoint{4.127042in}{1.476596in}}%
\pgfpathlineto{\pgfqpoint{4.127042in}{1.476596in}}%
\pgfpathlineto{\pgfqpoint{3.157691in}{1.516067in}}%
\pgfpathlineto{\pgfqpoint{1.890773in}{1.584935in}}%
\pgfpathlineto{\pgfqpoint{1.466389in}{1.587851in}}%
\pgfpathlineto{\pgfqpoint{1.091548in}{1.556795in}}%
\pgfpathlineto{\pgfqpoint{0.761265in}{1.651466in}}%
\pgfpathlineto{\pgfqpoint{0.761265in}{1.651466in}}%
\pgfpathclose%
\pgfusepath{stroke,fill}%
}%
\begin{pgfscope}%
\pgfsys@transformshift{0.000000in}{0.000000in}%
\pgfsys@useobject{currentmarker}{}%
\end{pgfscope}%
\end{pgfscope}%
\begin{pgfscope}%
\pgfpathrectangle{\pgfqpoint{0.592976in}{0.451389in}}{\pgfqpoint{3.702355in}{1.956011in}}%
\pgfusepath{clip}%
\pgfsetbuttcap%
\pgfsetroundjoin%
\definecolor{currentfill}{rgb}{0.835294,0.368627,0.000000}%
\pgfsetfillcolor{currentfill}%
\pgfsetfillopacity{0.200000}%
\pgfsetlinewidth{1.003750pt}%
\definecolor{currentstroke}{rgb}{0.835294,0.368627,0.000000}%
\pgfsetstrokecolor{currentstroke}%
\pgfsetstrokeopacity{0.200000}%
\pgfsetdash{}{0pt}%
\pgfsys@defobject{currentmarker}{\pgfqpoint{0.761265in}{1.191897in}}{\pgfqpoint{4.127042in}{1.510784in}}{%
\pgfpathmoveto{\pgfqpoint{0.761265in}{1.510784in}}%
\pgfpathlineto{\pgfqpoint{0.761265in}{1.499849in}}%
\pgfpathlineto{\pgfqpoint{1.091548in}{1.344084in}}%
\pgfpathlineto{\pgfqpoint{1.466389in}{1.316344in}}%
\pgfpathlineto{\pgfqpoint{1.890773in}{1.277472in}}%
\pgfpathlineto{\pgfqpoint{3.157691in}{1.221987in}}%
\pgfpathlineto{\pgfqpoint{4.127042in}{1.191897in}}%
\pgfpathlineto{\pgfqpoint{4.127042in}{1.226923in}}%
\pgfpathlineto{\pgfqpoint{4.127042in}{1.226923in}}%
\pgfpathlineto{\pgfqpoint{3.157691in}{1.250997in}}%
\pgfpathlineto{\pgfqpoint{1.890773in}{1.304030in}}%
\pgfpathlineto{\pgfqpoint{1.466389in}{1.332316in}}%
\pgfpathlineto{\pgfqpoint{1.091548in}{1.467128in}}%
\pgfpathlineto{\pgfqpoint{0.761265in}{1.510784in}}%
\pgfpathlineto{\pgfqpoint{0.761265in}{1.510784in}}%
\pgfpathclose%
\pgfusepath{stroke,fill}%
}%
\begin{pgfscope}%
\pgfsys@transformshift{0.000000in}{0.000000in}%
\pgfsys@useobject{currentmarker}{}%
\end{pgfscope}%
\end{pgfscope}%
\begin{pgfscope}%
\pgfpathrectangle{\pgfqpoint{0.592976in}{0.451389in}}{\pgfqpoint{3.702355in}{1.956011in}}%
\pgfusepath{clip}%
\pgfsetbuttcap%
\pgfsetroundjoin%
\definecolor{currentfill}{rgb}{0.800000,0.470588,0.737255}%
\pgfsetfillcolor{currentfill}%
\pgfsetfillopacity{0.200000}%
\pgfsetlinewidth{1.003750pt}%
\definecolor{currentstroke}{rgb}{0.800000,0.470588,0.737255}%
\pgfsetstrokecolor{currentstroke}%
\pgfsetstrokeopacity{0.200000}%
\pgfsetdash{}{0pt}%
\pgfsys@defobject{currentmarker}{\pgfqpoint{0.761265in}{1.877366in}}{\pgfqpoint{4.127042in}{2.155375in}}{%
\pgfpathmoveto{\pgfqpoint{0.761265in}{2.155375in}}%
\pgfpathlineto{\pgfqpoint{0.761265in}{2.146762in}}%
\pgfpathlineto{\pgfqpoint{1.091548in}{2.126007in}}%
\pgfpathlineto{\pgfqpoint{1.466389in}{2.022284in}}%
\pgfpathlineto{\pgfqpoint{1.890773in}{1.960169in}}%
\pgfpathlineto{\pgfqpoint{3.157691in}{1.914779in}}%
\pgfpathlineto{\pgfqpoint{4.127042in}{1.877366in}}%
\pgfpathlineto{\pgfqpoint{4.127042in}{1.891647in}}%
\pgfpathlineto{\pgfqpoint{4.127042in}{1.891647in}}%
\pgfpathlineto{\pgfqpoint{3.157691in}{1.931396in}}%
\pgfpathlineto{\pgfqpoint{1.890773in}{1.981764in}}%
\pgfpathlineto{\pgfqpoint{1.466389in}{2.036978in}}%
\pgfpathlineto{\pgfqpoint{1.091548in}{2.140481in}}%
\pgfpathlineto{\pgfqpoint{0.761265in}{2.155375in}}%
\pgfpathlineto{\pgfqpoint{0.761265in}{2.155375in}}%
\pgfpathclose%
\pgfusepath{stroke,fill}%
}%
\begin{pgfscope}%
\pgfsys@transformshift{0.000000in}{0.000000in}%
\pgfsys@useobject{currentmarker}{}%
\end{pgfscope}%
\end{pgfscope}%
\begin{pgfscope}%
\pgfpathrectangle{\pgfqpoint{0.592976in}{0.451389in}}{\pgfqpoint{3.702355in}{1.956011in}}%
\pgfusepath{clip}%
\pgfsetbuttcap%
\pgfsetroundjoin%
\definecolor{currentfill}{rgb}{0.792157,0.568627,0.380392}%
\pgfsetfillcolor{currentfill}%
\pgfsetfillopacity{0.200000}%
\pgfsetlinewidth{1.003750pt}%
\definecolor{currentstroke}{rgb}{0.792157,0.568627,0.380392}%
\pgfsetstrokecolor{currentstroke}%
\pgfsetstrokeopacity{0.200000}%
\pgfsetdash{}{0pt}%
\pgfsys@defobject{currentmarker}{\pgfqpoint{0.761265in}{0.540298in}}{\pgfqpoint{4.127042in}{0.963196in}}{%
\pgfpathmoveto{\pgfqpoint{0.761265in}{0.917466in}}%
\pgfpathlineto{\pgfqpoint{0.761265in}{0.898816in}}%
\pgfpathlineto{\pgfqpoint{1.091548in}{0.751731in}}%
\pgfpathlineto{\pgfqpoint{1.466389in}{0.709081in}}%
\pgfpathlineto{\pgfqpoint{1.890773in}{0.681996in}}%
\pgfpathlineto{\pgfqpoint{3.157691in}{0.595232in}}%
\pgfpathlineto{\pgfqpoint{4.127042in}{0.540298in}}%
\pgfpathlineto{\pgfqpoint{4.127042in}{0.680150in}}%
\pgfpathlineto{\pgfqpoint{4.127042in}{0.680150in}}%
\pgfpathlineto{\pgfqpoint{3.157691in}{0.689680in}}%
\pgfpathlineto{\pgfqpoint{1.890773in}{0.963196in}}%
\pgfpathlineto{\pgfqpoint{1.466389in}{0.729335in}}%
\pgfpathlineto{\pgfqpoint{1.091548in}{0.833740in}}%
\pgfpathlineto{\pgfqpoint{0.761265in}{0.917466in}}%
\pgfpathlineto{\pgfqpoint{0.761265in}{0.917466in}}%
\pgfpathclose%
\pgfusepath{stroke,fill}%
}%
\begin{pgfscope}%
\pgfsys@transformshift{0.000000in}{0.000000in}%
\pgfsys@useobject{currentmarker}{}%
\end{pgfscope}%
\end{pgfscope}%
\begin{pgfscope}%
\pgfsetrectcap%
\pgfsetmiterjoin%
\pgfsetlinewidth{1.254687pt}%
\definecolor{currentstroke}{rgb}{0.800000,0.800000,0.800000}%
\pgfsetstrokecolor{currentstroke}%
\pgfsetdash{}{0pt}%
\pgfpathmoveto{\pgfqpoint{0.592976in}{0.451389in}}%
\pgfpathlineto{\pgfqpoint{0.592976in}{2.407400in}}%
\pgfusepath{stroke}%
\end{pgfscope}%
\begin{pgfscope}%
\pgfsetrectcap%
\pgfsetmiterjoin%
\pgfsetlinewidth{1.254687pt}%
\definecolor{currentstroke}{rgb}{0.800000,0.800000,0.800000}%
\pgfsetstrokecolor{currentstroke}%
\pgfsetdash{}{0pt}%
\pgfpathmoveto{\pgfqpoint{4.295331in}{0.451389in}}%
\pgfpathlineto{\pgfqpoint{4.295331in}{2.407400in}}%
\pgfusepath{stroke}%
\end{pgfscope}%
\begin{pgfscope}%
\pgfsetrectcap%
\pgfsetmiterjoin%
\pgfsetlinewidth{1.254687pt}%
\definecolor{currentstroke}{rgb}{0.800000,0.800000,0.800000}%
\pgfsetstrokecolor{currentstroke}%
\pgfsetdash{}{0pt}%
\pgfpathmoveto{\pgfqpoint{0.592976in}{0.451389in}}%
\pgfpathlineto{\pgfqpoint{4.295331in}{0.451389in}}%
\pgfusepath{stroke}%
\end{pgfscope}%
\begin{pgfscope}%
\pgfsetrectcap%
\pgfsetmiterjoin%
\pgfsetlinewidth{1.254687pt}%
\definecolor{currentstroke}{rgb}{0.800000,0.800000,0.800000}%
\pgfsetstrokecolor{currentstroke}%
\pgfsetdash{}{0pt}%
\pgfpathmoveto{\pgfqpoint{0.592976in}{2.407400in}}%
\pgfpathlineto{\pgfqpoint{4.295331in}{2.407400in}}%
\pgfusepath{stroke}%
\end{pgfscope}%
\begin{pgfscope}%
\pgfsetbuttcap%
\pgfsetmiterjoin%
\definecolor{currentfill}{rgb}{1.000000,1.000000,1.000000}%
\pgfsetfillcolor{currentfill}%
\pgfsetfillopacity{0.800000}%
\pgfsetlinewidth{1.003750pt}%
\definecolor{currentstroke}{rgb}{0.800000,0.800000,0.800000}%
\pgfsetstrokecolor{currentstroke}%
\pgfsetstrokeopacity{0.800000}%
\pgfsetdash{}{0pt}%
\pgfpathmoveto{\pgfqpoint{4.475390in}{0.538404in}}%
\pgfpathlineto{\pgfqpoint{5.669946in}{0.538404in}}%
\pgfpathquadraticcurveto{\pgfqpoint{5.694946in}{0.538404in}}{\pgfqpoint{5.694946in}{0.563404in}}%
\pgfpathlineto{\pgfqpoint{5.694946in}{2.295385in}}%
\pgfpathquadraticcurveto{\pgfqpoint{5.694946in}{2.320385in}}{\pgfqpoint{5.669946in}{2.320385in}}%
\pgfpathlineto{\pgfqpoint{4.475390in}{2.320385in}}%
\pgfpathquadraticcurveto{\pgfqpoint{4.450390in}{2.320385in}}{\pgfqpoint{4.450390in}{2.295385in}}%
\pgfpathlineto{\pgfqpoint{4.450390in}{0.563404in}}%
\pgfpathquadraticcurveto{\pgfqpoint{4.450390in}{0.538404in}}{\pgfqpoint{4.475390in}{0.538404in}}%
\pgfpathlineto{\pgfqpoint{4.475390in}{0.538404in}}%
\pgfpathclose%
\pgfusepath{stroke,fill}%
\end{pgfscope}%
\begin{pgfscope}%
\definecolor{textcolor}{rgb}{0.150000,0.150000,0.150000}%
\pgfsetstrokecolor{textcolor}%
\pgfsetfillcolor{textcolor}%
\pgftext[x=4.869268in,y=2.175414in,left,base]{\color{textcolor}\sffamily\fontsize{9.000000}{10.800000}\selectfont Legend}%
\end{pgfscope}%
\begin{pgfscope}%
\pgfsetroundcap%
\pgfsetroundjoin%
\pgfsetlinewidth{1.505625pt}%
\definecolor{currentstroke}{rgb}{0.003922,0.450980,0.698039}%
\pgfsetstrokecolor{currentstroke}%
\pgfsetdash{}{0pt}%
\pgfpathmoveto{\pgfqpoint{4.500390in}{2.031664in}}%
\pgfpathlineto{\pgfqpoint{4.625390in}{2.031664in}}%
\pgfpathlineto{\pgfqpoint{4.750390in}{2.031664in}}%
\pgfusepath{stroke}%
\end{pgfscope}%
\begin{pgfscope}%
\definecolor{textcolor}{rgb}{0.150000,0.150000,0.150000}%
\pgfsetstrokecolor{textcolor}%
\pgfsetfillcolor{textcolor}%
\pgftext[x=4.850390in,y=1.987914in,left,base]{\color{textcolor}\sffamily\fontsize{9.000000}{10.800000}\selectfont Total time}%
\end{pgfscope}%
\begin{pgfscope}%
\pgfsetroundcap%
\pgfsetroundjoin%
\pgfsetlinewidth{1.505625pt}%
\definecolor{currentstroke}{rgb}{0.870588,0.560784,0.019608}%
\pgfsetstrokecolor{currentstroke}%
\pgfsetdash{}{0pt}%
\pgfpathmoveto{\pgfqpoint{4.500390in}{1.844164in}}%
\pgfpathlineto{\pgfqpoint{4.625390in}{1.844164in}}%
\pgfpathlineto{\pgfqpoint{4.750390in}{1.844164in}}%
\pgfusepath{stroke}%
\end{pgfscope}%
\begin{pgfscope}%
\definecolor{textcolor}{rgb}{0.150000,0.150000,0.150000}%
\pgfsetstrokecolor{textcolor}%
\pgfsetfillcolor{textcolor}%
\pgftext[x=4.850390in,y=1.800414in,left,base]{\color{textcolor}\sffamily\fontsize{9.000000}{10.800000}\selectfont kNN search}%
\end{pgfscope}%
\begin{pgfscope}%
\pgfsetroundcap%
\pgfsetroundjoin%
\pgfsetlinewidth{1.505625pt}%
\definecolor{currentstroke}{rgb}{0.007843,0.619608,0.450980}%
\pgfsetstrokecolor{currentstroke}%
\pgfsetdash{}{0pt}%
\pgfpathmoveto{\pgfqpoint{4.500390in}{1.656664in}}%
\pgfpathlineto{\pgfqpoint{4.625390in}{1.656664in}}%
\pgfpathlineto{\pgfqpoint{4.750390in}{1.656664in}}%
\pgfusepath{stroke}%
\end{pgfscope}%
\begin{pgfscope}%
\definecolor{textcolor}{rgb}{0.150000,0.150000,0.150000}%
\pgfsetstrokecolor{textcolor}%
\pgfsetfillcolor{textcolor}%
\pgftext[x=4.850390in,y=1.612914in,left,base]{\color{textcolor}\sffamily\fontsize{9.000000}{10.800000}\selectfont Create graph}%
\end{pgfscope}%
\begin{pgfscope}%
\pgfsetroundcap%
\pgfsetroundjoin%
\pgfsetlinewidth{1.505625pt}%
\definecolor{currentstroke}{rgb}{0.835294,0.368627,0.000000}%
\pgfsetstrokecolor{currentstroke}%
\pgfsetdash{}{0pt}%
\pgfpathmoveto{\pgfqpoint{4.500390in}{1.382153in}}%
\pgfpathlineto{\pgfqpoint{4.625390in}{1.382153in}}%
\pgfpathlineto{\pgfqpoint{4.750390in}{1.382153in}}%
\pgfusepath{stroke}%
\end{pgfscope}%
\begin{pgfscope}%
\definecolor{textcolor}{rgb}{0.150000,0.150000,0.150000}%
\pgfsetstrokecolor{textcolor}%
\pgfsetfillcolor{textcolor}%
\pgftext[x=4.850390in, y=1.425415in, left, base]{\color{textcolor}\sffamily\fontsize{9.000000}{10.800000}\selectfont Get \& prepare}%
\end{pgfscope}%
\begin{pgfscope}%
\definecolor{textcolor}{rgb}{0.150000,0.150000,0.150000}%
\pgfsetstrokecolor{textcolor}%
\pgfsetfillcolor{textcolor}%
\pgftext[x=4.850390in, y=1.281421in, left, base]{\color{textcolor}\sffamily\fontsize{9.000000}{10.800000}\selectfont obstacles}%
\end{pgfscope}%
\begin{pgfscope}%
\pgfsetroundcap%
\pgfsetroundjoin%
\pgfsetlinewidth{1.505625pt}%
\definecolor{currentstroke}{rgb}{0.800000,0.470588,0.737255}%
\pgfsetstrokecolor{currentstroke}%
\pgfsetdash{}{0pt}%
\pgfpathmoveto{\pgfqpoint{4.500390in}{1.050659in}}%
\pgfpathlineto{\pgfqpoint{4.625390in}{1.050659in}}%
\pgfpathlineto{\pgfqpoint{4.750390in}{1.050659in}}%
\pgfusepath{stroke}%
\end{pgfscope}%
\begin{pgfscope}%
\definecolor{textcolor}{rgb}{0.150000,0.150000,0.150000}%
\pgfsetstrokecolor{textcolor}%
\pgfsetfillcolor{textcolor}%
\pgftext[x=4.850390in, y=1.093921in, left, base]{\color{textcolor}\sffamily\fontsize{9.000000}{10.800000}\selectfont Merge road}%
\end{pgfscope}%
\begin{pgfscope}%
\definecolor{textcolor}{rgb}{0.150000,0.150000,0.150000}%
\pgfsetstrokecolor{textcolor}%
\pgfsetfillcolor{textcolor}%
\pgftext[x=4.850390in, y=0.949927in, left, base]{\color{textcolor}\sffamily\fontsize{9.000000}{10.800000}\selectfont edges}%
\end{pgfscope}%
\begin{pgfscope}%
\pgfsetroundcap%
\pgfsetroundjoin%
\pgfsetlinewidth{1.505625pt}%
\definecolor{currentstroke}{rgb}{0.792157,0.568627,0.380392}%
\pgfsetstrokecolor{currentstroke}%
\pgfsetdash{}{0pt}%
\pgfpathmoveto{\pgfqpoint{4.500390in}{0.719165in}}%
\pgfpathlineto{\pgfqpoint{4.625390in}{0.719165in}}%
\pgfpathlineto{\pgfqpoint{4.750390in}{0.719165in}}%
\pgfusepath{stroke}%
\end{pgfscope}%
\begin{pgfscope}%
\definecolor{textcolor}{rgb}{0.150000,0.150000,0.150000}%
\pgfsetstrokecolor{textcolor}%
\pgfsetfillcolor{textcolor}%
\pgftext[x=4.850390in, y=0.762427in, left, base]{\color{textcolor}\sffamily\fontsize{9.000000}{10.800000}\selectfont Add POI}%
\end{pgfscope}%
\begin{pgfscope}%
\definecolor{textcolor}{rgb}{0.150000,0.150000,0.150000}%
\pgfsetstrokecolor{textcolor}%
\pgfsetfillcolor{textcolor}%
\pgftext[x=4.850390in, y=0.618433in, left, base]{\color{textcolor}\sffamily\fontsize{9.000000}{10.800000}\selectfont attributes}%
\end{pgfscope}%
\begin{pgfscope}%
\pgfsetroundcap%
\pgfsetroundjoin%
\pgfsetlinewidth{1.003750pt}%
\definecolor{currentstroke}{rgb}{0.003922,0.450980,0.698039}%
\pgfsetstrokecolor{currentstroke}%
\pgfsetdash{}{0pt}%
\pgfpathmoveto{\pgfqpoint{0.761265in}{2.318490in}}%
\pgfpathlineto{\pgfqpoint{1.091548in}{2.318490in}}%
\pgfpathlineto{\pgfqpoint{1.466389in}{2.318490in}}%
\pgfpathlineto{\pgfqpoint{1.890773in}{2.318490in}}%
\pgfpathlineto{\pgfqpoint{3.157691in}{2.318490in}}%
\pgfpathlineto{\pgfqpoint{4.127042in}{2.318490in}}%
\pgfusepath{stroke}%
\end{pgfscope}%
\begin{pgfscope}%
\pgfsetbuttcap%
\pgfsetroundjoin%
\definecolor{currentfill}{rgb}{0.003922,0.450980,0.698039}%
\pgfsetfillcolor{currentfill}%
\pgfsetlinewidth{0.752812pt}%
\definecolor{currentstroke}{rgb}{1.000000,1.000000,1.000000}%
\pgfsetstrokecolor{currentstroke}%
\pgfsetdash{}{0pt}%
\pgfsys@defobject{currentmarker}{\pgfqpoint{-0.034722in}{-0.034722in}}{\pgfqpoint{0.034722in}{0.034722in}}{%
\pgfpathmoveto{\pgfqpoint{0.000000in}{-0.034722in}}%
\pgfpathcurveto{\pgfqpoint{0.009208in}{-0.034722in}}{\pgfqpoint{0.018041in}{-0.031064in}}{\pgfqpoint{0.024552in}{-0.024552in}}%
\pgfpathcurveto{\pgfqpoint{0.031064in}{-0.018041in}}{\pgfqpoint{0.034722in}{-0.009208in}}{\pgfqpoint{0.034722in}{0.000000in}}%
\pgfpathcurveto{\pgfqpoint{0.034722in}{0.009208in}}{\pgfqpoint{0.031064in}{0.018041in}}{\pgfqpoint{0.024552in}{0.024552in}}%
\pgfpathcurveto{\pgfqpoint{0.018041in}{0.031064in}}{\pgfqpoint{0.009208in}{0.034722in}}{\pgfqpoint{0.000000in}{0.034722in}}%
\pgfpathcurveto{\pgfqpoint{-0.009208in}{0.034722in}}{\pgfqpoint{-0.018041in}{0.031064in}}{\pgfqpoint{-0.024552in}{0.024552in}}%
\pgfpathcurveto{\pgfqpoint{-0.031064in}{0.018041in}}{\pgfqpoint{-0.034722in}{0.009208in}}{\pgfqpoint{-0.034722in}{0.000000in}}%
\pgfpathcurveto{\pgfqpoint{-0.034722in}{-0.009208in}}{\pgfqpoint{-0.031064in}{-0.018041in}}{\pgfqpoint{-0.024552in}{-0.024552in}}%
\pgfpathcurveto{\pgfqpoint{-0.018041in}{-0.031064in}}{\pgfqpoint{-0.009208in}{-0.034722in}}{\pgfqpoint{0.000000in}{-0.034722in}}%
\pgfpathlineto{\pgfqpoint{0.000000in}{-0.034722in}}%
\pgfpathclose%
\pgfusepath{stroke,fill}%
}%
\begin{pgfscope}%
\pgfsys@transformshift{0.761265in}{2.318490in}%
\pgfsys@useobject{currentmarker}{}%
\end{pgfscope}%
\begin{pgfscope}%
\pgfsys@transformshift{1.091548in}{2.318490in}%
\pgfsys@useobject{currentmarker}{}%
\end{pgfscope}%
\begin{pgfscope}%
\pgfsys@transformshift{1.466389in}{2.318490in}%
\pgfsys@useobject{currentmarker}{}%
\end{pgfscope}%
\begin{pgfscope}%
\pgfsys@transformshift{1.890773in}{2.318490in}%
\pgfsys@useobject{currentmarker}{}%
\end{pgfscope}%
\begin{pgfscope}%
\pgfsys@transformshift{3.157691in}{2.318490in}%
\pgfsys@useobject{currentmarker}{}%
\end{pgfscope}%
\begin{pgfscope}%
\pgfsys@transformshift{4.127042in}{2.318490in}%
\pgfsys@useobject{currentmarker}{}%
\end{pgfscope}%
\end{pgfscope}%
\begin{pgfscope}%
\pgfsetroundcap%
\pgfsetroundjoin%
\pgfsetlinewidth{1.003750pt}%
\definecolor{currentstroke}{rgb}{0.870588,0.560784,0.019608}%
\pgfsetstrokecolor{currentstroke}%
\pgfsetdash{}{0pt}%
\pgfpathmoveto{\pgfqpoint{0.761265in}{2.182346in}}%
\pgfpathlineto{\pgfqpoint{1.091548in}{2.208468in}}%
\pgfpathlineto{\pgfqpoint{1.466389in}{2.258117in}}%
\pgfpathlineto{\pgfqpoint{1.890773in}{2.273853in}}%
\pgfpathlineto{\pgfqpoint{3.157691in}{2.285173in}}%
\pgfpathlineto{\pgfqpoint{4.127042in}{2.291692in}}%
\pgfusepath{stroke}%
\end{pgfscope}%
\begin{pgfscope}%
\pgfsetbuttcap%
\pgfsetroundjoin%
\definecolor{currentfill}{rgb}{0.870588,0.560784,0.019608}%
\pgfsetfillcolor{currentfill}%
\pgfsetlinewidth{0.752812pt}%
\definecolor{currentstroke}{rgb}{1.000000,1.000000,1.000000}%
\pgfsetstrokecolor{currentstroke}%
\pgfsetdash{}{0pt}%
\pgfsys@defobject{currentmarker}{\pgfqpoint{-0.034722in}{-0.034722in}}{\pgfqpoint{0.034722in}{0.034722in}}{%
\pgfpathmoveto{\pgfqpoint{0.000000in}{-0.034722in}}%
\pgfpathcurveto{\pgfqpoint{0.009208in}{-0.034722in}}{\pgfqpoint{0.018041in}{-0.031064in}}{\pgfqpoint{0.024552in}{-0.024552in}}%
\pgfpathcurveto{\pgfqpoint{0.031064in}{-0.018041in}}{\pgfqpoint{0.034722in}{-0.009208in}}{\pgfqpoint{0.034722in}{0.000000in}}%
\pgfpathcurveto{\pgfqpoint{0.034722in}{0.009208in}}{\pgfqpoint{0.031064in}{0.018041in}}{\pgfqpoint{0.024552in}{0.024552in}}%
\pgfpathcurveto{\pgfqpoint{0.018041in}{0.031064in}}{\pgfqpoint{0.009208in}{0.034722in}}{\pgfqpoint{0.000000in}{0.034722in}}%
\pgfpathcurveto{\pgfqpoint{-0.009208in}{0.034722in}}{\pgfqpoint{-0.018041in}{0.031064in}}{\pgfqpoint{-0.024552in}{0.024552in}}%
\pgfpathcurveto{\pgfqpoint{-0.031064in}{0.018041in}}{\pgfqpoint{-0.034722in}{0.009208in}}{\pgfqpoint{-0.034722in}{0.000000in}}%
\pgfpathcurveto{\pgfqpoint{-0.034722in}{-0.009208in}}{\pgfqpoint{-0.031064in}{-0.018041in}}{\pgfqpoint{-0.024552in}{-0.024552in}}%
\pgfpathcurveto{\pgfqpoint{-0.018041in}{-0.031064in}}{\pgfqpoint{-0.009208in}{-0.034722in}}{\pgfqpoint{0.000000in}{-0.034722in}}%
\pgfpathlineto{\pgfqpoint{0.000000in}{-0.034722in}}%
\pgfpathclose%
\pgfusepath{stroke,fill}%
}%
\begin{pgfscope}%
\pgfsys@transformshift{0.761265in}{2.182346in}%
\pgfsys@useobject{currentmarker}{}%
\end{pgfscope}%
\begin{pgfscope}%
\pgfsys@transformshift{1.091548in}{2.208468in}%
\pgfsys@useobject{currentmarker}{}%
\end{pgfscope}%
\begin{pgfscope}%
\pgfsys@transformshift{1.466389in}{2.258117in}%
\pgfsys@useobject{currentmarker}{}%
\end{pgfscope}%
\begin{pgfscope}%
\pgfsys@transformshift{1.890773in}{2.273853in}%
\pgfsys@useobject{currentmarker}{}%
\end{pgfscope}%
\begin{pgfscope}%
\pgfsys@transformshift{3.157691in}{2.285173in}%
\pgfsys@useobject{currentmarker}{}%
\end{pgfscope}%
\begin{pgfscope}%
\pgfsys@transformshift{4.127042in}{2.291692in}%
\pgfsys@useobject{currentmarker}{}%
\end{pgfscope}%
\end{pgfscope}%
\begin{pgfscope}%
\pgfsetroundcap%
\pgfsetroundjoin%
\pgfsetlinewidth{1.003750pt}%
\definecolor{currentstroke}{rgb}{0.007843,0.619608,0.450980}%
\pgfsetstrokecolor{currentstroke}%
\pgfsetdash{}{0pt}%
\pgfpathmoveto{\pgfqpoint{0.761265in}{1.650967in}}%
\pgfpathlineto{\pgfqpoint{1.091548in}{1.544641in}}%
\pgfpathlineto{\pgfqpoint{1.466389in}{1.583228in}}%
\pgfpathlineto{\pgfqpoint{1.890773in}{1.580172in}}%
\pgfpathlineto{\pgfqpoint{3.157691in}{1.513270in}}%
\pgfpathlineto{\pgfqpoint{4.127042in}{1.470914in}}%
\pgfusepath{stroke}%
\end{pgfscope}%
\begin{pgfscope}%
\pgfsetbuttcap%
\pgfsetroundjoin%
\definecolor{currentfill}{rgb}{0.007843,0.619608,0.450980}%
\pgfsetfillcolor{currentfill}%
\pgfsetlinewidth{0.752812pt}%
\definecolor{currentstroke}{rgb}{1.000000,1.000000,1.000000}%
\pgfsetstrokecolor{currentstroke}%
\pgfsetdash{}{0pt}%
\pgfsys@defobject{currentmarker}{\pgfqpoint{-0.034722in}{-0.034722in}}{\pgfqpoint{0.034722in}{0.034722in}}{%
\pgfpathmoveto{\pgfqpoint{0.000000in}{-0.034722in}}%
\pgfpathcurveto{\pgfqpoint{0.009208in}{-0.034722in}}{\pgfqpoint{0.018041in}{-0.031064in}}{\pgfqpoint{0.024552in}{-0.024552in}}%
\pgfpathcurveto{\pgfqpoint{0.031064in}{-0.018041in}}{\pgfqpoint{0.034722in}{-0.009208in}}{\pgfqpoint{0.034722in}{0.000000in}}%
\pgfpathcurveto{\pgfqpoint{0.034722in}{0.009208in}}{\pgfqpoint{0.031064in}{0.018041in}}{\pgfqpoint{0.024552in}{0.024552in}}%
\pgfpathcurveto{\pgfqpoint{0.018041in}{0.031064in}}{\pgfqpoint{0.009208in}{0.034722in}}{\pgfqpoint{0.000000in}{0.034722in}}%
\pgfpathcurveto{\pgfqpoint{-0.009208in}{0.034722in}}{\pgfqpoint{-0.018041in}{0.031064in}}{\pgfqpoint{-0.024552in}{0.024552in}}%
\pgfpathcurveto{\pgfqpoint{-0.031064in}{0.018041in}}{\pgfqpoint{-0.034722in}{0.009208in}}{\pgfqpoint{-0.034722in}{0.000000in}}%
\pgfpathcurveto{\pgfqpoint{-0.034722in}{-0.009208in}}{\pgfqpoint{-0.031064in}{-0.018041in}}{\pgfqpoint{-0.024552in}{-0.024552in}}%
\pgfpathcurveto{\pgfqpoint{-0.018041in}{-0.031064in}}{\pgfqpoint{-0.009208in}{-0.034722in}}{\pgfqpoint{0.000000in}{-0.034722in}}%
\pgfpathlineto{\pgfqpoint{0.000000in}{-0.034722in}}%
\pgfpathclose%
\pgfusepath{stroke,fill}%
}%
\begin{pgfscope}%
\pgfsys@transformshift{0.761265in}{1.650967in}%
\pgfsys@useobject{currentmarker}{}%
\end{pgfscope}%
\begin{pgfscope}%
\pgfsys@transformshift{1.091548in}{1.544641in}%
\pgfsys@useobject{currentmarker}{}%
\end{pgfscope}%
\begin{pgfscope}%
\pgfsys@transformshift{1.466389in}{1.583228in}%
\pgfsys@useobject{currentmarker}{}%
\end{pgfscope}%
\begin{pgfscope}%
\pgfsys@transformshift{1.890773in}{1.580172in}%
\pgfsys@useobject{currentmarker}{}%
\end{pgfscope}%
\begin{pgfscope}%
\pgfsys@transformshift{3.157691in}{1.513270in}%
\pgfsys@useobject{currentmarker}{}%
\end{pgfscope}%
\begin{pgfscope}%
\pgfsys@transformshift{4.127042in}{1.470914in}%
\pgfsys@useobject{currentmarker}{}%
\end{pgfscope}%
\end{pgfscope}%
\begin{pgfscope}%
\pgfsetroundcap%
\pgfsetroundjoin%
\pgfsetlinewidth{1.003750pt}%
\definecolor{currentstroke}{rgb}{0.835294,0.368627,0.000000}%
\pgfsetstrokecolor{currentstroke}%
\pgfsetdash{}{0pt}%
\pgfpathmoveto{\pgfqpoint{0.761265in}{1.505470in}}%
\pgfpathlineto{\pgfqpoint{1.091548in}{1.414033in}}%
\pgfpathlineto{\pgfqpoint{1.466389in}{1.325327in}}%
\pgfpathlineto{\pgfqpoint{1.890773in}{1.290878in}}%
\pgfpathlineto{\pgfqpoint{3.157691in}{1.234099in}}%
\pgfpathlineto{\pgfqpoint{4.127042in}{1.205642in}}%
\pgfusepath{stroke}%
\end{pgfscope}%
\begin{pgfscope}%
\pgfsetbuttcap%
\pgfsetroundjoin%
\definecolor{currentfill}{rgb}{0.835294,0.368627,0.000000}%
\pgfsetfillcolor{currentfill}%
\pgfsetlinewidth{0.752812pt}%
\definecolor{currentstroke}{rgb}{1.000000,1.000000,1.000000}%
\pgfsetstrokecolor{currentstroke}%
\pgfsetdash{}{0pt}%
\pgfsys@defobject{currentmarker}{\pgfqpoint{-0.034722in}{-0.034722in}}{\pgfqpoint{0.034722in}{0.034722in}}{%
\pgfpathmoveto{\pgfqpoint{0.000000in}{-0.034722in}}%
\pgfpathcurveto{\pgfqpoint{0.009208in}{-0.034722in}}{\pgfqpoint{0.018041in}{-0.031064in}}{\pgfqpoint{0.024552in}{-0.024552in}}%
\pgfpathcurveto{\pgfqpoint{0.031064in}{-0.018041in}}{\pgfqpoint{0.034722in}{-0.009208in}}{\pgfqpoint{0.034722in}{0.000000in}}%
\pgfpathcurveto{\pgfqpoint{0.034722in}{0.009208in}}{\pgfqpoint{0.031064in}{0.018041in}}{\pgfqpoint{0.024552in}{0.024552in}}%
\pgfpathcurveto{\pgfqpoint{0.018041in}{0.031064in}}{\pgfqpoint{0.009208in}{0.034722in}}{\pgfqpoint{0.000000in}{0.034722in}}%
\pgfpathcurveto{\pgfqpoint{-0.009208in}{0.034722in}}{\pgfqpoint{-0.018041in}{0.031064in}}{\pgfqpoint{-0.024552in}{0.024552in}}%
\pgfpathcurveto{\pgfqpoint{-0.031064in}{0.018041in}}{\pgfqpoint{-0.034722in}{0.009208in}}{\pgfqpoint{-0.034722in}{0.000000in}}%
\pgfpathcurveto{\pgfqpoint{-0.034722in}{-0.009208in}}{\pgfqpoint{-0.031064in}{-0.018041in}}{\pgfqpoint{-0.024552in}{-0.024552in}}%
\pgfpathcurveto{\pgfqpoint{-0.018041in}{-0.031064in}}{\pgfqpoint{-0.009208in}{-0.034722in}}{\pgfqpoint{0.000000in}{-0.034722in}}%
\pgfpathlineto{\pgfqpoint{0.000000in}{-0.034722in}}%
\pgfpathclose%
\pgfusepath{stroke,fill}%
}%
\begin{pgfscope}%
\pgfsys@transformshift{0.761265in}{1.505470in}%
\pgfsys@useobject{currentmarker}{}%
\end{pgfscope}%
\begin{pgfscope}%
\pgfsys@transformshift{1.091548in}{1.414033in}%
\pgfsys@useobject{currentmarker}{}%
\end{pgfscope}%
\begin{pgfscope}%
\pgfsys@transformshift{1.466389in}{1.325327in}%
\pgfsys@useobject{currentmarker}{}%
\end{pgfscope}%
\begin{pgfscope}%
\pgfsys@transformshift{1.890773in}{1.290878in}%
\pgfsys@useobject{currentmarker}{}%
\end{pgfscope}%
\begin{pgfscope}%
\pgfsys@transformshift{3.157691in}{1.234099in}%
\pgfsys@useobject{currentmarker}{}%
\end{pgfscope}%
\begin{pgfscope}%
\pgfsys@transformshift{4.127042in}{1.205642in}%
\pgfsys@useobject{currentmarker}{}%
\end{pgfscope}%
\end{pgfscope}%
\begin{pgfscope}%
\pgfsetroundcap%
\pgfsetroundjoin%
\pgfsetlinewidth{1.003750pt}%
\definecolor{currentstroke}{rgb}{0.800000,0.470588,0.737255}%
\pgfsetstrokecolor{currentstroke}%
\pgfsetdash{}{0pt}%
\pgfpathmoveto{\pgfqpoint{0.761265in}{2.151934in}}%
\pgfpathlineto{\pgfqpoint{1.091548in}{2.134732in}}%
\pgfpathlineto{\pgfqpoint{1.466389in}{2.030423in}}%
\pgfpathlineto{\pgfqpoint{1.890773in}{1.971947in}}%
\pgfpathlineto{\pgfqpoint{3.157691in}{1.923719in}}%
\pgfpathlineto{\pgfqpoint{4.127042in}{1.885557in}}%
\pgfusepath{stroke}%
\end{pgfscope}%
\begin{pgfscope}%
\pgfsetbuttcap%
\pgfsetroundjoin%
\definecolor{currentfill}{rgb}{0.800000,0.470588,0.737255}%
\pgfsetfillcolor{currentfill}%
\pgfsetlinewidth{0.752812pt}%
\definecolor{currentstroke}{rgb}{1.000000,1.000000,1.000000}%
\pgfsetstrokecolor{currentstroke}%
\pgfsetdash{}{0pt}%
\pgfsys@defobject{currentmarker}{\pgfqpoint{-0.034722in}{-0.034722in}}{\pgfqpoint{0.034722in}{0.034722in}}{%
\pgfpathmoveto{\pgfqpoint{0.000000in}{-0.034722in}}%
\pgfpathcurveto{\pgfqpoint{0.009208in}{-0.034722in}}{\pgfqpoint{0.018041in}{-0.031064in}}{\pgfqpoint{0.024552in}{-0.024552in}}%
\pgfpathcurveto{\pgfqpoint{0.031064in}{-0.018041in}}{\pgfqpoint{0.034722in}{-0.009208in}}{\pgfqpoint{0.034722in}{0.000000in}}%
\pgfpathcurveto{\pgfqpoint{0.034722in}{0.009208in}}{\pgfqpoint{0.031064in}{0.018041in}}{\pgfqpoint{0.024552in}{0.024552in}}%
\pgfpathcurveto{\pgfqpoint{0.018041in}{0.031064in}}{\pgfqpoint{0.009208in}{0.034722in}}{\pgfqpoint{0.000000in}{0.034722in}}%
\pgfpathcurveto{\pgfqpoint{-0.009208in}{0.034722in}}{\pgfqpoint{-0.018041in}{0.031064in}}{\pgfqpoint{-0.024552in}{0.024552in}}%
\pgfpathcurveto{\pgfqpoint{-0.031064in}{0.018041in}}{\pgfqpoint{-0.034722in}{0.009208in}}{\pgfqpoint{-0.034722in}{0.000000in}}%
\pgfpathcurveto{\pgfqpoint{-0.034722in}{-0.009208in}}{\pgfqpoint{-0.031064in}{-0.018041in}}{\pgfqpoint{-0.024552in}{-0.024552in}}%
\pgfpathcurveto{\pgfqpoint{-0.018041in}{-0.031064in}}{\pgfqpoint{-0.009208in}{-0.034722in}}{\pgfqpoint{0.000000in}{-0.034722in}}%
\pgfpathlineto{\pgfqpoint{0.000000in}{-0.034722in}}%
\pgfpathclose%
\pgfusepath{stroke,fill}%
}%
\begin{pgfscope}%
\pgfsys@transformshift{0.761265in}{2.151934in}%
\pgfsys@useobject{currentmarker}{}%
\end{pgfscope}%
\begin{pgfscope}%
\pgfsys@transformshift{1.091548in}{2.134732in}%
\pgfsys@useobject{currentmarker}{}%
\end{pgfscope}%
\begin{pgfscope}%
\pgfsys@transformshift{1.466389in}{2.030423in}%
\pgfsys@useobject{currentmarker}{}%
\end{pgfscope}%
\begin{pgfscope}%
\pgfsys@transformshift{1.890773in}{1.971947in}%
\pgfsys@useobject{currentmarker}{}%
\end{pgfscope}%
\begin{pgfscope}%
\pgfsys@transformshift{3.157691in}{1.923719in}%
\pgfsys@useobject{currentmarker}{}%
\end{pgfscope}%
\begin{pgfscope}%
\pgfsys@transformshift{4.127042in}{1.885557in}%
\pgfsys@useobject{currentmarker}{}%
\end{pgfscope}%
\end{pgfscope}%
\begin{pgfscope}%
\pgfsetroundcap%
\pgfsetroundjoin%
\pgfsetlinewidth{1.003750pt}%
\definecolor{currentstroke}{rgb}{0.792157,0.568627,0.380392}%
\pgfsetstrokecolor{currentstroke}%
\pgfsetdash{}{0pt}%
\pgfpathmoveto{\pgfqpoint{0.761265in}{0.906338in}}%
\pgfpathlineto{\pgfqpoint{1.091548in}{0.780249in}}%
\pgfpathlineto{\pgfqpoint{1.466389in}{0.718713in}}%
\pgfpathlineto{\pgfqpoint{1.890773in}{0.802233in}}%
\pgfpathlineto{\pgfqpoint{3.157691in}{0.633730in}}%
\pgfpathlineto{\pgfqpoint{4.127042in}{0.598356in}}%
\pgfusepath{stroke}%
\end{pgfscope}%
\begin{pgfscope}%
\pgfsetbuttcap%
\pgfsetroundjoin%
\definecolor{currentfill}{rgb}{0.792157,0.568627,0.380392}%
\pgfsetfillcolor{currentfill}%
\pgfsetlinewidth{0.752812pt}%
\definecolor{currentstroke}{rgb}{1.000000,1.000000,1.000000}%
\pgfsetstrokecolor{currentstroke}%
\pgfsetdash{}{0pt}%
\pgfsys@defobject{currentmarker}{\pgfqpoint{-0.034722in}{-0.034722in}}{\pgfqpoint{0.034722in}{0.034722in}}{%
\pgfpathmoveto{\pgfqpoint{0.000000in}{-0.034722in}}%
\pgfpathcurveto{\pgfqpoint{0.009208in}{-0.034722in}}{\pgfqpoint{0.018041in}{-0.031064in}}{\pgfqpoint{0.024552in}{-0.024552in}}%
\pgfpathcurveto{\pgfqpoint{0.031064in}{-0.018041in}}{\pgfqpoint{0.034722in}{-0.009208in}}{\pgfqpoint{0.034722in}{0.000000in}}%
\pgfpathcurveto{\pgfqpoint{0.034722in}{0.009208in}}{\pgfqpoint{0.031064in}{0.018041in}}{\pgfqpoint{0.024552in}{0.024552in}}%
\pgfpathcurveto{\pgfqpoint{0.018041in}{0.031064in}}{\pgfqpoint{0.009208in}{0.034722in}}{\pgfqpoint{0.000000in}{0.034722in}}%
\pgfpathcurveto{\pgfqpoint{-0.009208in}{0.034722in}}{\pgfqpoint{-0.018041in}{0.031064in}}{\pgfqpoint{-0.024552in}{0.024552in}}%
\pgfpathcurveto{\pgfqpoint{-0.031064in}{0.018041in}}{\pgfqpoint{-0.034722in}{0.009208in}}{\pgfqpoint{-0.034722in}{0.000000in}}%
\pgfpathcurveto{\pgfqpoint{-0.034722in}{-0.009208in}}{\pgfqpoint{-0.031064in}{-0.018041in}}{\pgfqpoint{-0.024552in}{-0.024552in}}%
\pgfpathcurveto{\pgfqpoint{-0.018041in}{-0.031064in}}{\pgfqpoint{-0.009208in}{-0.034722in}}{\pgfqpoint{0.000000in}{-0.034722in}}%
\pgfpathlineto{\pgfqpoint{0.000000in}{-0.034722in}}%
\pgfpathclose%
\pgfusepath{stroke,fill}%
}%
\begin{pgfscope}%
\pgfsys@transformshift{0.761265in}{0.906338in}%
\pgfsys@useobject{currentmarker}{}%
\end{pgfscope}%
\begin{pgfscope}%
\pgfsys@transformshift{1.091548in}{0.780249in}%
\pgfsys@useobject{currentmarker}{}%
\end{pgfscope}%
\begin{pgfscope}%
\pgfsys@transformshift{1.466389in}{0.718713in}%
\pgfsys@useobject{currentmarker}{}%
\end{pgfscope}%
\begin{pgfscope}%
\pgfsys@transformshift{1.890773in}{0.802233in}%
\pgfsys@useobject{currentmarker}{}%
\end{pgfscope}%
\begin{pgfscope}%
\pgfsys@transformshift{3.157691in}{0.633730in}%
\pgfsys@useobject{currentmarker}{}%
\end{pgfscope}%
\begin{pgfscope}%
\pgfsys@transformshift{4.127042in}{0.598356in}%
\pgfsys@useobject{currentmarker}{}%
\end{pgfscope}%
\end{pgfscope}%
\end{pgfpicture}%
\makeatother%
\endgroup%

%					\end{figcenter}
%					\caption{Relative share of each task on the total import time.}
%				\end{subfigure}
			\end{figcenter}
			\caption{Details of graph generation times for the two datasets \enquote{OSM city} (above) and \enquote{OSM rural} (below).}
			\label{fig:eval-import-details}
		\end{figure}
	
	\subsection{Routing}
	
\section{Routing evaluation}

	\subsection{Correctness}
	
		% is the shortest route really the shortest

	\subsection{Usefulness of routes}
	
		% How realistic are the routes (= can I go there in real life)? If not: Why not?
		% How much shorter are the routes?