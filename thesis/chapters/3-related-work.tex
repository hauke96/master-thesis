% !TEX root = ../thesis.tex
% !TeX spellcheck = en_US

Scientific literature contains related work from many different areas, mostly from geodata, graph construction and networks as well as routing and (pedestrian) path planning.
This section gives an overview about closely related work from these areas.

\section{Graphs and networks}

	% with some assumptions like no dollinear vertices (even though these assumptions are not necessary): Constructing the visibility graph for n-line segments in O(n²) time (Emo Welzl, 1985)
	Constructing a visibility graph is an important part of this thesis.
	The construction of visibility graphs in $\bigo{n^2}$ time and space (with $n$ line segments), under the condition of no collinear coordinates and no intersecting line segments, was presented by Welzl in 1985\cite{welzl-visibility-graph}.
	A few years later, Overmars and Welzl presented two new methods reducing the space complexity to $\bigo{n}$\cite{overmars-weizl-visibility-graph}.
	Their new algorithms are based on Welzls earlier paper and the idea of a rotational sweep through neighboring vertices.
	
	% maybe, they use geom. routing to fill OSM gaps: Automatic extrapolation of missing road network data in OpenStreetMap (Funke, Schirrmeister, Storandt, 2015)

\section{Routing}

	% Shortest paths among obstacles in the plane (Mitchell, 1993)
	Shortly after Welzl and others published their work on visibility graphs, which can be used in combination with Dijkstra to find shortest paths, Mitchell presented a pure geometric method to find euclidean shortest paths around obstacles in the plain in 1993\cite{mitchell-shortest-path}.
	He used wavefronts propagating from the source towards the target vertex.
	Each origin of a wavefront can be traces back to the source giving the shortest path.
	Because this algorithm solves the \term*{single-source shortest-paths} problem just like the Dijkstra algorithm does on a network, this technique is also called the \term{continuous dijkstra} paradigm.
	
	\todo[inline]{Normal routing: Dijkstra, A*, and others?}
	
	% An Optimal Algorithm for Euclidean Shortest Paths in the Plane (Hershberger, Suri, 1999)
	Another approach solving the geometric \term*{single-source shortest-path} problem was presented by Hershberger and Suri one year later, in 1997.
	They also use the \term*{continuous dijkstra} paradigm but increase performance by introducing a special quad-tree like structure and smart subdivision of the plain.
	The result is then actually a path map allowing shortest-path request to be processed in only $\bigo{n \log n}$.
	
	% maybe: Efficient algorithms for Euclidean shortest path and visibility problems with polygonal obstacles (Kapoor, Maheshwari, 1988)

\section{Pedestrian path finding}
	
	A special case of the general \term[single-source shortest-path]{shortest-path} problem is pedestrian routing, also called pedestrian shortest-path finding.
	This problem is especially interesting since pedestrians have the maximum degree of flexibility/freedom compared to other traffic participants like bikes or cars.
	Therefore, geometric routing is a good way to find shortest-paths for pedestrians.
	
	Pedestrian path finding is, however, not only relevant for real world users, who want to know how to get from one location to another.
	Such algorithms are also used in (multi) agent simulations of pedestrian behavior.
	Scenarios in such simulations are for example evacuation or crowd behaviors.
	The area of pedestrian path planning and simulations is divided into two strategies: Network and field based path finding\cite[2]{hartmann-geodesic}.
	Work from both strategies will be covered here.
	
	\subsection{Field based navigation}
		
		% online routing: A Navigation Algorithm for Pedestrian Simulation in Dynamic Environments (Teknomo, Millonig, 2007)
		Teknomo and Millonig created a dynamic algorithm which does not pre-compute the path with one of the above mentioned approaches\cite{teknomo-millonig-routing}.
		This is interesting, since the probably most common and naive approach is to pre-compute the route used by an agent within a simulation.
		The term \enquote{dynamic} refers to the fact, that real world pedestrians often only consider the near neighborhood within their path planning.
		Even if a person knows the hole area, dynamic changes (closed doors, construction work, crowds blocking the way) can still occur at any point in time.
		Therefore, such dynamic considerations can be one aspect to make simulations more realistic.
		
		% maybe: pure geometric/geodesic approach with navigation fields: https://iopscience.iop.org/article/10.1088/1367-2630/12/4/043032/pdf
		The above mentioned dynamic routing by Teknomo and Millonig has strong similarities to approaches based on potential fields, like the dynamic path finding from Hartmann, 2010\cite{hartmann-geodesic}.
		He proposed a mechanism using vector fields in a cellular automaton model with hexagonal cells.
		Interestingly, this method can be interpreted as wavefronts, like in Mitchells algorithm described above, propagating through the vector field\cite[4]{hartmann-geodesic}.
			
		% unified pedestrian routing (they use the above navigation field graph generation): A Unified Pedestrian Routing Model for Graph-Based Wayfinding Built on Cognitive Principles (Kielar, et al. 2017)
	
	\subsection{Graph generation for pedestrian navigation}
	
		An alternative to field based approaches are graph based ones, where a graph is constructed and then used in navigation.
		The construction can be based on potential fields but also on the above mentioned \term*[visibility graph]{visibility graphs}.
		
		In fact, Gloor, Stucki and Nagel presented a pedestrian navigation model and a comparison of these two strategies regarding their performance\cite{gloor-hybrid-pedestrian-routing}.
		They developed this model for a pedestrian simulation in the Alps, where a hiking network should be enhanced with additional navigation information.
		Two approaches were presented: A potential field based method, which is used in navigation when needed, and a visibility graph, which is merged into the existing hiking network.
		The visibility graph strategy strongly relates to this thesis.
		For the performance evaluation, they used an evacuation scenario.
		The results showed, that both navigation mechanisms are comparably fast.
		
		A very similar approach, at least with visibility graphs as a base, was chosen by Kneidl, Borrmann and Hartmann to produce a sparse navigation graph\cite[5]{kneidl-borrmann-hartmann-navigation}.
		They chose the construction of a visibility graph over generalized Voronoi diagrams for more realistic edges.
		
		Another interesting paper on enhancing existing navigation graphs, like Gloor, Stucki and Nagel did, was presented in 2016 by Anita Graser by integrating visibility graph edges into the existing OSM data model\cite{graser-osm-open-spaces}.
		Again, the visibility graph was the preferred method of producing edges.
		Considered and discussed alternatives here were two skeletonization methods and a simple grid places over open spaces.