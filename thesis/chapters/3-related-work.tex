% !TEX root = ../thesis.tex
% !TeX spellcheck = en_US

Scientific literature contains related work from many different areas, mostly from geodata, graph construction and networks as well as routing and (pedestrian) path planning.
This section gives an overview about closely related work from these areas.

\section{Graphs and networks}

	% TODO Add other paper not based on visibility graph generation?

	\subsection{Visibility graphs}
	\label{subsec:related-work:visibility-graph}

		Constructing a visibility graph is an important part of this thesis.
		Even though numerous approaches to create such graph were presented over the last decades, the worst-case time complexity of this problem will always be quadratic.
		This is due to the worst case graph where all vertices are visible to each other resulting in $\Omega(n^2)$ (precisely $|E| \leq \frac{n^2-n}{n}$) many edges in the output graph.
		
		% with some assumptions like no collinear vertices (even though these assumptions are not necessary): Constructing the visibility graph for n-line segments in O(n²) time (Emo Welzl, 1985)
		The construction of visibility graphs in $\bigo{n^2}$ time and space (with $n$ line segments), under the condition of no collinear vertices and no intersecting line segments, was presented by Welzl in 1985\cite{welzl-visibility-graph}.
		
		A few years later in 1988, Overmars and Welzl presented two new methods reducing the space complexity to $\bigo{n}$ \cite{overmars-weizl-visibility-graph} and one of these methods has a time complexity of $\bigo{E \log n}$ with $E$ being the number of output edges.
		Their new algorithms are based on Welzls earlier paper and implement the idea of a rotational sweep through neighboring vertices.
		
		The time complexity of Overmars and Welzls approach was already output sensitive by the number of edges in the output graph \cite{ghosh-output-sensitive-vgraph}.
		Ghosh and Mount presented another output sensitive algorithm but with $\bigo{E + n \log n}$, which is a significant improvement, since $E$ is in $\bigo{n^2}$.
		However, their approach uses $\bigo{E + n}$ space compared to $\bigo{n}$ from the earlier algorithm.
		Ghosh and Mount use plain-sweeps to create a triangulation of the free space and introduced so called \emph{funnel sequences}, which are sets of vertices visible to parts of an edge.
		To avoid numerous edge cases, their approach assumes no collinear vertices and unique x-coordinates.
		
		Kapoor and Maheshwari introduced algorithms based on triangulations as well \cite{kapoor-shortest-path-vgraph}, but only use a certain subset of the visibility graph to actually determine the shortest path.
		Their approach to create the visibility graph has the same complexity as Ghosh and Mounts, but is simpler structured and therefore might lead to a better performance in practice.
		
		% evtl. noch: A NEW ALGORITHM FOR COMPUTING VISIBILITY GRAPHS OF POLYGONAL OBSTACLES IN THE PLANE

\section{Routing}

	As done in the previous chapter, this section covers geometric and network based routing.
	Related work regarding specific routing techniques for pedestrian navigation is described below in \cref{sec:pedestrian-path-planning}.
	
	\subsection{Geometric routing}
	
		% Shortest paths among obstacles in the plane (Mitchell, 1993)
		\label{related-work:mitchell}
		Shortly after Welzl and others published their work on visibility graphs, which can be used in combination with Dijkstra to find shortest paths, Mitchell presented a pure geometric method to find euclidean shortest paths around obstacles in the plain in 1993\cite{mitchell-shortest-path}.
		He used wavefronts propagating from the source towards the target vertex.
		Each origin of a wavefront can be traces back to its source giving the shortest path.
		Because this algorithm solves the \hyperref[subsubsec:single-source-shortest-path]{\term*{single-source shortest path} problem}, just like the Dijkstra algorithm does on a network, this technique is also called the \term{continuous dijkstra} paradigm.
		
		% An Optimal Algorithm for Euclidean Shortest Paths in the Plane (Hershberger, Suri, 1999)
		Another approach solving the geometric single-source shortest path problem was presented by Hershberger and Suri one year later, in 1997 \cite{hershberger-suri}.
		They also use the \term*{continuous dijkstra} paradigm but increase performance by introducing a special quad-tree like structure as a smart subdivision of the plane and also introduce approximate wavefronts.
		The difficult part is the collision handling of non neighboring wavelets.
		Maintaining some important properties is crucial to the efficiency when creating the subdivision, most important is the $\bigo{1}$ amount of neighboring cells around an edge.
		Their algorithm calculates a shortest path map allowing queries to be answered in only $\bigo{\log n}$.
		
	\subsection{Network based routing}

\section{Pedestrian path finding}
\label{sec:pedestrian-path-planning}
	
	A special case of the general single-source shortest path problem is pedestrian routing, also called pedestrian shortest-path finding.
	This problem is especially interesting since pedestrians have the maximum degree of flexibility/freedom compared to other traffic participants like bikes or cars.
	Therefore, geometric routing is a good way to find shortest-paths for pedestrians.
	
	Pedestrian path finding is, however, not only relevant for real world users, who want to know how to get from one location to another.
	Such algorithms are also used in (multi) agent simulations of pedestrian behavior.
	Scenarios in such simulations are for example evacuation or crowd behaviors.
	The area of pedestrian path planning and simulations is divided into two strategies: Network and field based path finding \cite[2]{hartmann-geodesic}.
	Work from both strategies will be covered here.
	
	\subsection{Field based and dynamic navigation}
	
		Next to geometric and network based routing, potential fields can be used to find shortest paths within this field.
		
		% online routing: A Navigation Algorithm for Pedestrian Simulation in Dynamic Environments (Teknomo, Millonig, 2007)
		Teknomo and Millonig created a dynamic algorithm which does not pre-compute the path with one of the above mentioned approaches\cite{teknomo-millonig-routing}.
		This is interesting, since the probably most common and naive approach is to pre-compute the route used by an agent within a simulation.
		The term \enquote{dynamic} refers to the fact, that real world pedestrians often only consider the near neighborhood within their path planning.
		Even if a person knows the hole area, dynamic changes (closed doors, construction work, crowds blocking the way) can still occur at any point in time.
		Therefore, considering such dynamic changes make agent based simulations more realistic.
		
		% maybe: pure geometric/geodesic approach with navigation fields: https://iopscience.iop.org/article/10.1088/1367-2630/12/4/043032/pdf
		The above mentioned dynamic routing by Teknomo and Millonig has strong similarities to approaches based on potential fields, like the dynamic path finding from Hartmann, 2010\cite{hartmann-geodesic}.
		He proposed a mechanism using vector fields in a cellular automaton model with hexagonal cells.
		Interestingly, this method can be interpreted as wavefronts, like in Mitchells algorithm described above, propagating through the vector field\cite[4]{hartmann-geodesic}.
			
		% unified pedestrian routing (they use the above navigation field graph generation): A Unified Pedestrian Routing Model for Graph-Based Wayfinding Built on Cognitive Principles (Kielar, et al. 2017)
	
	\subsection{Graph generation for path planning}
	
		An alternative to field based approaches are graph based ones, where a graph is constructed and then used in navigation.
		The construction can be based on potential fields but also on the \hyperref[subsec:related-work:visibility-graph]{above} mentioned \term*[visibility graph]{visibility graphs}.
		
		In fact, Gloor, Stucki and Nagel presented a pedestrian navigation model and a comparison of these two strategies regarding their performance\cite{gloor-hybrid-pedestrian-routing}.
		They developed this model for a pedestrian simulation in the Alps, where a hiking network should be enhanced with additional navigation information.
		Two approaches were presented: A potential field based method, which is used in navigation when needed, and a visibility graph, which is merged into the existing hiking network.
		The visibility graph strategy strongly relates to this thesis.
		For the performance evaluation, they used an evacuation scenario.
		The results showed, that both navigation mechanisms are comparably fast.
		
		A very similar approach, at least with visibility graphs as a base, was chosen by Kneidl, Borrmann and Hartmann to produce a sparse navigation graph \cite[5]{kneidl-borrmann-hartmann-navigation}.
		They chose the construction of a visibility graph over generalized Voronoi diagrams for more realistic edges.
		
		Their work was used by Kielar et al. as a basis for a generalized routing model for pedestrian simulations taking cognitive models of agents into account \cite[5]{kielar-unified-pedestrian-routing}.
		Cognitive models take into account that humans create their own mixed and fuzzy path finding behavior, which differs from optimal mathematical algorithms like Dijkstra.
		Basic behaviors of human crowds are herding and learning from others, which makes communication and observation between agents necessary \cite[4]{kielar-unified-pedestrian-routing}.
		
		Stefan Funke et al. also wanted to generate a navigation graph, but instead of filling open spaces with edges, they wanted to fill holes in an existing road graph, which was the OpenStreetMap network in their case \cite{funke-osm-extrapolation}.
		To do so, they developed a metric to detect holes in an existing graph based on the following idea:
		If the shortest path between two vertices is much longer than the euclidean distance, there is probably a hole in the network.
		However, this metric leads to false positives when large obstacles, such as a river, is between the two vertices.
		In this case, the euclidean and shortest path distances differ a lot, which is in deed correct in such cases.
		To handle such problems caused by obstacles, they discussed the use of Mitchells wavefront propagation algorithm as discussed \hyperref[related-work:mitchell]{above} \cite[4]{funke-osm-extrapolation}.
		Due to the complexity and lower performance, they decided to use a dense grid instead, cutting out edges intersecting obstacles and use this remaining grid for navigation.
		Unfortunately, the use of other graph generation algorithms (for example generating a visibility graph) was not discussed.
		
		Another interesting paper on enhancing existing navigation graphs, like Gloor, Stucki and Nagel did, was presented in 2016 by Anita Graser by integrating visibility graph edges into the existing OSM data model\cite{graser-osm-open-spaces}.
		Again, the visibility graph was the preferred method of producing edges.
		Considered and discussed alternatives here were two skeletonization methods and a simple grid places over open spaces.