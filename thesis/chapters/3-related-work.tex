% !TEX root = ../thesis.tex
% !TeX spellcheck = en_US

Scientific literature contains related work from many different areas, mostly from geodata, graph construction and networks as well as routing and (pedestrian) path planning.
This section gives an overview about closely related work from these areas.

% Geodata and networks
	% with some assumptions like no dollinear vertices (even though these assumptions are not necessary): Constructing the visibility graph for n-line segments in O(n²) time (Emo Welzl, 1985)
Constructing a visibility graph is an important part of this thesis, thus the construction of visibility graphs in $\bigo{n^2}$ time and space (with $n$ line segments), under the condition of no collinear coordinates and no intersecting line segments, was presented by Welzl in 1985\cite{welzl-visibility-graph}.
A few years later, Overmars and Welzl presented two new methods reducing the space complexity to $\bigo{n}$\cite{overmars-weizl-visibility-graph}.
Their new algorithms are based on Welzls earlier paper and the idea of a rotational sweep through neighboring vertices.
	
	% maybe, they use geom. routing to fill OSM gaps: Automatic extrapolation of missing road network data in OpenStreetMap (Funke, Schirrmeister, Storandt, 2015)

% Routing
	% Shortest paths among obstacles in the plane (Mitchell, 1993)
Shortly after Welzl and others published their work on visibility graphs, which can be used in combination with Dijkstra to find shortest paths, Mitchell presented a pure geometric method to find euclidean shortest paths around obstacles in the plain in 1993\cite{mitchell-shortest-path}.
He used wavefronts propagating from the source towards the target vertex.
Each origin of a wavefront can be traces back to the source giving the shortest path.
Because this algorithm solves the \term*{single-source shortest-paths} problem just like the Dijkstra algorithm does on a network, this technique is also called the \term{continuous dijkstra} paradigm.

	% An Optimal Algorithm for Euclidean Shortest Paths in the Plane (Hershberger, Suri, 1999)
	% maybe: Efficient algorithms for Euclidean shortest path and visibility problems with polygonal obstacles (Kapoor, Maheshwari, 1988)

% Pedestrian path finding
	% online routing: A Navigation Algorithm for Pedestrian Simulation in Dynamic Environments (Teknomo, Millonig, 2007)
	% maybe: pure geometric/geodesic approach with navigation fields: https://iopscience.iop.org/article/10.1088/1367-2630/12/4/043032/pdf
	% unified pedestrian routing (they use the above navigation field graph generation): A Unified Pedestrian Routing Model for Graph-Based Wayfinding Built on Cognitive Principles (Kielar, et al. 2017)