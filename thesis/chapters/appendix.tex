% !TEX root = ../thesis.tex
% !TeX spellcheck = en_US

In the following code example are given, showing the entry points of the implementation as well as the usage of the hybrid visibility graph.

\section*{Implementation of the graph generation and routing}

	\subsection*{Graph generation}
		
		In this section, insights on the implementation of the graph generation from the \texttt{HybridVisibilityGraphGenerator} class are given, covering mainly the methods evaluated in \Cref{chap:evaluation}.
		Next to the first and mandatory feature parameter, the $k=(k_b, k_n)$ for the kNN search can be specified with \texttt{visibilityNeighborBinCount} ($k_b$) and \texttt{visibilityNeighborsPerBin} ($k_n$).
		The remaining arguments are regex-like filtering expressions used to get the respective features from the input data.
		
		\begin{minted}{csharp}
public static HybridVisibilityGraph Generate(
	IEnumerable<IFeature> features,
	int visibilityNeighborBinCount = 36,
	int visibilityNeighborsPerBin = 10,
	string[]? obstacleExpressions = null,
	string[]? poiExpressions = null,
	string[]? roadExpressions = null)
{
	features = features.ToList(); // Prevent multiple enumerations
	
	// Getting, unwrapping and triangulatin gobstacles
	var obstacles = GetObstacles(features, obstacleExpressions);
	
	// Performing the kNN search
	var vertexNeighbors = VisibilityGraphGenerator.CalculateVisibleKnn(
		obstacles, visibilityNeighborBinCount, visibilityNeighborsPerBin
	);
	
	// Generate the hybrid visibility graph, which currently only contains
	// visibility edges
	var graph = CreateVisibilityGraph(
		vertexNeighbors, obstacles
	);
	
	// Merge the road edges into the visibility graph
	MergeRoadsIntoGraph(features, graph, roadExpressions);
	
	// Add attributes to POI nodes in the graph
	AddAttributesToPoiNodes(features, graph, poiExpressions);
	
	return graph;
}
		\end{minted}
		
	\subsection*{Routing}
	
		The following code example is a slightly simplified version of the actual routing method that can be used by agents.
		An example for the \texttt{heuristic} can be seen in the usage examples below.
		
		\begin{minted}{csharp}
public List<Position> OptimalPath(Position source, Position destination,
	Func<EdgeData, NodeData, double> heuristic)
{
	var sourceNode = AddPositionToGraph(source);
	var destNode = AddPositionToGraph(destination);
	
	// Connect the nodes to the graph, which creates new nodes and edges
	var (sourceNodes, sourceEdges) = ConnectNodeToGraph(sourceNode, false);
	var (destNodes, destEdges) = ConnectNodeToGraph(destinationNode, false);
	
	var result = Graph.AStarAlgorithm(sourceNode.Key, destNode.Key, heuristic);
	
	// Remove temporarily created nodes (which automatically removes the edges too)
	sourceEdges.Each(RemoveEdge);
	destEdges.Each(RemoveEdge);
	sourceNodes.Each(RemoveNode);
	destNodes.Each(RemoveNode);
	
	// Merge all coordinates of the edges into the final list
	return result
		.Aggregate(new List<Position>(), (list, edge) => { ... })
		.ToList();
}
		\end{minted}

\section*{Usage of the hybrid visibility graph}
	
	\subsection*{Graph generation}
	
		The graph generation itself just requires one method call accepting a list of \texttt{VectorStructuredData} objects representing all features (obstacles and roads) of the input data:
	
		\begin{minted}{csharp}
// Get the VectorStructuredData from the input features
var features = Features.Map(f => f.VectorStructured).ToList();

// Generate the hybrid visibility graph
this.HybridVisibilityGraph = HybridVisibilityGraphGenerator.Generate(features);
		\end{minted}
		
		\noindent
		This code can, for example, be used in a shared class such as a layer in MARS or any other stateful class storing the \texttt{HybridVisibilityGraph} instance.
		
	\subsection*{Routing}
	
		The entry point for routing requests is within the hybrid visibility graph class and therefore can be invoked by a single method call.
		Three methods are available and shown in the following example.
		The predefined weighted shortest path uses a function with a factor of 0.8 on any road edge.
		Using a custom weighting function is also possible, in this example a relatively strong preference for bridleways is defined.
	
		\begin{minted}{csharp}
// Using existing methods:
this.Route = this.HybridVisibilityGraph.ShortestPath(source, dest);
this.Route = this.HybridVisibilityGraph.WeightedShortestPath(source, dest);

// Using a custom weighting function:
var weightingFunction =(edge, _) =>
{
	edge.Data.TryGetValue("highway", out object? highwayAttribute);
	var factor = "bridleway".Equals(highwayAttribute)
		? 0.5
		: 1;
	return edge.Length * factor;
};

this.Route = this.HybridVisibilityGraph.WeightedShortestPath(source, dest);
		\end{minted}