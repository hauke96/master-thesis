% !TEX root = ../thesis.tex
% !TeX spellcheck = en_US

\section{Geospatial data}

	According to the ISO standard 19109:2015\cite{geolexica-202}, the term \textit{geospatial data} refers to "`data with implicit or explicit reference to a location relative to the Earth"'.
	In other words, data with addresses, coordinates or other location references belong to the class of geospatial data.
	
	Typical examples are customer databases, where each customer has an address, or weather measurement data, where each data point has a coordinate, altitude and timestamp.
	Latter therefore belongs to \textit{spatiotemporal data}, because is has an additional time dimension.
	Since addresses are not globally standardized, describe only a vague location and usually belong to a hole area, they are of less importance for this thesis.
	Next to the mentioned explicit references to locations, implicit references can be encoded within metadata of datasets.
	This is for example the case for the GeoTIFF format, storing location data in the matadata of a normal TIFF file\cite{ogc-geotiff}.
	The exact coordinate of a data point (in this case a pixel in the TIFF image), must then be calculated using this matadata.

	\subsection{Coordinate reference systems and projections}

	\subsection{Data structures}
	
	\subsection{File formats}
	
	\subsection{GIS}
	
	\subsection{OpenStreetMap}

\section{Routing}

	% fastest, shortest path, all shortest...
	
	\subsection{Graph based}
	
	\subsection{Geometric}

\section{Agent based systems and simulations}