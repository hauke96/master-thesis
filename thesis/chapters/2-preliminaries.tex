% !TEX root = ../thesis.tex
% !TeX spellcheck = en_US

\section{Geospatial data}

	According to the ISO standard 19109:2015\cite{geolexica-202}, the term \term{geospatial data} refers to \enquote{data with implicit or explicit reference to a location relative to the Earth}.
	In other words, data with addresses, coordinates or other location references belong to the class of geospatial data.
	
	Typical examples are customer databases, where each customer has an address, or weather measurement data, where each data point has a coordinate, altitude and timestamp.
	Latter therefore belongs to \term{spatiotemporal data}, because is has an additional time dimension.
	Since addresses are not globally standardized, describe only a vague location and usually belong to a hole area, they are of less importance for this thesis.
	Next to the mentioned explicit references to locations, implicit references can be encoded within metadata of datasets.
	This is for example the case for the GeoTIFF format, storing location data in the matadata of a normal TIFF file\cite{ogc-geotiff}.
	The exact coordinate of a data point (in this case a pixel in the TIFF image), must then be calculated using this matadata.

	\subsection{Coordinate reference systems and projections}
	
		Usually coordinates are used to specify exact locations in a specific domain (usually on the surface of the Earth).
		Unfortunately the Earth is not a perfect sphere but rather an ellipsoid.
		But even this ellipsoid is not smooth and has slightly different radii in different regions.
		Furthermore, parts of the surface move relative to each other (continental drift) and therefore change the coordinates of whole regions.
		Even though, the continental plates move slowly, this effect adds up to a significant distance rendering accurate measurements potentially useless\cite[7]{ordenance-survey-booklet}.
		These properties of the surface of Earth make it necessary to have a variety of different reference systems and projections for spatial data.
		
		A \todo{check this whole CRS thing for correctness}\term{coordinate reference system} (CRS) defines the combination of a coordinate system and an approximation of the surface of the Earth.
		The \term{coordinate system} defines, like a mathematical coordinate system, the extend and unit of coordinates.
		Types of coordinate systems are for example cartesian (with meters or foot as unit) and ellipsoidal (with latitude and longitude in degrees)\cite[11-13]{ordenance-survey-booklet}.
		The \term{horizontal datum} (also \term{Terrestrial Reference System} or \term{geodetic datum}) defines the origin and orientation of the coordinate system and is therefore a mapping from coordinate to location on Earth based on a specific ellipsoid.
		Depending on this ellipsoid, the datum either defines a global or local reference system.
		Latter is more accurate in certain places while a global system tries to find a good approximation for every location on Earth.
		For two dimensional Cartesian coordinate systems, a \term*[map projection]{projection} is needed to turn each  spherical location to a two dimensional planar coordinate\cite[17]{ordenance-survey-booklet}.
		
		For visualization purposes, a \term{map projection} turns coordinates of a specific CRS into flat and two dimensional coordinate for a map.
		This may be a different projection as used in a two dimensional Cartesian CRS and transformations between different CRS are needed.
		This visualizing map projection is independent of the data but has a strong effect on the visualization.\todo[inline]{Visualization part is a bit clumsy ...}

	\subsection{Data structures}
	
	\subsection{File formats}
	
	\subsection{GIS}
	
	\subsection{OpenStreetMap}

\section{Routing}

	% fastest, shortest path, all shortest...
	
	\subsection{Graph based}
	
	\subsection{Geometric}

\section{Agent based systems and simulations}