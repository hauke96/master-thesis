% !TEX root = ../thesis.tex
% !TeX spellcheck = en_US

% finding good routes is difficult, solved using A*/Dijkstra on networks
Traveling through a network of roads and paths often requires the use of a computer systems determining the optimal route from a source to a target location.
The software used to find such routes is called a routing engine and is based on algorithms optimized to find such routes.
In computer science these algorithms usually work on graphs and the problem to solve is called the \term{shortest-path problem}.
A graph is used as the central data structure abstracting the real world into vertices, which are connected by edges representing roads and paths.
Such graphs, however, can also represent other non-physical networks, for example trading relations between countries\citationNeeded.

If the network represents roads and paths, the edges are always just an abstraction of the real world situation.
Detailed geometric information, such as the width or number or lanes, are added as additional properties.
The geometry of the resulting shortest path, however, will just follow the given geometry of the network.
% networks have to be (manually) created and routing is fixed to network
Unfortunately the real world also contains polygonal shapes like market places, parking areas or parks, where classical routing algorithms can only find paths along the outer edges of these polygons.
Often auxiliary edges are added to connect opposite sides to allow routing algorithms to cross these areas\todo{cite OSM?}.
A hand full of lines can be added manually but filling large areas or even a whole city with these auxiliary paths is not feasible to be done manually.

% problem: Most real world destinations are not on the network
A similar problem arises with the exact source and target locations of real world routing.
Often these location are not part of the network, which is the case for most addresses, \term*{points of interest} (POIs) and locations manually chosen by the user.
They either have to be connected manually or a location on the network is chosen to to be the optimal source or target (e.g. by choosing the nearest  vertex or point on an edge).

% alternative: geometric routing avoiding obstacles
Next to a network based shortest path algorithm, pure geometric approaches exist to also determine shortest paths\index{geometric routing}.
Geometric approaches only know obstacles, which are geometries that cannot be passed.
There are two main strategies\citationNeeded:
Build a network around these obstacles and then use a classical shortest path algorithm to find the actual path or directly determine paths without building a network around obstacles.

The first approach creates a so called \term{visibility graph}, a graph with edged between all visible vertices\citationNeeded.
A network based shortest path algorithm is then able to find paths on that network.
It therefore can be seen as a kind of pre-processing.

The second approach does not create such graph and directly tries to find paths around the obstacles.
It uses the so called \term{continuous dijkstra paradigm}, which will be described later in detail\todo{describe in detail}.
A well fitting analogy for this approach are sound or water waves propagating through space, folding around obstacles and finally reaching the target.
Recursively following each wave back to its origin yields the shortest path from the source to the target\citationNeeded.
In fact, one can determine a tree of all shortest paths from the source with a given length, which means it behaves like Dijkstras algorithm, giving this paradigm its name\todo{are there sources on that?}.

% problem: performance(?) and insufficient data for useable routes
Current approaches for network based routing are well optimized and have time complexities of only $\bigo{|E|}$ (under the condition of integer edge weights and constant time multiplications) or $\bigo{|E| + |V| \cdot \log{|V|}}$ (for arbitrary but positive weights)\citationNeeded.

% this thesis: combine both worlds: Good data of networks and flexibility of geom. routing
This thesis proposes, implements and evaluates a hybrid routing algorithm that combines geometric with network based routing.
Such an hybrid approach will benefit from the performance of network based algorithms with the flexibility of geometric routing.
Calculating shortest paths is still fast but reaches arbitrary locations.
Normal real world user but also simulations, such as multi-agent simulations of pedestrians, can benefit from this algorithm.