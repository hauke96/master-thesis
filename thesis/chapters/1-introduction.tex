% !TEX root = ../thesis.tex
% !TeX spellcheck = en_US

\section{Motivation}
	
	Traveling through a network of roads and paths often requires the use of a computer system determining the optimal route from a source to a destination location.
	The software used to solve this problem is often called a \emph{routing engine} and uses shortest path algorithms.
	A graph is used underlying data structure, which abstracts the real world into vertices and edges, representing roads and ways.
	Detailed information, such as surface conditions or the number of lanes, are usually added to the geometries of the graph via attributes (usually key-value pairs).
	
	Algorithms for graph-based routing, such as the algorithms Dijkstra or A*, are optimized and used in numerous scientific and real-world applications including this work.
	To enhance performance in large networks, several speedup methods exist that further decrease query times allowing global routing request to be answered within milliseconds.\footnote{Tested for a routing request from Hamburg, Germany to Beijing, China on \href{https://www.openstreetmap.org/directions?engine=fossgis\_osrm\_car&route=53.55\%2C10.00\%3B39.91\%2C116.39}{openstreetmap.org} using the OSRM car profile which is based on Contraction Hierarchies (s. \Cref{subsubsec:ch}). The HTTP request returned the result in under 215ms -- including network latency of several milliseconds.}
	
	Aside from line-based shapes, which are very suitable for routing, the real world also consists of polygonal areas like market places, parking lots or parks, which can not be abstracted to a single line.
	Classical routing algorithms can therefore only find routes along the outer edges of these polygons.
	To cross these open areas, auxiliary paths are added to connect opposite sides of polygons.
	This is a common practice in OpenStreetMap, a crowdsourced geospatial database, for example on markets, squares or playgrounds.
	A low amount of lines can be added by hand, but manually filling large areas or even a whole city with these auxiliary paths is not feasible.
	Fortunately, there are geometric routing algorithms that find the so-called \term*{euclidean shortest path} across open spaces with the presence of impassible obstacles, such as buildings or walls.
	Two strategies exist to find path through open spaces, one uses graph-based routing in generated auxiliary paths and one only considered the obstacles in the open space.
	Both strategies will be covered in detail in \Cref{sec:geometric-routing}.
	
	A similar problem arises with the source and destination locations of real-world routing.
	Because things like addresses, \term*[point of interest]{points of interest} (\term[POI]{POIs}) and manually chosen locations are often not part of roads or ways, they are rarely part of the underlying routing graph and therefore not reachable by graph-based routing algorithms.
	Simple approaches to connect points to a graph (e.g. connecting to the nearest vertex) are often inaccurate.
	Such inaccuracies arise from the fact that the space between an arbitrary location and the actual road network may contain obstacles, such as walls or buildings.
	
	Even if only vertices of the routing graph are chosen, the quality of graph-based routing results is limited due to the abstractions of the road network.
	Missing sidewalks, inaccurate geometries and missing connections between parallel ways reduce the usefulness of graph-based routes, especially for routes determined for pedestrians.
	
\section{Problem and contributions}
	
	\begin{wrapfigure}{r}{0.5\textwidth}
%		\vspace{-1\baselineskip}
		\begin{figcenter}
			\includegraphics[width=\linewidth]{images/qgis-routing-osterstrasse}
		\end{figcenter}
		\caption[Comparison of normal routing with hybrid visibility routing.]{Comparison of a expected and realistic route (green) with a graph-based route (red) and an actual result of the algorithm presented in this work (blue).}
		\label{fig:osterstrasse-routing-vs-expected}
	\end{wrapfigure}

	The previous section mentions one of the disadvantages of graph-based routing:
	No assumptions can be made on whether source and destination locations are represented as vertices in the graph.
	Furthermore, trajectories of pedestrians in the real world are often not strictly following edges on routing graphs, which is especially the case for insufficiently connected open spaces.
	Routes strictly following the edges of a graph are therefore inaccurate for pedestrian routing\cite{graser-osm-open-spaces}, which means the calculated route contains detours and is therefore longer than the actual trajectory and additionally may not reach the exact destination location at all.
	Such inaccuracies significantly affect the quality of agent-based simulations, as well as the helpfulness and user experience of real-world applications, such as mobile navigation apps.

	In this thesis these issues are solved by proposing and designing an algorithm, called the \term{hybrid routing algorithm}, that combines the accuracy of geometric routing using visibility graphs with the optimizations and wide use of network-based routing.
	Additionally, an implementation of the proposed algorithm was created and evaluated regarding performance, correctness and usefulness.
	Determining shortest paths using the resulting approach has a simple structure, enables the use of common speedup techniques and reaches all accessible locations independent of the presence of corresponding vertices in the graph.
	Any other arbitrary location, for example a location chosen by the user, can be added to the graph.
	After connecting the newly added vertex to the graph, routing from and to this vertex is possible.
	Real-world applications as well as simulations, such as multi-agent models simulating the behavior of pedestrians, benefit from this algorithm.
	
\section{Structure of this thesis}

	First, the basics of spatial data, data formats, graph routing, geometric routing and agent-based simulations are covered in \Cref{chap:fundamentals}.
	The scientific work related to this thesis is presented in \Cref{chap:related-work} for graphs and networks, routing techniques and pedestrian path finding in particular.
	In \Cref{chap:design} the design of the system developed for this thesis is described and an overview of the different components and elements is given.
	A more detailed view of the implementation is presented in \Cref{chap:implementation} including technical aspects of the combination of network and geometric routing as well as used frameworks.
	The hybrid routing algorithm is then evaluated in \Cref{chap:evaluation} regarding performance, correctness and quality of the resulting routes.