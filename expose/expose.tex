% !TeX spellcheck = en_US

\documentclass[
	a4paper,
	11pt,
	twoside,
	twocolumn
]{article}

\usepackage[T1]{fontenc}
\usepackage[utf8]{inputenc}
\usepackage[ngerman,english]{babel}

%\usepackage{libertine}
\usepackage[sc]{mathpazo}
% Maybe this for math: \usepackage{libertinust1math}
\usepackage{courier}
\usepackage{microtype}

\usepackage[margin=2.25cm,columnsep=20pt]{geometry}
\usepackage{fancyhdr}

\usepackage{amsmath}
\usepackage{amsfonts}
\usepackage{amssymb}

\usepackage{titlesec}
\usepackage{titling}

\usepackage{lipsum}

\usepackage[toc, page]{appendix}
\usepackage[backend=bibtex8, style=numeric, sorting=none]{biblatex}
\usepackage{hyperref}
\usepackage[noabbrev]{cleveref}

%
% BibTeX
%
\addbibresource{sources.bib}

%
% Section styling
%
\renewcommand\thesection{\Roman{section}}
\renewcommand\thesubsection{\roman{subsection}}
\titleformat{\section}[block]{\large\scshape\centering}{\thesection.}{1em}{}
\titleformat{\subsection}[block]{\large}{\thesubsection.}{1em}{}

%
% Title styling
%
\setlength{\droptitle}{-4\baselineskip} % Move title up
\pretitle{\begin{center}\Huge\bfseries}
\posttitle{\end{center}}
\title{WIP: Improving Multi-Agent Simulations through Geometric Routing}
\author{%
	\textsc{Hauke Stieler} \\[1ex]
	\normalsize Universit\"at Hamburg \\ 
	\normalsize Department of Informatics \\
	\normalsize Databases and Information Systems \\
%	\normalsize \href{mailto:4stieler@informatik.uni-hamburg.de}{4stieler@informatik.uni-hamburg.de}
}
\date{\today}

%
% General page styling
%
\pagestyle{fancyplain}
\fancyfoot[C]{\thepage}
\renewcommand{\headrulewidth}{0pt}

\begin{document}
	\maketitle
	
	\section{Introduction}
		
		% Agent based simulations often uses spatial data.
		Simulating complex systems with numerous individuals is a complicated task.
		Agent-based models help to break down this complexity by simulating each individual separately with its own behavior and decision-making. \cite{macal2014introductory}
		Interactions are also an essential part of such simulating and affect other agents as well as the environment.
		
		An agent might be a simulated person but could also be a company, car or other non-living part of the simulated world.
		Many simulations use real people as agents to better understand human behaviors.
		This often involves a spatial environment where agents can move and interact with each other.
		Since random movement is rather uninteresting most of the time, algorithms finding optimal paths are involved. 
		
		% Common routing techniques only work on networks.
		In order to let agents move towards a certain target position, path finding and routing algorithms are used to calculate the optimal path.
		Normal path finding algorithms are well known, optimized and used in countless applications.
		Popular algorithms such as Dijkstra or A* (and many more) work on graphs with vertices and edges between them.
		These algorithms are well optimized and speed up methods, such as contraction hierarchies, can improve performance even more. \cite[57]{geisberger2008contraction}

		% Path finding of agents in geometric space.
		The world, however, does not consist of fixed lines.
		Even roads and sidewalks are areas where vehicles and pedestrians can theoretically walk freely on all possible directions.
		This is especially the case for larger public places such as station forecourts, pedestrian zones or markets.
		Here people have walking habits but are free to move wherever they want.
		
		% Network based routing has limits.
		Normal, network based routing reaches its limits here in terms of accuracy and possibilities for the agents.
		One example are points of interests (POIs), which are only reachable, when each POI is connected to the network of ways.
		Large open areas, such as a station forecourt, need crossing ways so that agents can walk across them, otherwise routing algorithms will route along the outer edge of those areas.
		To be able to find paths across open areas requires numerous additional lines in the network, but making them optimal and complete is an elaborate task.
		However, if the original dataset is considered to be correct, routes along the given ways are possible to use by agents and additional data (such as attributes regarding barrier-free accessibility) helps to find high quality routes.
		
		% Geometric routing can be beneficial.
		A geometric routing approach is not fraught with the mentioned problems regarding wide open areas.
		Agents, moving along a route determined by a geometric routing algorithm, will follow an optimal path across open areas, but avoiding obstacles, such as buildings or walls.
		Of course, certain areas such as highways and private property should not be crossed, which needs to me modeled in the data so that the algorithm can respect this.
		
		% Combination of routing algorithms might be beneficial
		Combining both approaches, network and geometric routing, could also combine both benefits.
		Using existing ways as long as possible and switching to geometric routing when needed would combine the two main benefits.
		Segments of a route following the network are possible and enriched with useful data, while segments crossing open areas do not need additional and manual preprocessing.
		Of course, such a technique requires data of high quality with all necessary barriers and zones that are not allowed to be entered.
		
	\section{Geometric routing}
	
		% multiple strategies: visibility graph vs. continuous dijkstra
		
		% visibility graph
		
		% cont. dijkstra
		
	\section{Considered scenario}
	
		% TODO describe this when scenario is fixed
		
	\section{Evaluation}
	
		% performance: preprocessing, routing
		
		% quality: usefulness, accuracy
	
	\section{Schedule}
	
		% TODO
	
	\printbibliography
\end{document}