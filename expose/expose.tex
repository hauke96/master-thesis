% !TeX spellcheck = en_US

\documentclass[
	a4paper,
	11pt,
	twoside,
	twocolumn
]{article}

\usepackage[T1]{fontenc}
\usepackage[utf8]{inputenc}
\usepackage[ngerman,english]{babel}

%\usepackage{libertine}
\usepackage[sc]{mathpazo}
% Maybe this for math: \usepackage{libertinust1math}
\usepackage{courier}
\usepackage{microtype}

\usepackage[margin=2.25cm,columnsep=20pt]{geometry}
\usepackage{fancyhdr}

\usepackage{amsmath}
\usepackage{amsfonts}
\usepackage{amssymb}

\usepackage{titlesec}
\usepackage{titling}

\usepackage{lipsum}

\usepackage[toc, page]{appendix}
\usepackage[backend=bibtex8, style=numeric, sorting=none]{biblatex}
\usepackage{hyperref}
\usepackage[noabbrev]{cleveref}

%
% BibTeX
%
\addbibresource{sources.bib}

%
% Section styling
%
\renewcommand\thesection{\Roman{section}}
\renewcommand\thesubsection{\roman{subsection}}
\titleformat{\section}[block]{\large\scshape\centering}{\thesection.}{1em}{}
\titleformat{\subsection}[block]{\large}{\thesubsection.}{1em}{}

%
% Title styling
%
\setlength{\droptitle}{-4\baselineskip} % Move title up
\pretitle{\begin{center}\Huge\bfseries}
\posttitle{\end{center}}
\title{WIP: Improving Multi-Agent Simulations through Geometric Routing}
\author{%
	\textsc{Hauke Stieler} \\[1ex]
	\normalsize Universit\"at Hamburg \\ 
	\normalsize Department of Informatics \\
	\normalsize Databases and Information Systems \\
%	\normalsize \href{mailto:4stieler@informatik.uni-hamburg.de}{4stieler@informatik.uni-hamburg.de}
}
\date{\today}

%
% General page styling
%
\pagestyle{fancyplain}
\fancyfoot[C]{\thepage}
\renewcommand{\headrulewidth}{0pt}

\begin{document}
	\maketitle
	
	\section{Introduction}
		
		% agent based sim. often spatial data
		
		Simulating complex systems with numerous individuals is a complicated task.
		Agent-based models help to break down this complexity by simulating each individual separately with its own behavior and decision making. \cite{macal2014introductory}
		Interactions are also an essential part of such simulating and affect other agents as well as the environment.
		
		An agent might be a simulated person but could also be a company, car or other non living part of the simulated world.
		Many simulations use real people as agents to better understand human behaviors.
		This often involves a spatial environment where agents can move and interact with each other.
		Since random movement is rather uninteresting most of the time, algorithms finding optimal paths are involved. 
		
		% common routing only on networks
		In order to let agents move towards a certain target position, path finding and routing algorithms are used to calculate the optimal path.
		Normal path finding algorithms are well known, optimized and used in countless applications.
		Popular algorithms such as Dijkstra or A* (and many more) work on graphs with vertices and edges between them.
		These algorithms are well optimized and speed up methods, such as contraction hierarchies, can improve performance even more.
		% TODO source on CH performance

		% path finding of agents geometric
		The world, however, does not consist of fixed lines.
		
		
		% geometr. routing or combination might be beneficial
		
	\section{Geometric routing}
	
		% multiple strategies: visibility graph vs. continuous dijkstra
		
		% visibility graph
		
		% cont. dijkstra
		
	\section{Considered scenario}
	
		% TODO describe this when scenario is fixed
		
	\section{Evaluation}
	
		% performance: preprocessing, routing
		
		% quality: usefulness, accuracy
	
	\section{Schedule}
	
		% TODO
	
	\printbibliography
\end{document}